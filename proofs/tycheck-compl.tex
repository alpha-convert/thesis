\begin{proof}
By induction on the derivation of $\Psi;\Theta;\Delta;\Omega;\Gamma \vdash e : \tau$, we prove both claims simultaneously.
In all cases, the erasure property is immediate from the inductive hypotheses-- we will elide this bit of the proof.
\begin{itemize}
  \item[(T-Var-1)] Suppose $\Psi ; \Theta ; \Delta ; \Omega ; \Gamma \vdash x : \tau$ from $x : \tau \in \Gamma$. By AT-Var-1, $\Psi ; \Theta ; \Delta ; \Omega ; \Gamma \vdash x \infers \tau \gens \top,\Gamma \setminus \{x : \tau\}$. Of course, $\Theta ; \Delta \vDash \top$, and $|x| = x$, so (1) is complete. By Theorem~\ref{thm:subty-refl}, $\Psi ; \Theta ; \Delta \vdash \tau \subty \tau : \star \gens \Phi$ with $\Theta ; \Delta \vDash \Phi$. By AT-Sub,  $\Psi ; \Theta ; \Delta ; \Omega ; \Gamma \vdash x \checks \tau \gens \top,\Gamma \setminus \{x : \tau\}$, as required for (2).
  
  \item[(T-Var-2)] Immediate.
  \item[(T-Unit)] Immediate.
  \item[(T-Base)] Immediate.
  \item[(T-Absurd)] Suppose $\Psi ; \Theta ; \Delta ; \Omega ; \Gamma \vdash \texttt{absurd} : \tau$ from $\Theta ; \Delta \vDash \bot$. By AT-Absurd,
  $\Psi ; \Theta ; \Delta ; \Omega ; \Gamma \vdash \texttt{absurd} \checks \tau \gens \bot,\Gamma$. Since $\Theta ; \Delta \vDash \bot$, this completes (1). For (2),
  use AT-Anno to get $\Psi ; \Theta ; \Delta ; \Omega ; \Gamma \vdash (\texttt{absurd} : \tau) \infers \tau \gens \bot,\Gamma$
  
  \item[(T-Nil)] Suppose $\Psi ; \Theta ; \Delta ; \Omega ; \Gamma\vdash \texttt{nil} : L^I \tau$ by way of
  $\Theta ; \Delta \vdash I : \mathbb{N}$ and
  $\Theta;\Delta \vDash I = 0$.
  By Theorem~\ref{thm:sort-compl}, there is a $\Phi$ such that
  $\Theta ; \Delta \vDash \Phi$, and
  $\Theta ; \Delta \vdash I : \mathbb{N} \gens \Phi$.
  By AT-Nil, $\Psi ; \Theta ; \Delta ; \Omega ; \Gamma\vdash \texttt{nil} \checks L^I \tau \gens \Phi \wedge (I = 0),\Gamma$.
  Of course, $\Theta ; \Delta \vDash \Phi \wedge (I = 0)$,
  completing (1). (2) follows by AT-Anno.
  
  \item[(T-Cons)] Suppose
  $\Psi ; \Theta ; \Delta ; \Omega ; \Gamma_1, \Gamma_2\vdash e_1 :: e_2 : L^I \tau$ by way of
  $\Psi ; \Theta ; \Delta ; \Omega ; \Gamma_1\vdash e_1 : \tau$,
  $\Psi ; \Theta ; \Omega ; \Gamma_2\vdash e_2 : L^{I-1} \tau$, and
  $\Theta ; \Delta \vDash I \geq 1$.
  By IH, there are $e_1'$, $\Phi_1$, $\Gamma_1'$ such that
  $|e_1'| = e_1$,
  $\Theta ; \Delta \vDash \Phi_1'$, and
  $\Psi ; \Theta ; \Delta ; \Omega ; \Gamma_1 \vdash e_1' \checks \tau \gens \Phi_1,\Gamma_1'$.
  Since $\Psi ; \Theta ; \Delta \vdash \Gamma_1,\Gamma_2 \wknto \Gamma_1$,
  by Theorem~\ref{thm:admits-weaken} we have $e_1''$, $\Phi_1'$, $\Gamma_1''$ such that
  $|e_1''| = |e_1'|$,
  $\Theta ; \Delta \vDash \Phi_1'$,
  $\Psi ; \Theta ; \Delta \vdash \Gamma_1'' \wknto (\Gamma_1,\Gamma_2) \setminus \Gamma_1$, and
  $\Psi ; \Theta ; \Delta ; \Omega ; \Gamma_1,\Gamma_2 \vdash e_1'' \checks \tau \gens \Phi_1',\Gamma_1''$.
  Since $(\Gamma_1,\Gamma_2) \setminus \Gamma_1 = \Gamma_2$,
  we have that
  $\Psi ; \Theta ; \Delta \vdash \Gamma_1'' \wknto \Gamma_2$.
  By IH, there are $e_2'$, $\Phi_2$, $\Gamma_2'$ such that
  $|e_2'| = e_2$,
  $\Theta ; \Delta \vDash \Phi_2$, and
  $\Psi ; \Theta ; \Delta ; \Omega ; \Gamma_2 \vdash e_2' \checks L^{I-1} \tau \gens \Phi_2,\Gamma_2'$.
  But again by Theorem~\ref{thm:admits-weaken}, we have $e_2''$, $\Phi_2'$, $\Gamma_2''$ such that
  $|e_2''| = |e_2'|$,
  $\Theta ; \Delta \vDash \Phi_2'$, and
  $\Psi ; \Theta ; \Delta ; \Omega ; \Gamma_1'' \vdash e_2' \checks L^{I-1} \tau \gens \Phi_2',\Gamma_2''$.
  By AT-Cons, we have that
  $\Psi ; \Theta ; \Delta ; \Omega ; \Gamma_1,\Gamma_2 \vdash e_1 :: e_2 \checks L^I \tau \gens \Phi_1' \wedge \Phi_2' \wedge (I \geq 1), \Gamma_2''$.
  But, $\Theta ; \Delta \vDash \Phi_1' \wedge \Phi_2' \wedge (I \geq 1)$ and so this completes (1). (2) follows immediately by AT-Anno.
  
  
  \item[(T-Match)] Suppose
  $\Psi ; \Theta ; \Delta ; \Omega ; \Gamma_1,\Gamma_2\vdash \texttt{match}(e,e_1,h.t.e_2) : \tau'$ by way of
  $\Psi ; \Theta ; \Delta ; \Omega ; \Gamma_1\vdash e : L^I \tau$,
  $\Psi ; \Theta ; \Delta, I = 0 ; \Omega ; \Gamma_2\vdash e_1 : \tau'$, and
  $\Psi ; \Theta ; \Delta, I \geq 1; \Omega ; \Gamma_2, h : \tau, t : L^I \tau \vdash e_2 : \tau'$.
  By IH, there are $e'$, $\Phi$, $\Gamma_1'$ such that
  $|e'| = e$,
  $\Theta ; \Delta \vDash \Phi$, and
  $\Psi ; \Theta ; \Delta ; \Omega ; \Gamma_1 \vdash e' \infers L^I \tau \gens \Phi, \Gamma_1'$.
  Since $\Psi ; \Theta ; \Delta \vdash \Gamma_1,\Gamma_2 \wknto \Gamma_1$,
  we have by Theorem~\ref{thm:admits-weaken} that there are $e''$, $\Phi'$, $\Gamma_1''$ such that
  $|e''| = |e'|$,
  $\Theta ; \Delta \vDash \Phi'$,
  $\Psi ; \Theta ; \Delta \vdash \Gamma_1'' \wknto \Gamma_2$, and
  $\Psi ; \Theta ; \Delta ; \Omega ; \Gamma_1,\Gamma_2 \vdash e'' \infers L^I \tau \gens \Phi', \Gamma_1''$.
  Again by IH, there are $e_1'$, $\Phi_1$, $\Gamma_2'$ such that
  $|e_1'| = e_1$,
  $\Theta ; \Delta, I = 0 \vDash \Phi_1$, and
  $\Psi ; \Theta ; \Delta, I = 0 ; \Omega ; \Gamma_2 \vdash e_1' \checks \tau' \gens \Phi_1,\Gamma_2'$.
  By Theorem~\ref{thm:ctx-sub-wkn},
  $\Psi ; \Theta ; \Delta, I = 0 \vdash \Gamma_1'' \wknto \Gamma_2$.
  Then, by Theorem~\ref{thm:admits-weaken}, there are $e_1''$, $\Phi_1'$, $\Gamma_2''$ such that
  $|e_1''| = |e_1'|$,
  $\Theta ; \Delta, I = 0 \vDash \Phi_1'$, and
  $\Psi ; \Theta ; \Delta, I = 0 ; \Omega ; \Gamma_1'' \vdash e_2'' \checks \tau' \gens \Phi_1',\Gamma_2''$.
  Invoking the IH once more, we have $e_2'$, $\Phi_2$, $\Gamma_3'$ such that
  $|e_2'| = e_2$,
  $\Theta ; \Delta, I \geq 1 \vdash \Phi_2$, and
  $\Psi ; \Theta ; \Delta, I \geq 1 ; \Omega ; \Gamma_2, h : \tau, t : L^I \tau \vdash e_2' \checks \tau' \gens \Phi_2,\Gamma_3'$.
  By Theorem~\ref{thm:ctx-sub-wkn} and Theorem~\ref{thm:ctx-sub-subset2},
  $\Psi ; \Theta ; \Delta, I \geq 1 \vdash \Gamma_1'', h : \tau, t : L^I \tau \wknto \Gamma_2, h : \tau, t : L^I \tau$.
  Then, by Theorem~\ref{admits-weaken}, there are $e_2''$, $\Phi_2'$, $\Gamma_3''$ such that
  $|e_2''| = |e_2'|$,
  $\Theta ; \Delta, I \geq 1 \vDash \Phi_2'$, and
  $\Psi ; \Theta ; \Delta, I \geq 1; \Omega ; \Gamma_1'', h : \tau, t : L^I \tau \vdash e_2' \checks \tau' \gens \Phi_2',\Gamma_3''$.
  By AT-Match,
  $\Psi ; \Theta ; \Delta ; \Omega ; \Gamma_1,\Gamma_2\vdash \texttt{match}(e,e_1,h.t.e_2) \checks \tau' \gens \Phi' \wedge (I = 0 \to \Phi_1') \wedge (I \geq 1 \to \Phi_2'), \Gamma_2'' \cap (\Gamma_3'' \setminus \{x,y\})$.
  Since $\Theta ; \Delta \vDash \Phi'$,  $\Theta ; \Delta, I = 0 \vDash \Phi_1'$, and $\Theta ; \Delta, I \geq 1 \vDash \Phi_2'$, we have equivalently that
  $\Theta ; \Delta \vDash \Phi' \wedge (I = 0 \to \Phi_1') \wedge (I \geq 1 \to \Phi_2')$. This completes (1), and (2) follows by AT-Anno.
  
  \item[(T-ExistI)] Suppose $\Psi ; \Theta ; \Delta ; \Omega ; \Gamma\vdash \texttt{pack}[I](e) : \exists i:S.\tau$ from
  $\Theta ; \Delta \vdash I : S$, and
  $\Psi ; \Theta ; \Delta ; \Omega ; \Gamma\vdash e : \tau[I/i]$.
  By Theorem~\ref{thm:idx-compl}, there is some $\Phi_1$ such that
  $\Theta ; \Delta \vDash \Phi_1$, and
  $\Theta ; \Delta \vdash I : S \gens \Phi_1$.
  By IH, there are $e'$, $\Phi_2$, $\Gamma'$ such that
  $\Psi ; \Theta ; \Delta ; \Omega ; \Gamma \vdash e' \checks \tau[I/i] \gens \Phi_2,\Gamma'$.
  By AT-ExistI,
  $\Psi ; \Theta ; \Delta ; \Omega ; \Gamma\vdash \texttt{pack}[I](e') : \exists i:S.\tau \gens \Phi_1 \wedge \Phi_2,\Gamma'$.
  Clearly, $|\texttt{pack}[I](e')| = \texttt{pack}[I](|e'|) = \texttt{pack}[I](e)$, and $\Theta ; \Delta \vDash \Phi_1 \wedge \Phi_2$.
  This completes (1), and (2) follows immediately by AT-Anno.
  
  \item[(T-ExistE)] Suppose
  $\Psi ; \Theta ; \Delta ; \Omega ; \Gamma_1,\Gamma_2\vdash \texttt{unpack } (i,x) = e_1 \texttt{ in } e_2 : \tau'$ from
  $\Psi ; \Theta ; \Delta ; \Omega ; \Gamma_1\vdash e_1 : \exists i : S.\tau$ and
  $\Psi ; \Theta, i : S ; \Delta ; \Omega ; \Gamma_2, x : \tau \vdash e_2 : \tau'$.
  By IH, there are $e_1'$, $\Phi_1$, $\Gamma_1'$ such that
  $|e_1'| = e_1$,
  $\Theta ; \Delta \vDash \Phi_1$, and
  $\Psi ; \Theta ; \Delta ; \Omega ; \Gamma_1 \vdash e_1' \infers \exists i : S. \tau \gens \Phi_1,\Gamma_1'$.
  Since $\Psi ; \Theta ; \Delta \vdash \Gamma_1,\Gamma_2 \wknto \Gamma_1$,
  we have by Theorem~\ref{thm:admits-weaken} that there are $e_1''$, $\Phi_1'$, $\Gamma_1''$ such that
  $|e_1''| = |e_1'|$,
  $\Theta ; \Delta \vDash \Phi_1'$,
  $\Psi ; \Theta ; \Delta \vdash \Gamma_1'' \wknto \Gamma_2$, and
  $\Psi ; \Theta ; \Delta ; \Omega ; \Gamma_1,\Gamma_2 \vdash e_1'' \infers \exists i : S. \tau \gens \Phi_1',\Gamma_1''$.
  By IH, there are $e_2'$, $\Phi_2$, $\Gamma_2'$ such that
  $|e_2'| = e_2$,
  $\Theta, i : S ; \Delta \vDash \Phi_2$, and
  $\Psi ; \Theta, i : S ; \Delta ; \Omega ; \Gamma_2,x : \tau \vdash e_2' \checks \tau' \gens \Phi_2,\Gamma_2'$.
  Since $\Psi ; \Theta ; \Delta \vdash \Gamma_1'',x : \tau \wknto \Gamma_2, x : \tau$,
  we have by Theorem~\ref{thm:admits-weaken} that there are $e_2''$, $\Phi_2'$, $\Gamma_2''$ such that
  $|e_2''| = |e_2'|$,
  $\Theta, i : S ; \Delta \vDash \Phi_2'$, and
  $\Psi ; \Theta, i : S ; \Delta ; \Omega ; \Gamma_1'', x: \tau \vdash e_2'' \checks \tau' \gens \Phi_2',\Gamma_2''$.
  By AT-ExistE,
  $\Psi ; \Theta ; \Delta ; \Omega ; \Gamma_1,\Gamma_2\vdash \texttt{unpack } (i,x) = e_1' \texttt{ in } e_2' : \tau' \gens \Phi_1' \wedge (\forall i : S. \Phi_2'),\Gamma_2'' \setminus \{x\}$,
  which completes (1). For (2), a single use of AT-Anno suffices.
  
  \item[(T-Lam)] Suppose $\Psi ; \Theta ; \Delta ; \Omega ; \Gamma\vdash \lambda x.e : \tau_1 \loli \tau_2$ from $\Psi ; \Theta ; \Delta ; \Omega ; \Gamma, x : \tau_1 \vdash e : \tau_2$. By IH, there are $e'$, $\Phi$, $\Gamma'$ so that $|e'| = e$, $\Theta ; \Delta \vDash \Phi$, and $\Psi ; \Theta ; \Delta ; \Omega ; \Gamma, x : \tau_1 \vdash e' \checks \tau_2 \gens \Phi,\Gamma'$. By AT-Lam, $\Psi ; \Theta ; \Delta ; \Omega ; \Gamma \vdash \lambda x. e' \checks \tau_1 \loli \tau_2 \tau_2 \gens \Phi,\Gamma' \setminus \{x : \tau_1\}$. But, $|\lambda x. e'| = \lambda x. |e'| = \lambda x.e$, which completes the proof of (1). For (2), AT-Anno gives that $\Psi ; \Theta ; \Delta ; \Omega ; \Gamma \vdash (\lambda x. e' : \tau_1 \loli \tau_2) \infers \tau_1 \loli \tau_2 \gens \Phi,\Gamma' \setminus \{x : \tau_1\}$, as required.
  
  \item[(T-App)] Suppose $\Psi ; \Theta ; \Delta ; \Omega ; \Gamma_1,\Gamma_2\vdash e_1 \, e_2 :  \tau_2$ from
  $\Psi ; \Theta ; \Delta ; \Omega ; \Gamma_1\vdash e_1 : \tau_1 \loli \tau_2$ and
  $\Psi ; \Theta ; \Delta ; \Omega ; \Gamma_2\vdash e_2 : \tau_1$.
  By IH, there are $e_1'$, $\Gamma_1'$, $\Phi_1$ such that
  $|e_1'| = e_1$,
  $\Theta ; \Delta \vDash \Phi_1$, and
  $\Psi ; \Theta ; \Delta ; \Omega ; \Gamma_1 \vdash e_1' \infers \tau_1 \loli \tau_2 \gens \Phi_1,\Gamma_1'$.
  Since $\Psi ; \Theta ; \Delta \vdash \Gamma_1,\Gamma_2 \wknto \Gamma_1$,
  we have by Theorem~\ref{thm:admits-weaken} that there are $e_1''$, $\Phi_1'$, $\Gamma_1''$ such that
  $|e_1''| = |e_1'|$,
  $\Theta ; \Delta \vDash \Phi_1'$,
  $\Psi ; \Theta ; \Delta \vdash \Gamma_1'' \wknto \Gamma_2$, and
  $\Psi ; \Theta ; \Delta ; \Omega ; \Gamma_1,\Gamma_1 \vdash e_1'' \infers \tau_1 \loli \tau_2 \gens \Phi_1',\Gamma_1''$.
  Then, by IH, there are $e_2'$, $\Gamma_2'$, and $\Phi_2$ such that
  $|e_2'| = e_2$,
  $\Theta ; \Delta \vDash \Phi_2$,
  $\Psi ; \Theta ; \Delta ; \Omega ; \Gamma_2 \vdash e_2' \checks \tau_1 \gens \Phi_2,\Gamma_2'$.
  But then by Theorem~\ref{thm:admits-weaken}, there are $e_2''$, $\Gamma_2''$, and $\Phi_2'$ such that
  $|e_2''| = |e_2'|$,
  $\Theta ; \Delta \vDash \Phi_2'$, and
  $\Psi ; \Theta ; \Delta ; \Omega; \Gamma_1'' \vdash e_2'' \checks \tau_1 \gens \Phi_2',\Gamma_2''$.
  Then, by AT-App, we have that
  $\Psi ; \Theta ; \Delta ; \Omega ; \Gamma_1,\Gamma_2 \vdash e_1'' \, e_2'' \infers \tau_2 \gens \Phi_1' \wedge \Phi_2',\Gamma_2''$.
  This completes (2). For (1), 
  we apply Theorem~\ref{thm:subty-refl} to find that there is some $\Phi_3$ such that $\Theta ; \Delta \vDash \Phi_3$, and
  $\Psi ; \Theta ; \Delta \vdash \tau_2 \subty \tau_2 : \star \gens \Phi_3$.
  Then, by AT-Sub, we have
  $\Psi ; \Theta ; \Delta ; \Omega ; \Gamma_1,\Gamma_2 \vdash e_1'' \, e_2'' \checks \tau_2 \gens \Phi_1' \wedge \Phi_2' \wedge \Phi_3,\Gamma_2''$
  as required for (2).
  
  \item[(T-TensorI)] Suppose $\Psi ; \Theta ; \Delta ; \Omega ; \Gamma_1,\Gamma_2\vdash \angles{e_1,e_2} ; \tau_1 \otimes \tau_2$ from $\Psi ; \Theta ; \Delta ; \Omega ; \Gamma_1\vdash e_1 : \tau_1$ and $\Psi ; \Theta ; \Delta ; \Omega ; \Gamma_2\vdash e_2 : \tau_2$.
  By IH, there are $e_1'$, $\Phi_1$, $\Gamma_1'$ such that $|e_1'| = e_1$, $\Theta ; \Delta \vDash \Phi_1$, and $\Psi ; \Theta ; \Delta ; \Omega ; \Gamma_1 \vdash e_1' \checks \tau_1 \gens \Phi_1,\Gamma_1'$. By Theorem~\ref{thm:admits-weaken}, there are $e_1''$, $\Phi_1'$, $\Gamma_1''$ such that $|e_1''| = |e_1'|$, $\Theta ; \Delta \vDash \Phi_1'$, $\Psi ; \Theta ; \Delta \vdash \Gamma_1'' \wknto (\Gamma_1,\Gamma_2) \setminus \Gamma_1$ and $\Psi ; \Theta ; \Delta ;\Omega ; \Gamma_1,\Gamma_2 \vdash e_1'' \checks \tau_2 \gens \Phi_1',\Gamma_1''$. But, $(\Gamma_1,\Gamma_2) \setminus \Gamma_1 = \Gamma_2$, since $\Gamma_1$ and $\Gamma_2$ are disjoint, and so we have that $\Psi ; \Theta ; \Delta \vdash \Gamma_1'' \wknto \Gamma_2$.
  By IH, there are $e_2'$, $\Phi_2$, $\Gamma_2'$ such that $|e_2'| = e_2$, $\Theta ; \Delta \vDash \Phi_2$,  and $\Psi ; \Theta ; \Delta ; \Omega ; \Gamma_2 \vdash e_2' \checks \tau_2 \gens \Phi_2,\Gamma_2'$. But by Theorem~\ref{thm:admits-weaken}, there are $e_2''$, $\Phi_2'$, $\Gamma_2''$ such that $|e_2''| = |e_2'|$, $\Theta ; \Delta \vDash \Phi_2'$, and $\Psi ; \Theta ; \Delta ; \Omega ; \Gamma_1'' \vdash e_2'' \checks \tau_2 \gens \Phi_2', \Gamma_2''$. By AT-TensorI,
  $\Psi ; \Theta ; \Delta ; \Omega ; \Gamma_1,\Gamma_2 \vdash \angles{e_1'',e_2''} \checks \tau_1 \otimes \tau_2 \gens \Phi_1' \wedge \Phi_2', \Gamma_2''$.
  We then verify the conditions: $|\angles{e_1'',e_2''}| = \angles{|e_1''|,|e_2''|} = \angles{|e_1'|,|e_2'|} = \angles{e_1,e_2}$, and also $\Theta ; \Delta \vDash \Phi_1' \wedge \Phi_2'$, completing (1). For (2), AT-Anno gives that $\Psi ; \Theta ; \Delta ; \Omega ; \Gamma_1,\Gamma_2 \vdash (\angles{e_1'',e_2''} : \tau_1 \otimes \tau_2) \infers \tau_1 \otimes \tau_2 \gens \Phi_1' \wedge \Phi_2', \Gamma_2''$, as required.
  
  \item[(T-TensorE)] Suppose $\Psi ; \Theta ; \Delta ; \Omega ; \Gamma_1,\Gamma_2\vdash \texttt{let } \angles{x,y} = e_1 \texttt{ in } e_2 : \tau'$ by way of
  $\Psi ; \Theta ; \Delta ; \Omega ; \Gamma_1\vdash e_1 : \tau_1 \otimes \tau_2$ and
  $\Psi ; \Theta ; \Delta ; \Omega ; \Gamma_2,x : \tau_1, y : \tau_2\vdash e_2 : \tau'$.
  By IH, there are $e_1'$, $\Phi_1$, $\Gamma_1'$ such that
  $|e_1'| = e_1$,
  $\Theta ; \Delta \vDash \Phi_1$, and
  $\Psi ; \Theta ; \Delta ; \Omega ; \Gamma_1 \vdash e_1' \infers \tau_1 \otimes \tau_2 \gens \Phi_1,\Gamma_1'$.
  Since $\Psi ; \Theta ; \Delta \vdash \Gamma_1,\Gamma_2 \wknto \Gamma_1$,
  we have by Theorem~\ref{thm:admits-weaken} that there are $e_1''$, $\Phi_1'$, $\Gamma_1''$ such that
  $|e_1''| = |e_1'|$,
  $\Theta ; \Delta \vDash \Phi_1'$,
  $\Psi ; \Theta ; \Delta \vdash \Gamma_1'' \wknto (\Gamma_1,\Gamma_2 \setminus \Gamma_1)$, and
  $\Psi ; \Theta ; \Delta ; \Omega ; \Gamma_1,\Gamma_2 \vdash e_1'' \infers \tau_1 \otimes \tau_2 \gens \Phi_1',\Gamma_1''$.
  Again by IH, there are $e_2'$, $\Phi_2$, $\Gamma_2'$ such that
  $|e_2'| = e_2$,
  $\Theta ; \Delta \vDash \Phi_2$, and
  $\Psi ; \Theta ; \Delta ; \Omega ; \Gamma_2,x : \tau_1, y : \tau_2 \vdash e_2' \checks \tau' \gens \Phi_2,\Gamma_2'$.
  Again by Theorem~\ref{admits-weaken}, since
  $\Psi ; \Theta ; \Delta \vdash \Gamma_1'',x : \tau_1, y : \tau_2 \wknto \Gamma_2,x : \tau_1, y : \tau_2$,
  there are $e_2''$, $\Phi_2'$, $\Gamma_2''$ such that
  $|e_2''| = |e_2'|$,
  $\Theta ; \Delta \vDash \Phi_2'$, and
  $\Psi ; \Theta ; \Delta, \Gamma_1'', x : \tau_1, y : \tau_2 \vdash e_2'' \checks \tau' \gens \Phi_2',\Gamma_2''$.
  By AT-TensorE,
  $\Psi ; \Theta ; \Delta ; \Omega;  \Gamma_1,\Gamma_2 \vdash \texttt{let } \angles{x,y} = e_1'' \texttt{ in } e_2'' \checks \tau' \gens \Phi_1' \wedge \Phi_2',\Gamma_2''$.
  This completes (1), and (2) follows by AT-Anno.
  
  \item[(T-WithI)] Suppose
  $\Psi ; \Theta ; \Delta ; \Omega ; \Gamma \vdash (e_1,e_2) : \tau_1 \amp \tau_2$ by way of
  $\Psi ; \Theta ; \Delta ; \Omega ; \Gamma \vdash e_1 : \tau_1$ and
  $\Psi ; \Theta ; \Delta ; \Omega ; \Gamma \vdash e_2 : \tau_2$.
  By IH, we have for $i = 1,2$ that $e_i'$, $\Phi_i$, $\Gamma_i$ such that
  $|e_i'| = e_i$,
  $\Theta ; \Delta \vDash \Phi_i$, and
  $\Psi ; \Theta ; \Delta ; \Omega ; \Gamma \vdash e_i' : \tau_i \gens \Phi_i,\Gamma_i$.
  By AT-WithI,
  $\Psi ; \Theta ; \Delta ; \Omega ; \Gamma \vdash (e_1',e_2') \checks \tau_1 \amp \tau_2 \gens \Phi_1 \wedge \Phi_2,\Gamma_1 \cap \Gamma_2$.
  This completes (1), and (2) follows by AT-Anno.
  
  \item[(T-Fst)] Suppose
  $\Psi ; \Theta ; \Delta ; \Omega ; \Gamma \vdash \texttt{fst}(e) : \tau_1$ from
  $\Psi ; \Theta ; \Delta ; \Omega ; \Gamma \vdash e : \tau_1 \amp \tau_2$.
  By IH, there are $e'$, $\Phi$, $\Gamma'$ such that
  $|e'| = e$,
  $\Theta ; \Delta \vDash \Phi$, and
  $\Psi ; \Theta ; \Delta ; \Omega ; \Gamma \vdash e' \infers \tau_1 \amp \tau_2 \gens \Phi,\Gamma'$.
  By AT-Fst,
  $\Psi ; \Theta ; \Delta ; \Omega ; \Gamma \vdash \texttt{fst}(e') \infers \tau_1 \gens \Phi,\Gamma'$.
  This completes (2). For (1), we invoke Theorem~\ref{thm:subty-refl} to get some $\Phi'$ such that
  $\Theta ; \Delta \vDash \Phi'$ and $\Psi ; \Theta ; \Delta \vdash \tau_1 \subty \tau_1 : \star \gens \Phi'$.
  Then, by AT-Sub,
  $\Psi ; \Theta ; \Delta ; \Omega ; \Gamma \vdash \texttt{fst}(e') \checks \tau_1 \gens \Phi \wedge \Phi',\Gamma'$,
  which completes (1).
  
  
  \item[(T-Snd)] Identical to T-Fst.
  \item[(T-Inl)] Suppose $\Psi ; \Theta ; \Delta ; \Omega ; \Gamma \vdash \texttt{inl}(e) : \tau_1 \oplus \tau_2$ from
  $\Psi ; \Theta ; \Delta ; \Omega ; \Gamma \vdash e : \tau_1$.
  By IH, there are $e'$, $\Phi$, $\Gamma'$ such that
  $|e'| = e$,
  $\Theta ; \Delta \vDash \Phi$, and
  $\Psi ; \Theta ; \Delta ; \Omega ; \Gamma \vdash e' \checks \tau_1 \gens \Phi,\Gamma'$.
  By AT-Inl,
  $\Psi ; \Theta ; \Delta ; \Omega ; \Gamma \vdash \texttt{inl}(e') \checks \tau_1 \oplus \tau_2 \gens \Phi,\Gamma'$.
  This completes (1), and (2) follows by AT-Anno.
  
  
  \item[(T-Inr)] Identical to T-Inl.
  \item[(T-Case)] Suppose
  $\Psi ; \Theta ; \Delta ; \Omega ; \Gamma_1,\Gamma_2 \vdash \texttt{case}(e,x.e_1,y.e_2) : \tau$ by way of
  $\Psi ; \Theta ; \Delta ; \Omega ; \Gamma_1 \vdash e : \tau_1 \oplus \tau_2$,
  $\Psi ; \Theta ; \Delta ; \Omega ; \Gamma_2, x: \tau_1 \vdash e_1 : \tau$, and
  $\Psi ; \Theta ; \Delta ; \Omega ; \Gamma_2, y: \tau_2 \vdash e_2 : \tau$.
  By IH, there are $e'$, $\Phi$, $\Gamma_1'$ such that
  $|e'| = e$,
  $\Theta ; \Delta \vDash \Phi$, and
  $\Psi ; \Theta ; \Delta ; \Omega ; \Gamma_1 \vdash e' \infers \tau_1 \oplus \tau_2 \gens \Phi,\Gamma_1'$.
  But since $\Psi ; \Theta ; \Delta \vdash \Gamma_1,\Gamma_2 \wknto \Gamma_1$, we have by Theorem~\ref{thm:admits-weaken}
  that there are $e''$, $\Phi'$, $\Gamma_1''$ such that
  $|e''| = |e'|$,
  $\Theta ; \Delta \vDash \Phi'$,
  $\Psi ; \Theta ; \Delta \vdash \Gamma_1'' \wknto (\Gamma_1,\Gamma_2) \setminus \Gamma_1$, and
  $\Psi ; \Theta ; \Delta ; \Omega ; \Gamma_1,\Gamma_2 \vdash e'' \infers \tau_1 \oplus \tau_2 \gens \Phi',\Gamma_1''$.
  Then, again by IH, there are $e_1'$, $\Phi_1$, $\Gamma_2'$ such that
  $|e_1'| = e_1$,
  $\Theta ; \Delta \vDash \Phi_1$, and
  $\Psi ; \Theta ; \Delta ; \Omega ; \Gamma_2, x : \tau_1 \vdash e_1' \checks \tau \gens \Phi_1,\Gamma_2'$.
  But since
  $\Psi ; \Theta ; \Delta \vdash \Gamma_1'', x : \tau_1 \wknto \Gamma_2, x : \tau_1$,
  we have by Theorem~\ref{thm:admits-weaken} that there are $e_1''$, $\Phi_1'$, $\Gamma_2''$ such that
  $|e_1''| = |e_1'|$,
  $\Theta ; \Delta \vDash \Phi_1'$, and
  $\Psi ; \Theta ; \Delta ; \Omega ; \Gamma_1'', x : \tau_1 \vdash e_1'' \checks \tau \gens \Phi_1',\Gamma_2''$.
  Applying IH one last time, we have $e_2'$, $\Phi_2$, $\Gamma_3'$ such that
  $|e_2'| = e_2$,
  $\Theta ; \Delta \vDash \Phi_2$, and
  $\Psi ; \Theta ; \Delta ; \Omega ; \Gamma_2, y : \tau_2 \vdash e_2' \checks \tau \gens \Phi_2,\Gamma_3'$.
  But, once more by Theorem~\ref{thm:admits-weaken}, there are $e_2''$, $\Phi_2'$, $\Gamma_3''$ such that
  $|e_2''| = |e_2'|$,
  $\Theta ; \Delta \vDash \Phi_2'$, and
  $\Psi ; \Theta ; \Delta ; \Omega ; \Gamma_1'', y : \tau_2 \vdash e_2'' \checks \tau \gens \Phi_2',\Gamma_3''$.
  Finally, by AT-Case, we have that
  $\Psi ; \Theta ; \Delta ; \Omega ; \Gamma_1,\Gamma_2 \vdash \texttt{case}(e'',x.e_1'',y.e_2'') \checks \tau \gens \Phi' \wedge \Phi_1' \wedge \Phi_2',(\Gamma_2'' \setminus \{x\}) \cap (\Gamma_3'' \setminus \{y\})$.
  This completes (1), and (2) follows immediately by AT-Anno.
  
  \item[(T-ExpI)] Suppose
  $\Psi ; \Theta ; \Delta ; \Omega ; \Gamma \vdash !e : !\tau$ by way of
  $\Psi ; \Theta ; \Delta ; \Omega ; \cdot \vdash e : \tau$.
  By IH, there are $e'$, $\Phi$, $\Gamma'$ such that
  $|e'| = e$,
  $\Theta ; \Delta \vDash \Phi$,and
  $\Psi ; \Theta ; \Delta ; \Omega; \cdot \vdash e' \checks \tau \gens \Phi,\Gamma'$.
  By AT-ExpI,
  $\Psi ; \Theta ; \Delta ; \Omega; \Gamma \vdash !e' \checks !\tau \gens \Phi,\Gamma$,
  which completes (1), and (2) follows by AT-Anno.        
  
  \item[(T-ExpE)] Suppose
  $\Psi ; \Theta ; \Delta ; \Omega ; \Gamma_1,\Gamma_2 \vdash \texttt{let } !x = e_1 \texttt{ in } e_2 : \tau'$ from
  $\Psi ; \Theta ; \Delta ; \Omega ; \Gamma_1 \vdash e_1  : !\tau$ and
  $\Psi ; \Theta ; \Delta ; \Omega, x : \tau ; \Gamma_2 \vdash e_2 : \tau'$.
  By IH, there are $e_1'$, $\Phi_1$, $\Gamma_1'$ such that
  $|e_1'| = e_1$,
  $\Theta ; \Delta \vDash \Phi_1$, and
  $\Psi ; \Theta ; \Delta ; \Omega ; \Gamma_1 \vdash e_1' \infers !\tau \gens \Phi_1,\Gamma_1'$.
  By Theorem~\ref{thm:admits-weaken}, there are $e_1''$, $\Phi_1'$, $\Gamma_1''$ such that
  $|e_1''| = |e_1'|$,
  $\Theta ; \Delta \vDash \Phi_1'$,
  $\Psi ; \Theta ; \Delta \vdash \Gamma_1'' \wknto (\Gamma_1,\Gamma_2) \setminus \Gamma_1$, and
  $\Psi ; \Theta ; \Delta ; \Omega; \Gamma_1,\Gamma_2 \vdash e_1'' \infers !\tau \gens \Phi_1',\Gamma_1''$.
  Again by IH, there are $e_2'$, $\Phi_2$, $\Gamma_2'$ such that
  $|e_2'| = e_2$,
  $\Theta ; \Delta \vDash \Phi_2$, and
  $\Psi ; \Theta ; \Delta ; \Omega, x: \tau ; \Gamma_2 \vdash e_2' \checks \tau' \gens \Phi_2,\Gamma_2'$.
  Then, since
  $\Psi ; \Theta ; \Delta \vdash \Gamma_1'' \wknto \Gamma_2$,
  we have by Theorem~\ref{admits-weaken} that there are $e_2''$, $\Phi_2'$, $\Gamma_2''$ such that
  $|e_2''| = |e_2|$,
  $\Theta ; \Delta \vDash \Phi_2'$, and
  $\Psi ; \Theta ; \Delta ; \Omega, x : \tau ; \Gamma_1'' \vdash e_2'' \checks \tau' \gens \Phi_2', \Gamma_2''$.
  Finally, by AT-ExpE, we have
  $\Psi ; \Theta ; \Delta ; \Omega ; \Gamma_1,\Gamma_2 \vdash \texttt{let } !x = e_1 \texttt{ in } e_2 \checks \tau' \gens \Phi_1' \wedge \Phi_2', \Gamma_2''$,
  as required for (1). (2) follows by AT-Anno.
  
  \item[(T-TAbs)] Suppose
  $\Psi ; \Theta ; \Delta ; \Omega ; \Gamma \vdash \Lambda \alpha. e : \forall \alpha : K.\tau$ by way of
  $\Psi, \alpha : K ; \Theta ; \Delta ; \Omega ; \Gamma \vdash e : \tau$.
  By IH, there are $e'$,$\Phi$, $\Gamma'$ such that
  $|e'| = e$,
  $\Theta ; \Delta \vDash \Phi$, and
  $\Psi, \alpha : K ; \Theta ; \Delta ; \Omega ; \Gamma \vdash e' \checks \tau \gens \Phi,\Gamma'$.
  By AT-TAbs
  $\Psi ; \Theta ; \Delta ; \Omega ; \Gamma \vdash e' \checks \forall \alpha : K. \tau \gens \Phi,\Gamma'$.
  This completes (1), and for (2), one use of AT-Anno suffices.
    
  \item[(T-TApp)] Suppose
  $\Psi ; \Theta ; \Delta ; \Omega ; \Gamma \vdash e [\tau'] : \tau[\tau'/\alpha]$ by way of
  $\Psi ; \Theta ; \Delta ; \Omega ; \Gamma \vdash e : \forall \alpha : K.\tau$ and
  $\Psi ; \Theta ; \Delta \vdash \tau' : K$.
  By IH, there are $e'$, $\Phi$, $\Gamma'$ such that
  $|e'| = e$,
  $\Theta ; \Delta \vDash \Phi$, and
  $\Psi ; \Theta ; \Delta ; \Omega ; \Gamma \vdash e' \infers \forall \alpha : K.\tau \gens \Phi,\Gamma'$.
  By Theorem~\ref{thm:kind-compl}, there is some $\Phi'$ such that
  $\Theta ; \Delta \vDash \Phi'$, and
  $\Psi ; \Theta ; \Delta \vdash \tau' : K \gens \Phi'$.
  Then, by AT-TApp,
  $\Psi ; \Theta ; \Delta ; \Omega ; \Gamma \vdash e' [\tau'] \infers \tau[\tau'/\alpha] \gens \Phi \wedge \Phi',\Gamma'$.
  This completes (2). For (1), we have
  by Theorem~\ref{thm:subty-refl}, there is some $\Phi''$ with
  $\Theta ; \Delta \vDash \Phi''$ and
  $\Psi ; \Theta ; \Delta \vdash \tau[\tau'/\alpha] \subty \tau[\tau'/\alpha] : \star \gens \Phi''$.
  Then, by AT-Sub,
  $\Psi ; \Theta ; \Delta ; \Omega ; \Gamma \vdash e' [\tau'] \infers \tau[\tau'/\alpha] \gens \Phi \wedge \Phi' \wedge \Phi'',\Gamma'$
  which completes (2).
  \item[(T-IAbs)] Suppose
  $\Psi ; \Theta ; \Delta ; \Omega ; \Gamma \vdash \Lambda i. e : \forall i : S. \tau$ by way of
  $\Psi ; \Theta, i : S ; \Delta ; \Omega ; \Gamma \vdash e : \tau$.
  By IH, there are $e'$, $\Phi$, $\Gamma'$ such that
  $|e'| = e$,
  $\Theta, i : S ; \Delta \vDash \Phi$, and
  $\Psi ; \Theta, i : S ; \Delta ; \Omega; \Gamma \vdash e' \checks \tau \gens \Phi,\Gamma'$.
  By AT-IAbs,
  $\Psi ; \Theta ; \Delta ; \Omega; \Gamma \vdash \Lambda i. e' \checks \forall i : S.\tau \gens \forall i : S.\Phi,\Gamma'$.
  Since $\Theta, i : S ; \Delta \vDash \Phi$, we also have $\Theta ; \Delta \vDash \forall i : S. \Phi$, which proves (1).
  (2) follows immediately by AT-Anno, as always.
  
  
  \item[(T-IApp)] Suppose
  $\Psi ; \Theta ; \Delta ; \Omega ; \Gamma \vdash e [I] : \tau[I/i]$ by way of
  $\Psi ; \Theta ; \Delta ; \Omega ; \Gamma \vdash e : \forall i : S.\tau$ and
  $\Theta ; \Delta \vdash I : S$.
  By IH, there are $e'$, $\Phi$, $\Gamma'$ such that
  $|e'| = e$,
  $\Theta ; \Delta \vDash \Phi$, and
  $\Psi ; \Theta ; \Delta ; \Omega; \Gamma \vdash e \infers \forall i : S. \tau \gens \Phi,\Gamma'$.
  By Theorem~\ref{thm:sort-compl}, there is some $\Phi'$ with
  $\Theta ; \Delta \vDash \Phi'$ and
  $\Theta ; \Delta \vdash I : S \gens \Phi'$.
  By AT-IApp,
  $\Psi ; \Theta ; \Delta ; \Omega; \Gamma \vdash e [I] \infers \tau[I/i] \gens \Phi \wedge \Phi',\Gamma'$.
  This completes (1), and (2) follows by AT-Anno.
  
  
  \item[(T-Fix)] Suppose $\Psi ; \Theta ; \Delta ; \Omega ; \Gamma \vdash \texttt{fix } x.e : \tau$ from
  $\Psi ; \Theta ; \Delta ; \Omega, x : \tau ; \cdot \vdash e : \tau$.
  By IH, there are $e'$, $\Phi$, $\Gamma'$ such that
  $|e'| = e$,
  $\Theta ; \Delta \vDash \Phi$, and
  $\Psi ; \Theta ; \Delta ; \Omega, x : \tau ; \cdot \vdash e' \checks \tau \gens \Phi,\Gamma'$.
  By AT-Fix,
  $\Psi ; \Theta ; \Delta ; \Omega ; \Gamma \vdash \texttt{fix } x.e' : \tau \gens \Phi,\Gamma$,
  as required. Since $|\texttt{fix }x.e'| = \texttt{fix }x.|e'| = \texttt{fix }x.e$, this completes (1). For (2),
  one use of AT-Anno suffices.
  
  \item[(T-Ret)] Suppose
  $\Psi ; \Theta ; \Delta ; \Omega ; \Gamma \vdash \texttt{ret}\; e : \M \, \phi(I,\vec{p}) \, \tau$ by way of
  $\Psi ; \Theta ; \Delta ; \Omega ; \Gamma \vdash e : \tau$.
  By IH, there are $e'$, $\Phi$, $\Gamma'$ such that
  $|e'| = e$,
  $\Theta ; \Delta \vDash \Phi$, and
  $\Psi ; \Theta ; \Delta ; \Omega ; \Gamma \vdash e \checks \tau \gens \Phi,\Gamma'$.
  By AT-Ret,
  $\Psi ; \Theta ; \Delta ; \Omega ; \Gamma \vdash \texttt{ret}\; e \checks \M \, \phi(I,\vec{p}) \, tau \gens \Phi,\Gamma'$.
  This completes (1), and (2) follows by a single use of AT-Anno.
  
  \item[(T-Bind)] 
  Suppose $\Psi ; \Theta ; \Delta ; \Omega ; \Gamma_1,\Gamma_2 \vdash \texttt{bind } x = e_1 \texttt{ in } e_2 : \M \, \phi(I,\vec{p} + \vec{q})\, \tau_2$ by way of
  $\Psi ; \Theta ; \Delta ; \Omega ; \Gamma_1 \vdash e_1 : \M \, \phi(I,\vec{p})\, \tau_1$, and
  $\Psi ; \Theta; \Delta ; \Omega ; \Gamma_2, x:\tau_1 \vdash e_2 : \M \, \phi(I,\vec{q})\, \tau_2$.
  By IH, there are $e_1'$, $\Phi_1$, $\Gamma_1'$ such that
  $|e_1'| = e_1$,
  $\Theta ; \Delta \vDash \Phi_1$, and
  $\Psi  ; \Theta ; \Delta ; \Omega ; \Gamma_1 \vdash e_1' \checks \M \, \phi(I,\vec{p}) \, \tau_1 \gens \Phi_1,\Gamma_1'$.
  Since $\Psi ; \Theta ; \Delta \vdash \Gamma_1,\Gamma_2 \wknto \Gamma_1$,
  we have by Theorem~\ref{thm:admits-weaken} $e_1''$, $\Gamma_1''$, $\Phi_1'$ such that
  $|e_1''| = |e_1'|$,
  $\Theta ; \Delta \vDash \Phi_1'$,
  $\Psi ; \Theta ; \Delta \vdash \Gamma_1'' \wknto (\Gamma_1,\Gamma_2) \setminus \Gamma_1$, and
  $\Psi ; \Theta ; \Delta ; \Omega ; \Gamma_1,\Gamma_2 \vdash e_1'' \checks \M \, \phi(I,\vec{p}) \, \tau_1 \gens \Phi_1',\Gamma_1''$.
  Noting that $(\Gamma_1,\Gamma_2) \setminus \Gamma_1 = \Gamma_2$, we have that
  $\Psi ; \Theta ; \Delta \vdash \Gamma_1'' \wknto \Gamma_2$.
  By IH, there are $e_2'$, $\Phi_2$, $\Gamma_2'$ such that
  $|e_2'| = e_2$,
  $\Theta ; \Delta \vDash \Phi_2$, and
  $\Psi ; \Theta; \Delta ; \Omega ; \Gamma_2, x:\tau_1 \vdash e_2' \checks \M \, \phi(I,\vec{q})\, \tau_2 \gens \Phi_2,\Gamma_2'$.
  But then by Theorem~\ref{thm:admits-weaken}, there are $e_2''$, $\Gamma_2''$, $\Phi_2'$ such that
  $|e_2''| = |e_2'|$,
  $\Theta ; \Delta \vDash \Phi_2'$, and
  $\Psi ; \Theta ; \Delta ; \Omega ; \Gamma_1'', x : \tau_1 \vdash e_2'' \infers \M \, \phi(I,\vec{q}) \, \tau_2 \gens \Phi_2',\Gamma_2''$.
  By Theorem~\ref{thm:subty-refl}, there is some $\Phi_3$ such that $\Theta ; \Delta \vDash \Phi_3$, and
  $\Psi ; \Theta ; \Delta \vdash \tau_2 \subty \tau_2 : \star \gens \Phi_3$.
  Then, by AS-Monad, $\Psi ; \Theta ; \Delta \vdash \M \, \phi(I,\vec{q}) \, \tau_2 \subty \M \, \phi(I,(\vec{p} + \vec{q}) - \vec{p}) \, \tau_2 : \star \gens \Phi_3 \wedge (I = I) \wedge (\vec{q} \leq (\vec{p} + \vec{q}) - \vec{p})$.
  Then, by AT-Sub,
  $\Psi ; \Theta ; \Delta ; \Omega ; \Gamma_1'', x : \tau_1 \vdash e_2'' \checks \M \, \phi(I,(\vec{p} + \vec{q}) - \vec{p}) \, \tau_2 \gens \Phi_2' \wedge  \Phi_3 \wedge (I = I) \wedge (\vec{q} \leq (\vec{p} + \vec{q}) - \vec{p}),\Gamma_2''$.
  Finally, by AT-Bind,
  $\Psi ; \Theta ; \Delta ; \Omega ; \Gamma_1,\Gamma_2 \vdash \texttt{bind } x = e_1'' \texttt{ in } e_2'' \checks \M \, \phi(I,\vec{p} + \vec{q})\, \tau_2 \gens \Phi_1' \wedge \Phi_2' \wedge \Phi_3 \wedge (I = I) \wedge (\vec{q} \leq (\vec{p} + \vec{q}) - \vec{p}), \Gamma_2''$, which completes (1). (2) follows immediately by AT-Anno.
  
  
  \item[(T-Tick)] Suppose $\Psi ; \Theta ; \Delta ; \Omega ; \Gamma \vdash \texttt{tick}[I|\vec{p}] : \M \, \phi(I,\vec{p})\, 1$ by way of
  $\Theta ; \Delta \vdash I : \mathbb{N}$ and
  $\Theta ; \Delta \vdash \vec{p} : \vec{\mathbb{R}^+}$.
  By Theorem~\ref{thm:sort-compl}, there is some $\Phi_1$ such that
  $\Theta ; \Delta \vDash \Phi_1$, and
  $\Theta ; \Delta \vdash I : \mathbb{N} \gens \Phi_1$.
  Again by Theorem~\ref{thm:sort-compl}, there is a $\Phi_2$ such that
  $\Theta ; \Delta \vDash \Phi_2$, and
  $\Theta ; \Delta \vdash \vec{p} : \vec{\mathbb{R}^+}$.
  By AT-Tick,
  $\Psi ; \Theta ; \Delta ; \Omega ; \Gamma \vdash \texttt{tick}[I|\vec{p}] \infers \M \, \phi(I,\vec{p})\, 1 \gens \Phi_1 \wedge \Phi_2$,
  which completes (2).
  For (1), it is easy to see that $\Psi ; \Theta ; \Delta \vdash \M \, \phi(I,\vec{p})\, 1 \subty \M \, \phi(I,\vec{p})\, 1 : \star \gens I = I \wedge \vec{p} \leq \vec{p}$, and of course $\Theta ; \Delta \vDash I = I \wedge \vec{p} \leq \vec{p}$.
  By AT-Sub,
  $\Psi ; \Theta ; \Delta ; \Omega ; \Gamma \vdash \texttt{tick}[I|\vec{p}] \checks \M \, \phi(I,\vec{p})\, 1 \gens \Phi_1 \wedge \Phi_2 \wedge  I = I \wedge \vec{p} \leq \vec{p}$, completing (1).
  
  \item[(T-Release)] Suppose
  $\Psi ; \Theta ; \Delta ; \Omega ; \Gamma_1,\Gamma_2 \vdash \texttt{release } x = e_1 \texttt{ in }e_2 : \M \, \phi(I,\vec{p}) \, \tau_2$ by way of
  $\Psi ; \Theta ; \Delta ; \Omega ; \Gamma_1 \vdash e_1 : [I | \vec{q}] \tau_1$ and
  $\Psi ; \Theta ; \Delta ; \Omega ; \Gamma_2, x : \tau \vdash e_2 : \M \, \phi(I,\vec{p} + \vec{q}) \, \tau_2$.
  By IH, there are $e_1'$, $\Phi_1$, $\Gamma_1'$ such that
  $|e_1'| = e_1$,
  $\Theta ; \Delta \vDash \Phi_1$, and
  $\Psi ; \Theta ; \Delta ; \Omega ; \Gamma_1 \vdash e_1' \infers [I|\vec{q}] \tau_1 \gens \Phi_1,\Gamma_1'$.
  Since $\Psi ; \Theta ;\Delta \vdash \Gamma_1,\Gamma_2 \wknto \Gamma_1$, we have by Theorem~\ref{thm:admits-weaken} that there are $e_1''$, $\Phi_1'$, $\Gamma_1''$ such that
  $|e_1''| = |e_1'|$,
  $\Theta ; \Delta \vDash \Phi_1'$,
  $\Psi ; \Theta ; \Delta \vdash \Gamma_1'' \wknto (\Gamma_1,\Gamma_2) \setminus \Gamma_1$, and
  $\Psi ; \Theta ; \Delta ; \Omega ; \Gamma_1,\Gamma_2 \vdash e_1'' \infers [I|\vec{q}] \tau_1 \gens \Phi_1',\Gamma_1''$.
  Then, by IH, there are $e_2'$, $\Phi_2$, $\Gamma_2'$ such that
  $|e_2'| = e_2$,
  $\Theta ; \Delta \vDash \Phi_2$, and
  $\Psi ; \Theta ; \Delta ; \Omega ; \Gamma_2, x : \tau \vdash e_2' \checks \M \, \phi(I,\vec{p} + \vec{q}) \, \tau_2 \gens \Phi_2,\Gamma_2'$.
  Since $\Psi ; \Theta ; \Delta \vdash \Gamma_1'' \wknto \Gamma_2$,
  we have by Theorem~\ref{thm:admits-weaken} that there are $e_2''$, $\Phi_2'$, $\Gamma_2''$ such that
  $|e_2''| = |e_2'|$,
  $\Theta ; \Delta \vDash \Phi_2'$, and
  $\Psi ; \Theta ; \Delta ; \Omega ; \Gamma_1'', x : \tau \vdash e_2'' \checks \M \, \phi(I,\vec{p} + \vec{q}) \, \tau \gens \Phi_2',\Gamma_2''$.
  Then, by AT-Relase,
  $\Psi ; \Theta ; \Delta ; \Omega ; \Gamma_1,\Gamma_2 \vdash \texttt{release } x = e_1'' \texttt{ in }e_2'' \checks \M \, \phi(I,\vec{p}) \, \tau_2 \gens (I = I) \wedge \Phi_1' \wedge \Phi_2',\Gamma_2'' \setminus \{x\}$.
  This completes (1), and (2) follows by AT-Anno.
  
  
  
  \item[(T-Store)] Suppose
  $\Psi ; \Theta ; \Delta ; \Omega ; \Gamma \vdash \texttt{store}[I|\vec{p}](e) : \M \, \phi(I,\vec{p}) \, ([I| \vec{p}] \, \tau)$ by way of
  $\Theta ; \Delta \vdash I : \mathbb{N}$,
  $\Theta ; \Delta \vdash \vec{p} : \vec{\mathbb{R}^+}$, and
  $\Psi ; \Theta ; \Delta ; \Omega ; \Gamma \vdash e : \tau$.
  By Theorem~\ref{thm:sort-compl}, there are $\Phi_1$, $\Phi_2$ such that
  $\Theta ; \Delta \vdash I : \mathbb{N} \gens \Phi_1$ and
  $\Theta ; \Delta \vdash \vec{p} : \vec{\mathbb{R}^+} \gens \Phi_2$
  with $\Theta ; \Delta \vDash \Phi_1 \wedge \Phi_2$.
  By IH, there are $e'$, $\Phi_3$, $\Gamma'$ such that
  $|e'| = e$,
  $\Theta ; \Delta \vDash \Phi_3$, and
  $\Psi ; \Theta ; \Delta ; \Omega; \Gamma \vdash e' \checks \tau \gens \Phi_3,\Gamma'$.
  By AT-Store,
  $\Psi ; \Theta ; \Delta ; \Omega ; \gamma \vdash \texttt{store}[I|\vec{p}](e') \checks \M \, \phi(I,\vec{p}) \, ([I| \vec{p}] \, \tau) \gens \Phi_1 \wedge \Phi_2 \wedge \Phi_3 \wedge (I \leq I \leq I) \wedge (\vec{p} \leq \vec{p} \leq \vec{p}),\Gamma'$.
  Of course, $\Theta ; \Delta \vDash (I \leq I \leq I) \wedge (\vec{p} \leq \vec{p} \leq \vec{p})$, and so
  we are done with (1). For (2), a single use of AT-Anno suffices.
  
    
  \item[(T-StoreConst)] Suppose
  $\Psi ; \Theta ; \Delta ; \Omega ; \Gamma \vdash \texttt{store}[I](e) : \M \, \phi(K,\texttt{const}(I)) \, ([I] \, \tau)$ by way of
  $\Theta ; \Delta \vdash I : \mathbb{N}$ and
  $\Psi ; \Theta ; \Delta ; \Omega ; \Gamma \vdash e : \tau$.
  By Theorem~\ref{thm:sort-compl}, there is some $\Phi_1$ such that
  $\Theta ; \Delta \vDash \Phi_1$, and
  $\Theta ; \Delta \vdash I : \mathbb{N} \gens \Phi_1$.
  By IH, there are $e'$, $\Phi_2$, and $\Gamma'$ such that
  $|e'| = e$,
  $\Theta ; \Delta \vDash \Phi_2$, and
  $\Psi ; \Theta ; \Delta \vdash e' \checks \tau \gens \Phi_2,\Gamma'$.
  Then, by AT-StoreConst,
  $\Psi ; \Theta ; \Delta ; \Omega ; \Gamma \vdash \texttt{store}[I](e') \checks \M \, \phi(K,\texttt{const}(I)) \, ([I] \, \tau) \gens \Phi_1 \wedge \Phi_2 \wedge (\texttt{const}(I) \leq \texttt{const}(I) \leq \texttt{const}(I)),\Gamma'$.
  Of course, $\Theta ; \Delta \vDash (\texttt{const}(I) \leq \texttt{const}(I) \leq \texttt{const}(I))$,
  which completes (1). For (2), one use of AT-Anno suffices.

  \item[(T-ReleaseConst)] Suppose
  $\Psi ; \Theta ; \Delta ; \Omega ; \Gamma_1,\Gamma_2 \vdash \texttt{release } x = e_1 \texttt{ in }e_2 : \M \, \phi(I,\vec{p}) \, \tau_2$ from
  $\Psi ; \Theta ; \Delta ; \Omega ; \Gamma_1 \vdash e_1 : [J] \tau_1$ and
  $\Psi ; \Theta ; \Delta ; \Omega ; \Gamma_2, x : \tau \vdash e_2 : \M \, \phi(I,\vec{p} + \texttt{const}(J)) \, \tau_2$.
  By IH, there are $e_1'$, $\Phi_1$, $\Gamma_1'$ such that
  $|e_1'| = e_1$,
  $\Theta ; \Delta \vDash \Phi_1$, and
  $\Psi ; \Theta ; \Delta ; \Omega ; \Gamma_1 \vdash e_1' \infers [J] \tau_1 \gens \Phi_1,\Gamma_1'$.
  Since $\Psi ; \Theta ; \Delta \vdash \Gamma_1,\Gamma_2 \wknto \Gamma_1$, we have by Theorem~\ref{thm:admits-weaken}
  that there are $e_1''$, $\Phi_1'$, $\Gamma_1''$ such that
  $|e_1''| = |e_1'|$,
  $\Theta ; \Delta \vDash \Phi_1'$,
  $\Psi ; \Theta ; \Delta \vdash \Gamma_1'' \wknto \Gamma_2$, and
  $\Psi ; \Theta ; \Delta ; \Omega; \Gamma_1,\Gamma_2 \vdash e_1'' \infers [J] \tau_1 \gens \Phi_1',\Gamma_1''$.
  Again by IH, there are $e_2'$, $\Phi_2$, $\Gamma_2'$ such that
  $|e_2'| = e_2$,
  $\Theta ; \Delta \vDash \Phi_2$, and
  $\Psi ; \Theta ; \Delta ; \Omega ; \Gamma_2, x : \tau \vdash e_2' \checks \M \, \phi(I,\vec{p} + \texttt{const}(J)) \, \tau_2 \gens \Phi_2,\Gamma_2'$.
  But, by Theorem~\ref{thm:admits-weaken}, there are $e_2''$, $\Phi_2'$, $\Gamma_2''$ such that
  $|e_2''| = |e_2'|$,
  $\Theta ; \Delta \vDash \Phi_2'$, and
  $\Psi ; \Theta ; \Delta ; \Omega ; \Gamma_1'', x: \tau \vdash e_2'' \checks  \M \, \phi(I,\vec{p} + \texttt{const}(J)) \, \tau_2 \gens \Phi_2',\Gamma_2''$.
  Then, by AT-ReleaseConst, 
  $\Psi ; \Theta ; \Delta ; \Omega ; \Gamma_1,\Gamma_2 \vdash \texttt{release } x = e_1'' \texttt{ in }e_2'' : \M \, \phi(I,\vec{p}) \, \tau_2 \gens \Phi_1' \wedge \Phi_2',\Gamma_2''$.
  This completes (1), and (2) follows by AT-Anno.
  
  \item[(T-Shift)] Suppose $\Psi ; \Theta ; \Delta ; \Omega ; \Gamma \vdash \texttt{shift}(e) : \M \, \phi(I,\vec{p}) \, \tau$ by way of
  $\Psi ; \Theta ; \Delta ; \Omega ; \Gamma \vdash e : \M \, \phi(I - 1,\lhd \vec{p}) \, \tau$ and
  $\Theta ; \Delta \vDash I \geq 1$.
  By IH, there are $e'$, $\Phi$, $\Gamma'$ such that
  $|e'| = e$,
  $\Theta ; \Delta \vDash \Phi$, and
  $\Psi  ; \Theta ; \Delta ; \Omega; \Gamma \vdash e' \checks \M \, \Phi(I-1,\lhd \vec{p}) \, \tau \gens \Phi,\Gamma'$.
  By AT-Shift,
  $\Psi  ; \Theta ; \Delta ; \Omega; \Gamma \vdash \texttt{shift}(e') \checks \M \, \Phi(I,\vec{p}) \, \tau \gens (I \geq 1) \wedge \Phi,\Gamma'$.
  Since $\Theta ; \Delta \vDash (I \geq 1) \wedge \Phi$, this completes (1), and (2) follows by AT-Anno.

  \item[(T-CImpI)] Suppose
  $\Psi ; \Theta ; \Delta ; \Omega ; \Gamma \vdash \Lambda .e : (\Phi' \Rightarrow \tau)$ by way of
  $\Psi ; \Theta ; \Delta,\Phi' ; \Omega ; \Gamma \vdash e : \tau$.
  By IH, there are $e'$, $\Phi$, $\Gamma'$ such that
  $|e'| = e$,
  $\Theta ; \Delta,\Phi' \vDash \Phi$, and
  $\Psi ; \Theta ; \Delta,\Phi'; \Omega ; \Gamma \vdash e' \checks \tau \gens \Phi,\Gamma'$.
  By AT-CImpI,
  $\Psi ; \Theta ; \Delta ; \Omega ; \Gamma \vdash \Lambda. e' \checks \Phi' \implies \tau \gens (\Phi' \to \Phi),\Gamma'$.
  This completes (1), and (2) follows by AT-Anno.
  
  \item[(T-CImpE)] Suppose
  $\Psi ; \Theta ; \Delta ; \Omega ; \Gamma \vdash e \{\} : \tau$ by way of
  $\Psi ; \Theta ; \Delta ; \Omega ; \Gamma \vdash e : \Phi' \Rightarrow \tau$ and
  $\Theta ; \Delta \vDash \Phi'$.
  By IH, there are $e'$, $\Phi$, $\Gamma'$ such that
  $|e'| = e$,
  $\Theta ; \Delta \vDash \Phi$, and
  $\Psi ; \Theta ; \Delta ; \Omega ; \Gamma \vdash e' \infers \Phi' \Rightarrow \tau \gens \Phi,\Gamma'$.
  By AT-CImpE,
  $\Psi ; \Theta ; \Delta ; \Omega ; \Gamma \vdash e \{\} \infers \tau \gens \Phi \wedge \Phi',\Gamma'$.
  Since $\Theta ; \Delta \vDash \Phi'$ and $\Theta ; \Delta \vDash \Phi$, we have $\Theta ; \Delta \vDash \Phi \wedge \Phi'$.
  This completes (2). For (2),
  by Theorem~\ref{thm:subty-refl}, there is some $\Phi''$ with
  $\Theta ; \Delta \vDash \Phi''$, and
  $\Psi ; \Theta ; \Delta \vdash \tau \subty \tau : \star \gens \Phi''$.
  Then, by AT-Sub,
  $\Psi ; \Theta ; \Delta ; \Omega ; \Gamma \vdash e \{\} \checks \tau \gens \Phi \wedge \Phi' \wedge \Phi'',\Gamma'$,
  which completes (1).
  
  \item[(T-CAndI)] Suppose
  $\Psi ; \Theta ; \Delta ; \Omega ; \Gamma \vdash <e> : \Phi' \amp \tau$ by way of
  $\Psi ; \Theta ; \Delta ; \Omega ; \Gamma \vdash e : \tau$ and
  $\Theta ; \Delta \vDash \Phi'$.
  By IH, there are $e'$, $\Phi$, $\Gamma'$ such that
  $|e'| = e$,
  $\Theta ; \Delta \vDash \Phi$, and
  $\Psi  ; \Theta ; \Delta ; \Omega ; \Gamma \vdash e' \checks \tau \gens \Phi,\Gamma'$.
  By AT-CAndI,
  $\Psi; \Theta ; \Delta ; \Omega ; \Gamma \vdash <e'> \checks \Phi' \amp \tau \gens \Phi \wedge \Phi',\Gamma'$.
  Since $\Theta ; \Delta \vDash \Phi$ and $\Theta ; \Delta \vDash \Phi'$, we have $\Theta ; \Delta \vDash \Phi \wedge \Phi'$,
  completing (1). (2) follows by AT-Anno.
  
  \item[(T-CAndE)] Suppose
  $\Psi ; \Theta ; \Delta ; \Omega ; \Gamma_1,\Gamma_2 \vdash \texttt{clet } x = e_1 \texttt{ in } e_2 : \tau'$ by way of
  $\Psi ; \Theta ; \Delta ; \Omega ; \Gamma_1 \vdash e_1 : \Phi' \amp \tau$, and
  $\Psi ; \Theta ; \Delta, \Phi' ; \Omega ; \Gamma_2, x : \tau \vdash e_2 : \tau'$.
  By IH, there are $e_1'$, $\Phi_1$, $\Gamma_1'$ such that
  $|e_1'| = e_1$,
  $\Theta ; \Delta \vDash \Phi_1$, and
  $\Psi ; \Theta ; \Delta ; \Omega ; \Gamma_1 \vdash e_1' \infers \Phi' \amp \tau \gens \Phi_1,\Gamma_1'$.
  By Theorem~\ref{thm:admits-weaken}, there are $e_1''$, $\Phi_1'$, $\Gamma_1''$ such that
  $|e_1''| = |e_1'|$,
  $\Theta ; \Delta \vDash \Phi_1'$,
  $\Psi ; \Theta ; \Delta \vdash \Gamma_1'' \wknto (\Gamma_1,\Gamma_2) \setminus \Gamma_1$, and
  $\Psi ; \Theta ; \Delta ; \Omega ; \Gamma_1,\Gamma_2 \vdash e_1'' \infers \Phi' \amp \tau \gens \Phi_1',\Gamma_1''$.
  Again by IH, there are $e_2'$, $\Phi_2$, $\Gamma_2'$ such that
  $|e_2'| = e_2$,
  $\Theta ; \Delta,\Phi' \vDash \Phi_2$, and
  $\Psi ; \Theta ; \Delta,\Phi' ; \Omega ; \Gamma_2,x : \tau \vdash e_2' \checks \tau' \gens \Phi_2,\Gamma_2'$.
  Since
  $\Psi ; \Theta ; \Delta \vdash \Gamma_1'' \wknto \Gamma_2$, we have by Theorem~\ref{thm:ctx-sub-wkn} and Theorem~\ref{thm:ctx-sub-subset2}
  that
  $\Psi ; \Theta ; \Delta,\Phi' \vdash \Gamma_1'',x : \tau \wknto \Gamma_2,x : \tau$,
  and so by Theorem~\ref{thm:admits-weaken}, there are $e_2''$, $\Phi_2'$, $\Gamma_2''$ such that
  $|e_2''| = |e_2'|$,
  $\Theta ; \Delta, \Phi' \vDash \Phi_2'$,
  $\Psi ; \Theta ; \Delta, \Phi' ; \Omega ;\Gamma_1'',x:\tau \vdash e_2'' \checks \tau' \gens \Phi_2',\Gamma_2''$.
  Then, by AT-CAndE,
  $\Psi ; \Theta ; \Delta ; \Omega ; \Gamma_1,\Gamma_2 \vdash \texttt{clet } x = e_1'' \texttt{ in } e_2'' : \tau' \gens \Phi_1' \wedge (\Phi' \to \Phi_2'),\Gamma_2''$.
  Since $\Theta ; \Delta \vDash \Phi_1'$ and $\Theta ; \Delta, \Phi' \vDash \Phi_2'$,
  we have $\Theta ; \Delta \vDash \Phi_1' \wedge (\Phi' \to \Phi_2')$
  This completes (1), and (2) follows by AT-Anno.
  
  \item[(T-Sub)] Suppose $\Psi ; \Theta ; \Delta ; \Omega ; \Gamma \vdash e : \tau$ from $\Psi ; \Theta ; \Delta ; \Omega ; \Gamma \vdash e : \tau'$ and $\Psi;\Theta;\Delta \vdash \tau' \subty \tau : \star$. By IH, there are $e',\Phi_1,\Gamma'$ so that $|e'| = e$, $\Theta ; \Delta \vDash \Phi_1$, and $\Psi ; \Theta ; \Delta ; \Omega ; \Gamma \vdash e' \infers \tau' \gens \Phi_1,\Gamma'$. By Theorem~\ref{thm:subty-compl}, there is $\Phi_2$ such that $\Theta ; \Delta \vDash \Phi_2$, and $\Psi ; \Theta ; \Delta \vdash \tau' \subty \tau : \star \gens \Phi_2$. By AT-Sub, $\Psi ; \Theta ; \Delta ; \Omega ; \Gamma \vdash e' \checks \tau \gens \Phi_1 \wedge \Phi_2,\Gamma'$, which completes (1). For (2), we apply AT-Anno to get $\Psi ; \Theta ; \Delta ; \Omega ; \Gamma \vdash (e' : \tau) \infers \tau \gens \Phi_1 \wedge \Phi_2,\Gamma'$, and are done.
  
  \item[(T-Weaken)] Suppose $\Psi ; \Theta ; \Delta ; \Omega' ; \Gamma' \vdash e : \tau$ from $\Psi ; \Theta ; \Delta ; \Omega ; \Gamma \vdash e : \tau$, $\Theta ; \Delta \vdash \Omega' \bdby \Omega$, and $\Theta ; \Delta \vdash \Gamma' \bdby \Gamma$. By IH, there are $e'$, $\Phi$, $\Gamma''$ so that $|e'| = e$, $\Theta ; \Delta \vDash \Phi$, and $\Psi ; \Theta ; \Delta ; \Omega ; \Gamma \vdash e : \tau \gens \Phi, \Gamma''$. By Theorem~\ref{thm:admits-weaken}, there are $e_1$, $\Phi_1$, $\Gamma_1$ so that $|e_1| = |e'|$, $\Theta ; \Delta \vDash \Phi_1$, and $\Psi ; \Theta ; \Delta ; \Omega' ; \Gamma' \vdash e_1 \checks \tau \gens \Phi_1,\Gamma_1$, and also that there are $e_2$, $\Phi_2$, $\Gamma_2$ so that $|e_2| = |e'|$, $\Theta ; \Delta \vDash \Phi_2$, and $\Psi ; \Theta ; \Delta ; \Omega' ; \Gamma' \vdash e_2 \infers \tau \gens \Phi_2,\Gamma_2$, which proves (1) and (2).
 
\end{itemize}

\end{proof}