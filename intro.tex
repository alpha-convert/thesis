%\red{(This intro is very tentative, and I'm holding off on fixing it until I know exactly what the rest of thesis looks like)}

%NOTES:
% Programmers want to understand/reason about not only the meaning and behavior of their programs,
% but also the resource usage
% How do we build tools to help them do this?
% They're already doing things, what exactly are they doing and why is it right?

As the importance of ubiquity of modern software increases, so too does its complexity. The burden of this complexity blowup lands squarely on the shoulders of software developers, who are asked to create increasingly intricate systems, with little extra help. In response to this, many developers have turned to languages and tooling to help ease the burden: this is exemplified by the rise of Rust, TypeScript, and other and modern strongly-typed languages \cite{rust} \cite{d-lang} \cite{typescript},  along with the explosion of interest in formally-verified software by way of theorem proving or other formal methods techniques \cite{ringer-et-al:qed}. While these practices go a long way to improve developer experience and confidence, they are limited in the domains of understanding that they improve. Notably, there are very few existing tools which help developers reason about the resource usage, algorithmic complexity, or performance of their software. The days when resource usage could be easily discerned from source code by eye are long gone, and yet very few techniques have stepped in to fill the void. Moreover, the techniques that software developers use to analyze algorithmic complexity in the absence of formal procedures, languages, or tooling are informal, fragile, and ad-hoc.

The long-term goal of the field of language-based resource analysis is to create languages, tools, and methods which fill this gap by providing programmers with the capability to statically reason about the resource cost of the programs they write. In this thesis, we will restrict our focus along two axes. First, we will only consider a particular resource: run-time cost.
% While other well-known resources such as memory or energy are of interest, program cost is, in the opinion of the author, primary among all resources in terms of importance. 
Second, we will only consider typed functional programs: the compositional nature of functional programs and the structure provided by types lend themselves greatly to the approaches to cost analysis we consider. Additionally, the techniques will all have the flavor of \textit{amortized} analysis \cite{tarjan:amortized-complexity}, a technique which we discuss in depth in \autoref{sec:amortized-primer} below.


%\subsection{Contributions}
%\red{Re-write this, no more categories}

We begin in \autoref{ch:lambda-amor} by exploring a project which serves as a tool for engineers to prove cost properties of their programs. In particular, we consider \lambdaamorimpl, a functional language for amortized cost analysis with an immensely expressive type system. The original creators of the core calculus on which \lambdaamorimpl is based \cite{rajani-et-al:popl21} considered it primarily as a unifying calculus for multiple resource-aware type systems. In this chapter, we consider it instead from a language designer's perspective and investigate what it takes to \textit{implement} the language. \lambdaamorimpl fills a point in the design space of resource aware languages analogous to that of dependently-typed languages like Agda \cite{norell:afp08}, Idris \cite{brady:jfp13}, and F* \cite{swamy-et-al:pldi13}, or refinement-typed languages like Liquid Haskell \cite{vazou-et-al:icfp14} in the space of verification techniques. \lambdaamorimpl's highly expressive type system  allows for intrinsic verification \cite{korkut-et-al:regex} of program costs, similarly to how dependent or refinement types allow for intrinsic verification of functional correctness. However, the same expressiveness that allows interesting amortized bounds to be verified in \lambdaamorimpl makes this implementation task no easy feat, and so we draw on a plethora of type system algorithmization techniques to eventually arrive at a calculus which is amenable to implementation. Finally, we present and evaluate an implementation of \lambdaamorimpl in OCaml.

Then, in \autoref{ch:rec-extr}, we move to considering a technique that functional programming practitioners \textit{already} use in analyzing the cost of their programs, namely the method of extracting and solving recurrences. As taught in computer science courses, the technique is easily understandable but fraught with potential for error and is entirely informal: there is no proven connection. Previous work by \citet{danner-et-al:plpv13} \cite{danner-et-al:icfp15} has put recurrence extraction on on firmer ground by formalizing the procedure and proving it correct: we refer to this as \textit{formal recurrence extraction}. With our formalization, there is no doubt that the extracted recurrences are meaningful, and give sound upper bounds on program cost. Formal recurrence extraction has been considered for a large and growing class of functional languages \cite{kavvos-et-al:popl20} \cite{danner2021denotational}. In this chapter, we extend it to handle recurrences for amortized cost.
\\

\subsubsection{Attributions and Funding}

\autoref{ch:lambda-amor} is based on work started during the author's Summer 2020 internship at The Max Planck Institute for Software Systems, and completed at Wesleyan University. The project was undertaken under the supervision of Prof. Deepak Garg. This work is unpublished.

\autoref{ch:rec-extr} was published and presented at the ACM SIGPLAN International Conference of Functional Programming, 2020. The work was conducted at Wesleyan University under the supervision of Profs. Daniel R. Licata and Norman Danner. This material is based upon work supported by the National Science Foundation under Grant Number CCF-1618203, the Air Force Office of Scientific Research under award number FA9550-16-1-0292, and the United States Air Force Research Laboratory under agreement number FA9550-15-1-0053.
%\red{Do I have to do attribution sentences here? This work was performed at ... with... funded by... presented at....}

%red{W/ grant thanks}

\section{Amortized Analysis Primer}
\label{sec:amortized-primer}
The intuition behind both developments in this thesis is rooted in amortized analysis, the classical algorithm analysis technique first presented by \citet{tarjan:amortized-complexity}. As such, we will provide a brief introduction to the technique here, in addition to presenting a few examples of its utility, one of which we will use as a running example.

Amortized analysis was initially conceived of as a technique for analyzing the worst-case cost of a sequence of operations on a data structure. Without amortization, such cost analyses can be very imprecise Naively, the worst-case cost of a sequence of operations is bounded by the sum of the worst-case cost for each operation. However, this usually fails to take into account internal data structure invariants which make it impossible (or not always true) that each operation in the sequence executes with its worst-case complexity. In short, amortized analysis allows us to peel back some of the abstraction barrier of a datatype in order to more closely analyze the cost of its operations in context. For a concrete example of this phenomenon, consider the (contrived, yet pedagogically useful) example of a binary counter shown in \autoref{fig:bin-counter}.

\begin{figure}
  \[
\begin{array}[t]{lcl}
\codeInc &:& \codebitlist \to \codebitlist \\
\codeInc\,[\,] &=& [1] \\
\codeInc\,(0 :: bs) &=& 1 :: bs \\
\codeInc\,(1 :: bs) &=& 0 :: \codeInc\,bs
\end{array}
\qquad
\begin{array}[t]{lcl}
\codeSet &:& \codenat \to \codebitlist \\
\codeSet\,0 &=& [\,] \\
\codeSet\,(S\,n) &=& \codeInc(\codeSet\,n)
\end{array}
\]
  %bit, counter, inc,set
  \caption{Binary Counter Data Structure}
  \label{fig:bin-counter}
\end{figure}

The type \texttt{bit} is either $0$ or $1$, and \texttt{bit list} represents a binary counter with the least significant bit at the head. The \texttt{inc} operation increments a counter by one, and the  \texttt{set} operation applies to $n$ iterates \texttt{inc} $n$ times, starting from the zero (empty) counter. To illustrate where a standard analysis goes wrong, we will naively analyze the cost of \texttt{set}. For simplicity, the only costly operations are cons (\texttt{::}) operations, which cost one unit of time each.

Given a counter of length $k$, \texttt{inc} performs at most $k + 1$ cons operations-- at worst, the counter is all ones, and \texttt{inc} must walk down the entire list flipping ones to zeroes, finishing by cons-ing a one onto the end. It's easy to see that after $i$ calls to \texttt{inc}, the counter has length bounded by $\left \lceil{\log_2 i}\right \rceil$, the number of needed for the binary representation of $i$. Thus, the total cost of \texttt{set n} is
\begin{align*}
  \sum_{i=1}^n \left\lceil{\log_2 i}\right \rceil + 1
  &\leq \sum_{i=1}^n \log_2 i + 2\\
  &\leq 2n + \sum_{i=1}^n \log_2 i\\
  &\leq 2n + \log_2(n!)\\
  &\leq 2n + n\log_2 n
\end{align*}

While it may be useful for some applications, this bound is not tight. To see why, consider the case where the counter is set to the value of $15_{10}$, or $1111_2$. A  call to increment on this counter costs the full $5$ cons operations, leaving the counter at $10000_2$. However, a subsequent call to \texttt{inc} only costs $1$ to flip the first bit. Indeed, very few calls to \texttt{inc} traverse the whole list-- the vast majority only flip one or two bits. This example illustrates the tension of doing these naive analysis, and provides the primary insight for amortized analysis: while one data structure operation may be expensive, it may also restructure the data structure in such a way which makes \textit{subsequent} operations cheaper than the worst case\footnote{
This is the genesis of the name \textit{amortized} analysis: expensive function operations effectively pay for subsequent ones to be cheaper, evoking \textit{amortization} from accounting.
}.

\subsection{Physicist's Method}
The most common method for operationalizing this insight is by the physicist's method of amortized analysis. This method proceeds by associating a data structure with a real-valued ``potential function" $\Phi : S \to \mathbb{R}$ on its states. The only restriction on potential functions is that they be nonnegative everywhere.
Then, for any sequence of data structure operations $f_i$ with costs $c_i$ and intermediate states $s_i$ (pictured in \autoref{fig:amortized-situation}, with $s_0$ initial and $s_i = f_i(s_{i-1})$), we may define the \textit{amortized cost} of each operation as:
$$
a_i = c_i + \Phi(s_i) - \Phi(s_{i-1})
$$

\begin{figure}
  \caption{The Generic Amortized Analysis Setup}
  \label{fig:amortized-situation}
\end{figure}

The amortized cost of an operation is the actual cost, plus the change in potential across it. When we sum the amortized cost of all the operations across the sequence, the sum telescopes:
\begin{align*}
  \sum_{i=1}^n a_i &= \sum_{i=1}^n c_i + \Phi(s_i) - \Phi(s_{i-1})\\
                   &= \Phi(s_n) - \Phi(s_0) + \sum_{i=1}^n c_i
\end{align*}
Then, if $\Phi(s_0) = 0$, we get that the total amortized cost is an upper bound on the total actual cost.
$$
\sum_{i=1}^n a_i \geq \sum_{i=1}^n c_i
$$
A good intuition for potential functions $\Phi$ is that $\Phi(s)$ represents the amount of work that subsequent operations have to do in order to modify the data structure from state $s$.
In practice, one usually picks potential functions such that when an expensive operation $f_i$ runs, $\Phi(s_{i-1})$ is very large, and $\Phi(s_i)$ is very small so that subsequent operations are cheap.

Returning to the binary counter example, the traditional choice of potential function is to take $\Phi(\texttt{xs})$ to be the number of 1-bits in \texttt{xs}.
For simplicity, we will denote the actual cost of a call to $\texttt{inc xs}$ by $C(\texttt{xs})$, and its amortized cost by $A(\texttt{xs}) = C(\texttt{xs}) + \Phi(\texttt{inc xs}) - \Phi(\texttt{xs})$.

\begin{theorem}
For all \texttt{xs:counter}, $A(\texttt{xs}) = 2$
\end{theorem}
\label{thm:bc-phys}
\begin{proof}
By a straightforward structural induction on \texttt{xs}.
\begin{itemize}
  \item ($\texttt{xs} = \texttt{[]}$): \texttt{inc []} does $1$ cons operation, $\Phi(\texttt{[]}) = 0$ and $\Phi(\texttt{[1]}) = 1$, so the amortized cost is $2$.
  \item ($\texttt{xs} = \texttt{y::ys}$): If $\texttt{y} = \texttt{0}$, then the \texttt{inc xs} does $1$ cons operation. Since $\Phi(\texttt{1::ys}) - \Phi(\texttt{y::ys}) = (1 + \Phi(\texttt{ys})) - \Phi(\texttt{ys}) = 1$, we again have that the amortized cost of \texttt{inc xs} is $2$. Now, suppose $\texttt{y} = \texttt{1}$. Then \texttt{inc xs} incurs $1$ cost from the cons operation, plus the cost of \texttt{inc ys}. So, we may compute:
  \begin{align*}
    A(\texttt{y::ys}) &= C(\texttt{y::ys}) + \Phi(\texttt{inc (y::ys)}) - \Phi(\texttt{y::ys})\\
    &= 1 + C(\texttt{ys}) + \Phi(\texttt{inc ys}) - (1 + \Phi(\texttt{ys}))\\
    &= C(\texttt{ys}) + \Phi(\texttt{inc ys}) - \Phi(\texttt{ys})\\
    &= A(\texttt{ys)}
  \end{align*}
  But by the inductive hypothesis, $A(\texttt{ys}) = 2$, which completes the proof.
\end{itemize}
\end{proof}

An immediate corollary of this fact is that the amortized cost of \texttt{set n} is bounded by $2n$, by the same telescoping argument as before. Most importantly, since the binary counter \texttt{[0]} has potential $0$, the amortized cost of \texttt{set n} is an upper bound on the \textit{actual} cost of \texttt{set n}. This last step is crucial. A priori, amortized costs give no information about the actual costs of program execution, and are only a bound on the actual cost if the final potential is greater than the initial.

\subsubsection{Single Function and Gas Tank Analyses}
Often, amortized analysis is applied to individual functions, rather than a sequence. This may be thought of as the length-one case of amortized analysis.
Given a function \texttt{f:a->b} and potential functions $\Phi_\texttt{a} : \texttt{a} \to \R$ and $\Phi_\texttt{b} : \texttt{b} \to \R$, we may define
the amortized cost of \textbf{f} as $A_\texttt{f}(\texttt{x}) = C_\texttt{f}(\texttt{x}) + \Phi_\texttt{b}(\texttt{f x}) - \Phi_\texttt{a}(\texttt{x})$,
where $C_\texttt{f}(\texttt{x})$ is the cost of \texttt{f}. As long as $\Phi_\texttt{b}(\texttt{f x}) \geq \Phi_\texttt{a}(\texttt{x})$, the amortized cost is an upper bound on the actual cost.

A common variation on this concept is to pick the potential functions such that the amortized cost is \textit{zero}. Then, the actual cost $C_\texttt{f}(\texttt{x})$ is exactly $\Phi_\texttt{a}(\texttt{x}) - \Phi_\texttt{b}(\texttt{f x})$. The usual intuition here is to think of the available potential as a sort of ``gas tank" which the function must siphon from to do work. Then, the total gas used, $\Phi_\texttt{a}(\texttt{x}) - \Phi_\texttt{b}(\texttt{f x})$, is an upper bound on the amount of work done by the function.


\subsection{Banker's Method}
While the physicist's method is powerful, some situations call for a more fine-grained analysis. In this case, we employ the so-called banker's method of amortized analysis. The banker's method works by introducing imaginary ``credits" to a data structure, which may be ``attached" to the values in a program. These credits are thought of to interact with the cost model of the language in a special way: credits can always be created from thin air at a cost of one unit of time, and then they can subsequently be discarded or ``spent" to decrease execution cost by one unit. In the setting of the banker's method, this is what we mean when we refer to ``amortized cost"-- the real cost of an operation, plus the cost of creating and spending credits along the way. Crucially, if an operation begins with no credits available, then the amortized cost must be an upper bound on the actual cost since credits must be created (incurring a cost of $1$) before they can be spent (decreasing the cost by $1$). All of this is best illustrated by returning to the binary counter example.

We begin by enforcing a credit invariant on values of type \texttt{counter}: every \texttt{1} bit must have a credit attached. It's worth confirming that the increment function is able to maintain this invariant: if the counter is empty, we spawn a credit, attach it to a \texttt{1} bit, and cons it to the front of the list.
If the least significant bit is \texttt{0}, we again spawn a credit, flip the bit to \texttt{1}, and attach the credit. Finally, if the least significant bit is \texttt{1}, then we detatch its credit, spend it, recurse down the tail, and finish by cons-ing a \texttt{0} onto the front. Of course, none of this is manifest in the code
\footnote{
Readers familiar with concurrent separation logic might find this idea familiar: credits are a form of ghost state.
}.
Just like with potential in the physicist's method, these proofs must happen off to the side on paper.

Finally, we may perform the analysis itself.

\begin{theorem}
For all \texttt{xs:counter} satisfying the credit invariant, the amortized cost of \texttt{inc xs} is $2$.
\end{theorem}
\label{thm:bc-bank}
\begin{proof}
We proceed by structural induction on \texttt{xs}.
\begin{itemize}
  \item ($\texttt{xs} = \texttt{[]}$): \texttt{inc []} does $1$ cons operation and spawns one credit, for an amortized cost of $2$.
  \item ($\texttt{xs} = \texttt{y::ys}$): If $\texttt{y} = \texttt{0}$, then \texttt{inc xs} does one cons operation and spawns a single credit, for an amortized cost of $2$. Finally, if $\texttt{y} = \texttt{1}$, then \texttt{inc xs} makes a single recursive call \texttt{inc ys}, which has amortized cost $2$, by inductive hypothesis. But then, the function spends the credit attached to \texttt{y}, which cancels out the cost $1$ incurred by cons-ing a \texttt{0} onto the result of the recursive call. In total, this case has $2$ amortized cost, as required.
\end{itemize}
\end{proof}

\subsection{Comparisons}

The reader may note that the proof of \autoref{thm:bc-bank} was remarkably similar to the proof of \autoref{thm:bc-phys}. This is, unsurprisingly, not by coincidence. The reason for this similarity is that the banker's method can be thought of as a concretization of the physicist's method. Rather than ``globally" assigning potential to the states of a data structure, the banker's method ``localizes" the potential, thought of as discrete credits, on  specific values in the state. Because of this, analyses with the banker's method have an ``operational" feel to them, while analyses using the physicist's method have a more ``calculational" flavor.

On the whole, the two methods of amortized analysis are essentially equivalent in power. Given a banker's method analysis, we may turn it into a physicist's method analysis by taking the potential function to be the total number of credits. Conversely, physicist's method analyses can be converted to use the banker's method by maintaining the invariant that $ \left\lceil{\Phi(\texttt{s})}\right \rceil$ credits be kept on the value \texttt{s}.

Proofs using the banker's method are often more tedious and traditionally less formal. In \autoref{ch:rec-extr}, we present a formalization of the banker's method by way of recurrence extraction. \autoref{ch:lambda-amor} presents \lambdaamor, which is based primarily on the physicist's method.

\section{Substructural and Modal Types Primer}
\label{sec:modal-and-substructural}
While \autoref{ch:rec-extr} is primarily concerned with approaching cost analysis from a recurrence extraction angle, it will, like \autoref{ch:lambda-amor}, involve the creation of a type system. The two type systems under consideration, those of \lambdaA and \lambdaamor, share a major feature which is common to most developments in resource analysis: they are both \textit{affine substructural} type systems. Most type systems have three ``non-logical" rules governing the behavior of their typing context. The rule that allows contexts to be freely considered up to permutations is called \textit{exchange}, the rule allowing variables to be dropped from the context is called \textit{weakening}, and the rule which allows variables to be duplicated is referred to as \textit{contraction}. The three rules are sometimes explicitly included in the list of rules generating a typing judgment, but more often than not they are omitted and derived as admissible rules.

Substructural type systems are ones in which some or all of the traditional structural rules governing typing contexts are disallowed. Disallowing all of them yields ordered types, and allowing only exchange yields linear types. Most importantly for this thesis however, is the combination of exchange and weakening (but not contraction), which yields an \textit{affine} type system. In an affine type system, contexts are sets, where each variable from the context may be used at most once.

The rules of an affine type system are set up in such a way that the contraction rule is inadmissible. In particular, every multi-premise rule ``splits" the context so that a variable cannot be used by more than one sub-term. For instance, consider the usual product introduction rule of a fully-structural type system:
$$
\inferrule{\Gamma \vdash e_1 : A \\ \Gamma \vdash e_2 : B}{\Gamma \vdash (e_1,e_2) : A \times B}
$$
is transformed into the following rule to support affine types:
$$
\inferrule{\Gamma_1 \vdash e_1 : A \\ \Gamma_2 \vdash e_2 : B}{\Gamma_1,\Gamma_2 \vdash (e_1,e_2) : A \otimes B}
$$
The context must be divided into two disjoint parts $\Gamma_1$ and $\Gamma_2$ to be given to the two premises\footnote{
The name of the connective changes also-- in a fully structural type system, positive and negative products are isomorphic, and thus written $\times$. Passing to a substructural type system disentangles the two notions.
}.

Affine types are a natural object of study in the resource-usage literature, since they naturally enforce a resource-usage restriction: using a variable ``consumes" it, which corresponds to the consumption of the resources the variable holds. In the type systems we will study for the remainder of this thesis, the notion of resource which our affine type systems will track will be the credits and potentials of amortized analysis. Indeed, non-duplication of variables in an affine type system can be leveraged to enforce the non-duplication requirement of credits and potential in the banker's and physicist's method of amortized analysis.

Of course, this restriction on the usage of variables is very strong from a programming perspective. Even incredibly simple programs often require the re-use of a single variable. The traditional solution to this is the introduction of a new type $!A$, which represents values of type $A$ that can be used arbitrarily many times. This $!$, usually referred to as the \textit{exponential modality} is our first example of a modality, or a unary operator on types. To ``implement" this modality in the type system, one usually adds a second context to the typing judgment of so-called exponential variables, which can be used multiple times. This context is fully structural-- it is not split in multi-premise rules. For instance, the product introduction rule becomes something like:
$$
\inferrule{\Omega;\Gamma_1 \vdash e_1 : A \\ \Omega; \Gamma_2 \vdash e_2 : B}{\Omega; \Gamma_1,\Gamma_2 \vdash (e_1,e_2) : A \otimes B}
$$
and we also add an exponential variable rule to use variables from the $\Omega$ context.
$$
\inferrule{ }{\Omega, x : A; \Gamma \vdash x : A}
$$
The modality itself is governed by a pair of simple introduction and elimination rules. A term can be made exponential with the introduction rule, so long as it does not depend on any affine resources:
$$
\inferrule{\Omega ; \cdot \vdash e : A}{\Omega ; \Gamma \vdash e : \, !A}
$$
The elimination rule allows for a term of type $!A$ to be bound as one of type $A$ in the exponential context, and thus be used many times:
$$
\inferrule{\Omega ; \Gamma_1 \vdash e : \,!A \\ \Omega, x :: A ; \Gamma_2 \vdash e' : B}{\Omega ; \Gamma_1,\Gamma_2 \vdash \texttt{let} \, !x \, = \, e \,\texttt{in}\, e' : B}
$$

Both \lambdaA and \lambdaamor make use of versions of this modality, but \lambdaA generalizes it to also allow $n$-affine types, whose values may be used at most $n$ times. In addition, each type system uses a \textit{graded} modality to quantify the usage of potential and credits. While the presentation (and terminology) is slightly different, the $!_r A$ modality of \lambdaA and the $[I] \, A$ modality of \lambdaamor can be thought of as essentially one and the same. 