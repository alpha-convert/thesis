\usetikzlibrary
{
	calc ,
	matrix ,
	arrows ,
	angles ,
	quotes ,
	intersections ,
	positioning ,
	fit ,
	shapes.geometric ,
	shapes.misc ,
	decorations ,
	decorations.markings ,
	decorations.pathmorphing ,
	decorations.pathreplacing ,
	decorations.text %,
	%graphdrawing % currently works only with LuaTex
}

\usepgflibrary
{
	arrows.meta
}

%\usegdlibrary % currently works with LuaTex only
%{
%	trees ,
%	force
%}


\usepackage{xcolor}

% customizations:
\tikzset
{
	% default diagram style
	diagram/.style =
	{
		line cap = butt ,
		line join = bevel ,
		arrows = -> ,
		> = angle 60 ,
		auto = left ,
		text height = 1.5ex , % i had this commented for some reason, but don't remember why.  needed for vertical label text alignment, i think.
		text depth = 0.25ex , % needed for baseline alignment (TiKZ quirk)
		align = center ,
	} ,
	% node style for code diagrams:
	code node/.append style =
	{
		every node/.append style =
		{
			execute at begin node = \begin{texttt} ,
			execute at end node = \end{texttt}
		}
	} ,
	% style for code diagrams:
	code diagram/.style =
	{
		diagram ,
		code node
	} ,
	% node style for math diagrams:
	math node/.append style =
	{
		every node/.append style =
		{
			execute at begin node = \begin{math} ,
			execute at end node = \end{math}
		}
	} ,
	% style for math diagrams:
	math diagram/.style =
	{
		diagram ,
		math node
	} ,
	% style for string diagrams:
	string diagram/.style =
	{
		math diagram ,
		arrows = - ,
	} ,
	% node placement with center anchor
	online/.style =
	{
		shape = rectangle ,
		rounded corners = 2mm ,
		inner sep = 1.5pt ,
		fill = white ,
		opacity = 1.0 ,
		anchor = center	
	} ,
	% style for double-shafted arrows
	double arrow/.style =
	{
		double distance between line centers = 2pt ,
		-implies % to get a single tip on a double arrow (TiKZ quirk)
	} ,
	% style for equality arrows
	equals/.style =
	{
		double distance between line centers = 2pt ,
		-implies , % to get a single tip on a double arrow (TiKZ quirk)
		arrows = -
	} ,
	% style for maps-to arrows
	mapto/.append style =
	{
		arrows = |-> ,
	} ,
	% style for paths
	paph/.append style =
	{
		decorate ,
		decoration = {snake , segment length = 5mm , amplitude = 0.5mm} ,
	} ,
	% style for left-bending inclusion arrows
	includel/.append style =
	{
		arrows = Hooks[right]-> ,
	} ,
	% style for right-bending inclusion arrows
	includer/.append style =
	{
		arrows = Hooks[left]-> ,
	} ,
	% style for monic arrows
	monomorphism/.append style =
	{
		arrows = >-> ,
	} ,
	% style for epic arrows
	epimorphism/.append style =
	{
		arrows = ->> ,
	} ,
	% style for cofibration arrows
	cofibration/.append style =
	{
		arrows = >-> ,
	} ,
	% style for fibration arrows
	fibration/.append style =
	{
		arrows = ->> ,
	} ,
	% style for weak equivalences
	equivalence/.append style =
	{
		decoration = {markings , mark = at position 1/2 with {\node[online , transform shape] {∼};}} ,
		postaction = {decorate} ,
	} ,
	% style for proarrows
	proarrow/.append style =
	{
		decoration = {markings , mark = at position 1/2 with {\draw [solid , -] (0 , -0.6ex) to (0 , 0.6ex);}} ,
		postaction = {decorate} ,
	} ,
	% style for string diagrams
	string/.append style =
	{
		arrows = - ,
	} ,
	% style for cofibration strings
	cofibration string/.append style =
	{
		decoration = {markings , mark = at position 1/4 with {\arrow{>}} , mark = at position 3/4 with {\arrow{>}}} ,
		postaction = {decorate} ,
	} ,
	% style for fibration strings
	fibration string/.append style =
	{
		decoration = {markings , mark = at position 4/8 with {\arrow{>}} , mark = at position 5/8 with {\arrow{>}}} ,
		postaction = {decorate} ,
	} ,
	% style for faking over-crossing
	overcross/.append style =
	{
		preaction = {draw = white , - , line width = 5pt}
	} ,
	% style for smaller labels
	label/.append style =
	{
		font = \scriptsize
	} ,
	% style for incoming string labels
	incoming/.append style =
	{
		pos = 0 ,
		anchor = south
	} ,
	% style for outgoing string labels
	outgoing/.append style =
	{
		pos = 1 ,
		anchor = north
	} ,
	% style for string labels entering from below
	outcoming/.append style =
	{
		pos = 0 ,
		anchor = north
	} ,
	% style for string labels exiting from above
	ingoing/.append style =
	{
		pos = 1 ,
		anchor = south
	} ,
	% style for labels for strings entering from above
	fromabove/.append style =
	{
		pos = 0 ,
		anchor = south
	} ,
	% style for labels for strings exiting to below
	tobelow/.append style =
	{
		pos = 1 ,
		anchor = north
	} ,
	% style for labels for strings entering from below
	frombelow/.append style =
	{
		pos = 0 ,
		anchor = north
	} ,
	% style for string labels exiting to above
	toabove/.append style =
	{
		pos = 1 ,
		anchor = south
	} ,
	% style for incoming string labels from left in double category
	fromleft/.append style =
	{
		pos = 0 ,
		anchor = east
	} ,
	% style for outgoing string labels to right in double category
	toright/.append style =
	{
		pos = 1 ,
		anchor = west
	} ,
	% style for incoming string labels from right in double category
	fromright/.append style =
	{
		pos = 0 ,
		anchor = west
	} ,
	% style for outgoing string labels to left in double category
	toleft/.append style =
	{
		pos = 1 ,
		anchor = east
	} ,
	% style for arrows quantified
	forall/.append style =
	{
		line width = 0.5pt ,
		%color = gray!75 % affects the labels too
	} ,
	% style for arrows asserted to exist
	exists/.append style =
	{
		densely dashed
	} ,
	% style for arrows that are part of the structure under consideration (deemphasized)
	structural/.append style =
	{
		opacity = 2/3 ,
		%color = gray!75 % affects the labels too
	} ,
	% styles for string diagrams
	0-cell/.style =
	{
		shape = rectangle ,
		rounded corners = 2mm ,
		inner sep = 2pt
	} ,
	1-cell/.style =
	{
		shape = rectangle ,
		rounded corners = 2mm ,
		inner sep = 1.5pt ,
		fill = white ,
		opacity = 1.0 ,
		anchor = center	
	} ,
	% general 2-cell
	2-cell/.style =
	{
		draw = black!75 ,
		fill = white ,
		opacity = 0.9 ,
		inner sep = 0.75mm ,
		minimum size = 4.0mm
	} ,
	% bead 2-cell
	bead/.style =
	{
		2-cell ,
		shape = rectangle ,
		rounded corners = 2.0mm
	} ,
	% diamond 2-cell
	diamond/.style =
	{
		2-cell ,
		shape = diamond
	} ,
	% dirac bra 2-cell
	bra/.style =
	{
		2-cell ,
		shape = isosceles triangle ,
		isosceles triangle apex angle = 70 ,
		shape border rotate = -90 ,
		minimum size = 5mm
	} ,
	% dirac ket 2-cell
	ket/.style =
	{
		2-cell ,
		shape = isosceles triangle ,
		isosceles triangle apex angle = 70 ,
		shape border rotate = 90 ,
		minimum size = 6mm
	} ,
	% box 2-cell
	box/.style =
	{
		2-cell ,
		shape = rectangle
	} ,
	% white dot 2-cell
	white dot/.style =
	{
		2-cell ,
		shape = circle ,
		minimum size = 1.5mm ,
		fill = white
	} ,
	% black dot 2-cell
	black dot/.style =
	{
		2-cell ,
		shape = circle ,
		minimum size = 1.5mm ,
		fill = gray
	} ,
	% plain dot 2-cell
	dot/.style =
	{
		2-cell ,
		shape = circle ,
		inner sep = 0mm ,
		minimum size = 0.5mm ,
		fill = black
	} ,
	% sheet style for surface diagrams
	sheet/.style =
	{
		draw = black!75 ,
		fill = gray!10 ,
		opacity = 0.9 ,
		fill opacity = 0.5
	} ,
	% torn edge for surfaces
	torn/.append style =
	{
		decoration = {random steps , segment length = 2pt , amplitude = 1pt}
	} ,
	% annotations for edge diagrams
	edge annotation/.append style =
	{
		anchor = west
	} ,
	% general style for rewrite callout
	callout/.append style =
	{
		minimum height = 2mm ,
		minimum width = 4mm ,
		draw ,
		draw opacity = 0.25
	} ,
	% style for rewrite callout - pre
	callout-pre/.append style =
	{
		callout ,
		inner sep = 1.2mm ,
		solid
	} ,
	% style for rewrite callout - post
	callout-post/.append style =
	{
		callout ,
		inner sep = 0.8mm ,
		dashed
	} ,
	% style for pullback wedge:
	pullback/.pic =
	{
		\draw [semithick , arrows = -] (-0.1 , 0.1) -- (0.1 , 0.1) -- (0.1 , -0.1) ;
	} ,
	% style for pushout wedge:
	pushout/.pic =
	{
		\draw [semithick , arrows = -] (-0.1 , 0.1) -- (-0.1 , -0.1) -- (0.1 , -0.1) ;
	} ,
}

% path replacing decoration for offset path
% from https://tex.stackexchange.com/questions/305161/perpendicular-shift-for-paths
\pgfdeclaredecoration{simple line}{start}
{
  \state{start}[width = +0pt,
                next state=step]{
    \pgfpathmoveto{\pgfpoint{0pt}{0pt}}
  }
  \state{step}[auto end on length    = 3pt,
               auto corner on length = 3pt,               
               width=+1pt]
  {
    \pgfpathlineto{\pgfpoint{1pt}{0pt}}
  }
  \state{final}
  {}
}
