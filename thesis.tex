\def\fileversion{1.0}
\def\filedate{April 2021}

\documentclass[final]{westhesis}
\makeatletter
\newcommand{\@chapapp}{\relax}%
\makeatother
\usepackage[utf8]{inputenc}
\usepackage{import}
%\usepackage[margin=30pt]{geometry}
\usepackage{amsmath}
%\usepackage{amsfonts}
%\usepackage{amssymb}
%\usepackage{graphicx}
%\usepackage{fourier}
%\usepackage{amsmath}
%\usepackage{amsfonts}
%\usepackage{amsthm}
%\usepackage{amssymb}
\usepackage{cmll}
%\usepackage{minted}
\usepackage{MnSymbol}
\usepackage{relsize}
\usepackage{enumerate}
\usepackage{proof}
\usepackage{mathpartir}
\usepackage{stmaryrd}
\usepackage{syntax}
%\usemintedstyle{bw}
\usepackage{graphics}
\usepackage{xcolor}
\usepackage[numbers]{natbib}
\usepackage{xspace}
\usepackage{mathrsfs}
\usepackage{tikz}
\import{./}{tikzstyle}
\usepackage[toc,page]{appendix}
\usepackage{chngcntr}
\usepackage{thmtools} 
\usepackage{thm-restate}


%for proofs
\usepackage{mdframed}
\usepackage{changepage}

%hyperref wants to be last :(
\usepackage[colorlinks,backref]{hyperref}
\usepackage[noabbrev,capitalize]{cleveref}


\title{Languages with Potential:\\
\normalsize{Types \& Recurrences for Formal Amortized Analysis}}
\author{Joseph W. Cutler}
\department{Mathematics and Computer Science}
\submitdate{April 2021}
\advisor{Daniel R. Licata}



%\newtheorem{theorem}{Theorem}[section]
\declaretheorem[name=Theorem,numberwithin=section]{theorem}
\newtheorem{corollary}{Corollary}[section]
\newtheorem{lemma}{Lemma}[section]
\newtheorem{definition}{Definition}[section]

\renewcommand{\chapterautorefname}{Chapter}
\renewcommand{\sectionautorefname}{Section}
\renewcommand{\theoremautorefname}{Theorem}
\renewcommand{\subsectionautorefname}{Section}
\renewcommand{\subsubsectionautorefname}{Section}
\renewcommand{\figureautorefname}{Figure}
\newcommand*{\Appendixautorefname}{Appendix}


\newcommand{\tytrans}[1]{\llbracket #1 \rrbracket}
\newcommand{\tertrans}[1]{\left\lVert#1\right\rVert}
\newcommand{\norm}[1]{\left\lVert#1\right\rVert}
\newcommand{\angles}[1]{\left\llangle #1 \right\rrangle}
\newcommand{\eval}{\downarrow}
\newcommand{\val}{\texttt{val}}
\newcommand{\bdby}{\sqsubseteq}
\newcommand{\wknto}{\sqsubseteq}
\newcommand{\valbd}{\sqsubseteq_{\texttt{val}}}
\newcommand{\subbd}{\sqsubseteq_{\texttt{sub}}}
\newcommand{\subst}[2]{\left[#1 \middle/ #2 \right]}

\newcommand{\savename}{\ensuremath{\texttt{save}}}
\newcommand{\xfername}{\ensuremath{\texttt{transfer}}}
\newcommand{\waitname}[0]{\ensuremath{\texttt{create}}}
\newcommand{\discname}{\texttt{spend}}
\let\createname\waitname
\let\spendname\discname
\newcommand{\save}[3]{\ensuremath{\texttt{save}^{#1}_{#2} \; #3}}
\newcommand{\disc}[2]{\ensuremath{\texttt{disc}_{#1} \; #2}}
\newcommand{\wait}[2]{\ensuremath{\texttt{wait}_{#1} \; #2}}
\newcommand{\tsfer}[6]{\ensuremath{\texttt{transfer}_{#1} \, !^{#2}_{#3} \, #4 = #5 \; \texttt{to} \; #6}}
\newcommand{\case}[5]{\ensuremath{\texttt{case} \, (#1,\, #2. #3 \, , \, #4 . #5})}
\newcommand{\inl}[1]{\ensuremath{\texttt{inl} \, #1}}
\newcommand{\inr}[1]{\ensuremath{\texttt{inr} \, #1}}
\newcommand{\elist}{\ensuremath{\texttt{[]}}}
\newcommand{\cons}[2]{\ensuremath{#1 \, :: \, #2}}
\newcommand{\listty}[1]{\ensuremath{\texttt{[}#1\texttt{]}}}
\newcommand{\delay}[1]{\ensuremath{\texttt{delay} \, #1}}
\newcommand{\susp}[1]{\ensuremath{\texttt{susp} \, #1}}
\newcommand{\force}[1]{\ensuremath{\texttt{force} \, #1}}
\newcommand{\psplit}[4]{\ensuremath{\texttt{split}(#1, \, #2.#3.#4)}}
\newcommand{\unit}{\ensuremath{\texttt{()}}}
\newcommand{\N}{\ensuremath{\mathbb{N}}}
\newcommand{\nrec}[3]{\ensuremath{\texttt{nrec}\left(#1,#2,#3\right)}}
\newcommand{\lrec}[3]{\ensuremath{\texttt{lrec}\left(#1,#2,#3\right)}}
\newcommand{\inj}{\overline}
\newcommand{\tick}[1]{\ensuremath{\texttt{tick} \; \, ; \; #1}}
\newcommand{\amp}{\ensuremath{\&}}
\newcommand{\amppair}[2]{\ensuremath{\langle #1,#2 \rangle}}
\newcommand{\scott}[1]{\ensuremath{\llbracket #1 \rrbracket}}
\newcommand{\snrec}{\ensuremath{\texttt{snrec}}}
\newcommand{\slrec}{\ensuremath{\texttt{slrec}}}
\newcommand{\scase}{\ensuremath{\texttt{scase}}}
\newcommand{\ret}[1]{\ensuremath{\texttt{ret} \, #1}}
\newcommand{\attach}[1]{\ensuremath{\texttt{attach} \, #1}}
\newcommand{\asplit}[5]{\ensuremath{\texttt{split}_{#1}(#2, \, #3.#4.#5)}}
\newcommand{\ccase}[5]{\ensuremath{\texttt{case} \, (#1,\, #2. #3 \, , \, #4 . #5})}
\newcommand{\acase}[6]{\ensuremath{\texttt{case}_{#1} \, (#2,\, #3. #4 \, , \, #5 . #6})}

\newcommand{\bbbc}{\mathbb{C}}
\newcommand{\tree}[1]{\ensuremath{\texttt{tree}\left(#1\right)}}
\newcommand{\trec}[6]{\ensuremath{\texttt{trec}\left(#1,#2,#3,#4,#5,#6\right)}}
\newcommand{\set}[1]{\ensuremath{\texttt{set}\left(#1\right)}}
\newcommand{\pack}[3]{\ensuremath{\texttt{pack}_{#1 = #2} #3}}
\newcommand{\unpack}[4]{\ensuremath{\texttt{unpack } (#1,#2) = #3 \texttt{ in } #4}}
\newcommand{\toC}[1]{\ensuremath{\text{to}\mathbb{C}(#1)}}


\DeclareMathOperator{\wb}{WB}
\DeclareMathOperator{\wbc}{WBc}
\DeclareMathOperator{\Hom}{Hom}

\newcommand{\loli}{\multimap}
\newcommand{\tensor}{\otimes}
\newcommand{\proves}{\vdash}
\newcommand{\ang}{^{\circ}}

\newcommand{\curry}[1]{\ensuremath{\text{curry}\left(#1\right)}}
\newcommand{\const}[1]{\ensuremath{\text{const} \left(#1\right)}}

\newcommand{\ds}{\ensuremath{\$}}

\newcommand{\gens}{\Rightarrow}
\newcommand{\infers}{\uparrow}
\newcommand{\checks}{\downarrow}
\newcommand{\M}{\mathbb{M}}
\newcommand{\subty}{<:}
\newcommand{\subtynf}{<:_{\texttt{nf}}}

\newcommand{\fv}{\texttt{fv}}
\newcommand{\bv}{\texttt{bv}}

\newcommand{\citehere}{\color{red}[cite]\color{black}}

\newcommand{\red}[1]{\color{red}#1\color{black}}

\newcommand{\lambdaamor}{$\lambda$-Amor\xspace}
\newcommand{\lambdaamorminus}{$\lambda-\text{Amor}^{-}$\xspace}
\newcommand{\dlambdaamor}{d$\lambda$-Amor\xspace}
\newcommand{\bilambdaamor}{bi$\lambda$-Amor\xspace}
\newcommand{\lambdaA}{$\lambda^A$\xspace}
\newcommand{\lambdaamorimpl}{\texttt{LambdaAmor}\xspace}

\newcommand{\R}{\mathbb{R}}
\newcommand{\Z}{\mathbb{Z}}


\newcommand{\potvec}{\vec{\mathbb{R}^+}}

\newcommand{\pvdash}{\vdash_p}

\newcommand{\codeInc}{\texttt{inc}}
\newcommand{\codeSet}{\texttt{set}}
\newcommand{\codebitlist}{\texttt{bit\ list}}
\newcommand{\codenat}{\texttt{nat}}

%theorems

\newcommand{\jtheorem}[2]{
  \vspace{0.5em}

  % \noindent\fcolorbox{CadetBlue}{white}{#1}
  %{\color{CadetBlue}\hrule height 0.5pt}

  \vspace{0.25em}

  \noindent \textbf{#1}

  \vspace{0.25em}

  %{\color{CadetBlue}\hrule height 0.5pt}

  \vspace{0.5em}

  \noindent \textit{Proof.} #2 \qed

  \vspace{1em}
}

\newcommand{\jtref}[1]
  {\textsc{#1}}

\newmdenv[
  usetwoside=false,
  topline=false,
  bottomline=false,
  rightline=false,
  leftmargin=0.2in,
  linewidth=0.75pt,
  skipabove=\topsep,
  skipbelow=\topsep,
  nobreak=false
]{leftrule}

\newcommand{\allrule}[1]{
  \vspace{\topsep}

  \noindent\hspace{10pt}\fbox{#1}

  \vspace{\topsep}
}

\newcommand{\jgivengoal}[2]{
  \vspace{0.5em}

  {
    \setlength{\parskip}{0em}
    \noindent $\blacktriangleright$ \textbf{Given:}
    {
      %\setlength{\parskip}{0.7em}
      \begin{adjustwidth}{0.1in}{0in}
        #1
      \end{adjustwidth}
      %\vspace{0.7em}
    }

    \noindent $\blacktriangleright$ \textbf{Goal:}

    %\vspace{0.35em}

    \begin{adjustwidth}{0.1in}{0in}
      \allrule{#2}
    \end{adjustwidth}

    %\vspace{0.35em}
  }
  \noindent \ignorespaces
}

\newcommand{\jgivengoalTwo}[3]{
  \vspace{0.5em}

  {
    \setlength{\parskip}{0em}
    \noindent $\blacktriangleright$ \textbf{Given:}
    {
      %\setlength{\parskip}{0.7em}
      \begin{adjustwidth}{0.1in}{0in}
        #1
      \end{adjustwidth}
      %\vspace{0.7em}
    }

    \noindent $\blacktriangleright$ \textbf{Goal A:}

    %\vspace{0.35em}

    \begin{adjustwidth}{0.1in}{0in}
      \allrule{#2}
    \end{adjustwidth}

    %\vspace{0.35em}

    \noindent $\blacktriangleright$ \textbf{Goal B:}

    %\vspace{0.35em}

    \begin{adjustwidth}{0.1in}{0in}
      \allrule{#3}
    \end{adjustwidth}

    %\vspace{0.35em}
  }
  \noindent \ignorespaces
}

\newcommand{\jcase}[3]{
  %\vspace{0.7em}

  \noindent $\blacktriangleright$ \textbf{Case #1:} \textit{#2.}

  {
    %\setlength{\parskip}{0.7em}
    \begin{leftrule}
      \vspace{0.35em}
      #3
    \end{leftrule}
  }
  
  \vspace{0.7em}

  \noindent \ignorespaces
}

\newcommand{\jnocase}[1]{
  {
    \setlength{\parskip}{0.7em}
    #1
  }
}

\newcommand{\jsubcase}[3]{
  \noindent $\blacktriangleright$ \textbf{Subcase #1:} \textit{#2.}

  {
    \setlength{\parskip}{0.7em}
    \begin{adjustwidth}{0.2in}{0in}
      #3
    \end{adjustwidth}
  }

  \noindent \ignorespaces
}


\newcommand{\caseText}[1]
  {\noindent #1}

\newcommand{\caseFact}[1]
  % {\\[2pt]\makebox[0.40in][r]{(#1)\ \ }}
  {\noindent \hspace{10pt}(#1)\hspace{5pt}}

\newcommand{\caseFactPl}[1]
  % {\\[2pt]\makebox[0.60in][r]{(#1)\ \ }}
  {\caseFact{#1}}


\begin{document}

\begin{abstract}
While decades of research into formal verification have brought provable correctness guarantees closer to being a part of developers' everyday reality, provable guarantees about algorithmic complexity are harder to come by. This is not for lack of effort. Algorithmic complexity and program cost don't play nicely with abstraction, and so they can prove a difficult target for the kinds of compositional analysis that formal tools handle best. However, the classical algorithm analysis technique known as amortized analysis \cite{tarjan:amortized-complexity} is promising in this regard: it allows one to selectively break abstraction barriers to precisely and compositionally calculate program costs. In this thesis, we leverage amortized analysis to make two contributions which make some progress towards our goal of provable cost guarantees for the masses.

We begin by taking a cue from the interactive theorem proving school of software verification. We develop a functional language called \lambdaamorimpl, which provides a rich refinement type system for in-language amortized cost analysis. We begin with a previously-developed core calculus called \lambdaamor \cite{rajani-et-al:popl21}, transform it to an \textit{algorithmic} type system which is amenable to implementation, and subsequently implement the language in OCaml.

Next, we move to considering a more lightweight method of analyzing the cost of programs, namely the extract-and-solve method of recurrences. This technique is already used (explicitly or otherwise) by practitioners of functional languages, and regularly included in introductory CS curricula. However, the technique is informal, error-prone, and not immediately applicable to amortized cost analysis. Following on work by \citet{danner-et-al:icfp15}, we formalize the process of amortized analysis by recurrence extraction as a language-to-language translation, and use this to prove the technique's correctness.
\end{abstract}


\begin{dedication}
\epigraph{Classical watches display time, but can hardly do anything else. This limitation is artificial: for instance, several people confessed to be often in want of mustard... and what is the point of knowing time if you cannot get mustard?}{Y.J. Ringard \cite{mustard-watches}}
\end{dedication}

\begin{acknowledgements}
Firstly and most importantly, I must thank my outstanding research advisor, Prof. Dan Licata. Dan took me under his wing when I was only a freshman, and spent countless hours of his valuable time gently teaching me the ways of programming language research. In the years since, he's become an amazing mentor, a tremendously skillful collaborator, and a great friend. For all this, I will be eternally grateful.

I am also incredibly lucky to have been simultaneously mentored by the wonderful Prof. Norman Danner. On top of being the spiritual leader of the Wesleyan PL cost-analysis group and the driving force in shaping my research interests, Norman is an absolutely stellar professor from whom I had the pleasure of taking multiple courses and seminars.

I would like to express my gratitude to my readers, Dan, Norman, and Robert Rose, for taking the time to read this tome.

A huge amount of thanks also goes out to Prof. Deepak Garg, who, over a high-top table at the hotel bar at POPL 2020, offered to host me at MPI-SWS for the internship that became Chapter 2 of this thesis. I also appreciate his generosity in agreeing to work with me virtually when a global pandemic decided to upend my summer-in-Saarbrucken plans.

I am very grateful for the joint friend/mentor-ship offered by Joomy Korkut, Prof. Alex Kavvos, and Mitchell Riley, all of whom seamlessly and rapidly alternate between being outstanding friends, kind mentors, and non-judgmental shoulders to cry on.

I would be remiss not to mention the rotating cast of characters with whom I've shared ESC341 and ESC345, my two ``offices" during my time at Wesleyan: Yulia, Pi, Rocco, Elliott, Vabuk, and Ed, you all made it worth trudging through the CT cold to come to work in Exley each day. Similarly, I would like to thank the many denizens of the 6th floor math lounge for providing yet another comfortable and fun working environment.

Of course, as is true of all such projects, this thesis would not have been possible without the emotional support of my wonderful housemates and friends. Sam, Shea, Rachel, the gang from $(\N \to \N) \to (\N \to \N) \to (\N \to \N)$, and the folks from the PL-twitterverse have all been instrumental in keeping me sane during the madness that has been 2020 and 2021.

Finally, I must acknowledge the tremendous support my family has provided me through college, and during the past year especially. Mom, Dad, and Nathaniel: I got to spent more time with you than I was expecting to this year, and I'm thankful for it every day. Cristian and Vyana: your astounding hospitality made it possible for me to weather the storm of our rocky COVID summer, and emerge on the other end with a head start on the contents of this thesis.
\end{acknowledgements}


\frontmatter
\maketitle
\makeabstract
\makededication
\makeack 
%\setcounter{tocdepth}{2}
\tableofcontents
%\listoftheorems[ignoreall,show={thm}]

\mainmatter


\hypersetup{allcolors = {blue}}


\chapter{Introduction}
\label{ch:intro}
As the importance of ubiquity of modern software increases, so too does its complexity. The burden of this complexity blowup lands squarely on the shoulders of software developers, who are asked to create increasingly intricate systems, with little extra help. In response to this, many developers have turned to tooling and languages to help ease the burden: this is exemplified by the rise of Rust and safe systems languages \citehere which provide safety guarantees, along with the integration of static analysis tools into the standard development practices of many large software companies. While these practices go a long way to improve developer experience, they are limited in the domains of understanding that they improve. Notably, there are very few existing tools which help developers reason about the algorithmic complexity and performance of their software. The days when the performance and resource usage could be easily discerned from source code by eye are long gone, and yet very few techniques have stepped in to fill the void.

The long-term goal of the field of language-based resource analysis is to create languages and tools which fill this gap by providing methods for statically determining a program's resource usage, and presenting this information to developers. Works (\textbf{??}) in this field usually come in one of two forms: techniques which which give resource guarantees to large classes of programs at once, and those which allow one to analyze a single program a time. The first category is typified by the creation of languages with resource-aware type systems which allow programmers to reason about their programs' resource usage as they write them. Some examples of projects of this type include Resource Aware ML \citehere, Granule \citehere, and Linear Haskell \citehere.

The second category is exemplified by verification techniques such as program logics and static analyses-- given a program, these tools can be used to semi-automatically derive or prove resource usage properties. Some well known examples of this class include Infer \citehere, (\textbf{??})

In this thesis, we will explore and extend one development in each of these two categories. More specifically, we will specialize our notion of resource to only consider algorithmic complexity or time cost, and examine two techniques which analyze it using amortized analysis.

\subsection{Contributions}


\section{Amortized Analysis Primer}
The intuition behind both developments in this thesis is rooted in amortized analysis, the classical algorithm analysis technique first presented by \citet{tarjan:amortized-complexity}. As such, we will provide a brief introduction to the technique here, in addition to presenting a few examples of its utility, one of which we will use as a running example.

Amortized analysis was initially conceived of as a technique for analyzing the worst-case cost of a sequence of operations on a data structure. Without amortization, such cost analyses can be very imprecise Naively, the worst-case cost of a sequence of operations is bounded by the sum of the worst-case cost for each operation. However, this usually fails to take into account internal data structure invariants which make it impossible (or not always true) that each operation in the sequence executes with its worst-case complexity. For a concrete example of this phenomenon, consider the (contrived, yet useful) example of a binary counter shown in Figure~\ref{fig:bin-counter}.

\begin{figure}
  %bit, counter, inc,set
  \caption{Binary Counter Data Structure}
  \label{fig:bin-counter}
\end{figure}

The type \texttt{counter} is a list of bits, with the least significant bit at the head. The \texttt{inc} operation increments the binary counter by one. To illustrate where a standard analysis goes wrong, we will analyze the cost of a sequence of $n$ increment operations, starting from the empty list-- this behavior is encapsulated by the \texttt{set}, which iterates \texttt{set} $n$ times. For simplicity, the only costly operations are cons (\texttt{::}) operations, which cost one unit of time each.

Given a counter of length $k$, \texttt{inc} performs at most $k + 1$ cons operations-- at worst, the counter is all ones, and \texttt{inc} must walk down the entire list flipping ones to zeroes, finishing by cons-ing a one onto the end. It's easy to see that after $i$ calls to \texttt{inc}, the counter has length bounded by $\left \lceil{\log_2 i}\right \rceil$, the number of needed for the binary representation of $i$. Thus, the total cost of \texttt{set n} is
\begin{align*}
  \sum_{i=1}^n \left\lceil{\log_2 i}\right \rceil + 1
  &\leq \sum_{i=1}^n \log_2 i + 2\\
  &\leq 2n + \sum_{i=1}^n \log_2 i\\
  &\leq 2n + \log_2(n!)\\
  &\leq 2n + n\log_2 n
\end{align*}

While it may be useful for some applications, this bound is not tight. To see why, consider the case where the counter is set to the value of $15_{10}$, or $1111_2$. A  call to increment on this counter costs the full $5$ cons operations, leaving the counter at $10000_2$. However, a subsequent call to \texttt{inc} only costs $1$ to flip the first bit. Indeed, very few calls to \texttt{inc} traverse the whole list-- the vast majority only flip one or two bits (\red{do the math here for fun}). This example illustrates the tension of doing these naive analysis, and provides the primary insight for amortized analysis: while one data structure operation may be expensive, it may also restructure the data structure in such a way which makes \textit{subsequent} operations cheaper than the worst case\footnote{
This is the genesis of the name \textit{amortized} analysis: expensive function operations effectively pay for subsequent ones to be cheaper, evoking \textit{amortization} from accounting.
}.

\subsection{Physicist's Method}
The most common method for operationalizing this insight is by the physicist's method of amortized analysis. This method proceeds by associating a data structure with a real-valued ``potential function" $\Phi : S \to \mathbb{R}$ on its states. The only restriction on potential functions is that they be nonnegative everywhere.
Then, for any sequence of data structure operations $f_i$ with costs $c_i$ and intermediate states $s_i$ (pictured in Figure~\ref{fig:amortized-situation}, with $s_0$ initial and $s_i = f_i(s_{i-1})$), we may define the \textit{amortized cost} of each operation as:
$$
a_i = c_i + \Phi(s_i) - \Phi(s_{i-1})
$$

\begin{figure}
  \caption{The Generic Amortized Analysis Setup}
  \label{fig:amortized-situation}
\end{figure}

The amortized cost of an operation is the actual cost, plus the change in potential across it. When we sum the amortized cost of all the operations across the sequence, the sum telescopes:
\begin{align*}
  \sum_{i=1}^n a_i &= \sum_{i=1}^n c_i + \Phi(s_i) - \Phi(s_{i-1})\\
                   &= \Phi(s_n) - \Phi(s_0) + \sum_{i=1}^n c_i
\end{align*}
Then, if $\Phi(s_0) = 0$, we get that the total amortized cost is an upper bound on the total actual cost.
$$
\sum_{i=1}^n a_i \geq \sum_{i=1}^n c_i
$$
A good intuition for potential functions $\Phi$ is that $\Phi(s)$ represents the amount of work subsequent operations have to do in order to modify the data structure.
In practice, one usually picks potential functions such that when an expensive operation $f_i$ runs, $\Phi(s_{i-1})$ is very large, and $\Phi(s_i)$ is very small such that subsequent operations are cheap.

Returning to the binary counter example, the traditional choice of potential function is to take $\Phi(\texttt{xs})$ to be the number of 1-bits in \texttt{xs}.
For simplicity, we will denote the actual cost of a call to $\texttt{inc xs}$ by $C(\texttt{xs})$, and its amortized cost by $A(\texttt{xs}) = C(\texttt{xs}) + \Phi(\texttt{inc xs}) - \Phi(\texttt{xs})$.

\begin{theorem}
For all \texttt{xs:counter}, $A(\texttt{xs}) = 2$
\end{theorem}
\label{thm:bc-phys}
\begin{proof}
By a straightforward structural induction on \texttt{xs}.
\begin{itemize}
  \item ($\texttt{xs} = \texttt{[]}$): \texttt{inc []} does $1$ cons operation, $\Phi(\texttt{[]}) = 0$ and $\Phi(\texttt{[1]}) = 1$, so the amortized cost is $2$.
  \item ($\texttt{xs} = \texttt{y::ys}$): If $\texttt{y} = \texttt{0}$, then the \texttt{inc xs} does $1$ cons operation. Since $\Phi(\texttt{1::ys}) - \Phi(\texttt{y::ys}) = (1 + \Phi(\texttt{ys})) - \Phi(\texttt{ys}) = 1$, we again have that the amortized cost of \texttt{inc xs} is $2$. Now, suppose $\texttt{y} = \texttt{1}$. Then \texttt{inc xs} incurs $1$ cost from the cons operation, plus the cost of \texttt{inc ys}. So, we may compute:
  \begin{align*}
    A(\texttt{y::ys}) &= C(\texttt{y::ys}) + \Phi(\texttt{inc (y::ys)}) - \Phi(\texttt{y::ys})\\
    &= (1 + C(\texttt{ys}) + \Phi(\texttt{inc ys}) - (1 + \Phi(\texttt{ys}))\\
    &= C(\texttt{ys}) + \Phi(\texttt{inc ys}) - \Phi(\texttt{ys})\\
    &= A(\texttt{ys)}
  \end{align*}
  But by the inductive hypothesis, $A(\texttt{ys}) = 2$, which completes the proof.
\end{itemize}
\end{proof}

An immediate corollary of this fact is that the amortized cost of \texttt{set n} is bounded by $2n$, by the same telescoping argument as before. Most importantly, since the binary counter \texttt{[0]} has potential $0$, the amortized cost of \texttt{set n} is an upper bound on the \textit{actual} cost of \texttt{set n}. This last step is crucial. A priori, amortized costs give no information about the actual costs of program execution, and are only a bound on the actual cost if the final potential is greater than the initial.

\subsubsection{Single Function and Gas Tank Analyses}
Often, amortized analysis is applied to individual functions, rather than a sequence. This may be thought of as the length-one case of amortized analysis.
Given a function \texttt{f:a->b} and potential functions $\Phi_\texttt{a} : \texttt{a} \to \R$ and $\Phi_\texttt{b} : \texttt{b} \to \R$, we may define
the amortized cost of \textbf{f} as $A_\texttt{f}(\texttt{x}) = C_\texttt{f}(\texttt{x}) + \Phi_\texttt{b}(\texttt{f x}) - \Phi_\texttt{a}(\texttt{x})$,
where $C_\texttt{f}(\texttt{x})$ is the cost of \texttt{f}. As long as $\Phi_\texttt{b}(\texttt{f x}) \geq \Phi_\texttt{a}(\texttt{x})$, the amortized cost is an upper bound on the actual cost.

A common variation on this concept is to pick the potential functions such that the amortized cost is \textit{zero}. Then, the actual cost $C_\texttt{f}(\texttt{x})$ is exactly $\Phi_\texttt{a}(\texttt{x}) - \Phi_\texttt{b}(\texttt{f x})$. The usual intuition here is to think of the available potential as a sort of ``gas tank" which the function must siphon from to do work. Then, the total gas used, $\Phi_\texttt{a}(\texttt{x}) - \Phi_\texttt{b}(\texttt{f x})$, is an upper bound on the amount of work done by the function.


\subsection{Banker's Method}
While the physicist's method is powerful, some situations call for a more fine-grained analysis. In this case, we employ the so-called banker's method of amortized analysis. The banker's method works by introducing imaginary ``credits" to a data structure, which may be ``attached" to the values in a program. These credits are thought of to interact with the cost model of the language in a special way: credits can always be created from thin air at a cost of one unit of time, and then they can subsequently be discarded or ``spent" to decrease execution cost by one unit. In the setting of the banker's method, this is what we mean when we refer to ``amortized cost"-- the real cost of an operation, plus the cost of creating and spending credits along the way. Crucially, if an operation begins with no credits available, then the amortized cost must be an upper bound on the actual cost since credits must be created (incurring a cost of $1$) before they can be spent (decreasing the cost by $1$). All of this is best illustrated by returning to the binary counter example.

We begin by enforcing a credit invariant on values of type \texttt{counter}: every \texttt{1} bit must have a credit attached. It's worth confirming that the increment function is able to maintain this invariant: if the counter is empty, we spawn a credit, attach it to a \texttt{1} bit, and cons it to the front of the list.
If the least significant bit is \texttt{0}, we again spawn a credit, flip the bit to \texttt{1}, and attach the credit. Finally, if the least significant bit is \texttt{1}, then we detatch its credit, spend it, recurse down the tail, and finish by cons-ing a \texttt{0} onto the front. Of course, none of this is manifest in the code
\footnote{
Readers familiar with concurrent separation logic might find this idea familiar: credits are a form of ghost state.
}.
Just like with potential in the physicist's method, these proofs must happen off to the side on paper.

Finally, we may perform the analysis itself.

\begin{theorem}
For all \texttt{xs:counter} satisfying the credit invariant, the amortized cost of \texttt{inc xs} is $2$.
\end{theorem}
\label{thm:bc-bank}
\begin{proof}
We proceed by structural induction on \texttt{xs}.
\begin{itemize}
  \item ($\texttt{xs} = \texttt{[]}$): \texttt{inc []} does $1$ cons operation and spawns one credit, for an amortized cost of $2$.
  \item ($\texttt{xs} = \texttt{y::ys}$): If $\texttt{y} = \texttt{0}$, then \texttt{inc xs} does one cons operation and spawns a single credit, for an amortized cost of $2$. Finally, if $\texttt{y} = \texttt{1}$, then \texttt{inc xs} makes a single recursive call \texttt{inc ys}, which has amortized cost $2$, by inductive hypothesis. But then, the function spends the credit attached to \texttt{y}, which cancels out the cost $1$ incurred by cons-ing a \texttt{0} onto the result of the recursive call. In total, this case has $2$ amortized cost, as required.
\end{itemize}
\end{proof}

\subsection{Comparisons}

The reader may note that the proof of Theorem~\ref{thm:bc-bank} was remarkably similar to the proof of Theorem~\ref{thm:bc-phys}. This is, unsurprisingly, not by coincidence. The reason for this similarity is that the banker's method can be thought of as a concretization of the physicist's method. Rather than ``globally" assigning potential to the states of a data structure, the banker's method ``localizes" the potential, thought of as discrete credits, on  specific values in the state. Because of this, analyses with the banker's method have an ``operational" feel to them, while analyses using the physicist's method have a more ``calculational" flavor.

On the whole, the two methods of amortized analysis are essentially equivalent in power. Given a banker's method analysis, we may turn it into a physicist's method analysis by taking the potential function to be the total number of credits. Conversely, physicist's method analyses can be converted to use the banker's method by maintaining the invariant that $ \left\lceil{\Phi(\texttt{s})}\right \rceil$ credits be kept on the value \texttt{s}.

Proofs using the banker's method are often more tedious and traditionally less formal. In Chapter~\ref{chap:rec-extr}, we present a formalization of the banker's method by way of recurrence extraction. Chapter~\ref{chap:lambda-amor} presents \lambdaamor, which is based primarily on the physicist's method.


\chapter{Amortized Analysis with Type Systems}
\label{ch:lambda-amor}
\section{Introduction}
Type systems are invaluable tools for developing robust and extensible modern software \citehere. Programming languages theorists have long understood the utility that type systems bring outside of the standard assurance that well-typed programs do not go wrong \citehere. Indeed, type systems can be designed to help programmers reason about myriad facets of their programs, including but certainly not limited to security and privacy \citehere, nondeterminism \citehere, computational effects \citehere, low-level representation details \red{[Kinds calling conventions]}, asynchronous communication \red{[Session types]}, staging \citehere, and program modularity \red{[Module systems]}. 
\\

Most relevant to this thesis, however, is the ability to create type systems which allow programmers to reason about their programs' resource usage. \red{more general background here about type systems for resource analysis}.


In this \red{chapter}, we will investigate and implement a variant of \lambdaamorminus (pronounced ``lambda amor minus"), a language with a type system for amortized cost analysis. Programs written in \lambdaamorminus have types which are annotated with costs and potential, such that every type-correct program in \lambdaamorminus is a valid amortized analysis. Moreover, the costs expressed in the types give a sound upper bound on the actual execution cost of the program. \lambdaamorminus is expressive enough to statically verify amortized cost bounds for a wide class of functional programs, from the traditional examples of amortized analysis, to fully general cost-polymorphic higher-order functions like \texttt{map} and \texttt{fold}, which aren't well handled by existing resource-analysis languages like Resource Aware ML \citehere. \red{Should I talk more about Lambda-Amor here?}

By and large, the original creators of \lambdaamorminus were interested in it as a unifying framework for amortized analysis type systems. There are many axes along which one may design a type system, and \lambdaamorminus does a good job of interpreting many different styles. In this work, however, we will primarily interest ourselves in \lambdaamorminus's usefulness as a programming language which provides strong type-based cost reasoning principles to the user. To this end, the primary goal of this work is to design a version of \lambdaamorminus called \dlambdaamor which is amenable to implementation. We do this by cutting out a fragment of \lambdaamorminus, and restricting its syntax somewhat-- this process will yield \dlambdaamor. However, \dlambdaamor is not immediately implementable, as its typing rules (just like \lambdaamorminus's) are written in declarative style, from which we cannot immediately construct an algorithm. 

The traditional solution to this is to create yet another type system-- an \textit{algorithmic} one, from which a type-checker can be easily implemented. For this purpose, we will introduce \bilambdaamor, a type system which encodes the same typing relations as \dlambdaamor, but is presented in a manner that is trivial to implement.

We will begin in Section~\ref{sec:lambdaamor-overview} by giving an overview of the concepts \lambdaamorminus's type system draws on. \lambdaamorminus includes two modalities-- unary operators on types for tracking cost and potential, respectively. To soundly manage this potential, \lambdaamorminus is based on an affine logic in which every variable may be used at most once so that values with potential cannot be duplicated. To make complex potential functions, \lambdaamor uses refinement types in the style of DML \citehere, which we review. Next, we discuss the main obstacle the original type system presents to implementation: constraint solving. Our solution to this problem is based on univariate polynomial potential functions in the style of Automated Amortized Resource Analysis (AARA) \citehere, which we briefly review.

Then, in Section~\ref{sec:dlambdaamor-syntax-and-types}, we will discuss the syntax and type system of \dlambdaamor in depth, providing intuition for the each of the judgments, and discussing selected rules from the type system. \dlambdaamor's type system is many-layered, with rules for type formation, term formation, and a smaller type system for the sub-language which governs the refinement types. We pay special attention to the rules which govern the cost analysis-specific language features, and describe them in detail.

In Section~\ref{sec:dlambdaamor-sound}, we sketch the soundness proof for \dlambdaamor, by showing that it may be embedded in \lambdaamorminus, and appealing to its soundness theorem featured in \citet{rajani-et-al:popl21}.

In Section~\textbf{??}, we explore some programs written in \dlambdaamor. These programs (and others) show off different facets of \dlambdaamor's type system, and serve to provide a comprehensive test suite against which we can evaluate our eventual implementation.

In Section~\ref{sec:bilambdaamor}, we present \bilambdaamor, the algorithmic version of \dlambdaamor. \bilambdaamor draws on numerous techniques from literature on algorithmizing type systems, such as bidirectional type inference \citehere, constraint output and solving \citehere, normalization \citehere, and the I/O method \citehere. We give an overview of these techniques, and then explore how they may be applied to \dlambdaamor to yield \bilambdaamor. We pay especially careful attention to a normalization procedure used in \bilambdaamor's subtyping, which represents the most nonstandard aspect of the algoithmic system.

In Section~\ref{sec:metatheory}, we prove that \bilambdaamor and \dlambdaamor are in fact (essentially) the same type system. This fact is a requirement for a good implementation, as it guarantees that our typechecker accurately and soundly types terms. The proof is broken into two parts. A proof of soundness tells us that when a typechecker derived from the algorithmic rules of \bilambdaamor confirms that an expression has a given type, our ``ground truth" declarative \dlambdaamor agrees. Dually, the proof of completeness ensures that every declaratively-derivable typing relationship in \dlambdaamor will be found by a typechecker which implements \bilambdaamor's algorithm.

Finally, in Section~\ref{sec:lambdaamor-impl}, we discuss \lambdaamorimpl, our OCaml implementation of the \dlambdaamor. In order to be a useful programming language, \lambdaamorimpl sports a top-level environment with multiple declaration types, on top of the simple typechecking prescribed by \bilambdaamor. We discuss these additions to the language, as well as the specific design choices made while building the artifact. We finish the section by writing the examples from Section~\textbf{??} in \lambdaamorimpl, and benchmarking our implementation.


\section{Overview of \dlambdaamor} 
\label{sec:lambdaamor-overview}
In this section, we will begin by presenting the overarching ideas which make \lambdaamor a useful language for resource analysis, and subsequently discuss the large subset \dlambdaamor, which we will focus on for the rest of the chapter.

\subsection{Cost and Potential Modalities}
One of the most basic insights that \lambdaamor takes advantage of in its design is that costly computation can be thought of an effect\footnote{
In fact, cost can also be thought of as a \textit{coeffect} \cite{girard-et-al:tcs92:bll}, and one of the major breakthroughs of \lambdaamor is the unification
of both styles of resource tracking in a single calculus.
}. When a program does work, it has an effect on the world, namely the effect of taking time. In this sense, nearly all ``pure" programming languages are impure, as they allow pervasive use of the effect of cost. Unlike most languages, \lambdaamor encapsulates this effect by forcing all costly computation to happen in a monad \citehere.

\subsubsection{Cost Monad}
 However, a simple monad is not enough. We care not only that a term may incur cost, but how much cost it can incur! For this purpose, \lambdaamor uses a \textit{graded} monad \citehere $\M \; I \; \tau$. A computation of this type is a computation which returns a value of type $\tau$, and may incur up to $I$ cost, where $I$ is drawn from the sort of positive real numbers. As a graded modality, this monad's operations interact with the grade in nontrivial ways: for instance, the ``pure" computation $\texttt{ret}(e)$ has type $\M \; 0 \; \tau$ when $e : \tau$. Of course, any pure term may be lifted to a monadic computation which incurs no cost (\red{How do I explain that things run...}). Most importantly, given a costly computation $e_1 : \M \; I_1 \; \tau_1$ and a continuation $x : \tau_2 \vdash e_2 : \M \; I_2 \; \tau_2$, they can be sequenced into a computation $\texttt{bind}\, x = e_1\, \texttt{in}\, e_2 : \M \; (I_1 + I_2) \; \tau_2$. Note that the costs add-- a computation which may take up to $I_1$ units of time followed by a computation which takes up to $I_2$ units of course takes at most $I_1 + I_2$ units. However, neither \texttt{ret} nor \texttt{bind} incurs any nontrivial cost: any program written using only \texttt{ret}s and \texttt{tick}s will have type $\M \, 0 \, \tau$. For this, \lambdaamor includes a term $\texttt{tick}[I]$ of type $M \, I \, \texttt{unit}$, which incurs cost $I$. This is the only construct in \lambdaamor which incurs any ``extra cost": the idea is that programmers insert \texttt{tick}s in front of the operations their specific cost model dictates are costly. This technique is widely used in the cost analysis literature \citehere, and so \lambdaamor also adopts it for simplicity.
 
But of course, this cost monad can only be half the story. In a language which seeks to provide types for amortized analysis, a mechanism for handling potential is required.
 
\subsubsection{Potential Modality and Affine Types}
In addition to the cost monad, \lambdaamor includes another graded modality for tracking potential. A term of type $[I] \; \tau$ can be thought of a term of type $\tau$ which stores $I$ potential\footnote{
In some senses, potential in \lambdaamor behaves more like the credits of the banker's method discussed in Section~\ref{sec:amortized-primer}-- it can be created and attached to specific values. To avoid confusion, we follow \citet{rajani-et-al:popl21} with the terminology of ``potential"
}, where $I$ is again drawn from a sort of positive real numbers.
The most important operation associated with the potential modality is the ability to use potential to offset the cost of a computation. Concretely, given a term $e_1 : [I] \; \tau_1$ and a monadic continuation $x : \tau_1 \vdash e_2 : \M \; (I + J) \; \tau_2$, we can form the computation $\texttt{release}\, x = e_1 \, \texttt{in}\, e_2 : \M \; J \; \tau_2$. The crucial aspect of this construction is the fact that the resulting computation requires at most $J$ units of time to run, while the initial computation $e_2$ required $I + J$. Intuitively, we think of this as the $I$ units of potential ``paying for" $I$ steps of computation. 

Potential may also be created, and attached to values. In \lambdaamor, these two functions are handled by the same construct. For terms $e : \tau$, we may form $\texttt{store}[I](e) : \M \; I \; ([I] \; \tau)$, which is a computation which runs for at most $I$ units of time, and returns a $\tau$ with $I$ potential attached. The fact that \texttt{store} incurs this cost is what justifies the term \texttt{release}-- the program has paid an ``extra" cost of $I$ to create $[I] \; \tau$, and thus can exercise this option to reduce the cost of a subsequent computation with \texttt{release}.

This dynamic between \texttt{store} and \texttt{release} forces a restriction on the type system-- variables can only be used at most once. Our argument for the soundness of \texttt{release} relies on an the assumption that the potential we are releasing has not already been released elsewhere, and so duplication of variables must be disallowed. Of course, this kind of restriction is very common-- we simply require that \lambdaamor be \textit{affine}: weaking of the context is allowed, but contraction is disallowed. 

\subsubsection{Refinement Types}
\label{sec:lambdaamor-overview-refty}
So far, the situation we've described would only allow types with \textit{constant} amounts of potential. For nontrivial analyses, this is wholly insufficient: the potential of a data structure must be able to depend on the size or other numerical parameters of that data structure. For this purpose, \lambdaamor includes \textit{refinement types} in the style of Dependent ML \citehere. Concretely, \lambdaamor includes length-refined lists: a value of type $L^n \tau$ is a list of length $n$, where $n$ is an \textit{index term}-- an term in a small language of arithmetic expressions over a set of variables. Further, these index terms which appear in refinements may also appear in potentials! For instance, $\left[n^2\right] \; (L^n \tau)$ is the type of lists of length $n$ with potential $n^2$.

\subsection{Potential Vectors and AARA}
The story we've just told about \lambdaamor's type system is loyal to the original presentation in \citep{rajani-et-al:popl21}, but somewhat inadequate for implementation purposes. As we will discuss in Section~\ref{sec:lambdaamor-impl}, efficient subtyping is necessary for implementation of \lambdaamor. However,
the inclusion of the potential and cost modalities presents a challenge. In order for $[I] \; \tau_1$ to be a subtype of $[J] \; \tau_2$, it must be that $\tau_1 \subty \tau_2$, and that $J \leq I$. But as discussed above, $I$ and $J$ are index terms, and may be polynomials in a set of index variables. Ideally, we would like to discharge these inequalities generated by subtyping by constraint solver, but even modern SMT solvers struggle to handle polynomial inequalities.

To solve this problem, we borrow a key idea from Automatic Amortized Resource Analysis (AARA) \citehere which will allow us to generate only linear constraints over index variables, while still allowing univariate polynomial potentials and cost. The main idea is to fix a clever ``basis" for the space of polynomials, and then represent polynomials as a vector of their coefficients with respect to that basis. The basis in question is chosen to satisfy one key property: if $f(n)$ is written in terms of the basis, then the coefficients of $f(n-1)$ may be efficiently determined from $f(n)$. This property gives rise to the ability to easily analyze list algorithms in \lambdaamor: when writing a function $([f(n)] \, (L^n \, \tau)) \loli \sigma$, it is simple to pattern match on the argument and determine the type of the tail $[f(n-1)] \, (L^{n-1} \, \tau)$ to pass to a recursive call.

In \dlambdaamor, we will mostly syntactically restrict potential functions to be of this form, with some exception. We show in Section \textbf{??} that this language may be trivially elaborated into the original \lambdaamor, and further in Section \textbf{??} we show that the restricted set of allowable potential functions are still expressive enough for practical purposes.
\red{transition...}

\textbf{How do I cite that literally all of this is from JanH}

\begin{definition}[Potential Vector]
For a fixed $k$, we call a vector of nonnegative reals $(a_0,\dots,a_k)$ a potential vector.
\end{definition}

\begin{definition}[$\phi$ Function]
For fixed $k$, we define $\phi : \N \times \R_{\geq 0}^k \loli \R$ to be
$$
\phi\left(n,(p_0,\dots,p_k)\right) = \sum_{i=0}^k p_i\binom{n}{i}
$$
where $\binom{n}{r}$ is the binomial coefficient. We refer to the first argument of $\phi$ as the ``base", and the second argument as the ``potential".
\end{definition}

With $\phi$ in hand, we redefine the cost and potential modalities. In \dlambdaamor, the cost modality is written as $M \, (I,\vec{p}) \, \tau$ and the potential modality is $[I|\vec{p}] \,  \tau$. These two types classify values of type $\tau$ which cost up to $\phi(I,\vec{p})$ units of time and posess $\phi(I,\vec{p})$ potential, respectively.

\begin{theorem}[Monotonicity and Additivity of $\Phi$]
Let $\vec{p}$ and $\vec{q}$ be potential vectors.
\begin{enumerate}
  \item If $\vec{p} \leq \vec{q}$ componentwise, then $\phi(n,\vec{p}) \leq \phi(n,\vec{q})$.
  \item $\phi(n,\vec{p} + \vec{q}) = \phi(n,\vec{p}) + \phi(n,\vec{q})$
\end{enumerate}
\end{theorem}

The fact that $\phi$ is monotone in its second argument allows us to reduce the problematic subtyping rule for potentials (and costs) to generating linear inequalities and equalities \red{this isn't really true in presence of the sum...}: $[I|\vec{p}] \, \tau_1$ is a subtype of $[J|\vec{q}]$ when $I = J$ and $\vec{q} \leq \vec{p}$ componentwise.

The additivity of $\phi$ also allows us to simplify the bind and release- given a computation $e_1 : \M \, (I,\vec{p}) \, \tau_1$ and a continuation $x : \tau_1 \vdash e_2 : \M \, (I,\vec{q}) \, \tau_2$,  we may perform the computations in sequence with $\texttt{bind}\, x = e_1 \, \texttt{in}\, e_2 : \M \, (I,\vec{p} + \vec{q}) \, \tau_2$.

The final ingredient of this new version of the cost and potential modalities is the ability to change base. To illustrate, consider the process of writing a function $L^n \tau \loli \M \, (n,\vec{p}) \, \sigma$. The recursive call on the tail of the input list will have type $\M \, (n-1,\vec{p}) \, \sigma$, but the function expects a return value of type $\M \, (n,\vec{p}) \, \sigma$. Since the \texttt{bind} requires that the argument and the continuation have the same base, the recursive call cannot be used in this context, rendering it useless. To fix this, we include a term \texttt{shift} in \dlambdaamor which ``promotes" a computation of type $\M \, (n-1,\vec{p}) \, \sigma$ to one of type $\M \, (n,\vec{q}) \, \sigma$, for a specific $\vec{q}$ determined by $\vec{p}$. This concept is likely familar to the reader familiar with AARA: in Resource Aware ML (an implementation of OCaml based on AARA) this construct is baked into the pattern match rule, while we make it explicit.

\begin{definition}[Additive Shift]
For $\vec{p} = (a_0,\dots,a_{k-1},a_k)$ a potential vector, we define $\lhd \vec{p} = (a_0 + a_1,\dots,a_{k-1} + a_k,a_k)$
\end{definition}

\begin{theorem}
For $n \geq 1$ and $\vec{p}$ a potential vector, $\phi(n,\vec{p}) = \phi(n-1,\lhd \vec{p})$
\label{thm:raml-shift}
\end{theorem}
\begin{proof}
It is straightforward to prove (either by combinatorial argument or direct computation) that $\binom{n-1}{i} + \binom{n-1}{i+1} = \binom{n}{i+1}$. Using this fact,
we may compute as follows:
\begin{align*}
  \phi(n-1,\lhd \vec{p}) &= \sum_{i=0}^k p_i\binom{n-1}{i} + \sum_{i=0}^{k-1} p_{i+1}\binom{n-1}{i}\\
                         &= p_0 + \sum_{i=1}^k p_i\binom{n-1}{i} + \sum_{i=0}^{k-1} p_{i+1}\binom{n-1}{i}\\
                         &= p_0 + \sum_{i=0}^{k-1} p_{i+1}\left(\binom{n-1}{i+1} + \binom{n-1}{i}\right)\\
                         &= p_0 + \sum_{i=0}^{k-1} p_{i+1}\binom{n}{i+1}\\
                         &= \sum_{i=0}^k p_i \binom{n}{i}\\
                         &= \phi(n,\vec{p})
\end{align*}
\end{proof}

\section{Syntax and Type System \dlambdaamor}
\label{sec:dlambdaamor-syntax-and-types}

\subsection{Syntax of \dlambdaamor}
In preparation to discuss \dlambdaamor's type system, we present its syntax in Figure~\ref{fig:dlambdaamor-syntax}.

\begin{figure}
$$
\begin{array}{llll}
\text{Base Sort} & bS & ::= & \mathbb{N} \; | \; \mathbb{R}^+ \; | \; \vec{\mathbb{R}^+}\\
\text{Sort} & S & ::= & bS \;|\; bS \to S\\
\text{Kinds} & K & ::= & \star \;|\; S \to K\\
\text{Constants} & c & ::= & n \in \mathbb{N} \;|\; r \in \mathbb{R}^+ \;|\; (c_0,\dots,c_k) \in \vec{\mathbb{R}^+}\\
\text{Index Term} & I,J,\vec{p},\vec{q} & ::= & 0,1,\dots | \; i,j \; | \; I + J \; | \; I - J \; | \; k \cdot I \; | \; \lambda i : bS. I \; | \; I \, J \; | \; \texttt{const}(I) \;|\; \sum_{i=I_0}^{I_1} J

\\
\text{Constraint} & \Phi & ::= & \top \; | \; \bot \; | \; \Phi_1 \wedge \Phi_2 \;|\; \Phi_1 \vee \Phi_2  \;|\; \Phi_1 \to \Phi_2 \;|\; \forall i : S. \Phi \;|\; \exists i : S.\Phi \;|\; I = J \;|\; I \leq J\\
\text{Type} & \tau & ::= & 1 \;|\; \alpha \;|\; \tau_1 \loli \tau_2 \;|\; \tau_1 \otimes \tau_2 \;|\; \tau_1 \amp \tau_2 \;|\; !\tau \;|\; \tau_1 \oplus \tau_2 \;|\; \forall i : S . \tau \;|\; \exists i : S. \tau \;|\; \forall \alpha : K. \tau \;|\; L^I \tau \;|\; \Phi \implies \tau \;|\; \Phi \amp \tau\\
&&& |\; \M(I,\vec{p}) \; \tau \;|\; [I|\vec{p}] \; \tau \;|\; [I] \; \tau \;|\; \lambda i : S. \tau \;|\; \tau \; I\\
\text{Expression} & e & ::= & 0,1,\dots \;|\; x \;|\; \lambda x. e \;|\; e_1 \; e_2 \;|\; (e : \tau) \;|\; \llangle e_1,e_2 \rrangle \;|\; \texttt{let} \; \llangle x,y \rrangle = e\; \texttt{in} \; e' \;|\; (e_1,e_2) \;|\; \texttt{fst}(e) \;|\; \texttt{snd}(e)\\
&&& |\; !e \;|\; \texttt{let} \; !x = e \; \texttt{in} \; e' \;|\; \texttt{inl}(e) \;|\; \texttt{inr}(e) \;|\; \texttt{case}(e,x.e_1,y.e_2) \;|\; \texttt{case}(e,e_1,x.e_2) \;|\; \texttt{fix}(x.e)\\
&&& |\; \Lambda i. e \;|\; e \; [I] \;|\; \Lambda \alpha.e \;|\; e \; [\tau] \;|\; \texttt{nil} \;|\; e_1 :: e_2 \;|\; \texttt{match}(e,e_1,x.y.e_2) \;|\; \texttt{pack}[I](e) \;|\; \texttt{unpack} \; (i,x) = e \; \texttt{in} \; e'\\
&&& |\; \Lambda. e \;|\; e \;\{\} \;|\; <e> \;|\; \texttt{clet} \; x = e \; \texttt{in} \; e' \;|\; \texttt{ret}(e) \;|\; \texttt{bind} \; x = e \; \texttt{in} \; e' \;|\; \texttt{tick}[I|\vec{p}] \;|\; \texttt{store}[I|\vec{p}](e) \\
&&& |\; \texttt{store}[I](e) \;|\; \texttt{release} \; x = e \; \texttt{in} \; e' \;|\; \texttt{shift}(e)
\end{array}
$$

\caption{Syntax of \dlambdaamor}
\label{fig:dlambdaamor-syntax}
\end{figure}

\subsubsection{Index Terms, Sorts, Kinds, and Constraints}
\dlambdaamor's refinement types are modeled in the style of DML \citehere, which takes the form of a two-level type system. As discussed in Section \textbf{??}, these refinements allow the user to assign types potential which depend on the sizes of data structures, such as the length of lists. These numerical values are denoted by \textit{index terms} ($I,J$) which decorate some of the types and surface syntax of \dlambdaamor. Index terms may be of three possible numerical \textit{base sorts}: natural numbers $\N$, positive real numbers $\R^+$, and potential vectors of some fixed length $k$, $\vec{\R^+}$. Additionally, \dlambdaamor also includes first-order sort-level functions.

The syntax of index terms themselves is generated by the standard arithmetic operations, along with constants, variables, and application/abstraction forms for the sort-level functions. Of special note are the \texttt{const} and $\Sigma$ constructs. For an index term $I$ of sort $\R^+$, the term $\texttt{const}(I)$ is of potential vector sort, and may be thought of as the ``constant" potential vector $(I,0,\dots,0)$, such that for all $n \N$, $\phi(n,\texttt{const}(I)) = I$.
The $\Sigma$ construct is as expected, although the upper bound is non-inclusive: the sum $\sum_{i=I_0}^{I_1} J$ sums from $J[I_0/i]$ to $J[(I_1-1)/i]$, as long as the range is nonempty, when the sum is of course zero.

\dlambdaamor also supports full System F-style impredicative polymorphism, as well as sort-indexed types. We denote the kind of types as $\star$. Note that sort-indexed types may have sort-level arrows in negative position, and so sort-function-indexed types are included also.

Finally, \dlambdaamor includes constraints over index terms, generated by conjunction, disjunction, implication, both kinds of quantification, as well as the trivially true and false propositions. Note that we will not provide a proof system for these constraints. Instead, we will only ever interact with constraints via an abstract satisfiability relation $\vDash$, and all the proofs of soundness and completeness in Section \textbf{??} will be relative to a decision procedure/oracle for $\vDash$.

\subsubsection{Types}
\dlambdaamor's types include all of the standard connectives from affine logic, namely positive and negative products ($\otimes$ and $\amp$), sums ($\oplus$), affine functions ($\loli$), and the exponential modality $! \tau$, whose values may be used more than once.
Of course, \dlambdaamor also supports a litany of more specialized types for amortized cost analysis.

Chief among these are the cost monad and potential types, A monadic type $M \, (I,\vec{p}) \, \tau$ classifies monadic computations of type $\tau$, which may incur up to $\phi(I,\vec{p})$ cost. The type formation rules (Figure \textbf{??}) ensure that $I$ is of sort $\N$, and $\vec{p}$ is of sort $\vec{\R^+}$. With the same restrictions on the sorts of its index terms, the potential type $[I|\vec{p}]\, \tau$ classifies values with at least $\phi(I,\vec{p})$ potential. In addition to the AARA-style potential, \dlambdaamor also has a ``constant potential" modality $[I] \, \tau$, whose values are those of type $\tau$, with $I = \phi(n,\texttt{const}(I))$ potential, for any $n$. While not strictly necessary for the theoretical development of \dlambdaamor, this modality is sometimes useful in practice.

Index variables may be quantified over in types with the $\forall i : S.\tau$ and $\exists i : S.\tau$ types, and polymorphic type variables are quantified over using the $\forall \alpha : K .\tau$ type constructor-- we do not support existential types, though there is no metatheoretical barrier to their inclusion.

As previously mentioned, the type of lists $L^I \, \tau$ is refined by length-- the values of this type all have length $I$. Next, \dlambdaamor also includes two ``constraint types", $\Phi \implies \tau$, and $\Phi \amp \tau$. Values of the first type are known to have type $\tau$ when $\Phi$ holds, and values of the second type are values of type $\tau$, along with an (irrelevant) proof of $\Phi$. As \lambdaamor has no error handling mechanism, this construct is helpful for statically preventing errors by encoding function pre and post-conditions in a type: for instance, the \texttt{head} function may be typed as $\forall n : \N. (n \geq 1) \implies (L^n \, \tau \loli \tau)$

Finally, \dlambdaamor's types include abstraction and application forms for indexed types. The abstraction form $\lambda i :S.\tau$ has kind $S \to K$ when $\tau$ has kind $K$, and so term variables will never have type $\lambda i : S.\tau$, as it is a higher-kinded type.

\subsubsection{Terms}
\label{}
While the original presentation of \lambdaamor takes great care to include only a barebones term syntax, \dlambdaamor will have to expand this syntax somewhat to ensure that the textual representation of a program is unambiguous for programming purposes. Practically, this means that every logical connective has explicit syntactic introduction and elimination forms, whereas this is handled silently in \lambdaamor.

The term syntax for all of the standard connectives should be familiar. The two products are distinguished by double angled brackets for positive pairs, and parentheses for negative pairs. All binders are un-annotated to reduce the burden on the programmer. Lists are constructed with nil and cons constructors, and the elimination form is a pattern match. The last standard inclusion is a fixpoint operator \texttt{fix}, which allows us to write recursive functions. 

The syntax associated to the amortized analysis constructs is likely less familiar. The monadic cost type $M \, (I,\vec{p}) \, \tau$ has three operations associated with it: $\textbf{ret}(e)$ and $\texttt{bind} \, x = e_1 \, \texttt{in}\, e_2$, the unit and bind of the monad, respectively, as well as $\texttt{tick}[I|\vec{p}]$, an atomic operation which incurs a cost of $\phi(I,\vec{p})$. The potential type has introduction form $\texttt{store}[I|\vec{p}](e)$ and elimination form $\texttt{release} \, x = e_1, \texttt{in} \, e_2$. Similarly, the \textit{constant} potential type has introduction form $\texttt{store}[I](e)$, and the same elimination syntax as the AARA-style potential type.

\subsection{Type System of \dlambdaamor}
\begin{figure}
\begin{mathpar}

\boxed{\Theta \vdash \Delta}

\boxed{\Theta ; \Delta \vdash I : S}

\boxed{\Theta ; \Delta \vdash \Phi \; \texttt{wf}}

\boxed{\Psi ; \Theta ; \Delta \vdash \tau : K}
\\
\boxed{\Psi ; \Theta ; \Delta \vdash \tau \subty \tau' : K}

\boxed{\Psi ; \Theta ; \Delta \vdash \Gamma \wknto \Gamma'}

\boxed{\Psi ; \Theta ; \Delta; \Omega ; \Gamma \vdash e : \tau}
\end{mathpar}
\label{fig:dlambdaamor-typing-judgments}
\caption{Judgment Forms of the \dlambdaamor Type System}
\end{figure}

In Figure~\ref{fig:dlambdaamor-typing-judgments}, we provide a listing of the judgments which make up \dlambdaamor's type system. Selected rules are presented in Figure~\ref{fig:dlambdaamor-selected-typing-rules}, and a listing of all rules can be found in Appendix \textbf{??}.

\subsubsection{Contexts}
Judgments in \dlambdaamor have as many as five contexts.
 Contexts $\Psi$ map type variables to their kinds. $\Theta$ is an index variable context, which maps index variables to their sorts. $\Delta$ is a list of constraints, which are assumptions of the judgment-- constraints in $\Delta$ may mention variables in $\Theta$, and so there is a weak form of dependence between the two contexts. The final two contexts $\Omega$ and $\Gamma$ are term variable contexts, which map variables to their types. The context $\Omega$ is referred to as the exponential context, and it contains variables which may be used more than once: i.e. are not subject to the affine restriction.
\footnote{
One may think of all types in $\Omega$ implicitly beginning with $!$, and imagine the variable rule for exponential variables to be silently inserting the counit $!\tau \loli \tau$. This dual-context construction is standard in the study of modal types. \cite{kavvos:lmcs}
}. Finally, the context $\Gamma$ lists the rest of the variables, which may be used at most once.

To avoid questions of exchange, we consider all of the contexts except for $\Delta$ up to permutations. Indeed, we will frequently treat contexts like sets, testing membership. Further, it will be useful later on to take intersections, unions, and differences of contexts: these operations will only be defined when both operations involved are subsets of a common superset.

\subsubsection{Index Terms and their Sorts}
\begin{figure}
\begin{mathpar}
\inferrule[I-Var]{i : S \in \Theta}{\Theta ; \Delta \vdash i : S}

\inferrule[I-Plus]{\Theta ; \Delta \vdash I : bS\\ \Theta ; \Delta \vdash J : bS}{\Theta ; \Delta \vdash I + J : bS}

\inferrule[I-Minus]{\Theta ; \Delta \vdash I : bS\\ \Theta ; \Delta \vdash J : bS\\ \Theta;\Delta \vDash I \geq J}{\Theta ; \Delta \vdash I - J : bS}

\inferrule[I-Shift]{\Theta ; \Delta \vdash I : \vec{\mathbb{R}^+}}{\Theta ; \Delta \vdash \; \lhd I : \vec{\mathbb{R}^+}}

\inferrule[I-Lam]{\Theta, i : bS ; \Delta \vdash I : S}{\Theta ; \Delta \vdash \lambda i : bS. I : bS \to S}

\inferrule[I-App]{\Theta ; \Delta \vdash I : bS \to S\\ \Theta ; \Delta \vdash J : bS}{\Theta ; \Delta \vdash I \; J : S}\\

\inferrule[I-Sum]{\Theta;\Delta \vdash I_0 : \mathbb{N}\\ \Theta;\Delta \vdash I_1 : \mathbb{N}\\ 
                 \Theta,i : \N;\Delta, I_0 \leq i < I_1 \vdash J : bS}
                 {\Theta;\Delta \vdash \sum_{i=I_0}^{I_1} J : bS}

\noindent\makebox[\linewidth]{\rule{\textwidth}{0.4pt}}

\inferrule[K-Var]{\alpha : K \in \Psi}{\Psi ; \Theta ; \Delta \vdash \alpha : K}

\inferrule[K-Unit]{ }{\Psi ; \Theta ; \Delta \vdash 1 : \star}

\inferrule[K-Monad]{ \Theta ; \Delta \vdash I : \mathbb{N}\\ \Theta ; \Delta \vdash \vec{p} : \vec{\mathbb{R}^+}\\ \Psi ; \Theta ; \Delta \vdash \tau : \star}{\Psi ; \Theta ; \Delta \vdash \M(I,\vec{p}) \tau : \star}

\inferrule[K-Pot]{\Theta ; \Delta \vdash I : \mathbb{N} \\ \Theta ; \Delta \vdash \vec{p} : \vec{\mathbb{R}^+} \\ \Psi ; \Theta ; \Delta \vdash \tau : \star}{\Psi ; \Theta ; \Delta \vdash [I|\vec{p}] \tau : \star}

\inferrule[K-FamLam]{\Psi ; \Theta, i : S ; \Delta \vdash \tau : K}{\Psi ; \Theta ; \Delta \vdash \lambda i : S. \tau : S \to K}

\inferrule[K-FamApp]{\Psi ; \Theta ; \Delta \vdash \tau : S \to K\\ \Theta ; \Delta \vdash I : S}{\Psi ; \Theta ; \Delta \vdash \tau \; I : K}

\noindent\makebox[\linewidth]{\rule{\textwidth}{0.4pt}}

\inferrule[C-Conj]{\Theta ; \Delta \vdash \Phi_1 \texttt{ wf}\\ \Theta ; \Delta \vdash \Phi_2 \texttt{ wf}}{\Theta;\Delta \vdash \Phi_1 \wedge \Phi_2 \texttt{ wf}}

\inferrule[C-Eq]{\Theta ; \Delta \vdash I : bS\\ \Theta ; \Delta \vdash J : bS}{\Theta ; \Delta \vdash I = J \texttt{ wf}}
\end{mathpar}
\label{fig:dlambdaamor-selected-sort-kind-constr-rules}
\caption{Selected Algorithmic Sort, Kind, and Constraint Rules}
\end{figure}


The rules that make up the sort system for index terms (prefixed I-) are mostly self-explanatory: we ensure that arguments to arithmetic operators have the same sorts.
Since all three base sorts ($\N$, $\R^+$, $\vec{\R^+}$) are nonnegative, the rule for subtraction $I - J$ must ensure that $I \geq J$. As discussed in Section~\textbf{??}, the rule I-ConstVec shows how \texttt{const} promotes index terms of sort $\R^+$ to sort $\vec{\R^+}$. Finally, the I-Lam and I-App rules give the introduction and elimination rules for the index-level functions.

\subsubsection{Types and their Kinds}
The type formation rules (prefixed K-) for \dlambdaamor are very straightforward. All types have kind $\star$, with the exception of the index type abstraction and elimination forms. The rule K-FamLam ensures that an indexed type $\lambda i : S. \tau$ has kind $S \to K$ when $\tau$ has kind $K$, and the indexed-type application $\tau \, I$ has kind $K$ when $\tau$ has kind $S \to K$ and $I$ is of sort $S$, as seen in K-FamApp.

\subsubsection{Subtyping}
The majority of the rules for subtyping in \dlambdaamor (prefixed by S-) are the standard congruences for logical connectives. The rules for the types involved in cost analysis for refinements, however, warrant some discussion.

The rule S-Monad gives the subtyping relation for the cost monad: $\M \, (I,\vec{q}) \, \tau_1 \subty \M (J,\vec{p}) \,\tau_2$ when $\tau_1 \subty \tau_2$, $I = J$, and $\vec{q} \leq \vec{p}$ componentwise. The soundness of this rule relies on the fact that $\phi(n,\vec{q}) \leq \phi(n,\vec{p})$ when $\vec{q} \leq \vec{p}$, and the fact that the cost annotations represent \textit{upper bounds}-- it is safe to use a computation which incurs less cost in a context which expects one that incurs more. Dually, it is always safe to throw away potential in the subtyping rules for the two potential modalities, S-Pot and S-ConstPot.

In addition to the subtyping rules at base kind, the rules S-FamLam and S-FamApp govern the subtyping of indexed types. The rule S-FamLam states that subtyping at kind $S \to K$ is simply generated by pointwise subtyping in the codomain $K$, while S-FamApp is a standard congruence rule. Note that S-FamApp requires that the two arguments be equal: since we do not require that indexed types be monotone, this is the strongest possible form of the rule.

Finally, the rules S-FamBeta-$1$ and S-FamBeta-$2$ serve to include $\beta$-equality of type families in the subtyping relation. The combination of these rules with S-FamApp and S-FamLam makes the subtyping relation not syntax directed at higher kind, which provides another barrier to simple implementation, as discussed in Section\textbf{??}.

\subsubsection{Context Well-Formedness Judgments and Context Subsumption}
The judgment $\Theta ; \Delta \vdash \Phi \; \texttt{wf}$ ensures that $\Phi$ is a well-formed context: all index terms mentioned in it are sort-correct, and relations are only judged between index terms of the same sort.

\dlambdaamor also requires two auxiliary context-well-formedness judgments: $\Theta \vdash \Delta \; \texttt{wf}$ and $\Psi ; \Theta ; \Delta \vdash \Gamma \; \texttt{wf}$. The former ensures that the constraints in the context $\Delta$ are well-typed with respect to the context $\Theta$, and the latter ensures that all of the types in $\Gamma$ have kind $\star$.

Finally, The judgment $\Psi ; \Theta ; \Delta \vdash \Gamma' \wknto \Gamma$ determines when we may relax a context $\Gamma'$ to a weaker one $\Gamma$ with the T-Weaken rule. Intuitively, this judgment encodes the permission to weaken a context as a kind of record subtyping.


\subsubsection{Terms and their Types}
\begin{figure}
\begin{mathpar}
\inferrule[T-Var-1]
{x : \tau \in \Gamma}{\Psi ; \Theta ; \Delta ; \Omega ; \Gamma\vdash x : \tau}

\inferrule[T-TensorI]
{
\Psi ; \Theta ; \Delta ; \Omega ; \Gamma_1\vdash e_1 : \tau_1\\
\Psi ; \Theta ; \Delta ; \Omega ; \Gamma_2\vdash e_2 : \tau_2\\
}{
\Psi ; \Theta ; \Delta ; \Omega ; \Gamma_1,\Gamma_2\vdash \angles{e_1,e_2} ; \tau_1 \otimes \tau_2
}

\inferrule[T-TensorE]
{
\Psi ; \Theta ; \Delta ; \Omega ; \Gamma_1\vdash e : \tau_1 \otimes \tau_2\\
\Psi ; \Theta ; \Delta ; \Omega ; \Gamma_2,x : \tau_1, y : \tau_2\vdash e' : \tau'
}{
\Psi ; \Theta ; \Delta ; \Omega ; \Gamma_1,\Gamma_2\vdash \texttt{let } \angles{x,y} = e \texttt{ in } e' : \tau'
}

\\

\inferrule[T-Nil]
{ }
{\Psi ; \Theta ; \Delta ; \Omega ; \Gamma\vdash \texttt{nil} : L^0 \tau}

\inferrule[T-Cons]
{\Psi ; \Theta ; \Delta ; \Omega ; \Gamma_1\vdash e_1 : \tau\\
\Psi ; \Theta ; \Omega ; \Gamma_2\vdash e_2 : L^{I} \tau
}
{\Psi ; \Theta ; \Delta ; \Omega ; \Gamma_1, \Gamma_2\vdash e_1 :: e_2 : L^{I + 1} \tau}
\\

\inferrule[T-Match]
{
\Psi ; \Theta ; \Delta ; \Omega ; \Gamma_1\vdash e : L^I \tau\\
\Psi ; \Theta ; \Delta, I = 0 ; \Omega ; \Gamma_2\vdash e_1 : \tau'\\
\Psi ; \Theta ; \Delta, I \geq 1; \Omega ; \Gamma_2, h : \tau, t : L^I \tau \vdash e_2 : \tau'\\
}
{\Psi ; \Theta ; \Delta ; \Omega ; \Gamma_1,\Gamma_2\vdash \texttt{match}(e,e_1,h.t.e_2) : \tau'}


\inferrule[T-Ret]
{
\Psi ; \Theta ; \Delta ; \Omega ; \Gamma \vdash e : \tau
}{
\Psi ; \Theta ; \Delta ; \Omega ; \Gamma \vdash \texttt{ret}\; e : \M \, (I,\vec{0}) \, \tau
}

\inferrule[T-Tick]
{
\Theta ; \Delta \vdash I : \mathbb{N}\\
\Theta ; \Delta \vdash \vec{p} : \vec{\mathbb{R}^+}
}{
\Psi ; \Theta ; \Delta ; \Omega ; \Gamma \vdash \texttt{tick}[I|\vec{p}] : \M \, (I,\vec{p})\, 1
}


\inferrule[T-Bind]
{
\Psi ; \Theta ; \Delta ; \Omega ; \Gamma_1 \vdash e_1 : \M \, (I,\vec{p})\, \tau_1\\
\Psi ; \Theta; \Delta ; \Omega ; \Gamma_2, x:\tau_1 \vdash e_2 : \M \, (I,\vec{q})\, \tau_2\\
}{
\Psi ; \Theta ; \Delta ; \Omega ; \Gamma_1,\Gamma_2 \vdash \texttt{bind } x = e_1 \texttt{ in } e_2 : \M \, (I,\vec{p} + \vec{q})\, \tau_2
}

\inferrule[T-Release]
{
\Psi ; \Theta ; \Delta ; \Omega ; \Gamma_1 \vdash e_1 : [I | \vec{q}] \tau_1\\
\Psi ; \Theta ; \Delta ; \Omega ; \Gamma_2, x : \tau \vdash e_2 : \M \, (I,\vec{p} + \vec{q}) \, \tau_2\\
}{
\Psi ; \Theta ; \Delta ; \Omega ; \Gamma_1,\Gamma_2 \vdash \texttt{release } x = e_1 \texttt{ in }e_2 : \M \, (I,\vec{p}) \, \tau_2
}
\\

\inferrule[T-Store]
{
\Theta ; \Delta \vdash I : \mathbb{N}\\
\Theta ; \Delta \vdash \vec{p} : \vec{\mathbb{R}^+}\\
\Psi ; \Theta ; \Delta ; \Omega ; \Gamma \vdash e : \tau\\
}{
\Psi ; \Theta ; \Delta ; \Omega ; \Gamma \vdash \texttt{store}[I|\vec{p}](e) : \M \, (I,\vec{p}) \, ([I| \vec{p}] \, \tau)
}


\inferrule[T-Sub]
{
\Psi ; \Theta ; \Delta ; \Omega ; \Gamma \vdash e : \tau'\\
\Psi;\Theta;\Delta \vdash \tau' \subty \tau
}{
\Psi ; \Theta ; \Delta ; \Omega ; \Gamma \vdash e : \tau
}

\inferrule[T-Weaken]
{
\Psi ; \Theta ; \Delta ; \Omega ; \Gamma \vdash e : \tau\\
\Theta ; \Delta \vdash \Omega' \bdby \Omega\\
\Theta ; \Delta \vdash \Gamma' \bdby \Gamma\\
}{
\Psi ; \Theta ; \Delta ; \Omega' ; \Gamma' \vdash e : \tau
}
\end{mathpar}
\label{fig:dlambdaamor-selected-typing-rules}
\caption{Selected \dlambdaamor rules}
\end{figure}

The typing rules for all of the logical connectives have the standard caveats for an affine type system: affine arrow introduction T-ArrI binds variable $x : A$ in the affine context $\Gamma$. Multi-premise rules like tensor introduction (T-TensorI) and sum elimination (T-Case) require splitting the affine context to type the premises. As usual, ``parallel" premises such as the two arms of a case may share affine resources, as only one branch will be taken at runtime. The exponential modality $!\tau$ also has the standard rules: T-ExpI ensures that one may only introduce a value of type $!\tau$ when the affine context is empty, and T-ExpE destructs a value of type $!\tau$ by binding an exponential variable of type $\tau$ for use in the continuation.

Of greater interest are the rules for the cost analysis and refinement type-related constructs. The return of the cost monad lifts a pure value $e : \tau$ to a monadic computation $\texttt{ret}(e)$ which incurs no cost. So, the rule T-Ret types $\texttt{ret}(e)$ at $\M(I,\vec{0}) \, \tau$ for any index term $I$ of sort $\N$, where $\vec{0}$ is the length $k$ vector of $0$s. Since $\phi(I,\vec{0}) = 0$ independent of the base $I$, this rule has the desired effect. Meanwhile, the bind of the cost monad sums the costs of the computation and the continuation. T-Bind operationalizes this by typing $\texttt{bind}\, x = e \, \texttt{in} \, e' \; : \M(I,\vec{p} + \vec{q}) \, \tau_2$ when $e : \M(I,\vec{p}) \, \tau_1$ and $x : \tau_1 \vdash e' : \M(I,\vec{q}) \, \tau_2$/ The soundness of this rule is justified by the linearity of the $\phi$ function, as outlined in Section~\textbf{??}.

The two operations for the potential modality are carefully constructed to work harmoniously with the cost monad. Firstly, given a term $e : \tau$, the rule T-Store allows us to store $\phi(I,\vec{p})$ potential on the term by incurring that amount of cost: this takes the form of assigning the type $\M(I,\vec{p}) \, \left([I|\vec{p}] \, \tau\right)$ to the term $\texttt{store}[I|\vec{p}](e)$. Note that to access the underlying potential, one must first $\texttt{bind}$ the computation, in effect incurring the requisite $\phi(I,\vec{p})$ cost to have access to the potential. Dually, the rule T-Release gives the typing for using potential. The potential on a term $e : [I|\vec{p}] \, \tau_1$ can be to pay for a monadic continuation $x : \tau_1 \vdash e' : \M(I,\vec{q} + \vec{p}) \, \tau_2$ to get
$\texttt{store}\, x = e \, \texttt{in} \, e' : \M(I,\vec{q}) \, \tau_2$. Of course, the rules for constant potentials follow a similar pattern: the constant store expression $\texttt{store}[J](e)$ has type $\M(I,\texttt{const}(J)) \, \left([J] \, \tau\right)$ by T-StoreConst.
We note that the type system enforces a discipline that all potential-related activities happen inside the cost monad, which greatly simplifies the type soundness proof found in \citet{rajani-et-al:popl21}.

The list type ($L^I \, \tau$) is length-indexed, and so its typing rules are somewhat more involved than the standard ones. To enforce the length refinement, the rules T-Nil and T-Cons specify that the empty list $\texttt{[]}$ has type $L^0 \tau$, while a cons list $e :: e'$ has type $L^{I+1} \tau$ for $e : \tau$ and $e' :: L^I \tau$. The list elimination rule T-Match is more or less standard, but the two branches are typed under extra constraints in the constraint context $\Delta$. If the scrutinee has type $L^I \tau$, then the nil case of the match is typed under the assumption that $I = 0$. Meanwhile the cons case is given the assumption $I \geq 1$, and the tail of the list is bound as having type $L^{I-1} \tau$.

In addition to the length-refined lists, the refinement type portion of \dlambdaamor's type system also includes index term quantifiers in types ($\forall,\exists$), as well as the two constraint types ($\Phi \amp \cdot$, $\Phi \implies \cdot$/).  The treatment of the quantifiers is standard: the rules T-ILam and T-ExistE bind index variables in the index context $\Theta$, while the rules T-IApp and T-ExistI substitute in index terms provided by the syntax. The rules for the constraint types operate in a similarly dual fashion.

Finally, \dlambdaamor includes two special ``structural" rules. The first is a subtyping rule T-Sub, which may be used to downcast the type of a term to a less precise one. This rule has no syntactic form, and thus may be inserted anywhere in a derivation. The second is T-Weaken, which allows for the weakening of the two term variable contexts, $\Omega$ and $\Gamma$. As previously mentioned, the weakening relation on which this rule depends includes subtyping, and so a weaker context may include less precise types, and not just fewer available variables.


\subsubsection{Presuppositions}
All of the judgments presented so far are ``raw" judgments-- one may mechanically derive a proof of one using the inference rules, without regard for whether or not the judgment makes any sense. Traditionally, the requisite assumptions for stating a judgment in a sensical manner are known as \textit{presuppositions}. For example, the sort-checking judgment $\Theta ; \Delta \vdash I : S$ requires that the constraint context $\Delta$ be well-formed with respect to $\Theta$. There are many ways of handling these, but in this work we choose to make them explicit. Each raw judgment form has an associated judgment form which packages together the requisite well-formedness presuppositions for that judgment. We denote this by a subscript $p$ on the turnstile.

\begin{definition}
We say that $\Theta ; \Delta \pvdash I : S$ when $\Theta \vdash \Delta \; \texttt{wf}$ and $\Theta ; \Delta \vdash I : S$.
\end{definition}

\begin{definition}
We write $\Psi ; \Theta ; \Delta \pvdash \tau : K$ to mean that $\Theta \vdash \Delta \; \texttt{wf}$ and $\Psi ; \Theta ; \Delta \vdash \tau : K$
\end{definition}

\begin{definition}
We write $\Psi ; \Theta ; \Delta \pvdash \tau \subty \tau' : K$ to mean that
\begin{enumerate}
  \item $\Theta \vdash \Delta \; \texttt{wf}$
  \item $\Psi ; \Theta ; \Delta \vdash \tau : K$
  \item $\Psi ; \Theta ; \Delta \vdash \tau' : K$
  \item $\Psi ; \Theta ; \Delta \vdash \tau \subty \tau' : K$
\end{enumerate}
\end{definition}

\begin{definition}
We say $\Psi ; \Theta ; \Delta \pvdash \Gamma \wknto \Gamma'$ to mean that
\begin{enumerate}
  \item $\Theta \vdash \Delta \; \texttt{wf}$
  \item $\Psi ; \Theta ; \Delta \vdash \Gamma \; \texttt{wf}$
  \item $\Psi ; \Theta ; \Delta \vdash \Gamma' \; \texttt{wf}$
  \item $\Psi ; \Theta ; \Delta \vdash \Gamma \wknto \Gamma'$
\end{enumerate}
\end{definition}

\begin{definition}
We say that $\Psi ; \Theta ; \Delta ; \Omega ; \Gamma \pvdash e : \tau$ when
\begin{enumerate}
  \item $\Theta \vdash \Delta \; \texttt{wf}$
  \item $\Psi ; \Theta ; \Delta \vdash \Omega \; \texttt{wf}$
  \item $\Psi ; \Theta ; \Delta \vdash \Gamma \; \texttt{wf}$
  \item $\Psi ; \Theta ; \Delta \vdash \tau : \star$
  \item $\Psi ; \Theta ; \Delta ; \Omega \vdash e : \tau$
\end{enumerate}
\end{definition}

\subsection{Admissibilities and Metatheory of \dlambdaamor}
\red{(Need the cuts and weakens here... maybe explain the quasi-circularity and knot-untying here?)}


\section{Semantics and Soundness of \dlambdaamor}
For \dlambdaamor to be useful, its type system must be \textit{sound}. In this context, soundness means that the statically-predicted execution costs from the types given to programs are in fact actual upper bounds on the programs' real execution cost. To prove that \dlambdaamor's type system is sound in this way, we will appeal to a version of the soundness proof of \lambdaamorminus. As discussed in Section~\textbf{??}, \lambdaamorminus differs from \dlambdaamor mainly in its treatment of potentials and costs. In fact the two languages are sufficiently similar (by design, of course) that there is a straightforward embedding of \dlambdaamor into \lambdaamorminus. This embedding is cost-preserving, and so the soundness of \dlambdaamor follows immediately from the soundness of \lambdaamorminus. Formally, we do not present a true embedding into \lambdaamorminus, as it does not have a sort of potential vectors! However, potential vectors can be trivially added to \lambdaamorminus: the kripke logical relation which forms the basis for its soundness proof never inspects index terms, and conflates index terms with the semantic objects they denote. For this reason, we freely consider \lambdaamorminus as having a sort of potential vectors in the style of \dlambdaamor.

In Section~\textbf{??}, we present the operational semantics for \dlambdaamor upon which the soundness theorem is based. This semantics is a big-step cost-indexed operational semantics: the cost indices are the concrete notion of cost that will be bounded by the statically-predicted costs in the soundness theorem.

Then, in Section~\textbf{??}, we sketch the embedding of \dlambdaamor into \lambdaamorminus, and further sketch proofs that the cost semantics of \dlambdaamor coincides with that of \lambdaamorminus under the embedding, as well as the overall soundness theorem of \dlambdaamor. Because of \red{(insert excuse)}, we will not present the full details of the translation here, but the strategy is clear, and we see no barriers to its formalization.


\subsection{Operational Semantics of \dlambdaamor}
To pin down the exact cost of programs written in \dlambdaamor, we provide a \textit{cost semantics} for the language: a big-step operational semantics which is indexed by the cost of evaluation.

Operationally, \dlambdaamor behaves like a call-by-name monadic version of PCF. The cost semantics, for which selected rules are presented in Figure~\ref{fig:dlambdaamor-selected-operational-rules}, consists of two separate judgments. First is a \textit{pure} evaluation relation: $e \Downarrow v$, which evaluates an expression of type $\tau$ to a value of the same type. Evaluations in this relation are not thought to incur any cost: in fact, the set of values includes all of the monadic computations, which must be \textit{forced}. This is accomplished with the \textit{forcing} evaluation relation $e \Downarrow^\kappa v'$, which relates monadic values of type $\M \, I \, \tau$ to values of type $\tau$. Selected rules from these two judgments can be found in Figure~\ref{fig:selected-sem-rules}.

The rules for the pure evaluation relation are straightforward- as all monadic terms are values, the pure relation simply behaves like a big-step evaluation relation for by-name PCF. The rules for the refinement syntax at term level behave as if the syntax for refinements has been erased at runtime- they contribute nothing meaningful to the operational semantics.

The rules for the forcing relation warrant some discussion. Since all monadic computations are values, the forcing relation depends on the pure relation to evaluate sub-expressions. For instance, the forcing relation evaluates $\texttt{ret}(e)$ to $v$ in $0$ steps when $e \Downarrow v$. Of course, the pure relation will take some steps of computation by performing $\beta$-redexes, but we will not consider these to be \textit{costly}, and thus do not need to be accounted for in the forcing relation.

Most importantly, the $\texttt{tick}[I|\vec{p}]$ term evaluates with cost $\phi(I,\vec{p})$ to the trivial value $()$. This rule encodes the heretofore intutitive cost behavior of the type $\M (I,\vec{p}) \, \tau$, by explicitly assigning the atomic costly operation the cost $\phi(I,\vec{p})$ in our cost semantics.  The final cost-monadic term, the \texttt{bind}, is assigned cost in a purely compositional way. The evaluation of \texttt{bind} proceeds like the evaluation of a let-binding, where the costs of forcing the argument and then the subsequent continuation are added, and given as the total cost.

Finally, the two potential-related operations incur no semantic cost. This may come as a surprise-- the statically predicted cost for the \texttt{store} operation (for example) is the amount of potential to be allocated. However, this cost is entirely for bookeeping purposes to ensure that potentially is used soundly: it is not truly incurred when the program runs. Similarly, the \texttt{release} operation runs identically to \texttt{bind}: it is simply a monadic sequencing. This ``ghost" nature of potential at runtime is congruent with the way we think about amortized analysis. Recalling the notation of Section~\textbf{??}, the operational semantics give the costs $C(f)$, while the static types encode the amortized cost $A(f) + \Delta\Phi$.

\begin{figure}
\label{fig:selected-sem-rules}
\caption{Selected Rules of \dlambdaamor's Cost Semantics}
\end{figure}

\subsection{Embedding of \dlambdaamor in \lambdaamorminus}
The translation of \dlambdaamor into \lambdaamorminus requires little insight: we simply compile the costs and potentials written abstractly as a base and potential vector to the $\phi$ function applied to a pair. Concretely, the meat of the translation on types consists of two rules: the \dlambdaamor cost type $\M(I,\vec{p}) \, \tau$ is translated to the \lambdaamorminus $\M \left(\phi(I,\vec{p})\right)\, \tau'$, and the potential type $[I|\vec{p}] \, \tau$ is translated to $\left[\phi(I,\vec{p})\right] \, \tau'$, where $\tau'$ is the translation of $\tau$. These translations respect rules of the two type systems: the monotonicity and additivity of the $\phi$ function from Theorem~\textbf{??} justify the translations of the \texttt{bind} and \texttt{release} operations, as well as the subtyping rule for costs and potentials. The rest of the translation is primarily an erasure. \lambdaamorminus's syntax includes no explicit index terms or types at the term level, and so these are all erased. In particular, the shift operation is erased, a move which is justified by Theorem~\ref{thm:raml-shift}. This notion is codified by the erasure theorem below.

For the remainder of the section, we will write the embedding on all syntactic forms as $(\cdot)^\circ$. Further, the typing judgments of \lambdaamorminus will be distinguished by a superscript minus on their turnstiles, while the semantic judgments will be distinguished by a subscript minus on the big-step down-arrow.

\begin{theorem}
\label{thm:dla-trans-sound}
If $\Psi ; \Theta ; \Delta ; \Omega ; \Gamma \pvdash e : \tau$ then $\Psi^\circ ; \Theta^\circ ; \Delta^\circ ; \Omega^\circ ; \Gamma^\circ \vdash_{-} e^\circ : \tau^\circ$
\end{theorem}


\subsubsection{Statement of Soundness of \dlambdaamor}
To prove the soundness of \dlambdaamor, we begin by noting that its operational semantics are preserved under the erasure to \lambdaamorminus. This is of course to be expected: \dlambdaamor's cost semantics is simply that of \lambdaamorminus, only written in terms of the abstract costs $\phi(I,\vec{p})$.

\begin{theorem}
\label{thm:dla-trans-sem-sound}
If $e \Downarrow^\kappa v$, then $e^\circ \Downarrow_{-}^\kappa v^\circ$
\end{theorem}

From this, the soundness theorem for \dlambdaamor follows immediately: the actual cost of running a closed monadic computation is bounded above by its statically-predicted amortized cost.

\begin{theorem}
\label{thm:dlambdaamor-sound}
If $\cdot \vdash e : \M \, (I,\vec{p}) \, \tau$ and $e \Downarrow^\kappa v$, then $\kappa \leq \phi(I,\vec{p})$.
\end{theorem}
\begin{proof}
By Theorem~\ref{thm:dla-trans-sound} and Theorem~\ref{thm:dla-trans-sem-sound}, we have that $\cdot \vdash e^\circ  : \M \, \phi(I,\vec{p}) \, \tau^\circ$, and $e^\circ \Downarrow^\kappa v^\circ$. Then, by Theorem 1 of \citet{rajani-et-al:popl21}, we have that $\kappa \leq \phi(I,\vec{p})$, as required.
\end{proof}



\section{Examples of Programs in \dlambdaamor}
Having just spent a great many pages discussing the technical details of \dlambdaamor's syntax, type system, and semantics, we now arrive at the fun part: analyzing the cost of programs! In this section, we will present a number of examples of programs written in \dlambdaamor, each of which exemplifies a different component of its cost analysis capabilities. These examples will loosely follow the presentation of Section 3 of \citet{rajani-et-al:popl21}, where more in-depth discussion can be found.

\subsubsection{Add One}

We begin with a (very) simple example to demonstrate the utility of \dlambdaamor's AARA-style costs. Consider writing a function $\texttt{addOne}$, which adds one to each integer in a list. If we assume the cost model that natural number addition costs one unit of time, the function would have type $\forall n : \N. \, L^n(\texttt{nat}) \loli \M \, (n,\langle 0,1 \rangle) \, \left(L^n(\texttt{nat})\right)$. Recalling the intended meaning of the AARA-style cost functions, this means that \texttt{addOne} costs $\phi(n,\langle 0,1 \rangle) = n$ in total, where $n$ is the length of the input list (and also the output). Of course, this makes sense, as each entry in the list incurs a single cost, to add one to it. The term for this type can be found in Figure~\ref{fig:example-dlambdaamor-addone}. The operational aspects of the program are exactly what one expects from an instance of map. More interesting are the cost-related aspects of the code. In the cons branch, we immediately \texttt{shift}. This allows us to provide a term of type  $\M \, (n-1,\langle 1,1 \rangle) \, \left(L^n(\texttt{nat})\right)$ in place of the expected type $\M \, (n,\langle 0,1 \rangle) \, \left(L^n(\texttt{nat})\right)$. This shift is required to perform the recursive call on the tail: \textbf{??} has type $\M \, (n-1,\langle 0,1 \rangle) \, \left(L^{n-1} (\texttt{nat})\right)$, which can only be bound into a continuation which results in something of type $\M \, (n-1, \_) \, \_$. Further, the shift ``exposes" the one constant cost, which is incurred by the tick (which we attribute to the addition). This raises a crucial point: a ``hole" in a program expecting $\M \, (n,\langle 0,1 \rangle) \, \tau$ cannot accept a term of type $\M \, (n, \langle 1,0 \rangle) \, \tau$, despite this being semantically sound.

\begin{figure}
\label{fig:example-dlambdaamor-addone}
\caption{\texttt{addOne} function in \dlambdaamor}
\end{figure}


This example can also be performed using potentials, rather than costs. Instead of a function which incurs $n$ cost, we can instead think of \texttt{addOne} as a free-to-execute function which expects $n$ potential. One possible choice for this function's type is:
$$\forall n : \N .\, [n|\langle 0,1 \rangle] \, 1 \loli L^n(\texttt{nat}) \loli \M \, (n,\langle 0,0 \rangle) \, \left(L^n(\texttt{nat})\right)$$
This style is indicative of ``gas-cost" analyses, as we expect $n$ gas up front to run, and spend it all towards performing the additions. For technical reasons relating to expressivity\footnote{
In short, coeffect-style analyses require the use of potentials.
}, we often use this style (preferring the type $[I] \, \tau \loli \M \, 0 \, \sigma$ over the type $\tau \loli \M \, I \, \sigma$) even in cost analyses which are not amortized. \red{Should this be said elsewhere?}

Another option is a type which attaches a single potential to each element of the input list, in a style indicative of the Banker's method:
$$
\forall n : \N .\, L^n\left([1] \, \texttt{nat}\right) \loli \M \, (n,\langle 0,0 \rangle) \, \left(L^n(\texttt{nat})\right)
$$
The terms corresponding to both of these types can be found in Appendix~\textbf{??}. Of course, this cost analysis is tight and fairly uninteresting: it requires no ``real" amortized analysis.

To illustrate the utility of \dlambdaamor as a platform for actual amortized analysis, we show how a few classic examples of amortized analysis can be performed by writing the programs in \dlambdaamor.

\subsubsection{Insertion Sort}
\red{Talk about how you can EZ PZ do quadratic analyses}

\subsubsection{Functional Queue}
The first example of amortized analysis is the traditional functional queue \citehere. Here, a queue is represented as a pair of lists, $l_f$ and $l_r$, which we refer to as the front and rear lists, respectively. To enqueue an element, we cons it to the head of the front list, and to dequeue, an element is removed from the head of the rear list. If the rear list is empty when a dequeue operation is issued, the front list is reversed into the rear.

If we assume that cons operations are the only costly operation, and that they each incur one cost, this dequeue operation has worst-case complexity $O(n)$ where $n$ is the size of the queue (the sum of the sizes of $l_f$ and $l_r$), since it sometimes needs to reverse the entire front list. However, by employing the banker's method, we may enforce the invariant that each element of the front list carries two credits to be used to pay for its eventual reversal. Under this scheme, both enqueue and dequeue are constant time.

This entire informal analysis is captured formally by the types\footnote{
We will sometimes write $\M \, \vec{p} \, \tau$ to mean $\forall j : \N. \M \, (j,\vec{p}) \, \tau$.
} of the enqueue and dequeue operations in \dlambdaamor. To encode this analysis, we define a queue to be of type $ L^n([2] \, \tau) \otimes L^m \, \tau$: a pair of $\tau$-lists, where the front has $2$ potential on each of its $n$ elements.

The enqueue function has the following fairly obvious type.

$$
\texttt{enq} \; : \; \forall n,m : \N. \, [3] \, 1 \loli \tau \loli L^n([2] \, \tau) \otimes L^m \, \tau \loli \M \, \langle 0 \rangle \, \left(L^{n+1}([2] \, \tau) \otimes L^m \, \tau\right)
$$

From a queue and three extra potential, we may enqueue a single element, resulting in queue with one more element on its front list, for no cost. The term implementing \texttt{enc} can be found in Figure~\ref{fig:example-dlambdaamor-enc}. The type of dequeue is somewhat more involved, since the sizes of the output lists are not a simple function of this inputs. In addition, the function has a precondition: the queue cannot be empty. These two numerical restrictions provide a nice illustration of \dlambdaamor's refinement types.

$$
\texttt{deq} \; : \; \forall m,n : \N. (m + n > 0) \implies L^n([2] \, \tau) \otimes L^m \, \tau \loli \M \langle 0 \rangle \left(\exists n',m' : \N. (n' + m' + 1 = n + m) \amp \left(L^n([2] \, \tau) \otimes L^m \, \tau\right)\right)
$$

\texttt{deq} takes a nonempty queue, and produces another queue which has one element removed. The implementation of \texttt{deq} relies on a function \texttt{move}, which reverses the rear list into the front. The terms for \texttt{move} and \texttt{dec} can be found in Appendix~\textbf{??}.
\red{Do i want to do the binary counter here?}


\begin{figure}
\label{fig:example-dlambdaamor-enc}
\caption{\texttt{enc} function in \dlambdaamor}
\end{figure}

\subsubsection{Cost-Parametric Map}
While many existing languages and type systems for (amortized) resource analysis also support higher-order functions, the allowable analyses with higher-order functions are limited. One such limitation is that function arguments to higher-order functions are usually assumed to be constant-cost: for instance, in the cost analysis of a map, each application of the mapping function is assumed to incur the same amount of cost.

To improve on this, we employ a cost family $C : \N \to \R^+$ to encode the costs of each application of the function: the $i$-th call to the function is thought to incur $C(i)$ cost. Then in total, the map function incurs $\sum_{0 \leq i < n} C(i)$ cost. This analysis is reified in the type of map:
$$
\texttt{map} \; : \; \forall \alpha,\beta : \star. \forall C : \N \to \R^+. \forall n : \N. \, !\left(\forall i : \N. [C \, i] \, 1 \loli \texttt{Nat}(i) \loli \alpha \loli \M \, \langle 0 \rangle\,  \beta\right)
\loli !\texttt{Nat}(n)
\loli L^n \,\alpha \loli \M \, \langle \texttt{const}\left(\sum_{0 \leq i < n} C(i)\right) \rangle\, \left(L^n \, \beta\right)
$$

Most importantly, the mapping function has type $!\left(\forall i : \N. [C \, i] \, 1 \loli \texttt{Nat}(i) \loli \alpha \loli \M \, \langle 0 \rangle\,  \beta\right)$. Since it must be applied to each element of the list, its type is $!$-ed to ensure it may be duplicated. The function is parameterized by the index $i$ on which it operates. To ensure that the mapping function at $i$ is actually only ever used at index $i$, the mapping function takes an additional argument of type $\texttt{Nat}(i)$, which is the singleton type of natural numbers equal to $i$\footnote{
This is simply an alias for $L^i \, 1$.
}. Finally, the mapping function requires $C \, i$ potential to run, and incurs no amortized cost, which ensures that its actual cost is bounded by $C \, i$.

Given the mapping function, the function \texttt{map} then transforms an $L^n \, \alpha$ into a monadic computation of an $L^n \, \beta$, incurring $\sum_{0 \leq i < n} C(i)$ amortized (\red{You should go back and make sure you're calling the static costs ``amortized cost" everywhere, that's nice.}) cost. As usual, the term implementing map can be found in Appendix~\textbf{??}


\subsection{Church Numerals}
A similar trick can be used to write other cost-parametric higher-order functions. One particularly interesting instance of this is the iteration function: from a  function $\texttt{f} : \tau \loli \tau$, this functional computes $\texttt{f}^n : \tau \loli \tau$: this function is perhaps better known as the church numeral $n$. If \texttt{f} has cost $c$, then $\texttt{f}^n$ clearly has cost $cn$. In \dlambdaamor, we can give this function a \textit{far} more precise type which encodes a very strong analysis of the cost behavior of church numerals.

First, we generalize the monomorphic iteration to iteration over a sequence of types: the  church numeral $n$ accepts a sequence of maps $\alpha \, i \loli \alpha \, (i + 1)$ for any $\N$-indexed family of types $\alpha$, and produces a function $\alpha \, 0 \loli \alpha \, n$. Next, we allow for the transition maps to be \textit{costly}. Similarly to the map, we index by a cost family $C : \N \to \R^+$ to allow for the possibility that the functions each have different amortized costs.  The intuitive cost analysis is again similar to that of map. To add some nontrivial cost into the mix, we will require that all function applications incur one cost. The church numeral $n$ applies the sequence of maps in order, each incurring $C \, i$ cost for $0 \leq i < n$, so in total, the resulting function $\alpha \, 0 \loli \alpha \, n$ has cost $n + \sum_{i < n} C \, i$, which accounts for the costs to run the functions, plus the $1$ cost to apply each of them. In the full type of church numerals, these costs are represented as potentials in negative position. Finally, we define the type of church numerals $\texttt{Nat}(n)$ \red{(Notational overload here... do we stick with the original?)} as an indexed type of kind $\N \to \star$.

$$
\texttt{Nat} \, : \, \N \to \star \, = ??
$$

With this type in hand, we can begin to write down church numerals! The church numeral zero is trivial: it is essentially the identity. Moreover, it should be intuitively clear (by parametricity) that the term shown below is the \textit{only} inhabitant of its type.

$$
\texttt{zero} \, : \, \texttt{Nat} \, 0 = ??
$$

More interesting however, are the church numeral operations! Most basic among them is the successor function, of type $\forall n : \N. [2] \, 1 \loli \texttt{Nat} \, n \loli \M \, \langle 0 \rangle \, \left(\texttt{Nat} \, (n+1)\right)$. The basic idea is simple: given a church numeral $N$, we produce a new one by iterating from $0$
to $n$ by $N$, and then applying one last transition function. The full term, however, is fairly involved, and it can be found (along with church numeral addition) in Appendix~\textbf{??}

\section{\bilambdaamor}
\label{sec:bilambdaamor}
In order for \dlambdaamor to be useful as a programming language, it must be implementable! While a declarative type system on paper is useful for modeling and proving purposes, it has limited utility from a language design standpoint. While \dlambdaamor calculus described in Section~\ref{sec:dlambdaamor} is far more implementation-ready than its predecessor \lambdaamorminus, the rules of the type system do not provide us with an obvious implementation method. Traditionally, one hopes to implement a type system in a manner similar to implementing a definitional interpreter \red{cite reynolds here}. For each judgment of the type system, the programmer writes a function which essentially runs a proof search for that judgment.

For some of the judgments of \dlambdaamor, such as the sort-assignment judgment for index terms, a proof search procedure seems straightforward to define. For others, such as subtyping or the type-assignment judgment for terms, a few features of the type system present five immediate challenges.

\begin{enumerate}
  \item The main typing judgment is ambiguous. It is not at all clear which rule to apply at any given step of building a derivation, since the subtyping and weakening rules can always be tried at each stage. Indeed, one could always implement proof search for \dlambdaamor using backtracking, but it is preferable to avoid this if possible. Instead, we would like our implementation-ready calculus \bilambdaamor \red{mention the name earlier} to be \textit{syntax-directed} in the sense that the outermost syntax of the current term informs us which typing rule must be applied next to build a successful derivation. 
  
  \item \dlambdaamor includes full System F impredicative polymorphism, but a well-known result of \red{(figure out who)} states that type inference for System F is undecidable. Hence, we will not be able to design a type inference algorithm for \dlambdaamor. A natural second option is to shoot for implementing a type checker. Unfortunately, this too has its limitations. To implement proper type checking, the syntax of \dlambdaamor would have to be changed such that every variable binder includes a type annotation. This is a heavy burden on the programmer: annotating binders with types is tedious, error prone, and generally uninteresting\footnote{
As \citet{pierce:lics03} notes: ``The more interesting your types get, the less fun it is to write them down!"
%https://www.cis.upenn.edu/~bcpierce/papers/tng-lics2003-slides.pdf
  }. Instead, \bilambdaamor adopts \textit{bidirectional type checking}, a technique pioneered by \citehere which trades off some of the generality of full type inference for added ergonomics over standard type checking. The mechanics of this technique are discussed in Section~\textbf{??}
  
  \item \dlambdaamor's subtyping relation provides a challenge which should be familiar to the reader who is versed in the implementation of dependent type theories. The inclusion of the two subtyping rules S-Fam-Beta1 and S-FamBeta2 (found in Figure~\ref{fig:dlambdaamor-selected-typing-rules} or Figure~\textbf{??}) mean that the deciding the subtyping relation includes deciding $\beta$ equality at the type level. Luckily, the equational theory of types is simpler than that of a simply-typed lambda calculus, since the type-level lambda in \dlambdaamor $\lambda i : S. \tau$ ranges over index terms, not types. This allows for a very simple single-pass normalization procedure which decides the subtyping relation: this is discussed in Section~\textbf{??}.
  
  \item Many of the crucial rules of the \dlambdaamor subtyping relation include constraint satisfiability premises of the form $\Theta ; \Delta \vDash \Phi$. These premises will need to be discharged by an SMT solver. However, repeatedly pausing the subtyping algorithm to send constraints to a solver is inefficient. Instead, we would prefer to do one pass of typechecking, followed by a single call to the solver. To achieve this, the judgments of \bilambdaamor ``output" constraints. The intended meaning of this is that when the constraints are valid, the declarative version of the same judgment is derivable.
  
  \item The final barrier to implementation comes not from the refinement type or cost analysis features of \dlambdaamor, but simply from the fact that it is an affine type system. As an illustration, consider the typing rule T-TensorI: the ``input" context to the typechecker must be split into two disjoint parts which can be used in the two premises. This choice is nondeterministic: there is no way to know a priori what allocation of resources to give to each premise until later. To solve this, we employ a classical technique for implementing substructural type systems, known as \red{(is it though?)} the IO method.

\end{enumerate}


\subsection{Overview of Solutions (change this name)}

Below, we present the solutions to these five problems that we choose to adopt. All five solutions are well-known techniques, but to our knowledge \red{try a bit harder to make sure this is true before you say it...} \bilambdaamor is the first type system to show that they may all be simultaneously integrated into a single system. Moreover, since the five techniques are orthogonal, we present each feature of \bilambdaamor in isolation for a significantly more simple language. In Section~\textbf{??}, we show how all of the techniques are applied to \dlambdaamor to form the algorithmic type system \bilambdaamor.

\subsubsection{Bidirectional Type Systems}
\label{sec:bilambdaamor-overview-bidir}
Bidirectional type checking, also known as ``local type checking" is a type system algorithmization technique pioneered by \red{Pierce and Turner}. The technique works by separating the type checking judgment $\Gamma \vdash e : \tau$ of a declarative type system into two algorithmic judgments: $\Gamma \vdash e \checks \tau$ and $\Gamma \vdash e \infers  \tau$, which are read ``$e$ checks against $\tau$" and ``$e$ infers $\tau$" (sometimes ``synthesizes"), respectively. These two judgments are mutually-recursively defined in a specific manner. The process of turning a declarative type system into a bidirectional algorithmic one is straightforward to the point of mechanical: Dunfield and Pfenning \citehere provide a simple-to-follow recipe for this conversion, which extends from the simple type system they consider all the way to \dlambdaamor. 

Syntax-directed algorithmic type systems presented in a bidirectional style are trivially implementable: the implementation strategy is built into the structure of the rules. To implement a bidirectional type system, one writes two mutually-recursive functions \texttt{check:ctx->tm->typ->bool} and \texttt{infer:ctx->tm->typ} by recursion on the term input: the recursive calls are guided by the premises of each rule. Note that the types of these functions indicate the intended \textit{modes} of the three positions of the judgment, in the sense of logic programming. In the checking judgment, all positions are imagined to be \textit{inputs}, while the inference judgment indicates that the type position is an \textit{output} of the judgment.

As alluded to earlier, the ``inference" of the judgment $\Gamma \vdash e \infers \tau$ is not full inference, but merely ``local" inference: this judgment is derivable when enough information is present the form of $e$ to determine its type. This is in contrast to full type inference, where the type of a term may not be fully known until its type constraints are put in the context of those from the larger term in which it sits. For this reason, every syntactic form in the language has either an inference or checking rule: if requiring one of the premises to be inference gathers enough information to determine the type of the conclusion, then that conclusion will be an inference judgment. Otherwise, the judgment will be checking.

%\subsubsection{Subsumption and Annotation}

To mediate between the two judgments, bidirectional type systems include two special rules. First, is the rule which is traditionally referred to as ``subsumption": to show that $e \checks \tau$, it suffices to show that $e \infers \tau$. In other words, if $e$ can infer a type, then it checks against that type. This rule is usually strengthened by subtyping:
$$
\infer{\Gamma \vdash e \checks \tau}{\Gamma \vdash e \infers \tau' & \tau' \subty \tau}
$$ For $e$ to check against $\tau$, it suffices for $e$ to synthesize a more precise type $\tau'$.

Going in the other direction from a checking premise to an infering conclusion is somewhat more involved. In general, the desired converse rule is not true: there will always be terms such that $e \checks \tau$ but it is not the case that $e \infers \tau$. To remedy this, bidirectional type systems introduce a new piece of syntax to the declarative language on which they're based: annotations. When $e$ checks against $\tau$, the annotated term $(e : \tau)$ infers the type $\tau$:
$$
\infer{\Gamma \vdash (e : \tau) \infers \tau}{\Gamma \vdash e \checks \tau}
$$

These annotations must be manually added to terms by the programmer as they write the program. However, the only place where annotations are truly required are at the sites of \textit{bare $\beta$-redexs}. For example, to check the term $(\lambda x. e)\, e'$,, it must be annotated as $(\lambda x.e : \tau \to \sigma) \, e'$. Since most programs only contain bare $\beta$-redexes in the form of let-bindings, this requirement is both predictable and fairly ergonomic.

It is important to remember that these annotations are \textit{not} present in a declarative syntax. It will eventually be useful (when discussing the relation between \bilambdaamor and \dlambdaamor) to have the ability to talk about the ``underlying" declarative term of an algorithmic term, which is achieved by simply removing all type annotations $(e : \tau)$ from a term. We usually denote this $|e|$, when $e$ is an algorithmic term, and sometimes refer to it as the \textit{erasure} of a term. The erasure can be trivially defined by recursion on raw terms, with the critical case being $|(e : \tau)| = |e|$.

% \subsubsection{Soundness and Completeness}
As of yet, the relationship between a declarative calculus and its bidirectional algorithmic counterpart has been left unstated. However, the point of the bidirectional calculus is to be able to algorithmically generate declarative derivations! To this end, one always requires that the bidirectional type system be \textit{sound} for the declarative one.
\begin{theorem}[Bidirectional Soundness]
If $\Gamma \vdash e \checks \tau$, then $\Gamma \vdash e : \tau$
\end{theorem}
In other words, running $\texttt{check}(\Gamma,e,\tau)$ and getting \texttt{true} is sufficient to show that $e$ in fact has type $\tau$.

Conversely, completeness is also desirable, but not strictly necessary for bidirectional type systems. However, the most obvious statement of completeness ($\Gamma \vdash e : \tau$ implies $\Gamma \vdash e \checks \tau$) does not hold! This is because of the annotation requirement: the term $e$ may contain un-annotated bare $\beta$-redexes. For this reason, the following slightly weaker theorem is used as the completeness result for bidirectional type systems.
\begin{theorem}[Bidirectional Completeness]
If $\Gamma \vdash e : \tau$, then there exists $e'$ such that $\Gamma \vdash e' \checks \tau$, and $|e'| = e$, where $|e'|$ is the annotation-erasure of $e'$.
\label{thm:bidir-compl-example}
\end{theorem}

When proven constructively, this completeness result encodes an algorithm which inserts annotations into the term $e$ so that the resulting term checks against $\tau$. When Theorem~\ref{thm:bidir-compl-example} is proven directly by induction, the algorithm it encodes introduces far more annotations than is often strictly necessary: we improve on this with our completeness proof of \bilambdaamor in Section~\textbf{??} by proving an equivalent statement whose constructive proof inserts fewer annotations than the standard theorem.

\subsubsection{Algorithmic Subtyping and Normalization}
To implement the subsumption rule mentioned above, a decision procedure for the subtyping relation $\tau \subty \tau'$ is required. However, \dlambdaamor's subtyping is not immediately implementable for two important reasons.

Firstly, like \dlambdaamor's typing relation, it is not syntax directed: the transitivity rule S-Trans can be used at any step of a derivation. Similarly, the reflexivity rule T-Refl conflicts with all of the congruence rules. To avoid a backtracking implementation, it will be necessary to design an algorithmic subtyping relation for \bilambdaamor which includes neither of these rules. Of course, the algorithmic subtyping will need to be sound and complete for the declarative one. This requirement means that the algorithmic subtyping relation will need to have reflexivity and transitivity as admissible rules: in effect, we will need to prove identity and cut elimination.

The second (and more pernicious) problem is the inclusion of indexed types. While many refinement type systems (including DML \citehere, on which \lambdaamor's refinement types are based) include indexed types \citehere, they are usually implemented only as types of the form $\forall i : S. \tau$ of kind $\star$. While useful, these indexed types are not fully general, as their abstraction and application is controlled by term-level introduction and elimination rules. Instead, \dlambdaamor includes indexed types of the form $\lambda i : S. \tau$, which allow the programmer to use a richer set of types. But, the inclusion of type-level abstractions and applications requires the subtyping relation to include $\beta$ equalities for these indexed type families (S-Fam-Beta{1,2}): without them, the subtyping relation would not be able to judge relations like $(\lambda i : \N. L^i\, \tau) \, 3 \subty L^3 \, \tau$, where the subtying relation holds up to $\beta$ equality.

The inclusion of the two $\beta$-inequalities makes a simple algorithmic subtyping relation unlikely, since any way of deciding the subtyping relation must also decide $\beta$ equality of this small lambda calculus at the type level. However, the situation is sufficiently simple that we can get away with a fairly low-powered solution. To this end, \bilambdaamor's subtying relation is split into two phases. First, both types are evaluated (or \textit{normalized}) to normal forms, and then judged for subtyping by a relation which only contains the congruence rules. Since the abstractions $\lambda i : S.\tau$ range over \textit{index terms} and not types, a $\beta$ reduct has strictly fewer type connectives than its redex. For this reason, the normalization can be implemented in a single pass: substituting an index term for a free variable in a type in normal form yields another type in normal form. This two-phase algorithmic subtyping relation, as well as the normalization proof, are discussed in detail in Section~\textbf{??}

\subsubsection{Constraint Generation}
As motivated in Section~\textbf{?? intro}, most of the changes to \lambdaamorminus that result in \dlambdaamor are there for the purpose of simplifying the constraint-solving process that arises as a part of subtyping. Efficiently handling these constraints is crucial to an efficient implementation. For this reason, it is useful to defer the discharging of these constraint satisfiability premises of rules until \textit{after} the typechecking pass has finished.

We operationalize this in \bilambdaamor by designing each judgment to ``output" a constraint: we replace declarative judgments $\mathcal{J}$ with algorithmic ones $\mathcal{J} \gens \Phi$. For instance, the declarative sort-checking judgment $\Theta ; \Delta \vdash I : S$ of \dlambdaamor corresponds to the algorithmic judgment $\Theta ; \Delta \vdash I : S \gens \Phi$ from \bilambdaamor. The intended meaning of this (and the shape of the soundness theorem for an algorithmic judgment with a constraint output) is that if we can derive $\mathcal{J} \gens \Phi$ and $\Phi$ holds, then $\mathcal{J}$ is derivable.

This scheme is pervasive. Since every judgment in \dlambdaamor either has a rule with a constraint satisfaction premise or depends on one that does, every judgment in \bilambdaamor must emit constraints. The pattern in transforming a declarative judgment to an algorithmic one which emits constraints is fairly uniform: the output constraint of a rule is essentially the conjunction of the constraints output by its premises. One must also ensure that implications and quantifiers are inserted for constraints and index variables bound in premises: the logical structure of the output constraint mirrors the structure of the premises.

\red{Talk a bit here about where these arise in other places}

\subsubsection{I/O Method}
\label{sec:bilambdaamor-overview-io}
To maintain the soundness of potentials, \dlambdaamor has an affine type system. On top of the implementation challenges created by the fancier aspects of \dlambdaamor's type system, its affine-ness presents a well-understood barrier to implementation. To illustrate, consider writing the following case of the \texttt{check:ctx->tm->typ->bool} function from earlier.\footnote{
To simplify some of the presentation of this section, we will specialize to the non-bidirectional setting, and work in a simply-typed language where binders are fully annotated. \red{Should I say this in the main copy?}
}

$$
\texttt{check gamma Pair(e1,e2) Tensor(t1,t2) = } ??
$$

This case corresponds to the introduction rule for tensor,

%As outlined in Section~\textbf{??}, the implementation of bidirectional type systems usually takes the form of two mutually recursive functions \texttt{check:ctx->tm->typ->bool} and \texttt{infer:ctx->tm->typ}. These functions are implemented recursively on the second argument, and the recursive calls for a specific case are dictated by the premises of the corresponding typing rule. However, in the presence of an affine type system, this clean story is somewhat complicated. To illustrate, consider this simplified version of a first cut at the algorithmic tensor introduction rule.
$$
\infer{\Gamma_1,\Gamma_2 \vdash (e_1,e_2) : \tau_1 \otimes \tau_2}{\Gamma_1 \vdash e_1 : \tau_1 & \Gamma_2 \vdash e_2 : \tau_2}
$$


It is not at all clear how to proceed in this case. The tensor introduction rule prescribes that we make two recursive calls \texttt{check gamma1 e1 t1} and \texttt{check gamma2 e2 t2}, but provides no direction how to obtain \texttt{gamma1} and \texttt{gamma2} from \texttt{gamma}: the rule is presented in the standard way so that the two halves of the context are given at the outset.

This problem has two naive solutions. Firstly, one could analyze the structure of \texttt{e1} and \texttt{e2} to determine the variables they each use, and partition the context accordingly. Of course, this is very inefficient: even if done with a pre-processing step, this adds at least one pass through the term. Secondly, one could split the context \textit{symbolically}, and generate yet more constraints to unify at the end of the typechecking process.

Instead of either of these, we adopt a more principled approach based on the work of \citet{cervesato:tcs00}. In short, we extend the main typing judgment with yet another output-- this time a second context, which contains the variables which were unused in typing the term. A simplified version of the typing judgment takes the form $\Gamma \vdash e : \tau \gens \Gamma'$, where $\Gamma$ is the \textit{input context}, and $\Gamma'$ is the \textit{output context}. The key idea of this setup (known sometimes as the I/O method) is that we may thread the contexts through the premises of a rule as follows:

$$
\infer{\Gamma \vdash (e_1,e_2) : \tau_1 \otimes \tau_2 \gens \Gamma_2}{\Gamma \vdash e_1 : \tau_1 \gens \Gamma_1 & \Gamma_1 \vdash e_2 : \tau_2 \gens \Gamma_2}
$$

The first premise (the first component of the pair) has access to the entire input context, and it outputs $\Gamma_1$, the variables in $\Gamma$ which were unused in typing $e_1$. This context is then used as the \textit{input} context for checking $e_2$: since affine variables may be used at most once, the only variables which $e_2$ may access are those unused by $\Gamma_2$. This property is enforced ``in parallel" by splitting the context up front in the declarative rules, but it may similarly be enforced ``sequentially" by lazily deciding which premises may use which variables in this algorithmic styles.

The key rule in designing an algorithmic type system which uses the I/O method is of course the affine variable rule. When a variable is used, it must be removed from the output context:

$$
\infer{\Gamma \vdash x : \tau \gens \Gamma \setminus \{x\}}{x : \tau \in \Gamma}
$$

Moreover, this I/O method will be trivial to implement. We simply change the type of \texttt{check} and \texttt{infer} to output a context as well. Since these functions both receive and output a context, one can think of typechecking with the I/O method as happening inside a state monad of contexts, as opposed to the usual reader monad.

While this solution is clearly preferable to the naive ones efficiency-wise, it is not at all clear that this way of algorithmizing an affine type system is sound, much less complete, for the standard presentation of the rules. The proof of soundness is fairly straightforward, and relies on a simple intuition about the output context: if we can derive that $\Gamma \vdash e : \tau \gens \Gamma'$, then the variables used by $e$ are precisely $\Gamma \setminus \Gamma'$. Writing $\Gamma \vdash e : \tau$ (with no output context) as the declarative typing relation, we can prove

\begin{theorem}[Soundness of the I/O Method]
If $\Gamma \vdash e : \tau \gens \Gamma'$, then $\Gamma \setminus \Gamma' \vdash e : \tau$
\end{theorem}

Note that $\Gamma \setminus \Gamma'$ is well-defined because $\Gamma' \subseteq \Gamma$, a fact which must be proven by induction over the algorithmic rules.

The completeness theorem is simpler to state, but harder to prove.

\begin{theorem}[Completeness of the I/O Method]
If $\Gamma \vdash e : \tau$, then there is some $\Gamma'$ such that $\Gamma \vdash e : \tau \gens \Gamma'$
\end{theorem}

The proof of this theorem relies on the fact while weakening is not admissible for the (affine) declarative type system, it \textit{is} admissible for an algorithmic type system using the I/O method: if new variables are added to the input context, then they simply ``flow through" the judgment to the output context, and are left unused.

\begin{theorem}[Admissibility of Weakening for the I/O Method]
If $\Gamma \vdash e : \tau \gens \Gamma'$, then for all $\Gamma''$, we have that $\Gamma,\Gamma'' \vdash e : \tau \gens \Gamma',\Gamma''$
\end{theorem}

\subsection{Normalization of Types}
To circumvent the issue of deciding $\beta$-equality of types as a part of \bilambdaamor's subtyping routine, we employ a normalization (or evaluation) procedure to eliminate all $\beta$-redexes from a type. Once these $\beta$-redexes have been eliminated, subtyping only requires congruence rules. The normalization proof that we describe in this section is a normalization relative to the equational theory induced by \dlambdaamor's subtyping relation, that is to say: we will eventually prove that a type and its normal form are mutual subtypes of each other \textit{with the subtyping relation of \dlambdaamor}. This may seem strange-- after all, the normalization is required for \bilambdaamor's subtyping relation. However, we will see in Section~\textbf{??} that to prove the completeness of \bilambdaamor's algorithmic subtyping (Theorem~\textbf{??}), a normalization proof for \dlambdaamor's subtyping is exactly what's required. Moreover, the eventual soundness and completeness theorems for algorithmic subtyping will allow us to transport this normalization result to \bilambdaamor's type system when required.

This normalization procedure computes \textit{normal forms} for types, which should be thought of as canonical representatives of the $\beta$-equivalence classes of types. These normal forms can characterized syntactically: we present a pair of relations $\tau \, \texttt{ne}$ and $\tau \, \texttt{nf}$, which judge a type to be neutral or normal, respectively. Neutral types are those which can be of arrow kind, but will not induce any $\beta$-redexes when applied to, while normal types are types which include no $\beta$-redexes. The former are required to define the latter: the type $\tau \, I$ is only in normal form when $\tau$ is not of the form $\lambda i : S.\tau'$. The rules generating these two relations can be found in Appendix~\textbf{??}.

Before we present the normalization function, let us briefly take a moment to discuss why the solution we are about to present is incredibly simple. Proofs of normalization for most calculi require fairly high-powered proof techniques such as logical relations or categorical arguments. The inherent complexity of normalization proofs stems from the fact that straightforward induction on terms rarely works, since one would need to induct on substitution instances of lambda terms which are not subterms of the original term. However, \dlambdaamor's type-level lambdas do not range over types, they range over terms. Because of this, a substitution instance $\tau[I/i] : K$ of an open type $i : S \vdash \tau : K$ has the exact same number of type connectives as the open term. Further, substituting an index term into a type cannot introduce any new redexes, and so any substitution instance of an open type in normal form is also normal. These observations are codified in the following theorems.

\begin{theorem}
$\#\texttt{eval}(\tau) \leq \#\tau$, where $\#(\cdot)$ denotes the number of connectives in a type.
\end{theorem}
\begin{proof}
By induction on $\#\tau$
\end{proof}

\red{Very strange spacing here... fix this}

\begin{theorem}
~\begin{enumerate}
  \item If $\tau \; \texttt{ne}$ then $\tau[I/i] \; \texttt{ne}$
  \item If $\tau \; \texttt{nf}$ then $\tau[I/i] \; \texttt{nf}$
\end{enumerate}
\label{thm:idx-subst-nf}
\end{theorem}
\begin{proof}
We prove the two claims simultaneously by induction on the derivations of $\tau \; \texttt{ne}$ and $\tau \; \texttt{nf}$.
\end{proof}

Because of these simplifying factors, we can define an evaluation function \texttt{eval} defined inductively on the structure of types which computes normal forms.
The most important clauses of the definition can be found in Figure~\ref{fig:selected-eval-rules}. For all of the logical connectives, the definition proceeds compositionally-- the remaining rules can be found in Figure~\textbf{??}

\red{Factor out figures into sep. files}
\begin{figure}
\begin{mathpar}
  \texttt{eval}(\alpha) = \alpha
  
  \texttt{eval}(\lambda i : S. \tau) = \lambda i : S. \texttt{eval}(\tau)
  
  \\  
  
  \texttt{eval}(\tau \; I) = \begin{cases}
   \tau'[I/i] & \texttt{eval}(\tau) = \lambda i : S. \tau' \\
   \texttt{eval}(\tau) \; I & \text{otherwise}
                              \end{cases}
\end{mathpar}
\label{fig:selected-eval-rules}
\caption{Selected Clauses of the \texttt{eval} Function}
\end{figure}

The most important (and only nontrivial) clause of the definition is the application case. To evaluate $\tau \; I$, we begin by evaluating $\tau$. If its normal form
is a lambda, we simply perform the $\beta$-reduction. Note that we do not need to evaluate this substitution instance, as it must already be in normal form by Theorem~\ref{thm:idx-subst-nf}, assuming the correctness of the \texttt{eval} function. Otherwise, we simply re-apply the index term $I$.

Of course, it is not immediately clear that this function in fact computes what we want! In order for \texttt{eval} function to be a normalization procedure, its image must consist only of types in normal form, and every type must be equivalent to its evaluation. Note that we do not prove the stronger property that equivalence is completely characterized by syntactic equality of normal forms (up to satisfied equality of index terms). While almost certainly true, this property requires a bit more work to prove and is not required for the discussion in Section~\textbf{??}, and so we omit it. Finally, we must also prove that the \texttt{eval} function preserves kinds-- this proof follows the same inductive structure as the proof of normalization, and so we bundle them together. We present the case for evaluation below, and the remainder of the cases can be found in Appendix~\textbf{??}. The Normalization Theorem does depend on a small canonical forms lemma: types of arrow kind in normal form must either be lambdas or neutral.

\begin{theorem}[Canonical Forms for $S \to K$]
If $\Psi ; \Theta ; \Delta \vdash \tau : S \to K$ and $\tau \; \texttt{nf}$, then either:
\begin{enumerate}
  \item $\tau = \lambda i : S.\tau'$ with $\tau' \; \texttt{nf}$
  \item $\tau \; \texttt{ne}$
\end{enumerate}
\end{theorem}


\begin{theorem}[Normalization Theorem]
If $\Psi ; \Theta ; \Delta \pvdash \tau : K$, then:
\begin{enumerate}
  \item $\Psi ; \Theta ; \Delta \pvdash \texttt{eval}(\tau) : K$
  \item $\Psi ; \Theta ; \Delta \pvdash \tau \equiv \texttt{eval}(\tau) : K$
  \item $\texttt{eval}(\tau) \; \texttt{nf}$
\end{enumerate}
\label{thm:norm-thm}
\end{theorem}

\red{Include the cut case here.}

\begin{theorem}
~$\texttt{eval}(\tau[J/i]) = \texttt{eval}(\tau)[J/i]$
\label{thm:idx-subst-eval}
\end{theorem}

\section{Algorithmic Type System of \bilambdaamor}
The syntax of \bilambdaamor is nearly identical to that of \dlambdaamor: this is of course by design, as \bilambdaamor is intended to be an implementable version of \dlambdaamor. The only difference is the addition of the type annotation syntax $(e : \tau)$ described in Section~\textbf{??}. The main change between the two type systems is in the forms of the judgments. Some judgments change in only minor ways: the sort-checking, kind-checking, and constraint well-formedness judgments are all the same as in \dlambdaamor, with the exception of the added constraint outputs as described in Section~\textbf{??}. The subtyping judgment also sports a constraint output, but is also is split into two, with first a ``normal form subtyping" relation which judges one type to be a subtype of another when both are in normal form, and then the general algorithmic subtyping relation which relates two types by normalizing them and then relating them via the normal form subtyping relation. Finally, the typing judgment changes the most: it splits into a checking ($\downarrow$) and inferring/synthesis ($\uparrow$) judgment to support bidirectional type inference, with added constraint outputs for solving and unused variable context output for the I/O method. These judgment forms are all shown in Figure~\ref{fig:bilambdaamor-typing-judgments}.

\begin{figure}
\label{fig:bilambdaamor-typing-judgments}
\caption{Judgment Forms of the \bilambdaamor Type System}
\end{figure}

\subsubsection{Sorts, Kinds, and Well-Formed Constraints}
\begin{figure}
\begin{mathpar}
\inferrule[AI-Var]{i : S \in \Theta}{\Theta ; \Delta \vdash i : S \gens \top}

\inferrule[AI-Plus]{\Theta ; \Delta \vdash I : bS \gens \Phi_1 \\ \Theta ; \Delta \vdash J : bS \gens \Phi_2}{\Theta ; \Delta \vdash I + J : bS \gens \Phi_1 \wedge \Phi_2}

\inferrule[AI-Minus]{\Theta ; \Delta \vdash I : bS \gens \Phi_1 \\ \Theta ; \Delta \vdash J : bS \gens \Phi_2}{\Theta ; \Delta \vdash I - J : bS \gens \Phi_1 \wedge \Phi_2 \wedge (I \geq J)}


\inferrule[AI-Sum]{\Theta;\Delta \vdash I_0 : \mathbb{N} \gens \Phi_1\\ \Theta;\Delta \vdash I_1 : \mathbb{N} \gens \Phi_2\\ 
                 \Theta,i : \N;\Delta, I_0 \leq i < I_1 \vdash J : bS \gens \Phi_3}
                 {\Theta;\Delta \vdash \sum_{i=I_0}^{I_1} J : bS \gens \Phi_1 \wedge \Phi_2 \wedge \forall i : \N.(I_0 \leq i < I_1 \to \Phi_3)}

\noindent\makebox[\linewidth]{\rule{\textwidth}{0.4pt}}

\inferrule[AC-Top]{ }{\Theta;\Delta \vdash \top \; \texttt{wf}\gens \top}

\inferrule[AC-Bot]{ }{\Theta;\Delta \vdash \bot \; \texttt{wf}\gens \top}

\inferrule[AC-Conj]{\Theta ; \Delta \vdash \Phi_1\; \texttt{wf} \gens \Phi_1' \\ \Theta ; \Delta \vdash \Phi_2\; \texttt{wf} \gens \Phi_2'}{\Theta;\Delta \vdash \Phi_1 \wedge \Phi_2\; \texttt{wf} \gens \Phi_1' \wedge \Phi_2'}

\inferrule[AC-Forall]{\Theta, i : S ; \Delta \vdash \Phi\; \texttt{wf} \gens \Phi'}{\Theta ; \Delta \vdash \forall i : S. \Phi\; \texttt{wf} \gens \forall i : S. \Phi'}

\inferrule[AC-Eq]{\Theta ; \Delta \vdash I : bS \gens \Phi_1\\ \Theta ; \Delta \vdash J : bS \gens \Phi_2}{\Theta ; \Delta \vdash I = J\; \texttt{wf} \gens \Phi_1 \wedge \Phi_2}

\noindent\makebox[\linewidth]{\rule{\textwidth}{0.4pt}}

\inferrule[AK-Unit]{ }{\Psi ; \Theta ; \Delta \vdash 1 : \star \gens \top}

\inferrule[AK-Tensor]{\Psi ; \Theta ; \Delta \vdash \tau_1 : \star \gens \Phi_1\\ \Psi ; \Theta ; \Delta \vdash \tau_2 : \star \gens \Phi_2}{\Psi ; \Theta ; \Delta \vdash \tau_1 \otimes \tau_2 : \star \gens \Phi_1 \wedge \Phi_2}

\inferrule[AK-IForall]{\Psi ; \Theta, i : S ; \Delta \vdash \tau : \star \gens \Phi}{\Psi ; \Theta ; \Delta \vdash \forall i : S. \tau : \star \gens \forall i :S.\Phi}

\inferrule[AK-Monad]{ \Theta ; \Delta \vdash I : \mathbb{N} \gens \Phi_1\\ \Theta ; \Delta \vdash \vec{p} : \vec{\mathbb{R}^+} \gens \Phi_2\\ \Psi ; \Theta ; \Delta \vdash \tau : \star \gens \Phi_3}{\Psi ; \Theta ; \Delta \vdash \M(I,\vec{p}) \tau : \star \gens \Phi_1 \wedge \Phi_2 \wedge \Phi_3}

\end{mathpar}
\label{fig:bilambdaamor-selected-sort-kind-constr-rules}
\caption{Selected Algorithmic Sort, Kind, and Constraint Rules}
\end{figure}

As we will see is true for the majority of the judgments of \bilambdaamor, the majority of the rules from \dlambdaamor carry over with only minor modification. Although they do form the typing rules for a (small) language embedded in \bilambdaamor, the sort-assignment, kind-assignment and well-formedness judgments for index terms, types, and constraints respectively, do not require a bidirectional treatment. This is because all binders in these three syntactic categories are fully annotated, and so we can easily implement sort/kind inference and checking without any difficulty. Similarly, since the index and type variable contexts are fully structural, there is no need for the I/O method. Hence, the only modification to these three judgments is the addition of the constraint output.

Intuitively, the three judgments all have very simple meanings: for instance, $\Theta ; \Delta \vdash I : S \gens \Phi$ is intended to mean that when $\Phi$ is satisfied, $\Theta ; \Delta \vdash I$ holds declaratively, and similarly for the other two judgment forms. This intuition is made formal by the soundness proofs in Section~\textbf{??}.

We present a few selected rules from these judgments in Figure~\ref{fig:bilambdaamor-selected-sort-kind-constr-rules}. As mentioned earlier, the vast majority of rules are carried over from \dlambdaamor: two good examples are AI-Plus and AC-Conj, which follow an identical structure to their declarative counterparts, and simply conjoin the output contexts from the premises in the conclusion.

Some declarative rules have an instance of the constraint satisfaction relation as a premise: for example, the rule I-Minus requires $\Theta ; \Delta \vDash I \geq J$ to judge $\Theta ; \Delta \vdash I - J : bS$. In the algorithmic judgments of \bilambdaamor, these constraints are conjoined onto the output constraint of the conclusion. The algorithmic rule corresponding to I-Minus, AI-Minus, exemplifies this pattern. It has two premises to check that the two subterms $I$ and $J$ are of the proper sort, which emit constraints $\Phi_1$ and $\Phi_2$, respectively. The output constraint is then $\Phi_1 \wedge \Phi_2 \wedge (I \geq J)$. This is constructed in such a way that our eventual soundness theorem will be simple, if the output constraint is valid, then so is $I \geq J$, and we can thus use $\Theta ; \Delta \vDash I \geq J$ to construct a declarative proof that $I - J$ is sort-correct.

In a similar manner, in rules where premises bind index variables or assume constraints, the bound variable or constraint must be introduced to the conclusion's output constraint to maintain the well-formedness of output constraints. As an example, consider the rules AK-Forall and AC-Forall. Their premises output constraints $\Phi$ which may (and usually do) mention the universal index variable $i : S$, which is bound in the context $\Theta$. Then in the conclusion, this variable is no longer present, and so $\Phi$ need not be well-formed. To fix this, we explicitly quantify over the index variable $i$ in the conclusion's output constraint. 

The assumption context $\Delta$ bears a similar requirement, as illustrated by the rule AI-Sum. When the constraint $I_0 \leq i \leq I_1$ is assumed in a premise which emits a constraint $\Phi_3$, the output constraint is transformed to $I_0 \leq i \leq I_1 \to \Phi_3$ to preserve the meaning of the judgment.

Finally, it's worth noting a potential confusion about the algorithmic constraint well-formedness judgment, $\Theta ; \Delta \vdash \Phi \; \texttt{wf} \gens \Phi'$.
The output constraint $\Phi'$ does not encode the truth of $\Phi$. The soundness proof will make this concrete, but knowing that $\Theta ; \Delta \vDash \Phi'$ only implies that $\Phi$ is well-formed, but it need not be valid.

\subsubsection{Algorithmic Subtyping}

\begin{figure}
\begin{mathpar}
\inferrule[AS-Unit]{ }{\Psi ; \Theta ; \Delta \vdash 1 \subtynf 1 : \star \gens \top}

\inferrule[AS-Tensor]{\Psi ; \Theta ; \Delta \vdash \tau_1 \subtynf \tau_1' : \star  \gens \Phi_1 \\ \Psi ; \Theta ; \Delta \vdash \tau_2 \subtynf \tau_2' : \star \gens \Phi_2}{\Psi ; \Theta ; \Delta \vdash \tau_1 \otimes \tau_2 \subtynf \tau_1' \otimes \tau_2' : \star \gens \Phi_1 \wedge \Phi_2}

\inferrule[AS-Monad]{\Psi ; \Theta ; \Delta \vdash \tau_1 \subtynf \tau_2 : \star \gens \Phi}{\Psi ; \Theta ; \Delta \vdash \M(I,\vec{q}) \tau_1 \subtynf \M(J,\vec{p}) \tau_2 : \star \gens (I = J) \wedge (\vec{q} \leq \vec{p}) \wedge \Phi}

\inferrule[AS-Pot]{\Psi ; \Theta ; \Delta \vdash \tau_1 \subtynf \tau_2 : \star \gens \Phi}{\Psi ; \Theta ; \Delta \vdash [I|\vec{q}] \tau_1 \subtynf [J|\vec{p}] \tau_2 : \star \gens (I = J) \wedge (\vec{p} \leq \vec{q}) \wedge \Phi}

\inferrule[AS-FamLam]{\Psi ; \Theta, i : S ; \Delta \vdash \tau_1 \subtynf \tau_2 : K \gens \Phi}{\Psi ; \Theta ; \Delta \vdash \lambda i : S. \tau_1 \subtynf \lambda i : S. \tau_2 : S \to K \gens \forall i : S. \Phi}

\inferrule[AS-FamApp]{\Psi ; \Theta ; \Delta \vdash \tau_1 \subtynf \tau_2 : S \to K \gens \Phi}{\Psi ; \Theta ; \Delta \vdash \tau_1 \; I \subtynf \tau_2 \; J : K \gens (I = J) \wedge \Phi}

\inferrule[AS-Normalize]{\Psi ; \Theta ; \Delta \vdash \texttt{eval}(\tau_1) \subtynf \texttt{eval}(\tau_2) : K \gens \Phi}{\Psi ; \Theta ; \Delta \vdash \tau_1 \subty \tau_2 : K\gens \Phi}
\end{mathpar}
\label{fig:bilambdaamor-selected-subty-rules}
\caption{Selected Algorithmic Subtyping Rules}
\end{figure}

In \bilambdaamor, there is not one subtyping judgment, but two. The first, which we will refer to as ``normal form" subtyping (denoted $\subtynf$),  judges one type to be a subtype of another when both are in normal form. This relation contains all of the congruence rules from \dlambdaamor's subtyping relation. All of the congruence rules in the normal form subtyping relation are simply transcriptions of their declarative counterparts. Just like with sort/kind-checking, constraint-validity premises are shuffled to the constraint output of the conclusion, and variables bound in premises are quantified over. Deciding a subtyping relation which only includes congruences is of course trivial, and so this relation is certainly algorithmic.

The second judgment (denoted $\subty$) is generated by a single rule, AT-Normalize. AT-Normalize encodes the first step of our two-step subtyping algorithm.
To show that $\Psi ; \Theta ; \Delta \vdash \tau_1 \subty \tau_2 : K \gens \Phi$, it suffices (and indeed it is necessary) to first normalize $\tau_1$ and $\tau_2$, and then judge that their normal forms are related by the normal form subtyping judgment, $\Psi ; \Theta ; \Delta \vdash \texttt{eval}(\tau_1) \subtynf \texttt{eval}(\tau_2) : K \gens \Phi$.
 
This strategy should be familiar to the reader familiar with implementing-dependently typed languages. When implementing a dependent type theory, it is necessary to check equality of types, which may of course include programs. To do so, one first normalizes the types, and then checks them for syntactic equality-- \bilambdaamor's algorithmic subtyping is simply a directed version of this.

Two distinct phases and normalization aside, the remaining way that \bilambdaamor's subtyping differs from \dlambdaamor's is in the removal of two rules. The rules S-Refl and S-Trans from \dlambdaamor are not included in our algorithmic subtyping relation, as they are not syntax-directed. In Section~\textbf{??}, we show that reflexivity and transitivity are admissible for types in normal form, and that these results may be lifted to the full relation through evaluation.


\subsubsection{Bidirectional Typing Rules}

\begin{figure}
\begin{mathpar}
\inferrule[AT-Var-1]
{x : \tau \in \Gamma}{\Psi ; \Theta ; \Delta ; \Omega ; \Gamma\vdash x \infers \tau \gens \top, \Gamma \setminus \{x : \tau\}}

\inferrule[AT-Lam]
{
\Psi ; \Theta ; \Delta ; \Omega ; \Gamma, x : \tau_1 \vdash e \checks \tau_2, \gens \Phi, \Gamma'
}{
\Psi ; \Theta ; \Delta ; \Omega ; \Gamma\vdash \lambda x.e \checks \tau_1 \loli \tau_2 \gens \Phi, \Gamma' \setminus \{x : \tau_1\}
}

% Is this right? Do we just thread the vars to the conclusion to be closed over later?
\inferrule[AT-App]
{
\Psi ; \Theta ; \Delta ; \Omega ; \Gamma\vdash e_1 \infers \tau_1 \loli \tau_2 \gens \Phi_1, \Gamma_1\\
\Psi ; \Theta ; \Delta ; \Omega ; \Gamma_1\vdash e_2 \checks \tau_1 \gens \Phi_2, \Gamma_2
}{
\Psi ; \Theta ; \Delta ; \Omega ; \Gamma\vdash e_1 \, e_2 \infers  \tau_2 \gens \Phi_1 \wedge \Phi_2, \Gamma_2
}


\inferrule[AT-TensorI]
{
\Psi ; \Theta ; \Delta ; \Omega ; \Gamma\vdash e_1 \checks \tau_1 \gens \Phi_1, \Gamma_1\\
\Psi ; \Theta ; \Delta ; \Omega ; \Gamma_1\vdash e_2 \checks \tau_2 \gens \Phi_2, \Gamma_2\\
}{
\Psi ; \Theta ; \Delta ; \Omega ; \Gamma\vdash \angles{e_1,e_2} \checks \tau_1 \otimes \tau_2 \gens \Phi_1 \wedge \Phi_2,\Gamma_2
}

\inferrule[AT-TensorE]
{
\Psi ; \Theta ; \Delta ; \Omega ; \Gamma\vdash e \infers \tau_1 \otimes \tau_2 \gens \Phi_1, \Gamma_1\\
\Psi ; \Theta ; \Delta ; \Omega ; \Gamma_1,x : \tau_1, y : \tau_2\vdash e' \checks \tau' \gens \Phi_2,\Gamma_2
}{
\Psi ; \Theta ; \Delta ; \Omega ; \Gamma\vdash \texttt{let } \angles{x,y} = e \texttt{ in } e' \checks \tau' \gens \Phi_1 \wedge \Phi_2, \Gamma_2 \setminus \{x,y\}
}

\inferrule[AT-Ret]
{
\Psi ; \Theta ; \Delta ; \Omega ; \Gamma \vdash e \checks \tau \gens \Phi,\Gamma'
}{
\Psi ; \Theta ; \Delta ; \Omega ; \Gamma \vdash \texttt{ret } e \checks \M \, \phi(I,\vec{p}) \, \tau \gens \Phi, \Gamma'
}

\inferrule[AT-Bind]
{
\Psi ; \Theta ; \Delta ; \Omega ; \Gamma \vdash e_1 \infers \M \, (J,\vec{p})\, \tau_1 \gens \Phi_1,\Gamma_1\\
\Psi ; \Theta; \Delta ; \Omega ; \Gamma_1, x:\tau_1 \vdash e_2 \checks \M \, (I,\vec{q} - \vec{p})\, \tau_2 \gens \Phi_2,\Gamma_2\\
\Phi = (\vec{q} \geq \vec{p}) \wedge (I =J)  \wedge \Phi_1 \wedge \Phi_2
}{
\Psi ; \Theta ; \Delta ; \Omega ; \Gamma \vdash \texttt{bind } x = e_1 \texttt{ in } e_2 \checks \M \, (I,\vec{q})\, \tau_2 \gens \Phi, \Gamma_2 \setminus \{x : \tau_1\}
}

\inferrule[AT-Release]
{
\Psi ; \Theta ; \Delta ; \Omega ; \Gamma \vdash e_1 \infers [J | \vec{q}] \tau_1 \gens \Phi_1,\Gamma_1\\
\Psi ; \Theta ; \Delta ; \Omega ; \Gamma_1, x : \tau \vdash e_2 \checks \M \, (I,\vec{p} + \vec{q}) \, \tau_2 \gens \Phi_2, \Gamma_2
}{
\Psi ; \Theta ; \Delta ; \Omega ; \Gamma \vdash \texttt{release } x = e_1 \texttt{ in }e_2 \checks \M \, (I,\vec{p}) \, \tau_2 \gens (I = J \wedge \Phi_1 \wedge \Phi_2), \Gamma_2 \setminus \{x\}
}

\inferrule[AT-Store]
{
\Theta ; \Delta \vdash K : \N \gens \Phi_1\\
\Theta ; \Delta \vdash \vec{w} : \vec{\mathbb{R}^+} \gens \Phi_2\\
\Psi ; \Theta ; \Delta ; \Omega ; \Gamma \vdash e \checks \tau \gens \Phi_3,\Gamma'\\
\Phi =  \Phi_1 \wedge \Phi_2 \wedge\Phi_3 \wedge  (\vec{p} \leq \vec{w} \leq \vec{q}) \wedge (I = J = K)
}{
\Psi ; \Theta ; \Delta ; \Omega ; \Gamma \vdash \texttt{store}[K|\vec{w}](e) \checks \M \, \phi(I,\vec{q}) \, ([J | \vec{p}] \, \tau) \gens \Phi, \Gamma'
}



\inferrule[AT-Sub]
{
\Psi ; \Theta ; \Delta ; \Omega ; \Gamma \vdash e \infers \tau' \gens \Phi_1,\Gamma'\\
\Psi;\Theta;\Delta \vdash \tau' \subty \tau : \star \gens \Phi_2
}{
\Psi ; \Theta ; \Delta ; \Omega ; \Gamma \vdash e \checks \tau \gens \Phi_1 \wedge \Phi_2,\Gamma'
}

\inferrule[AT-Anno]
{
\Psi ; \Theta ; \Delta ; \Omega ; \Gamma \vdash e \checks \tau \gens \Phi,\Gamma'
}{
\Psi ; \Theta ; \Delta ; \Omega ; \Gamma \vdash (e : \tau) \infers \tau \gens \Phi,\Gamma'
}
\end{mathpar}
\caption{Selected Algorithmic Typing Rules}
\label{fig:bilambdaamor-selected-typing-rules}
\end{figure}

As expected, the typing judgments of \bilambdaamor change the most. For one, we pass to a bidirectional type system. As discussed in Section~\textbf{??}, this process is fairly standardized, and so the reader who has seen bidirectional type systems in the past will find no surprises in \bilambdaamor. The typing judgment is split in two, yielding a mutually recursive pair of checking and inference judgments. Secondly, typing judgment sports a constraint output in a manner identical to all of the other algorithmic judgments discussed so far. Finally, to handle the affine context $\Gamma$ in an algorithmic way, we employ the I/O method from Section~\textbf{??}, adding an output context of unused variables $\Gamma'$, which are threaded through rules in a state-passing manner.

While all of these algorithmization techniques were described in the abstract in Section~\textbf{??}, understanding how they work in the context of a type system as feature-rich as \bilambdaamor is another matter entirely. To this end, we take some time to describe the selected rules presented in Figure~\ref{fig:bilambdaamor-selected-typing-rules}.

We begin with AT-Var-1, which allows us to use variables from the affine context. When $x : \tau \in \Gamma$, the term $x$ infers the type $\tau$. Using this rule in a derivation counts as a use of $x$, and so $x$ must be removed from the output context, as it is no longer unused.

The pair of rules AT-Lam and AT-App exhibit a common pattern which is common to nearly all negative logical connectives in \bilambdaamor. For introduction form, both the conclusion and premise are checking. In the elimination form, the conclusion as well as the principal judgment (the judgment typing the term being eliminated) are inferring, while all other premises check
\footnote{
The connection between bidirectional type systems and polarization/focusing which makes this pattern so ubiquitous in the rules of \bilambdaamor is deep, beautiful, and not fully understood. A wonderful overview of work on the subject, as well as exposition about how to bidirectionalizing your own declarative type systems can be found in a paper by \citet{dunfield19:bidir-survey}.
}.
The AT-Lam rule also illustrates a small oddity of the I/O method when applied to affine types. Since variables \textit{can} be left unsued, it's possible for the $\lambda$-bound variable $x$ to end up in the output context $\Gamma'$ of the premise checking the body of the lambda. For this reason, we must explicitly remove $x$ from the context of unused variables as it falls out of scope, lest it be possible to typecheck terms like $\angles{\lambda x. (), x}$, where a bound variable escapes its scope. The rule AT-App also illustrates how the ``threading" aspect of the I/O method is easily combined with the two kinds of typing judgments. To check $e_1 \, e_2$ in context $\Gamma$, the type of $e_1$ is inferred, returning unused variables $\Gamma_1$. Then, the type of $e_2$ is checked \textit{in context $\Gamma_1$}. That judgment ``returns" $\Gamma_2$, which is then used as the output judgment for the checking conclusion.

Dually, the rules AT-TensorI and AT-TensorE are a simple instance of the bidirectional rules for a positive logical connective. The introduction form has checking premises and conclusions, just like the negatives. On the other hand, the elimination form has an inferring principle judgment, but checking conclusion-- this is because positive elims all take the form of a (many-armed) let-binding, and the type of the continuation cannot be inferred locally. Because of this let-binding style, most positive elims must remove bound variables from the output context to deal with the same scoping issue as AT-Lam.

The remaining rules for logical connectives following a similar pattern: their bidirectional behavior is predetermined by logical concerns discovered by prior work in the area. Algorithmizing the rules for nonlogical connectives, however, requires quite a bit more work and cleverness.

As a case study, consider the rule T-Bind from \dlambdaamor:

$$
\inferrule
{
\Psi ; \Theta ; \Delta ; \Omega ; \Gamma_1 \vdash e_1 : \M \, (I,\vec{p})\, \tau_1\\
\Psi ; \Theta; \Delta ; \Omega ; \Gamma_2, x:\tau_1 \vdash e_2 : \M \, (I,\vec{q})\, \tau_2\\
}{
\Psi ; \Theta ; \Delta ; \Omega ; \Gamma_1,\Gamma_2 \vdash \texttt{bind } x = e_1 \texttt{ in } e_2 : \M \, (I,\vec{p} + \vec{q})\, \tau_2
}
$$

This plainly follows the let-binding style of positive elimination forms (despite not being a logical connective), and so the same direction pattern seems like a good choice. This rule has no constraint solving premises, and so the output constraints can be conjoined together. Finally, this term has the form of a let-binding, and so we thread contexts through the premises, removing $x$ in the conclusion. These three choices lead to the following first cut at an algorithmic bind rule:

$$
\inferrule
{
\Psi ; \Theta ; \Delta ; \Omega ; \Gamma \vdash e_1 \infers \M \, (I,\vec{p})\, \tau_1 \gens \Phi_1,\Gamma_1\\
\Psi ; \Theta; \Delta ; \Omega ; \Gamma_1, x:\tau_1 \vdash e_2 : \M \, (I,\vec{q})\, \tau_2 \gens \Phi_2,\Gamma_2\\
}{
\Psi ; \Theta ; \Delta ; \Omega ; \Gamma \vdash \texttt{bind } x = e_1 \texttt{ in } e_2 : \M \, (I,\vec{p} + \vec{q})\, \tau_2 \gens \Phi_1 \wedge \Phi_2, \Gamma_2 \setminus \{x\}
}
$$

Unfortunately, this rule is insufficient for potentially subtle reasons. When we implement \bilambdaamor, the checking judgment is implemented as a function (essentially) of type \texttt{ctx -> tm -> ty -> bool}, which proceeds by a (very large) case analysis on the term and type arguments. The algorithmic bind rule corresponds to the case where the term is the constructor for bind, and the type is the cost monad. However, we hope to not match further into the type, to match the index term $\vec{p} + \vec{q}$. Indeed, the second component of the cost need not be syntactically a sum of potential vectors! To fix this, we take a slightly different approach. Instead of typing the conclusion at type $\M \, (I,\vec{p} + \vec{q})\, \tau_2$, we will instead have it check against the type $\M \, (I, \vec{q})\, \tau_2$, so long as the continuation checks against $\M \, (I,\vec{q} - \vec{p}) \, \tau_2$ (when $\vec{q} \geq \vec{p}$). Intuitively, this new rule encodes the same logic: the total amortized cost of the composite is the sum of the costs of $e_1$ and $e_2$ 

A similar situation plays out if we consider the first component of the cost pair. The rule above indicates that the first components in the three monadic types need to be \textit{identical}. Just like requiring that the second component be \textit{literally} a sum, this is far too strong a condition: we only need require that they are provably equal. This leads us to the completed AT-Bind rule, as shown in Figure~\ref{fig:bilambdaamor-selected-typing-rules}. A nearly identical game is played with the rule for the potential elimination form, AT-Release: we generalize the potentials to have syntactically but provably equal bases.

The introduction rules for monads and potentials also require some tweaking. To illustrate, we consider the declarative store rule, T-Store.
$$
\inferrule
{
\Theta ; \Delta \vdash I : \mathbb{N}\\
\Theta ; \Delta \vdash \vec{p} : \vec{\mathbb{R}^+}\\
\Psi ; \Theta ; \Delta ; \Omega ; \Gamma \vdash e : \tau\\
}{
\Psi ; \Theta ; \Delta ; \Omega ; \Gamma \vdash \texttt{store}[I|\vec{p}](e) : \M \, (I,\vec{p}) \, ([I| \vec{p}] \, \tau)
}
$$
We bidirectionalize this in a straightforward manner, by making both the premise and the conclusion checking. The constraint and context outputs are similarly trivial: they are passed from the output of the premise directly to the conclusion. We are then faced with yet another matching problem: the $I$s and $\vec{p}$s in the term and type are required to be syntactically equal. It is clear how to generalize the bases: we allow all three to be different, but provably equal. The proper formulation for the coefficient components is less clear, however. Inspiration comes from considering the ranges of sound but imprecise typings for the positions. In order for 
$\texttt{store}[K|\vec{w}](e)$ to check against $\M \, (I,\vec{q}) \, \left([J|\vec{p}]\right)$ when $I = J = K$, it ought to be allowable for $\vec{w}$ to be smaller than $\vec{q}$, and for $\vec{p}$ to be smaller than $\vec{w}$. When we ask for $\phi(I,\vec{q})$ potential, it is sound to overpay, and underdeliver. The final rule, AT-Store, allows just this.

This optimization is justified by the subtyping rules AS-Pot and AS-Monad: an alternate way of thinking of AT-Store is that it's the ``basic" AT-Store derived from the declarative version, with subtyping baked in.


The last two interesting rules to be discussed are AT-Sub and AT-Anno. These rules are not analogues of rules which were present in \dlambdaamor. Instead, they are the two bidirectional-specific rules discussed in Section~\textbf{??} which allow us to mediate between the checking and inference judgments. When a synactic form whose corresponding rule has a checking conclusion (such as a lambda) is placed in a position where its expected to infer (such as the principal position of application), an annotation must be introduced. However, in the opposite situation, a term whose rule has an inferring conclusion may always be used in a checking position, so long as the type which is inferred is more specific than the one the term is being checked against.


\subsubsection{Well-formedness and Presuppositions}

The judgments of \bilambdaamor presented thusfar have all been \textit{raw} judgments, in the same sense that we have presented no well-formedness restrictions. Just like in \dlambdaamor, we restrict the positions of each relation by well-formedness presuppositions. Again, these are denoted with a subscript $p$ on the turnstile. Unlike, \dlambdaamor, these presuppositions are algorithmic in the sense that they use the corresponding judgments from \bilambdaamor to impose restrictions. 
%\red{Do I just want to write out the presupps here?}

\begin{definition}
We say that $\Theta ; \Delta \pvdash I : S$ when $\Theta \vdash \Delta \; \texttt{wf}$ and $\Theta ; \Delta \vdash I : S$.
\end{definition}

\begin{definition}
We write $\Psi ; \Theta ; \Delta \pvdash \tau : K$ to mean that $\Theta \vdash \Delta \; \texttt{wf}$ and $\Psi ; \Theta ; \Delta \vdash \tau : K$
\end{definition}

\begin{definition}
We write $\Psi ; \Theta ; \Delta \pvdash \tau \subty \tau' : K$ to mean that
\begin{enumerate}
  \item $\Theta \vdash \Delta \; \texttt{wf}$
  \item $\Psi ; \Theta ; \Delta \vdash \tau : K$
  \item $\Psi ; \Theta ; \Delta \vdash \tau' : K$
  \item $\Psi ; \Theta ; \Delta \vdash \tau \subty \tau' : K$
\end{enumerate}
\end{definition}

\begin{definition}
We say that $\Psi ; \Theta ; \Delta ; \Omega ; \Gamma \pvdash e : \tau$ when
\begin{enumerate}
  \item $\Theta \vdash \Delta \; \texttt{wf}$
  \item $\Psi ; \Theta ; \Delta \vdash \Omega \; \texttt{wf}$
  \item $\Psi ; \Theta ; \Delta \vdash \Gamma \; \texttt{wf}$
  \item $\Psi ; \Theta ; \Delta \vdash \tau : \star$
  \item $\Psi ; \Theta ; \Delta ; \Omega \vdash e : \tau$
\end{enumerate}
\end{definition}

%Note the lack of structural rules

\section{Soundness and Completeness of \bilambdaamor with respect to \dlambdaamor}
\label{sec:metatheory}
With the algorithmic system of \bilambdaamor in place, the time has come to prove theorems about it. Ideally, we would like to prove that it behaves exactly the same as \dlambdaamor. That way, when we build the implementation of \bilambdaamor in Section~\textbf{??}, we will know that (a) every program typechecked by our implementation declarative has the proper type in \dlambdaamor, and that (b) every well-typed program in \dlambdaamor \textit{can be} checked by our implementation. In this context, these two properties are known as soundness and completeness\footnote{
This may seem backwards to the reader already familiar with the terms-- we think of the declarative system as giving a ``ground truth" semantics of which terms have which types, and the algorithmic system as a proof system in which one may manually derive proofs of well-typedness. From this perspective, soundness and completeness are as described above.
}, respectively.

As is usually the case with such things, the soundness proofs are very straightforward. This is because \bilambdaamor is far more strict and structured than \dlambdaamor, so it is always fairly easy to lift a \bilambdaamor derivation with its strictures to a proof in \dlambdaamor. This mismatch simultaneously makes completeness quite difficult to prove: compiling a proof of one of the judgments of \dlambdaamor down to a structured one in \bilambdaamor requires some work in general. For this reason, we will begin with proving the soundness theorems, and subsequently move to proving completeness.

The general shape of the soundness theorems are all the same: for every algorithmic judgment $\mathcal{J} \gens \Phi$ (and corresponding declarative judgment $\mathcal{J}$) we prove that if there is a derivation of $\mathcal{J} \gens \Phi$ and $\Phi$ is valid, then there is a derivation of $\mathcal{J}$. Individual theorems may vary-- the inclusion of bidirectionality and the I/O method complicates the statement of soundness for typing-- but this is the main flavor. This pattern justifies the intended use of the algorithmic system: we derive algorithmic judgments using the implementation, which outputs constraints. If the constraints are solvable by a solver, the corresponding declarative judgment holds.

Dually, the completeness theorems have the ``opposite" shape: if $\mathcal{J}$ holds, then there is some solvable $\Phi$ such that the corresponding algorithmic judgment $\mathcal{J} \gens \Phi$ is derivable. Of course, the same caveats apply for judgment forms with more bells and whistles. This theorem structure justifies that our intended usage of the implementation covers all possible uses of the declarative system\footnote{Note that this justification requires that the constraint output for all algorithmic judgments are deterministic, and thus the existence is unique. Our system of course validates this assumption, but this implicit requirement is important to understand.}: combined with soundness, it tells us that the implementation always succeeds to derive a proof of a declarative judgment, if one exists.

\subsubsection{Soundness of Index Terms, Constraints, Contexts, and Types}
The four most basic algorithmic judgments of \bilambdaamor mirror their declarative counterparts rule-for-rule: the only ``real" modification is the addition of constraint output. This uniformity means that the soundness proofs are fairly trivial single-pass inductions on derivations. Each of these proofs comes in two parts. First, we prove that the soundness holds as a a statement about ``raw" judgments by omitting the presuppositions. These theorems are garbage in, garbage out: malformed judgments in \bilambdaamor are sent to malformed judgments in \dlambdaamor. Afterwards, we prove that the presuppositions are preserved, and so well-formed judgments in \bilambdaamor are sent to well-formed judgments in \dlambdaamor. This two-step process is only necessary because the presuppositions have a mutually inductive structure: to untangle the knot, we must first prove the raw statements, and then repackage them with the required presuppositions afterwards.

\red{(For the love of god, don't forget to put the raw proofs in the appendix)}

Below, we will only present the versions of the theorems with presuppositions included: the gory details can be found in Appendix~\textbf{??}. All of the proofs proceed by elementary inductions on derivations, occasionally using easy properties about constraint validity.

\begin{theorem}[Soundness of Index Context Well-Formedness]
If $\Theta \vdash \Delta \; \texttt{wf} \gens \Phi$ and $\Theta ; \cdot \vDash \Phi$, then $\Theta \vdash \Delta \; \texttt{wf}$
\label{thm:idx-ctx-wf-sound}
\end{theorem}

\begin{theorem}[Soundness of Sort Checking]
If $\Theta;\Delta \pvdash I : S \gens \Phi$ and $\Theta;\Delta \vDash \Phi$, then $\Theta;\Delta \pvdash I : S$ 
\label{thm:sort-sound}
\end{theorem}

\begin{theorem}[Soundness of Constraint Well-Formedness]
If $\Theta ; \Delta \pvdash \Phi \texttt{ wf} \gens \Phi'$ and $\Theta ; \Delta \vDash \Phi'$ then $\Theta ; \Delta \pvdash \Phi \texttt{ wf}$
\label{thm:constr-sound}
\end{theorem}

\begin{theorem}[Soundness of Kind Checking]
If $\Psi ; \Theta ; \Delta \pvdash \tau : K \gens \Phi$ and $\Theta ; \Delta \vDash \Phi$ then $\Psi ; \Theta ; \Delta \pvdash \tau : K$.
\label{thm:kind-sound}
\end{theorem}

\subsubsection{Soundness of Subtyping}
Subtyping provides a significantly more interesting soundness proof than the prior cases: we must justify that \bilambdaamor's two-step normalize-then-compare strategy is in fact sound for the declarative type system. The proof proceeds in two parts corresponding to the two judgments- we first prove soundness for the normal form subtyping, and then lift it, using the results about normalization from Section~\textbf{??}, to the full algorithmic subtyping relation.

\begin{theorem}[Soundness of Subtyping for Normal Forms]
If $\Psi ; \Theta ; \Delta \pvdash \tau_1 \subtynf \tau_2 : K \gens \Phi$ and $\Theta ; \Delta \vDash \Phi$ then $\Psi ; \Theta ; \Delta \pvdash \tau_1 \subty \tau_2 : K$
\label{thm:subtynf-sound}
\end{theorem}

\begin{theorem}[Soundness of Subtyping]
If $\Psi ; \Theta ; \Delta \pvdash \tau_1 \subty\tau_2 : K \gens \Phi$ and $\Theta ; \Delta \vDash \Phi$, then $\Psi ; \Theta ; \Delta \pvdash \tau_1 \subty \tau_2 : K$
\label{thm:subty-sound}
\end{theorem}
\begin{proof}
There is only one case: $\Psi ; \Theta ; \Delta \pvdash \tau_1 \subty\tau_2 : K \gens \Phi$ by way of $\Psi ; \Theta ; \Delta \vdash \texttt{eval}(\tau_1) \subtynf \texttt{eval}(\tau_2) : K \gens \Phi$ with $\Theta ; \Delta \vDash \Phi$. By Theorem~\ref{thm:subtynf-sound}, $\Psi ; \Theta ; \Delta \vdash \texttt{eval}(\tau_1) \subty \texttt{eval}(\tau_2) : K$. By Theorem~\ref{thm:norm-thm} and two uses of S-Trans, $\Psi ; \Theta ; \Delta \pvdash \tau_1 \subty \tau_2 : K$, as required.
\end{proof}

\subsubsection{Soundness of Typechecking}

As is to be expected, the soundness of the bidirectional type-checking judgment is the most involved. Before we can prove it, a few small lemmata are required.
First, we prove (as was noted before) that the output contexts of the I/O method in both typing judgments have a strong regularity condition: the output context is always a subset\footnote{
This containment is almost always strict: in fact, a corollary of Theorem~\ref{thm:tycheck-sound}, is that the containment is strict unless the term is closed.
}
of the input context. \red{(Should I even be talking about this? Or should I skip to the main theorem)}

\begin{theorem}
If $\Psi ; \Theta ; \Delta ; \Omega ; \Gamma \vdash e \updownarrow \tau \gens \Phi, \Gamma'$ then $\Gamma' \subseteq \Gamma$.
\label{thm:lsc}
\end{theorem}

The next lemma is required for cases of Theorem~\ref{thm:tycheck-sound} where variables are bound in premises and subsequently removed in the conclusion. In essence, it proves compatibility between the set difference operator which removes variables from the output, and context extension.

\begin{theorem}
If $\Gamma' \subseteq \Gamma$, then
$(\Gamma, x : \tau) \setminus \Gamma' \subseteq \Gamma \setminus (\Gamma' \setminus \{x : \tau\}), x : \tau$. Moreover, if:
~\begin{itemize}
  \item $\Theta \vdash \Delta \; \texttt{wf}$
  \item $\Psi ; \Theta ; \Delta \vdash \Gamma \; \texttt{wf}$
  \item $\Psi ; \Theta ; \Delta \vdash \tau : \star$
\end{itemize}
Then $\Psi ; \Theta ; \Delta \pvdash \Gamma \setminus (\Gamma' \setminus \{x : \tau\}), x : \tau \wknto (\Gamma, x : \tau) \setminus \Gamma'$.
\label{thm:tycheck-sound-lemma}
\end{theorem}
\begin{proof}
The first part follows by an elementary set-theoretic containment proof, and the second is immediate by applying the presuppositions.
\end{proof}

Because the two judgments (checking and inference) are mutually inductively defined, we must prove each judgment's corresponding soundness theorem simultaneously. The theorem must also handle two as-of-yet untreated differences of \bilambdaamor and \dlambdaamor. First, the syntax of algorithmic terms is different from the declarative ones. To translate an algorithmic term to a declarative one, we rely on the erasure transformation from Section~\ref{sec:bilambdaamor-overview-bidir} to remove all type annotations from a term. Second, the use of the I/O method means we must incorporate the ``context-strengthening"-style completeness theorem from Section~\ref{sec:bilambdaamor-overview-io}. All together, we arrive at the following theorem.

\begin{theorem}[Soundness of Type Checking/Inference]
~\begin{enumerate}
 \item If $\Psi;\Theta;\Delta;\Omega;\Gamma \pvdash e \checks \tau \gens \Phi, \Gamma'$ and $\Theta;\Delta \vDash \Phi$ then $\Psi;\Theta;\Delta;\Omega;\Gamma \setminus \Gamma' \pvdash |e| : \tau$
 \item If $\Psi;\Theta;\Delta;\Omega;\Gamma \pvdash e \infers \tau \gens \Phi, \Gamma'$ and $\Theta;\Delta \vDash \Phi$ then $\Psi;\Theta;\Delta;\Omega;\Gamma \setminus \Gamma' \pvdash |e| : \tau$
\end{enumerate}
\label{thm:tycheck-sound}
\end{theorem}

The proof of Theorem~\ref{thm:tycheck-sound} is not particularly enlightening: we prove both claims simultaneously by induction on the judgment premises, liberally applying Theorem~\ref{thm:tycheck-sound-lemma} when binders are used.


\subsubsection{Completeness of Sorts, Constraints, Contexts, and Kinds}
Perhaps expectedly, the four basic judgments admit very simple completeness proofs. Similarly to their soundness proofs, these are all proved by single-pass inductions on derivations. Again, these proofs are split into two parts to untie the knot: we first prove completeness of ``raw" judgments, and then repackage the theorems with presuppositions after all of the raw theorems have been proven.


\begin{theorem}
If $\Theta;\Delta \pvdash I : S$, then $\Theta;\Delta \pvdash I : S \gens \Phi$ and $\Theta;\Delta \vDash \Phi$.
\label{thm:sort-compl}
\end{theorem}

\begin{theorem}
If $\Theta \vdash \Delta \; \texttt{wf}$ then $\Theta \vdash \Delta \; \texttt{wf} \gens \Phi$ with $\Theta ; \cdot \vDash \Phi$
\label{thm:idx-ctx-wf-compl}
\end{theorem}

\begin{theorem}
If $\Theta ; \Delta \pvdash \Phi \; \texttt{wf}$, then $\Theta ; \Delta \pvdash \Phi \; \texttt{wf} \gens \Phi'$ with $\Theta ; \Delta \vDash \Phi'$
\label{thm:constr-compl}
\end{theorem}

\begin{theorem}
If $\Phi;\Theta;\Delta \pvdash \tau : K$, then $\Phi;\Theta;\Delta \pvdash \tau : K \gens \Phi$ with $\Theta ; \Delta \vDash \Phi$
\label{thm:kind-compl}
\end{theorem}

\subsubsection{Completeness of Subtyping}
The proof that \bilambdaamor's subtyping is complete is perhaps the most exciting proof we will see. As has been discussed numerous times, \dlambdaamor's inclusion of index term-indexed types means that proving the algorithmic subtyping complete is tantamount to deciding $\beta$-equivalence of a limited lambda calculus. It may be tempting\footnote{
I certainly was tempted.
} to attempt to split the proof of completeness of subtyping into two statements: one could attempt to first prove that the algorithmic normal form subtyping relation is complete for types in normal form: i.e. that if $\tau_1 \subty \tau_2$ in \dlambdaamor with $\tau_1,\tau_2 \, \texttt{nf}$, then there is some solvable $\Phi$ such that $\tau_1 \subtynf \tau_2 \gens \Phi$. Unfortunately, this is not easily provable: if the premise is a use of transitivity, the cut type may not be in normal form, and thus the inductive hypothesis cannot be applied.

The actual proof of completeness of algorithmic subtyping relies on two key admissibility theorems, namely of reflexivity and transitivity. Since \dlambdaamor's subtyping includes the rules S-Refl and S-Trans but \bilambdaamor's doesn't include analogues of these (for the purposes of syntax-directedness), the algorithmic subtyping must be able to emulate these rules whenever they occur in a declarative derivation. In both cases, the proof proceeds by proving the statement for normal forms, and then lifting the result to the full algorithmic relation through normalization.

\begin{theorem}[Reflexivity of Algorithmic Subtyping for Neutral Forms]
If $\Psi ; \Theta ;  \Delta \pvdash \tau : K$ and $\tau \;\texttt{ne}$, then $\Psi ; \Theta ;  \Delta \pvdash \tau\subtynf \tau : K \gens \Phi$ with $\Theta ; \Delta \vDash \Phi$
\label{thm:subtyne-refl}
\end{theorem}

\begin{theorem}[Reflexivity of Algorithmic Subtyping for Normal Forms]
If $\Psi ; \Theta ;  \Delta \pvdash \tau : K$ and $\tau \;\texttt{nf}$, then $\Psi ; \Theta ;  \Delta \pvdash \tau\subtynf \tau : K \gens \Phi$ with $\Theta ; \Delta \vDash \Phi$
\label{thm:subtynf-refl}
\end{theorem}

\begin{theorem}[Reflexivity of Algorithmic Subtyping]
If $\Psi ; \Theta ;  \Delta \pvdash \tau : K$ then $\Psi ; \Theta ;  \Delta \pvdash \tau\subty \tau : K \gens \Phi$ with $\Theta ; \Delta \vDash \Phi$
\label{thm:subty-refl}
\end{theorem}
\red{(Do I want this in the new style?)}
\begin{proof}
By AS-Normalize, it suffices to show that $\Psi ; \Theta ;  \Delta \vdash \texttt{eval}(\tau) \subtynf \texttt{eval}(\tau) : K \gens \Phi$. By Theorem~\ref{thm:norm-thm}, we have that $\texttt{eval}(\tau) \; \texttt{nf}$, and by S-Refl, $\Psi ; \Theta ; \Delta \vdash \texttt{eval}(\tau) \subty \texttt{eval}(\tau) : K$, By Theorem~\ref{thm:subtynf-refl}, we have $\Psi ; \Theta ;  \Delta \vdash \texttt{eval}(\tau) \subtynf \texttt{eval}(\tau) : K \gens \Phi$ and $\Theta ; \Delta \vDash \Phi$, as required. 
\end{proof}

\begin{theorem}[Transitivity of Algorithmic Subtyping for Normal Forms]
If $\Psi ; \Theta ; \Delta \pvdash \tau_1 \subtynf \tau_2 : K \gens \Phi_1$ and $\Psi ; \Theta ; \Delta \pvdash \tau_2 \subtynf \tau_3 : K \gens \Phi_2$ with $\Theta ; \Delta \vDash \Phi_1 \wedge \Phi_2$, then $\Psi ; \Theta ; \Delta \vdash \tau_1 \subtynf \tau_3 : K \gens \Phi$ such that $\Theta ; \Delta \vDash \Phi$.
\label{thm:subtynf-trans}
\end{theorem}

\begin{theorem}[Transitivity of Algorithmic Subtyping]
\label{thm:subty-trans}
If $\Psi ; \Theta ; \Delta \pvdash \tau_1 \subty \tau_2 : K \gens \Phi_1$ and $\Psi ; \Theta ; \Delta \pvdash \tau_2 \subty \tau_3 : K \gens \Phi_2$ with $\Theta ; \Delta \vDash \Phi_1 \wedge \Phi_2$, then $\Psi ; \Theta ; \Delta \pvdash \tau_1 \subty \tau_3 : K \gens \Phi$ and $\Theta ; \Delta \vDash \Phi$
\end{theorem}
%\textbf{Rewrite in new style}
\begin{proof}
By inversion, $\Psi ; \Theta ; \Delta \vdash \texttt{eval}(\tau_1) \subtynf \texttt{eval}(\tau_2) : K \gens \Phi_1$ and $\Psi ; \Theta ; \Delta \vdash \texttt{eval}(\tau_2) \subtynf \texttt{eval}(\tau_3) : K \gens \Phi_1$ By Theorem~\ref{thm:norm-thm} and Theorem~\ref{thm:subtynf-trans}, 
$\Psi ; \Theta ; \Delta \vdash \texttt{eval}(\tau_1) \subtynf \texttt{eval}(\tau_3) : K \gens \Phi$ with $\Theta ; \Delta \vDash \Phi$. Then, by AS-Normalize,
$\Psi ; \Theta ; \Delta \vdash \tau_1 \subty \tau_3 : K \gens \Phi$, as required.
\end{proof}

The following theorem is essentially a subtyping version of Theorem~\textbf{??} \red{(cut instances of normal forms)}: not only does index term substitution preserve the property that types are in normal form, it also preserves all subtyping relations. This theorem depends on a series of theorems that index-term substitution is admissible for all of the preceeding judgments. These are all proved (in Appendix~\textbf{??}) by appealing to the corresponding substitution theorem in \dlambdaamor, and taking a round trip through soundness and completeness for the judgment in question.

\begin{theorem}[Admissibility of Normal Form Subtyping Substitution]
\label{thm:subtynf-idx-subst}
~Suppose the following:
 \begin{itemize}
   \item $\Psi ; \Theta, i : S ; \Delta \pvdash \tau_1 \subtynf \tau_2 : K \gens \Phi$ with $\Theta ; \Delta \vDash \Phi$ and $\Theta \vdash \Delta \; \texttt{wf}$.
   \item $\Theta ; \Delta \pvdash I : S \gens \Phi_1$ with $\Theta ; \Delta \vDash \Phi_1$
   \item $\Theta ; \Delta \pvdash J : S \gens \Phi_2$ with $\Theta ; \Delta \vDash \Phi_2$ 
   \item $\Theta ; \Delta \vDash I = J$
 \end{itemize}
 Then, $\Psi ; \Theta ; \Delta \pvdash \tau_1[I/i] \subtynf \tau_2[J/i] : K \gens \Phi'$ for some $\Phi'$ with $\Theta ; \Delta \vDash \Phi'$.
\end{theorem}

A corollary for the above admissibility theorem is that type evaluation essentially commutes with type family application. This theorem is pivotal for proving the AS-FamApp case of Theorem~\ref{thm:subty-compl} below.

\begin{theorem}[Type Family Application Commutes with Evaluation]
If $\Psi ; \Theta ; \Delta \pvdash \texttt{eval}(\tau_1) \subtynf \texttt{eval}(\tau_2) : S \to K \gens \Phi$ and $\Theta ; \Delta \vDash \Phi \wedge I = J$ with $\Theta ; \Delta \pvdash I : S$ and $\Theta ; \Delta \pvdash J : S$ then 
$\Psi ; \Theta ; \Delta \pvdash \texttt{eval}(\tau_1 \; I) \subtynf \texttt{eval}(\tau_2 \; J) : K \gens \Phi'$ for some $\Theta ; \Delta \vDash \Phi'$.
\label{thm:eval-app-lemma}
\end{theorem}
\begin{proof}
By inversion on $\Psi ; \Theta ; \Delta \vdash \texttt{eval}(\tau_1) \subtynf \texttt{eval}(\tau_2) : S \to K \gens \Phi$.
\begin{itemize}
  \item For the first case, suppose the derivation was $\Psi ; \Theta ; \Delta \vdash \lambda i : S. \tau_1' \subtynf \tau_2' : S \to K \gens \Phi$
  from $\Psi ; \Theta, i : S ; \Delta \vdash \tau_1' \subtynf \tau_2' : K \gens \Phi'$. By Theorem~\ref{thm:subtynf-idx-subst},
  $\Psi ; \Theta ; \Delta \pvdash \tau_1'[I/i] \subtynf \tau_2'[J/i] : K \gens \Phi'$, for some $\Theta ; \Delta \vDash \Phi'$. But $\texttt{eval}(\tau_1 \; I) = \tau_1'[I/i]$ and $\texttt{eval}(\tau_2 \; J) = \tau_2'[J/i]$.
  \item Now, suppose the derivation was $\Psi ; \Theta ; \Delta \vdash \tau_1' \; L_1 \subtynf \tau_2' \; L_2 : S \to K \gens \Phi \wedge (L_1 = L_2)$, where $\texttt{eval}(\tau_1) = \tau_1' \; L_1$ and $\texttt{eval}(\tau_2) = \tau_2 \; L_2$. These must both be $\texttt{ne}$, since they are both applications, and therefore
  $\texttt{eval}(\tau_1) \; I = \texttt{eval}(\tau_1 \; I)$ and $\texttt{eval}(\tau_2) \; J = \texttt{eval}(\tau_2 \; J)$, as required.
\end{itemize}
\end{proof}

Finally, we prove the full completeness of algorithmic subtyping. The proof proceeds by a single induction on the hypothesis. The reflexivity and transitivity cases are handled by Theorems \ref{thm:subty-refl} and \ref{thm:subty-trans}. We present the most interesting case (for S-Fam-Beta1) below, and leave the remaining cases for Appendix~\textbf{??}.

\begin{theorem}[Completeness of Algorithmic Subtyping]
If $\Psi ; \Theta ; \Delta \pvdash \tau_1 \subty \tau_2 : K$ then  $\Psi ; \Theta ; \Delta \pvdash \tau_1 \subty \tau_2 : K \gens \Phi$ and $\Theta ; \Delta \vDash \Phi$.
\label{thm:subty-compl}
\end{theorem}
\begin{proof}
~\begin{itemize}
   \item[(S-Fam-Beta1)] Suppose $\Psi ; \Theta ; \Delta \vdash (\lambda i : S. \tau) \; J \subty \tau[J/i] : K$. By AS-Normalize, it suffices to show that
   $\Psi ; \Theta ; \Delta \vdash \texttt{eval}((\lambda i : S. \tau) \; J) \subtynf \texttt{eval}(\tau[J/i]) : K \gens \Phi$ and $\Theta ; \Delta \vDash \Phi$.
   But, $\texttt{eval}((\lambda i : S. \tau) \; J) = \texttt{eval}(\tau)[J/i]$ by definition, and $\texttt{eval}(\tau[J/i]) = \texttt{eval}(\tau)[J/i]$ by Theorem~\ref{thm:idx-subst-eval}. By Theorem~\ref{thm:norm-thm},  $\Psi ; \Theta ; \Delta \vdash \texttt{eval}(\tau)[J/i] : K$, and so by Theorem~\ref{thm:subtynf-refl}, we have that $\Psi ; \Theta ; \Delta \vdash \texttt{eval}(\tau)[J/i] \subtynf \texttt{eval}(\tau)[J/i] : K \gens \Phi$ with $\Theta ; \Delta \vDash \Phi$ as required.
 \end{itemize}
\end{proof}

\subsubsection{Completeness of Typechecking}

Finally, we arrive at the completeness of \bilambdaamor's typechecking algorithm. To begin, we must prove the admissibility of \dlambdaamor's weakening rule (T-Weaken) in \bilambdaamor. This requires a fairly sizable and involved simultaneous induction on the checking and inference judgments, which must account for all of the bells and whistles of the bidirectional typechecking with constraints and I/O contexts. The theorem is best understood as a \bilambdaamor -specific version of the mock I/O weakening theorem from Section~\ref{sec:bilambdaamor-overview-io}. When we weaken the affine context, the added variables flow through, and remain unused.

\red{(How the hell do I make this readable??)}

\begin{theorem}[Admissibility of Algorithmic Weakening]
~\begin{enumerate}
  \item If $\Psi ; \Theta ; \Delta ; \Omega ; \Gamma \pvdash e \checks \tau \gens \Phi,\Gamma''$ with $\Theta ; \Delta \vDash \Phi$, then whenever $\Psi ; \Theta ; \Delta \pvdash \Gamma' \wknto \Gamma$ and $\Psi ; \Theta ; \Delta \pvdash \Omega' \wknto \Omega$, there are $\Phi_1$, $e_1$, $\Gamma_1$ so that $|e_1| = |e|$, $\Theta ; \Delta \vDash \Phi_1$, $\Psi ; \Theta ; \Delta \pvdash \Gamma_1 \wknto \Gamma' \setminus \Gamma$, $\Psi ; \Theta ; \Delta \pvdash \Gamma_1 \wknto \Gamma''$, and $\Psi ; \Theta ; \Delta ; \Omega' ; \Gamma' \pvdash e_1 \checks \tau \gens \Phi_1,\Gamma_1$.
  \item If $\Psi ; \Theta ; \Delta ; \Omega ; \Gamma \pvdash e \infers \tau \gens \Phi,\Gamma''$ with $\Theta ; \Delta \vDash \Phi$, then whenever $\Psi ; \Theta ; \Delta \pvdash \Gamma' \wknto \Gamma$ and $\Psi ; \Theta ; \Delta \pvdash \Omega' \wknto \Omega$, there are $\Phi_2$, $e_2$, $\Gamma_2$ so that $|e_2| = |e|$, $\Theta ; \Delta \vDash \Phi_2$, $\Psi ; \Theta ; \Delta \pvdash \Gamma_2 \wknto \Gamma' \setminus \Gamma$, $\Psi ; \Theta ; \Delta \pvdash \Gamma_2 \wknto \Gamma''$ and $\Psi ; \Theta ; \Delta ; \Omega' ; \Gamma' \pvdash e_2 \infers \tau \gens \Phi_2,\Gamma_2$.
\end{enumerate}
\label{thm:admits-weaken}
\end{theorem}

The actual statement of completeness is easily understandable. For any declarative type assignment, we can always annotate the term with types so that it can either check or infer, while outputting a valid constraint. We note that it is not strictly necessary to prove the theorem in this form-- the careful reader may have noticed that (2) is implied by (1) and a single use of AT-Anno. This method is in fact the traditional way of proving bidirectional completeness. However, the algorithm its proof encodes inserts many unnecessary annotations: any term in inference position will be annotated, even if the term is already a syntactic form whose rule has inferring conclusion. However, by providing an inductive hypothesis for an inference judgment at every stage, we give terms which may infer their own types the opportunity to do so. For this reason, the algorithm which the proof of completeness encodes inserts far fewer annotations than the traditional one.

\begin{theorem}[Completeness of Type Checking/Inference]
If $\Psi;\Theta;\Delta;\Omega;\Gamma \pvdash e : \tau$, then:
\begin{enumerate}
  \item There are $e'$, $\Phi'$, $\Gamma'$ such that $|e'| = e$, $\Theta ; \Delta \vDash \Phi'$, and $\Psi;\Theta;\Delta;\Omega;\Gamma \pvdash e' \checks \tau \gens \Phi', \Gamma'$.
  \item There are $e''$, $\Phi''$, $\Gamma''$ such that $|e''| = e$, $\Theta ; \Delta \vDash \Phi''$, and $\Psi;\Theta;\Delta;\Omega;\Gamma \pvdash e'' \infers \tau \gens \Phi'',\Gamma''$
\end{enumerate}
\label{thm:tycheck-compl}
\end{theorem}


\section{Implementation of \lambdaamor}
\label{sec:lambdaamor-impl}
To exhibit the practical use of \dlambdaamor, we present an implementation of \bilambdaamor, which we will simply refer to as \lambdaamorimpl. The implementation consists of approximately \textbf{??} lines of OCaml, and is freely available at the URL below, under a \textbf{??} license.
$$
\text{URL}
$$
Functionally, the implementation sports a typechecker and interpreter for \dlambdaamor, as well as a command-line interface with a REPL \red{rm this if you haven't done it...} for interactive use.

We begin the section by discussing the format of programs in \lambdaamorimpl. In order to use \lambdaamorimpl as a programming language, we require some language features other than the type-checking and evaluation of single expressions, as modeled by \bilambdaamor. To this end, we introduce four top-level declaration forms which can be composed to form programs in \lambdaamorimpl. Next, we discuss the artifact itself in some depth: giving an overview of the project's structure, elaborating on design decisions, and remarking on a few places where the implementation departs from the theory.
 Next, we show \lambdaamorimpl in action, by typechecking and evaluating the examples from Section~\textbf{??}, as well as exhibiting some of the different modes of use and interaction we envision for languages like this one. We also exhibit some of the ergonomic features available in \lambdaamorimpl which are not present in the core theory. Finally, we present an experimental evaluation of the tool, and compare it with existing resource-aware languages.
 

\subsection{Declarations and Structure of \lambdaamorimpl Programs}
\begin{figure}
$$
\begin{array}{|l|l|}
\hline
\text{Let Declaration} & \texttt{let x:t = e}\\
\text{Do Declaration} & \texttt{do x:t <- e}\\
\text{Type Alias Declaration} & \texttt{type a:k = t}\\
\text{Index Alias Declaration} & \texttt{idx i:s = it}\\
\hline
\end{array}
$$
\caption{Declarations in \lambdaamorimpl}
\label{fig:lambdaamorimmpl-decls}
\end{figure}

Programs in \lambdaamorimpl are lists of \textit{declarations}, which can be any of four forms: let-bound definitions, type and index term aliases, and \texttt{do}-declarations. The syntax for each can be seen in Figure~\ref{fig:lambdaamorimmpl-decls}. These four declaration types are allow programmers to ergonomically write interesting programs by composing them from smaller ones, and building up abstractions. All four declarations should be understood as simply exposing existing judgmental structure from \dlambdaamor to the programmer, and as such they do not add or subtract from the expressive power of the language.

Index term and type aliases are somewhat self-explanatory: a programmer may give names to types and index terms they wish to use later. Since types in \lambdaamorimpl can get quite complex, liberal use of type aliases is often very helpful. Importantly, both index term and type aliases may be of higher sort and kind, respectively, and so a user can give names to type families and index functions, too.

\texttt{do}-declarations are reminiscent of top-level interaction in Haskell's GHCI interpreter. In GHCI, computations in the IO monad entered at top level are not only evaluated, but forced for effect. The \texttt{do} of \lambdaamorimpl serves a similar role by allowing monadic computations which have previously been built up to be run, and have their \textit{actual} run-time costs computed. The reader may find it helpful to think of \texttt{do}-declarations as being a \texttt{bind} into an ambient cost-monadic context, again in a manner similar to GHCI.

Let-bindings in \lambdaamorimpl behave just like top-level bindings in any other functional language. The only twist is that all top-level let-bound variables are required to be exponential variables, and hence cannot use any affine variables bound by \texttt{do}-declarations.


\subsection{Overview of Phases}
\begin{figure}
$$
\begin{array}{|l|l|}
\hline
\text{File} & \text{Description}\\
\hline
\texttt{src/constr_elab.ml} & \text{Constraint elaboration phase to eliminate } \potvec\\
\texttt{src/ctx.ml} & \text{Typing context module}\\
\texttt{src/env.ml} & \text{Typing environment module}\\
\texttt{src/fresh_var.ml} & \text{Globably-unique variable generation}\\
\texttt{src/freshen.ml} & \text{AST freshening pass}\\
\texttt{src/interp.ml} & \text{Definitional interpreter}\\
\texttt{src/lexer.mll} & \text{OCamlLex lexer specification file}\\
\texttt{src/main.ml} & \text{Entry point and command line interface}\\
\texttt{src/normalize.ml} & \text{Type normalization routine}\\
\texttt{src/parser.mly} & \text{OCamlYacc parser specification file}\\
\texttt{src/support.ml} & \text{Debug info}\\
\texttt{src/syntax.ml} & \text{AST datatypes and helpers}\\
\texttt{src/tycheck.ml} & \text{Main bidirectional typechecking algorithms}\\
\texttt{src/tyerror.ml} & \text{Error handling monad for typechecking}\\
\texttt{src/why_backend.ml} & \text{Code to interface with Why3}\\
\texttt{src/why_trans.ml} & \text{Translate constraints into Why3 goals}\\
\texttt{la.why} & \text{Why3 supplementary theory file}\\
\hline
\end{array}
$$
\caption{File Structure of \lambdaamorimpl}
\label{fig:lambdaamorimpl-file-structure}
\end{figure}
A program to be run by \lambdaamorimpl follows a straightforward path. It is first lexed and parsed from its textual form into an abstract syntax representation. This abstract syntax is then passed to the typechecker, which closely follows the algorithmic approach prescribed by \bilambdaamor. This typechecking emits constraints, which are then passed to an SMT solver using the Why3 prover frontend \citehere. If the constraints come back valid, the program can then be passed to the built-in definitional interpreter \citehere, which runs the program according to the cost semantics of \citehere.

For the remainder of this section, we will take a tour through the implementation and design choices of each of the phases of the language's execution. Each of these roughly corresponds to a single module in the \lambdaamorimpl source, so a table of files along with their descriptions can be found in Figure~\ref{fig:lambdaamorimpl-file-structure}. Finally, some of the structure of the implementation is borrowed and inspired from previous developments in resource-aware and bidirectional type systems, and so we are careful to flag our predecessors for each pass.


\subsubsection{Lexing and Parsing}
\lambdaamorimpl uses off-the-shelf OCaml lexer and parser generators, \texttt{ocamllex} and \texttt{ocamlyacc} \citehere. While not the most performant options, these do fine for our purposes. The syntax of \lambdaamorimpl was carefully chosen to resemble the syntax of \bilambdaamor as much as possible while retaining an unambiguous grammar, and while ensuring that users need not type unicode symbols.

While the language syntax is closely modeled on that of \bilambdaamor, it must be extended to support the top-level features introduced for ease of use in \lambdaamorimpl. The only change to the term syntax is the introduction of a syntax (wildcards/underscores) for typed holes in \lambdaamorimpl, in the style of Haskell \citehere, or Hazel \citehere, which allow a programmer to typecheck partial programs, and be informed about what types the checker expects to fill the holes.

The lexer and parser are specified in \texttt{src/lexer.mll}, respectively \texttt{src/parser.mly}. The parser emits an abstract syntax representation of a program-- the type of these syntax trees, as well as all of the abstract syntax of the language, is found in \texttt{src/syntax.ml}.


\subsubsection{AST Freshening Pass}
Since this development includes no mechanized metatheory, all variables in \lambdaamorimpl are represented as strings for simplicity. This, of course, poses complications for the substitution operation. In order to resolve this once and for all, the AST of a program is fed to a ``freshener" immediately after parsing, which $\alpha$-converts all terms so that every bound variable is globally unique. This pass is modeled off of a similar one from the Granule language \citehere, a language with a sophisticated modal type system for quantatative static program analysis.

\subsubsection{Normalization}
Thanks to the theoretical simplicity of \lambdaamor's type normalization algorithm, \lambdaamorimpl's implementation of it is similarly simple: the code (in \texttt{src/normalize.ml}) is only $\approx 150$ lines. The procedure works in two passes, first by evaluating object-language types into a meta-language type of types \textit{in normal form}, and then quoting back. In practice, most types in \lambdaamor programs are in normal form. To avoid unnecessarily normalizing types, we tag type with a status bit which signals if it's already in normal form.

\subsubsection{Bidirectional Type Checking}
The core typechecking algorithm of \lambdaamorimpl is very faithful to the core algorithmic calculus presented in Section~\ref{sec:bilambdaamor}. Search functions for each of the four user-facing judgments (sort-checking, kind-checking, subtyping, typechecking) are implemented in the file \texttt{src/tycheck.ml}. The sort-checking and kind-checking judgments both operate on fully annotated terms. For this reason, we implement full inference and checking for both: with sort/kind as output and input, respectively. Subtyping is implemented as expected: both types are normalized, and then passed to a helper function which decides the normal form subtyping relation of \bilambdaamor. Finally, the main pair of type checking and inference judgments are implemented in the usual bidirectional style as a pair of mutually recursive functions. All of these functions, in addition to their usual return types (unit for checking functions, sort/kind/type for inference functions) also return the constraints output by their corresponding judgments, to be passed to the solver.

In order to simplify the lives of programmers, we do deviate slightly from the core calculus in a few places. First, the type checking and inference judgments include a few ``parallel rules": instances where the bidirectional rule has a checking or inference conclusion, but we also include a case for the other mode. While not strictly required for completeness, these extra rules can make programming more ergonomic. Next, we always normalize the output of the type inference function: this is helpful in cases where the type of a term inferred in an elimination position has a $\beta$-redex as its head, and not the expected connective. This is also clearly still sound and complete, as this behavior can be emulated by adding an annotation in the requisite elimination position. Finally, we note the behavior of typed holes in \lambdaamorimpl. This feature is a practical necessity in languages with type systems as complex as ours. Fortunately, the bidirectional framework makes them simple to implement: when the type checking or inference judgments hit the hole, checking is halted, and the expected type of the hole (in the case of a checking judgment) as well as the current context is printed for the user.

In order to simplify some of the boilerplate involved in implementing the functions corresponding to each algorithmic judgment, we introduce a monadic discipline inspired by the implementation of BiRelCost \citehere. We use a combined state/error monad called \texttt{'a checker} to simultaneously handle the four fully structural contexts and the substructural one via the I/O method (hence state, not reader), as well as managing type errors. OCaml's \texttt{let*} syntax allows us to cleanly write the typechecker in a manner similar to \texttt{do}-notation in Haskell, while a preponderance of useful monadic combinators lifted from BiRelCost make for very readable code.

Of course, the typechecker must not only handle the core term calculus, but also the top-level declaration features. Because of the inclusion of type and index term aliases, the state part of the \texttt{checker} monad must also include a type and index term environment which binds aliases to their values, on top of the existing type contexts. The top-level declarations require more choices to be made. 

Top level term declarations $\texttt{let x : t = e}$ are implicitly typed as exponential terms-- we erase the affine context before checking them, and the variable \texttt{x} given type \texttt{t} in the exponential context $\Omega$. This allows functions declared at the top level to be used many times, instead of just once, which is of course the intended pattern of use for a top-level definition.

The only terms which are bound in the affine context at top level are variables resulting from the \texttt{do} declarations, which are checked to have monadic type. Since the result of a monadic computation can store potential, the result of a \texttt{do} declaration must not be duplicated.

\subsubsection{Constraint Elaboration}
The language of \dlambdaamor's constraints includes equations and inequalities over potential vectors (index terms of sort $\potvec$). We eventually plan to send these constraints to an SMT solver, yet solvers don't have theories this fairly unusual type of variable-length vectors over a monoid. While many solvers allow us to define our own theories, it would be preferable to instead appeal to built-in (and highly optimized) real arithmetic theories of modern SMT solvers. For this purpose, all constraints output from the typechecker are elaborated to transform equalities over potential vectors to componentwise equalities over reals. This transformation is functionally identical to the index term component of the \dlambdaamor-to-\lambdaamorminus compilation pass sketched in Section~\textbf{??}, and is implemented in \texttt{src/constr_elab.ml}

\subsubsection{Constraint Solving}
The actual constraint solving of \lambdaamorimpl is handled by the Why3 platform \citehere. Why3 is a unified frontend for a number of SMT solvers, which allows the user to switch between the proovers of their choosing. After the constraint elaboration phase, the constraints are translated into a format understandable by Why3 using its OCaml API. We then interface with the prover by building a Why3 proof goal for each constraint emitted by the typechecker. This set of proof goals is then checked in sequence by the prover, and the results are reported to the user.

\subsubsection{Interpreter}
Finally, a type-correct program can be interpreted. The interpreter included with \lambdaamorimpl is a straightforward definitional implementaion of the big-step cost-indexed operational semantics of \dlambdaamor: nothing too fancy. When the interpreter is invoked, all of the \texttt{do} declarations are run. The cost semantics tallies up the total actual cost of running a single declaration, and presents it, along with the statically predictied cost to the user. By the soundness theorem of \dlambdaamor, the predicted cost will always be an upper bound on the actual cost.

\subsection{Examples and Use}

\subsection{Experimental Evaluation and Comparison}


\section{Related and Future Work}



\chapter{Amortized Analysis by Recurrence Extraction}
\label{ch:rec-extr}

\section{Introduction}\label{sec:intro}

% An important aspect of programming is predicting how much of certain
% resources, such as time or space, a program will require to execute.  A very
A common technique for analyzing the asymptotic resource
complexity of functional programs is the
\emph{extract-and-solve} method, in which one extracts a recurrence
expressing an upper bound on the cost of the program in terms of the size of
its input, and then solves the recurrence to obtain a big-$O$ bound.
Typically, the connection between the original program and the extracted
recurrence is left informal, relying on an intuitive understanding that the
extracted recurrence correctly models the program.  Previous
work~\cite{danner-et-al:plpv13,danner-et-al:icfp15,hudson,kavvos-et-al:popl20,
danner-licata:jfp-in-prep} has begun to explore more formal techniques for
relating programs and extracted recurrences.  The process of extracting a
recurrence consists of two phases.  The first is a monadic translation into
the writer monad~$\bbbc\times\cdot$, translating a program to also
``output'' its cost along with its value.  We call the result a
\emph{syntactic recurrence}, and at function type, the result is essentially
a function that maps a value to a pair consisting of the cost of evaluating
that function along with its result.  At higher type, the syntactic
recurrence maps a recurrence for the argument to a recurrence for the
result.  A \emph{bounding logical relation} relates programs to syntactic
recurrences, and the fundamental \emph{bounding theorem} states that a
program and its syntactic recurrence are related, which in particular
implies that its actual runtime cost is bounded by the extracted prediction.
Since inductive values are translated to (essentially) themselves, this
phase does not abstract values to sizes; in effect, the syntactic recurrence
describes the cost of the program in terms of its actual arguments.  The
second phase performs this size abstraction by interpreting (the language
of) syntactic recurrences in a denotational model.  The interpretation of
each type is intended to be a domain of sizes for values of that type, and
different models can implement different notions of size.  For example, a
list value (i.e., the list type and constructors) may be interpreted by its
length in one model, or even more exotic notions of size, such as the number
of pairwise inversions (as required for an analysis of insertion sort) for a
list of numbers.  Thus the interpretation of the syntactic recurrence
extracted from a source program (what we might call the \emph{semantic
recurrence}) is a function that maps sizes (of source-program values) to a
bound on the cost of that program on those values.  
It is these semantic recurrences
that match the recurrences that arise from the typical ``extract-and-solve''
approach to analyzing program cost.  Our previous work develops this
methodology for functional programs with numbers and
lists~\cite{danner-et-al:plpv13}, inductive types with structural
recursion~\cite{danner-et-al:icfp15}, general
recursion~\cite{kavvos-et-al:popl20}, and
let-polymorphism~\cite{danner-licata:jfp-in-prep}.

As an example that demonstrates both the approach and a weakness of the
underlying technique for cost analysis that it formalizes, let us consider
the binary increment function, a standard motivating example for amortized
analysis:
% We define the usual types $\codebitlist$ and $\codenat$ and the
% functions
\begin{small}
\[
\begin{array}[t]{lcl}
\codeInc &:& \codebitlist \to \codebitlist \\
\codeInc\,[\,] &=& [1] \\
\codeInc\,(0 :: bs) &=& 1 :: bs \\
\codeInc\,(1 :: bs) &=& 0 :: \codeInc\,bs
\end{array}
\qquad
\begin{array}[t]{lcl}
\codeSet &:& \codenat \to \codebitlist \\
\codeSet\,0 &=& [\,] \\
\codeSet\,(S\,n) &=& \codeInc(\codeSet\,n)
\end{array}
\]
\end{small}

\noindent
The value part of a monadic translation of a function into~$\bbbc\times\cdot$
is a function into a pair, but here we
sugar that into a pair of functions, which may be mutually recursive.  We
denote the cost and value components by $(\cdot)_c$ and $(\cdot)_p$,
respectively (this notation is explained in
\autoref{sec:monadic-translation}), and charge one unit of cost for each
$::$ operation:
\begin{small}
\[
\begin{array}[t]{lcl}
\codeInc_c &:& \codebitlist \to \bbbc \\
\codeInc_c\,[] &=& 1 \\
\codeInc_c\,(0 :: bs) &=& 1 \\
\codeInc_c\,(1 :: bs) &=& 1 + \codeInc_c\,bs
\\ \\
\codeSet_c &:& \codenat \to \bbbc \\
\codeSet_c\,0 &=& 0 \\
\codeSet_c\,(S\,n) &=& \codeSet_c(n) + \codeInc_c(\codeSet_p\,n)
\end{array}
\qquad
\begin{array}[t]{lcl}
\codeInc_p &:& \codebitlist \to \codebitlist \\
\codeInc_p\,[] &=& [1] \\
\codeInc_p\,(0 :: bs) &=& 1 :: bs \\
\codeInc_p\,(1 :: bs) &=& 0 :: \codeInc_p\,bs
\\ \\
\codeSet_p &:& \codenat \to \codebitlist \\
\codeSet_p\,0 &=& [] \\
\codeSet_p\,(S\,n) &=& \codeInc_p(\codeSet_p\,n)
\end{array}
\]
\end{small}
% We see that function composition is treated as a monadic bind, with the
% ``effect'' of including the cost of the argument.  

We obtain the usual recurrences that we expect when we interpret these
syntactic recurrences in an appropriate denotational semantics.  We
interpret $\codebitlist$ and $\codenat$ by~$\N$, the natural numbers, and
interpret the constructors so that a $\codebitlist$ is interpreted by its
length and a $\codenat$ by its value.  Doing so, we obtain semantic
recurrences for the the cost and size of~$\codeInc$:
\[
\begin{aligned}
T_{\codeInc}(0) &= 1 \\
T_{\codeInc}(n+1) &= \max\{1, 1 + T_{\codeInc}(n)\}
\end{aligned}
\qquad
\begin{aligned}
S_{\codeInc}(0) &= 1 \\
S_{\codeInc}(n+1) &= \max\{1 + n, 1 + S_{\codeInc}(n)\}
\end{aligned}
\]
The usual techniques (in the semantics) then allow us to conclude that
$T_{\codeInc}(n) \leq n + 1$ and $S_{\codeInc}(n) \leq n + 1$, which are correct
and tight bounds on the cost and size of the $\codeInc$ function.  The
semantic recurrences for $\codeSet$ are
\[
\begin{aligned}
T_{\codeSet}(0) &= 0 \\
T_{\codeSet}(n+1) &= T_{\codeSet}(n) + T_{\codeInc}(S_{\codeSet}(n)) \\
                  &\leq T_{\codeSet}(n) + S_{\codeSet}(n) + 1
\end{aligned}
\qquad
\begin{aligned}
S_{\codeSet}(0) &= 0 \\
S_{\codeSet}(n+1) &= S_{\codeInc}(S_{\codeSet}(n)) \\
                  &\leq S_{\codeSet}(n) + 1
\end{aligned}
\]
and so we conclude that $S_{\codeSet}(n)\leq n$ and hence
$T_{\codeSet}(n)\in O(n^2)$, both of which are correct, but not tight,
bounds.  

On the one hand, through syntactic recurrence extraction, the bounding
theorem, and soundness of the semantics, we have a formal connection between
the original programs and the semantic recurrences that bound their cost and
size.  On the other, this example demonstrates a well-understood weakness in
the informal technique:  while the cost of a composition of functions is
bounded by the composition of their costs, the bound is not necessarily
tight.  The tight bound is usually established with some form of amortized
analysis, and \emph{the goal of this paper is to provide a formalization of
the banker's method for amortized analysis comparable to the formalization
of \cite{danner-et-al:plpv13,danner-et-al:icfp15,hudson} for non-amortized
analysis.}

% However, one kind of analysis not covered by this previous work is
% \emph{amortized analysis}, where the cost of expensive operations is
% redistributed to less expensive ones, yielding a more precise bound for a
% collection of operations.  A standard motivating example of amortized
% analysis is implementing a binary counter using a list of bits, with an
% operation $\texttt{set : nat -> [bit]}$ for setting the counter to a
% particular number, implemented using a helper function $\texttt{inc :
%   [bit] -> [bit]}$ for incrementing the counter by one:
% 
% \begin{tabular}{lll}
%   \texttt{inc []} = \texttt{[1]} & \hspace{0.5in} &  \texttt{set 0} = \texttt{[]}\\
%   \texttt{inc (0::\texttt{bs})} = \texttt{1::bs} & \qquad & \texttt{set n} = \texttt{inc (set (n-1))} \\
%   \texttt{inc (1::\texttt{bs})} = \texttt{0::(inc bs)}\\
% \end{tabular}
% 
% \noindent For simplicity, we define the cost to be the number of times a bit is
% flipped ($0$ to $1$ or $1$ to $0$).  Following the usual
% extract-and-solve method, we might obtain worst-case recurrences
% $T_{\texttt{inc}}(l) = 1 + T_{\texttt{inc}}(l-1)$ (where $l$ is the
% length of the input list) and $T_{\texttt{set}}(n) =
% T_{\texttt{set}}(n-1) + T_{\texttt{inc}}(n)$ (where $n$ is the number
% the counter is set to), so $T_{\texttt{inc}} \in O(n)$ and
% $T_{\texttt{set}} \in O(n^2)$.  This analysis requires bounding the
% length of the input list to $\texttt{inc}$ by $n$, while a more precise
% analysis might observe that this length is at most $O(\log n)$, and obtain
% $O(n \log n)$ for $\texttt{set}$.  However, both of these bounds are
% imprecise: in fact, $T_{\texttt{set}}\in O(n)$ overall, intuitively
% because $\texttt{inc}$ is constant time when the first bit is zero, and
% the increments $0 \to 1 \to 2 \to \ldots \to n$ execute this case often
% enough.

The \emph{banker's method}
for amortized analysis~\cite{tarjan:amortized-complexity}
permits one to ``prepay'' time
cost to generate ``credits'' that are ``spent'' later to reduce time
cost, rearranging the accounting of costs from one portion of a program
to another (in particular, generating a credit costs 1 unit of time,
while spending a credit reduces the cost by 1 unit of time).  In this
example, we maintain the invariant that one credit is attached to every $1$ bit in
the counter representation.  The \emph{amortized cost} of flipping a bit
from $0$ to $1$ is then $2$ units of time---one for the actual bit flip
plus one to generate the credit. However, the amortized cost of flipping
a bit from $1$ to $0$ is $0$ units of time---the bit flip takes one unit
of time, but that is paid for by the credit.  Using these new amortized
costs, we can see that $T_{\texttt{inc}}(n)$ is $O(1)$ amortized: in the
case where the first bit is $0$, we flip it to $1$, which costs $2$
units of time, and stop. In the case where the first bit is $1$, we flip
it \emph{for free} to $0$, and then make a recursive call, which
inductively is bounded by 2. So $T_{\texttt{inc}}(n) = 2$, which means
that $T_{\texttt{set}}(n) = 2n$, amortized. Since a single run of
$\texttt{set}$ starts with no credits, its actual cost will be bounded
by the amortized cost $2n$: all of the credits spent during the call to
$\texttt{set}$, which subtract from the cost, must have been created
earlier, incurring a cost which balances out the gain garnered from
spending it.

% In this paper, we extend the formal approach to recurrence extraction
% (most directly following
% \cite{danner-et-al:plpv13,danner-et-al:icfp15,hudson}) to the
% accounting/banker's method for amortized analysis.  This requires us to move

Formalizing recurrence extraction for the banker's method for amortized
analysis requires us to move
from a relatively standard source language based on the simply-typed
$\lambda$-calculus with inductive datatypes to a more specialized one.
We do not expect amortization policies (e.g.\ generate a credit when
flipping a bit from 0 to 1, to be spent when flipping a bit from 1 to 0)
to be automatically inferable in the general case---these policies are the part
of an amortized analysis that requires the most cleverness.  To notate
these policies, we use an \emph{intermediate language} $\lambda^A$
(\autoref{sec:la}) \footnote{
In an unfortunate coincidence, the recurrence extraction project including $\lambda^A$ was developed concurrently with the \lambdaamor project by disjoint sets of authors. This led to a name collision that we hope will not cause the reader too much grief.
}, which has ``effectful'' operations for
generating and spending credits ($\waitname$ and $\discname$), as well
as a modal type operator $!_\ell$ for associating credits with values
(e.g.\ storing a credit with each 1 in a bit list).  The type~$!_\ell A$
classifies a value of type $A$ that has $\ell$ credits associated with
it.  To correctly manage credits, this intermediate language is based on
a form of linear logic, which prevents spending the same credit more
than once; in particular, $\lambda^A$ is an affine lambda calculus with
all of the standard connectives $\otimes, \oplus, \&, \multimap, !$ plus
multiplicities $!^k A$ (where $k$ is a positive number) for tracking
multiple-use values.  The type structure of the intermediate language is
inspired by the credits (written as $\Diamond$) of
\cite{hofmann02diamonds,hofmann03diamonds-journal}, $n$-linear types
(e.g. \cite{girard-et-al:tcs92:bll,reed:names-useless,mcbride:plenty-o-nuttin,atkey:lics18}),
and the uses of credits and linear logic in in automatic amortized
resource analysis (AARA)
(e.g. \cite{hofmannjost03aara,hoffmann-et-al:toplas12:multivariate-amortized,knoth+19resourceguided}).
% However, our goal for the intermediate language is different than the
% above works: we formalize an extract-and-solve-a-recurrence technique
% for amortized analysis.  

% To this end, we give a translation of the intermediate language
% $\lambda^A$ into a \emph{recurrence language} $\lambda^{\bbbc}$.  The
% recurrence language, following \cite{danner-et-al:icfp15,hudson}, is a
The target of the monadic translation is the \emph{recurrence
language}~$\lambda^{\bbbc}$, which,
following~\cite{danner-et-al:icfp15,hudson}, is a
standard simply-typed $\lambda$-calculus with a base type for costs
(linearity is not needed at this stage). It is equipped with an
inequality judgment $E \le_T E'$ that can be used to express upper
bounds.  The translation we define here extracts a recurrence for the
\emph{amortized} cost of the program (where the costs have been
``rearranged''), by translating the credit generation and spending
operations in $\lambda^A$ to modifications of the cost.  We define a
bounding relation (a cross-language logical relation) for the amortized
case, and prove that a term is related to its extraction.  As a
corollary, we obtain that the amortized cost of running a program from
$\lambda^A$ is bounded by the cost component of its translation into
$\lambda^{\bbbc}$; for programs that use no external credits, this gives
a bound on its actual cost as well.  The recurrence language, recurrence
extraction and bounding theorem are described in \autoref{sec:cl}.
Next, we use a denotational semantics of the recurrence language in
preorders, similar to~\cite{danner-et-al:icfp15}, to justify the
consistency of the recurrence language $\le$ judgment, and to simplify
and solve extracted recurrences (\autoref{sec:preorder}).

The version of $\lambda^A$ and the recurrence extraction presented
through \autoref{sec:preorder} allows a statically fixed number of
credits to be stored with each element of a data structure (e.g. 1
credit on element of a list, so $n$ credits overall).  For some
analyses, it is necessary to choose the number of credits stored with an
element dynamically.  For example, when analyzing
splay trees \cite{sleator-tarjan-85}, the number of
credits stored at each node in the tree is a function of the size of the
subtree rooted at that node, which varies for different tree nodes.  To
support such analyses, we extend $\lambda^A$
with existential quantifiers over credit variables
in \autoref{sec:ex}, and use them to code
a portion of \citet{okasaki:purely-functional-data-structures}'s
analysis of splay trees in our system.  

% While our recurrence extraction and the denotational semantics in
% preorders are given as automatic language-to-language translations,
% there are two phases of the analysis that, for this paper, require
% manual intervention.  On the front-end, a source program written in a
% standard functional language must be annotated with its credit usage
% policy by translating it into $\lambda^A$, and on the back-end the
% extracted recurrence must be simplified and solved.  
% We diagram the
The process of extracting and solving a recurrence in diagrammed in
\autoref{fig:pipeline}.
% , where the first and last steps are manual and
% the middle two are automatic.  
%% %: the manual annotation of a program in
%% $\lambda^A$, followed by the automatic translation of $\lambda^A$ into
%% the recurrence language $\lambda^\bbbc$, followed by the general
%% semantics of $\lambda^A$ in preorders, followed by manual simplification
%% and solving, as illustrated
While automation of the annotation and solving steps
is a worthwhile goal,
% , and there are interesting
% questions about how to automate the annotation and solution steps, 
our
main motivation in this paper is to formally justify the
extract-and-solve method for amortized analysis, a technique that we teach and that is
typically used by practitioners.  Connecting the extracted recurrence in
terms of user-defined notions of size to the operational cost is the
least justified step in this process, and so a formal account of it has
important foundational value.  It could likewise have important
practical value: because students and practitioners are trained in the
use of cost recurrences, reverse-engineering a recurrence that yields a
worse-than-expected cost bound to the (mis)implementation may require
a lower cognitive load than doing the same with more
sophisticated techniques.  Moreover, though this technique is less
automated than others, it can handle at least some examples that
existing techniques cannot---to our knowledge, splay trees cannot be
analyzed by the existing automatic techniques.
We give a detailed comparison with related work in \autoref{sec:related-work}.
%% Relative to the previous work on both linear type systems for amortized
%% analysis and formal recurrence extraction, the main contribution of this
%% paper is the combination of these two ingredients: the design of a
%% linear type system for credit tracking and a recurrence extraction
%% procedure that together satisfy a bounding theorem, providing a formal
%% justification for applying the extract-and-solve method to amortized
%% anlayses.  



% We would eventually like
% to automate the annotation and solving steps as well, but here we focus
% on proving the correctness of the extraction of amortized recurrences.  

%% This allows us to use equational reasoning in the posets to simplify
%% our recurrences down to the point where they resemble the kind of
%% recurrence that one would naively write down, and subsequently solve
%% them.  Models of the recurrence language have been studied and
%% formalized in Agda in Hudson's Master's Thesis \cite{hudson}, and our
%% presentation of the semantics differs from this one only in adding
%% cases for the new connectives.

\begin{figure}[t]
  %\vspace{-.25in}
  %\tikzset
{
		line cap = butt ,
		line join = bevel ,
		arrows = -> ,
		> = angle 60 ,
		auto = left ,
		text depth = 0.25ex , % needed for baseline alignment (TiKZ quirk)
		align = center ,
}

\begin{tikzpicture}
	% nodes
	\node (source) at (0,0) {Source \\ language};
	\node [draw] at ($(source) + (3.5,0)$ ) (affine) {Intermediate \\ language $\lambda^A$};
	\node [draw] at ($(affine) + (3.5,0)$ ) (syntactic) {Recurrence \\ language $\lambda^\bbbc$};
	\node [draw] at ($(syntactic) + (3.5,0)$) (semantic) {Semantic \\ recurrences} ;
	
	\draw[dotted] (source) to node [auto] {Annotate} (affine);
	\draw (affine) to node [auto] {$\norm{-}$} (syntactic);
	\draw (syntactic) to node [auto] {$\scott{-}$} (semantic);
	\draw[dotted, loop above] (semantic) to node [auto] {Solve} (semantic);
	
	%\node (source) at (0 , 0) {source \\ language} ;
	%\node (syntactic recurrence) at (3/2 , 0) {recurrence \\ language \\ (syntactic recurrence)} ;
	%\node (semantic recurrence) at (5/2 , 0) {denotational  cost \\ semantics \\ (semantic recurrence)} ;
	%\node [draw , dashed] (extraction) at ($ (source) + (4/6 , 0) $) {recurrence \\ extraction} ;
	%\draw (source) to (extraction) ;
	%\draw (extraction) to (syntactic recurrence) ;
	%\draw (syntactic recurrence) to (semantic recurrence) ;
	%\draw (syntactic recurrence) to node [auto] {$\scott{-}$} (semantic recurrence) ;
\end{tikzpicture}

  \caption{Recurrence Extraction Pipeline}
  \label{fig:pipeline}
\end{figure}


\section{Intermediate Language \texorpdfstring{$\lambda^A$}{}}\label{sec:la}

In this section we discuss the static and operational semantics of
$\lambda^A$, which is an \emph{affine} lambda calculus---it permits
weakening (unused variables) but not contraction (duplication of
variables).  It includes some standard connectives of linear logic, such
as positive/eager/multiplicative products ($\otimes$ and $1$),
sums/coproducts ($\oplus$), and functions ($\loli$), as well as
negative/lazy/additive products ($\amp$).  The language has two basic
datatypes, natural numbers ($\N$) and (eager) lists ($\listty A$), both
with structural recursion (though we expect these techniques to extend
to all strictly positive inductive
types~\cite{danner-et-al:icfp15,danner-licata:jfp-in-prep}).
%% it also includes suspensions, which are used to combine case
%% analysis and recursion into one elimination form without always
%% requiring recursive calls to be run, by suspending the recursive call.

In addition to these, $\lambda^A$ contains some constructs specific to
its role as an intermediate language for expressing amortized analyses.
First, instead of fixing the operational costs of $\lambda^A$'s programs
themselves, we include a \texttt{tick} operation which costs 1 unit of
time, and assume that the translation of a program into $\lambda^A$ has
annotated the program with sufficient ticks to model the desired
operational cost~\cite{danielsson:popl08} (for example, we can
charge only for bit flips in the above binary counter
program).

Second, we have operations \waitname\ and \discname\ for creating and spending
credits, which respectively increase and decrease the
\emph{amortized} cost of the program \textit{without changing} the true
operational cost.
% These should be thought of as structural rules or
% effects; they are not specific to any type.

Third, we have a type constructor $!_\ell A$, where a value of this type
is a value of type $A$ with $\ell$ credits attached; its introduction
and elimination rules allow for the movement of credits around a
program.  The combination of of \discname\/ and the $!_\ell$ modality
motivates our affine type system: because spending credits decreases the
amortized cost of a program, we must ensure that a credit is spent only
once, so credits should not be duplicated; because credits can be stored
in values, values cannot in general be duplicated as well.  However,
$\lambda^A$ does allow credit weakening---choosing not to spend
available credits---because this increases the amortized cost (relative
to spending the credits), and we are interested in upper bounds on
running time.  While the basic affine type system allows a variable to
be used only once, to simplify the expression of programs that use a
variable a fixed number of times, we use $n$-linear types (see e.g.
\cite{girard-et-al:tcs92:bll,reed:names-useless,mcbride:plenty-o-nuttin,atkey:lics18}),
where variables are annotated with a multiplicity $k$, and can be used
at most $k$ times.\footnote{While Girard's notation for multiplicities
  is $!_k A$~\cite{girard-et-al:tcs92:bll}, we write superscripts
  following~\citet{atkey:lics18}, and write subscripts for the
  credit-storing modality, which is used more frequently in our system.}
This is internalized by a modality $!^k A$, which represents an $A$ that
can be used at most $k$ times.  We additionally allow $k$ to be
$\infty$, in which case $!^\infty A$ is the usual exponential of linear
logic, allowing unrestricted use.  Using this modality, standard
functional programs can be coded in $\lambda^A$, but our current
recurrence extraction does not handle the $!^\infty$ fragment very well,
as explained below---at present, we use $!^\infty$ mainly as a technical
device for typing recursors.  It is technically convenient to combine
the two modalities into one type former $!^k_\ell A$, which represents
an $A$ that can be used $k$ times, which also has $\ell$ credits
attached (total, not $\ell$ credits with each use).  Because $k$ is a
coefficient but $\ell$ is an additive constant, the individual
modalities are recovered as $!^k A := !^k_0 A$ and $!_\ell A := !^1_\ell
A$.  In pure affine logic, one can think of $!^k_\ell A$ as $X \otimes
\ldots \otimes X \otimes A \otimes \ldots \otimes A$ with $\ell$ $X$s
and $k$ $A$'s (in the case where $k$ and $\ell$ are finite), for an
atomic proposition $X$ representing a single credit.  However, our
judgmental presentation is easier to work with for our bounding relation
and theorem below, and the $n$-linear modality $!^k A$ ensures that
additional invariant that it is the \emph{same} value that can be used
$k$ times, i.e. it only allows the diagonal of $A \otimes \ldots \otimes
A$.

\begin{figure}
  %\begin{small}
\[\begin{array}{lll}
\text{Types} & A,B,C & ::= \N \; | \; \listty A \; |\; A \loli B \;|\; A \otimes B \;|\; A \oplus B \;|\; A \amp B \;|\; !^k_\ell A\\
\text{Terms} & M,N & ::= x \; | \; \texttt{tick} \, ;\, M \;|\; \texttt{create}_\ell \; M \;|\; \texttt{spend}_\ell \; M \;|\;
\texttt{save}^k_\ell \; M  \;|\; \tsfer {k'} k {\ell} x M N\\
&& \quad | \; \lambda x.M \;|\; M \; N \;|\; \inl M \;|\; \inr M \;|\;
\texttt{case}_{k'}(M,x.N_1,y.N_2) \;|\; \amppair{M}{N} \;|\;
\pi_1 M \;|\; \pi_2 M\\
&& \quad |\; \texttt{split}(M,\, x.y.N) \;|\;
0 \;|\; S(M) \;|\; \texttt{nrec}(M,N_1,N_2) \;|\; \texttt{[]} \;|\;
\cons{M}{N} \;|\; \texttt{lrec}(M,N_1,N_2)\\
\end{array}
\]
\end{small}

  \vspace{-0.2in}
  \caption{$\lambda^A$ Grammar}
  \label{fig:la-bnf}
\end{figure}

\subsection{Type System}

In Fig.~\ref{fig:la-ty-rules} we define
a typing judgment of the form
$\Gamma \vdash_f M : A$, where $\Gamma$ is a standard context $x_1:A_1,
x_2:A_2, \ldots, x_n : A_n$ and $f$ is a \textit{resource} term of the
form $a_1 x_1 + a_2 x_2 + \ldots + a_n x_n + \ell$, where
$x_1,\ldots,x_n$ are the variables in $\Gamma$ and $a_i$ and $\ell$ are
natural numbers or $\infty$.  The resource term $f$ can be
thought of as annotating each variable $x_i$ with the number of times
$a_i$ that it is allowed to occur, and additionally annotating the
judgment with a nonnegative ``bank'' $\ell$ of available credits
that can be used.  For example, the judgment $x : A, y : B, z : C
\vdash_{3x+2y+0z+2} M : D$, means that $M$ is a term of type $D$, which
may use $x$ at most $3$ times, $y$ at most twice, $z$ not at all, and
has access to $2$ credits.  We consider these resource terms up to the
usual arithmetic identities (associativity, unit, commutativity,
distributivity, $0 f = 0$, $\infty k = \infty$ otherwise, etc.).  In the
admissible substitution rule, we write $g[f/x]$ to denote the result of
normalizing the textual substitution of $f$ for $x$ in $g$ according to
these identities; e.g. $(3x+2y+2)[10a+11b+3/x] = 30a+33b+2y+11$. 
Our judgmental presentation of $n$-linear types differs from some 
existing ones-- the reader more familiar with Girard's BLL~\cite{girard-et-al:tcs92:bll}
may read $\Gamma \vdash_f M : A$ as analogous to $!_{\vec{f}} \Gamma \vdash M : A$
-- but this type system was derived as an instance of a general framework for modal
types~\cite{lsr}, which, for our purposes, simplifies the presentation of standard 
metatheorems like substitution. Note that
the resource terms $f$ play a different role than the resource
polynomials in Bounded Linear Logic and AARA~\cite{girard-et-al:tcs92:bll,hoffmann-et-al:toplas12:multivariate-amortized}, 
which provide a mechanism for measuring the size and credit allocation in a
data structure.  The resource terms are also affine in the sense of a
polynomial---the exponent of every variable is 1, except for the constant term
$\ell$---but we will avoid this meaning of affine to avoid
confusion with ``affine logic'' (allowing weakening but not contraction).

\begin{figure}
  %\begin{small}
  \begin{mathpar}

\text{Admissible:}
\qquad
\infer{\Gamma \vdash_g M : A}
      {\Gamma \vdash_f M : A & g \ge f}
\qquad
\infer{\Gamma,y:B \vdash_{f+0y} M : A}
      {\Gamma \vdash_f M : A}
\qquad
\infer{\Gamma \vdash_{g[f/x]} N[M/x] : B}
      {\Gamma \vdash_f M : A&
        \Gamma, x : A \vdash_g N : B &
      }

\end{mathpar}

\hrule

\begin{mathpar}

\infer{\Gamma, x : A \vdash_{f+x} x : A}{}

\infer{\Gamma \vdash_f \tick M : A}{\Gamma \vdash_f M : A}

\infer{\Gamma \vdash_f \wait \ell M : A}
{\Gamma \vdash_{f+\ell} M : A}

\infer{\Gamma \vdash_{f+\ell} \disc \ell M : A}
{\Gamma \vdash_f M : A}

\\

\infer{
\Gamma \vdash_g \save k \ell M : !^k_\ell A
}{
\Gamma \vdash_f M : A
& kf+\ell \leq g
}

\infer{
\Gamma \vdash_{k'f+g} \tsfer {k'} k \ell x M N : B
}{
\Gamma \vdash_f M : !^k_\ell A & \Gamma,x : A \vdash_{g+k'(kx+\ell)} N : B
}
\\

\infer{
\Gamma \vdash_f \lambda x.M : A \loli B
}{
\Gamma, x : A \vdash_{f + x} M : B
}

\infer{
\Gamma \vdash_{f + g} M \; N : B
}
{
\Gamma \vdash_f M : A \loli B & \Gamma \vdash_g N : A
}
\\

\infer{
\Gamma \vdash_f \inl M : A \oplus B
}{\Gamma \vdash_f M : A}

\infer{
\Gamma \vdash_f \inr M : A \oplus B
}{\Gamma \vdash_f M : B}

\infer{
\Gamma \vdash_{k'f + g} \acase {k'} M x {N_1} y {N_2} : C
}{
  \begin{array}{l}
    \Gamma \vdash_f M : A \oplus B \\
    \Gamma, x : A \vdash_{g+k'x} N_1 : C \\ \Gamma, y : B \vdash_{g+k'y}
    N_2 : C
  \end{array}
}

\\

\infer{\Gamma \vdash_f \amppair M N : A \amp B}{\Gamma \vdash_f M : A & \Gamma \vdash_f N : B}
\qquad
\infer{\Gamma \vdash_f \pi_i M : A_i}{\Gamma \vdash_f M : A_1 \amp A_2}
 
\\

\infer{\Gamma \vdash_f () : 1}{}

\infer{\Gamma \vdash_{f+g} (M,N) : A \tensor B}{\Gamma \vdash_f M : A & \Gamma \vdash_g N : B}

\infer{\Gamma \vdash_{k'f+g} \asplit {k'} M x y N : C}{\Gamma \vdash_f M : A \tensor B & \Gamma,x:A,y:B \vdash_{g+k'(x+y)} N : C}

\\

\infer{
\Gamma \vdash_f 0 : \N
}{}

\infer{
\Gamma \vdash_f S(M) : \N
}{
\Gamma \vdash_f M : \N
}

\infer{
\Gamma \vdash_{f+g_1+g_2} \nrec M {N_1} {N_2} : C
}{
  \begin{array}{l}
    \Gamma \vdash_f M : \N
    \\
\Gamma \vdash_{g_1} N_1 : 1 \loli C
\\
\Gamma\vdash_{g_2} N_2 : !_0^\infty(\N \otimes (1 \loli C) \loli C)
  \end{array}
}


\\

\infer{\Gamma \vdash_f \elist : \listty A}{}
\qquad
\infer{\Gamma \vdash_{f+g} \cons{M_1}{M_2} : \listty A}{
\Gamma \vdash_f M_1 : A
&
\Gamma \vdash_g M_2 : \listty A
}
\qquad
\infer{\Gamma \vdash_{f+g_1+g_2} \lrec M {N_1} {N_2} : C}
      {\begin{array}{l}
          \Gamma \vdash_f M : \listty A \\
          \Gamma \vdash_{g_1} N_1 : 1 \loli C \\
          \Gamma \vdash_{g_2} N_2 : !_0^\infty(A\tensor(\listty A \amp
          C) \loli C)
        \end{array}}

\end{mathpar}
\end{small}

  \vspace{-0.2in}
  \caption{$\lambda^A$ Typing Rules}
  \label{fig:la-ty-rules}
\end{figure}

\subsubsection{Structural Rules.}

The rules make three structural principles admissible:

\begin{restatable}[Admissible structural rules]{theorem}{lastructural} \label{thm:la-structural}\hfill
  \begin{itemize}
\item Resource Weakening: Write $g \ge f$ for the coefficient-wise
  partial order on resource terms ($a_1 x_1 + a_2 x_2 + \ldots + \ell
  \ge$ $b_1 x_1 + b_2 x_2 + \ldots + \ell'$ iff $a_i \ge b_i$ for all
  $i$ and $\ell \ge \ell'$).  Then if $\; \Gamma \vdash_f M : A$ and $g
  \geq f$ then $\Gamma \vdash_g M : A$.

\item Variable Weakening:
If $\Gamma \vdash_f M : A$ and $y$ does not occur in $\Gamma$, then $\Gamma,y:B \vdash_{f+0y} M : A$.
  
\item
    Substitution: 
If $\Gamma \vdash_f M : A$ and $\Gamma, x : A \vdash_g N : B$, then
$\Gamma \vdash_{g[f/x]} N[M/x] : B$

  \end{itemize}
\end{restatable}
\begin{proof}
By induction on derivations.
\end{proof}

First, we can weaken the resource subscript, allowing more uses of a
variable or more credits in the bank (e.g.\ if $\cdot \vdash_3 M : A$,
then $\cdot \vdash_5 M : A$).  Second, we can weaken a context
to include an unused variable (we write $f+0y$ for emphasis, but by
equating resource terms up to arithmetic identities, this is just $f$).
Third, we can substitute one term into another, performing the
corresponding substitution on resource terms.  The idea is that, if $N$
uses a variable $x$ say $3$ times, then it requires 3 times the
resources needed to make $M$ to duplicate $M$ three times; this
multiplication occurs when substituting $f$ for the occurrence of $x$ in
$g$.

\subsubsection{Multiplicative/Additive Rules in $n$-linear Style.}
In the $n$-linear types style of presentation, rules of linear logic
that traditionally split the context (e.g. $\otimes$ introduction,
$\loli$ elimination) sum the resources used in each premise, but keep
the same underlying variable context $\Gamma$ in all premises.  For
example, in a positive pair $(M,N) : A \otimes B$, if $M$ is allowed to
use $x$ 3 times and $N$ is allowed to use $x$ 4 times, then the whole
pair must be allowed to use $x$ 7 times.  As a special case, if a
variable is not allowed to occur in, e.g., $N$, it can be marked with a
coefficient of 0.  On the other hand, rules for additives (e.g. pairing
for $A \& B$) use the same resource term in multiple premises.  While
the elimination rule for $\oplus$ is additive in sequent calculus style,
in natural deduction there is some summing because it builds in a cut
for the term being case-analyzed.

\subsubsection{Ticks, and Creating/Spending Credits.}\label{ssec:wdt}

The $\texttt{tick} \; ; \; M$ construct is used to mark program points that
are intended to incur one unit of time cost (e.g.\ bit flips in the binary
counter example); it uses the same resources as $M$.  

$\waitname$ is the means to create credits, where $\waitname_\ell$
gives $M$ access to $\ell$ extra credits to use, along with whatever
resources are present in the ambient context; formally, this is
represented by adding to the ``bank'' in the premise of the typing rule
for $M$.  In the operational semantics and recurrence extraction below,
\waitname\/ adds $\ell$ steps to the amortized cost of $M$---it is used to
``prepay'' for later costs.  

$\discname$ is the means to spend credits, where $\discname_\ell$ spends
$\ell$ credits; because credits can only be spent once, these $\ell$
credits in the conclusion of the typing rule are not also available in
the premise for $M$.  In the operational semantics/recurrence
extraction, \discname\/ subtracts $\ell$ steps from the amortized cost of
$M$---it is used to take advantage of prepaid steps.  Note that
\discname\/ satisfies the same typing judgments as an instance of
resource weakening (because $f + \ell \ge f$); the ``silent'' weakening
does not change the amortized cost, but instead is a case where our recurrence extraction might
obtain a non-tight upper-bound.

\subsubsection{$!^k_\ell$ Modality.}
Instead of having two separate modalities, one for $n$-use types and the
other for types storing credits, we combine them into a single modality
$!^k_\ell A$. A value of type $!^k_\ell A$ is a $k$-use $A$ with $\ell$
credits attached (not $k \cdot \ell$ credits, which is what one would
expect if each use had $\ell$ credits attached---though that could be
modeled by the type $!^k_0 (!^1_\ell A)$).  While we write $a$ and
$\ell$ for nonnegative numbers or $\infty$, we restrict $k$ to range
over a \emph{positive} number or $\infty$ -- i.e. we do not allow a
``zero-use'' modality $!^0_\ell A$, which would complicate the
erasure of $\lambda^A$ to regular simply typed lambda calculus.

The introduction rule for $!^k_\ell$ says that if we can prove $M$ has
type~$A$ with~$f$ resources, then a version of~$M$ that can be used $k$
times requires $kf$ resources.  If in addition, $\ell$ credits are to be
attached, then $kf+\ell$ resources are required.  Intuitively, one can
think of $\save k \ell M$ as the act of running $M$ once to obtain its
value, but repeating whatever requirement it imposes on the bank $k$
times, which justifies making $k$ uses of its value, and then attaching
$\ell$ credits to this value.  In order to make resource weakening
admissible in general, it is necessary to build weakening into this
rule (the second premise).

The elimination rule for the modality allows for the credit stored
on a term to be released into the ambient context of another in order to
be redistributed or spent. We first present a simplified version, and
then explain the general version. Given $\Gamma \vdash_f M : !^k_\ell
A$, we essentially have $k$ copies of an $A$, along with $\ell$ extra
credits. Given a term $N$ which can use $k$ copies of an $A$ and $\ell$
credits, $\Gamma,y : A \vdash_{ky + \ell} N : C$, we can form the term
$\Gamma \vdash_{f} \texttt{transfer} \, !^k_\ell y = M \; \texttt{to} \;
N : C$, which, intuitively, deconstructs $M$ into its $k$-usable value
and $\ell$ credits, and moves them to $N$, where they can be used. On
top of this version, we make two modifications. Firstly, $N$ should have
access to resources other than just what's provided to it by $M$-- so we
add a resource term $g$ available in $N$ (and
therefore required to type the
$\texttt{transfer}$). Secondly, it may be necessary at the site of the
transfer to further duplicate the $M : !^k_\ell A$ --- this is required to
prove a fusion law below, for example.  To support this,
we parameterize the $\texttt{transfer}$ term
by another number, $k'$, arriving at the version of the rule
presented in \autoref{fig:la-ty-rules}, which should be thought of as
eliminating $k'$ copies of a $!^k_\ell A$ at once.
The rules for other positive types ($\oplus,\otimes$) similarly permit elimination of multiple copies at
once.

The $!$ modality satisfies the following interactions with other logical
connectives, where we write $A \dashv \vdash B$ to mean
interprovability/functions in both directions:

\begin{restatable}[Fusion Laws]{theorem}{fusion}\hfill
\label{thm:fusion}
\begin{enumerate}
  \item $!^{k_1k_2}_{\ell_1 + k_1 \cdot \ell_2} A \dashv \vdash \, !^{k_1}_{\ell_1} !^{k_2}_{\ell_2} A$
  \item $!^k_{\ell_1 + \ell_2} (A \otimes B) \dashv \vdash \, !^k_{\ell_1} A \otimes !^k_{\ell_1} B$
  \item $!^k_\ell (A \oplus B) \dashv \vdash \, !^k_\ell A \oplus !^k_\ell B$
\end{enumerate}
\end{restatable}

\subsubsection{Natural Number Recursor} \label{sec:ns-rules}

For natural numbers, while the rules for zero and successor are standard,
the recursor takes a bit of explanation.  We think of
the recursor \texttt{nrec} as a function constant of type
$\N \loli (1 \loli C) \loli !^\infty_0(\N \times (1 \loli C) \loli C) 
 \loli C$.
The base case is ``thunked'' because we think of $\loli$ as
a call-by-value function type, but the base case should not be evaluated until the recurrence
argument is~$0$.  The ordinary type for the step function (inductive case)
would be $(\N \times C \loli C)$, but we also suspend the recursive
call, to allow for a simple case analysis that chooses not to use the
recursive call.  
The $!^\infty_0$ modality surrounding the step function is needed to
ensure that the step function itself does not use any ambient credits,
which is necessary because the step function is applied repeatedly by
the recursor ($n$ times if the value of $M$ is $n$).  Without this
restriction, one could, for example, iterate a step function that spends
$k$ credits to subtract $Mk$ credits from the amortized cost, while only
having $k$ credits in the bank to spend.  For example, without the use of $!^\infty_0$, the term
$\cdot\vdash_1\nrec 7 {\lambda\_.0}{\lambda\_.\disc 1 0} : \N$ typechecks
with only one credit in the ambient bank, but 
intuitively subtracts 7 from the amortized cost, rather than just the
1 credit that was allowed.  
We solve this problem using the type $!^\infty_0 A$ (where $A$ is the ordinary
type of the step function $\N \otimes (1 \loli C) \loli C$),
which represents an infinitely duplicable $A$ that stores no additional credits.
Being infinitely duplicable is an over-approximation, because
the step function really only needs to be run $M$ times, but
being more precise would require reasoning about such values in the
type system.

In the common case, the step function will use other
infinite-use variables but no credits from the bank.  A typical 
typing derivation for this case, where $H$ is the type of a helper
function and $A$ is the type of the step function, would be
\[
\infer{f : H \vdash_{\infty (\infty f) = \infty f} \save{\infty}{0}{N_2'} : !^\infty_0 A}
      {f : H \vdash_{\infty f} N_2' : A}
\]
Using this as the third premise of the typing rule of \texttt{nrec}, we
see that such an \texttt{nrec} itself requires only the credits demanded
by the number argument ($M$) and base case ($N_1$), assuming $f$ is
substituted by a helper function that uses no credits.

The way in which the $!^\infty$ modality ``prevents'' the use of credits
from the bank is somewhat subtle: a step function \emph{can} use credits
from the bank, but this will require the bank to be infinite in the
conclusion.  This is because the introduction rule for $!^\infty_0$ 
inflates any finite resources to $\infty$ in the conclusion:
\[
\infer{f : H \vdash_{\infty(2f + 3) = \infty f + \infty} \save{\infty}{0}{N_2'} : !^\infty_0 A}
      {f : H \vdash_{2x + 3} N_2' : A}
\]
Thus, the step function is only permitted to use credits from the bank
when the bank has $\infty$ credits in the conclusion, while we are
generally interested in programs that use finitely many credits.  

%% To solve this problem, one could enforce that the step function of a
%% recursor not use any outside resources whatsoever- but this prevents
%% the step function from using any variables that don't carry credit,
%% which should be allowable. We get around this limitation by making two
%% moves. First, we allow the step function to use outside resources that
%% carry credit- but at the cost of our predicted bounds. If a step
%% function captures credit from outside, the recurrence we will extract
%% for the recursor term will blow up to infinity, as we can no longer
%% guarantee anything about the cost of the function. Secondly, we
%% leverage the fact that terms which carry no credit are infinitely
%% duplicable. In other words, they have type $!^\infty_0 A$. To give
%% some intuition about how this works, we consider the case of closed
%% terms (which are of course the ones that we will evaluate). Note that
%% in a term typed in the empty context, the resource term it's typed
%% with can always be strengthened to just a number of credits- since the
%% term we are typing cannot refer to any variables. We range over such
%% resource terms with $a,b$. The introduction rule for $!^\infty_0$ says
%% that $\cdot \vdash_{\infty \cdot a} \save \infty 0 M : !^\infty_0 A$
%% when $\cdot \vdash_a M : A$. If $a \neq 0$, then $\infty \cdot a =
%% \infty$, which as we will see in \autoref{sec:cl}, will make the
%% recurrence we extract trivial. However, if $a = 0$, then $\infty \cdot
%% a = 0$, and so $\cdot \vdash_0 \save \infty 0 M : !^\infty_0 A$, and
%% hence the term uses no credit. \textbf{This section could be cleaned
%%   up- revisit.} So, in order to prevent the step function from
%% re-using credits, we wrap it in $!^\infty_0$, giving us the type of
%% the $\N$-recursor,
%% $$
%% \N \loli (1 \loli C) \loli \, !^\infty_0(\N \times (\susp C) \loli C) \loli C
%% $$

\subsubsection{List Recursor}
The list recursor $\lrec {M} {N_1} {N_2}$ has the same ``credit
capture'' problem as the recursor on naturals, which we solve using
$!^\infty_0$.  The list recursor has another challenge, though,
because unlike a natural number, the values of the list can themselves
store credits.  Because of this, to prevent credits from being
duplicated, in the cons case, the recursor may use \emph{either} the
tail of the list or the recursive result, but not both.  We code this
using an internal choice/negative product $\&$.  The negative product
will itself be treated as a lazy type constructor, where an $A \with B$
pair is a value even when the $A$ and $B$ are not, so we do not need
to further thunk the recursive result $C$ here.

%% NOTE: can include counterexample in long version

%% However, induce another problem, whose solution is the source of the
%% negative product ($\&$) in the type of the step function. Suppose for
%% the purposes of illustration that the type of the list recursor were
%% $\listty A \loli (1 \loli C) \loli !^\infty_0(A \otimes (\listty A
%% \otimes (\susp C)) \loli C) \loli C$.  Using some syntactic sugar and
%% non-exhaustive pattern matching to simplify the presentation, consider
%% the term

%% $$
%% \infer{
%%  \cdot \vdash_0 \texttt{lrec}([v_1,v_2],\lambda z. z ,\save \infty 0 {(\lambda (y,(z,r)). \tsfer {1} {1} {\ell} \_ y {(\disc \ell M)})}) : 1
%% }{
%%   \ldots
%% }
%% $$
%% where $v_1,v_2$ are values of type $!^1_\ell A$, and
%% $$
%% M = \lrec z {\lambda\_. \force r} {\save \infty 0 {(\lambda (y',(\_,r')). \tsfer {1} {1} \ell \_ {y'} {(\disc \ell {(\force r')})} )}}  
%% $$

%% In essence, this term will perform a single loop through $[v_1,v_2]$, but spends the credits on the first two elements both times, which ends up with the credit on $v_2$ being used twice, which should not be allowable. This stems from the fact that the step function has access to \textit{both} the tail of the list and the recursive call on the tail-which allows us to play tricks like that above and use each element twice. To solve this, we enforce that the step function only be given access to the tail of the list or the recursive call, but not both. This pattern of access to a pair is sometimes known as an \textit{internal choice} (\textbf{Cite someone here}?) and is central to the use of the negative presentation of the product type in linear type systems. We use this negative product in the type of the argument to the step function for the list recursor:
%% $$
%% \texttt{lrec} : \listty A \loli (1 \loli C) \loli \, !^\infty_0(A \otimes (\listty A \amp C) \loli C) \loli C  
%% $$

\subsection{Operational Semantics for \texorpdfstring{$\lambda^A$}{the
intermediate language}} \label{sec:la-sem}

We present a call-by-value big-step operational semantics for
$\lambda^A$ in \autoref{fig:la-sem-rules}, whose primary judgment
form is $M \downarrow^{(n,r)} v$, which means that $M$ evaluates to the
value $v$ with cost $(n,r)$.  The first component of the cost, $n$ (a
non-negative number) indicates the \textit{real cost} of evaluating $M$,
in this case the number of $\texttt{ticks}$ performed while evaluating
$M$.  The second component, $r$ (which can be any integer), tracks
$\waitname$s and $\discname$s --- the (possibly negative) sum total of
credits created and spent while evaluating $M$, where creating is
positive and spending is negative.  The \emph{amortized cost} of
evaluating $M$ is $n + r$: the number of ``actual" steps taken, plus the
number of credits created, minus the number spent.

One reason we separate $n$ and $r$ in the judgment form is that there is
a straightforward \emph{erasure} of $\lambda^A$ to ordinary simply typed
$\lambda$-calculus (STLC with a \texttt{tick} operation), in which
evaluating the STLC program has cost (number of ticks) $n$.  Briefly,
this translation translates $!^k_\ell A$ to $A$, translates all of the
linear connectives to their unrestricted counterparts, drops all
\waitname, \discname, \texttt{save} term constructors, and translates
\texttt{transfer} to a \texttt{let}.  The definition of $n$ in each of
our inference rules for $M \downarrow^{(n,r)} v$ is the same as the
usual cost for STLC with a tick operation, so this erasure preserves
cost.  Because of this erasure, the $n$ in $M \downarrow^{(n,r)} v$ is a
meaningful cost to bound. Further, the distinction between $n$ and $r$
is why we have separate terms $\waitname$ and $\texttt{tick}$:
$\texttt{tick}$ increases the operational cost which should be preserved
under erasure, while $\texttt{create}$ increase the amortized cost only.

\begin{figure}
  %\begin{small}
  \begin{mathpar}
\infer{\tick M \downarrow^{(1+n,r)} v}{M \downarrow^{(n,r)} v}

\infer{\wait {\ell} M \downarrow^{(n,r+\ell)} v}
{M \downarrow^{(n,r)} v}

\infer{\disc {\ell} M \downarrow^{(n,r-\ell)} v}
{M \downarrow^{(n,r)} v}

\\

\infer{
\save k {\ell} M \downarrow^{(n,kr)} \save k {\ell} v
}{
M \downarrow^{(n,r)} v
}

\infer{
\tsfer {k'} k {\ell} x M N \downarrow^{(n_1 + n_2,k'r_1 + r_2)} v
}{
M \downarrow^{(n_1,r_1)} \save k {\ell} {v_1} & N[v_1/x] \downarrow^{(n_2,r_2)} v
}

\\

\infer{\lambda x.M \downarrow^{(0,0)} \lambda x.M}{}
\qquad
\infer{
M \; N \downarrow^{(n_1+n_2+n_3,r_1+r_2+r_3)} v
}{
  M \downarrow^{(n_1,r_1)} \lambda x.M'&
  N \downarrow^{(n_2,r_2)}v_1&
  M'[v_1/x] \downarrow^{(n_3,r_3)} v
}
\\

\infer{
\inr M \downarrow^{(n,r)} \inr v
}{
M \downarrow^{(n,r)} v
}
\qquad
\infer{
\acase {k'} M x {N_1} y {N_2} \downarrow^{(n_1+n_2,k'r_1+r_2)} v
}{
M \downarrow^{(n_1,r_1)} \inr v_1
&
N_2[v_1/x] \downarrow^{(n_2,r_2)} v
}

\\

\infer{
\inl M \downarrow^{(n,r)} \inl v
}{
M \downarrow^{(n,r)} v
}
\qquad
\infer{
\acase {k'} M x {N_1} y {N_2} \downarrow^{(n_1+n_2,k'r_1+r_2)} v
}{
M \downarrow^{(n_1,r_1)} \inl v_1
&
N_1[v_1/x] \downarrow^{(n_2,r_2)} v
}

\\

\infer{\amppair M N \downarrow^{(0,0)} \amppair M N}{}
 
\infer{\pi_i M \downarrow^{(n_1+n_2,r_1+r_2)} v}{M \downarrow^{(n_1,r_1)} \amppair {N_1} {N_2} & N_i \downarrow^{(n_2,r_2)} v}

\\


\infer{(M,N) \downarrow^{(n_1+n_1,r_1+r_2) }(v_1,v_2)}{
M \downarrow^{(n_1,r_1)} v_1
&
N \downarrow^{(n_2,r_2)} v_2
}

\infer{\asplit {k'} M x y N \downarrow^{(n_1 + n_2,k'r_1 + r_2)} v}{
M \downarrow^{(n_1,r_1)} (v_1,v_2)
&
N[v_1/x,v_2/y] \downarrow^{(n_2,r_2)} v
}

\\

\infer{0 \downarrow^{(0,0)} 0}{}

\infer{
S(M) \downarrow^{(n,r)} S(v)
}{M \downarrow^{(n,r)} v}

\infer{(\,) \downarrow^{(0,0)} (\,)}{}


\\

\infer{
\nrec M {N_1} {N_2} \downarrow^{(n_1+n_2+n_3+n_4,r_1+r_2+r_3 + r_4)} v
}{
M \downarrow^{(n_1,r_1)} 0
& N_1 \downarrow^{(n_2,r_2)} \lambda x.N_1'
& N_2 \downarrow^{(n_3,r_3)} v'
& N_1'[()/x] \downarrow^{(n_4,r_4)} v
}

\\

\infer{
\nrec M {N_1} {N_2} \downarrow^{(n_1+n_2+n_3+n_4,r_1+r_2+r_3+r_4)} v
}{
\begin{array}{l}
M \downarrow^{(n_1,r_1)} S(v_1)\\
N_2 \downarrow^{(n_2,r_2)} \save \infty 0 {(\lambda x.N_2')}\\
N_1 \downarrow^{(n_3,r_3)} \lambda x.N_1'\\
N_2'[(v_1,\lambda z. {(\nrec {v_1} {\lambda x.N_1'} {\save \infty 0 {(\lambda x.N_2')}} )}) / x] \downarrow^{(n_4,r_4)} v
\end{array}
}

\\


\infer{\lrec M {N_1} {N_2} \downarrow^{(n_1+n_2+n_3+n_4,r_1+r_2+r_3+r_4)} v}
{M \downarrow^{(n_1,r_1)} [] & N_1 \downarrow^{(n_2,r_2)} \lambda x.N_1'
& N_2 \downarrow^{(n_3,r_3)} \save \infty 0 {(\lambda x.N_2')} & N_1'[()/x] \downarrow^{(n_4,r_4)} v
}

\infer{\lrec M {N_1} {N_2} \downarrow^{(n_1+n_2+n_3+n_4,r_1+r_2+r_3+r_4)} v}
{
\begin{array}{l}
M \downarrow^{(n_1,r_1)} \cons {v_1} {v_2} \\
N_2 \downarrow^{(n_2,r_2)} \save \infty 0 {(\lambda x.N_2')}\\
N_1 \downarrow^{(n_3,r_3)} \lambda x.N_1'\\
N_2'[(v_1,\amppair {v_2} {\lrec {v_2} {\lambda x.N_1'} {\save \infty 0 (\lambda x.N_2')} })/x] \downarrow^{(n_4,r_4)} v
\end{array}
}

\end{mathpar}
\end{small}

  \caption{$\lambda^A$ Operational Semantics}
  \label{fig:la-sem-rules}
\end{figure}

As discussed in
\autoref{ssec:wdt}, $\wait \ell M$ creates $\ell$ credits for $M$ to
use for the price of $\ell$ units of time cost, whereas
    %% When $M \downarrow^{(n,r)} v$, then $\wait \ell M \downarrow^{(n,r +
    %%   \ell)} v$. If $M$ evaluates with an amortized cost of $n + r$,
    %% $\wait \ell M$ evaluates with an amortized cost of $n + r + \ell$,
    %% $\ell$ more than the cost of $M$. Of course, for this extra time,
    %% $M$ is given $\ell$ credits to play with.  Dually to \waitname,
\discname\/ subtracts from the amortized cost of an expression ---
a speedup which is paid for by the $\ell$ credits which the body is no
longer allowed to use.  Both are reflected by corresponding changes
to~$r$.

    %% If $M \downarrow^{(n,r)} v$, then $\disc \ell M
    %% \downarrow^{(n,r-\ell)} v$, and so the amortized cost of $\disc \ell
    %% M$ is $\ell$ less than that of $M$.

The operational intuition for $\save{k}{\ell}{M} : \: !^k_\ell A$ is that
it runs $M$ once, but repeats whatever effect this had on the credit
bank $k$ times, which justifies using the credits in the value of $M$
$k$ times.  (The erasure to STLC discussed above runs $M$ only once, not
$k$ times---which would be challenging when $k$ is $\infty$.)  Formally,
this means that the $n$ in the conclusion is just the $n$ in the
premise, but the $r$ is multiplied by $k$.  Running
$\texttt{save}^k_\ell$ does \textit{not} add $\ell$ to the $r$ component
because \texttt{save} does not create credits (adding to the amortized
cost), but only attaches some already existing credits to the value $v$.
Recall that \texttt{transfer} detaches the credits from a $!^k_\ell$
value, and allows for them, along with the $k$ copies of the value, to
be used in another term. The evaluation rule says that, in order to
evaluate $\tsfer {k'} k \ell x M N$, we first evaluate $M$ to a
\texttt{save} value, and then evaluate the substitution instance
$N[v_1/x]$. The $k'$ in \texttt{transfer} means to repeat the evaluation
of $M$ $k'$ times, allowing $k \cdot k'$ uses in the body of $N$, so
this (similarly to \texttt{save}) repeats the credit effects $r_1$ of
$M$ $k'$ times in the conclusion.  The other positive elimination forms are similar.  
  
% The natural number recursor \texttt{nrec} can be thought of as a
% call-by-value function constant of type $\N \to (1 \to C) \loli !^\infty_0(\N
% \times \susp C) \loli C$, so it can require some calculation to compute
% the branches to a function (unlike a typical recursor construct, where
% the branches are terms with free variables), and the first thing done in
% both rules is to evaluate $N_1$ and $N_2$ to values.
% %% This behavior simplifies cost analysis, and,
% %% because one nearly always passes such a function values, is in practice
% %% no different from the traditional recursor syntax which takes as
% %% arguments terms with free variables.
% %% The other nonstandard aspect of
% %% \texttt{nrec} is the fact that the base case is suspended
% %% (\textbf{better way to say this?}). This is due to technical details of
% %% how the bounding theorem (\autoref{thm:bounding}) is proved.
% The only other non-standard aspect is that recursive call in the
% successor case is delayed (the $\lambda \_. \ldots$), which is designed
% so that a case analysis that does not use the recursive call will have
% the correct running time.  The operational semantics for list recursion
% \texttt{lrec} is similar to \texttt{nrec}.

%% The first evaluation case for \texttt{nrec} is straightforward. When $M$
%% evaluates to $\elist$, we evaluate the remaining arguments, and then
%% evaluate the base case. But, when $M \downarrow^{(n_1,r_1)} S(v_1)$, the
%% remaining arguments are evaluated: $N_1 \downarrow^{(n_2,r_2)} \lambda
%% x.N_1'$, $N_2 \downarrow^{(n_3,r_3)} \lambda x.N_2'$. Next, the step
%% function, $\lambda x.N_2'$ is evaluated with two arguments. The first is
%% $v_1$, the ``predecessor" of $M$. The second is a delayed computation of
%% the recursive call on $v_1$. The result of this evaluation is returned
%% as the result of the whole recursor.

%% When $M \downarrow^{(n_1,r_1)} \cons{v_1}
%% {v_2}$, the remaining arguments are evaluated as usual: $N_1
%% \downarrow^{(n_2,r_2)} \lambda x.N_1'$ and $N_2 \downarrow^{(n_3,r_3)}
%% \save \infty 0 (\lambda x.N_2')$. The step function $\lambda x.N_2'$ is
%% then ``unboxed" from under the \texttt{save}, and passed two
%% arguments. The first is $v_1$, the head of the list. The second is a
%% ``negative pair" of the tail of the list, and the recursive call on the
%% tail-- this allows the step function to use either the tail or the
%% recursive call, sidestepping the credit multi-use issue discussed in
%% \autoref{sec:ns-rules}. The result of this function evaluation is
%% then returned as the final value.

\subsection{Syntactic Properties}

%% In order for $\lambda^A$ to actually express valid amortized analyses,
%% the amortized cost of a complete program must be larger than the actual
%% cost. While not yet a statement about the recurrences we intend to
%% extract, a theorem of this sort lends plausibility to the language we
%% have presented. The exact statement of the theorem takes inspiration
%% from CLRS \cite[equation 17.1]{CLRS} which states that for an amortized
%% analysis to be valid, the total amortized cost of a program must be an
%% upper bound on its total actual cost.
%% For $\lambda^A$, this entails proving that for all closed terms $\dot
%% \vdash_0 M : A$, when $M \downarrow^{(n,r)} v$, then $n + r \geq n$. Of
%% course, it suffices to show that $r \geq 0$. Instead, we strengthen the
%% induction hypotheses and prove the following theorem, which also serves
%% as the preservation theorem:

In the operational semantics judgment $M \downarrow^{(n,r)} v$, we
think of $n + r$ (the actual cost $n$ plus the credit difference $r$) as
the amortized cost of the program.  A key property of amortized analysis
is that the amortized cost is an upper bound on the true cost, which
means in this case that $n + r \ge n$, so we would like $r \ge 0$.
While $r$ is in general allowed to be a negative number, it is
controlled by the credits $a$ of the typing judgment $\cdot \vdash_a M
: A$, intuitively because it is only \discname\/ operations that
subtract from $r$, and \discname\/ operations are only allowed when the
type system deems there to be sufficient credits available.  Thus, we
will be able to prove that $r \ge 0$ for well-typed terms.  To do so,
we strengthen the induction hypotheses to prove that $\cdot \vdash_a M :
A$ and $M \downarrow^{(n,r)} v$ imply $a + r \geq 0$, which gives $r
\geq 0$ for closed programs that use no external credits (so $a = 0$),
which is what a ``main'' function is expected to be (e.g. \texttt{set}
in the binary counter example).  It is technically convenient to combine
this with a preservation result, stating that the credits of $v$ is in
fact $a + r$ (the resource term in a typing judgment must be
non-negative, so $a + r \geq 0$ is in fact a prerequisite for even
asserting that $\cdot \vdash_{a+r} v : A$).  The proofs of the following are
relatively straightforward and may be found
%\begin{icfp2020}in the full version of this
%paper~\citep{cutler-et-al:icfp2020-full}.\end{icfp2020}
\autoref{appendix:b}.%\end{arxiv}


\begin{restatable}[Preservation Bound]{theorem}{pres}
\label{thm:pres}
If $\cdot \vdash_a M : A$ and $M \downarrow^{(n,r)} v$, then $a + r \geq 0$ and $\cdot \vdash_{a + r} v : A$. 
\end{restatable}

We also have that values evaluate in 0 steps:
\begin{restatable}[]{theorem}{valevalzero}\label{thm:val-eval-none}
If $v$ is a value, and $v \downarrow^{(n,r)} v$, then $n = r = 0$.
\end{restatable}

and that values of type $\N$ contain no credits:
\begin{restatable}[Resource strengthening for $\N$]{theorem}{natstren}\label{thm:nat-strengthening}
If $\cdot \vdash_a v : \N$, then $\cdot \vdash_0 v : \N$
\end{restatable}



\subsection{Binary Counter Annotation}

\begin{figure}
  \begin{small}
  $$
\begin{array}{l}
\cdot \vdash_0 \texttt{inc} := \lambda b. \texttt{lrec}(b,\lambda \_. \tick {\cons {\wait 1 ({\inl ({\save 1 1 {()}})}) } {\elist} },\\
  \hspace{4em} \texttt{save}^{\infty}_0 (\lambda (a,tr). \texttt{case}_1(a,\_.{\tick {\cons {\wait 1 ({\inl ({\save 1 1 {()}})})} {\pi_1 tr} } },\\
      \hspace{15.75em} y. \texttt{transfer}_1 \; !^1_1 \_ = y \; \texttt{to} \\
       \hspace{17em} {\disc 1 {(\tick {\cons {\inl {()}} {\pi_2 tr}})}}))) : \listty {\texttt{bit}} \loli \listty {\texttt{bit}}
\\\\
\vdash_{0} \texttt{set} := \lambda n. \texttt{nrec}(n,\lambda \_. \elist,
  \texttt{save}^\infty_0(\lambda p.
   \asplit 1 {p} {\_} x {\texttt{inc} \; (x \; ())}
  )
) : \N \loli \listty{\texttt{bit}}
\end{array}
$$
\end{small}

  \vspace{-0.15in}
  \caption{Binary Counter Terms in $\lambda^A$}
  \label{fig:bc-term}
\end{figure}

As an example, we translate the binary counter program from
\autoref{sec:intro} to $\lambda^A$, decorating the program with
\createname, \spendname, \savename, and \xfername\/ in order to emulate the
analysis described in \autoref{sec:intro}.  Since the analysis
stores credits on 1 bits, the type of bits is $\texttt{bit} = 1 \oplus
!^1_1 1$; a value $\inl {(\,)}$ represents a $0$ bit, and a value $\inr
{(\save 1 1 {(\, )})}$ represents a $1$~bit, with a credit attached. A
binary number is represented as a list of bits, $\listty
{\texttt{bit}}$.
The cost of interest is the number of bit flips, so 
we insert $\texttt{tick}$s everywhere a bit is flipped from
$0$ to $1$ or vice versa. Next, to handle the credits, we
$\createname$ and subsequently $\texttt{save}$ a credit when we
flip a bit from $0$ to $1$, and $\xfername$ then $\spendname$
when flipping bits from $0$ to $1$.
This annotation is shown in \autoref{fig:bc-term} -- for simplicity, we use \texttt{inc} as a meta-level name for the term
implementing the function, so its occurrence in \texttt{set} really means
a copy of that entire term (to do this at the object level, we could
alternatively think of a top-level definition of \texttt{inc} as binding
an infinite-use variable).

%% Because $\lambda^A$ is affine, we must take care to format the
%% annotation in a way that is type correct and does not re-use
%% resources. This is mostly trivial given the way that $\lambda^A$'s type
%% system mirrors the analysis we are trying to capture-- but requires
%% special attention for helper functions like \texttt{inc} that will be
%% used more than once.  Formally, we present \texttt{inc} as a term $\cdot
%% \vdash_0 \texttt{inc} : \listty{\texttt{bit}} \loli
%% \listty{\texttt{bit}}$, and \texttt{set} as a term which has infinite
%% use of a variable $inc$ which has the same type as \texttt{inc}, ie:
%% $inc : \listty{\texttt{bit}} \loli \listty{\texttt{bit}} \vdash_{\infty
%%   \cdot inc} \texttt{set} : \N \loli \listty{\texttt{bit}}$. Since the
%% \texttt{inc} term is a value, $\save \infty 0 {(\texttt{inc})}$ costs 0
%% to evaluate, and so substituting it into $\texttt{set}$ with a mediating
%% \texttt{transfer} in between incurs no extra cost. The annotated
%% $\lambda^A$ terms for the binary counter are presented in
%% \autoref{fig:bc-term}.  

\section{Recurrence Language \texorpdfstring{$\lambda^\bbbc$}{}, Amortized Recurrence Extraction, and Bounding Theorem} \label{sec:cl}

Next, we define a translation from $\lambda^A$ into a \emph{recurrence
  language} $\lambda^{\bbbc}$. Unlike $\lambda^A$, $\lambda^{\bbbc}$ has
a fully structural (weakening and contraction) type system, and no
special constructs for amortized analysis (it is mostly unchanged from
\cite{danner-et-al:icfp15,hudson}). Further, because we view $\lambda^\bbbc$
as a syntatx for mathematical expressions, it is designed as a call-by-name language--
this is in contrast to $\lambda^A$, which is by-value.
The recurrence translation takes a function in
$\lambda^A$ to a function that outputs the original function's cost in
$\lambda^\bbbc$, using a cost type $\bbbc$ (which we will often
take to be integers).  Formally, $\bbbc$ can be any commutative ring
with an $\infty$ element, the typical example being the (``tropical'')
max-plus ring on the integers, i.e. integers with addition and binary
maxes.  Some of the typing rules for $\lambda^\bbbc$ are presented in
\autoref{fig:lc-rules}.

Relative to our previous work, the main conceptual change for supporting
amortized analysis is that, instead of extracting recurrences for the
true cost of a program ($n$ in $M \downarrow^{(n,r)} v$), we extract
recurrences that given an upper bound on the program's amortized cost $n
+ r$, which is itself a bound on the true cost for programs which begin
with an empty bank of credits.

\begin{figure}[h]
  \begin{small}
  \begin{mathpar}
\infer{
\Gamma, x : T \vdash x : T
}{}
\qquad
\infer{
\Gamma \vdash k : \bbbc
}{
k \in \mathbb{Z}
}
\qquad
\infer{
\Gamma \vdash E_1 + E_2 : \bbbc
}{
\Gamma \vdash E_1 : \bbbc
&
\Gamma \vdash E_2 : \bbbc
}
\qquad
\infer{
\Gamma \vdash () : 1
}{}

\\

\infer{
\Gamma \vdash E_1 \; E_2 : T_2
}{
\Gamma \vdash E_1 : T_1 \to T_2
&
\Gamma \vdash E_2 : T_1
}
\qquad
\infer{
\Gamma \vdash \lambda x.E : T_1 \to T_2
}{
\Gamma, x : T_1 \vdash E : T_2
}
\qquad
\infer{
\Gamma \vdash (E_1,E_2) : T_1 \times T_2
}{
\Gamma \vdash E_1 : T_1
&
\Gamma \vdash E_2 : T_2
}
\qquad
\infer{
\Gamma \vdash \pi_i E : T_i
}{
\Gamma \vdash E : T_1 \times T_2
}

\\

\infer{
\Gamma \vdash \inl E : T_1 + T_2
}{
\Gamma \vdash E : T_1
}
\qquad
\infer{
\Gamma \vdash \inr E : T_1 + T_2
}{
\Gamma \vdash E : T_2
}
\qquad
\infer{
\Gamma \vdash \ccase E x {E_1} y {E_2} : T
}{
\Gamma \vdash E : T_1 + T_2
&
\Gamma, x : T_1 \vdash E_1 : T
&
\Gamma, y : T_2 \vdash E_2 : T
}

\\

\infer{
\Gamma \vdash 0 : \N
}{}
\qquad
\infer{
\Gamma \vdash S(E) : \N
}{
\Gamma \vdash E : \N
}
\qquad
\infer{
\Gamma \vdash \nrec {E} {E_1} {E_2} : T
}{
\Gamma \vdash E : \N
&
\Gamma \vdash E_1 : 1 \to T
&
\Gamma \vdash E_2 : \N \times T \to T
}

\\

\infer{
\Gamma \vdash \elist : \listty T
}{}
\qquad
\infer{
\Gamma \vdash \cons {E_1} {E_2} : \listty T
}{
\Gamma \vdash E_1 : T
&
\Gamma \vdash E_2 : \listty T
}
\qquad
\infer{
\Gamma \vdash \lrec E {E_1} {E_2} : T
}{
  \begin{array}{l}
    \Gamma \vdash E : \listty {T_1}\\
\Gamma \vdash E_1 : 1 \to T\\
\Gamma \vdash E_2 : T_1\times(\listty {T_1} \times T) \to T
  \end{array}
}
\end{mathpar}
\end{small}

  \vspace{-0.25in}
  \caption{Recurrence Language $\lambda^\bbbc$ Definition}
  \label{fig:lc-rules}
\end{figure}

\subsection{Monadic Translation from \texorpdfstring{$\lambda^A$}{intermediate
language} to \texorpdfstring{$\lambda^\bbbc$}{recurrence language}}\label{ssec:mt}
\label{sec:monadic-translation}

Following~\cite{danner-et-al:plpv13,danner-et-al:icfp15}, a function $A
\loli B$ in $\lambda^A$ will be translated to a function \mbox{$\angles{A}
\to \bbbc \times \angles{B}$}, where for a $\lambda^A$ type $A$, a value of
$\lambda^\bbbc$ type $\angles{A}$ represents the size of a value in
$\lambda^A$.  Intuitively, this means that a function in $\lambda^A$ is
translated to a $\lambda^{\bbbc}$ function that, in terms of the size of the
input, gives the cost of running the function on that argument and the size
of the output.  Generalized to higher-type, ``size'' is properly viewed as
``use-cost;'' it is a property that tells us how the value affects the cost
of a computation that uses it.  In an unfortunate terminological clash,
prior work~\cite{danner-royer:ats-lmcs} refers to this concept as
\emph{potential} (as in ``potential cost'' or ``future cost''), with no
intentional connotation of potential functions from the physicist's method
of amortized analysis.  In order to keep this work consistent with the
sequence of papers it follows, and since $\lambda^A$ is based on the
banker's method, we will only use ``potential" to refer to the use-cost of a
value, and so call $\angles{A}$ the \emph{potential type} for~$A$ and a
value of type~$\angles A$ a \emph{potential}.  The size of the output is
needed for the translation to be compositional: the recurrence extracted for
a term should be composed of the recurrences extracted for its subterms, but
the cost of e.g.\ a function application depends on the size of the argument
itself, not just its cost.  A recurrence extraction of this form can be
packaged as a monadic translation into the writer monad $\bbbc \times A$.  

As discussed in \autoref{sec:intro}, the proper notion of size for a
specific datatype may vary from analysis to analysis. To this end, we
follow~\cite{danner-et-al:icfp15} in deferring the
abstraction of values as sizes to denotational semantics of
$\lambda^\bbbc$ defined in \autoref{sec:preorder}, which allows the
same recurrence extraction and bounding theorem to be reused for
multiple models with different notions of size.

We call the pair of a cost and a potential a \textit{complexity}.  The
translation consists of three separate functions, the definitions of
which are shown in \autoref{fig:rec-extr}. Firstly, $\angles{\cdot}$
takes a type $A$ in $\lambda^A$ and maps it to the type $\angles{A}$
whose elements are the potentials of type $A$. We extend this to contexts pointwise:
$\angles{\Gamma,x : A} = \angles{\Gamma},x:\angles{A}$.
The second is $\norm{A}
:= \bbbc \times \angles{A}$, which takes a type $A$ to the corresponding
type of complexities. Finally, we overload $\norm{\cdot}$ to denote the
recurrence extraction function from terms of $\lambda^A$ to terms in
$\lambda^\bbbc$.  For convenience, when $E : \bbbc \times T$, we often
write $\pi_1 E$ as $E_c$ (cost) and $\pi_2 E$ as $E_p$ (potential).
\footnote{We regard the subscript notation as binding tighter than
ordinary projection: i.e. $\pi_1E_p = \pi_1(E_p)$.
}
We
also use special notation for adding a cost to a complexity, writing
$E+_c E'$ for $(E + E'_c,E'_p)$ when $E : \bbbc$ and $E' : \bbbc \times
T$.

\begin{figure}
  % \begin{small}
% \[
% \begin{array}{l}
% \norm{A} = \bbbc \times \angles{A} \\
% \angles{!^k_\ell A} = \angles{A} \\
% \angles{1} = 1 \\
% \angles{A \otimes B} = \angles{A} \times \angles{B} \\
% \angles{A \oplus B} = \angles{A} + \angles{B} \\
% \angles{A \loli B} = \angles{A} \to \norm{B} \\
% \angles{A \amp B} = \norm{A} \times \norm{B} \\
% \angles{\N} = \N \\
% \angles{\listty A} = \listty {\angles A} \\
% \\
% \norm{x} = (0,x)\\
% \norm{()} = (0,())\\
% \norm{\tick M} = 1 +_c \norm{M}\\
% \norm{(M,N)} = (\norm{M}_c + \norm{N}_c,(\norm{M}_p,\norm{N}_p))\\
% \norm{\asplit {k'} M x y N} = k'\norm{M}_c +_c \norm{N}[\pi_1\norm{M}_p/x,\pi_2\norm{M}_p/y]\\
% \norm{\lambda x.M} = (0,\lambda x.\norm{M})\\
% \norm{M \; N} = (\norm{M}_c + \norm{N}_c) +_c \norm{M}_p \; \norm{N}_p\\
% \norm{\inl M} = (\norm{M}_c,\inl {\norm{M}_p})\\
% \norm{\inr M} = (\norm{M}_c,\inr {\norm{M}_p})\\
% \norm{\acase {k'} M x {N_1} y {N_2}} = k'\norm{M}_c + \ccase {\norm{M}_p} x {\norm{N_1}} y {\norm{N_2}}\\
% \norm{0} = (0,0)\\
% \norm{S(M)} = (\norm{M}_c,S(\norm{M}_p))\\
% \norm{\elist} = (0,\elist)\\
% \norm{\cons M N} = (\norm{M}_c + \norm{N}_c,\cons {\norm{M}_p} {\norm{N}_p})\\
% \norm{\tsfer {k'} k \ell x M N} = k'\norm{M}_c +_c \norm{N}[\norm{M}_p/x]\\
% \norm{\save k \ell M} = (k\norm{M}_c,\norm{M}_p)\\
% \norm{\wait \ell M} = \ell +_c \norm{M}\\
% \norm{\disc \ell M} = (-\ell) +_c \norm{M}\\
% \norm{\amppair M N} = (0,(\norm{M},\norm{N}))\\
% \norm{\pi_i M} = \norm{M}_c +_c \pi_i\left(\norm{M}_p\right)\\
% \norm{\nrec M {N_1} {N_2}} = (\norm{M}_c + \norm{N_1}_c + \norm{N_2}_c) +_c \nrec {\norm{M}_p} {\norm{N_1}_p} {\lambda x. \norm{N_2}_p(\pi_1 x,\lambda z.\pi_2 x)}\\
% \norm{\lrec M {N_1} {N_2}}=(\norm{M}_c+\norm{N_1}_c+\norm{N_2}_c)+_c\lrec {\norm{M}_p} {\norm{N_1}_p} {\lambda x. \norm{N_2}_p (\pi_1 x,((0,\pi_1 \pi_2 x),\pi_2\pi_2 x)) }\\
% \end{array}
% \]
% \end{small}

\begin{small}
\[
\begin{array}{c}
\norm{A} = \bbbc \times \angles{A} \\
\angles{1} = 1 \qquad
\angles{A \otimes B} = \angles{A} \times \angles{B}\qquad
\angles{A \oplus B} = \angles{A} + \angles{B}\qquad
\angles{A \loli B} = \angles{A} \to \norm{B} \\
\angles{!^k_\ell A} = \angles{A} \qquad
\angles{A \amp B} = \norm{A} \times \norm{B} \\
\angles{\N} = \N\qquad
\angles{\listty A} = \listty {\angles A} \\
\end{array}
\]

\[
\begin{array}{c}
\norm{x} = (0,x) \\
\norm{()} = (0,()) \qquad
\norm{(M,N)} = (\norm{M}_c + \norm{N}_c,(\norm{M}_p,\norm{N}_p)) \qquad
\norm{\pi_i M} = \norm{M}_c +_c \pi_i\left(\norm{M}_p\right)\\ \\
\norm{\inl M} = (\norm{M}_c,\inl {\norm{M}_p}) \qquad
\norm{\inr M} = (\norm{M}_c,\inr {\norm{M}_p})\\
\norm{\acase {k'} M x {N_1} y {N_2}} = k'\norm{M}_c + \ccase {\norm{M}_p} x {\norm{N_1}} y {\norm{N_2}}\\ \\
\norm{\lambda x.M} = (0,\lambda x.\norm{M}) \qquad
\norm{M \; N} = (\norm{M}_c + \norm{N}_c) +_c \norm{M}_p \; \norm{N}_p\\ \\
\norm{\amppair M N} = (0,(\norm{M},\norm{N}))\\
\norm{\asplit {k'} M x y N} = k'\norm{M}_c +_c \norm{N}[\pi_1\norm{M}_p/x,\pi_2\norm{M}_p/y]\\ \\
\norm{0} = (0,0) \qquad
\norm{S(M)} = (\norm{M}_c,S(\norm{M}_p)) \\ \\
\norm{\elist} = (0,\elist) \qquad
\norm{\cons M N} = (\norm{M}_c + \norm{N}_c,\cons {\norm{M}_p} {\norm{N}_p}) \\ \\
\norm{\tick M} = 1 +_c \norm{M}\\
\norm{\tsfer {k'} k \ell x M N} = k'\norm{M}_c +_c \norm{N}[\norm{M}_p/x] \qquad
\norm{\save k \ell M} = (k\norm{M}_c,\norm{M}_p)\\
\norm{\wait \ell M} = \ell +_c \norm{M} \qquad
\norm{\disc \ell M} = (-\ell) +_c \norm{M}\\ \\
\norm{\nrec M {N_1} {N_2}} = (\norm{M}_c + \norm{N_1}_c + \norm{N_2}_c) +_c \nrec {\norm{M}_p} {\norm{N_1}_p} {\lambda x. \norm{N_2}_p(\pi_1 x,\lambda z.\pi_2 x)}\\
\hspace{-10em}\norm{\lrec M {N_1} {N_2}}=(\norm{M}_c+\norm{N_1}_c+\norm{N_2}_c)+_c\\
\hspace{12em}\lrec {\norm{M}_p} {\norm{N_1}_p} {\lambda x. \norm{N_2}_p (\pi_1 x,((0,\pi_1 \pi_2 x),\pi_2\pi_2 x)) }\\
\end{array}
\]
\end{small}

  \caption{Recurrence Extraction}
  \label{fig:rec-extr}
\end{figure}

Overall, the idea is that a term is translated to a function from
potentials of its context to complexities of its type:
\begin{restatable}[Extraction Preserves Types]{theorem}{extrsound}\label{thm:extr-sound}
If $\Gamma \vdash_a M : A$ then $\angles{\Gamma} \vdash \norm M : \norm A$
\end{restatable}

We comment on some of the less obvious aspects of this translation:

\begin{itemize}
  \item $!^k_\ell A$: The type translation erases the $!^k_\ell$ modality. 
  
  \item $A \amp B$: Since the negative product in $\lambda^A$ is lazy, a
    value of type $A \amp B$ is a pair of un-evaluated terms. Thus, the
    potential of a term of type $A \amp B$ must include the cost of
    evaluating each term, since that will factor into the cost of
    using such a value.
  
  \item $\texttt{tick}$: Since $\tick M$ evaluates with (true cost and)
    amortized cost $1$ higher than $M$'s, the cost component of
    $\norm{\tick M}$ is $1 + \norm{M}_c$.

  \item $\texttt{save}^k_\ell$: The extracted amortized cost of $\save k
    \ell M$ is $k$ times the extracted cost of $M$, with the potential
    remaining the same.  This is in principle a non-exact bound, because we
    are conceptually multiplying the operational amortized cost of $M
    \downarrow^{(n,r)} v$, which is $n + r$, by $k$, whereas the
    operational semantics gives the more precise $n + k r$.  We view
    this as a consequence of the fact that amortized analyses extract
    recurrences for the amortized cost $n+r$, rather than $n$ and $r$
    separately. However, this inflation is not a
    problem for our uses of $!^\infty$ in typing recursors because the
    branches of the recursor are usually values, which have 0 cost, and
    $\infty \times 0 = 0$. In future work, we might consider a recurrence
    translation into the $\bbbc \times \bbbc \times A$ monad, with
    separate extractions of $n$ and $r$, if more precision is needed.
    This would allow for $\lambda^A$ to be used in the place of the
    (linear fragment) of the source language in previous 
    work~\cite{danner-et-al:icfp15}. Embedding that language into the $!^\infty$
    fragment of $\lambda^A$ and then extracting recurrences into 
    $\bbbc \times \bbbc \times A$ would yield the same results as
    applying the non-amortized recurrence extraction. We emphasize
    that the loss of precision from not making this change has no bearing
    on \textit{amortized} algorithm analyses, it would only
    allow for \textit{non-amortized} analyses to also be performed
    with $\lambda^A$-- but such analyses are already handled by prior work \cite{danner-et-al:icfp15,kavvos-et-al:popl20}
    
  \item $\texttt{transfer}$: A similar imprecision arises with respect
    to the multiplicity $k'$ here, but otherwise $\texttt{transfer}$ is
    translated like a \texttt{let}.  

%   \item $\waitname_\ell$: $\wait{\ell}(M)$ adds $\ell$ to the
%     extracted amortized cost recurrence.   
% 
%   \item $\discname_\ell$: $\disc \ell M$ subtracts $\ell$ from the
%     extracted amortized cost recurrence.
% 
  \item $\texttt{nrec}$: As in the operational semantics, because we
    think of the recursor as a call-by-value function constant, some
    cost is in principle incurred for evaluating the branches to
    function values, though the branches are usually values in practice.

  \item \sloppypar $\texttt{lrec}$: The type of the step function in a
    list recursor is $!^\infty_0(A \otimes (\listty A \amp C) \loli C)$,
    and the potential translation of this type is
    %$\angles{!^\infty_0(A \otimes (\listty A \amp C) \loli C)} =
    \mbox{$\angles{A} \times \left(\left(\bbbc \times \listty {\angles
      A}\right) \times \left(\bbbc \times \angles C\right)\right) \to
    \bbbc \times \angles C$}. However, this does not match the
    required type of the step function of the list recursor in
    $\lambda^\bbbc$, which must be $T_1 \times (\listty {T_1} \times
    T_2) \to T_2$.  Taking $T_1 = \angles{A}$ and $T_2 = \bbbc \times
    \angles C$, the translation of the step function additionally
    requires a $\bbbc$ input representing the cost of the tail of the
    list.  However, lists are eager, so the step function is always applied
    to a value, so we can supply $0$ cost here.
\end{itemize}


\subsection{Recurrence Language Inequality Judgment}\label{sec:so}

$\lambda^\bbbc$ has a syntactic inequality judgment $\Gamma
\vdash E_1 \leq_T E_2$ (\autoref{fig:syn-ord}), which intuitively means
that the recurrence $E_1$ is bounded above by $E_2$.  For now, we
include only those inequalities that are necessary to prove the
bounding theorem; this allows for the most models of the
recurrence language, and additional axioms valid in particular models
can be added in order to simplify recurrences syntactically.  The
necessary axioms are congruence in the principal positions of
elimination forms, as well as the fact that $\beta$-reducts are bounded
above by their redexes.  We often omit the context and type subscript
from $\Gamma \vdash E_1 \leq_T E_2$, writing $E_1 \leq_T E_2$ or $E_1
\leq E_2$, though formally it is a relation on well-typed terms in
context. This relation is primarily a technical device to provide closure
properties for the bounding relation. Because of this, we omit a more lengthy
discussion of the relation here, and refer the reader to the prior work
\cite{danner-et-al:icfp15} which introduces this type of relation.

\begin{figure}
  \begin{mathpar}

  \mathcal{C} ::= [] \enskip 
              | \enskip \pi_0 \mathcal{C} \enskip 
              | \enskip \pi_1 \mathcal{C} \enskip
              | \enskip \mathcal{C} \; E \enskip
              | \enskip \ccase {\mathcal{C}} x {E} y {E'} \enskip 
              | \enskip \nrec {\mathcal{C}} {E_1} {E_2} \enskip
              | \enskip \lrec {\mathcal{C}} {E_1} {E_2}

\infer{
\Gamma \vdash \mathcal{C}[E_0] \leq_T \mathcal{C}[E_1]
}{
\Gamma, x : T' \vdash \mathcal{C}[x] : T
&
\Gamma \vdash E_0 \leq_{T'} E_1
}

\infer{
\Gamma \vdash E \leq_T E
}{}

\infer{
\Gamma \vdash E_1 \leq_T E_3
}{
\Gamma \vdash E_1 \leq_T E_2
&
\Gamma \vdash E_2 \leq_T E_3
}

\infer{
\Gamma \vdash E_1[E/x] \leq_T \ccase {\inl E} x {E_1} y {E_2}
}{}

\infer{
\Gamma \vdash E_2[E/x] \leq_T \ccase {\inr E} x {E_1} y {E_2}
}{}

\infer{
\Gamma \vdash E[E'/x] \leq_T (\lambda x.E) \; E'
}{}

\infer{
\Gamma \vdash E_i \leq_{T_i} \pi_i(E_1,E_2)
}{}
\\
\infer{
\Gamma \vdash E_1 \; () \leq_T \nrec 0 {E_1} {E_2}
}{}

\infer{
\Gamma \vdash E_2 \; (E,\nrec E {E_1} {E_2}) \leq_T \nrec {S(E)} {E_1} {E_2}
}{}
\\
\infer{
\Gamma \vdash E_1 \; () \leq_T \lrec {\elist} {E_1} {E_2}
}{}

\infer{
\Gamma \vdash E_2 \; (E,(E', \lrec {E'} {E_1} {E_2})) \leq_T \lrec {\cons {E} {E'}} {E_1} {E_2}
}{}

\end{mathpar}

  \vspace{-0.25in}
  \caption{Syntactic Ordering on $\lambda^\bbbc$}
  \label{fig:syn-ord}
\end{figure}

\subsection{Bounding Relation and Its Closure Properties}

The correctness of the recurrence extraction is stated in terms of a
logical relation between terms in $\lambda^A$ and terms in
$\lambda^\bbbc$. The intended meaning is that the $\lambda^\bbbc$
recurrence term is an upper bound on the $\lambda^A$ term's cost and
potential.

\begin{definition}[Bounding Relation] \label{def:bounding}
When $\cdot \vdash_a M : A$ and $\cdot \vdash E : \norm{A}$, then $M \bdby^{A,a} E$ if and only if, when $M \downarrow^{(n,r)} v$,
\begin{itemize}
  \item $n \leq E_c - r$
  \item $v \valbd^{A,a+r} E_p$
\end{itemize}
When $\cdot \vdash_a v : A$ and $\cdot \vdash E : \angles A$, we define $v
\valbd^{A,a} E$ by induction on $A$.
\begin{itemize}
    \item $\save k \ell v \valbd^{!^k_\ell A,c} E$ if there exists $d \geq 0$ so that $kd + \ell \leq c$, and $v \valbd^{A,d} E$
    \item $\lambda x.M \valbd^{A \loli B, c} E$ if whenever $v \valbd^{A,d} E'$, we have that $M[v/x] \bdby^{B,c+d} E \; E'$
    \item $(v_1,v_2) \valbd^{A_1 \otimes A_2,a} E$ if there are $a_1,a_2$ such that $a_1+a_2 = a$ and $v_i \valbd^{A_i,a_i} \pi_i E$ for $i \in \{1,2\}$
    \item $[] \valbd^{\listty A,a} E$ iff $[] \leq_{\listty {\angles A}} E$
    \item $\cons {v_1} {v_2} \valbd^{\listty A,a} E$ iff there are $E_1,E_2$ with $\cons {E_1} {E_2} \leq_{\listty {\angles A}} E$, and there are $a_1,a_2$ such that $a_1 + a_2 = a$ such that $v_1 \valbd^{A,a_1} E_1$ and $v_2 \valbd^{\listty A, a_2} E_2$.
    \item $0 \valbd^{\N,a} E$ iff $0 \leq E$
    \item $S(v) \valbd^{\N,a} E$ iff there is some $E'$ such that $S(E') \leq_\N E$, and $v \valbd^{\N,a} E'$
    \item $\inl v \valbd^{A \oplus B,a} E$ if there exists $E'$ such that $\inl E' \leq_{\angles A} E$ and $v \valbd^{A,a} E'$.
    \item $\inr v \valbd^{A \oplus B,a} E$ if there exists $E'$ such that $\inr E' \leq_{\angles B} E$ and $v \valbd^{B,a} E'$.
    \item $() \valbd^{1,a} E$ if $() \leq_1 E$.
    \item $\amppair M N \valbd^{A \amp B,a} E$ if $M \bdby^{A,a} \pi_1 E$, and $N \bdby^{B,a} \pi_2 E$.
\end{itemize}
We extend the value bounding relation to substitutions pointwise: $\theta \subbd^{\Gamma,\sigma} \Theta$ if for all $x : A \in \Gamma$, $\theta(x) \valbd^{A,\sigma(x)} \Theta(x)$. Finally, we define the bounding relation for open terms: when $\Gamma \vdash_f M : A$, we say that $M \bdby E$ if for all $\theta \subbd^{\Gamma,\sigma} \Theta$, we have $M[\theta] \bdby^{A,f[\sigma]} E[\Theta]$.
\end{definition}

The \emph{term/expression bounding relation} $M \bdby^{A,a} E$ says
first that the cost component of $E$ is an upper bound on the amortized
cost of $M$, which is $n + r \leq E_c$ (since we will eventually be
interested in bounding the actual cost of evaluating $M$, we write this
as $n \leq E_c - r$).  Additionally, expression bounding says that the
potential component of $E$ is an ``upper bound'' on the value that $M$
evaluates to; this is expressed via a mutually-defined type-varying
\emph{value bounding relation} $M \valbd^{A,a} E$.  The value bounding
relation is defined first by induction on the type $A$, and the cases
for natural numbers and lists have a local induction on the number/list
value as well.\footnote{In general, it is necessary to define the
  relations for inductive types inductively~\cite{danner-et-al:icfp15},
  but the values of $\N$ and $\listty{A}$ are simple enough that
  induction on values suffices here.}  We write the credit bank $a$ as a
parameter of the bounding relations, but it is a presupposition that
this number is the same one that was used to type check $\cdot \vdash_a
\{M,v\} : A$ (because the bounding relation is on closed terms, the resource
subscript is just a single number $a$).  
% For example, in the case for
% $\texttt{save}$, we essentially would like that $v \valbd^{A,a} E$
% implies $\save k \ell v \valbd^{!^k_\ell A,ka + \ell} E$. But, to define
% $\save k \ell v \valbd^{A,c} E$ in general, we require that there is a
% $d \geq 0$ so that $v \valbd^{A,d} E$ and $c \geq k d + \ell$.  The
% remaining cases of the definition modify $a$ in the same way as the
% typing rules.
%% Thus, $\save 1 3 v$ should not be bounded by some $E$
%% in a context with only $2$ credits.


%% It is important to note that the definitions of the value and expression
%% bounding logical relations are mutually recursive: a function value
%% $\lambda x.M$ is value bounded by a recurrence term $E$ if, for all
%% values $v$ bounded by $E'$, the substitution instance $M[v/x]$ is
%% \textit{expression} bounded by $E \; E'$, with suitable credits
%% everywhere.

We extend the bounding relation to open terms by considering all closing
substitutions: a term $\Gamma \vdash_f M : A$ is bounded by $E$ if for
every substitution $\theta$ which is bounded pointwise by $\Theta$ with
some credit function $\sigma$, then the closed term $M[\theta]$ is
bounded by $E[\Theta]$ with $f[\sigma]$ credits.  In this definition,
$\sigma$ gives a number of credits $a_i$ for each variable $x_i$,
because $\theta$ is a substitution of closed terms for variables $(\cdot
\vdash_{a_1} v_1 : A_1) / x_1, (\cdot \vdash_{a_2} v_2 : A_2) / x_2,
\ldots$.

\subsection{Bounding Theorem}

As usual for a logical relation, we first require some lemmas about the
bounding relation, before a main loop proving the fundamental theorem
that terms are related to their extractions.  The proofs of the following
theorems can be found
%\begin{icfp2020}in the full version of this
%paper~\citep{cutler-et-al:icfp2020-full}.\end{icfp2020}
%\begin{arxiv}
in \autoref{appendix:b}.%\end{arxiv}
% \processifversion{arxiv}{in \autoref{sec:appendix}.}
% \processifversion{icfp2020}{in the full version of this
% paper~\citep{cutler-et-al:icfp2020-full}.}

First, we have an analogue of \autoref{thm:nat-strengthening}:

\begin{theorem}[$\N$-strengthening]
For all $\cdot \vdash_a v : \N$, if $v \valbd^{\N,a} E$, then $v \valbd^{\N,0} E$.
\end{theorem}

Second, we can weaken a bound by recurrence language inequality:

\begin{restatable}[Weakening]{theorem}{weakening} \hfill
\label{thm:weakening}
\begin{enumerate}
    \item If $M \bdby^{A,a} E$, and $E \leq_{\norm{A}} E'$, then $M \bdby^{A,a} E'$
    \item If $v \valbd^{A,a} E$, and $E \leq_{\llangle A \rrangle} E'$, then $v \valbd^{A,a} E'$
\end{enumerate}
\end{restatable}
%% PROOF
%% \weakening*
\begin{proof}
We prove 1 and 2 simultaneously by induction on $A$.
\begin{enumerate}
  \item Suppose $M \bdby^{A,a} E$, and $E \leq_{\mathbb{C} \times \angles{A}} E'$. We need to show that $M \bdby^{A,a} E'$. Suppose $M \downarrow^{(n,r)} v$.
  We need to show:
  \begin{itemize}
    \item $n \leq E'_c - r$
    \item $v \valbd^{A,a+r} E'_p$
  \end{itemize}
  But, since $M \bdby^{A,a} E$
  \begin{itemize}
    \item $n \leq E_c - r$
    \item $v \valbd^{A,a+r} E_p$
  \end{itemize} 
  so, it suffices to show that $E_c \leq_{\mathbb{C}} E'_c$ and $E_p \leq_{\angles A} E'_p$, which is true by the $\pi_1(-)$ and $\pi_2(-)$ congruences, recalling that $(-)_c$ and $(-)_p$ are simply $\pi_1$ and $\pi_2$.
  \item Let $E \leq_{\angles A} E'$. We have a few cases to consider.
  \begin{itemize}
    \item[($!$)] Suppose $\save k l v \valbd^{!^k_l A,a} E$. We must show that $\save k l v \valbd^{!^k_l A,a} E'$. We know that there is a $d \geq 0$ such that $ka+l \leq d$, and $v \valbd^{A,d} E$. So, by IH, $v \valbd^{A,d} E'$, and hence $\save k l v \valbd^{!^k_l A,a} E'$, as required.
    \item[($\loli$)] Suppose $\lambda x.M \valbd^{A\loli B,a} E$. We need to show that $\lambda x.M \valbd^{A\loli B,a} E'$. Let $v \valbd^{A,b} E_v$. Then, $M[v/x] \bdby^{B,a+b} E \; E_v$. Using the application congruence and 1, $M[v/x] \bdby^{B,a+b} E' \; E_v$. Since $v,b,E_v$ were chosen arbitrarily, $\lambda x.M \valbd^{A \loli B,a} E'$ as required.
    \item[($\tensor$)] Suppose $(v_1,v_2) \valbd^{A_1\otimes A_2,a} E$. Then, there are $a_1,a_2$ such that $a_1 + a_2 = a$, and $v_i \valbd^{A_i,a_i} \pi_i E$, and so by $\pi_i$-congruence and the IH, $v_i \valbd^{A_i,a_i} \pi_i E'$, so $(v_1,v_2) \valbd^{A_1\otimes A_2,a} E'$, as required.
    \item[($\listty{A}$)] Both cases are immediate by transitivity.
    \item[($\N$)] Both cases are immediate by transitivity.
    \item[($\oplus$)] Both cases are immediate by transitivity.
    \item[($A \amp B$)] Suppose $\amppair {M_1} {M_2} \valbd^{A_1 \amp A_2,a} E$. Then, for $i \in \{1,2\}$, $M_i \bdby^{A_i,a} \pi_i E$. By $\pi_i$-congruence, $\pi_i E \leq_{\norm{A}} \pi_i E'$, and so by IH from 1, we know that $M_i \bdby^{A_i,a} \pi_i E'$, and are done.
  \end{itemize}
\end{enumerate}
\end{proof}


Next, we have an analogue of resource weakening in
\autoref{thm:la-structural}:

\begin{restatable}[Credit Weakening]{theorem}{credwkn}
If $a_1 \leq a_2$, then:
\begin{enumerate}
  \item[(1)] If $M \bdby^{A,a_1} E$, then $M \bdby^{A,a_2} E$
  \item[(2)] If $v \valbd^{A,a_1} E$, then $v \valbd^{A,a_2} E$ 
\end{enumerate}
\end{restatable}

Next, we have inductive lemmas that will be used in the recursor cases
of the fundamental theorem:

\begin{restatable}[$\N$-Recursor]{theorem}{nreclemma}
\label{thm:nrec-lemma}
If $\lambda x.N_1' \valbd^{1 \loli C,c_3} E_1$, $\lambda x.N_2' \valbd^{\N
\otimes (1 \loli C) \loli C,d} E_2$ with $d \geq 0$, then $\forall n \geq 0$, if $\inj n \valbd^{\N,0} E$, then $\nrec {\inj n} {\lambda x.N_1'} {\save \infty 0 (\lambda x.N_2')} \bdby^{C,c_3 + \infty \cdot d} \nrec E {E_1} {\lambda p. E_2 \; (\pi_1 p, \lambda z.\pi_2 p)}$
\end{restatable}

\begin{restatable}[$\listty A$-Recursor]{theorem}{lreclemma}
\label{thm:lrec-lemma}
If $\lambda x.N_1' \valbd^{1 \loli C,c_1} E_1$ and $\lambda x.N_2'
\valbd^{A \otimes (\listty A \amp C) \loli C,c_2} E_2$, then for all
values $\cdot \vdash_d v : \listty A$ such that $v \valbd^{\listty A,d}
E$, we have that\\
$\lrec v {\lambda x.N_1'} {\save \infty 0 {(\lambda x.N_2')}} \bdby^{C,c_1+d + \infty \cdot c_2} \lrec E {E_1} {\lambda x. E_2 (\pi_1 x,((0,\pi_1 \pi_2 x),\pi_2\pi_2 x)) }$
\end{restatable}
%% PROOF
%% \lreclemma*
\begin{proof}
We proceed by induction on the derivation of $\cdot \vdash_d v : \listty A$.
First, suppose $v = []$.  To show that $\lrec {[]} {\lambda x.N_1'} {\save \infty 0 {(\lambda x.N_2')}} \bdby^{C,c_1+d+\infty \cdot c_2} \lrec E {E_1} {\lambda x. E_2 (\pi_1 x,((0,\pi_1 \pi_2 x),\pi_2\pi_2 x))}$, assume that  $\lrec {[]} {\lambda x.N_1'} {\save \infty 0 {(\lambda x.N_2')}} \downarrow^{(n,r)} v$. By inversion, it was by way of $N_1'[()/x] \downarrow^{(n,r)} v$. It suffices to show 
\begin{itemize}
   \item $n \leq \lrec E {E_1} {\lambda x. E_2 (\pi_1 x,((0,\pi_1 \pi_2 x),\pi_2\pi_2 x))}_c - r$
   \item $v\ \valbd^{C,c_1+d+\infty \cdot c_2 + r} \lrec E {E_1} {\lambda x. E_2 (\pi_1 x,((0,\pi_1 \pi_2 x),\pi_2\pi_2 x)) }_p$
\end{itemize}
Since $() \leq_1 ()$, $() \valbd^{1,d} ()$, so $N_1'[()/x] \bdby^{C,c_1+d} E_1 \; ()$, and so
\begin{itemize}
  \item $n \leq (E_1 \; ())_c - r$
  \item $v \valbd^{C,c_1+d+r} (E_1 \; ())_p$
\end{itemize}
But, $\infty \cdot c_2 > 0$ since $c_2 > 0$, and so by credit weakening, $v \valbd^{C,c_1+d+\infty \cdot c_2 +r} (E_1 \; ())_p$. Note that, by assumption, $[] \valbd^{1,d} E$, which means that $[] \leq_{\listty {\angles A}} E$. So,
$$
E_1 \; () \leq \lrec {[]} {E_1} {\lambda x. E_2 (\pi_1 x,((0,\pi_1 \pi_2 x),\pi_2\pi_2 x)) } \leq \lrec E {E_1} {\lambda x. E_2 (\pi_1 x,((0,\pi_1 \pi_2 x),\pi_2\pi_2 x)) }
$$
and so we are done by weakening.

Otherwise, suppose $v = \cons {v_1} {v_2}$. To show that $\lrec {\cons {v_1} {v_2}} {\lambda x.N_1'} {\save \infty 0 {(\lambda x.N_2')}} \bdby^{C,c_1+d+\infty \cdot c_2} \lrec E {E_1} {\lambda x. E_2 (\pi_1 x,((0,\pi_1 \pi_2 x),\pi_2\pi_2 x)) }$, suppose $\lrec {\cons {v_1} {v_2}} {\lambda x.N_1'} {\save \infty 0 {(\lambda x.N_2')}} \downarrow^{(n,r)} v$. By inversion, it was by $N_2'[(v_1,\amppair {v_2} {\lrec {v_2} {\lambda x.N_1'} {\save \infty 0 {(\lambda x.N_2')}}})/x] \downarrow^{(n,r)} v$. It suffices to show:
\begin{itemize}
  \item $n \leq \lrec E {E_1} {\lambda x. E_2 (\pi_1 x,((0,\pi_1 \pi_2 x),\pi_2\pi_2 x)) }_c - r$
  \item $v \valbd^{C,c_1+d+\infty \cdot c_2 + r} \lrec E {E_1} {\lambda x. E_2 (\pi_1 x,((0,\pi_1 \pi_2 x),\pi_2\pi_2 x)) }_p$
\end{itemize}

Since $\cons {v_1} {v_2} \valbd^{\listty A,d} E$, there are $d_1,d_2 \geq 0$ such that $d_1 + d_2 = d$, along with $E'$,$E''$ such that $v_1 \valbd^{A,d_1} E'$ and $v_2 \valbd^{\listty A,d_2} E''$, and $\cons {E'} {E''} \leq E$

By IH, $\lrec {v_2} {\lambda x.N_1'} {\save \infty 0 {(\lambda x.N_2')}} \bdby^{C,c_1 + d_2 + \infty \cdot c_2} \lrec {E''} {E_1} {\ldots}$. 
Since $v_2 \valbd^{\listty A,d_2} E''$, $v_2 \bdby^{\listty A,d_2} (0,E'')$, 
and since $c_1 + \infty \cdot c_2 \geq 0$, we have by credit weakening that $v_2 \bdby^{\listty A,c_1 + d_2 + \infty \cdot c_2} (0,E'')$. 
So, 
$\amppair {v_2} {\lrec {v_2} {\lambda x.N_1'} {\save \infty 0 {(\lambda x.N_2')}}} \valbd^{\listty A \amp C,c_1 + d_2 + \infty \cdot c_2} ((0,E''),\lrec {E''} {E_1} {\ldots})$. 
Further, using the fact that $d_1 + d_2 = d$, 

$$
\begin{array}{l}
(v_1,\amppair {v_2} {\lrec {v_2} {\lambda x.N_1'} {\save \infty 0 {(\lambda x.N_2')}}}) \valbd^{A \otimes (\listty A \amp C), c_1 + d + \infty \cdot c_2}\\ (E',((0,E''),\lrec {E''} {E_1} {\ldots}))
\end{array}
$$
Thus, since $\lambda x.N_2' \valbd^{A \otimes (\listty A \amp C) \loli C,c_2} E_2$, we have (using the fact that $c_2 + \infty \cdot c_2 = \infty \cdot c_2$)
$$
N_2'[(v_1,\amppair {v_2} {\lrec {v_2} {\lambda x.N_1'} {\save \infty 0 {(\lambda x.N_2')}}})/x] \bdby^{C,c_1+d+\infty \cdot c_2} E_2 \; (E',((0,E''),\lrec {E''} {E_1} {\ldots}))
$$
By definition, this means that
\begin{itemize}
  \item $n \leq (E_2 \; (E',((0,E''),\lrec {E''} {E_1} {\ldots})))_c - r$
  \item $v \valbd^{c_1+d+\infty \cdot c_2 + r} (E_2 \; (E',((0,E''),\lrec {E''} {E_1} {\ldots})))_p$
\end{itemize}
We then compute:
\begin{align*}
&E_2 \; (E',((0,E''),\lrec {E''} {E_1} {\ldots}))\\
&\leq (\lambda x. E_2 (\pi_1 x,((0,\pi_1 \pi_2 x),\pi_2\pi_2 x))) \; (E',(E'',\lrec {E''} {E_1} {\lambda x. E_2 (\pi_1 x,((0,\pi_1 \pi_2 x),\pi_2\pi_2 x))}))\\
&\leq \lrec {\cons {E'} {E''}} {E_1} {\lambda x. E_2 (\pi_1 x,((0,\pi_1 \pi_2 x),\pi_2\pi_2 x))s}\\
&\leq \lrec {E} {E_1} {\lambda x. E_2 (\pi_1 x,((0,\pi_1 \pi_2 x),\pi_2\pi_2 x)) }
\end{align*}

and hence we are done by weakening.
\end{proof}


Using these, we prove the main result:

\begin{restatable}[Bounding Theorem]{theorem}{bounding}
\label{thm:bounding}
If $\Gamma \vdash_f M : A$, then $M \bdby^A \norm{M}$
\end{restatable}

Finally, for terms that use no external credits, the true cost is
bounded by the extracted recurrence: 

\begin{corollary}[True cost bounding] \label{cor:true-cost}
If $\cdot \vdash_0 M : A$ and $M \downarrow^{(n,r)} v$ then $n \le
\norm{M}_c$.
\end{corollary}
\begin{proof}
By \autoref{thm:bounding}, we have $n \le \norm{M}_c - r$, but by
preservation~(\autoref{thm:pres}), we have that $0 + r \ge 0$, so
$n \le \norm{M}_c$.  
\end{proof}

\subsection{Binary Counter Recurrences}

\begin{figure}
  $$
\begin{array}{l}
\cdot \vdash \norm{\texttt{inc}} := (0,\lambda bs. \texttt{lrec}(bs,\lambda\_. (2,\cons {(\inr {()})} {\elist}),\\
\hspace{9em} \lambda p. (\lambda x. \texttt{case}(\pi_1 x,\\
    \hspace{16em}\_. {(2 + (\pi_1 \pi_2 x)_c,\cons {(\inr {()})} {(\pi_1\pi_2 x)_p})} \\
    \hspace{16em}\_. {((\pi_2\pi_2 x)_c,\cons {(\inl {()})} {(\pi_2\pi_2 x)_p})})\\
 \hspace{10em})(\pi_1 p, ((0,\pi_1 \pi_2 p),\pi_2 \pi_2 p)))) : \bbbc \times (\listty{1 + 1} \to \bbbc \times \listty{1 + 1}
\\\\
\cdot \vdash \norm{\texttt{set}} := (0,\lambda n. \texttt{nrec}(n,\lambda \_. (0,\elist),\lambda u.
{(0,\lambda p. (\pi_2 p \; ())_c +_c \norm{\texttt{inc}}_p (\pi_2 p \; ())_p)_p}\\
\hspace{17em}(\pi_1 u,\lambda \_. \pi_2 u)
)) : \bbbc \times (\N \to \bbbc \times \listty{1 + 1})
\end{array}
$$

  \vspace{-0.25in}
  \caption{Binary Counter Recurrences in $\lambda^\bbbc$}
  \label{fig:bc-rec}
\end{figure}

As an example, the binary counter program in $\lambda^A$
(\autoref{fig:bc-term}) is translated by the recurrence extraction
translation to the terms in \autoref{fig:bc-rec}.
Next, we will use a denotational semantics of the recurrence language to
simplify these recurrences to the desired closed form.

\section{Recurrence Language Semantics} \label{sec:preorder}

The final step of our technique is to simplify recurrences to closed
forms.  This can be done semantically, in a denotational model of the
recurrence languages, or syntactically, by adding axioms to the
inequality judgment $\Gamma \vdash E \le_T E'$ corresponding to
properties true in a particular model.  Here, we will work in a
denotational model of $\lambda^\bbbc$ in preorders, which mostly follows
previous work~\cite{danner-et-al:plpv13,danner-et-al:icfp15,hudson}.
%% particular \cite{hudson}, where all of the
%% relevant theorems are formally verified in Agda.

\subsection{Semantic Interpretation}

We describe the semantic interpretation of $\lambda^\bbbc$ in preorders
here, and highlight the differences from \cite{hudson}, which gives a
similar presentation with mechanized proofs.

The semantics of types and terms is given in
\autoref{fig:sem-interp}, omitting function and product types, which are interpreted using the standard cartesian product and exponential objects of preorders.  For each type $A$ of $\lambda^\bbbc$, we
associate a partially ordered set $\scott{A}$ equipped with a top
element ($\infty$) and binary maximums ($\vee$) for which the top
element is an annihilator.
%% We sometimes regard a poset as a ``thin''
%% category, where the objects are the elements of the poset, and each
%% $\Hom_{\scott{A}}(x,y)$ is inhabited by a singleton if $x \leq y$ in
%% $\scott{A}$.
We write $1$ for the one-element poset, and $\N \cup \infty$ for the
natural numbers with an infinite element added, with the usual $0 \le 1
\le 2 \le \ldots \le \infty$ total order, and $\mathbb{Z} \cup \infty$
for the integers with an infinite element added, with the usual total
order.  We write $P \times Q$ for the cartesian product of posets with
the pointwise order, and $Q^P$ for the poset of monotone functions from
$P$ to $Q$, ordered pointwise; these have binary maxes and top elements
given pointwise.  We write $P + Q /\mathord\sim$ for the ``coalesced'' sum,
which first takes the disjoint union of $P$ and $Q$, with only
$\texttt{inl}(x) \le \texttt{inl}(y)$ if $x \le_P y$ and similarly for
\texttt{inr}, and then equates $\texttt{inl}(\infty_P)$ and
$\texttt{inr}(\infty_Q)$ to create a top element $\infty_{P+Q/\mathord\sim}$;
binary maxes are defined using maxes in $P$ and $Q$ for two elements
whose injections match, and to be $\infty$ otherwise.  The translation
on types is extended to contexts: $\scott{\cdot} = 1$,
$\scott{\Gamma,x:A} = \scott{\Gamma} \times \scott{A}$. Finally, we
interpret terms of $\lambda^\bbbc$ as \textit{monotone} (but not
necessarily infinity- or max-preserving) maps\footnote{ We write the
  composition of maps $f : A \to B$ and $g : B \to C$ in diagrammatic
  order, $f ; g : A \to C$.  } from the interpretation of their contexts
into the interpretation of their types. These maps are morphisms in the category
\textbf{Poset} of partially ordered sets and monotone maps, and
so we write them as elements of $\Hom_{\textbf{Poset}}(A,B)$, the
set of monotone maps between posets $A$ and $B$.

\begin{figure}
  \begin{small}
\[
\begin{array}{l}
\scott{\bbbc} = \mathbb{Z} \cup \{\infty\} \\
\scott{\N} = \N \cup \{\infty\}\\
\scott{\listty T} = \N \cup \{\infty\}\\
%% \scott{T_1 \times T_2} = \scott{T_1} \times \scott{T_2}\\
\scott{T_1 + T_2} = \left(\scott{T_1} + \scott{T_2}\right)/\sim \text{where } \inl{\infty} \sim \inr{\infty} \\
%% \scott{T_1 \to T_2} = \scott{T_2}^{\scott{T_1}}\\
\\
\scott{\Gamma,x:T, \Gamma' \vdash x : T} = \pi_1^k ; \pi_2 \text{ where } \left|\Gamma'\right| = k\\
\\
\scott{\Gamma \vdash k : \bbbc} = \const k\\
\scott{\Gamma \vdash E_1 + E_2 : \bbbc} = (\scott{\Gamma \vdash E_1 : \bbbc},\scott{\Gamma \vdash E_2 : \bbbc}) ; +\\
\\
\scott{\Gamma \vdash () : 1} = \const {()}\\
\\

\scott{\Gamma \vdash \inl E : T_1 + T_2} = \scott{\Gamma \vdash E : T_1} ; \texttt{inl}\\

\scott{\Gamma \vdash \inr E : E_1 + E_2} = \scott{\Gamma \vdash E : E_2} ; \texttt{inr}\\
\scott{\Gamma \vdash \ccase {E} x {E_1} y {E_2} : T} = \left(1_{\scott{\Gamma}},\scott{\Gamma \vdash E : T_1 + T_2}\right) ; \scase(\scott{\Gamma,x : T_1 \vdash E_1 : T},\scott{\Gamma, y : T_2 \vdash E_2 : T})\\
\texttt{scase} \in \Hom_{Poset}\left(C^{G \times A}\times C^{G \times B},C^{G\times(A + B)}\right)\\
\scase(f,g)(\gamma,\inl a) = f(\gamma,a) \vee g(\gamma,\infty)\\
\scase(f,g)(\gamma,\inr b) = f(\gamma,\infty) \vee g(\gamma,b)\\
\\

%% \scott{\Gamma \vdash \lambda x.E : T_1 \to T_2} = \curry{\scott{\Gamma,x : T_1 \vdash E : T_2}}\\
%% \scott{\Gamma \vdash E_1 E_2 : T_2} = (\scott{\Gamma \vdash E_1 : T_1 \to T_2},\scott{\Gamma \vdash E_2 : T_1});\texttt{ev}
%% \\
%% \\

%% \scott{\Gamma \vdash (E_1,E_2) : T_1 \times T_2} = (\scott{\Gamma \vdash E_1 : T_1},\scott{\Gamma \vdash E_2 : T_2})\\
%% \scott{\Gamma \vdash \pi_i E : T_i} = \scott{\Gamma \vdash E : T_1 \times T_2} ; \pi_i\\
%% \\

\scott{\Gamma \vdash 0 : \N} = \const 0\\
\scott{\Gamma \vdash S(M) : \N} = \scott{\Gamma \vdash M : \N} ; S\\
\scott{\Gamma \vdash \nrec {E} {E_1} {E_2} : T} = \left(1_{\scott{\Gamma}}, \scott{\Gamma \vdash E : \N}\right) ; \snrec(\scott{\Gamma \vdash E_1 : 1 \to T},\scott{\Gamma \vdash E_2 : \N\times T \to T})\\
\texttt{snrec} \in \Hom_{Poset}\left({\left(C^1\right)}^G\times {\left(C^{\N\times C}\right)}^G,C^{G\times\N}\right)\\
\snrec(f,g)(\gamma,0) = f(\gamma)()\\
\snrec(f,g)(\gamma,n+1) = g(\gamma)(n,\snrec(f,g)(\gamma,n)) \vee f(\gamma)()\\

\\
\scott{\Gamma \vdash \elist : \listty A} = \const 0\\
\scott{\Gamma \vdash \cons {E_1} {E_2} : \listty A} = \scott{\Gamma \vdash E_2 : \listty A} ; S\\
\scott{\Gamma \vdash \lrec E {E_1} {E_2} : T} = \left(1_{\scott{\Gamma}}, \scott{\Gamma \vdash E : \listty T'}\right) ; \\
\hspace{11em}\slrec(\scott{\Gamma \vdash E_1 : 1 \to T},\scott{\Gamma \vdash E_2 : T' \times (\listty T' \times T) \to T})\\
\texttt{slrec} \in  \Hom_{Poset}\left({\left(C^1\right)^G} \times {\left(C^{A \times (\N \times C)}\right)^G},C^{G \times \N}\right)\\
\slrec(f,g)(\gamma,0) = f(\gamma)()\\
\slrec(f,g)(\gamma,n+1) = g(\gamma)(\infty,(n,\slrec(f,g)(\gamma,n))) \vee f(\gamma)()\\

\end{array}
\]
\end{small}

  \caption{Semantic Interpretation Definition}
  \label{fig:sem-interp}
\end{figure}

In \autoref{fig:sem-interp}, we show some representative cases of the
interpretation of terms for sums, natural numbers and lists.  For costs,
the interpretation of cost constants and addition uses the elements and
addition of $\mathbb{Z} \cup \infty$.
% First, we consider the introduction forms for positive types.  For sums,
% the coalesced sum $P + Q / \sim$ has monotone injection functions from
% both $P$ and $Q$.  
In this model, we interpret both natural numbers and
lists as $\N \cup \infty$; for lists, this interprets a list as its
length.  $\N \cup \infty$ has a 0 element and a monotone successor
function $S$, where $S(\infty) = \infty$; these are used to interpret
0/the empty list and successor/cons.
The elimination forms for positives are more complex, and use some
auxiliary monotone functions (which are the morphisms in the category of
posets):
\begin{restatable}{theorem}{auxsemlemma}\label{thm:aux-sem-lemma}
For any posets $A,B,C,G$ with $\infty$ and $\vee$,
\begin{enumerate}
  \item $\texttt{snrec} \in \Hom_{Poset}\left({\left(C^1\right)}^G\times {\left(C^{\N\times C}\right)}^G,C^{G\times\N}\right)$
  \item $\texttt{slrec} \in  \Hom_{Poset}\left({\left(C^1\right)^G} \times {\left(C^{A \times (\N \times C)}\right)^G},C^{G \times \N}\right)$
  \item $\texttt{scase} \in \Hom_{Poset}\left(C^{G \times A}\times C^{G \times B},C^{G\times(A + B)}\right)$
\end{enumerate}
\end{restatable}


The definition of \texttt{scase} is required to respect the quotienting
$\texttt{inl}(\infty) = \texttt{inr}(\infty)$; by maxing each branch the
image of $\infty$ from the other branch, we obtain $f(\gamma,\infty)
\vee g(\gamma,\infty)$ as the image of both of those.  The definition of
\texttt{snrec} is required to be monotone in the $0 \le 1 \le \ldots \le
\infty$ ordering; taking the maximum of the base case and the inductive
step achieves this, because it forces the image of 1 to dominate the
image of 0.  The definition of \texttt{slrec} is similar; the new
question that arises is that, because we have abstracted lists as their
lengths, forgetting the elements, we do not have a value for the head of
the list to supply to $g$ (which, when we use this operation, will be
the translation of the cons branch given to the $\lambda^\bbbc$
recursor).  Here, we always supply $\infty$ as the head list element,
which is sufficient when the analysis really does not require any
information about the elements of the list (otherwise, one can make a
model where lists are interpreted more precisely than as their
lengths~\cite{danner-et-al:icfp15,danner-licata:jfp-in-prep}).

The interpretation satisfies standard soundness theorems, the 
proofs of which 
%\begin{icfp2020}can be found in the full version of this
%paper~\citep{cutler-et-al:icfp2020-full}.\end{icfp2020}
%\begin{arxiv}
are in \autoref{appendix:b}.%\end{arxiv}

%% The majority of the cases for the following theorems are formally
%% verified in Hudson (\cite{hudson}), and so we omit all but the new
%% ones for \texttt{nrec} and \texttt{lrec}.

\begin{restatable}[Compositionality]{theorem}{semsubst}
\label{thm:sem-subst}
If $\Gamma, x : T_1 \vdash E : T_2$, and $\Gamma \vdash E' : T_1$, then $\scott{\Gamma \vdash E[E'/x] : T_2} =  \left(1_{\scott{\Gamma}},\scott{\Gamma \vdash E' : T_1}\right) ; \scott{\Gamma, x:T_1 \vdash E : T_2}$
\end{restatable}


\begin{restatable}[Soundness (Terms)]{theorem}{interpsound}
\label{thm:term-soundness}
If $\Gamma \vdash E : T$, then $\scott{\Gamma \vdash E : T} \in \Hom\left(\scott{\Gamma},\scott{T}\right)$
\end{restatable}


\begin{restatable}[Soundness (Inequality)]{theorem}{preordsound}
\label{thm:soundness-inequality}
If $\Gamma \vdash E \leq E'$, then for all $\gamma \in \scott{\Gamma}$, $\scott{\Gamma \vdash E : T}(\gamma) \leq \scott{\Gamma \vdash E' : T}(\gamma)$
\end{restatable}

\subsection{Binary Counter Conclusion}

We interpret the binary counter recurrences from \autoref{fig:bc-rec}
in preorders by unfolding the definitions in
\autoref{fig:sem-interp}; the result is shown in
\autoref{fig:bc-poset}.  For the function \texttt{inc}, this yields a
monotone map $\scott{\norm{\texttt{inc}}_p} \in \Hom(1,\N \to \Z \times
\N)$, which is (essentially) a function from an input list size to the
cost of evaluation and the length of the output.  For the function
\texttt{set}, this yields a monotone map $\scott{\norm{\texttt{set}}}
\in \Hom(1,\Z\times(\N \to \Z \times \N))$, which is a pair of a cost
(the cost of evaluating the function definition --- $0$ since
$\texttt{set}$ is a value) and a function from input size to the cost of
evaluation and the length of the output.
% This discussion is dep'd, since we inline \scott{\norm{inc}_p} into the defn of set in all 3 figures.
%For \texttt{set}, we are left with a map $\scott{\norm{\texttt{set}}} \in \Hom(1 \times (\N \to \N \times N),\N \times (\N \to \N \times \N))$ which takes an increment function as an argument and returns a function from numbers to a costs and lengths of output. 
% Dep'd Because of inlining 
%The terms in \autoref{fig:bc-poset} are slightly different from this- we substitute $\norm{\texttt{inc}}_p$ into the body of $\norm{\texttt{set}}$ and then apply \autoref{thm:sem-subst} in order to make the simplification a bit easier. 
%Overall, we are left with $\scott{\norm{\texttt{inc}}_p} \in \Hom(1,\N \to \N \times \N)$ and $\scott{\norm{\texttt{set}}\left[\norm{\texttt{inc}}_p/inc\right]} \in \Hom(1,\N \times (\N \to \N \times \N))$

\begin{figure}
  % Exact:
%$$
%\scott{\norm{\texttt{inc}}_p} = \lambda \gamma. \lambda bs. \snrec(
%     \lambda \gamma.\lambda z. (2,1)
%     \lambda \gamma. \lambda  p. (\lambda x. \texttt{case}(
%                                               \lambda \gamma'. (2 + \pi_1\pi_1\pi_2\pi_2\pi_1 \gamma', 1 + \pi_2\pi_1\pi_2\pi_2\pi_1 \gamma'),
%                                               \lambda \gamma'. (\pi_1\pi_2\pi_2\pi_2\pi_1 \gamma', 1 +  \pi_2\pi_2\pi_2\pi_2\pi_1 \gamma')
%                                              )(((\gamma,p),x),\pi_1x)
%                                 (\pi_1 p,((0,\pi_1\pi_2 p),\pi_2 \pi_2 p))
%)
%$$
\begin{small}
\[
\begin{array}{l}
\scott{\norm{\texttt{inc}}_p} = \lambda \gamma. \lambda bs. \slrec(\lambda \gamma.\lambda z. (\framebox{2},1),\\
\hspace{12.5em}     \lambda \gamma. \lambda  p. \texttt{case}(\lambda x. (\framebox{2}, 1 + \pi_1\pi_2 p),\\
\hspace{18.25em}                                               \lambda x. (\framebox{$\pi_1\pi_2 \pi_2 p$}, 1 +  \pi_2\pi_2 \pi_2 p)\\
\hspace{18em}                                              )(\gamma',\pi_1 \pi_1 p)\\
\hspace{16em} \text{where } \gamma' = ((\gamma,p),(\pi_1 p,((0,\pi_1\pi_2 p),\pi_2 \pi_2 p)))\\
\hspace{12.25em}       )(\gamma,bs)\\

\scott{\norm{\texttt{set}}_p} = \lambda \gamma. (0,\lambda n.\snrec(\lambda \gamma'.\lambda x.(\framebox{$0$},0)\\
\hspace{12em} \lambda \gamma. \lambda p. (\framebox{$\pi_1\pi_2p + \pi_2(\scott{\norm{inc}_p}()(\pi_2 \pi_2 p))$},\pi_2(\scott{\norm{inc}_p}()(\pi_2 \pi_2 p)))
))

%\scott{\norm{\texttt{set}}} = \lambda \gamma . ($0$,\lambda n. \snrec(\lambda \gamma. \lambda x. (\framebox{$0$},0),\\
%\hspace{13em}                \lambda \gamma.\lambda p. (\framebox{$\pi_1\pi_2p + \pi_1(\scott{\norm{\texttt{inc}}_p}()(\pi_2\pi_2 p))$}, \\
%\hspace{16.5em}                                          \pi_2(\scott{\norm{\texttt{inc}}_p}()(\pi_2\pi_2 p))))\\
%\hspace{12.5em}                                         ((\gamma,n),n))
\end{array}
\]
\end{small}

  \vspace{-0.2in}
  \caption{Binary Counter Recurrences Interpreted}
  \label{fig:bc-poset}
\end{figure}

We have boxed the parts of the term that are related to computing the
cost.  The boxed portions of \texttt{inc} express that its amortized
cost is 2 on the empty list (to create a 1 bit with a credit), is 2 when
the bit is 0, and is exactly the same number of steps as the recursive
call when the bit is 1.  The boxed portions of \texttt{set} express that
for zero it costs 0, and for successor it costs the recursive call plus
the cost of \texttt{inc} on the potential of the output of the recursive
call.  However, because we will show that \texttt{inc} turns out to be
constant amortized time, we do not need to bound the potential of the
output of \texttt{set}.  Intuitively, to see that \texttt{inc} has
constant amortized time, observe that the \texttt{slrec} will always
supply the $\infty$ bit as the head of the list, which by definition of
the coalesced sum is both true and false, so the case is effectively the
maximum of $2$ and $\pi_1 \pi_2 \pi_1 p$.  Thus, we effectively have
recurrence where $T_{\texttt{inc}}(0) = 2$ and $T_{\texttt{inc}}(n) = 2
\vee T_{\texttt{inc}}(n-1)$, which solves to $T(n) = 2$ by induction.
Substituting this into the recurrence for \texttt{set}, we have
essentially $T_{\texttt{set}}(0) = 0$ and $T_{\texttt{set}}(n) =
T_{\texttt{set}}(n-1) + 2$, which is of course $O(n)$.  More formally,
we can show by induction that for all $n \geq 0$,
$(\scott{\norm{\texttt{inc}}_p}()(n))_c \leq 2$, and that for all $n$,
$(\scott{\norm{\texttt{set}}_p}()(n))_c \leq 2n$,
establishing bounds on these recurrences in this denotational semantics
in preorders.  

By the bounding theorem (Corollary~\ref{cor:true-cost}), we have that,
for the true operational cost $m$ of evaluating $\texttt{set}(n)
\downarrow^{(m,r)} v$, we have $m \le_\bbbc {\norm{\texttt{set}}_p}(n)_c$
in terms of the syntactic preorder judgment in $\lambda^\bbbc$.  By the
soundness of the interpretation in preorders
(\autoref{thm:soundness-inequality}), we have that $m
\le_{\mathbb{Z} \sqcup \infty} \scott{\norm{\texttt{set}}_p}()(n)_c$ in
the preorder model.  Therefore, by transitivity, we have $m \le 2n$ in
the preorder model, so our technique proves that the true operational
cost $m$ of setting the binary counter to $n$ is in fact $O(n)$,
as desired.

%% Of course, this result is not unique to the binary counter
%% example. Indeed, given any ``classical" amortized analysis that uses the
%% accounting method (with credits), it is in principle straightforward to
%% perform the same annotation, recurrence extraction, and interpretation
%% to arrive at bounds similar to those described in literature. \textbf{Is
%%   this a good enough way to dance around the fact that we haven't done
%%   any other examples?}


\section{Variable-Credit Extension}
\label{sec:ex}

The version of $\lambda^A$ described thus far supports amortized analyses
where the amount of credit stored on each element of a data structure is
fixed (e.g. $\listty{!_2 A}$ is a list with 2 credits on each element).
However, in some important amortized analyses, different amounts of credit
must be stored in different parts of a data structure---e.g. for balanced
binary search trees implemented via splay trees~\cite{sleator-tarjan-85},
the number of credits stored on each node is a function of the size of the
subtree rooted at that node.  In this section, we show that adding
existential quantification over credit amounts to $\lambda^A$ suffices to
analyze such examples, using a portion of splay trees as an example.  Using
existentials, a value of type $\exists \alpha.!_\alpha A$ is a value of type
$A$ which carries $\alpha$ credits, for some $\alpha$; for example, a tree
whose elements are of type $\exists \alpha.!_\alpha \mathbb{N}$ stores a
variable number of credits with the number on each node. In keeping with our
methodology of doing as much of an analysis as possible in the recurrence
language and its semantics, the fact that a particular piece of code uses
existentials to implement a desired credit policy will not be tracked by the
type system, but proved after recurrence extraction.  An alternative
approach would be to enrich $\lambda^A$ with some form of indexed or
dependent types to track the sizes of data structures in the type system,
but such an extension is not necessary for our approach.  
The proofs of the results in this section
%\begin{icfp2020}can be found in the full version of this
%paper~\citep{cutler-et-al:icfp2020-full}.\end{icfp2020}
%\begin{arxiv}
are in \autoref{appendix:b}.%\end{arxiv}

%% This type of credit policy lends itself to a dependently-typed intermediate language, with judgments of the form $n : \N \vdash_{n + \texttt{valOf}(n)\cdot \ell} M : !^1_{n \cdot \ell} A$, where the amount of credit stored in a $!A$ could depend on some function of term variables. In order to simplify the presentation and focus on specifics of cost analysis rather than dependent types, we will take a simpler approach, which suffices to encode the examples that we are interested in.

\subsection{Existential Types in $\lambda^A$}
To support existential quantifiers over credits, we extend the main typing judgment to be one of the form $\Delta | \Gamma \vdash_f M : A$, where $\Delta = \alpha_1,\dots,\alpha_n$ is a list of ``credit variables''. Any of the $\alpha_i$ can occur free in the types in $\Gamma$, the resource term $f$, the term $M$, or the type $A$. Credit variables $\alpha$ range over \textit{credit terms} $c$, which are (finite) sums of credit variables like $\alpha,\beta$ and credit constants $\ell$ --- i.e. $\alpha_1 + \alpha_2 + \ldots + \alpha_n + l$.  We write $\Delta \vdash c \texttt{  credit}$ to mean that a credit term is well-formed from the variables in $\Delta$.  We consider credit terms up to the usual equations for addition on natural numbers.  These credit terms can then be used as the ``bank'' in resource terms: the resource term $3x + 2y + (\alpha + 2)$ describes a context where one can use $x$ $3$ times, $y$ twice, and has access to the credit term $\alpha + 2$ credits. Most importantly, credit terms are now allowed to appear in the subscript of the $!$ modality (generalizing the natural number constants $\ell$ allowed above): a term $\alpha \mid \Gamma \vdash_f M : !_\alpha A$ with is an $A$ with $\alpha$ credits attached.
%% Of course, credit terms can also be $c = \ell$, a constant amount of %% credit.
We add a new type $\exists \alpha . A$ for existentially quantifying over credit variables.
A value of type $\exists \alpha . A$ is a value of type $A[c/\alpha]$, for some credit term $c$.  Such a value does not store the ability to \emph{use} the credits $c$ --- it stores a number of credits itself.
However, combining the existential with the $!$ modality,
a value of type $\exists \alpha. !_\alpha A$ is an $A$ with $c$ credits attached, for some credit term $c$.
The operational semantics is defined for terms with no free credit variables, so its structure remains unchanged.

\begin{figure}
  \begin{small}
  \[
  \begin{array}{cc}
\infer{\Delta | \Gamma \vdash_f \pack \alpha c M : \exists \alpha . A}{\Delta | \Gamma \vdash_f M : A[c/\alpha] & \Delta,\alpha \vdash A & \Delta \vdash c \texttt{ credit}}
&
\infer{\pack \alpha \ell M \downarrow^{(n,r)} \pack \alpha \ell v}{M \downarrow^{(n,r)} v}
\\
\infer{\Delta | \Gamma \vdash_{f + g} \unpack \alpha x M N : C}{\Delta | \Gamma \vdash_f M : \exists \alpha.A & \Delta,\alpha|\Gamma,x : A \vdash_{g+x} N : C & \Delta \vdash C}
&
\infer{\unpack \alpha x M N \downarrow^{(n_1+n_2,r_1+r_2)} v}{M \downarrow^{(n_1,r_1)} \pack \alpha\ell {v_1} & N[\ell/\alpha,v_1/x] \downarrow^{(n_2,r_2)} v}
  \end{array}
  \]
\end{small}

  \caption{Extension of $\lambda^A$ with existential types}
  \label{fig:la-ex-rules}
\end{figure}

The typing rules and operational semantics for existential types are presented in \autoref{fig:la-ex-rules}.
The terms for existentials are standard $\texttt{pack}$/$\texttt{unpack}$ terms.
The operational semantics of \texttt{pack} and \texttt{unpack} are also standard; because we only evaluate closed terms, the credit term being packed/unpacked with the value will always be a (closed) natural number $\ell$.

%% To evaluate $\pack \alpha \ell M$ we simply evaluate the argument $M$. To evaluate $\unpack \alpha x M N$, we evaluate $M$ to a package $\pack \alpha \ell {v_1}$, plug $\ell$ in for $\alpha$ and $v_1$ for $x$ in $N$, and evaluate.

%% To introduce $\exists \alpha. A$, one must provide a credit term $c$ as a witness for $\alpha$, as well as a term of type $A[c/\alpha]$. The corresponding introduction rule, \texttt{pack}, creates an opaque ``package" of the witness $c$ along with the term of type $A[c/\alpha]$.

%% We can subsequently use the \texttt{unpack} rule to take a package $M : \exists \alpha . A$, and allow for it to be used in a continuation term $\alpha|x : A \vdash N : C$. Crucially to the behavior as an existential, the $\alpha$ is not reified as $c$ when the package is unpacked, and the $\alpha$ cannot appear free in the motive type $C$.

The rest of the rules for $\lambda^A$ are mostly unchanged, so we do not repeat them: they are obtained from the rules in \autoref{fig:la-ty-rules} by carrying the credit variable context $\Delta$ through all of the rules, and,
in the $!^k_c$ modality and the \texttt{save}, \texttt{transfer},
\texttt{create}, and \texttt{spend} terms, the natural number constants
$\ell$ are generalized to credit terms $c$ constructed from these variables.
% The main difference is that the subscript of the ranges over credit terms in order to reflect the storage of variable amounts of credit on a value. To this end, when attaching $c$ credits to a term, one must provide proof that the credit term $c$ is in fact well formed, given the current credit variable context.
Finally, since the resource terms may contain free credit variables, the ordering judgment on resource terms must be augmented with a credit variable context, and the ordering itself extended to contain the coefficient-wise ordering on credit variables.
The operational semantics for these constructs in unchanged, because closed credit terms are
precisely the credit values $\ell$ used above.

For this extension, substitution and type preservation are stated as follows:

\begin{restatable}[Substitution]{theorem}{substext}\label{thm:subst-ext}
$\;$
\begin{itemize}
  \item If $\Delta \vdash c \texttt{ credit}$ and $\Delta,\alpha \vdash c' \texttt{ credit}$, then $\Delta \vdash c'[c/\alpha] \texttt{ credit}$
  \item If $\Delta \vdash c \texttt{ credit}$ and $\Delta,\alpha|\Gamma\vdash_f M : A$, then $\Delta|\Gamma[c/\alpha] \vdash_{f[c/\alpha]} M[c/\alpha] : A[c/\alpha]$
\end{itemize}
\end{restatable}

\begin{restatable}[Preservation]{theorem}{presext}\label{thm:pres-ext}
If $\cdot | \cdot \vdash_a M : A$ and $M \downarrow^{(n,r)} v$, then $a + r \geq 0$ and $\cdot | \cdot \vdash_{a + r} v : A$.
\end{restatable}

\subsection{Extracting Recurrences for Existentials}
\begin{figure}
  \begin{small}
  \[
  \begin{array}{rcl}
\angles{\exists \alpha. A} & = & \$ \times \angles{A}\\
\norm{\pack \alpha c M} & = & (\norm{M}_c,(c,\norm{M}_p))\\
\norm{\unpack \alpha x M N}  &= & \norm{M}_c +_c \norm{N}[\pi_1\norm{M}_p/\alpha,\pi_2\norm{M}_p/x]\\
\norm{\wait c M} & = & (\toC{c} + \norm{M}_c,\norm{M}_)\\
\norm{\disc c M} & = & (-\toC{c} + \norm{M}_c,\norm{M}_p)\\
  \end{array}
  \]
\end{small}

  \caption{Recurrence extraction for credit existentials}
  \label{fig:la-ex}
\end{figure}

Recall that the recurrence extraction in \autoref{fig:rec-extr}
erases the $!^k_\ell A$ modalities and translates $\wait \ell M$ and $\disc \ell M$ by adding/subtracting $\ell$ to/from the amortized cost.
Since we now allow credit variables $\alpha$, such as those coming from unpacking an existential type, in the credit position of $\waitname$/$\discname$, the recurrence extraction will need to refer to the values chosen for $\alpha$ in order to know how much to add/subtract to/from the amortized cost.
Thus, we add a type $\$$ to the recurrence language, the values of which are numbers of credits, represented by natural numbers.  The credit context $\Delta$ is translated to recurrence language variables of type $\$$
(i.e. $\angles{\Delta,\alpha} = \angles{\Delta},\alpha : \$$), while existential types $\exists \alpha.A$ are translated to pairs  $\$ \times \angles{A}$.  A simple pair suffices because the $!$ modality is erased by $\angles{\cdot}$, and this is the only place where credit terms can occur in the syntax of types, so all occurrences of $\alpha$ under the binder are removed, and $\angles{A}$ is a closed type.

We show the new and changed cases of recurrence extraction in \autoref{fig:la-ex}.  The introduction and elimination rules for $\exists \alpha . A$ translate to the corresponding introduction and elimination forms for $\$ \times \angles{A}$.
For \texttt{create} and \texttt{spend}, in principle, we would like the cost component of $\wait c M$ to be $c + \norm{M}_c$, but this will not type check, given that $c : \$$ but $\norm{M}_c : \bbbc$.
Recalling that costs $\bbbc$, though axiomatized as a monoid with some operations, are morally integers, we add a coerction $\texttt{to}\mathbb{C} : \$ \to \bbbc$, which is morally the inclusion of natural numbers into integers.

%% Most notably, this extraction, as shown in \autoref{fig:lc-ex} necessitates that credit variables appear in recurrences. For example, the term $\alpha | \cdot \vdash_3 \pack \beta {\alpha + 2} M : \exists \alpha. A$ extracts to the recurrence $(\norm{M}_c,(\alpha + 2,\norm{M}_p))$.
%% For these recurrences to be well-typed, the the type-level context $\Delta$ must be demoted in the recurrence extraction to be included in the $\lambda^\bbbc$ term variable context. Once this is established, the extraction for \texttt{pack} and \texttt{unpack} is straightforwardly defined.

\begin{restatable}[Extraction Preserves Types]{theorem}{extrsoundex}\label{thm:extr-sound-ex}
If $\Delta | \Gamma \vdash_f M : A$, then $\angles{\Delta},\angles{\Gamma} \vdash \norm{M} : \norm{A}$
\end{restatable}


\subsection{Bounding Relation and Bounding Theorem}

%% With the recurrence extraction in place, we move to modifying and re-stating the bounding theorem. Similarly to the previous sections, the bounding relation itself requires little modification to handle the extensions to $\lambda^A$ and $\lambda^\bbbc$.

The definition of the bounding relation for values (Definition~\ref{def:bounding}) is extended  with
\begin{itemize}
\item
  $\pack \alpha \ell v \valbd^{\exists \alpha.A,a} E$ iff $\ell \leq_{\$} \pi_1 E$ and $v \valbd^{A[\ell/\alpha],a} \pi_2 E$
\end{itemize}
Recalling that $E : \angles{\exists \alpha. A} = \$ \times \angles{A}$, this simply states that the amount of credit packed by $\alpha$ is bounded by the amount described by $\pi_1 E$, and that the value packed with the credit amount is in fact bounded by $\pi_2 E$. We remark that this definition may give the careful reader pause-- inducting on a substitution instance of an existential type where the existential variable ranges over \textit{types} leads to well-definedness issues.
But, our existential variables range over \textit{credits}, so we may simply regard a closed substitution instance of a type $\alpha \vdash A \, \mathsf{type}$ as a smaller type than $A$.

The definition of the bounding relation for open terms must also be modified to quantify over closing substitutions for the credit context, as well as the term context.
First, if $\omega$ is a substitution of credit amounts $\ell$ for credit variables, and $\Omega$ is a substitution of closed terms of type $\$$ for recurrence language variables, then $\omega \bdby^{\Delta} \Omega$ means that for all $\alpha \in \Delta$, $\omega(\alpha) \leq_\$ \Omega(\alpha)$.
Then for $\Delta | \Gamma \vdash_f M : A$ we write $M \bdby^A E$ if for all $\omega \bdby^{\Delta} \Omega$ and for all $\theta \bdby^{\Gamma[\omega],\sigma} \Theta$, we have that $M[\omega,\theta] \bdby^{A[\omega],f[\omega,\sigma]} E[\Omega,\Theta]$.
Using this notation, the bounding theorem is
\begin{restatable}[Bounding Theorem]{theorem}{boundingex}
\label{thm:bounding-ex}
If $\Delta | \Gamma \vdash_f M : A$, then $M \bdby^A \norm{M}$
\end{restatable}
\noindent and the cases which differ from the original \autoref{thm:bounding} are proved in the supplementary materials.


\subsection{Splay Tree Analysis}
We now describe somewhat informally how to use the above machinery to
analyze splay trees; the complete formalism is given 
%\begin{icfp2020}in the full version of this
%paper~\citep{cutler-et-al:icfp2020-full}.\end{icfp2020}
%\begin{arxiv}
in \autoref{appendix:b} (\autoref{fig:trec-rules}).%\end{arxiv}
Following 
Okasaki's presentation~\cite{okasaki:purely-functional-data-structures},
the key operation is
a $\texttt{split} : (A \times \tree A) \to \tree A \times \tree A$ function that splits a given tree into elements larger and smaller than a given pivot. Insertion, deletion, union, intersection, difference etc. can be all implemented from \texttt{split} and a $\texttt{join}$ operation that combines two sorted trees where all the elements of the first are less than the elements of the second. Showing that $\texttt{split}$ is amortized $O(\log n)$ time, where $n$ is the size of the tree, is the most difficult part of the amortized analysis, and implies the desired time bounds for the other operations.  The key idea of splay trees is that each access rearranges the tree so that accessing the same element twice in a row is quicker the second time. In Okasaki's presentation, this rearrangement takes place in $\texttt{split}$, which performs a series of tree rotations. These rotations ensure that the amortized cost of $\texttt{split}$ (amortized over any sequence of binary search tree operations) is $O(\log n)$, even though the tree is not always balanced.
The most challenging cases of the code unpack the tree to depth two, and rotate the output if they traverses the same direction twice while searching for the pivot:
$$
\begin{array}{l}
split \; p \; (N (x,N (y,a_{11},a_{12}), N (z,a_{21},a_{22}))) | \; x \geq p \mathbin{\&\&} y \geq p = \\
\hspace{1em} (small, N (y,big,N (x,a_{12},N (z,a_{21},a_{22})))) \texttt{ where } (small,big) = \texttt{split} \; p \; a_{11}\\
\end{array}
$$
Okasaki's analysis of split maintains the invariant that there are $\varphi(t) = \lceil{\lg (|t| + 1)}\rceil$ credits associated with the root of every subtree $t$ in a splay tree, and uses the potential/physicists method to analyze the amortized cost.  

The addition of existentials to $\lambda^A$ allows us to encode this analysis, by giving \texttt{split} the type $A \otimes \tree {\exists   \alpha.!^1_\alpha A} \loli \tree {\exists \alpha.!^1_\alpha A} \otimes \tree {\exists \alpha.!^1_\alpha A}$, and using code to maintain the invariant that each of these $\alpha$'s are precisely $\varphi(t)$.

%% Then, when a node is encountered on a path down the tree, its credits are spent. The credit invariant is subsequently re-established by \texttt{create}-ing enough credits as the recursive calls unwind.

\subsubsection{Creating Variable Amounts of Credit.}

\begin{figure}
  \vspace{-0.1in}
  \[
\begin{array}{rcl}
 N_1 & = & {\lambda \_. \pack \alpha 0 {(\save 1 0 {()})}}\\
 N_2 & = & {\lambda (\_,(\alpha,\save \alpha 1 {()})). \wait 1 {(\pack \beta {\alpha + 1} {\save {\alpha + 1} {1} {()}})}}\\
\texttt{spawn}(n) &=& \nrec {n} {N_1} {\save \infty 0 {N_2}}  : \exists \alpha . !_\alpha^1 1
\end{array}
\]

  \vspace{-0.2in}
  \caption{$\lambda^A$ term for the \texttt{spawn} function}
  \label{fig:ghost-loop}
\end{figure}

To maintain this invariant, we will sometimes need to \waitname\/ amounts of credit determined by a run-time natural number, like $\varphi(t)$ for some tree $t$---but the primitive \wait{c}{M} term allows for waiting only for a credit term $c$, which cannot depend on run-time values.
However, we can write a recursive loop that spawns a number of credits
dependent on a run-time value, and package this as a function
$\texttt{spawn} : \N \loli \exists \alpha . !^1_\alpha 1$
such that the $\alpha$ packed in the result of $\texttt{spawn}(n)$ is (the credit term representing) $n$.
The implementation of \texttt{spawn}\/ is shown in \autoref{fig:ghost-loop}---at a high level,
the term loops $\texttt{create}_1$ in a $\N$-recursor, using a credit existential as a counter variable.
In this example, and throughout this section, we use pattern-matching notation as syntactic sugar for the elimination rules for positive types like $\exists,!,\otimes$, with the convention that matching on the result of a thunked recursive call implicitly forces it.

In \autoref{sec:la-sem}, we argued that
the $n$ component in the operational cost semantics $M \downarrow^{n,r} v$ captures the actual operational cost of an erasure to simply-typed $\lambda$-calculus, as long as $\texttt{tick}$s in $\lambda^A$ are inserted for each STLC $\beta$-redex.  Because we do not include any \texttt{tick} terms in \texttt{spawn}, its abstract operational cost $n$ is zero.  Thus, to realize this cost semantics, $\texttt{spawn}$ must be erased before actually running the program.  Fortunately, a simple program optimization suffices to do this: translate $\lambda^A$ to simply-typed $\lambda$-calclus by dropping both the $\exists$ and $!$ types and the associated terms, at which point \texttt{spawn} has type $\N \to 1$; then replace all terms of type $1$ with the trivial value.
That is, we think of \texttt{spawn} as a \emph{ghost loop} --- code that is meant for the extracted recurrence, but not intended to actually be run.  

\subsubsection{Definition of Trees in $\lambda^A$.}
Extending $\lambda^A$ with the requisite \texttt{tree} type constructor and its rules follows both previous work~\cite{danner-et-al:icfp15}
and the pattern illustrated with lists above.
The type of trees is essentially $\tree A = \texttt{Emp} \; | \; \texttt{N of } A \otimes \N \otimes \tree A \otimes \tree A $.
The $\N$ argument caches the size of the tree, making the function $\texttt{size} : \tree A \loli \N \otimes \tree A$ --- which projects out that field and then rebuilds the tree\footnote{The tree can be rebuilt   because values of type $\N$ are duplicable--- there is a diagonal map   $\N \loli \N \otimes \N$. Also, we will often use \texttt{size} as a   function $\tree A \loli \N$, and silently contract the second   projection for re-use of the argument.} --- constant time.
To support coding the \texttt{split} function described above, we directly add a recursor that performs a two-level pattern match,
with cases for the empty tree, for a node with one child or the other empty and the other is another node, and for a node with two nodes as children; in the latter case, the recursor provides recursive calls on
all four subtrees.  
% The details are in the supplementary materials--- see \autoref{fig:trec-rules}.  

% The details of this recursor along with its corresponding recurrence and proofs can be found in the Appendix-- but it is precisely what is required to write the \texttt{split} function in $\lambda^A$. Of course, the same can be accomplished with general recursion as handled in prior work~\cite{popl20}, but we instead opt for the ad hoc solution, rather than develop the theory of amortized analysis with general recursion.

\subsubsection{Splay Tree Implementation.}
We define a \textit{splay tree} to be a binary search tree $t : \tree {\exists \alpha. !^\infty_\alpha A}$ satisfying the property that if $\texttt{size}(t) = n$, then if $t = N(\_,m,t_0,t_1)$, then $t_0$ and $t_1$ are splay trees, and for $\scott{\norm{t}_p} = N((\alpha,\_),\_,\_)$, we have $\alpha = \phi(n)$.
In other words, the credit invariant holds at each node in the tree. We note that each element of the tree not only carries $\alpha$ credits, but is also infinitely usable since we are required to compare nodes in the tree more than constantly many times. This causes no issues for the extracted recurrences, because keys in the tree are always values. We then prove a lemma which states that \texttt{split} preserves the splay tree property --- i.e. that the existentially quantified credits stored in the tree satisfy the desired invariant.  

\begin{lemma} \label{lem:splay-tree-invariant}
If $t : \tree {\exists \alpha. !^\infty_\alpha A}$ is a splay tree and $\texttt{split}(t) \downarrow (t_0,t_1)$, then $t_0$ and $t_1$ are also splay trees.
\end{lemma}

To illustrate the $\lambda^A$ term for \texttt{split}, we show one key
case of the recursor, which corresponds to the snippet given at the
beginning of this section and to \cite[Theorem
  5.2]{okasaki:purely-functional-data-structures}.  For this case, we
are in the situation where the root, labeled by $x$, has two subtrees,
$y$ with subtrees $a_{11},a_{12}$, and $z$ with subtrees
$,a_{21},a_{22}$. If the pivot is less than both $x$ and $y$, we recur
on the leftmost subtree $a_{11}$, which produces the elements of
$a_{11}$ that are smaller and bigger than the pivot.  Then $smaller$
contains all the elements of the original tree smaller than the pivot.
The elements bigger than the pivot are $bigger$ and everything else from
the original tree; we combine these together into a new tree, performing
a rotation to put $y$ at the root.

The $\lambda^A$ version of this term, presented in \autoref{fig:split}, annotates the above code with some additional information about the sizes of trees, and with some code for manipulating credits.  The variables $x,y,z$ are the values of type $A$ at the root and its immediate children; these come with existentially-quantified numbers of credits $\alpha,\beta,\gamma$ ($\alpha$ credits are stored with $x$, $\beta$ with $y$, and $\gamma$ with $z$), and also with natural numbers caching the sizes of the subtrees that they are the roots of ($n_1,n_2,n_3$ respectively).
The variables $a_{ij}$ stand for the four subtrees with their (suspended) recursive call outputs; we write $\texttt{split}(p,a_{11})$ for projecting and forcing the recursive call, and write $a_{ij}$ for projecting the other subtrees.  The credit manipulation involves spending the credits $\alpha$ and $\beta$ stored with $x$ and $y$ in the input tree (we do not spend $z$, because the $z$ node is left unchanged in the output), calculating the sizes of the new nodes $t'$ and $s'$ that will be part of the output, and \texttt{spawn}ing credits corresponding to $\varphi$ of these sizes. The term presented in \autoref{fig:split} is one branch of one of the step functions passed to the $\texttt{treerec}$ which forms the outermost structure of \texttt{split}.

\begin{figure}
  \vspace{-0.1in}
  % \begin{small}
%   $$
% \begin{array}{l}
% \lambda ((\alpha,\save \infty \alpha x),n_1,(\beta,\save \infty \beta y),n_2,(\gamma,\save \infty \gamma z),n_3,a_{11},a_{12},a_{21},a_{22}).\\
% \hspace{1em} \texttt{if } x \geq p \mathbin{\&\&} y \geq p \texttt{then} \, \texttt{tick} ; \; \texttt{let}\\
%  \hspace{3em} (small,big) = \disc {\alpha + \beta} {(\texttt{split}(p,a_{11}))}\\
%  \hspace{3em} d = N(\pack \alpha \gamma {(\save \infty \gamma z)},n_3,a_{21},a_{22})\\
%  \hspace{3em} n_{12} = \texttt{size}(a_{12})\\
%  \hspace{3em} n_{big} = \texttt{size}(big)\\
%  \hspace{3em} t'_{size} = 1 + n_{12} + n_3 \\
%  \hspace{3em} s'_{size} = 2 + n_{big} + n_{12} + n_3\\
%  \hspace{3em} ((\alpha',\_),(\beta',\_)) = (\texttt{spawn}(\varphi(t'_{size})),\texttt{spawn}(\varphi(s'_{size})))\\
%  \hspace{3em} t' = N(\pack \alpha {\alpha'} {(\save \infty {\alpha'} x)},t'_{size},a_{12},d)\\
%  \hspace{3em} s' = N(\pack \alpha {\beta'} {(\save \infty {\beta'} y)},s'_{size},big,t')\\
% \hspace{2.5em}\texttt{in} \, (small,s') \, \texttt{end}\\
% \hspace{1em} \texttt{else} \ldots
% \end{array}
% $$
% \end{small}

\begin{small}
  $$
\begin{array}{l}
\lambda ((\alpha,\save \infty \alpha x),n_1,(\beta,\save \infty \beta y),n_2,(\gamma,\save \infty \gamma z),n_3,a_{11},a_{12},a_{21},a_{22}).\\
\texttt{if } x \geq p \mathbin{\&\&} y \geq p \texttt{then} \, \texttt{tick} ; \; \texttt{let}\\
 \hspace{2em} 
 \begin{array}{l@{\,}c@{\,}ll@{\,}c@{\,}l}
 (small,big) &=& \disc {\alpha + \beta} {(\texttt{split}(p,a_{11}))} & 
 d &=& N(\pack \alpha \gamma {(\save \infty \gamma z)},n_3,a_{21},a_{22})\\
 n_{12} &=& \texttt{size}(a_{12}) &
 n_{big} &=& \texttt{size}(big)\\
 t'_{size} &=& 1 + n_{12} + n_3 &
 s'_{size} &=& 2 + n_{big} + n_{12} + n_3\\
 ((\alpha',\_),(\beta',\_)) &=& (\texttt{spawn}(\varphi(t'_{size})),\texttt{spawn}(\varphi(s'_{size}))) &
 t' &=& N(\pack \alpha {\alpha'} {(\save \infty {\alpha'} x)},t'_{size},a_{12},d)\\
 s' &=& N(\pack \alpha {\beta'} {(\save \infty {\beta'} y)},s'_{size},big,t')
 \end{array} \\
\hspace{1em}\texttt{in} \, (small,s') \, \texttt{end}\\
\texttt{else} \ldots
\end{array}
$$
\end{small}

  \vspace{-0.3in}
  \caption{Part of the $\lambda^A$ term for \texttt{split}}
    \vspace{-0.1in}
  \label{fig:split}
\end{figure}

To analyze splay trees, we pass this $\lambda^A$ term through recurrence extraction and the preorder semantics and then prove the following:
\begin{theorem}
If $t : \tree {\exists \alpha . !^\infty_\alpha A}$ is a splay tree with $\texttt{size}(t) = n$, then for any $v : A$, $\scott{\norm{\texttt{split}(t,v)}_c} \leq 1 + 2\varphi(\scott{\norm{\texttt{size}}_p(t)}) \in O(\lg n)$.
\end{theorem}
\begin{proof}
As an example, we show the case for the code in \autoref{fig:split}. The cost component of the extracted recurrence is
\[
1-\alpha - \beta + \scott{\norm{\texttt{split}(p,a_{11})}} + \varphi(1 + n_{12} + n_3) + \varphi(2 + n_{big} + n_{12} + n_3)
\]
The 1 comes from the $\texttt{tick}$; $\alpha$ and $\beta$ are subtracted because they are \texttt{spent}; and the $\varphi$ of the sizes of $t'$ and $s'$ are added because they are \texttt{create}d.  
By definition, $1 + n_{12} + n_3 = \scott{\norm{\texttt{size}}_p(t')}$ and $2 + n_{big} + n_{12} + n_3 = \scott{\norm{\texttt{size}}_p(s')}$. By the credit invariant, $\alpha = \varphi(\scott{\norm{\texttt{size}}_p(t)})$, and $\beta = \varphi(\scott{\norm{\texttt{size}}_p(s)})$, where $s$ is the subtree of $t$ rooted at $y$. Rewriting by these and commuting terms, the extracted recurrence is precisely
\[1 + \scott{\norm{\texttt{split}(p,a_{11})}} + \varphi(\scott{\norm{\texttt{size}}_p(s')}) + \varphi(\scott{\norm{\texttt{size}}_p(t')}) - \varphi(\scott{\norm{\texttt{size}}_p(s)}) - \varphi(\scott{\norm{\texttt{size}}_p(t)})
\]
which Okasaki~\cite[Theorem 5.2]{okasaki:purely-functional-data-structures} proves is bounded by $1 + 2\varphi(\texttt{size}(t))$, as required.
\end{proof}

\section{Related Work}
\label{sec:related-work}
Techniques for extracting (asymptotic)
cost information from high-level program source
code is a project that is almost as old as studying programming languages.
% (and here we will not even broach the subject of worst-case execution time
% (WCET) analysis, which focuses on low-level timing analysis on specific
% hardware).  
For non-amortized analysis of
functional languages, we have examples from the 1970s and
1980s by \citet{wegbreit:cacm75}, \citet{lematayer:toplas88}, and
Rosendahl~\cite{rosendahl:auto-complexity-analysis}.  The idea of
simultaneously extracting information about cost and size, and defining the
size of a function to be a function itself (leading to higher-order
recurrences) has its roots in \citet{danner-royer:ats-lmcs},
which in turn draws from ideas in \citet{shultis:complexity},
\citet{sands:thesis}, and Van Stone~\cite{vanstone:thesis}.  Using
bounded modal operators to describe resource usage goes back at least to
\citet{girard-et-al:tcs92:bll}, and Orchard et~al.\ have
recently incorporated these ideas into the Granule
language~\cite{orchard-et-al:icfp19:graded-modal-types}.  Perhaps the
work that is closest in spirit to ours is Benzinger's ACA system for
analyzing call-by-name \textsc{Nuprl} programs~\cite{benzinger:tcs04}.  From
a cost-annotated operational semantics, he extracts a ``symbolic semantics''
that is similar in flavor to our recurrence language and extracted
recurrences, although without amortization.  The symbolic semantics yields
higher-order recurrences, which he reduces to first-order recurrences that
can be analyzed with a computer algebra system.  
% All of this work focuses on
% upper bounds for worst-case cost and does not address settings in which
% tight bounds for a composition of functions are not obtained by composing
% the corresponding bounds.

There is also extensive work on recurrence extraction from first-order
imperative languages.  The COSTA project
\citep{albert-et-al:jar11,albert-et-al:tcs12:cost-analysis,albert-et-al:tocl13:inference}
takes Java bytecode as its source language, extracts cost relations
(essentially, non-deterministic cost recurrences), and solves them for upper
bounds.  In this line of work, \citet{alonso-blas-genaim:sas12} and
\citet{flores-montoya:fm16} investigate the failure to derive tight upper
bounds in settings where amortized analysis is typically deployed.  They
trace the issue to the fact that typically cost relations do not depend on
the results of the analyzed functions.
% , and hence the computation of bounds
% cannot make use of such information.  
Making this possible allows more
precise constraints which, when solved, yield tighter bounds.  The
dependency on output corresponds roughly to total accumulated savings,
% (in
% the nomenclature of the physicist's approach to amortized analysis, the
% potential of a value), 
and they infer an appropriate potential function (in the terminology of the
physicist's method),
modulo a choice of templates.
% that are used to guide the inference process.
To analogize with our work, they delay the determination of the credit
policy until solving for upper bounds of extracted recurrences, whereas we
specify the credit policy as part of the source program, which directly
yields a recurrence for cost that takes the policy into account.

% I think this discussion of Alonso-Blas & Genaim is correct, but I'm not
% entirely sure.

Two recent approaches that handle amortized analysis for functional programs
are Timed~ML (TiML, \cite{wang-et-al:oopsla17:timl}) and automatic amortized
resource analysis (AARA,
\cite{hoffmann-et-al:toplas12:multivariate-amortized,hoffmann-shao:esop15:parallel,hoffmann-et-al:popl17,niu-hoffmann:lpar18}).
In TiML, ML type and function definitions are annotated with indices that
convey size information.  The notion of size is left unspecified and
the indices are very flexible, and can
include constraints such as those required to define red-black trees.  Type
inference generates verification conditions.  Depending on the details of
the annotations, solving the verification conditions provides exact or
asymptotic bounds on the cost of the original program.  The focus is on
worst-case analysis, but the annotation language is sufficiently rich to
encode the physicist's method of amortized analysis.  Although it is not
part of their focus, the formalism does not appear to enable analysis of
higher-order functions whose cost depends on the complexity behavior of the
function arguments.

AARA provides a type inference system for resource bound analysis of
higher-order functional programs that incorporates amortization.  Credit
allocation is built into the type system itself.
% in such a way that unused
% credits can be used in other parts of the typing.  
Soundness says that the
net credit change during evaluation is bounded by the net credit change
described by the typing.  AARA focuses primarily on strict languages, but
\citet{jost-et-al:jar17} use similar ideas to analyze programs
under lazy evaluation.
% , where amortization seems necessary for almost any
% analysis (see, e.g.,
% Okasaki~\cite{okasaki:purely-functional-data-structures}).  
% To summarize the
% technical difference between our approach and that of AARA, 
% Whereas for us a typing
% judgment $\Gamma\vdash_f\cdot  : A$ identifies credit usage and the term
% determines credit allocation (via $\texttt{save}$, $\waitname$, $\discname$,
% etc.), 
In AARA, the credit allocation and usage is described
in the type judgment.
Type inference generates constraints, and the solution of these constraints
is essentially a credit allocation strategy.  
% A type can be inferred if a
% bound can be defined in terms of a set of base functions for describing the
% strategy.  
Our approach describes usage in the type judgment, but
requires the strategy to be explicit in the program (via 
$\texttt{save}$, $\waitname$, $\discname$, etc.), which
places a greater burden on the programmer.  However, reasoning about that
strategy (e.g., establishing a credit invariant) in the semantics may
provide more flexibility, though that requires more investigation.
% One place this difference in detail plays out is where and how automation
% may play a role.  For AARA, automation arises via a constraint-based type
% inference algorithm, which determines the credit allocation strategy, and as
% implemented, finds bounds if they can be defined in terms of a set of base
% functions for describing the strategy.  For the approach described here,
% automation could arise in the solution of the extracted recurrences, which
% does not \emph{a priori} depend on a pre-defined set of base functions.
% Solving these recurrences is a project that requires future work.  It could
% also be that defining the strategy in the intermediate language, and then
% reasoning about that strategy (e.g., establishing a credit invariant) in the
% semantics may provide more flexibility, though again that requires more
% investigation.

We note that the technical differences between TiML and AARA and our
approach arise from a difference in what we might consider the
philosophical underpinnings.  TiML and AARA introduce novel type systems
with a goal of inferring cost bounds to the greatest extent possible.  Those
bounds are extracted as part of the type inference procedure.  This is not
how most programmers conceptualize a cost analysis, and our
interest is in staying as close to typical informal analyses as we can.
While $\lambda^A$ is a novel type system, the novelty exists solely in order
to make the programmer be explicit about how credits are allocated and used.
This task is part of a banker's-method analysis, though it is usually stated
informally (``put one credit on each~$1$ in the bit list'').  After that, it
is extraction of ordinary (semantic) recurrences which one hopes to be able
to bound using whatever methods are at the programmer's disposal.



\bibliographystyle{plainnat}

\bibliography{refs}

\appendix
\renewcommand\chaptername{Appendix}
\counterwithin{theorem}{chapter}
\counterwithin{figure}{chapter}

\chapter{}
\label{appendix:a}
\textbf{SOMETHING IS MESSED UP WITH APPENDIX THEOREM NUMBERING... FIX THIS}
\section{Rules of \dlambdaamor}

\section{\dlambdaamor Theorems and Proofs}

\ctxwfstreng*
\begin{proof}
Immediate from the fact that $\Psi ; \Theta ; \Delta \vdash \Gamma \; \texttt{wf}$ exactly when $\Psi ; \Theta ; \Delta \vdash \tau : \star$ for every $x : \tau \in \Gamma$.
\end{proof}

\conwfstreng*
\begin{proof}
Immediate by inversion.
\end{proof}

\begin{theorem}[Raw Admissibility of Weakening for Sort Checking]
If $\Theta ; \Delta \vdash I : S$ and $\Theta' \supseteq \Theta$, then $\Theta' ; \Delta \vdash I : S$
\end{theorem}

\begin{theorem}[Raw Admissibility of Weakening for Constraint Well-Formedness]
If $\Theta ; \Delta \vdash \Phi \; \texttt{ wf}$ and $\Theta' \supseteq \Theta$, then $\Theta' ; \Delta \vdash \Phi \; \texttt{ wf}$
\end{theorem}

\begin{theorem}[Admissibility of Weakening for Constraint Context Well-Formedness]
If $\Theta \vdash \Delta \; \texttt{wf}$ and $\Theta' \supseteq \Theta$, then $\Theta' \vdash \Delta \; \texttt{wf}$
\end{theorem}

\begin{theorem}[Admissibility of Weakening for Sort Checking]
If $\Theta ; \Delta \pvdash I : S$ and $\Theta' \supseteq \Theta$, then $\Theta' ; \Delta \pvdash I : S$
\end{theorem}

\begin{theorem}[Admissibility of Weakening for Constraint Well-Formedness]
If $\Theta ; \Delta \pvdash \Phi \; \texttt{wf}$ and $\Theta' \supseteq \Theta$, then $\Theta' ; \Delta \pvdash \Phi \; \texttt{ wf}$
\end{theorem}

\begin{theorem}[Index Substitution for Sort Checking]
If $\Theta, j : S_1 ; \Delta \pvdash I : S_2$ and $\Theta ; \Delta \pvdash J : S_1$ then $\Theta ; \Delta \pvdash I[J/j] : S_2$
\label{thm:idx-idx-subst}
\end{theorem}
\jtheorem{Proof of \autoref{thm:raw-idx-idx-subst}}{
By induction on the derivation of $\Theta, j : S_1 ; \Delta \vdash I : S_2$. The case for I-Var is immediate.

\jcase{1}{I-Plus}{
\jgivengoal{
    \caseFact{1} $\Theta, j : S_1 ; \Delta \vdash I_1 + I_2 : bS$
    
    \caseFact{2} $\Theta, j : S_1 ; \Delta \vdash I_1: bS$
    
    \caseFact{3} $\Theta, j : S_1 ; \Delta \vdash I_2 : bS$
    
    \caseFact{4} $\Theta \vdash \Delta \; \texttt{wf}$
  }{
      $\Theta ; \Delta[J/j] \vdash (I + J) [J/j] : bS$
  }
  \caseText{By IH on (2) and (3)}
  
  \caseFact{5} $\Theta ; \Delta[J/j] \vdash I_1[J/j] : bS$
  
  \caseFact{6} $\Theta ; \Delta[J/j] \vdash I_2[J/j] : bS$
  
  \caseText{By I-Plus}
  
  \caseFact{7} $\Theta ; \Delta[J/j] \vdash I_1[J/j] + I_2[J/j]: bS$
  
  \caseText{Goal follows by (7).}
}

\jcase{2}{I-Minus}{
\jgivengoal{
    \caseFact{1} $\Theta, j : S_1 ; \Delta \vdash I_1 - I_2 : bS$
    
    \caseFact{2} $\Theta, j : S_1 ; \Delta \vdash I_1: bS$
    
    \caseFact{3} $\Theta, j : S_1 ; \Delta \vdash I_2 : bS$
    
    \caseFact{4} $\Theta, j : S_1 ; \Delta \vDash I_1 \geq I_2$
    
    \caseFact{5} $\Theta \vdash \Delta \; \texttt{wf}$
  }{
      $\Theta ; \Delta[J/j] \vdash (I+1 - I_2) [J/j] : bS$
  }
  \caseText{By IH on (2) and (3)}
  
  \caseFact{5} $\Theta ; \Delta[J/j] \vdash I_1[J/j] : bS$
  
  \caseFact{6} $\Theta ; \Delta[J/j] \vdash I_2[J/j] : bS$
  
  \caseText{Instantiating the quantifier in (4) and using (5)}
  
  \caseFact{7} $\Theta ; \Delta[J/j] \vDash I_1[J/j] \geq I_2[J/j]$
  
  \caseText{By I-Minus on (5) (6) (7)}
  
  \caseFact{8} $\Theta ; \Delta[J/j] \vdash I_1[J/j] - I_2[J/j]: bS$
  
  \caseText{Goal follows by (8).}
}

\jcase{3}{I-Times-$\R^+$}{
 \jgivengoal{
   \caseFact{1} $\Theta, j : S_1 ; \Delta \vdash c \cdot I : \R^+$
   
   \caseFact{2} $\Theta, j : S_1 ; \Delta \vdash I : \R^+$
   
   \caseFact{3} $c \in \R^+$
 }{
   $\Theta ; \Delta[I/i] \vdash (c \cdot I)[J/j] : \R^+$
 }
 \caseText{By IH on (2)}
 
 \caseFact{4} $\Theta ; \Delta[J/j] \vdash I[J/j] : \R^+$
 
 \caseText{By I-Times-$\R^+$ on (3) and (4)}
 
 \caseFact{5} $\Theta ; \Delta[J/j] \vdash c \cdot I[J/j] : \R^+$
 
 \caseText{Goal follows by (5)}
}

\jcase{4}{I-Times-$\potvec$}{Identical to I-Times-$\R^+$}

\jcase{5}{I-Times-$\N$}{Identical to I-Times-$\R^+$}

\jcase{6}{I-Shift}{
  \jgivengoal{
    \caseFact{1} $\Theta, j : S ; \Delta \vdash \lhd I : \potvec$
    
    \caseFact{2} $\Theta, j : S ; \Delta \vdash I : \potvec$
  }{
    $\Theta ; \Delta[J/j] \vdash \left(\lhd I\right)[J/j] : \potvec$
  }
  
  \caseText{By IH on (2)}
  
  \caseFact{3} $\Theta ; \Delta[J/j] \vdash I[J/j] : \potvec$
  
  \caseText{By I-Shift on (3)}
  
  \caseFact{4} $\Theta ; \Delta[J/j] \vdash \lhd I[J/j] : \potvec$
  
  \caseText{Goal follows immediately from (4)}
}

\jcase{7}{I-Lam}{
  \jgivengoal{
    \caseFact{1} $\Theta, j : S_1 ; \Delta \vdash \lambda i : S_2. I : S_2 \to S_3$
    
    \caseFact{2} $\Theta, j : S_1, i : S_2 ; \Delta \vdash I : S_3$
  }{
    $\Theta ; \Delta[J/j] \vdash \lambda i : S_2. I[J/j] : S_2 \to S_3$  
  }
  
  \caseText{By IH on (2)}
  
  \caseFact{3} $\Theta, i : S_2 ; \Delta[J/j] \pvdash I[J/j] : S_3$
  
  \caseText{By I-Lam on (3)}
  
  \caseFact{4} $\Theta ; \Delta[J/j] \vdash \lambda i : S_2. I[J/j] : S_2 \to S_3$
  
  \caseText{Goal follows immediately by (4)}
}

\jcase{8}{I-App}{
  \jgivengoal{
    \caseFact{1} $\Theta, j : S ; \Delta \vdash I_1 \; I_2 : S_2$
    
    \caseFact{2} $\Theta, j : S; \Delta \vdash I_1 : S_1 \to S_2$
    
    \caseFact{3} $\Theta, j : S; \Delta \vdash I_2 : S_1$
  }{
    $\Theta ; \Delta[J/j] \vdash (I_1 \; I_2)[J/j] : S_2$,
  }
  
  \caseText{By IH on (2) and (3)}
  
  \caseFact{4} $\Theta ; \Delta[J/j] \vdash I_1[J/j] : S_1 \to S_2$
  
  \caseFact{5} $\Theta ; \Delta[J/j] \vdash I_2[J/j] : S_1$
  
  \caseText{By I-App on (4) and (5)}
   
  \caseFact{6} $\Theta ; \Delta[J/j] \vdash I_1[J/j] \; I_2[J/j] : S_2$
  
  \caseText{Goal follows from (6)}
}

\jcase{9}{I-Const}{
  \jgivengoal{
     \caseFact{1} $\Theta, j : S ; \Delta \vdash \texttt{const}(I) : \potvec$
     
     \caseFact{2} $\Theta, j : S ; \Delta \vdash I : \R^+$
  }{
    $\Theta ; \Delta[J/j] \vdash \texttt{const}(I)[J/j] : \potvec$
  }
  \caseText{By IH on (2)}
  
  \caseFact{3} $\Theta ; \Delta[J/j] \vdash I[J/j] : \R^+$
  
  \caseText{By I-Const on (3)}
  
  \caseFact{4} $\Theta ; \Delta[J/j] \vdash \texttt{const}(I[J/j]) : \potvec$
  
  \caseText{Goal follows from (4)}
}

\jcase{10}{I-$\N$-Lit} {Immediate.}

\jcase{11}{I-$\R^+$-Lit} {Immediate.}

\jcase{12}{I-$\potvec$-Lit} {Immediate.}
}
\iffalse
\begin{proof}
By induction on the derivation of $\Theta, j : S_1 ; \Delta \vdash I : S_2$.
\begin{itemize}
  \item[(I-Var)] Immediate.
  \item[(I-Plus)] Suppose $\Theta, j : S_1 ; \Delta \vdash I_1 + I_2 : bS$ by way of
  $\Theta, j : S_1 ; \Delta \vdash I_1: bS$ and
  $\Theta, j : S_1 ; \Delta \vdash I_2 : bS$.
  Applying IH twice, we have that
   and
  $\Theta ; \Delta \pvdash I_2[J/j] : bS$.
  By I-Plus, $\Theta ; \Delta \pvdash I_1[J/j] + I_2[J/j]: bS$, as required.
  \item[(I-Minus)] Suppose $\Theta, j : S_1 ; \Delta \vdash I_1 - I_2 : bS$ by way of
  $\Theta, j : S_1 ; \Delta \vdash I_1: bS$,
  $\Theta, j : S_1 ; \Delta \vdash I_2 : bS$, and
  $\Theta, j : S_1 ; \Delta \vDash I_1 \geq I_2$
  Applying IH twice, we have that
  $\Theta ; \Delta \pvdash I_1[J/j] : bS$ and
  $\Theta ; \Delta \pvdash I_2[J/j] : bS$.
  Moreover, we have $\Theta ; \Delta \vDash \forall j : S_1. (I_1 \geq I_2)$ since $\Theta \vdash \Delta \; \texttt{wf}$, and so
  $\Theta ; \Delta \vDash I_1[J/j] \geq I_2[J/j]$
  By I-Minus, $\Theta ; \Delta \pvdash I_1[J/j] - I_2[J/j]: bS$, as required.
  \item[(I-Times-$\R^+$)] Suppose $\Theta , j : S \vdash c \cdot I : \R$ from
  $\Theta, i : S \vdash I : \R$ and
  $c \in \R$.
  By IH,
  $\Theta \pvdash I[J/j] : \R$
  and so by I-Times-$\R^+$,
  $\Theta \pvdash c \cdot I[J/j] : \R$
  \item[(I-Times-$\vec{R}^+$)] Identical to I-Times-$\R^+$.
  \item[(I-Times-$\N$)] Identical to I-Times-$\R^+$.
  \item[(I-Shift)] Suppose $\Theta, j : S ; \Delta \vdash \lhd I : \potvec$ by way of $\Theta, j : S ; \Delta \vdash I : \potvec$.
  By IH,
  $\Theta ; \Delta \pvdash I[J/j] : \potvec$,
  and by I-Shift,
  $\Theta ; \Delta \pvdash \lhd I[J/j] : \potvec$
  as required.
  \item[(I-Lam)] Suppose $\Theta, j : S_1 ; \Delta \vdash \lambda i : S_2. I : S_2 \to S_3$ by way of
  $\Theta, j : S_1, i : S_2 ; \Delta \vdash I : S_3$.
  By IH,
  $\Theta, i : S_2 ; \Delta \pvdash I[J/j] : S_3$,
  and by I-Lam,
  $\Theta ; \Delta \pvdash \lambda i : S_2. I[J/j] : S_2 \to S_3$ as required.
  \item[(I-App)] Suppose $\Theta, j : S ; \Delta \vdash I_1 \; I_2 : S_2$ by way of
  $\Theta, j : S; \Delta \vdash I_1 : S_1 \to S_2$ and
  $\Theta, j : S; \Delta \vdash I_2 : S_1$.
  By IH twice, we have that
  $\Theta ; \Delta \pvdash I_1[J/j] : S_1 \to S_2$ and
  $\Theta ; \Delta \pvdash I_2[J/j] : S_1$.
  By I-App,
  $\Theta ; \Delta \pvdash I_1[J/j] \; I_2[J/j] : S_2$, as required.
  \item[(I-Const)] Suppose $\Theta, j : S ; \Delta \vdash \texttt{const}(I) : \potvec$
  by way of
  $\Theta, j : S ; \Delta \vdash I : \R$.
  By IH,
  $\Theta ; \Delta \pvdash I[J/j] : \R$,
  and so by I-Const,
  $\Theta ; \Delta \pvdash \texttt{const}(I)[J/j] : \potvec$
  as required.
  \item[(I-\* -Lit)] Immediate by the presupposition that $\Theta \vdash \Delta \; \texttt{wf}$.
\end{itemize}
\end{proof}
\fi

\begin{theorem}[Index Substitution for Constraint Well-Formedness]
If $\Theta, i : S ; \Delta \pvdash \Phi \; \texttt{wf}$ and $\Theta ; \Delta \pvdash I : S$ then $\Theta ; \Delta \pvdash \Phi[I/i] \; \texttt{wf}.$
\label{thm:constr-idx-subst}
\end{theorem}
\begin{proof}
By induction on the derivation of $\Theta, i ; S ; \Delta \vdash \Phi \; \texttt{wf}$.
\begin{enumerate}
  \item[(C-Top)] Immediate.
  \item[(C-Bot)] Immediate.
  \item[(C-Conj)] Suppose $\Theta, i : S ; \Delta \vdash \Phi_1 \wedge \Phi_2 \; \texttt{wf}$ by way of
  $\Theta, i : S ; \Delta \vdash \Phi_1 \; \texttt{wf}$ and
  $\Theta, i : S ; \Delta \vdash \Phi_2 \; \texttt{wf}$.
  By IH twice,
  $\Theta; \Delta \pvdash \Phi_1[I/i] \; \texttt{wf}$ and
  $\Theta; \Delta \pvdash \Phi_2[I/i] \; \texttt{wf}$.
  Then, by C-Conj,
  $\Theta ; \Delta \pvdash \Phi_1[I/i] \wedge \Phi_2[I/i] \; \texttt{wf}$
  as required.
  \item[(C-Disj)] Suppose $\Theta, i : S ; \Delta \vdash \Phi_1 \vee \Phi_2 \; \texttt{wf}$ by way of
  $\Theta, i : S ; \Delta \vdash \Phi_1 \; \texttt{wf}$ and
  $\Theta, i : S ; \Delta \vdash \Phi_2 \; \texttt{wf}$.
  By IH twice,
  $\Theta; \Delta \pvdash \Phi_1[I/i] \; \texttt{wf}$ and
  $\Theta; \Delta \pvdash \Phi_2[I/i] \; \texttt{wf}$.
  Then, by C-Disj,
  $\Theta ; \Delta \pvdash \Phi_1[I/i] \vee \Phi_2[I/i] \; \texttt{wf}$
  as required.
  
  \item[(C-Impl)]  Suppose $\Theta, i : S ; \Delta \vdash \Phi_1 \to \Phi_2 \; \texttt{wf}$ by way of
  $\Theta, i : S ; \Delta \vdash \Phi_1 \; \texttt{wf}$ and
  $\Theta, i : S ; \Delta, \Phi_1 \vdash \Phi_2 \; \texttt{wf}$.
  By IH twice,
  $\Theta; \Delta \pvdash \Phi_1[I/i] \; \texttt{wf}$ and
  $\Theta; \Delta, \Phi_1 \pvdash \Phi_2[I/i] \; \texttt{wf}$.
  Then, by C-Impl,
  $\Theta ; \Delta \pvdash \Phi_1[I/i] \to \Phi_2[I/i] \; \texttt{wf}$
  as required.
  
  \item[(C-Forall)] Suppose $\Theta, i : S ; \Delta \vdash \forall j : S'. \Phi \; \texttt{wf}$ by way of
  $\Theta, i : S, j : S' ; \Delta \vdash\Phi \; \texttt{wf}$.
  By IH,
  $\Theta, j : S' ; \Delta \pvdash \Phi[I/i] \; \texttt{wf}$.
  By C-Forall,
  $\Theta;\Delta \pvdash \forall j : S'. \Phi[I/i] \; \texttt{wf}$, as required.
  
  \item[(C-Exists)] Suppose $\Theta, i : S ; \Delta \vdash \exists j : S'. \Phi \; \texttt{wf}$ by way of
  $\Theta, i : S, j : S' ; \Delta \vdash\Phi \; \texttt{wf}$.
  By IH,
  $\Theta, j : S' ; \Delta \pvdash \Phi[I/i] \; \texttt{wf}$.
  By C-Exists,
  $\Theta;\Delta \pvdash \exists j : S'. \Phi[I/i] \; \texttt{wf}$, as required.
  
  \item[(C-Eq)] Immediate by Theorem~\ref{thm:idx-idx-subst}.
  \item[(C-Leq)] Immediate by Theorem~\ref{thm:idx-idx-subst}.
  \item[(C-Lt)] Immediate by Theorem~\ref{thm:idx-idx-subst}.
\end{enumerate}
\end{proof}


\begin{theorem}[Admissibility of Weakening for Type Formation]
If $\Psi ; \Theta ; \Delta \pvdash \tau : K$, $\Psi' \supseteq \Psi$, and $\Theta' \supseteq \Theta$, then
$\Psi' ; \Theta' ; \Delta \pvdash \tau : K$.
\end{theorem}

\begin{theorem}[Admissibility of Weakening for Term Context Wellformedness]
If $\Psi ; \Theta ; \Delta \vdash \Gamma \; \texttt{wf}$, $\Psi' \supseteq \Psi$, and $\Theta' \supseteq \Theta$, then
$\Psi' ; \Theta' ; \Delta \vdash \Gamma \; \texttt{wf}$.
\end{theorem}

\typeidxsubst*
\begin{proof}
By induction on the derivation of $\Psi ; \Theta, i : S ; \Delta \vdash \tau : K$.
\begin{enumerate}
  \item[(K-Var)] Immediate.
  \item[(K-Unit)] Immediate.
  \item[(K-Arr)] Suppose $\Psi ; \Theta, i : S ; \Delta \vdash \tau_1 \loli \tau_2 : \star$ from
  $\Psi ; \Theta, i : S ; \Delta \vdash \tau_1 : \star$ and
  $\Psi ; \Theta, i : S ; \Delta \vdash \tau_2 : \star$.
  By IH twice, we have that
  $\Psi ; \Theta ; \Delta \pvdash \tau_1[I/i] : \star$ and
  $\Psi ; \Theta ; \Delta \pvdash \tau_2[I/i] : \star$.
  But then by K-Arr, we have
  $\Psi ; \Theta ; \Delta \pvdash \tau_1[I/i] \loli \tau_2[I/i]: \star$
  as required.
  \item[(K-Tensor)] Suppose $\Psi ; \Theta, i : S ; \Delta \vdash \tau_1 \otimes \tau_2 : \star$ from
  $\Psi ; \Theta, i : S ; \Delta \vdash \tau_1 : \star$ and
  $\Psi ; \Theta, i : S ; \Delta \vdash \tau_2 : \star$.
  By IH twice, we have that
  $\Psi ; \Theta ; \Delta \pvdash \tau_1[I/i] : \star$ and
  $\Psi ; \Theta ; \Delta \pvdash \tau_2[I/i] : \star$.
  But then by K-Tensor, we have
  $\Psi ; \Theta ; \Delta \pvdash \tau_1[I/i] \otimes \tau_2[I/i]: \star$
  as required.
  \item[(K-With)] Suppose $\Psi ; \Theta, i : S ; \Delta \vdash \tau_1 \amp \tau_2 : \star$ from
  $\Psi ; \Theta, i : S ; \Delta \vdash \tau_1 : \star$ and
  $\Psi ; \Theta, i : S ; \Delta \vdash \tau_2 : \star$.
  By IH twice, we have that
  $\Psi ; \Theta ; \Delta \pvdash \tau_1[I/i] : \star$ and
  $\Psi ; \Theta ; \Delta \pvdash \tau_2[I/i] : \star$.
  But then by K-With, we have
  $\Psi ; \Theta ; \Delta \pvdash \tau_1[I/i] \amp \tau_2[I/i]: \star$
  as required.
  \item[(K-Sum)] Suppose $\Psi ; \Theta, i : S ; \Delta \vdash \tau_1 \oplus \tau_2 : \star$ from
  $\Psi ; \Theta, i : S ; \Delta \vdash \tau_1 : \star$ and
  $\Psi ; \Theta, i : S ; \Delta \vdash \tau_2 : \star$.
  By IH twice, we have that
  $\Psi ; \Theta ; \Delta \pvdash \tau_1[I/i] : \star$ and
  $\Psi ; \Theta ; \Delta \pvdash \tau_2[I/i] : \star$.
  But then by K-Sum, we have
  $\Psi ; \Theta ; \Delta \pvdash \tau_1[I/i] \oplus \tau_2[I/i]: \star$
  as required.
  \item[(K-Bang)] Suppose $\Psi ; \Theta, i : S ; \Delta \vdash !\tau : \star$ by way of
  $\Psi ; \Theta, i : S ; \Delta \vdash \tau : \star$.
  By IH,
  $\Psi ; \Theta ; \Delta \pvdash \tau[I/i] : \star$,
  and so by K-Bang,
  $\Psi ; \Theta ; \Delta \pvdash \tau[I/i] : !\star$.
  \item[(K-IForall)] Suppose $\Psi ; \Theta, i : S;  \Delta \vdash \forall j : S'. \tau : \star$ by way of
  $\Psi ; \Theta, i : S, j : S' ; \Delta \vdash \tau : \star$.
  By IH,
  $\Psi ; \Theta, j : S' ; \Delta \pvdash \tau[I/i] : \star$.
  By K-IForall,
  $\Psi ; \Theta ; \Delta \pvdash \forall j : S', \tau[I/i] : \star$, as required.
  \item[(K-IExists)] Suppose $\Psi ; \Theta, i : S;  \Delta \vdash \exists j : S'. \tau : \star$ by way of
  $\Psi ; \Theta, i : S, j : S' ; \Delta \vdash \tau : \star$.
  By IH,
  $\Psi ; \Theta, j : S' ; \Delta \pvdash \tau[I/i] : \star$.
  By K-IExists,
  $\Psi ; \Theta ; \Delta \pvdash \exists j : S', \tau[I/i] : \star$, as required.
  \item[(K-List)] Suppose $\Psi ; \Theta, i : S ; \Delta \vdash L^J \tau : \star$ by way of
  $\Theta, i : S ; \Delta \vdash J : \N$ and
  $\Psi ; \Theta, i : S ; \Delta \vdash \tau : \star$.
  By \autoref{thm:idx-idx-subst},
  $\Theta ; \Delta \pvdash J[I/i] : \N$.
  By IH,
  $\Psi ; \Theta ; \Delta \pvdash \tau[I/i] : \star$.
  By K-List,
  $\Psi ; \Theta l \Delta \pvdash L^{J[I/i]}\left(\tau[I/i]\right)$,
  as required.
  
  \item[(K-Impl)] Suppose $\Psi ; \Theta, i : S ; \Delta \vdash \Phi \implies \tau : \star$ by way of
  $\Psi ; \Theta, i : S ; \Delta \vdash \tau : \star$ and
  $\Theta, i : S ; \Delta \vdash \Phi \; \texttt{wf}$.
  By IH,
  $\Psi ; \Theta ; \Delta \pvdash \tau[I/i] : \star$.
  By \autoref{thm:constr-idx-subst},
  $\Theta ; \Delta \pvdash \Phi[I/i] \; \texttt{wf}$.
  Then, by K-Conj,
  $\Psi ; \Theta ; \Delta \vdash \Phi[I/i] \amp \tau[I/i] : \star$
  as required.
  Suppose $\Psi ; \Theta, i : S ; \Delta \vdash \Phi \amp \tau : \star$ by way of
  $\Psi ; \Theta, i : S ; \Delta \vdash \tau : \star$ and
  $\Theta, i : S ; \Delta \vdash \Phi \; \texttt{wf}$.
  By IH,
  $\Psi ; \Theta ; \Delta \pvdash \tau[I/i] : \star$.
  By \autoref{thm:constr-idx-subst},
  $\Theta ; \Delta \pvdash \Phi[I/i] \; \texttt{wf}$.
  Then, by K-Implies,
  $\Psi ; \Theta ; \Delta \pvdash \Phi[I/i] \implies \tau[I/i] : \star$
  as required.
  
  \item[(K-Monad)] Suppose
  $\Psi ; \Theta, i : S ; \Delta \vdash \M(J,\vec{p}) \tau : \star$
  by way of
  $\Theta, i : S ; \Delta \vdash J : \mathbb{N}$,
  $\Theta, i : S ; \Delta \vdash \vec{p} : \potvec$, and
  $\Psi ; \Theta, i : S ; \Delta \vdash \tau : \star$.
  By \autoref{thm:idx-idx-subst}.
  $\Theta ; \Delta \pvdash J[I/i] : \mathbb{N}$ and
  $\Theta ; \Delta \pvdash \vec{p}[I/i] : \potvec$.
  By IH,
  $\Psi ; \Theta ; \Delta \pvdash \tau[I/i] : \star$.
  Then, by K-Monad,
  $\Psi ; \Theta  \Delta \pvdash \M(J[I/i],\vec{p}[I/i]) \tau[I/i] : \star$
  
  \item[(K-Pot)] Suppose
  $\Psi ; \Theta, i : S ; \Delta \vdash [J|\vec{p}] \tau : \star$
  by way of
  $\Theta, i : S ; \Delta \vdash J : \mathbb{N}$,
  $\Theta, i : S ; \Delta \vdash \vec{p} : \potvec$, and
  $\Psi ; \Theta, i : S ; \Delta \vdash \tau : \star$.
  By \autoref{thm:idx-idx-subst}.
  $\Theta ; \Delta \pvdash J[I/i] : \mathbb{N}$ and
  $\Theta ; \Delta \pvdash \vec{p}[I/i] : \potvec$.
  By IH,
  $\Psi ; \Theta ; \Delta \pvdash \tau[I/i] : \star$.
  Then, by K-Pot,
  $\Psi ; \Theta  \Delta \pvdash \left[J[I/i],\vec{p}[I/i]\right] \tau[I/i] : \star$
  
  \item[(K-ConstPot)] Suppose
  $\Psi ; \Theta, i : S ; \Delta \vdash [J] \; \tau : \star$
  by way of
  $\Psi ; \Theta, i : S ; \Delta \vdash \tau : \star$ and
  $\Theta, i : S ; \Delta \vdash J : \potvec$.
  By IH,
  $\Psi ; \Theta ; \Delta \pvdash \tau[I/i] : \star$.
  By \autoref{thm:idx-idx-subst},
  $\Theta ; \Delta \pvdash J[I/i] : \potvec$.
  By K-ConstPot,
  $\Psi ; \Theta ; \Delta \pvdash \left[J[I/i]\right] \; \left(\tau[I/i]\right) : \star$
  as required.
  
  \item[(K-FamLam)] Suppose
  $\Psi ; \Theta, i : S ; \Delta \vdash \lambda j : S'. \tau : S' \to K$
  by way of
  $\Psi ; \Theta, i : S, j ; S' ; \Delta \vdash \tau : K$.
  By IH,
  $\Psi ; \Theta, j ; S' ; \Delta \pvdash \tau[I/i] : K$.
  By K-FamLam,
  $\Psi ; \Theta ; \Delta \pvdash \lambda j : S'.\tau[I/i] : S' \to K$
  as required.
  
  \item[(K-FamApp)] Suppose
  $\Psi ; \Theta, i : S ; \Delta \vdash \tau \; J : K$ by way of
  $\Theta, i : S ; \Delta \vdash J : S$ and
  $\Psi ; \Theta, i : S ; \Delta \vdash \tau : S \to K$.
  By \autoref{thm:idx-idx-subst},
  $\Theta ; \Delta \pvdash J[I/i] : S$.
  By IH,
  $\Psi ; \Theta ; \Delta \pvdash \tau[I/i] : S \to K$.
  By K-FamApp,
  $\Psi ; \Theta ; \Delta \pvdash \left(\tau[I/i]\right) \; \left(J[I/i]\right) : K$
  as required.
  
\end{enumerate}
\end{proof}

\begin{theorem}[Admissibility of Weakening for Subtyping]
Suppose $\Psi ; \Theta ; \Delta \pvdash \tau \subty \tau' : K$, $\Psi' \supseteq \Psi$, and $\Theta' \supseteq \Theta$.
Then, $\Psi' ; \Theta' ; \Delta \pvdash \tau \subty \tau'$.
\end{theorem}

\subtystreng*
\begin{proof}
Routine induction.
\end{proof}


\begin{theorem}[Context Subsumption Includes Subset]
If $\Psi ; \Theta ; \Delta \pvdash \Gamma'$ and $\Gamma \subseteq \Gamma'$ as sets, then $\Psi ; \Theta ; \Delta \pvdash \Gamma' \wknto \Gamma$
\label{thm:ctx-sub-subset1}
\end{theorem}
\begin{proof}
By induction on $|\Gamma|$.
If $\Gamma = \emptyset$, this is immediate by CS-Emp.
Now suppose $\Gamma = \Gamma'', x : \tau$. Then, since $\Gamma \subseteq \Gamma'$, we have $x : \tau \in \Gamma'$, and $\Gamma'' \subseteq \Gamma' \setminus \{x : \tau\}$. Moreover, by S-Refl, $\Psi ; \Theta ; \Delta \vdash \tau \subty \tau : \star$, and so we are done by IH.
\end{proof}

\ctxsubsubset*
\begin{proof}
By an easy induction on $|\Gamma|$.
\end{proof}

\begin{theorem}[Admissibility of Weakening for Context Subsumption]
If $\Psi ; \Theta ; \Delta \pvdash \Gamma \wknto \Gamma'$ and $\Psi' \supseteq \Psi$, $\Theta' \supseteq \Theta$, and $\Delta' \supseteq \Delta$, then
$\Psi' ; \Theta' ; \Delta' \pvdash \Gamma \wknto \Gamma'$
\label{thm:ctx-sub-wkn}
\end{theorem}

\begin{theorem}[Strengthening for Context Subsumption]
Suppose $\Psi ; \Theta ; \Delta \vdash \Gamma,\Gamma' \texttt{ wf}$.
If $\Psi' ; \Theta' ; \Delta \pvdash \Gamma \wknto \Gamma'$ with $\Theta' \supseteq \Theta$ and $\Psi' \supseteq \Psi$, then $\Psi ; \Theta ; \Delta \pvdash \Gamma \wknto \Gamma'$.
\end{theorem}
\label{thm:ctx-sub-streng}
\begin{proof}
Immediate by Theorem~\ref{thm:ctx-sub-subset2} and Theorem~\ref{thm:subty-streng}
\end{proof}

\ctxsubswap*
\begin{proof}
By Theorem~\ref{thm:ctx-sub-subset2}, it suffices to show that for every $x : \tau \in \Gamma_2'$, there is some $\tau'$ such that $x : \tau' \in \Gamma_2'$,
and $\Psi ; \Theta ; \Delta \vdash \tau' \subty \tau : \star$. Suppose $x : \tau \in \Gamma_2'$. Then, $x : \tau \in \Gamma_1'$. Further, since $\Psi' ; \Theta' ;  \Delta' \vdash \Gamma_2 \wknto \Gamma_2'$, there is some $\tau'$ such that $x : \tau' \in \Gamma_2$. But then, $x : \tau' \in \Gamma_1$, and so since  $\Psi ; \Theta ; \Delta \vdash \Gamma_1 \wknto \Gamma_1'$, we have that $\Psi ; \Theta ; \Delta \vdash \tau' \subty \tau : \star$, by Theorem~\ref{thm:ctx-sub-subset2}.
\end{proof}

\subsection{Normalization}

\canonforms*
\begin{proof}
Inversion on $\Psi ;\Theta ; \Delta \vdash \tau : S \to K$  and then $\tau \; \texttt{nf}$.
\end{proof}

\idxsubstnf*
\begin{proof}
By an easy simultaneous induction. This is only true because we don't require index terms inside a type to be in normal form for the type to be in normal form.
\end{proof}

\idxsubsteval*
\begin{proof}
Induction on $\tau$.
\end{proof}

%Define $\# \tau$ to be the number of type connectives in $\tau$.

\normthm*
\begin{proof}
%By induction on $\# \tau$. The base cases (base type and type variables) are immediate. For the inductive case, we break into cases on the syntax of $\tau$, inverting $\Psi ;\Theta ; \Delta \vdash \tau : K$-- since this judgment is syntax directed, we may write this as if it were cases over the derivation.

By induction on $\Psi ; \Theta ; \Delta \vdash \tau : K$.

\begin{enumerate}
  \item[(K-Var)] Immediate.
  \item[(K-Unit)] Immediate.  
  \item[(K-Arr)] Suppose ${\Psi ; \Theta ; \Delta \vdash \tau_1 \loli \tau_2 : \star}$ from $\Psi ; \Theta ; \Delta \vdash \tau_1 : \star$ and  $\Psi ; \Theta ; \Delta \vdash \tau_2 : \star$. By IH, we have for $i \in \{1,2\}$
  \begin{enumerate}[1.]
    \item $\Psi ; \Theta ; \Delta \vdash \texttt{eval}(\tau_i) : \star$
    \item $\Psi ; \Theta ; \Delta \vdash \tau_i \equiv \texttt{eval}(\tau_i) : \star$
    \item $\texttt{eval}(\tau_i) \; \texttt{nf}$.
  \end{enumerate}
  Note that $\texttt{eval}(\tau_1 \loli \tau_2) = \texttt{eval}(\tau_1) \loli \texttt{eval}(\tau_2)$. Then,
  \begin{enumerate}[1.]
    \item ${\Psi ; \Theta ; \Delta \vdash \texttt{eval}(\tau_1) \loli \texttt{eval}(\tau_2) : \star}$ by K-Arr
    \item $\Psi ; \Theta ; \Delta \vdash \tau_1 \loli \tau_2 \equiv \texttt{eval}(\tau_1) \loli \tau_2 : \star$ by two uses of S-Arr
    \item Since $\tau_i \; \texttt{nf}$, we have that $\tau_1 \loli \tau_2 \; \texttt{nf}$.
  \end{enumerate}
  as required.
  \item[(K-Tensor)] Suppose ${\Psi ; \Theta ; \Delta \vdash \tau_1 \otimes \tau_2 : \star}$ from $\Psi ; \Theta ; \Delta \vdash \tau_1 : \star$ and  $\Psi ; \Theta ; \Delta \vdash \tau_2 : \star$. By IH, we have for $i \in \{1,2\}$
  \begin{enumerate}[1.]
    \item $\Psi ; \Theta ; \Delta \vdash \texttt{eval}(\tau_i) : \star$
    \item $\Psi ; \Theta ; \Delta \vdash \tau_i \equiv \texttt{eval}(\tau_i) : \star$
    \item $\texttt{eval}(\tau_i) \; \texttt{nf}$.
  \end{enumerate}
  Note that $\texttt{eval}(\tau_1 \otimes \tau_2) = \texttt{eval}(\tau_1) \otimes \texttt{eval}(\tau_2)$. Then,
  \begin{enumerate}[1.]
    \item ${\Psi ; \Theta ; \Delta \vdash \texttt{eval}(\tau_1) \loli \texttt{eval}(\tau_2) : \star}$ by K-Tensor
    \item $\Psi ; \Theta ; \Delta \vdash \tau_1 \otimes \tau_2 \equiv \texttt{eval}(\tau_1) \otimes \tau_2 : \star$ by two uses of S-Tensor
    \item Since $\tau_i \; \texttt{nf}$, we have that $\tau_1 \otimes \tau_2 \; \texttt{nf}$.
  \end{enumerate}
  as required.
  \item[(K-With)] Suppose ${\Psi ; \Theta ; \Delta \vdash \tau_1 \amp \tau_2 : \star}$ from $\Psi ; \Theta ; \Delta \vdash \tau_1 : \star$ and  $\Psi ; \Theta ; \Delta \vdash \tau_2 : \star$. By IH, we have for $i \in \{1,2\}$
  \begin{enumerate}[1.]
    \item $\Psi ; \Theta ; \Delta \vdash \texttt{eval}(\tau_i) : \star$
    \item $\Psi ; \Theta ; \Delta \vdash \tau_i \equiv \texttt{eval}(\tau_i) : \star$
    \item $\texttt{eval}(\tau_i) \; \texttt{nf}$.
  \end{enumerate}
  Note that $\texttt{eval}(\tau_1 \amp \tau_2) = \texttt{eval}(\tau_1) \amp \texttt{eval}(\tau_2)$. Then,
  \begin{enumerate}[1.]
    \item ${\Psi ; \Theta ; \Delta \vdash \texttt{eval}(\tau_1) \amp \texttt{eval}(\tau_2) : \star}$ by K-With
    \item $\Psi ; \Theta ; \Delta \vdash \tau_1 \amp \tau_2 \equiv \texttt{eval}(\tau_1) \amp \tau_2 : \star$ by two uses of S-With
    \item Since $\tau_i \; \texttt{nf}$, we have that $\tau_1 \amp \tau_2 \; \texttt{nf}$.
  \end{enumerate}
  as required.
  \item[(K-Sum)] Suppose ${\Psi ; \Theta ; \Delta \vdash \tau_1 \oplus \tau_2 : \star}$ from $\Psi ; \Theta ; \Delta \vdash \tau_1 : \star$ and  $\Psi ; \Theta ; \Delta \vdash \tau_2 : \star$. By IH, we have for $i \in \{1,2\}$
  \begin{enumerate}[1.]
    \item $\Psi ; \Theta ; \Delta \vdash \texttt{eval}(\tau_i) : \star$
    \item $\Psi ; \Theta ; \Delta \vdash \tau_i \equiv \texttt{eval}(\tau_i) : \star$
    \item $\texttt{eval}(\tau_i) \; \texttt{nf}$.
  \end{enumerate}
  Note that $\texttt{eval}(\tau_1 \amp \tau_2) = \texttt{eval}(\tau_1) \oplus \texttt{eval}(\tau_2)$. Then,
  \begin{enumerate}[1.]
    \item ${\Psi ; \Theta ; \Delta \vdash \texttt{eval}(\tau_1) \oplus \texttt{eval}(\tau_2) : \star}$ by K-Sum
    \item $\Psi ; \Theta ; \Delta \vdash \tau_1 \oplus \tau_2 \equiv \texttt{eval}(\tau_1) \oplus \tau_2 : \star$ by two uses of S-Sum
    \item Since $\tau_i \; \texttt{nf}$, we have that $\tau_1 \oplus \tau_2 \; \texttt{nf}$.
  \end{enumerate}
  as required.
  \item[(K-Bang)] Suppose ${\Psi ; \Theta ; \Delta \vdash !\tau : \star}$ from ${\Psi ; \Theta ; \Delta \vdash \tau : \star}$.
  By IH, we have that
  \begin{enumerate}[1.]
   \item ${\Psi ; \Theta ; \Delta \vdash \texttt{eval}(\tau) : \star}$
   \item ${\Psi ; \Theta ; \Delta \vdash \tau \equiv \texttt{eval}(\tau) : \star}$
   \item $\texttt{eval}(\tau) \; \texttt{nf}$
  \end{enumerate}
  Then, noting that $\texttt{eval}(!\tau) = !\texttt{eval}(\tau)$,
  \begin{enumerate}[1.]
   \item ${\Psi ; \Theta ; \Delta \vdash !\texttt{eval}(\tau) : \star}$ by K-Bang
   \item ${\Psi ; \Theta ; \Delta \vdash !\tau \equiv !\texttt{eval}(\tau) : \star}$ by S-Bang
   \item Since $\texttt{eval}(\tau) \; \texttt{nf}$, $!\texttt{eval}(\tau) \; \texttt{nf}$
  \end{enumerate}
  as required.
  \item[(K-IForall)] Suppose ${\Psi ; \Theta ; \Delta \vdash \forall i : S. \tau : \star}$ from ${\Psi ; \Theta, i : S ; \Delta \vdash \tau : \star}$.
  By IH,
  \begin{enumerate}[1.]
   \item ${\Psi ; \Theta, i : S ; \Delta \vdash \texttt{eval}(\tau) : \star}$.
   \item ${\Psi ; \Theta, i : S ; \Delta \vdash \tau \equiv \texttt{eval}(\tau) : \star}$.
   \item $\texttt{eval}(\tau) \; \texttt{nf}$
  \end{enumerate}
  Then, noting that $\texttt{eval}(\forall i : S. \tau) = \forall i : S.\texttt{eval}(\tau)$, we have that
  \begin{enumerate}[1.]
   \item ${\Psi ; \Theta ; \Delta \vdash \forall i : S. \texttt{eval}(\tau) : \star}$ by K-IForall
   \item ${\Psi ; \Theta ; \Delta \vdash \forall i : S.\tau \equiv \forall i : S. \texttt{eval}(\tau) : \star}$ by S-IForall twice
   \item $\forall i : S. \texttt{eval}(\tau) \; \texttt{nf}$ since $\texttt{eval}(\tau) \; \texttt{nf}$
  \end{enumerate}
  as required.
  \item[(K-IExists)] Suppose ${\Psi ; \Theta ; \Delta \vdash \exists i : S. \tau : \star}$ from ${\Psi ; \Theta, i : S ; \Delta \vdash \tau : \star}$.
  By IH,
  \begin{enumerate}[1.]
   \item ${\Psi ; \Theta, i : S ; \Delta \vdash \texttt{eval}(\tau) : \star}$.
   \item ${\Psi ; \Theta, i : S ; \Delta \vdash \tau \equiv \texttt{eval}(\tau) : \star}$.
   \item $\texttt{eval}(\tau) \; \texttt{nf}$
  \end{enumerate}
  Then, noting that $\texttt{eval}(\exists i : S. \tau) = \exists i : S.\texttt{eval}(\tau)$, we have that
  \begin{enumerate}[1.]
   \item ${\Psi ; \Theta ; \Delta \vdash \exists i : S. \texttt{eval}(\tau) : \star}$ by K-IExists
   \item ${\Psi ; \Theta ; \Delta \vdash \exists i : S.\tau \equiv \forall i : S. \texttt{eval}(\tau) : \star}$ by S-IExists twice
   \item $\exists i : S. \texttt{eval}(\tau) \; \texttt{nf}$ since $\texttt{eval}(\tau) \; \texttt{nf}$
  \end{enumerate}
  as required.
  \item[(K-TForall)] Suppose ${\Psi ; \Theta ; \Delta \vdash \forall \alpha : K. \tau : \star}$ from ${\Psi, \alpha : K; \Theta ; \Delta \vdash \tau : \star}$.
  By IH,
  \begin{enumerate}[1.]
   \item ${\Psi, \alpha : K ; \Theta ; \Delta \vdash \texttt{eval}(\tau) : \star}$.
   \item ${\Psi, \alpha : K ; \Theta ; \Delta \vdash \tau \equiv \texttt{eval}(\tau) : \star}$.
   \item $\texttt{eval}(\tau) \; \texttt{nf}$
  \end{enumerate}
  Then, noting that $\texttt{eval}(\forall \alpha : K. \tau) = \forall \alpha : K.\texttt{eval}(\tau)$, we have that
  \begin{enumerate}[1.]
   \item ${\Psi ; \Theta ; \Delta \vdash \forall \alpha : K. \texttt{eval}(\tau) : \star}$ by K-TForall
   \item ${\Psi ; \Theta ; \Delta \vdash \forall \alpha : K.\tau \equiv \forall \alpha : K. \texttt{eval}(\tau) : \star}$ by S-TForall twice
   \item $\forall \alpha : K. \texttt{eval}(\tau) \; \texttt{nf}$ since $\texttt{eval}(\tau) \; \texttt{nf}$
  \end{enumerate}
  as required.
  \item[(K-List)] Suppose ${\Psi ; \Theta ; \Delta \vdash L^I \tau : \star}$ from $\Psi ; \Theta ; \Delta \vdash \tau : \star$ and $\Theta ; \Delta \vdash I : \N$.
  By IH, we have
  \begin{enumerate}[1.]
    \item $\Psi ; \Theta ; \Delta \vdash \texttt{eval}(\tau) : \star$
    \item $\Psi ; \Theta ; \Delta \vdash \tau \equiv \texttt{eval}(\tau) : \star$
    \item $\texttt{eval}(\tau) \; \texttt{nf}$
  \end{enumerate}
  Then, recalling that $\texttt{eval}\left(L^I \,\tau\right) = L^I\left(\texttt{eval}(\tau)\right)$, we have
  \begin{enumerate}[1.]
    \item $\Psi ; \Theta ; \Delta \vdash L^I\left(\texttt{eval}(\tau)\right) : \star$ by K-List, with $\Theta ; \Delta \vdash I : \N$
    \item $\Psi ; \Theta ; \Delta \vdash L^I \, \tau \equiv L^I\left(\texttt{eval}(\tau)\right) : \star$ by S-List, using the fact that $\Theta ; \Delta \vDash I = I$
    \item $L^I\left(\texttt{eval}(\tau)\right) \; \texttt{nf}$ because $\texttt{eval}(\tau) \; \texttt{nf}$
  \end{enumerate}
  as required.
  \item[(K-Conj)] Suppose ${\Psi ; \Theta ; \Delta \vdash \Phi \amp \tau : \star}$ from $\Psi ; \Theta ; \Delta \vdash \tau : \star$ and $\Theta ; \Delta \vdash \Phi \texttt{ wf}$. By  IH, we have:
  \begin{enumerate}[1.]
    \item $\Psi ; \Theta ; \Delta \vdash \texttt{eval}(\tau) : \star$
    \item $\Psi ; \Theta ; \Delta \vdash \tau \equiv \texttt{eval}(\tau) : \star$
    \item $\texttt{eval}(\tau) \; \texttt{nf}$
  \end{enumerate}
  Then, noting that $\texttt{eval}(\Phi \amp \tau) = \Phi \amp \texttt{eval}(\tau)$, we can conclude:
  \begin{enumerate}[1.]
    \item $\Psi ; \Theta ; \Delta \vdash \Phi \amp\texttt{eval}(\tau) : \star$ by K-Conj with $\Theta ; \Delta \vdash \Phi \texttt{ wf}$.
    \item $\Psi ; \Theta ; \Delta \vdash \Phi \amp \tau \equiv \Phi \amp\texttt{eval}(\tau) : \star$ by two uses of S-Conj, using the fact that $\Theta ; \Delta \vDash \Phi \to \Phi$.
    \item $\Phi \amp\texttt{eval}(\tau) \; \texttt{nf}$ since $\texttt{eval}(\tau) \; \texttt{nf}$
  \end{enumerate}
  as required.
  \item[(K-Impl)] Suppose ${\Psi ; \Theta ; \Delta \vdash \Phi \implies \tau : \star}$ from $\Psi ; \Theta ; \Delta \vdash \tau : \star$ and $\Theta ; \Delta \vdash \Phi \texttt{ wf}$. By  IH, we have:
  \begin{enumerate}[1.]
    \item $\Psi ; \Theta ; \Delta \vdash \texttt{eval}(\tau) : \star$
    \item $\Psi ; \Theta ; \Delta \vdash \tau \equiv \texttt{eval}(\tau) : \star$
    \item $\texttt{eval}(\tau) \; \texttt{nf}$
  \end{enumerate}
  Then, noting that $\texttt{eval}(\Phi \implies \tau) = \Phi \implies \texttt{eval}(\tau)$, we can conclude:
  \begin{enumerate}[1.]
    \item $\Psi ; \Theta ; \Delta \vdash \Phi \implies\texttt{eval}(\tau) : \star$ by K-Impl with $\Theta ; \Delta \vdash \Phi \texttt{ wf}$.
    \item $\Psi ; \Theta ; \Delta \vdash \Phi \implies \tau \equiv \Phi \amp\texttt{eval}(\tau) : \star$ by two uses of S-Impl, using the fact that $\Theta ; \Delta \vDash \Phi \to \Phi$.
    \item $\Phi \implies\texttt{eval}(\tau) \; \texttt{nf}$ since $\texttt{eval}(\tau) \; \texttt{nf}$
  \end{enumerate}
  as required.
  \item[(K-Monad)] Suppose ${\Psi ; \Theta ; \Delta \vdash \M(I,\vec{p}) \tau : \star}$ from $\Psi ; \Theta ; \Delta \vdash \tau : \star$ with $ \Theta ; \Delta \vdash I : \mathbb{N}$ and $\Theta ; \Delta \vdash \vec{p} : \vec{\mathbb{R}^+}$. Then, by IH,
  \begin{enumerate}[1.]
    \item $\Psi ; \Theta ; \Delta \vdash \texttt{eval}(\tau) : \star$
    \item $\Psi ; \Theta ; \Delta \vdash \tau \equiv \texttt{eval}(\tau) : \star$
    \item $\texttt{eval}(\tau) \; \texttt{nf}$
  \end{enumerate}
  Note that $\texttt{eval}(\M(I,\vec{p}) \tau) = \M(I,\vec{p})(\texttt{eval}(\tau))$, and so we may conclude:
  \begin{enumerate}
    \item $\Psi ; \Theta ; \Delta \vdash \M(I,\vec{p})(\texttt{eval}(\tau)) : \star$ by K-Monad with $ \Theta ; \Delta \vdash I : \mathbb{N}$ and $\Theta ; \Delta \vdash \vec{p} : \vec{\mathbb{R}^+}$
    \item $\Psi ; \Theta ; \Delta \vdash \M(I,\vec{p}) \tau \equiv \M(I,\vec{p})(\texttt{eval}(\tau)) : \star$ by two uses of S-Monad, using the fact that $\Theta ; \Delta \vdash (I = I) \wedge (\vec{p} \leq \vec{p})$.
    \item $\M(I,\vec{p})(\texttt{eval}(\tau)) \; \texttt{nf}$ since $\texttt{eval}(\tau) \; \texttt{nf}$
  \end{enumerate}
  \item[(K-Pot)] Suppose ${\Psi ; \Theta ; \Delta \vdash [I|\vec{p}] \tau : \star}$ from $\Psi ; \Theta ; \Delta \vdash \tau : \star$ with $\Theta ; \Delta \vdash I : \mathbb{N}$ and $\Theta ; \Delta \vdash \vec{p} : \vec{\mathbb{R}^+}$. By IH, we have that
    \begin{enumerate}[1.]
    \item $\Psi ; \Theta ; \Delta \vdash \texttt{eval}(\tau) : \star$
    \item $\Psi ; \Theta ; \Delta \vdash \tau \equiv \texttt{eval}(\tau) : \star$
    \item $\texttt{eval}(\tau) \; \texttt{nf}$
  \end{enumerate}
  Then, noting that $\texttt{eval}([I|\vec{p}] \tau) = [I|\vec{p}] (\texttt{eval}(\tau))$, we may conclude that
  \begin{enumerate}[1.]
    \item $\Psi ; \Theta ; \Delta \vdash [I|\vec{p}] (\texttt{eval}(\tau)) : \star$ by K-Pot with $\Theta ; \Delta \vdash I : \mathbb{N}$ and $\Theta ; \Delta \vdash \vec{p} : \vec{\mathbb{R}^+}$
    \item  $\Psi ; \Theta ; \Delta \vdash [I|\vec{p}] \tau \equiv [I|\vec{p}] (\texttt{eval}(\tau)) : \star$ by two uses of S-Pot, using the fact that $\Theta ; \Delta \vdash (I = I) \wedge (\vec{p} \leq \vec{p})$.
  \end{enumerate}
  
  \item[(K-ConstPot)] Suppose ${\Psi ; \Theta ; \Delta \vdash [I] \; \tau : \star}$ from $\Psi ; \Theta ; \Delta \vdash \tau : \star$ and $\Theta ; \Delta \vdash I : \mathbb{R}^+$. By IH,
  \begin{enumerate}[1.]
    \item $\Psi ; \Theta ; \Delta \vdash \texttt{eval}(\tau) : \star$
    \item $\Psi ; \Theta ; \Delta \vdash \tau \equiv \texttt{eval}(\tau) : \star$
    \item $\texttt{eval}(\tau) \; \texttt{nf}$
  \end{enumerate}
  Then, noting that $\texttt{eval}([I] \; \tau) = [I] \; \texttt{eval}(\tau)$, we can conclude:
  \begin{enumerate}
   \item $\Psi ; \Theta ; \Delta \vdash [I] \; \texttt{eval}(\tau) : \star$ by K-ConstPot with $\Theta ; \Delta \vdash I : \mathbb{R}^+$
   \item $\Psi ; \Theta ; \Delta \vdash [I] \; \tau \equiv [I] \; \texttt{eval}(\tau) : \star$ by S-ConstPot, using $\Theta ; \Delta \vDash I = I$.
   \item $[I] \; \texttt{eval}(\tau) \; \texttt{nf}$ because $\texttt{eval}(\tau) \; \texttt{nf}$
  \end{enumerate}
  as required.
  
  \item[(K-FamLam)] Suppose ${\Psi ; \Theta ; \Delta \vdash \lambda i : S. \tau : S \to K}$ from ${\Psi ; \Theta, i : S ; \Delta \vdash \tau : K}$. By IH,
  \begin{enumerate}[1.]
   \item $\Psi ; \Theta, i : S ; \Delta \vdash \texttt{eval}(\tau) : K$
   \item $\Psi ; \Theta, i : S ; \Delta\vdash \tau \equiv \texttt{eval}(\tau) : K$
   \item $\texttt{eval}(\tau) \; \texttt{nf}$
   %\item $\#\texttt{eval}(\tau) \leq \# \tau$
\end{enumerate}
  By definition, $\texttt{eval}(\lambda i : S. \tau) = \lambda i : S. \texttt{eval}(\tau)$. Then, we can proceed to prove the four claims:
  \begin{enumerate}[1.]
    \item By K-FamLam, $\Psi ; \Theta ; \Delta \vdash \lambda i : S. \texttt{eval}(\tau) : S \to K$.
    \item By S-FamLam in both directions, $\Psi ; \Theta ; \Delta\vdash \lambda i : S. \tau \equiv \lambda i : S. \texttt{eval}(\tau) : K$
    \item Since $\texttt{eval}(\tau) \; \texttt{nf}$, $\lambda i : S. \texttt{eval}(\tau) \; \texttt{nf}$ also.
    %\item Finally, $\#\texttt{eval}(\lambda i : S. \tau) = \#(\lambda i : S. \texttt{eval}(\tau)) = 1 + \#\texttt{eval}(\tau) \leq 1 + \#\tau = \#(\lambda i : S. \tau)$
  \end{enumerate}
  
  \item[(K-FamApp)] Suppose ${\Psi ; \Theta ; \Delta \vdash \tau \; I : K}$ from $\Psi ; \Theta ; \Delta \vdash \tau : S \to K$ and  $\Theta ; \Delta \vdash I : S$. By IH,
  we have that
  \begin{enumerate}[1.]
   \item $\Psi ; \Theta ; \Delta \vdash \texttt{eval}(\tau) : S \to K$
   \item $\Psi ; \Theta ; \Delta \vdash \tau \equiv \texttt{eval}(\tau) : S \to K$
   \item $\texttt{eval}(\tau) \; \texttt{nf}$
   %\item $\#\texttt{eval}(\tau) \leq \# \tau$
  \end{enumerate}
  By \autoref{thm:canon-forms}, we have two possibilities for $\texttt{eval}(\tau)$. First, suppose that  $\texttt{eval}(\tau) \; \texttt{ne}$. Then, $\texttt{eval}(\tau \; I) = \texttt{eval}(\tau) \; I$, and so we can easily prove the claims:
  \begin{enumerate}[1.]
   \item By K-FamApp, since $\Theta ; \Delta \vdash I : S$, we have that $\Psi ; \Theta ; \Delta \vdash \texttt{eval}(\tau) \; I : K$.
   \item By S-FamApp in both directions, we have that $\Psi ; \Theta ; \Delta \vdash \tau \; I \equiv \texttt{eval}(\tau) \; I : S \to K$
   \item Since $\texttt{eval}(\tau) \; \texttt{ne}$, we have that $\texttt{eval}(\tau) \; I \; \texttt{ne}$, and so $\texttt{eval}(\tau) \; I \; \texttt{nf}$.
   %\item $\#(\texttt{eval}(\tau) \; I) = 1 + \#\texttt{eval}(\tau) \leq 1 + \#\tau = \#(\tau \; I)$
  \end{enumerate}
  Otherwise, suppose that $\texttt{eval}(\tau) = \lambda i : S. \tau'$ with $\tau' \; \texttt{nf}$ and $\Psi ; \Theta , i : S ; \Delta \vdash \tau' : K$. In this case, $\texttt{eval}(\tau \; I) = \tau'[I/i]$:
  \begin{enumerate}[1.]
   \item By \textbf{Substitution} with $\Theta ; \Delta \vdash I : S$, we have that $\Psi ; \Theta ; \Delta \vdash \tau'[I/i] : K$.
   \item We already know that $\Psi ; \Theta ; \Delta \vdash \tau \equiv \texttt{eval}(\tau) : S \to K$, but $\texttt{eval}(\tau) = \lambda i : S. \tau'$, and so
   $\Psi ; \Theta ; \Delta \vdash \tau \equiv \lambda i : S.\tau' : S \to K$. By S-FamApp, $\Psi ; \Theta ; \Delta \vdash \tau \; I \equiv (\lambda i : S.\tau') \; I : K$. Postcomposing with both directions of S-Fam-Beta-\{1,2\}, we have that $\Psi ; \Theta ; \Delta \vdash \tau \; I \equiv \tau'[I/i] : K$, as required.
   \item Since $\tau'\; \texttt{nf}$, we have by \autoref{thm:idx-subst-nf} that $\tau'[I/i] \; \texttt{nf}$.
  \end{enumerate}
  
\end{enumerate}
\end{proof}

\begin{theorem}
~\begin{itemize}
  \item If $\tau \; \texttt{nf}$, then $\texttt{eval}(\tau) = \tau$
  \item If $\tau \; \texttt{ne}$, then $\texttt{eval}(\tau) = \tau$
\end{itemize}
\label{thm:norm-idemp}
\end{theorem}
\begin{proof}
Simultaneous induction.
\end{proof}

\section{Rules of \bilambdaamor}

\section{Soundness and Completeness}

\begin{theorem}[Raw Soundness of Sort Checking/Inference]
If $\Theta;\Delta \vdash I : S \gens \Phi$ and $\Theta;\Delta \vDash \Phi$, then $\Theta;\Delta \vdash I : S$ 
\label{thm:raw-sort-sound}
\end{theorem}
\jtheorem{Proof of \autoref{thm:raw-sort-sound}}{


\jgivengoal{
  \caseFact{1} $\Theta ; \Delta \vdash I : S \gens \Phi$
  
  \caseFact{2} $\Theta ; \Delta \vDash \Phi$
}{
  $\Theta ; \Delta \vdash I : S$
}

%\caseText{By induction on $\Theta ; \Delta \vdash I : S \gens \Phi$}

\jcase{1}{AI-Var}{Immediate.}

\jcase{2}{AI-Plus}{
  \jgivengoal{
    \caseFact{1} $\Theta ; \Delta \vdash I + J : bS \gens \Phi_1 \wedge \Phi_2$
    
    \caseFact{2} $\Theta ; \Delta \vDash \Phi_1 \wedge \Phi_2$
    
    \caseFact{3} $\Theta ; \Delta \vdash I : bS \gens \Phi_1$
    
    \caseFact{4} $\Theta ; \Delta \vdash J : bS \gens \Phi_2$
  }{
   $\Theta ; \Delta \vdash I + J : bS$  
  }
  
  \caseText{By IH on (3), since by (2) $\Theta ; \Delta \vDash \Phi_1$}
  
  \caseFact{5} $\Theta ; \Delta \vdash I : bS$
  
  \caseText{By IH on (4), since by (2) $\Theta ; \Delta \vDash \Phi_2$}
  
  \caseFact{6} $\Theta ; \Delta \vdash J : bS$  
  
  \caseText{Goal follows by I-Plus}
}

\jcase{3}{AI-Minus}{
  \jgivengoal{
    \caseFact{1} $\Theta ; \Delta \vdash I - J : bS \gens \Phi_1 \wedge \Phi_2 \wedge (I \geq J)$
    
    \caseFact{2} $\Theta ; \Delta \vDash \Phi_1 \wedge \Phi_2 \wedge (I \geq J)$
    
    \caseFact{3} $\Theta ; \Delta \vdash I : bS \gens \Phi_1$
    
    \caseFact{4} $\Theta ; \Delta \vdash J : bS \gens \Phi_2$
  }{
   $\Theta ; \Delta \vdash I - J : bS$    
  }
  
  \caseText{By IH on (3), since by (2) $\Theta ; \Delta \vDash \Phi_1$}
  
  \caseFact{5} $\Theta ; \Delta \vdash I : bS$
  
  \caseText{By IH on (4), since by (2) $\Theta ; \Delta \vDash \Phi_2$}
  
  \caseFact{6} $\Theta ; \Delta \vdash J : bS$  
  
  \caseText{Goal follows by I-Plus, using the fact that $\Theta ; \Delta \vDash I \geq J$ from (2)}
}

\jcase{4}{AI-Times-*}{Immediate by IH.}

\jcase{5}{AI-Shift}{
  \jgivengoal{
    \caseFact{1} $\Theta ; \Delta \vdash \lhd I : \potvec \gens \Phi$
    
    \caseFact{2} $\Theta ; \Delta \vDash \Phi$  
    
    \caseFact{3} $\Theta ; \Delta \vdash I : \potvec \gens \Phi$
  }{
    $\Theta ; \Delta \vdash \lhd I : \potvec$
  }
  
  \caseText{By IH on (3)}
  
  \caseFact{4} $\Theta ; \Delta \vdash I : \potvec$
  
  \caseText{Goal follows by I-Shift on (4)}
}

\jcase{6}{AI-Lam}{
  \jgivengoal{
    \caseFact{1} $\Theta ; \Delta \vdash \lambda i : S. I : S \to S' \gens \forall i : S. \Phi$
    
    \caseFact{2} $\Theta ; \Delta \vDash \forall i : S. \Phi$  
    
    \caseFact{3} $\Theta, i : S ; \Delta \vdash I : S' \gens \Phi$
  }{
    $\Theta ; \Delta \vdash \lambda i : S. I : S \to S'$
  }
  
  \caseText{Equivalently to (2)}
  
  \caseFact{4} $\Theta, i : S ; \Delta \vDash \Phi$
  
  \caseText{By IH on (3) with (4)}
  
  \caseFact{5} $\Theta, i : S; \Delta \vdash I : S'$
  
  \caseText{Goal follows by I-Lam on (5)}
}

\jcase{7}{AI-App}{
  \jgivengoal{
   \caseFact{1} $\Theta ; \Delta \vdash I \; J : S' \gens \Phi_1 \wedge \Phi_2$
   
   \caseFact{2} $\Theta ; \Delta \vDash \Phi_1 \wedge \Phi_2$
   
   \caseFact{3} $\Theta ; \Delta \vdash I : S \to S' \gens \Phi_1$
   
   \caseFact{4} $\Theta ; \Delta \vdash J : S \gens \Phi_2$
  }{
    $\Theta ; \Delta \vdash I \; J : S'$  
  }
  
  \caseText{By IH on (3)}
  
  \caseFact{5} $\Theta ; \Delta \vdash I : S \to S'$
  
  \caseText{By IH on (4)}
  
  \caseFact{6} $\Theta ; \Delta \vdash J : S$
  
  \caseText{Goal follows by I-App on (5) and (6)}
}

\jcase{8}{AI-Sum}{
  \jgivengoal{
    \caseFact{1} $\Theta;\Delta \vdash \sum_{i=I_0}^{I_1} J : bS \gens \Phi_1 \wedge \Phi_2 \wedge \forall i : \N.(I_0 \leq i \leq I_1 \to \Phi_3)$
    
    \caseFact{2} $\Theta ; \Delta \vDash \Phi_1 \wedge \Phi_2 \wedge \forall i : \N.(I_0 \leq i \leq I_1 \to \Phi_3)$
    
    \caseFact{3} $\Theta;\Delta \vdash I_0 : \mathbb{N} \gens \Phi_1$
    
    \caseFact{4}  $\Theta;\Delta \vdash I_1 : \mathbb{N} \gens \Phi_2$
    
    \caseFact{5} $\Theta,i : \N;\Delta, I_0 \leq i \leq I_1 \vdash J : bS \gens \Phi_3$
  }{
     $\Theta;\Delta \vdash \sum_{i=I_0}^{I_1} J : bS$
  }
  
  \caseText{From (2)}
  
  \caseFact{6} $\Theta, i : \N ; \Delta, (I_0 \leq i \leq I_1) \vDash \Phi_3$
  
  \caseText{By IH on (5) using (6)}
  
  \caseFact{7} $\Theta,i : \N;\Delta, I_0 \leq i \leq I_1 \vdash J : bS$
  
  \caseText{By IH on (3)}
  
  \caseFact{8} $\Theta;\Delta \vdash I_0 : \mathbb{N}$
  
  \caseText{By IH on (4)}
  
  \caseFact{9} $\Theta;\Delta \vdash I_1 : \mathbb{N}$
  
  \caseText{Goal follows by I-Sum on (7), (8), (9)}
}

\jcase{9}{AI-*-Lit}{Immediate.}

}

\begin{theorem}[Raw Soundness of Constraint Well-Formedness]
If $\Theta ; \Delta \vdash \Phi \gens \Phi'$ and $\Theta ; \Delta \vDash \Phi'$ then $\Theta ; \Delta \vdash \Phi \texttt{ wf}$
\label{thm:raw-constr-sound}
\end{theorem}
\begin{proof}
By induction on $\Theta ; \Delta \vdash \Phi \gens \Phi'$
\begin{itemize}
 \item[AC-Top] Immediate.
 \item[AC-Bot] Immediate.
 \item[AC-Conj] Suppose $\Theta;\Delta \vdash \Phi_1 \wedge \Phi_2 \gens \Phi_1' \wedge \Phi_2'$ from $\Theta ; \Delta \vdash \Phi_1 \gens \Phi_1'$ and $ \Theta ; \Delta \vdash \Phi_2 \gens \Phi_2'$ with $\Theta ; \Delta \vDash \Phi_1' \wedge \Phi_2'$. By IH, $\Theta ; \Delta \vdash \Phi_1 \texttt{ wf}$ and $\Theta ; \Delta \vdash \Phi_2 \texttt{ wf}$. By C-Conj, $\Theta;\Delta \vdash \Phi_1 \wedge \Phi_2 \texttt{ wf}$
 \item[AC-Disj]Suppose $\Theta;\Delta \vdash \Phi_1 \vee \Phi_2 \gens \Phi_1' \wedge \Phi_2'$ from $\Theta ; \Delta \vdash \Phi_1 \gens \Phi_1'$ and $ \Theta ; \Delta \vdash \Phi_2 \gens \Phi_2'$ with $\Theta ; \Delta \vDash \Phi_1' \wedge \Phi_2'$. By IH, $\Theta ; \Delta \vdash \Phi_1 \texttt{ wf}$ and $\Theta ; \Delta \vdash \Phi_2 \texttt{ wf}$. By C-Conj, $\Theta;\Delta \vdash \Phi_1 \vee \Phi_2 \texttt{ wf}$
 \item[AC-Impl] Suppose $\Theta;\Delta \vdash \Phi_1 \to \Phi_2 \gens \Phi_1' \wedge \Phi_2'$ from $\Theta ; \Delta \vdash \Phi_1 \gens \Phi_1'$ and $ \Theta ; \Delta \vdash \Phi_2 \gens \Phi_2'$ with $\Theta ; \Delta \vDash \Phi_1' \wedge \Phi_2'$. By IH, $\Theta ; \Delta \vdash \Phi_1 \texttt{ wf}$ and $\Theta ; \Delta \vdash \Phi_2 \texttt{ wf}$. By C-Conj, $\Theta;\Delta \vdash \Phi_1 \to \Phi_2 \texttt{ wf}$
 \item[AC-Forall] Suppose $\Theta ; \Delta \vdash \forall i : S. \Phi \gens \forall i : S. \Phi'$
 from $\Theta, i : S ; \Delta \vdash \Phi \gens \Phi'$ with $\Theta ; \Delta \vDash \forall i : S. \Phi'$. Equivalently, $\Theta , i : S; \Delta \vDash \Phi'$, and so by IH, $\Theta, i : S ; \Delta \vdash \Phi \texttt{ wf}$ and so by C-Forall, $\Theta ; \Delta \vdash \forall i : S. \Phi \texttt{ wf}$
 
 \item[AC-Exists] Suppose $\Theta ; \Delta \vdash \exists i : S. \Phi \gens \forall i : S. (\Phi \to \Phi')$ from $\Theta, i : S ; \Delta, \Phi \vdash \Phi \gens \Phi'$ with $\Theta ; \Delta \vDash \forall i : S.(\Phi \to \Phi')$. Equivalently, $\Theta , i : S ; \ Theta, \Phi \vDash \Phi'$. By IH, $\Theta, i : S ; \Delta, \Phi \vdash \Phi \texttt{ wf}$, and by C-Exists, $\Theta ; \Delta \vdash \forall i : S. \Phi$ as required.
 \item[AC-Eq] Suppose $\Theta ; \Delta \vdash I = J \gens \Phi_1 \wedge \Phi_2$ from $\Theta ; \Delta \vdash I : bS \gens \Phi_1$ and  $\Theta ; \Delta \vdash J : bS \gens \Phi_2$ with $\Theta ; \Delta \vDash \Phi_1 \wedge \Phi_2$. By \textbf{Soundness of Sort Checking}, $\Theta ; \Delta \vdash I : bS$ and $\Theta ; \Delta \vdash J : bS$, and so by C-Eq, $\Theta ; \Delta \vdash I = J$.
 \item[AC-Leq] Suppose $\Theta ; \Delta \vdash I \leq J \gens \Phi_1 \wedge \Phi_2$ from $\Theta ; \Delta \vdash I : bS \gens \Phi_1$ and  $\Theta ; \Delta \vdash J : bS \gens \Phi_2$ with $\Theta ; \Delta \vDash \Phi_1 \wedge \Phi_2$. By \textbf{Soundness of Sort Checking}, $\Theta ; \Delta \vdash I : bS$ and $\Theta ; \Delta \vdash J : bS$, and so by C-Leq, $\Theta ; \Delta \vdash I \leq J$.
 \item[AC-Lt] Suppose $\Theta ; \Delta \vdash I < J \gens \Phi_1 \wedge \Phi_2$ from $\Theta ; \Delta \vdash I : bS \gens \Phi_1$ and  $\Theta ; \Delta \vdash J : bS \gens \Phi_2$ with $\Theta ; \Delta \vDash \Phi_1 \wedge \Phi_2$. By \textbf{Soundness of Sort Checking}, $\Theta ; \Delta \vdash I : bS$ and $\Theta ; \Delta \vdash J : bS$, and so by C-Lt, $\Theta ; \Delta \vdash I < J$.
\end{itemize}
\end{proof}

\idxctxwfsound*
\begin{proof}
By induction on $\Theta \vdash \Delta \; \texttt{wf} \gens \Phi$. The base case is immediate. Suppose $\Theta \vdash \Delta, \Phi \; \texttt{wf} \gens \Phi_1 \wedge (\bigwedge \Delta \to \Phi_2)$ with $\Theta ; \cdot \vDash \Phi_1 \wedge (\bigwedge \Delta \to \Phi_2)$, by way of
$\Theta \vdash \Delta \; \texttt{wf} \gens \Phi_1$ and $\Theta ; \Delta \vdash \Phi \; \texttt{wf} \gens \Phi_2$.
Since $\Theta ; \cdot \vDash \Phi_1$, we have by IH that $\Theta \vdash \Delta \; \texttt{wf}$. By Theorem~\ref{thm:raw-sort-sound}, since $\Theta ; \Delta \vDash \Phi_2$, we have that $\Theta ; \Delta \vdash \Phi_2 \; \texttt{wf}$, and so $\Theta \vdash \Delta, \Phi \; \texttt{wf}$, as required.
\end{proof}

\sortsound*
\begin{proof}
Immediate by Theorem~\ref{thm:raw-sort-sound} and Theorem~\ref{thm:idx-ctx-wf-sound}
\end{proof}


\constrsound*
\begin{proof}
Immediate by Theorem~\ref{thm:raw-constr-sound} and Theorem~\ref{thm:idx-ctx-wf-sound}
\end{proof}

\begin{theorem}[Raw Soundness of Kind Checking/Inference]
If $\Psi ; \Theta ; \Delta \vdash \tau : K \gens \Phi$ and $\Theta ; \Delta \vDash \Phi$ then $\Psi ; \Theta ; \Delta \vdash \tau : K$.
\label{thm:raw-kind-sound}
\end{theorem}
\jtheorem{Proof of \autoref{thm:raw-kind-sound}}{
  \jgivengoal{
    \caseFact{1} $\Psi ; \Theta ; \Delta \vdash \tau : K \gens \Phi$  
    
    \caseFact{2} $Theta ; \Delta \vDash \Phi$
  }{
    $\Psi ; \Theta ; \Delta \vdash \ tau : K$  
  }
  
  \caseText{By Induction on $\Psi ; \Theta ; \Delta \vdash \tau : K \gens \Phi$}
  
  \jcase{1}{AK-Var}{Immediate.}
  
  \jcase{2}{AK-Unit}{Immediate.}
  
  \jcase{3}{AK-Arr}{
    \jgivengoal{
      \caseFact{1} $\Psi ; \Theta ; \Delta \vdash \tau_1 \loli \tau_2 : \star \gens \Phi_1 \wedge \Phi_2$    
      
      \caseFact{2} $\Theta ; \Delta \vDash \Phi_1 \wedge \Phi_2$
      
      \caseFact{3} $\Psi ; \Theta ; \Delta \vdash \tau_1 : \star \gens \Phi_1$
      
      \caseFact{4} $\Psi ; \Theta ; \Delta \vdash \tau_2 : \star \gens \Phi_2$
    }{
      $\Psi ; \Theta ; \Delta \vdash \tau_1 \loli \tau_2 : \star$
    }
    
    \caseText{By IH on (3)}
    
    \caseFact{5} $\Psi ; \Theta ; \Delta \vdash \tau_1 : \star$
    
    \caseText{By IH on (4)}
    
    \caseFact{6} $\Psi ; \Theta ; \Delta \vdash \tau_2 : \star$
    
    \caseText{Goal follows by K-Arr on (5) and (6)}
  }
  
  \jcase{4}{AK-Tensor}{
    \jgivengoal{
      \caseFact{1} $\Psi ; \Theta ; \Delta \vdash \tau_1 \otimes \tau_2 : \star \gens \Phi_1 \wedge \Phi_2$    
      
      \caseFact{2} $\Theta ; \Delta \vDash \Phi_1 \wedge \Phi_2$
      
      \caseFact{3} $\Psi ; \Theta ; \Delta \vdash \tau_1 : \star \gens \Phi_1$
      
      \caseFact{4} $\Psi ; \Theta ; \Delta \vdash \tau_2 : \star \gens \Phi_2$
    }{
      $\Psi ; \Theta ; \Delta \vdash \tau_1 \otimes \tau_2 : \star$
    }
    
    \caseText{By IH on (3)}
    
    \caseFact{5} $\Psi ; \Theta ; \Delta \vdash \tau_1 : \star$
    
    \caseText{By IH on (4)}
    
    \caseFact{6} $\Psi ; \Theta ; \Delta \vdash \tau_2 : \star$
    
    \caseText{Goal follows by K-Tensor on (5) and (6)}
  }
  
  \jcase{5}{AK-With}{
    \jgivengoal{
      \caseFact{1} $\Psi ; \Theta ; \Delta \vdash \tau_1 \amp \tau_2 : \star \gens \Phi_1 \wedge \Phi_2$    
      
      \caseFact{2} $\Theta ; \Delta \vDash \Phi_1 \wedge \Phi_2$
      
      \caseFact{3} $\Psi ; \Theta ; \Delta \vdash \tau_1 : \star \gens \Phi_1$
      
      \caseFact{4} $\Psi ; \Theta ; \Delta \vdash \tau_2 : \star \gens \Phi_2$
    }{
      $\Psi ; \Theta ; \Delta \vdash \tau_1 \amp \tau_2 : \star$
    }
    
    \caseText{By IH on (3)}
    
    \caseFact{5} $\Psi ; \Theta ; \Delta \vdash \tau_1 : \star$
    
    \caseText{By IH on (4)}
    
    \caseFact{6} $\Psi ; \Theta ; \Delta \vdash \tau_2 : \star$
    
    \caseText{Goal follows by K-With on (5) and (6)}
  }
  
  \jcase{6}{AK-Sum}{
    \jgivengoal{
      \caseFact{1} $\Psi ; \Theta ; \Delta \vdash \tau_1 \oplus \tau_2 : \star \gens \Phi_1 \wedge \Phi_2$    
      
      \caseFact{2} $\Theta ; \Delta \vDash \Phi_1 \wedge \Phi_2$
      
      \caseFact{3} $\Psi ; \Theta ; \Delta \vdash \tau_1 : \star \gens \Phi_1$
      
      \caseFact{4} $\Psi ; \Theta ; \Delta \vdash \tau_2 : \star \gens \Phi_2$
    }{
      $\Psi ; \Theta ; \Delta \vdash \tau_1 \oplus \tau_2 : \star$
    }
    
    \caseText{By IH on (3)}
    
    \caseFact{5} $\Psi ; \Theta ; \Delta \vdash \tau_1 : \star$
    
    \caseText{By IH on (4)}
    
    \caseFact{6} $\Psi ; \Theta ; \Delta \vdash \tau_2 : \star$
    
    \caseText{Goal follows by K-Sum on (5) and (6)}
  }
  
  \jcase{7}{AK-Bang}{
    \jgivengoal{
      \caseFact{1} $\Psi ; \Theta ; \Delta \vdash !\tau: \star \gens \Phi$
      
      \caseFact{2} $\Theta ; \Delta \vDash \Phi$
      
      \caseFact{3} $\Psi ; \Theta ; \Delta \vdash \tau : \star \gens \Phi$
    }{
      $\Psi ; \Theta ; \Delta \vdash !\tau : \star$
    }
    
    \caseText{By IH on (3)}
    
    \caseFact{4} $\Psi ; \Theta ; \Delta \vdash \tau : \star$
    
    \caseText{Goal follows by K-Bang on (5)}
  }
  
  \jcase{8}{AK-IForall}{
    \jgivengoal{
      \caseFact{1} $\Psi ; \Theta ; \Delta \vdash \forall i : S. \tau: \star \gens \forall i : S. \Phi$
      
      \caseFact{2} $\Theta ; \Delta \vDash \forall i : S. \Phi$
      
      \caseFact{3} $\Psi ; \Theta, i : S ; \Delta \vdash \tau : \star \gens \Phi$
    }{
      $\Psi ; \Theta ; \Delta \vdash \forall i : S. \tau : \star$
    }
    
    \caseText{Equivalently to (2)}
    
    \caseFact{5} $\Theta, i : S ; \Delta \vDash \Phi$
    
    \caseText{By IH on (3) using (5)}
    
    \caseFact{6} $\Psi ; \Theta, i : S ; \Delta \vdash \tau : \star$
    
    \caseText{Goal follows by K-IForall on (6)}
  }
  
  \jcase{9}{AK-IExists}{
    \jgivengoal{
      \caseFact{1} $\Psi ; \Theta ; \Delta \vdash \exists i : S. \tau: \star \gens \forall i : S. \Phi$
      
      \caseFact{2} $\Theta ; \Delta \vDash \forall i : S. \Phi$
      
      \caseFact{3} $\Psi ; \Theta, i : S ; \Delta \vdash \tau : \star \gens \Phi$
    }{
      $\Psi ; \Theta ; \Delta \vdash \exists i : S. \tau : \star$
    }
    
    \caseText{Equivalently to (2)}
    
    \caseFact{5} $\Theta, i : S ; \Delta \vDash \Phi$
    
    \caseText{By IH on (3) using (5)}
    
    \caseFact{6} $\Psi ; \Theta, i : S ; \Delta \vdash \tau : \star$
    
    \caseText{Goal follows by K-IExists on (6)}
  }
  
  \jcase{10}{AK-TForall}{
    \jgivengoal{
      \caseFact{1} $\Psi ; \Theta ; \Delta \vdash \forall \alpha : K. \tau : \star \gens \Phi$
      
      \caseFact{2} $\Theta ; \Delta \vDash \Phi$
      
      \caseFact{3} $\Psi, \alpha : K ; \Theta ; \Delta \vdash \tau : \star \gens \Phi$
    }{
      $\Psi ; \Theta ; \Delta \vdash \forall \alpha : K. \tau : \star$
    }
    
    \caseText{By IH on (3)}
    
    \caseFact{4} $\Psi, \alpha : K ; \Theta ; \Delta \vdash \tau : \star$
    
    \caseText{Goal follows by K-TForall on (5)}
  }
  
  \jcase{11}{AK-List}{
    \jgivengoal{
      \caseFact{1} $\Psi ; \Theta ; \Delta \vdash L^I \tau: \star \gens \Phi_1 \wedge \Phi_2$
      
      \caseFact{2} $\Theta ; \Delta \vDash \Phi_1 \wedge \Phi_2$
      
      \caseFact{3} $\Psi ; \Theta ; \Delta \vdash \tau : \star \gens \Phi_1$
      
      \caseFact{4} $\Theta ; \Delta \vdash I : \N \gens \Phi2$
    }{
      $\Psi ; \Theta ; \Delta \vdash L^I \tau : \star$
    }
    
    \caseText{By IH on (3)}
    
    \caseFact{5} $\Psi ; \Theta ; \Delta \vdash \tau : \star$
    
    \caseText{By \autoref{thm:raw-sort-sound} on (4)}
    
    \caseFact{6} $\Theta ; \Delta \vdash I : \N$
    
    \caseText{Goal follows by K-List on (5) and (6)}
  }
  
  \jcase{12}{AK-Conj}{
    \jgivengoal{
      \caseFact{1} $\Psi ; \Theta ; \Delta \vdash \Phi \amp \tau: \star \gens \Phi_1 \wedge \Phi_2$
      
      \caseFact{2} $\Theta ; \Delta \vDash \Phi_1 \wedge \Phi_2$
      
      \caseFact{3} $\Psi ; \Theta ; \Delta \vdash \tau : \star \gens \Phi_1$
      
      \caseFact{4} $\Theta ; \Delta \vdash \Phi \; \texttt{wf} \gens \Phi2$
    }{
      $\Psi ; \Theta ; \Delta \vdash \Phi \amp \tau : \star$
    }
    
    \caseText{By IH on (3)}
    
    \caseFact{5} $\Psi ; \Theta ; \Delta \vdash \tau : \star$
    
    \caseText{By \autoref{thm:raw-constr-sound} on (4)}
    
    \caseFact{6} $\Theta ; \Delta \vdash \Phi \; \texttt{wf}$
    
    \caseText{Goal follows by K-Conj on (5) and (6)}
  }
  
  \jcase{13}{AK-Impl}{
    \jgivengoal{
      \caseFact{1} $\Psi ; \Theta ; \Delta \vdash \Phi \implies \tau: \star \gens \Phi_1 \wedge (\Phi \to \Phi_2)$
      
      \caseFact{2} $\Theta ; \Delta \vDash \Phi_1 \wedge (\Phi \to \Phi_2)$
      
      \caseFact{3} $\Theta ; \Delta \vdash \Phi \; \texttt{wf} \gens \Phi1$
      
      \caseFact{4} $\Psi ; \Theta ; \Delta,\Phi \vdash \tau : \star \gens \Phi_2$
    }{
      $\Psi ; \Theta ; \Delta \vdash \Phi \implies \tau : \star$
    }

    
    \caseText{By \autoref{thm:raw-constr-sound} on (4)}
    
    \caseFact{5} $\Theta ; \Delta \vdash \Phi \; \texttt{wf}$
    
    \caseText{From (2)}
    
    \caseFact{6} $\Theta ; \Delta, \Phi \vDash \Phi_2$
    
    \caseText{By IH on (4) with (6)}
    
    \caseFact{7} $\Psi ; \Theta ; \Delta,\Phi \vdash \tau : \star$
    
    \caseText{Goal follows by K-Impl on (5) and (7)}
  }
  
  \jcase{14}{AK-Monad}{
    \jgivengoal{
      \caseFact{1} $\Psi ; \Theta ; \Delta \vdash \M \, (I,\vec{p}) \, \tau: \star \gens \Phi_1 \wedge \Phi_2 \wedge \Phi_3$
      
      \caseFact{2} $\Theta ; \Delta \vDash \Phi_1 \wedge \Phi_2 \wedge \Phi_3$
      
      \caseFact{3} $\Psi ; \Theta ; \Delta \vdash \tau : \star \gens \Phi_1$
      
      \caseFact{4} $\Theta ; \Delta \vdash I : \N \gens \Phi_2$
      
      \caseFact{5} $\Theta ; \Delta \vdash \vec{p} : \potvec \gens \Phi_3$
    }{
      $\Psi ; \Theta ; \Delta \vdash \M \, (I,\vec{p}) \, \tau : \star$
    }
    
    \caseText{By IH on (3)}
    
    \caseFact{6} $\Psi ; \Theta ; \Delta \vdash \tau : \star$
    
    \caseText{By \autoref{thm:raw-sort-sound} on (4)}
    
    \caseFact{7} $\Theta ; \Delta \vdash I : \N$

    \caseText{By \autoref{thm:raw-sort-sound} on (5)}
    
    \caseFact{8} $\Theta ; \Delta \vdash \vec{p} : \potvec$    
    
    \caseText{Goal follows by K-Monad on (6), (7), and (8)}
  }
  
  \jcase{15}{AK-Pot}{
    \jgivengoal{
      \caseFact{1} $\Psi ; \Theta ; \Delta \vdash [I|\vec{p}] \, \tau: \star \gens \Phi_1 \wedge \Phi_2 \wedge \Phi_3$
      
      \caseFact{2} $\Theta ; \Delta \vDash \Phi_1 \wedge \Phi_2 \wedge \Phi_3$
      
      \caseFact{3} $\Psi ; \Theta ; \Delta \vdash \tau : \star \gens \Phi_1$
      
      \caseFact{4} $\Theta ; \Delta \vdash I : \N \gens \Phi_2$
      
      \caseFact{5} $\Theta ; \Delta \vdash \vec{p} : \potvec \gens \Phi_3$
    }{
      $\Psi ; \Theta ; \Delta \vdash [I|\vec{p}] \, \tau : \star$
    }
    
    \caseText{By IH on (3)}
    
    \caseFact{6} $\Psi ; \Theta ; \Delta \vdash \tau : \star$
    
    \caseText{By \autoref{thm:raw-sort-sound} on (4)}
    
    \caseFact{7} $\Theta ; \Delta \vdash I : \N$

    \caseText{By \autoref{thm:raw-sort-sound} on (5)}
    
    \caseFact{8} $\Theta ; \Delta \vdash \vec{p} : \potvec$    
    
    \caseText{Goal follows by K-Pot on (6), (7), and (8)}
  }
  
  \jcase{16}{AK-ConstPot}{
    \jgivengoal{
      \caseFact{1} $\Psi ; \Theta ; \Delta \vdash [I] \, \tau: \star \gens \Phi_1 \wedge \Phi_2$
      
      \caseFact{2} $\Theta ; \Delta \vDash \Phi_1 \wedge \Phi_2$
      
      \caseFact{3} $\Psi ; \Theta ; \Delta \vdash \tau : \star \gens \Phi_1$
      
      \caseFact{4} $\Theta ; \Delta \vdash I : \R \gens \Phi_2$
    }{
      $\Psi ; \Theta ; \Delta \vdash [I] \, \tau : \star$
    }
    
    \caseText{By IH on (3)}
    
    \caseFact{5} $\Psi ; \Theta ; \Delta \vdash \tau : \star$
    
    \caseText{By \autoref{thm:raw-sort-sound} on (4)}
    
    \caseFact{6} $\Theta ; \Delta \vdash I : \R$
    
    \caseText{Goal follows by K-ConstPot on (5), and (6)}
  }
  
  \jcase{17}{AK-FamLam}{
   \jgivengoal{
     \caseFact{1} $\Psi ; \Theta ; \Delta \vdash \lambda i : S.\tau : S \to K \gens \forall i : S. \Phi$
     
     \caseFact{2} $\Theta ; \Delta \vDash \forall i : S. \Phi$   
     
     \caseFact{3} $\Psi ; \Theta, i : S ; \Delta \vdash \tau : K \gens \Phi$
   }{
     $\Psi ; \Theta ; \Delta \vdash \lambda i : S.\tau : S \to K$   
   }
   
   \caseText{Equivalently to (2)}
   
   \caseFact{4} $\Theta , i : S; \Delta \vDash \Phi$
   
   \caseText{By IH on (3) with (4)}
   
   \caseFact{5} $\Psi ; \Theta, i : S ; \Delta \vdash \tau : K$
   
   \caseText{Goal follows by K-FamLam on (5)}
  }
  
  \jcase{18}{AK-FamApp}{
   \jgivengoal{
     \caseFact{1} $\Psi ; \Theta ; \Delta \vdash \tau \; I : K \gens \Phi_1 \wedge \Phi_2$
     
     \caseFact{2} $\Theta ; \Delta \vDash \Phi_1 \wedge \Phi_2$
     
     \caseFact{3} $\Psi ; \Theta ; \Delta \vdash \tau : S \to K \gens \Phi_1$
     
     \caseFact{4} $\Theta ; \Delta \vdash I : S \gens \Phi_2$
   }{
    $\Psi ; \Theta ; \Delta \vdash \tau \; I : K$   
   }
   
   \caseText{By IH on (3)}
   
   \caseFact{5} $\Psi ; \Theta ; \Delta \vdash \tau : S \to K$
   
   \caseText{By \autoref{thm:raw-sort-sound} on (4)}
   
   \caseFact{6} $\Theta ; \Delta \vdash I : S$
   
   \caseText{Goal follows by K-FamApp on (5) and (6)}
  }
}

\kindsound*
\begin{proof}
Immediate by Theorem~\ref{thm:raw-kind-sound} and Theorem~\ref{thm:idx-ctx-wf-sound}
\end{proof}

\begin{theorem}[Raw Soundness of Subtyping for Normal Forms]
If $\Psi ; \Theta ; \Delta \vdash \tau_1 \subtynf \tau_2 : K \gens \Phi$ and $\Theta ; \Delta \vDash \Phi$ then $\Psi ; \Theta ; \Delta \vdash \tau_1 \subty \tau_2 : K$\label{thm:raw-subtynf-sound}
\end{theorem}
\jtheorem{Proof of \autoref{thm:raw-subtynf-sound}}{
  \jgivengoal{
    \caseFact{1} $\Psi ; \Theta ; \Delta \vdash \tau_1 \subtynf \tau_2 : K \gens \Phi$
    
    \caseFact{2} $\Theta ; \Delta \vDash \Phi$
  }{
   $\Psi ; \Theta ; \Delta \vdash \tau_1 \subty \tau_2 : K$  
  }
  
  \jcase{1}{AS-Unit}{Immediate.}
  \jcase{2}{AS-Var}{Immediate.}
  
  \jcase{3}{AS-Arr}{
   \jgivengoal{
     \caseFact{1} $\Psi ; \Theta ; \Delta \vdash \tau_1 \loli \tau_2 \subtynf \tau_1' \loli \tau_2' : \star \gens \Phi_1 \wedge \Phi_2$
     
     \caseFact{2} $\Theta ; \Delta \vDash \Phi_1 \wedge \Phi_2$   
     
     \caseFact{3} $\Psi ; \Theta ; \Delta \vdash \tau_1' \subtynf \tau_1 : \star \gens \Phi_1$
     
     \caseFact{4} $\Psi ; \Theta ; \Delta \vdash \tau_2 \subtynf \tau_2' : \star \gens \Phi_2$
   }{
     $\Psi ; \Theta ; \Delta \vdash \tau_1 \loli \tau_2 \subty \tau_1' \loli \tau_2' : \star$   
   }
   
   \caseText{By IH on (3)}
   
   \caseFact{5} $\Psi ; \Theta ; \Delta \vdash \tau_1' \subty \tau_1 : \star$

   \caseText{By IH on (4)}   
   
   \caseFact{6} $\Psi ; \Theta ; \Delta \vdash \tau_2 \subty \tau_2' : \star$
   
   \caseText{Goal follows by S-Arr on (5) and (6)}
  }
  
  \jcase{4}{AS-Tensor}{
   \jgivengoal{
     \caseFact{1} $\Psi ; \Theta ; \Delta \vdash \tau_1 \otimes \tau_2 \subtynf \tau_1' \otimes \tau_2' : \star \gens \Phi_1 \wedge \Phi_2$
     
     \caseFact{2} $\Theta ; \Delta \vDash \Phi_1 \wedge \Phi_2$   
     
     \caseFact{3} $\Psi ; \Theta ; \Delta \vdash \tau_1 \subtynf \tau_1' : \star \gens \Phi_1$
     
     \caseFact{4} $\Psi ; \Theta ; \Delta \vdash \tau_2 \subtynf \tau_2' : \star \gens \Phi_2$
   }{
     $\Psi ; \Theta ; \Delta \vdash \tau_1 \otimes \tau_2 \subty \tau_1' \otimes \tau_2' : \star$   
   }
   
   \caseText{By IH on (3)}
   
   \caseFact{5} $\Psi ; \Theta ; \Delta \vdash \tau_1' \subty \tau_1 : \star$

   \caseText{By IH on (4)}   
   
   \caseFact{6} $\Psi ; \Theta ; \Delta \vdash \tau_2 \subty \tau_2' : \star$
   
   \caseText{Goal follows by S-Tensor on (5) and (6)}
  }
  
  \jcase{5}{AS-With}{
   \jgivengoal{
     \caseFact{1} $\Psi ; \Theta ; \Delta \vdash \tau_1 \amp \tau_2 \subtynf \tau_1' \amp \tau_2' : \star \gens \Phi_1 \wedge \Phi_2$
     
     \caseFact{2} $\Theta ; \Delta \vDash \Phi_1 \wedge \Phi_2$   
     
     \caseFact{3} $\Psi ; \Theta ; \Delta \vdash \tau_1 \subtynf \tau_1' : \star \gens \Phi_1$
     
     \caseFact{4} $\Psi ; \Theta ; \Delta \vdash \tau_2 \subtynf \tau_2' : \star \gens \Phi_2$
   }{
     $\Psi ; \Theta ; \Delta \vdash \tau_1 \amp \tau_2 \subty \tau_1' \amp \tau_2' : \star$   
   }
   
   \caseText{By IH on (3)}
   
   \caseFact{5} $\Psi ; \Theta ; \Delta \vdash \tau_1' \subty \tau_1 : \star$

   \caseText{By IH on (4)}   
   
   \caseFact{6} $\Psi ; \Theta ; \Delta \vdash \tau_2 \subty \tau_2' : \star$
   
   \caseText{Goal follows by S-With on (5) and (6)}
  }
  
  \jcase{6}{AS-Sum}{
   \jgivengoal{
     \caseFact{1} $\Psi ; \Theta ; \Delta \vdash \tau_1 \oplus \tau_2 \subtynf \tau_1' \oplus \tau_2' : \star \gens \Phi_1 \wedge \Phi_2$
     
     \caseFact{2} $\Theta ; \Delta \vDash \Phi_1 \wedge \Phi_2$   
     
     \caseFact{3} $\Psi ; \Theta ; \Delta \vdash \tau_1 \subtynf \tau_1' : \star \gens \Phi_1$
     
     \caseFact{4} $\Psi ; \Theta ; \Delta \vdash \tau_2 \subtynf \tau_2' : \star \gens \Phi_2$
   }{
     $\Psi ; \Theta ; \Delta \vdash \tau_1 \oplus \tau_2 \subty \tau_1' \oplus \tau_2' : \star$   
   }
   
   \caseText{By IH on (3)}
   
   \caseFact{5} $\Psi ; \Theta ; \Delta \vdash \tau_1' \subty \tau_1 : \star$

   \caseText{By IH on (4)}   
   
   \caseFact{6} $\Psi ; \Theta ; \Delta \vdash \tau_2 \subty \tau_2' : \star$
   
   \caseText{Goal follows by S-Sum on (5) and (6)}
  }
  
  \jcase{7}{AS-Bang}{
   \jgivengoal{
     \caseFact{1} $\Psi ; \Theta ; \Delta \vdash !\tau_1 \subtynf !\tau_2 : \star \gens \Phi$
     
     \caseFact{2} $\Theta ; \Delta \vDash \Phi$
     
     \caseFact{3} $\Psi ; \Theta ; \Delta \vdash \tau_1 \subtynf \tau_2 : \star \gens \Phi$
    
   }{
     $\Psi ; \Theta ; \Delta \vdash !\tau_1 \subty !\tau_2 : \star$
   }
   
   \caseText{By IH on (3)}
   
   \caseFact{4} $\Psi ; \Theta ; \Delta \vdash \tau_1 \subty \tau_2 : \star$
   
   \caseText{Goal follows by S-Bang on (4)}
  }
  
  \jcase{8}{AS-IForall}{
   \jgivengoal{
     \caseFact{1} $\Psi ; \Theta ; \Delta \vdash \forall i : S.\tau_1 \subtynf \forall i : S. \tau_2 : \star \gens \forall i : S. \Phi$
     
     \caseFact{2} $\Theta ; \Delta \vDash \forall i : S. \Phi$
     
     \caseFact{3} $\Psi ; \Theta, i : S ; \Delta \vdash \tau_1 \subtynf \tau_2 : \star \gens \Phi$
    
   }{
     $\Psi ; \Theta ; \Delta \vdash \forall i : S. \tau_1 \subty \forall i : S. \tau_2 : \star$
   }
   
   \caseText{Equivalently to (2)}
   
   \caseFact{4} $\Theta, i : S; \Delta \vDash \Phi$
   
   \caseText{By IH on (3), using (4)}
   
   \caseFact{5} $\Psi ; \Theta ; \Delta \vdash \tau_1 \subty \tau_2 : \star$
   
   \caseText{Goal follows by S-IForall on (5)}
  }
  
  \jcase{9}{AS-IExists}{
   \jgivengoal{
     \caseFact{1} $\Psi ; \Theta ; \Delta \vdash \exists i : S.\tau_1 \subtynf \exists i : S. \tau_2 : \star \gens \forall i : S. \Phi$
     
     \caseFact{2} $\Theta ; \Delta \vDash \forall i : S. \Phi$
     
     \caseFact{3} $\Psi ; \Theta, i : S ; \Delta \vdash \tau_1 \subtynf \tau_2 : \star \gens \Phi$
    
   }{
     $\Psi ; \Theta ; \Delta \vdash \exists i : S. \tau_1 \subty \exists i : S. \tau_2 : \star$
   }
   
   \caseText{Equivalently to (2)}
   
   \caseFact{4} $\Theta, i : S; \Delta \vDash \Phi$
   
   \caseText{By IH on (3), using (4)}
   
   \caseFact{5} $\Psi ; \Theta ; \Delta \vdash \tau_1 \subty \tau_2 : \star$
   
   \caseText{Goal follows by S-IExists on (5)}
  }
  
  \jcase{10}{AS-TForall}{
   \jgivengoal{
     \caseFact{1} $\Psi ; \Theta ; \Delta \vdash \forall \alpha : K. \tau_1 \subtynf \forall \alpha : K. \tau_2 : \star \gens \Phi$
     
     \caseFact{2} $\Theta ; \Delta \vDash \Phi$
     
     \caseFact{3} $\Psi, \alpha : K ; \Theta ; \Delta \vdash \tau_1 \subtynf \tau_2 : \star \gens \Phi$
    
   }{
     $\Psi ; \Theta ; \Delta \vdash \forall \alpha : K. \tau_1 \subty \forall \alpha : K. \tau_2 : \star$
   }
   
   \caseText{By IH on (3)}
   
   \caseFact{4} $\Psi, \alpha : K ; \Theta ; \Delta \vdash \tau_1 \subty \tau_2 : \star$
   
   \caseText{Goal follows by S-TForall on (4)}
  }
  
  \jcase{11}{AS-List}{
   \jgivengoal{
     \caseFact{1} $\Psi ; \Theta ; \Delta \vdash L^I \tau_1 \subtynf L^J \tau_2 : \star \gens \Phi \wedge (I = J)$
     
     \caseFact{2} $\Theta ; \Delta \vDash \Phi \wedge (I = J)$
     
     \caseFact{3} $\Psi ; \Theta ; \Delta \vdash \tau_1 \subtynf \tau_2 : \star \gens \Phi$
   }{
     $\Psi ; \Theta ; \Delta \vdash L^I \tau_1 \subty L^J \tau_2 : \star$
   }
   
   \caseText{By IH on (3)}
   
   \caseFact{4} $\Psi ; \Theta ; \Delta \vdash \tau_1 \subty \tau_2 : \star \gens \Phi$
   
   \caseText{From (2)}
   
   \caseFact{5} $\Theta ; \Delta \vDash I = J$
   
   \caseText{Goal follows by S-List on (4) and (5)}
  }
  
  \jcase{12}{AS-Conj}{
   \jgivengoal{
     \caseFact{1} $\Psi ; \Theta ; \Delta \vdash \Phi_1 \amp \tau_1 \subtynf \Phi_2 \amp \tau_2 : \star \gens \Phi \wedge (\Phi_1 \to \Phi_2)$
     
     \caseFact{2} $\Theta ; \Delta \vDash \Phi \wedge (\Phi_1 \to \Phi_2)$
     
     \caseFact{3} $\Psi ; \Theta ; \Delta \vdash \tau_1 \subtynf \tau_2 : \star \gens \Phi$
   }{
     $\Psi ; \Theta ; \Delta \vdash \Phi_1 \amp \tau_1 \subty \Phi_2 \amp \tau_2 : \star$
   }
   
   \caseText{By IH on (3)}
   
   \caseFact{4} $\Psi ; \Theta ; \Delta \vdash \tau_1 \subty \tau_2 : \star \gens \Phi$
   
   \caseText{From (2)}
   
   \caseFact{5} $\Theta ; \Delta \vDash \Phi_1 \to \Phi_2$
   
   \caseText{Goal follows by S-Conj on (4) and (5)}
  }
  
  \jcase{13}{AS-Impl}{
   \jgivengoal{
     \caseFact{1} $\Psi ; \Theta ; \Delta \vdash \Phi_1 \implies \tau_1 \subtynf \Phi_2 \implies \tau_2 : \star \gens (\Phi_2 \to \Phi) \wedge (\Phi_2 \to \Phi_1)$
     
     \caseFact{2} $\Theta ; \Delta \vDash (\Phi_2 \to \Phi) \wedge (\Phi_2 \to \Phi_1)$
     
     \caseFact{3} $\Psi ; \Theta ; \Delta,\Phi_2 \vdash \tau_1 \subtynf \tau_2 : \star \gens \Phi$
   }{
     $\Psi ; \Theta ; \Delta \vdash \Phi_1 \implies \tau_1 \subty \Phi_2 \implies \tau_2 : \star$
   }
   
   \caseText{By IH on (3)}
   
   \caseFact{4} $\Psi ; \Theta ; \Delta, \Phi_2 \vdash \tau_1 \subty \tau_2 : \star \gens \Phi$
   
   \caseText{From (2)}
   
   \caseFact{5} $\Theta ; \Delta \vDash (\Phi_2 \to \Phi) \wedge (\Phi_2 \to \Phi_1)$
   
   \caseText{Goal follows by S-Impl on (4) and (5)}
  }
  
  \jcase{14}{AS-Monad}{
   \jgivengoal{
    \caseFact{1} $\Psi ; \Theta ; \Delta \vdash \M \, (I,\vec{p}) \, \tau_1 \subtynf \M \, (J,\vec{q}) \, \tau_2 : \star \gens \Phi \wedge (I = J) \wedge (\vec{p} \leq \vec{q})$
    
    \caseFact{2} $\Theta ; \Delta \vDash \Phi \wedge (I = J) \wedge (\vec{p} \leq \vec{q})$
    
    \caseFact{3} $\Psi ; \Theta ; \Delta \vdash \tau_1 \subtynf \tau_2 : \star \gens \Phi$
   }{
     $\Psi ; \Theta ; \Delta \vdash \M \, (I,\vec{p}) \, \tau_1 \subty \M \, (J,\vec{q}) \, \tau_2 : \star$
   }
   
   \caseText{By IH on (3)}
   
   \caseFact{4} $\Psi ; \Theta ; \Delta \vdash \tau_1 \subty \tau_2 : \star$
   
   \caseText{From (2)}
   
   \caseFact{5} $\Theta ; \Delta \vDash I = J$
   
   \caseFact{6} $\Theta ; \Delta \vDash \vec{p} \leq \vec{q}$
   
   \caseText{Goal follows by S-Monad on (4), (5), and (6)}
  }
  
  \jcase{15}{AS-Pot}{
   \jgivengoal{
    \caseFact{1} $\Psi ; \Theta ; \Delta \vdash [I|\vec{p}] \, \tau_1 \subtynf [J|\vec{q}] \, \tau_2 : \star \gens \Phi \wedge (I = J) \wedge (\vec{p} \geq \vec{q})$
    
    \caseFact{2} $\Theta ; \Delta \vDash \Phi \wedge (I = J) \wedge (\vec{p} \geq \vec{q})$
    
    \caseFact{3} $\Psi ; \Theta ; \Delta \vdash \tau_1 \subtynf \tau_2 : \star \gens \Phi$
   }{
     $\Psi ; \Theta ; \Delta \vdash [I|\vec{p}] \, \tau_1 \subty [J|\vec{q}] \, \tau_2 : \star$
   }
   
   \caseText{By IH on (3)}
   
   \caseFact{4} $\Psi ; \Theta ; \Delta \vdash \tau_1 \subty \tau_2 : \star$
   
   \caseText{From (2)}
   
   \caseFact{5} $\Theta ; \Delta \vDash I = J$
   
   \caseFact{6} $\Theta ; \Delta \vDash \vec{p} \geq \vec{q}$
   
   \caseText{Goal follows by S-Pot on (4), (5), and (6)}
  }
  
  \jcase{16}{AS-ConstPot}{
   \jgivengoal{
    \caseFact{1} $\Psi ; \Theta ; \Delta \vdash [I] \, \tau_1 \subtynf [J] \, \tau_2 : \star \gens \Phi \wedge (I \geq J)$
    
    \caseFact{2} $\Theta ; \Delta \vDash \Phi \wedge (I \geq J)$
    
    \caseFact{3} $\Psi ; \Theta ; \Delta \vdash \tau_1 \subtynf \tau_2 : \star \gens \Phi$
   }{
     $\Psi ; \Theta ; \Delta \vdash [I] \, \tau_1 \subty [J] \, \tau_2 : \star$
   }
   
   \caseText{By IH on (3)}
   
   \caseFact{4} $\Psi ; \Theta ; \Delta \vdash \tau_1 \subty \tau_2 : \star$
   
   \caseText{From (2)}
   
   \caseFact{5} $\Theta ; \Delta \vDash I \geq J$
   
   \caseText{Goal follows by S-Const on (4) and (5)}
  }
  
  \jcase{17}{AS-FamLam}{
   \jgivengoal{
    \caseFact{1} $\Psi ; \Theta ; \Delta \vdash \lambda i : S.\tau_1 \subtynf \lambda i :S .\tau_2 :S \to K \gens \forall i : S. \Phi$
    
    \caseFact{2} $\Theta ; \Delta \vDash \forall i : S. \Phi$
    
    \caseFact{3} $\Psi ; \Theta, i : S; \Delta \vdash \tau_1 \subtynf \tau_2 : K \gens \Phi$
   }{
     $\Psi ; \Theta ; \Delta \vdash \lambda i : S.\tau_1 \subty \lambda i :S .\tau_2 :S \to K$
   }
   \caseText{Equivalently to (2)}
   
   \caseFact{4} $\Theta, i : S ; \Delta \vDash \Phi$
   
   \caseText{By IH on (3), using (4)}
   
   \caseFact{5} $\Psi ; \Theta, i : S: \Delta \vdash \tau_1 \subty \tau_2 : K$
   
   \caseText{Goal follows by S-FamLam on (5)}
  }
  
  \jcase{18}{AS-FamApp}{
   \jgivengoal{
     \caseFact{1} $\Psi ; \Theta ; \Delta \vdash \tau_1 \; I \subtynf \tau_2 \; J : K \gens (I = J) \wedge \Phi$ 
     
     \caseFact{2} $\Theta ; \Delta \vDash (I = J) \wedge \Phi$
     
     \caseFact{3} $\Psi ; \Theta ; \Delta \vdash \tau_1 \subtynf \tau_2 : S \to K \gens \Phi$
   }{
     $\Psi ; \Theta ; \Delta \vdash \tau_1 \; I \subty \tau_2 \; J : K$
   }
   
   \caseText{By IH on (3)}
   
   \caseFact{4} $\Psi ; \Theta ; \Delta \vdash \tau_1 \subty \tau_2 ; S \to K$
   
   \caseText{From (2)}
   
   \caseFact{5} $\Theta ; \Delta \vDash I = J$
   
   \caseText{Goal follows by S-FamApp on (4) and (5).}
  }

}

\iffalse
\begin{proof}
 We proceed by induction on $\Psi ; \Theta ; \Delta \vdash \tau_1 \subtynf \tau_2 : K \gens \Phi$.

\begin{itemize}
  \item[AS-Monad] Suppose $\Psi ; \Theta ; \Delta \vdash M(I,\vec{q}) \tau_1 \subtynf M(J,\vec{p}) \tau_2 : \star \gens (I = J) \wedge (\vec{q} \leq \vec{p}) \wedge \Phi$ by way of $\Psi ; \Theta ; \Delta \vdash \tau_1 \subtynf \tau_2 : \star \gens \Phi$ with $\Theta ; \Delta \vDash (I = J)$,  $\Theta ; \Delta \vDash \vec{q} \leq \vec{p}$, and $\Theta ; \Delta \vDash \Phi$. By IH, $\Psi ; \Theta ; \Delta \vdash \tau_1 \subty \tau_2 : \star$, and since $\Theta ; \Delta \vDash (I = J)$ and $\Theta ; \Delta \vDash \vec{q} \leq \vec{p}$, we have by S-Monad that $\Psi ; \Theta ; \Delta \vdash M(I,\vec{q}) \tau_1 \subty M(J,\vec{p}) \tau_2 : \star$.
  \item[AS-Pot] Suppose $\Psi ; \Theta ; \Delta \vdash [I|\vec{q}] \tau_1 \subtynf [J|\vec{p}] \tau_2 : \star \gens (I = J) \wedge (\vec{p} \leq \vec{q}) \wedge \Phi$ by way of $\Psi ; \Theta ; \Delta \vdash \tau_1 \subtynf \tau_2 : \star \gens \Phi$, with  $\Theta ; \Delta \vDash (I = J)$,  $\Theta ; \Delta \vDash \vec{p} \leq \vec{q}$, and $\Theta ; \Delta \vDash \Phi$. By IH, $\Psi ; \Theta ; \Delta \vdash \tau_1 \subty \tau_2 : \star$. Using the fact that $\Theta ; \Delta \vDash (I = J)$ and $\Theta ; \Delta \vDash \vec{p} \leq \vec{q}$, we have by S-Pot that $\Psi ; \Theta ; \Delta \vdash [i|\vec{q}] \tau_1 \subty [j|\vec{p}] \tau_2 : \star$.
  \item[AS-ConstPot] Suppose ${\Psi ; \Theta ; \Delta \vdash [I] \tau_1 \subtynf [J] \tau_2 : \star \gens \Phi \wedge (J \leq I)}$ by way of ${\Psi ; \Theta ; \Delta \vdash \tau_1 \subtynf \tau_2 : \star \gens \Phi}$ with $\Theta ; \Delta \vDash \Phi$ and $\Theta ; \Delta \vDash J \leq I$. By IH, $\Psi ; \Theta ; \Delta \vdash \tau_1 \subty \tau_2 : \star$, and so using $\Theta ; \Delta \vDash J \leq I$, we have by S-ConstPot that $\Psi ; \Theta ; \Delta \vdash [I] \tau_1 \subty [J] \tau_2 : \star$.
  \item[AS-FamLam] Suppose $\Psi ; \Theta ; \Delta \vdash \lambda i : S. \tau_1 \subtynf \lambda i : S. \tau_2 : S \to K \gens \forall i : S. \Phi$ by way of ${\Psi ; \Theta, i : S ; \Delta \vdash \tau_1 \subtynf \tau_2 : K \gens \Phi}$, with $\Theta ; \Delta \vDash \forall i : S. \Phi$. Equivalently, $\Theta, i : S; \Delta \vDash \Phi$, and so by IH, $\Psi ; \Theta, i : S ; \Delta \vdash \tau_1 \subty \tau_2 : K$, which by S-FamLam means $\Psi ; \Theta ; \Delta \vdash \lambda i : S. \tau_1 \subty \lambda i : S. \tau_2 : S \to K $.
  \item[AS-FamApp] Suppose that $\Psi ; \Theta ; \Delta \vdash \tau_1 \; I \subtynf \tau_2 \; J : K \gens (I = J) \wedge \Phi$ by way of $\Psi ; \Theta ; \Delta \vdash \tau_1 \subtynf \tau_2 : S \to K \gens \Phi$ and $i,j : S  \in \Theta$, with $\Theta ; \Delta \vDash I = J$ and $\Theta ; \Delta \vDash \Phi$. By IH, $\Psi ; \Theta ; \Delta \vdash \tau_1 \subty \tau_2 : S \to K \gens \Phi$, and by S-FamApp, $\Psi ; \Theta ; \Delta \vdash \tau_1 \; I \subty \tau_2 \; J : K$.
\end{itemize}

\end{proof}

\fi

\subtynfsound*
\begin{proof}
Immediate by Theorem~\ref{thm:raw-subtynf-sound} and Theorem~\ref{thm:idx-ctx-wf-sound}
\end{proof}

\subtysound*
\begin{proof}
There is only one case: $\Psi ; \Theta ; \Delta \vdash \tau_1 \subty\tau_2 : K \gens \Phi$ by way of $\Psi ; \Theta ; \Delta \vdash \texttt{eval}(\tau_1) \subtynf \texttt{eval}(\tau_2) : K \gens \Phi$ with $\Theta ; \Delta \vDash \Phi$. By Theorem~\ref{thm:subtynf-sound}, $\Psi ; \Theta ; \Delta \vdash \texttt{eval}(\tau_1) \subty \texttt{eval}(\tau_2) : K$. By Theorem~\ref{thm:norm-thm} and two uses of S-Trans, $\Psi ; \Theta ; \Delta \pvdash \tau_1 \subty \tau_2 : K$, as required.
\end{proof}

\begin{theorem}[Output Context is Uniquely Determined]
For any $\Psi,\Theta,\Delta,\Gamma,e,\Phi$, there is at most one $\Gamma'$ so that $\Psi ; \Theta ; \Delta ; \Omega ; \Gamma \vdash e \updownarrow \tau \gens \Phi, \Gamma'$
\label{thm:ctx-uniquely-determined}
\end{theorem}
\begin{proof}
Inspection of rules.
\end{proof}

\lsc*
\begin{proof}
Induction on $\Psi ; \Theta ; \Delta ; \Omega ; \Gamma \vdash e \updownarrow \tau \gens \Phi, \Gamma'$.
\end{proof}

\begin{theorem}[Output Context is Well-Formed]
If $\Psi ; \Theta ; \Delta ; \Omega ; \Gamma \pvdash e \updownarrow \tau \gens \Phi,\Gamma'$ then $\Psi ; \Theta ; \Delta \vdash \Gamma' \; \texttt{wf} \gens \Phi'$ with $\Theta ; \Delta \vDash \Phi'$.
\end{theorem}
\begin{proof}
An easy induction on $\Gamma'$, using the fact that $\Psi ; \Theta ; \Delta \vdash \Gamma \; \texttt{wf} \gens \Phi''$ with $\Theta ; \Delta \vDash \Phi''$, and Theorem~\ref{thm:lsc}
\end{proof}

\begin{theorem}[Algorithmic Well-Formedness of Context Operations]
If $\Gamma_1,\Gamma_2 \subseteq \Gamma$ such that $\Psi ; \Theta ; \Delta \vdash \Gamma \; \texttt{wf}$, then the following are true:
\begin{enumerate}
  \item $\Psi ; \Theta ; \Delta \vdash \Gamma_1,\Gamma_2 \; \texttt{wf}$
  \item $\Psi ; \Theta ; \Delta \vdash \Gamma_1 \cap \Gamma_2 \; \texttt{wf}$
  \item $\Psi ; \Theta ; \Delta \vdash \Gamma_1\setminus\Gamma_2 \; \texttt{wf}$
\end{enumerate}
\end{theorem}

\theorem[Raw Soundness of Type Checking/Inference]{
~\begin{enumerate}
 \item If $\Psi;\Theta;\Delta;\Omega;\Gamma \vdash e \checks \tau \gens \Phi, \Gamma'$ and $\Theta;\Delta \vDash \Phi$ then $\Psi;\Theta;\Delta;\Omega;\Gamma \setminus \Gamma' \vdash |e| : \tau$
 \item If $\Psi;\Theta;\Delta;\Omega;\Gamma \vdash e \infers \tau \gens \Phi, \Gamma'$ and $\Theta;\Delta \vDash \Phi$ then $\Psi;\Theta;\Delta;\Omega;\Gamma \setminus \Gamma' \vdash |e| : \tau$
\end{enumerate}
}
%\textbf{Redo in new style}
\label{thm:raw-tycheck-sound}
\begin{proof}
We prove both claims simultaneously by induction over $\Psi;\Theta;\Delta;\Omega;\Gamma \vdash e \checks \tau \gens \Phi, \Gamma'$ and $\Psi;\Theta;\Delta;\Omega;\Gamma \vdash e \infers \tau \gens \Phi, \Gamma'$.
We will often use Theorem~\ref{thm:lsc} and Theorem~\ref{thm:ctx-sub-subset-1} silently, usually baked into calls to weakening.
\begin{enumerate}
  \item[AT-Var-1] Suppose $\Psi ; \Theta ; \Delta ; \Omega ; \Gamma\vdash x \infers \tau \gens \top, \Gamma \setminus \{x : \tau\}$ by way of $x : \tau \in \Gamma$. Clearly, $x : \tau \in \{x : \tau\}$, and so by T-Var-1, $\Psi ; \Theta ; \Delta ; \Omega ; x : \tau \vdash x : \tau$, as required.
  \item[AT-Var-2] Suppose $\Psi ; \Theta ; \Delta ; \Omega ; \Gamma\vdash x \infers \tau \gens \top, \Gamma$ by way of $x : \tau \in \Omega$. Then, by T-Var-2, $\Psi ; \Theta ; \Delta ; \Omega ; \cdot \vdash x : \tau$ as required.
  
  \item[AT-Unit] Immediate. 
  \item[AT-Base] Immediate. 
  \item[AT-Absurd] Immediate. 
  \item[AT-Nil] Suppose $\Psi ; \Theta ; \Delta ; \Omega ; \Gamma\vdash \texttt{nil} \checks L^I \tau \gens I = 0, \Gamma$  with $\Theta ; \Delta \vDash I = 0$. Then, by T-Nil, $\Psi ; \Theta ; \Delta ; \Omega ; \cdot \vdash \texttt{nil} : L^0 \tau$. By S-List and S-Refl, $\Psi ; \Theta ; \Delta \vdash L^0 \tau \subty L^I \tau : \star$. By T-Sub, $\Psi ; \Theta ; \Delta ; \Omega ; \cdot \vdash \texttt{nil} : L^I \tau$, as required.
 
  
  \item[AT-Cons] 
  \textbf{REDO THIS CASE}  
  Suppose $\Psi ; \Theta ; \Delta ; \Omega ; \Gamma\vdash e_1 :: e_2 \checks L^I \tau \gens (I \geq 1) \wedge \Phi_1 \wedge \Phi_2, \Gamma_2$ by way of $\Psi ; \Theta ; \Delta ; \Omega ; \Gamma\vdash e_1 \checks \tau \gens \Phi_1, \Gamma_1$ and $\Psi ; \Theta ; \Delta ; \Omega ; \Gamma_1\vdash e_2 \checks L^{I-1} \tau \gens \Phi_2, \Gamma_2$ with $\Theta ; \Delta \vDash I \geq 1$, $\Theta ; \Delta \vDash \Phi_1$ and $\Theta ; \Delta \vDash \Phi_2$. By IH, $\Psi ; \Theta ; \Delta ; \Omega ; \Gamma \setminus \Gamma_1 \vdash |e_1| : \tau$, and $\Psi ; \Theta ; \Delta ; \Omega ; \Gamma_1 \setminus \Gamma_2 \vdash e_2 : L^{I-1} \tau$. Then, because $\Theta ; \Delta \vDash I \geq 1$, by T-Weaken and T-Cons, $\Psi ; \Theta ; \Delta ; \Omega ; \Gamma \setminus \Gamma_2 \vdash |e_1 :: e_2| : L^I \tau$.
  
  \item[AT-Match] Suppose $\Psi ; \Theta ; \Delta ; \Omega ; \Gamma\vdash \texttt{match}(e,e_1,h.t.e_2) \checks \tau' \gens \Phi_1 \wedge \Phi_\texttt{body}, \Gamma'$ from:
  \begin{enumerate}
    \item $\Psi ; \Theta ; \Delta ; \Omega ; \Gamma\vdash e \infers L^I \tau \gens \Phi_1, \Gamma_1$
    \item $\Psi ; \Theta ; \Delta, I = 0 ; \Omega ; \Gamma_1\vdash e_1 \checks \tau' \gens \Phi_2,\Gamma_2$
    \item $\Psi ; \Theta ; \Delta, I \geq 1; \Omega ; \Gamma_1, h : \tau, t : L^{I-1} \tau \vdash e_2 \checks \tau' \gens \Phi_3,\Gamma_3$
  \end{enumerate}
  with $\Phi_\texttt{body} = (I = 0 \to \Phi_2) \wedge (I \geq 1 \to \Phi_3 )$, $\Gamma' = \Gamma_2 \cap (\Gamma_3 \setminus \{h,t\})$, and $\Theta ; \Delta \vDash \Phi_1$, $\Theta ; \Delta, I = 0 \vDash \Phi_2$, and $\Theta ; \Delta, I \geq 1 \vDash \Phi_3$. Then, by IH we have:
  \begin{enumerate}
    \item $\Psi ; \Theta ; \Delta ; \Omega ; \Gamma \setminus \Gamma_1 \vdash |e| : L^I \tau$
    \item $\Psi ; \Theta ; \Delta, I = 0 ; \Omega ; \Gamma_1 \setminus \Gamma_2 \vdash |e_1| : \tau'$
    \item $\Psi ; \Theta ; \Delta, I \geq 1; \Omega ; (\Gamma_1, h : \tau, t : L^{I-1} \tau) \setminus \Gamma_3 \vdash |e_2| : \tau'$
  \end{enumerate}
  Then, by T-weakening, we have that $\Psi ; \Theta ; \Delta, I = 0 ; \Omega ; \Gamma_1 \setminus \Gamma' \vdash |e_1| : \tau'$ and $\Psi ; \Theta ; \Delta, I \geq 1; \Omega ; \Gamma_1 \setminus \Gamma', h : \tau, t : L^{I-1} \tau \vdash |e_2| : \tau'$. Then, by T-match and one more T-weakening,
  $\Psi ; \Theta ; \Delta ; \Omega ; \Gamma \setminus \Gamma' \vdash |\texttt{match}(e,e_1,h.t.e_2)| : \tau'$ as required.
  
  \item[AT-ExistI] Suppose $\Psi ; \Theta ; \Delta ; \Omega ; \Gamma\vdash \texttt{pack}[I](e) \checks \exists i:S.\tau \gens \Phi_1 \wedge \Phi_2, \Gamma'$ by way of $\Theta ; \Delta \vdash I : S \gens \Phi_1$ and $\Psi ; \Theta ; \Delta ; \Omega ; \Gamma\vdash e \checks \tau[I/i] \gens \Phi_2,\Gamma'$, with $\Theta ; \Delta \vDash \Phi_1$ and $\Theta ; \Delta \vDash \Phi_2$. By \textbf{Soundness of Sort Checking},  $\Theta ; \Delta \vdash I : S$. By IH, $\Psi ; \Theta ; \Delta ; \Omega ; \Gamma\setminus \Gamma' \vdash |e| \checks \tau[I/i]$. By T-ExistI, $\Psi ; \Theta ; \Delta ; \Omega ; \Gamma \setminus \Gamma'\vdash |\texttt{pack}[I](e)| : \exists i:S.\tau$, as required.

  \item[AT-ExistE] Suppose ${
\Psi ; \Theta ; \Delta ; \Omega ; \Gamma\vdash \texttt{unpack } (i,x) = e \texttt{ in } e' \checks \tau' \gens \Phi , \Gamma_2 \setminus \{x\}
}$ by way of $\Psi ; \Theta ; \Delta ; \Omega ; \Gamma\vdash e \infers \exists i : S.\tau \gens \Phi_1, \Gamma_1$ and $\Psi ; \Theta, i : S ; \Delta ; \Omega ; \Gamma_1, x : \tau \vdash e' \checks \tau' \gens \Phi_2, \Gamma_2$ with $\Phi = \Phi_1 \wedge (\forall i : S. \Phi_2)$ and $\Theta ; \Delta \vDash \Phi_1$, $\Theta ; \Delta \vDash \forall i : S. \Phi_2$. By IH, $\Psi ; \Theta ; \Delta ; \Omega ; \Gamma \setminus \Gamma_1 \vdash e : \exists i : S.\tau$. Since $\Theta, i :S ; \Delta \vDash \Phi_2$, we have by IH that $\Psi ; \Theta, i : S ; \Delta ; \Omega ; (\Gamma_1, x : \tau) \setminus \Gamma_2 \vdash e' : \tau'$. By weakening, $\Psi ; \Theta, i : S ; \Delta ; \Omega ; (\Gamma_1 \setminus (\Gamma_2 \setminus \{x : \tau\})), x : \tau \vdash e' : \tau'$, and so by T-ExistE, $
\Psi ; \Theta ; \Delta ; \Omega ; (\Gamma \setminus \Gamma_1), (\Gamma_1 \setminus (\Gamma_2 \setminus \{x : \tau\})) \vdash |\texttt{unpack } (i,x) = e \texttt{ in } e'| : \tau'$, and hence by weakening, ${
\Psi ; \Theta ; \Delta ; \Omega ; \Gamma \setminus (\Gamma_2 \setminus \{x : \tau\})\vdash |\texttt{unpack } (i,x) = e \texttt{ in } e'| : \tau'
}$ as required.
  \item[AT-Lam] Suppose $\Psi ; \Theta ; \Delta ; \Omega ; \Gamma\vdash \lambda x.e \checks \tau_1 \loli \tau_2 \gens \Phi, \Gamma' \setminus \{x : \tau_1\}$ by way of $\Psi ; \Theta ; \Delta ; \Omega ; \Gamma, x : \tau_1 \vdash e \checks \tau_2, \gens \Phi, \Gamma'$, with $\Theta ; \Delta \vDash \Phi$. By IH, $\Psi ; \Theta ; \Delta ; \Omega ; (\Gamma, x : \tau_1) \setminus \Gamma' \vdash e : \tau_2$. By weakening,  $\Psi ; \Theta ; \Delta ; \Omega ; (\Gamma \setminus (\Gamma' \setminus \{x : \tau_1\})), x : \tau_1 \vdash e : \tau_2$. Then, by T-Lam,  $\Psi ; \Theta ; \Delta ; \Omega ; \Gamma \setminus (\Gamma' \setminus \{x : \tau_1\}) \vdash \lambda x.e : \tau_1 \loli \tau_2$ as required.
  \item[AT-App] Suppose $\Psi ; \Theta ; \Delta ; \Omega ; \Gamma\vdash e_1 \, e_2 \infers  \tau_2 \gens \Phi_1 \wedge \Phi_2, \Gamma_2
$ by way of $\Psi ; \Theta ; \Delta ; \Omega ; \Gamma\vdash e_1 \infers \tau_1 \loli \tau_2 \gens \Phi_1, \Gamma_1$ and $\Psi ; \Theta ; \Delta ; \Omega ; \Gamma_1\vdash e_2 \checks \tau_1 \gens \Phi_2, \Gamma_2$ with $\Theta ; \Delta \vDash \Phi_1$ and $\Theta ; \Delta \vDash \Phi_2$. By IH, $\Psi ; \Theta ; \Delta ; \Omega ; \Gamma \setminus \Gamma_1 \vdash |e_1| : \tau_1 \loli \tau_2$, and also by IH $\Psi ; \Theta ; \Delta ; \Omega ; \Gamma_1 \setminus \Gamma_2 \vdash |e_2| : \tau_1$, and so by T-App, $\Psi ; \Theta ; \Delta ; \Omega ; (\Gamma \setminus \Gamma_1),(\Gamma_1 \setminus \Gamma_2)\vdash |e_1 \, e_2| :  \tau_2$. Then, by weakening,  $\Psi ; \Theta ; \Delta ; \Omega ; \Gamma \setminus \Gamma_2\vdash |e_1 \, e_2| :  \tau_2$ as required.
  \item[AT-TensorI] Suppose $\Psi ; \Theta ; \Delta ; \Omega ; \Gamma\vdash \angles{e_1,e_2} \checks \tau_1 \otimes \tau_2 \gens \Phi_1 \wedge \Phi_2,\Gamma_2$ by way of $\Psi ; \Theta ; \Delta ; \Omega ; \Gamma\vdash e_1 \checks \tau_1 \gens \Phi_1, \Gamma_1$ and $\Psi ; \Theta ; \Delta ; \Omega ; \Gamma_1\vdash e_2 \checks \tau_2 \gens \Phi_2, \Gamma_2$ with $\Theta ; \Delta \vDash \Phi_1$ and $\Theta ; \Delta \vDash \Phi_2$. By IH, $\Psi ; \Theta ; \Delta ; \Omega ; \Gamma \setminus \Gamma_1 \vdash |e_1| : \tau_1$. Also by IH, $\Psi ; \Theta ; \Delta ; \Omega ; \Gamma_1 \setminus \Gamma_2 \vdash |e_2| : \tau_2$. By T-TensorI and weakening, $\Psi ; \Theta ; \Delta ; \Omega ; \Gamma \setminus \Gamma_2\vdash |\angles{e_1,e_2}| : \tau_1 \otimes \tau_2$.
  \item[AT-TensorE] Suppose $\Psi ; \Theta ; \Delta ; \Omega ; \Gamma\vdash \texttt{let } \angles{x,y} = e \texttt{ in } e' \checks \tau' \gens \Phi_1 \wedge \Phi_2, \Gamma_2 \setminus \{x : \tau_1,y : \tau_2\}$ by way of $\Psi ; \Theta ; \Delta ; \Omega ; \Gamma\vdash e \infers \tau_1 \otimes \tau_2 \gens \Phi_1, \Gamma_1$ and $\Psi ; \Theta ; \Delta ; \Omega ; \Gamma_1,x : \tau_1, y : \tau_2\vdash e' \checks \tau' \gens \Phi_2,\Gamma_2$ with $\Theta ; \Delta \vDash \Phi_1$ and $\Theta ; \Delta\vDash \Phi_2$. By two uses of the IH, $\Psi ; \Theta ; \Delta ; \Omega ; \Gamma \setminus \Gamma_1 \vdash |e| : \tau_1 \otimes \tau_2$ and $\Psi ; \Theta ; \Delta ; \Omega ; (\Gamma_1,x : \tau_1, y : \tau_2) \setminus \Gamma_2 \vdash |e'| : \tau'$. By weakening, $\Psi ; \Theta ; \Delta ; \Omega ; (\Gamma_1 \setminus (\Gamma_2 \setminus \{x : \tau_1, y : \tau_2\})),x : \tau_1, y : \tau_2\ \vdash |e'| : \tau'$. By T-TensorE and weakening, $\Psi ; \Theta ; \Delta ; \Omega ; \Gamma\setminus (\Gamma_2 \setminus \{x : \tau_1,y : \tau_2\}) \vdash |\texttt{let } \angles{x,y} = e \texttt{ in } e'| : \tau'$
  \item[AT-WithI] Suppose $\Psi ; \Theta ; \Delta ; \Omega ; \Gamma \vdash (e_1,e_2) \checks \tau_1 \amp \tau_2 \gens \Phi_1 \wedge \Phi_2, \Gamma_1 \cap \Gamma_2$ by way of $\Psi ; \Theta ; \Delta ; \Omega ; \Gamma \vdash e_1 \checks \tau_1 \gens \Phi_1, \Gamma_1$ and $\Psi ; \Theta ; \Delta ; \Omega ; \Gamma \vdash e_2 \checks \tau_2 \gens \Phi_2, \Gamma_2$ with $\Theta ; \Delta \vDash \Phi_1$ and $\Theta ; \Delta \vDash \Phi_2$. By applying the IH twice, $\Psi ; \Theta ; \Delta ; \Omega ; \Gamma \setminus \Gamma_1 \vdash |e_1| : \tau_1$, and $\Psi ; \Theta ; \Delta ; \Omega ; \Gamma \setminus \Gamma_2 \vdash |e_2| : \tau_2$. By T-WithI and weakening,
$\Psi ; \Theta ; \Delta ; \Omega ; \Gamma \setminus (\Gamma_1 \cap \Gamma_2)\vdash (e_1,e_2) : \tau_1 \amp \tau_2$
  \item[AT-Fst] Suppose $\Psi ; \Theta ; \Delta ; \Omega ; \Gamma \vdash \texttt{fst}(e) \infers \tau_1 \gens \Phi,\Gamma'$ by way of $\Psi ; \Theta ; \Delta ; \Omega ; \Gamma \vdash e \infers \tau_1 \amp \tau_2 \gens \Phi,\Gamma'$ with $\Theta ; \Delta \vDash \Phi$. By IH, $\Psi ; \Theta ; \Delta ; \Omega ; \Gamma \setminus \Gamma' \vdash |e| \infers \tau_1 \amp \tau_2$, and by T-Fst, $\Psi ; \Theta ; \Delta ; \Omega ; \Gamma \setminus \Gamma' \vdash \texttt{fst}(e) : \tau_1$
  \item[AT-Snd] Suppose $\Psi ; \Theta ; \Delta ; \Omega ; \Gamma \vdash \texttt{snd}(e) \infers \tau_2 \gens \Phi,\Gamma'$ by way of $\Psi ; \Theta ; \Delta ; \Omega ; \Gamma \vdash e \infers \tau_1 \amp \tau_2 \gens \Phi,\Gamma'$ with $\Theta ; \Delta \vDash \Phi$. By IH, $\Psi ; \Theta ; \Delta ; \Omega ; \Gamma \setminus \Gamma' \vdash |e| \infers \tau_1 \amp \tau_2$, and by T-Snd, $\Psi ; \Theta ; \Delta ; \Omega ; \Gamma \setminus \Gamma' \vdash \texttt{snd}(e) : \tau_2$
  \item[AT-Inl] Suppose $\Psi ; \Theta ; \Delta ; \Omega ; \Gamma \vdash \texttt{inl}(e) \checks \tau_1 \oplus \tau_2 \gens \Phi,\Gamma'$ by way of $\Psi ; \Theta ; \Delta ; \Omega ; \Gamma \vdash e \checks \tau_1 \gens \Phi,\Gamma'$ with $\Theta ; \Delta \vDash \Phi$. By IH, $\Psi ; \Theta ; \Delta ; \Omega ; \Gamma \setminus \Gamma' \vdash |e| : \tau_1$. By T-Inl,  $\Psi ; \Theta ; \Delta ; \Omega ; \Gamma \setminus \Gamma' \vdash |\texttt{inl}(e)| : \tau_1 \oplus \tau_2$ as required.
  \item[AT-Inr] Suppose $\Psi ; \Theta ; \Delta ; \Omega ; \Gamma \vdash \texttt{inr}(e) \checks \tau_1 \oplus \tau_2 \gens \Phi,\Gamma'$ by way of $\Psi ; \Theta ; \Delta ; \Omega ; \Gamma \vdash e \checks \tau_2 \gens \Phi,\Gamma'$ with $\Theta ; \Delta \vDash \Phi$. By IH, $\Psi ; \Theta ; \Delta ; \Omega ; \Gamma \setminus \Gamma' \vdash |e| : \tau_2$. By T-Inr,  $\Psi ; \Theta ; \Delta ; \Omega ; \Gamma \setminus \Gamma' \vdash |\texttt{inr}(e)| : \tau_1 \oplus \tau_2$ as required.
  \item[AT-Case] Suppose $\Psi ; \Theta ; \Delta ; \Omega ; \Gamma \vdash \texttt{case}(e,x.e_1,y.e_2) \checks \tau \gens \Phi_1 \wedge \Phi_2 \wedge \Phi_3, \Gamma'$ by way of
  \begin{itemize}
    \item $\Psi ; \Theta ; \Delta ; \Omega ; \Gamma \vdash e \infers \tau_1 \oplus \tau_2 \gens \Phi_1, \Gamma_1$
    \item $\Psi ; \Theta ; \Delta ; \Omega ; \Gamma_1, x: \tau_1 \vdash e_1 \checks \tau \gens \Phi_2,\Gamma_2$
    \item $\Psi ; \Theta ; \Delta ; \Omega ; \Gamma_1, y: \tau_2 \vdash e_2 \checks \tau \gens \Phi_3,\Gamma_3$
  \end{itemize}
  with $\Theta ; \Delta \vDash \Phi_i$, $i \in \{1,2,3\}$ . By IH, we have that
  \begin{itemize}
    \item $\Psi ; \Theta ; \Delta ; \Omega ; \Gamma \setminus \Gamma_1 \vdash |e| : \tau_1 \oplus \tau_2$
    \item $\Psi ; \Theta ; \Delta ; \Omega ; (\Gamma_1, x: \tau_1) \setminus \Gamma_2 \vdash |e_1| : \tau$
    \item $\Psi ; \Theta ; \Delta ; \Omega ; (\Gamma_1, y: \tau_2) \setminus \Gamma_2 \vdash |e_2| : \tau$
  \end{itemize}
  By T-weakening, we get that $\Psi ; \Theta ; \Delta ; \Omega ; \Gamma_1 \setminus \Gamma', x : \tau_1 \vdash |e_1| : \tau$ and $\Psi ; \Theta ; \Delta ; \Omega ; \Gamma_1 \setminus \Gamma', y : \tau_2 \vdash |e_2| : \tau$. So, by T-Case and another T-Weakening, we have that
  $\Psi ; \Theta ; \Delta ; \Omega ; \Gamma \setminus \Gamma' \vdash |\texttt{case}(e,x.e_1,y.e_2)| : \tau$ as required.
  
  \item[AT-ExpI] Suppose $\Psi ; \Theta ; \Delta ; \Omega ; \Gamma \vdash !e \checks !\tau \gens \Phi, \Gamma$ from $\Psi ; \Theta ; \Delta ; \Omega ; \cdot \vdash e \checks \tau \gens \Phi, \Gamma'$ with $\Theta ; \Delta \vDash \Phi$. By IH, $\Psi ; \Theta ; \Delta ; \Omega ; \cdot \vdash e : \tau$, and then by T-ExpI, $\Psi ; \Theta ; \Delta ; \Omega ; \cdot \vdash !e : !\tau$ as required.
  \item[AT-ExpE] Suppose $\Psi ; \Theta ; \Delta ; \Omega ; \Gamma \vdash \texttt{let } !x = e \texttt{ in } e' \checks \tau' \gens \Phi_1 \wedge \Phi_2, \Gamma_2$ by way of $\Psi ; \Theta ; \Delta ; \Omega ; \Gamma \vdash e \infers !\tau \gens \Phi_1,\Gamma_1$ and $\Psi ; \Theta ; \Delta ; \Omega, x : \tau ; \Gamma_1 \vdash e' \checks \tau' \gens \Phi_2,\Gamma_2$. 
  \item[AT-TAbs] Suppose $\Psi ; \Theta ; \Delta ; \Omega ; \Gamma \vdash \Lambda \alpha. e \checks \forall \alpha : K.\tau \gens \Phi,\Gamma'$ by way of $\Psi, \alpha : K ; \Theta ; \Delta ; \Omega ; \Gamma \vdash e \checks \tau \gens \Phi, \Gamma'$ with $\Theta ; \Delta \vDash \Phi$. By IH,  $\Psi, \alpha : K ; \Theta ; \Delta ; \Omega ; \Gamma\setminus \Gamma' \vdash |e| : \tau$. By T-TAbs, $\Psi ; \Theta ; \Delta ; \Omega ; \Gamma \setminus \Gamma' \vdash |\Lambda \alpha. e| : \forall \alpha : K.\tau$ as required.
  
  \item[AT-TApp] Suppose $\Psi ; \Theta ; \Delta ; \Omega ; \Gamma \vdash e [\tau'] \infers \tau[\tau'/\alpha] \gens \Phi_1 \wedge \Phi_2, \Gamma'$ by way of $\Psi ; \Theta ; \Delta ; \Omega ; \Gamma \vdash e \infers \forall \alpha : K.\tau \gens \Phi_1, \Gamma'$ and $\Psi ; \Theta ; \Delta \vdash \tau' : K \gens \Phi_2$, with $\Theta ; \Delta \vDash \Phi_1$ and $\Theta ; \Delta \vDash \Phi_2$. By IH, $\Psi ; \Theta ; \Delta ; \Omega ; \Gamma \setminus \Gamma' \vdash |e| : \forall \alpha : K.\tau$. By \textbf{Soundness of Kind Checking}, $\Psi ; \Theta ; \Delta \vdash \tau' : K$. So, by T-TApp,  $\Psi ; \Theta ; \Delta ; \Omega ; \Gamma\setminus \Gamma' \vdash |e [\tau']| : \tau[\tau'/\alpha]$ as required.
  
  \item[AT-IAbs]
  Suppose $\Psi ; \Theta ; \Delta ; \Omega ; \Gamma \vdash \Lambda i. e \checks \forall i : S. \tau \gens \forall i : S. \Phi, \Gamma'$ by way of $\Psi ; \Theta, i : S ; \Delta ; \Omega ; \Gamma \vdash e \checks \tau \gens \Phi, \Gamma'$ with $\Theta ; \Delta \vDash \forall i : S. \Phi$. Equivalently, $\Theta , i : S; \Delta \vDash \Phi$, and so by IH, $\Psi ; \Theta, i : S ; \Delta ; \Omega ; \Gamma \setminus \Gamma' \vdash |e| : \tau$. By T-IAbs, $\Psi ; \Theta ; \Delta ; \Omega ; \Gamma\setminus \Gamma' \vdash |\Lambda i. e| : \forall i : S. \tau$ as required.
  
  \item[AT-IApp] Suppose $\Psi ; \Theta ; \Delta ; \Omega ; \Gamma \vdash e [I] \infers \tau[I/i] \gens \Phi_1 \wedge \Phi_2,\Gamma'$ by way of $\Psi ; \Theta ; \Delta ; \Omega ; \Gamma \vdash e \infers \forall i : S.\tau \gens \Phi_1,\Gamma'$ and $\Theta ; \Delta \vdash I : S \gens \Phi_2$ with $\Theta ; \Delta \vDash \Phi_1$ and $\Theta ; \Delta \vDash \Phi_2$. By IH, $\Psi ; \Theta ; \Delta ; \Omega ; \Gamma \setminus \Gamma' \vdash |e| : \forall i : S.\tau$, and by \textbf{Soundness of Sort Checking}, $\Theta ; \Delta \vdash I : S$. Then, by T-IApp, $\Psi ; \Theta ; \Delta ; \Omega ; \Gamma\setminus \Gamma' \vdash |e [I]| : \tau[I/i]$ as required.
  
  \item[AT-Fix] Suppose $\Psi ; \Theta ; \Delta ; \Omega ; \Gamma \vdash \texttt{fix } x.e \checks \tau \gens \Phi,\Gamma$ by way of $\Psi ; \Theta ; \Delta ; \Omega, x : \tau ; \cdot \vdash e \checks \tau \gens \Phi,\Gamma'$ with $\Theta ; \Delta \vDash \Phi$. By IH, $\Psi ; \Theta ; \Delta ; \Omega, x : \tau ; \cdot \vdash |e| : \tau$. By T-Fix, $\Psi ; \Theta ; \Delta ; \Omega ; \cdot \vdash |\texttt{fix } x.e| : \tau$ as required.
  
  \item[AT-Ret] Suppose $\Psi ; \Theta ; \Delta ; \Omega ; \Gamma \vdash \texttt{ret } e \checks \M \, \phi(I,\vec{p}) \, \tau \gens \Phi, \Gamma'$ by way of $\Psi ; \Theta ; \Delta ; \Omega ; \Gamma \vdash e \checks \tau \gens \Phi,\Gamma'$ with $\Theta ; \Delta \vDash \Phi$. By IH, $\Psi ; \Theta ; \Delta ; \Omega ; \Gamma \setminus \Gamma' \vdash |e| : \tau$. By T-Ret, $\Psi ; \Theta ; \Delta ; \Omega ; \Gamma \setminus \Gamma' \vdash |\texttt{ret } e| : \M \, \phi(I,\vec{p}) \, \tau$  as required.
  
  \item[AT-Bind] Suppose $\Psi ; \Theta ; \Delta ; \Omega ; \Gamma \vdash \texttt{bind } x = e_1 \texttt{ in } e_2 \checks \M \, \phi(I,\vec{q})\, \tau_2 \gens \Phi, \Gamma_2 \setminus \{x : \tau_1\}$ by way of
  \begin{itemize}
    \item $\Psi ; \Theta ; \Delta ; \Omega ; \Gamma \vdash e_1 \infers \M \, \phi(J,\vec{p})\, \tau_1 \gens \Phi_1,\Gamma_1$
    \item $\Psi ; \Theta; \Delta ; \Omega ; \Gamma_1, x:\tau_1 \vdash e_2 \checks \M \, \phi(I,\vec{q} - \vec{p})\, \tau_2 \gens \Phi_2,\Gamma_2$
  \end{itemize}
  with $\Phi = (\vec{q} \geq \vec{p}) \wedge (I =J)  \wedge \Phi_1 \wedge \Phi_2$ and $\Theta ; \Delta \vDash \Phi$. By IH, we have that:
  \begin{itemize}
    \item $\Psi ; \Theta ; \Delta ; \Omega ; \Gamma \setminus \Gamma_1 \vdash |e_1| : \M \, \phi(J,\vec{p})\, \tau_1$
    \item $\Psi ; \Theta; \Delta ; \Omega ; (\Gamma_1, x:\tau_1) \setminus \Gamma_2 \vdash |e_2| : \M \, \phi(I,\vec{q} - \vec{p})\, \tau_2$
  \end{itemize}
  Since $\Theta ; \Delta \vDash I = J$, $\Psi ; \Theta ; \Delta \vdash M(J,\vec{p})\, \tau_1 \subty M(I,\vec{p}) \, \tau_1 \; : \star$, we have by T-Sub that
  $\Psi ; \Theta ; \Delta ; \Omega ; \Gamma \setminus \Gamma_1 \vdash |e_1| : \M \, \phi(I,\vec{p})\, \tau_1$. By T-Weaken,
  $\Psi ; \Theta; \Delta ; \Omega ; (\Gamma_1 \setminus (\Gamma_2 \setminus \{x\})), x:\tau_1 \vdash |e_2| : \M \, \phi(I,\vec{q} - \vec{p})\, \tau_2$, and so by T-Bind, T-Weaken, and T-Sub (with $\Theta ; \Delta \vDash \vec{p} + \vec{q} - \vec{p} = \vec{q}$), we have that
  $\Psi ; \Theta ; \Delta ; \Omega ; \Gamma \setminus (\Gamma_2 \setminus \{x : \tau_1\}) \vdash |\texttt{bind } x = e_1 \texttt{ in } e_2| : \M \, \phi(I,\vec{q})\, \tau_2$ as required.
  
  \item[AT-Tick] Suppose $\Psi ; \Theta ; \Delta ; \Omega ; \Gamma \vdash \texttt{tick}[I|\vec{p}] \infers \M \, \phi(I,\vec{p})\, 1 \gens \Phi_1 \wedge \Phi_2,\Gamma$ from $\Theta ; \Delta \vdash I : \N \gens \Phi_1$ and $\Theta ; \Delta \vdash \vec{p} : \vec{\mathbb{R}^+} \gens \Phi_1$ with $\Theta ; \Delta \vDash \Phi_1$ and $\Theta ; \Delta \vDash \Phi_2$. By \textbf{Soundness of Sort checking}, $\Theta ; \Delta \vdash I : \N$ and $\Theta ; \Delta \vdash \vec{p} : \vec{\mathbb{R}^+}$. Then, by T-Tick, $\Psi ; \Theta ; \Delta ; \Omega ; \cdot \vdash |\texttt{tick}[I|\vec{p}]| : \M \, \phi(I,\vec{p})\, 1$ as required.
  
  \item[AT-Release] Suppose $\Psi ; \Theta ; \Delta ; \Omega ; \Gamma \vdash \texttt{release } x = e_1 \texttt{ in }e_2 \checks \M \, \phi(I,\vec{p}) \, \tau_2 \gens (I = J \wedge \Phi_1 \wedge \Phi_2), \Gamma_2 \setminus \{x : \tau_1\}$ from $\Psi ; \Theta ; \Delta ; \Omega ; \Gamma \vdash e_1 \infers [J | \vec{q}] \tau_1 \gens \Phi_1,\Gamma_1$ and $\Psi ; \Theta ; \Delta ; \Omega ; \Gamma_1, x : \tau_1 \vdash e_2 \checks \M \, \phi(I,\vec{p} + \vec{q}) \, \tau_2 \gens \Phi_2, \Gamma_2$ with $\Theta ;  \Delta \vDash \Phi_i$ and $\Theta ; \Delta \vDash I = J$. By IH, we have that $\Psi ; \Theta ; \Delta ; \Omega ; \Gamma \setminus \Gamma_1 \vdash |e_1| : [J | \vec{q}] \tau_1$  and $\Psi ; \Theta ; \Delta ; \Omega ; (\Gamma_1, x : \tau_1) \setminus \Gamma_2 \vdash |e_2| : \M \, \phi(I,\vec{p} + \vec{q}) \, \tau_2$. Since $\Theta ; \Delta \vDash I = J$, we have that $\Psi ; \Theta ; \Delta \vdash [J|\vec{q}] \tau_1 \subty [I|\vec{q}] \tau_1 : \star$, and so by T-Sub, $\Psi ; \Theta ; \Delta ; \Omega ; \Gamma \setminus \Gamma_1 \vdash |e_1| : [I | \vec{q}] \tau_1$. By T-Weaken, we have that $\Psi ; \Theta ; \Delta ; \Omega ; (\Gamma_1 \setminus (\Gamma_2 \setminus \{x : \tau_1\})) , x : \tau_1 \vdash |e_2| : \M \, \phi(I,\vec{p} + \vec{q}) \, \tau_2$. Then, by T-Release and one more T-Weaken, we have that $\Psi ; \Theta ; \Delta ; \Omega ; \Gamma \setminus (\Gamma_2 \setminus \{x : \tau_1\}) \vdash |\texttt{release } x = e_1 \texttt{ in }e_2| : \M \, \phi(I,\vec{p}) \, \tau_2$
  \item[AT-Store] Suppose $\Psi ; \Theta ; \Delta ; \Omega ; \Gamma \vdash \texttt{store}[K|\vec{w}](e) \checks \M \, \phi(I,\vec{q}) \, ([J | \vec{p}] \, \tau) \gens \Phi, \Gamma'$ from
  \begin{itemize}
    \item $\Theta ; \Delta \vdash K : \N \gens \Phi_1$
    \item $\Theta ; \Delta \vdash \vec{w} : \vec{\mathbb{R}^+} \gens \Phi_2$
    \item $\Psi ; \Theta ; \Delta ; \Omega ; \Gamma \vdash e \checks \tau \gens \Phi_3,\Gamma'$
  \end{itemize}
  with $\Phi =  \Phi_1 \wedge \Phi_2 \wedge\Phi_3 \wedge  (\vec{p} \leq \vec{w} \leq \vec{q}) \wedge (I = J = K)$
  and $\Theta ; \Delta \vDash \Phi$. By \textbf{soundness of sort checking}, $\Theta ; \Delta \vdash K : \N$ and $\Theta ; \Delta \vdash \vec{w} : \vec{\mathbb{R}^+}$. By IH, $\Psi ; \Theta ; \Delta ; \Omega ; \Gamma \setminus \Gamma' \vdash |e| : \tau$. By T-Store,
   $\Psi ; \Theta ; \Delta ; \Omega ; \Gamma \vdash |\texttt{store}[K|\vec{w}](e)| : \M \, \phi(K,\vec{w}) \, ([K | \vec{w}] \, \tau)$. Then, since $\Theta ; \Delta \vDash \vec{p} \leq \vec{w} \leq \vec{q}$ and $\Theta ; \Delta \vDash I = J = K$ we have by T-Sub that $\Psi ; \Theta ; \Delta ; \Omega ; \Gamma \vdash |\texttt{store}[K|\vec{w}](e)| : \M \, \phi(I,\vec{q}) \, ([J | \vec{p}] \, \tau)$, as required.
  
  \item[AT-StoreConst] Suppose $\Psi ; \Theta ; \Delta ; \Omega ; \Gamma \vdash \texttt{store}[J](e) \checks \M \, \phi(K,\vec{p}) \, ([I] \, \tau) \gens \Phi, \Gamma'$ from $\Psi ; \Theta ; \Delta ; \Omega ; \Gamma \vdash e \checks \tau \gens \Phi_1,\Gamma'$ and
 $\Theta ; \Delta \vdash J \checks \mathbb{R} \gens \Phi_2$ 
 with $\Phi = (\texttt{const}(I) \leq \texttt{const}(J) \leq \vec{p}) \wedge \Phi_1 \wedge \Phi_2$ 
 and $\Theta ; \Delta \vDash \Phi$. 
 By IH, $\Psi ; \Theta ; \Delta ; \Omega ; \Gamma \setminus \Gamma' \vdash |e| : \tau$.  By \textbf{soundness of sort checking}, $\Theta ; \Delta \vdash J : \mathbb{R}$. By T-StoreConst, $\Psi ; \Theta ; \Delta ; \Omega ; \Gamma \setminus \Gamma' \vdash |\texttt{store}[J](e)| : \M \, \phi(K,\texttt{const}(J)) \, ([J] \, \tau)$. Because $\Theta ; \Delta \vDash \texttt{const}(I) \leq \texttt{const}(J) \leq \vec{p}$, we have by T-Sub that
 $\Psi ; \Theta ; \Delta ; \Omega ; \Gamma \setminus \Gamma' \vdash |\texttt{store}[J](e)| : \M \, \phi(K,\vec{p}) \, ([I] \, \tau)$, as required.
  
  \item[AT-ReleaseConst] Suppose $\Psi ; \Theta ; \Delta ; \Omega ; \Gamma \vdash \texttt{release } x = e_1 \texttt{ in }e_2 \checks \M \, \phi(I,\vec{p}) \, \tau_2 \gens \Phi_1 \wedge \Phi_2, \Gamma_2 \setminus \{x\}$
  from $\Psi ; \Theta ; \Delta ; \Omega ; \Gamma \vdash e_1 \infers [J] \tau_1 \gens \Phi_1,\Gamma_1$
  and $\Psi ; \Theta ; \Delta ; \Omega ; \Gamma_1, x : \tau_1 \vdash e_2 \checks \M \, \phi(I,\vec{p} + \texttt{const}(J)) \, \tau_2 \gens \Phi_2, \Gamma_2$
  with $\Theta ; \Delta \vDash \Phi_1 \wedge \Phi_2$.
  By IH, $\Psi ; \Theta ; \Delta ; \Omega ; \Gamma \setminus \Gamma_1 \vdash |e_1| : [J] \tau_1$
  and $\Psi ; \Theta ; \Delta ; \Omega ; (\Gamma_1, x : \tau_1) \setminus \Gamma_2 \vdash |e_2| : \M \, \phi(I,\vec{p} + \texttt{const}(J)) \, \tau_2$. By T-Weaken,
  $\Psi ; \Theta ; \Delta ; \Omega ; (\Gamma_1 \setminus (\Gamma_2 \setminus \{x : \tau_1\})), x :\tau_1\vdash |e_2| : \M \, \phi(I,\vec{p} + \texttt{const}(J)) \, \tau_2$. By T-ReleaseConst and T-Weaken, 
  $\Psi ; \Theta ; \Delta ; \Omega ; \Gamma \setminus (\Gamma_2 \setminus \{x\}) \vdash |\texttt{release } x = e_1 \texttt{ in }e_2| : \M \, \phi(I,\vec{p}) \, \tau_2$

  \item[AT-Shift] Suppose $\Psi ; \Theta ; \Delta ; \Omega ; \Gamma \vdash \texttt{shift}(e) \checks M \, \phi(I,\vec{q}) \, \tau \gens (I \geq 1) \wedge \Phi, \Gamma'$ from $\Psi ; \Theta ; \Delta  ; \Omega ; \Gamma \vdash e \checks M \, \phi(I - 1,\lhd \vec{q}) \, \tau \gens \Phi, \Gamma'$ with $\Theta ; \Delta \vDash \Phi$ and $\Theta ; \Delta \vDash  \geq 1$. By IH, $\Psi ; \Theta ; \Delta  ; \Omega ; \Gamma \setminus \Gamma' \vdash |e| : M \, \phi(I - 1,\lhd \vec{q}) \, \tau$. Since $\Theta ; \Delta \vDash I \geq 1$, we have by T-Shift that $\Psi ; \Theta ; \Delta ; \Omega ; \Gamma \setminus \Gamma' \vdash |\texttt{shift}(e)| : M \, \phi(I,\vec{q}) \, \tau$ as required.
  
  \item[AT-CImpI] Suppose $\Psi ; \Theta ; \Delta ; \Omega ; \Gamma \vdash \Lambda .e \checks (\Phi' \Rightarrow \tau) \gens (\Phi' \to \Phi),\Gamma'$ from $\Psi ; \Theta ; \Delta,\Phi' ; \Omega ; \Gamma \vdash e \checks \tau \gens \Phi,\Gamma'$ with $\Theta ; \Delta \vDash \Phi' \to \Phi$, or equivalently, $\Theta ; \Delta, \Phi' \vDash \Phi$. Then by IH, $\Psi ; \Theta ; \Delta,\Phi' ; \Omega ; \Gamma \setminus \Gamma' \vdash |e| : \tau$  Then, by T-CImpI, $\Psi ; \Theta ; \Delta ; \Omega ; \Gamma \setminus \Gamma' \vdash |\Lambda .e| : (\Phi' \Rightarrow \tau)$ as required.
  \item[AT-CImpE] Suppose $\Psi ; \Theta ; \Delta ; \Omega ; \Gamma \vdash e \{\} \infers \tau \gens \Phi \wedge \Phi',\Gamma'$ from $\Psi ; \Theta ; \Delta ; \Omega ; \Gamma \vdash e \infers (\Phi' \Rightarrow \tau) \gens \Phi,\Gamma'$ with $\Theta ; \Delta \vDash \Phi$. By IH, $\Psi ; \Theta ; \Delta ; \Omega ; \Gamma \setminus \Gamma' \vdash |e| : (\Phi' \Rightarrow \tau)$. By T-CImpE, $\Psi ; \Theta ; \Delta ; \Omega ; \Gamma \setminus \Gamma' \vdash |e \{\}| : \tau$ as required.
  \item[AT-CAndI] Suppose $\Psi ; \Theta ; \Delta ; \Omega ; \Gamma \vdash <e> \checks \Phi' \amp \tau \gens \Phi \wedge \Phi',\Gamma'$ from $\Psi ; \Theta ; \Delta ; \Omega ; \Gamma \vdash e \checks \tau \gens \Phi,\Gamma'$ with $\Theta ; \Delta \vDash \Phi$ and $\Theta ; \Delta \vDash \Phi'$. By IH, $\Psi ; \Theta ; \Delta ; \Omega ; \Gamma \setminus \Gamma' \vdash |e| : \tau$ By T-CAnd, $\Psi ; \Theta ; \Delta ; \Omega ; \Gamma \setminus \Gamma' \vdash |<e>| : \Phi' \amp \tau$ as required.
  \item[AT-CAndE] Suppose $\Psi ; \Theta ; \Delta ; \Omega ; \Gamma \vdash \texttt{clet } x = e \texttt{ in } e' \checks \tau' \gens \Phi_1 \wedge (\Phi' \to \Phi_2),\Gamma_2 \setminus \{x : \tau\}$ from $\Psi ; \Theta ; \Delta ; \Omega ; \Gamma \vdash e \infers \Phi' \amp \tau \gens \Phi_1,\Gamma_1$ and $\Psi ; \Theta ; \Delta, \Phi' ; \Omega ; \Gamma_1, x : \tau \vdash e' \checks \tau' \gens \Phi_2, \Gamma_2$ with $\Theta ; \Delta \vDash \Phi_1$ and $\Theta ; \Delta \vDash (\Phi' \to \Phi_2)$. By IH,
   $\Psi ; \Theta ; \Delta ; \Omega ; \Gamma \setminus \Gamma_1 \vdash |e| : \Phi' \amp \tau$. Since $\Theta; \Delta,\Phi' \vDash \Phi_2$, we have by IH that
   $\Psi ; \Theta ; \Delta, \Phi' ; \Omega ; (\Gamma_1, x : \tau) \setminus \Gamma_2 \vdash |e'| : \tau'$. By T-Weaken,
   $\Psi ; \Theta ; \Delta, \Phi' ; \Omega ; (\Gamma_1 \setminus (\Gamma_2 \setminus \{x : \tau\})), x : \tau\vdash |e'| : \tau'$, and so by T-CAndE and T-Weaken, 
   $\Psi ; \Theta ; \Delta ; \Omega ; \Gamma \setminus (\Gamma_2 \setminus \{x : \tau\}) \vdash |\texttt{clet } x = e \texttt{ in } e'| : \tau'$ as required.
  
 
  \item[AT-Sub] Suppose $\Psi ; \Theta ; \Delta ; \Omega ; \Gamma \vdash e \checks \tau \gens \Phi_1 \wedge \Phi_2,\Gamma'$ by way of $\Psi ; \Theta ; \Delta ; \Omega ; \Gamma \vdash e \infers \tau' \gens \Phi_1,\Gamma'$ and $\Psi;\Theta;\Delta \vdash \tau' \subty \tau : \star \gens \Phi_2$ with $\Theta ; \Delta \vDash \Phi_1$ and $\Theta ; \Delta \vDash \Phi_2$. By IH,  $\Psi ; \Theta ; \Delta ; \Omega ; \Gamma \setminus \Gamma' \vdash |e| : \tau'$. By \textbf{Soundness of Subtyping}, $\Psi;\Theta;\Delta \vdash \tau' \subty \tau : \star$. Then, by T-Sub, $\Psi ; \Theta ; \Delta ; \Omega ; \Gamma\setminus \Gamma' \vdash |e| : \tau$ as required.
  
  
  \item[AT-Anno] Suppose $\Psi ; \Theta ; \Delta ; \Omega ; \Gamma \vdash (e : \tau) \infers \tau \gens \Phi,\Gamma'$ by way of $\Psi ; \Theta ; \Delta ; \Omega ; \Gamma \vdash e \checks \tau \gens \Phi,\Gamma'$ with $\Theta ; \Delta \vDash \Phi$. By IH, $\Psi ; \Theta ; \Delta ; \Omega ; \Gamma \setminus \Gamma' \vdash |e| : \tau$. Then, because $|e| = |(e : \tau)|$, we are done.
   
\end{enumerate}
\end{proof}


\tychecksound*
\begin{proof}
Immediate by \ref{thm:raw-tycheck-sound}.
\end{proof}

%%%
% COMPLETENESS
%%%



\begin{theorem}[Raw Completeness of Sort Checking/Inference]
If $\Theta;\Delta \vdash I : S$, then $\Theta;\Delta \vdash I : S \gens \Phi$ and $\Theta;\Delta \vDash \Phi$.
%\textbf{REDO THIS, FIX Proof file Name}
\label{thm:raw-sort-compl}
\end{theorem}
\begin{proof}
By induction on $\Theta;\Delta \vdash I : S$.

\begin{itemize}
  \item[(I-Var)] Immediate by AI-Var with $\Phi = \top$.
  \item[(I-Plus)] Suppose $\Theta ; \Delta \vdash I + J : bS$ from $\Theta ; \Delta \vdash I : bS$ and $\Theta ; \Delta \vdash J : bS$. By IH, $\Theta ; \Delta \vdash I : bS \gens \Phi_1$ and $\Theta ; \Delta \vdash J : bS \gens \Phi_2$ with $\Theta ; \Delta \vDash \Phi_1$ and $\Theta; \Delta \vDash  \Phi_2$. By AI-Plus, $\Theta ; \Delta \vdash I + J : bS \gens \Phi_1 \wedge \Phi_2$
  \item[(I-Minus)] Suppose $\Theta ; \Delta \vdash I - J : bS$ from $\Theta ; \Delta \vdash I : bS$ and $\Theta ; \Delta \vdash J : bS$, and $\Theta ; \Delta \vDash I \geq J$. By IH, $\Theta ; \Delta \vdash I : bS \gens \Phi_1$ and $\Theta ; \Delta \vdash J : bS \gens \Phi_2$ with $\Theta ; \Delta \vDash \Phi_1$ and $\Theta; \Delta \vDash  \Phi_2$. By AI-Minus, $\Theta ; \Delta \vdash I - J : bS \gens \Phi_1 \wedge \Phi_2 \wedge (I \geq J)$
  \item[(I-Times-$\mathbb{R}$)]
  \item[(I-Times-$\vec{\mathbb{R}}$)] 
  \item[(I-Times-$\mathbb{N}$)]
  \item[(I-Shift)] Suppose $\Theta ; \Delta \vdash \; \lhd I : \vec{\mathbb{R}^+}$ from  $\Theta ; \Delta \vdash I : \vec{\mathbb{R}^+}$. By IH, $\Theta ; \Delta \vdash I : \vec{\mathbb{R}^+} \gens \Phi$, and $\Theta ; \Delta \vDash \Phi$.  By AI-Shift,  $\Theta ; \Delta \vdash \; \lhd I : \vec{\mathbb{R}^+} \gens \Phi$, as required.
  \item[(I-Lam)] Suppose $\Theta ; \Delta \vdash \lambda i : bS. I : bS \to S$ from $\Theta, i : bS ; \Delta \vdash I : S$. By IH, $\Theta, i : bS ; \Delta \vdash I : S \gens \Phi$ with $\Theta ; \Delta \vDash  \Phi$. By AI-Lam, $\Theta ; \Delta \vdash \lambda i : bS. I : bS \to S \gens \Phi$.
  \item[(I-App)]
  \item[(I-Sum)]
\end{itemize}
\end{proof}

\begin{theorem}[Raw Completeness of Constraint Well-Formedness]
If $\Theta ; \Delta \vdash \Phi \; \texttt{wf}$, then $\Theta ; \Delta \vdash \Phi \; \texttt{wf} \gens \Phi'$ with $\Theta ; \Delta \vDash \Phi'$
\label{thm:raw-constr-compl}
\end{theorem}

\idxctxwfcompl*
\begin{proof}
By induction on $\Theta \vdash \Delta \; \texttt{wf}$. The base case is immediate.
Suppose $\Theta \vdash \Delta,\Phi \; \texttt{wf}$ by way of $\Theta \vdash \Delta \; \texttt{wf}$ and $\Theta ; \Delta \vdash \Phi \; \texttt{wf}$.
By IH, there is some $\Phi_1$ such that $\Theta \vdash \Delta \; \texttt{wf} \gens \Phi_1$ with $\Theta ; \cdot \vDash \Phi_1$. By Theorem~\ref{thm:raw-constr-compl}, there is $\Phi_2$ such that $\Theta ; \Delta \vdash \Phi \; \texttt{wf} \gens \Phi_2$ with $\Theta ; \Delta \vDash \Phi_2$. Equivalently, $\Theta ; \cdot \vDash \bigwedge \Delta \to \Phi_2$, and so $\Theta ; \cdot \vDash \Phi_1 \wedge (\bigwedge \Delta \to \Phi_2)$. Then, by AWF-CCtx-Ne, $\Theta \vdash \Delta,\Phi \; \texttt{wf} \gens \Phi_1 \wedge (\bigwedge \Delta \to \Phi_2)$, as required.
\end{proof}

\sortcompl*
\begin{proof}
Immediate by Theorem~\ref{thm:raw-sort-compl} and Theorem~\ref{thm:idx-ctx-wf-compl}
\end{proof}

\constrcompl*
\begin{proof}
Immediate by Theorem~\ref{thm:raw-constr-compl} and Theorem~\ref{thm:idx-ctx-wf-compl}
\end{proof}

\begin{theorem}[Raw Completeness of Kind Checking/Inference]
If $\Phi;\Theta;\Delta \vdash \tau : K$, then $\Phi;\Theta;\Delta \vdash \tau : K \gens \Phi$ with $\Theta ; \Delta \vDash \Phi$
\label{thm:raw-kind-compl}
\end{theorem}
\jtheorem{Proof of \autoref{thm:raw-kind-compl}}{
  \jgivengoal{
    \caseFact{1} $\Psi ; \Theta ; \Delta \vdash \tau : K$  
  }{
    $\Phi;\Theta;\Delta \vdash \tau : K \gens \Phi$ and $\Theta ; \Delta \vDash \Phi$  
  }
  
  \jcase{1}{K-Var}{Immediate.}
  \jcase{2}{K-Unit}{Immediate.}
  \jcase{3}{K-Arr}{
    \jgivengoal{
      \caseFact{1} $\Psi ; \Theta ; \Delta \vdash \tau_1 \loli \tau_2 : \star$
      
      \caseFact{2} $\Psi ; \Theta ; \Delta \vdash \tau_1 : \star$
      
      \caseFact{3} $\Psi ; \Theta ; \Delta \vdash \tau_2 : \star$      
    }{
       $\Psi ; \Theta ; \Delta \vdash \tau_1 \loli \tau_2 : \star \gens \Phi'$ and $\Theta ; \Delta \vDash \Phi'$
    }
    \caseText{By IH on (2) and (3)}
    
    \caseFact{4} $\Psi ; \Theta ; \Delta \vdash \tau_1 : \star \gens \Phi_1$
    
    \caseFact{5} $\Psi ; \Theta ; \Delta \vdash \tau_2 : \star \gens \Phi_2$
    
    \caseFact{6} $\Theta ; \Delta \vDash \Phi_1 \wedge \Phi_2$
    
    \caseText{By AK-Arr on (4) and (5)}
    
    \caseFact{7} $\Psi ; \Theta ; \Delta \vdash \tau_1 \loli \tau_2 : \star \gens \Phi_1 \wedge \Phi_2$
    
    \caseText{The Goal follows by (6) and (7), with $\Phi' = \Phi_1 \wedge \Phi_2$}
  }
  
  \jcase{4}{K-Tensor}{
    \jgivengoal{
      \caseFact{1} $\Psi ; \Theta ; \Delta \vdash \tau_1 \otimes \tau_2 : \star$
      
      \caseFact{2} $\Psi ; \Theta ; \Delta \vdash \tau_1 : \star$
      
      \caseFact{3} $\Psi ; \Theta ; \Delta \vdash \tau_2 : \star$
    }{
       $\Psi ; \Theta ; \Delta \vdash \tau_1 \otimes \tau_2 : \star \gens \Phi'$ and $\Theta ; \Delta \vDash \Phi'$
    }
    \caseText{By IH on (2) and (3)}
    
    \caseFact{4} $\Psi ; \Theta ; \Delta \vdash \tau_1 : \star \gens \Phi_1$
    
    \caseFact{5} $\Psi ; \Theta ; \Delta \vdash \tau_2 : \star \gens \Phi_2$
    
    \caseFact{6} $\Theta ; \Delta \vDash \Phi_1 \wedge \Phi_2$
    
    \caseText{By AK-Tensor on (4) and (5)}
    
    \caseFact{7} $\Psi ; \Theta ; \Delta \vdash \tau_1 \otimes \tau_2 : \star \gens \Phi_1 \wedge \Phi_2$
    
    \caseText{The Goal follows by (6) and (7), with $\Phi' = \Phi_1 \wedge \Phi_2$}
  }
  
  \jcase{5}{K-With}{
    \jgivengoal{
      \caseFact{1} $\Psi ; \Theta ; \Delta \vdash \tau_1 \amp \tau_2 : \star$
      
      \caseFact{2} $\Psi ; \Theta ; \Delta \vdash \tau_1 : \star$
      
      \caseFact{3} $\Psi ; \Theta ; \Delta \vdash \tau_2 : \star$
    }{
       $\Psi ; \Theta ; \Delta \vdash \tau_1 \amp \tau_2 : \star \gens \Phi'$ and $\Theta ; \Delta \vDash \Phi'$
    }
    \caseText{By IH on (2) and (3)}
    
    \caseFact{4} $\Psi ; \Theta ; \Delta \vdash \tau_1 : \star \gens \Phi_1$
    
    \caseFact{5} $\Psi ; \Theta ; \Delta \vdash \tau_2 : \star \gens \Phi_2$
    
    \caseFact{6} $\Theta ; \Delta \vDash \Phi_1 \wedge \Phi_2$
    
    \caseText{By K-With on (4) and (5)}
    
    \caseFact{7} $\Psi ; \Theta ; \Delta \vdash \tau_1 \amp \tau_2 : \star \gens \Phi_1 \wedge \Phi_2$
    
    \caseText{The Goal follows by (6) and (7), with $\Phi' = \Phi_1 \wedge \Phi_2$}
  }
  
  \jcase{6}{K-Sum}{
    \jgivengoal{
      \caseFact{1} $\Psi ; \Theta ; \Delta \vdash \tau_1 \oplus \tau_2 : \star$
      
      \caseFact{2} $\Psi ; \Theta ; \Delta \vdash \tau_1 : \star$
      
      \caseFact{3} $\Psi ; \Theta ; \Delta \vdash \tau_2 : \star$
    }{
       $\Psi ; \Theta ; \Delta \vdash \tau_1 \oplus \tau_2 : \star \gens \Phi'$ and $\Theta ; \Delta \vDash \Phi'$
    }
    \caseText{By IH on (2) and (3)}
    
    \caseFact{4} $\Psi ; \Theta ; \Delta \vdash \tau_1 : \star \gens \Phi_1$
    
    \caseFact{5} $\Psi ; \Theta ; \Delta \vdash \tau_2 : \star \gens \Phi_2$
    
    \caseFact{6} $\Theta ; \Delta \vDash \Phi_1 \wedge \Phi_2$
    
    \caseText{By AK-Sum on (4) and (5)}
    
    \caseFact{7} $\Psi ; \Theta ; \Delta \vdash \tau_1 \oplus \tau_2 : \star \gens \Phi_1 \wedge \Phi_2$
    
    \caseText{The Goal follows by (6) and (7), with $\Phi' = \Phi_1 \wedge \Phi_2$}
  }
  
  \jcase{7}{K-Bang}{
    \jgivengoal{
      \caseFact{1} $\Psi ; \Theta ; \Delta \vdash !\tau: \star$
      \caseFact{2} $\Psi ; \Theta ; \Delta \vdash \tau : \star$
    }{
       $\Psi ; \Theta ; \Delta \vdash !\tau : \star \gens \Phi'$ and $\Theta ; \Delta \vDash \Phi'$
    }
    \caseText{By IH on (2)}
    
    \caseFact{3} $\Psi ; \Theta ; \Delta \vdash \tau : \star \gens \Phi'$
    
    \caseFact{4} $\Theta ; \Delta \vDash \Phi'$
    
    \caseText{By AK-Bang on (3)}
    
    \caseFact{4} $\Psi ; \Theta ; \Delta \vdash !\tau : \star \gens \Phi'$
  }
  
  
  \jcase{8}{K-IForall}{
    \jgivengoal{
      \caseFact{1} $\Psi ; \Theta ; \Delta \vdash \forall i : S. \tau : \star$
      
      \caseFact{2} $\Psi ; \Theta, i : S ; \Delta \vdash \tau : \star$
    }{
      $\Psi ; \Theta ; \Delta \vdash \forall i : S. \tau : \star \gens \Phi$ and $\Theta ; \Delta \vDash \Phi$    
    }
    \caseText{By IH on (2)}
    
    \caseFact{3} $\Psi ; \Theta, i : S ; \Delta \vdash \tau : \star \gens \Phi$
    
    \caseFact{4} $\Theta, i : S ; \Delta \vDash \Phi$
    
    \caseText{Equivalently to (4)}
    
    \caseFact{5} $\Theta ; \Delta \vDash \forall i : S. \Phi$
    
    \caseText{By AK-IForall on (3)}
    
    \caseFact{6} $\Psi ; \Theta ; \Delta \vdash \forall i : S. \tau : \star \gens \forall i : S. \Phi$
  }
  
  \jcase{9}{K-IExists}{
    \jgivengoal{
      \caseFact{1} $\Psi ; \Theta ; \Delta \vdash \exists i : S. \tau : \star$
      
      \caseFact{2} $\Psi ; \Theta, i : S ; \Delta \vdash \tau : \star$
    }{
      $\Psi ; \Theta ; \Delta \vdash \exists i : S. \tau : \star \gens \Phi$ and $\Theta ; \Delta \vDash \Phi$    
    }
    \caseText{By IH on (2)}
    
    \caseFact{3} $\Psi ; \Theta, i : S ; \Delta \vdash \tau : \star \gens \Phi$
    
    \caseFact{4} $\Theta, i : S ; \Delta \vDash \Phi$
    
    \caseText{Equivalently to (4)}
    
    \caseFact{5} $\Theta ; \Delta \vDash \forall i : S. \Phi$
    
    \caseText{By AK-IExists on (3)}
    
    \caseFact{6} $\Psi ; \Theta ; \Delta \vdash \exists i : S. \tau : \star \gens \forall i : S. \Phi$
  }
  
  \jcase{10}{K-List}{
    \jgivengoal{
      \caseFact{1} $\Psi ; \Theta ; \Delta \vdash L^I \tau : \star$
      
      \caseFact{2} $\Theta ; \Delta \vdash I : \N$
      
      \caseFact{3} $\Psi ; \Theta ; \Delta \vdash \tau : \star$
    }{
      $\Psi ; \Theta ; \Delta \vdash L^I \tau : \star \gens \Phi$ and $\Theta ; \Delta \vDash \Phi$
    }
    \caseText{By IH on (3)}
    
    \caseFact{4} $\Psi ; \Theta ; \Delta \vdash \tau : \star \gens \Phi_1$
    
    \caseFact{5} $\Theta ; \Delta \vDash \Phi_1$
    
    \caseText{By \autoref{thm:raw-sort-compl} on (2)}
    
    \caseFact{6} $\Theta ; \Delta \vdash I : \N \gens \Phi_2$
    
    \caseFact{7} $\Theta ; \Delta \vDash \Phi_2$
    
    \caseText{By AK-List on (4) and (6)}
    
    \caseFact{8} $\Psi ; \Theta ; \Delta \vdash L^I \tau : \star \gens \Phi_1 \wedge \Phi_2$
    
    \caseText{The goal follows from (5), (7), (8) with $\Phi = \Phi_1 \wedge \Phi_2$}
  }
  
  \jcase{11}{K-Conj}{
    \jgivengoal{
      \caseFact{1} $\Psi ; \Theta ; \Delta \vdash \Phi \amp \tau : \star$
      
      \caseFact{2} $\Theta ; \Delta \vdash \Phi \texttt{ wf}$
      
      \caseFact{3} $\Psi ; \Theta ; \Delta \vdash \tau : \star$
    }{
      $\Psi ; \Theta ; \Delta \vdash \Phi \amp \tau : \star \gens \Phi$ and $\Theta ; \Delta \vDash \Phi$
    }
    \caseText{By \autoref{thm:raw-constr-compl} on (2)}
    
    \caseFact{4} $\Theta ; \Delta \vdash \Phi \; \texttt{wf} \gens \Phi_1$
    
    \caseFact{5} $\Theta ; \Delta \vDash \Phi_!$
    
    \caseText{By IH on (3)}
    
    \caseFact{6} $\Psi ; \Theta ; \Delta \vdash \tau : \star \gens \Phi_2$
    
    \caseFact{7} $\Theta ; \Delta \vDash \Phi_2$
    
    \caseText{By AK-Conj on (4) and (6)}
    
    \caseFact{8} $\Psi ; \Theta ; \Delta \vdash \Phi \amp \tau : \star \gens \Phi_1 \wedge \Phi_2$
    
    \caseText{The Goal follows by (5), (7), (8)}
  }
  
  \jcase{12}{K-Impl}{
    \jgivengoal{
      \caseFact{1} $\Psi ; \Theta ; \Delta \vdash \Phi \implies \tau : \star$
      
      \caseFact{2} $\Theta ; \Delta \vdash \Phi \texttt{ wf}$
      
      \caseFact{3} $\Psi ; \Theta ; \Delta \vdash \tau : \star$
    }{
      $\Psi ; \Theta ; \Delta \vdash \Phi \implies \tau : \star \gens \Phi$ and $\Theta ; \Delta \vDash \Phi$
    }
    \caseText{By \autoref{thm:raw-constr-compl} on (2)}
    
    \caseFact{4} $\Theta ; \Delta \vdash \Phi \; \texttt{wf} \gens \Phi_1$
    
    \caseFact{5} $\Theta ; \Delta \vDash \Phi_!$
    
    \caseText{By IH on (3)}
    
    \caseFact{6} $\Psi ; \Theta ; \Delta \vdash \tau : \star \gens \Phi_2$
    
    \caseFact{7} $\Theta ; \Delta \vDash \Phi_2$
    
    \caseText{By AK-Impl on (4) and (6)}
    
    \caseFact{8} $\Psi ; \Theta ; \Delta \vdash \Phi \implies \tau : \star \gens \Phi_1 \wedge \Phi_2$
    
    \caseText{The Goal follows by (5), (7), (8)}
  }
  

  \jcase{13}{K-Monad}{
    \jgivengoal{
      \caseFact{1} $\Psi ; \Theta ; \Delta \vdash \M(I,\vec{p}) \tau : \star$
      
      \caseFact{2} $\Theta ; \Delta \vdash I : \N$
      
      \caseFact{3} $\Theta ; \Delta \vdash \vec{p} : \potvec$
      
      \caseFact{4} $\Psi ; \Theta ; \Delta \vdash \tau : \star$
    }{
      $\Psi ; \Theta ; \Delta \vdash \M(I,\vec{p}) \tau : \star \gens \Phi$ and $\Theta ; \Delta \vDash \Phi$
    }
    \caseText{By \autoref{thm:raw-sort-compl} on (2) and (3)}
    
    \caseFact{5} $\Theta ; \Delta \vdash I : \N \gens \Phi_1$
    
    \caseFact{6} $\Theta ; \Delta \vdash \vec{p} : \potvec \gens \Phi_2$
    
    \caseFact{7} $\Theta ; \Delta \vDash \Phi_1 \wedge \Phi_2$
    
    \caseText{By IH on (4)}
    
    \caseFact{8} $\Psi ; \Theta ; \Delta \vdash \tau : \star \gens \Phi_3$
    
    \caseFact{9} $\Theta ; \Delta \vDash \Phi_3$
    
    \caseText{By AK-Monad on (5), (6), (8)}
    
    \caseFact{10} $\Psi ; \Theta ; \Delta \vdash \M(I,\vec{p}) \tau : \star \gens \Phi_1 \wedge \Phi_2 \wedge \Phi_3$
    
    \caseText{Goal follows by (7), (9), (10)}
  
  }
  
  \jcase{14}{K-Pot}{
    \jgivengoal{
      \caseFact{1} $\Psi ; \Theta ; \Delta \vdash [I|\vec{p}] \tau : \star$
      
      \caseFact{2} $\Theta ; \Delta \vdash I : \N$
      
      \caseFact{3} $\Theta ; \Delta \vdash \vec{p} : \potvec$
      
      \caseFact{4} $\Psi ; \Theta ; \Delta \vdash \tau : \star$
    }{
      $\Psi ; \Theta ; \Delta \vdash [I|\vec{p}] \tau : \star \gens \Phi$ and $\Theta ; \Delta \vDash \Phi$
    }
    \caseText{By \autoref{thm:raw-sort-compl} on (2) and (3)}
    
    \caseFact{5} $\Theta ; \Delta \vdash I : \N \gens \Phi_1$
    
    \caseFact{6} $\Theta ; \Delta \vdash \vec{p} : \potvec \gens \Phi_2$
    
    \caseFact{7} $\Theta ; \Delta \vDash \Phi_1 \wedge \Phi_2$
    
    \caseText{By IH on (4)}
    
    \caseFact{8} $\Psi ; \Theta ; \Delta \vdash \tau : \star \gens \Phi_3$
    
    \caseFact{9} $\Theta ; \Delta \vDash \Phi_3$
    
    \caseText{By AK-Pot on (5), (6), (8)}
    
    \caseFact{10} $\Psi ; \Theta ; \Delta \vdash [I|\vec{p}] \tau : \star \gens \Phi_1 \wedge \Phi_2 \wedge \Phi_3$
    
    \caseText{Goal follows by (7), (9), (10)}  
  }
  
  \jcase{15}{K-ConstPot}{
  
    \jgivengoal{
      \caseFact{1} $\Psi ; \Theta ; \Delta \vdash [I] \; \tau : \star$
      
      \caseFact{2} $\Theta ; \Delta \vdash I : \R^+$
      
      \caseFact{3} $\Psi ; \Theta ; \Delta \vdash \tau : \star$
    }{
     $\Psi ; \Theta ; \Delta \vdash [I] \; \tau : \star \gens \Phi$ and $\Theta ; \Delta \vDash \Phi$
    }
    
    \caseText{By \autoref{thm:raw-sort-compl} on  (2)}
    
    \caseFact{4} $\Theta ; \Delta \vdash I : \R^+ \gens \Phi_1$
    
    \caseFact{5} $\Theta ; \Delta \vDash \Phi_1$
    
    \caseText{By IH on (3)}
    
    \caseFact{6} $\Psi ; \Theta ; \Delta \vdash \tau : \star \gens \Phi_2$
    
    \caseFact{7} $\Theta ; \Delta \vDash \Phi_2$
    
    \caseText{Applying AK-ConstPot to (4) and (6)}
    
    \caseFact{8} $\Psi ; \Theta ; \Delta \vdash [I] \; \tau : \star \gens \Phi_1 \wedge \Phi_2$
    
    \caseText{Goal follows by (5), (7), (8)}
 
   }
   
   \jcase{16}{K-FamLam}{
     \jgivengoal{
       \caseFact{1} $\Psi ; \Theta ; \Delta \vdash \lambda i : S. \tau : S \to K$
       
       \caseFact{2} $\Psi ; \Theta, i : S ; \Delta \vdash \tau : K$
     }{
       $\Psi ; \Theta ; \Delta \vdash \lambda i : S. \tau : S \to K \gens \Phi$ and $\Theta ; \Delta \vDash \Phi$
     }
     \caseText{By IH on (2)}
     
     \caseFact{3} $\Psi ; \Theta, i : S; \Delta \vdash \tau : K \gens \Phi$
     
     \caseFact{4} $\Theta, i : S; \Delta \vDash \Phi$
     
     \caseText{By AK-FamLam on (3)}
     
     \caseFact{5} $\Psi ; \Theta ; \Delta \vdash \lambda i : S. \tau : S \to K \gens \forall i : S. \Phi$
     
     \caseText{Equivalently to (4)}
     
     \caseFact{6} $\Theta ; \Delta \vDash \forall i : S. \Phi$
   }
   
  
  \jcase{17}{K-FamApp}{
    \jgivengoal{
      \caseFact{1} $\Psi ; \Theta ; \Delta \vdash \tau \; I : K$
      
      \caseFact{2} $\Psi ; \Theta ; \Delta \vdash \tau : S \to K$
      
      \caseFact{3} $\Theta ; \Delta \vdash I : S$
    }{
      $\Psi ; \Theta ; \Delta \vdash \tau \; I : K \gens \Phi$ and $\Theta ; \Delta \vDash \Phi$    
    }
    \caseText{By IH on (2)}
    
    \caseFact{4} $\Psi ; \Theta ; \Delta \vdash \tau : S \to K \gens \Phi_1$
    
    \caseFact{5} $\Theta ; \Delta \vDash \Phi_1$
    
    \caseText{By \autoref{thm:raw-sort-compl} on (3)}
    
    \caseFact{6} $\Theta ; \Delta \vdash I : S \gens \Phi_2$
    
    \caseFact{7} $\Theta ; \Delta \vDash \Phi_2$
    
    \caseText{By AK-FamApp on (4) and (6)}
    
    \caseFact{8} $\Psi ; \Theta ; \Delta \vdash \tau \; I : K \gens \Phi_1 \wedge \Phi_2$
    
    \caseText{Goal is done by (5), (7), and (8)}
  }
}   

\kindcompl*
\begin{proof}
Immediate by Theorem~\ref{thm:raw-kind-compl} and Theorem~\ref{thm:idx-ctx-wf-compl}.
\end{proof}

\subtynerefl*
\jtheorem{Proof of \autoref{thm:subtyne-refl}}{

\jgivengoal{
  \caseFact{1} $\Psi ; \Theta ;  \Delta \pvdash \tau : K$
  
  \caseFact{2} $\tau \;\texttt{nf}$
}{
  $\Psi ; \Theta ;  \Delta \pvdash \tau\subtynf \tau : K \gens \Phi$ with $\Theta ; \Delta \vDash \Phi$
}

\caseText{By induction on (1) and inversion on (2)}

\jcase{1}{K-Var}{Immediate}

\jcase{2}{K-Unit}{Immediate}

\jcase{3}{K-FamApp}{
  \jgivengoal{
    \caseFact{1} ${\Psi ; \Theta ; \Delta \pvdash \tau \; I : K}$
    
    \caseFact{2} $\tau \; I \; \texttt{ne}$
    
    \caseFact{3} $\Psi ; \Theta ; \Delta \vdash \tau : S \to K$
    
    \caseFact{4} $\Theta ; \Delta \vdash I : S$
  }{
    $\Psi ; \Theta ;  \Delta \pvdash \tau \; I \subtynf \tau \; I: K \gens \Phi$ with $\Theta ; \Delta \vDash \Phi$
  }
  
  \caseText{By inversion on (2)}
  
  \caseFact{5} $\tau \; \texttt{ne}$
  
  \caseText{By IH on (3)}
  
  \caseFact{6} $\Psi ; \Theta ; \Delta \pvdash \tau \subtynf \tau : S \to K \gens \Phi$
  
  \caseFact{7} $\Theta ; \Delta \vDash \Phi$
  
  \caseText{By AK-FamApp on (6)}
  
  \caseFact{8} $\Psi ; \Theta ; \Delta \vdash \tau \; I \subtynf \tau \; I : K \gens \Phi \wedge (I = I)$
  
  \caseText{We re-establish the presupposition for (8) by applying \autoref{thm:sort-sound} to $\Theta ; \Delta \vdash I : S$ from (1)}
  
  \caseFact{9} $\Psi ; \Theta ; \Delta \pvdash \tau \; I \subtynf \tau \; I : K \gens \Phi \wedge (I = I)$
  
  \caseText{By (7), (9), and the fact that $\Theta ; \Delta \vDash I = I$}
}
  
}

\iffalse
\begin{proof}
By induction on $\Psi ; \Theta ;  \Delta \vdash \tau : K$. Inverting $\tau \;\texttt{ne}$, we see that the only possible cases are K-Var, K-Unit, and K-FamApp. The K-Var and K-Unit cases are immediate by AS-Var and AS-Unit.

Otherwise, suppose ${\Psi ; \Theta ; \Delta \vdash \tau \; I : K}$ from $\Psi ; \Theta ; \Delta \vdash \tau : S \to K$ with $\Theta ; \Delta \vdash I : S$. 

Here, $\tau \; I \; \texttt{ne}$ by way of $\tau \texttt{ne}$, and so by IH, $\Psi ; \Theta ; \Delta \vdash \tau \subty \tau : S \to K \gens \Phi$ with $\Theta ; \Delta \vDash \Phi$. By AS-FamApp, $\Psi ; \Theta ; \Delta \vdash \tau \; I \subty \tau \; I: S \to K \gens \Phi\wedge (I=I)$, and of course $\Theta ; \Delta \vDash I = I$, and so we are done.
\end{proof}
\fi

\subtynfrefl*
\jtheorem{Proof of Theorem~\ref{thm:subtynf-refl}}{

  \jgivengoal{
    \caseFact{1} $\Psi ; \Theta ;  \Delta \pvdash \tau : K$
    
    \caseFact{2} $\tau \;\texttt{nf}$  
  }{
    $\Psi ; \Theta ;  \Delta \pvdash \tau\subtynf \tau : K \gens \Phi$ with $\Theta ; \Delta \vDash \Phi$
  }
  \caseText{By induction on (1), followed by inversion on (2)}
  
  \jcase{1}{K-Var}{Immediate by Theorem~\ref{thm:subtyne-refl}}
  
  \jcase{2}{K-Unit}{Immediate by Theorem~\ref{thm:subtyne-refl}}
  
  \jcase{3}{K-Arr}{
    \jgivengoal{
      \caseFact{1} ${\Psi ; \Theta ; \Delta \pvdash \tau_1 \loli \tau_2 : \star}$
      
      \caseFact{2} $\tau_1$ and $\tau_2$ \texttt{nf}
      
      \caseFact{3} ${\Psi ; \Theta ; \Delta \vdash \tau_1 : \star}$
      
      \caseFact{4} ${\Psi ; \Theta ; \Delta \vdash \tau_2 : \star}$
    }{
      $\Psi ; \Theta ; \Delta \pvdash \tau_1 \loli \tau_2 \subtynf \tau_1 \loli \tau_2 : \star \gens \Phi$ and $\Theta ; \Delta \vDash \Phi$
    }
    
    \caseText{By IH on (3) and (4)}
    
    \caseFact{5} $\Psi ; \Theta ; \Delta \pvdash \tau_1 \subtynf \tau_1 : \star \gens \Phi_1$
    
    \caseFact{6} $\Psi ; \Theta ; \Delta \pvdash \tau_2 \subtynf \tau_2 : \star \gens \Phi_2$
    
    \caseFact{7} $\Theta ; \Delta \vDash \Phi_1 \wedge \Phi_2$
    
    \caseText{By AS-Arr on (5) and (6)}
    
    \caseFact{8} $\Psi ; \Theta ; \Delta \pvdash \tau_1 \loli \tau_2 \subtynf \tau_1 \loli \tau_2 : \star \gens \Phi_1 \wedge \Phi_2$
    
    \caseText{Goal is immediate from (7), (8)}
  }
  
  
  \jcase{4}{K-Tensor}{
    \jgivengoal{
      \caseFact{1} ${\Psi ; \Theta ; \Delta \pvdash \tau_1 \otimes \tau_2 : \star}$
      
      \caseFact{2} $\tau_1$ and $\tau_2$ \texttt{nf}
      
      \caseFact{3} ${\Psi ; \Theta ; \Delta \vdash \tau_1 : \star}$
      
      \caseFact{4} ${\Psi ; \Theta ; \Delta \vdash \tau_2 : \star}$
    }{
      $\Psi ; \Theta ; \Delta \pvdash \tau_1 \otimes \tau_2 \subtynf \tau_1 \otimes \tau_2 : \star \gens \Phi$ and $\Theta ; \Delta \vDash \Phi$
    }
    
    \caseText{By IH on (3) and (4)}
    
    \caseFact{5} $\Psi ; \Theta ; \Delta \pvdash \tau_1 \subtynf \tau_1 : \star \gens \Phi_1$
    
    \caseFact{6} $\Psi ; \Theta ; \Delta \pvdash \tau_2 \subtynf \tau_2 : \star \gens \Phi_2$
    
    \caseFact{7} $\Theta ; \Delta \vDash \Phi_1 \wedge \Phi_2$
    
    \caseText{By AS-Tensor on (5) and (6)}
    
    \caseFact{8} $\Psi ; \Theta ; \Delta \pvdash \tau_1 \otimes \tau_2 \subtynf \tau_1 \otimes \tau_2 : \star \gens \Phi_1 \wedge \Phi_2$
    
    \caseText{Goal is immediate from (7), (8)}
  }
  
  \jcase{5}{K-With}{
    \jgivengoal{
      \caseFact{1} ${\Psi ; \Theta ; \Delta \pvdash \tau_1 \amp \tau_2 : \star}$
      
      \caseFact{2} $\tau_1$ and $\tau_2$ \texttt{nf}
      
      \caseFact{3} ${\Psi ; \Theta ; \Delta \vdash \tau_1 : \star}$
      
      \caseFact{4} ${\Psi ; \Theta ; \Delta \vdash \tau_2 : \star}$
    }{
      $\Psi ; \Theta ; \Delta \pvdash \tau_1 \amp \tau_2 \subtynf \tau_1 \amp \tau_2 : \star \gens \Phi$ and $\Theta ; \Delta \vDash \Phi$
    }
    
    \caseText{By IH on (3) and (4)}
    
    \caseFact{5} $\Psi ; \Theta ; \Delta \pvdash \tau_1 \subtynf \tau_1 : \star \gens \Phi_1$
    
    \caseFact{6} $\Psi ; \Theta ; \Delta \pvdash \tau_2 \subtynf \tau_2 : \star \gens \Phi_2$
    
    \caseFact{7} $\Theta ; \Delta \vDash \Phi_1 \wedge \Phi_2$
    
    \caseText{By AS-With on (5) and (6)}
    
    \caseFact{8} $\Psi ; \Theta ; \Delta \pvdash \tau_1 \amp \tau_2 \subtynf \tau_1 \amp \tau_2 : \star \gens \Phi_1 \wedge \Phi_2$
    
    \caseText{Goal is immediate from (7), (8)}
  }
  
  \jcase{6}{K-Sum}{
    \jgivengoal{
      \caseFact{1} ${\Psi ; \Theta ; \Delta \pvdash \tau_1 \oplus \tau_2 : \star}$
      
      \caseFact{2} $\tau_1$ and $\tau_2$ \texttt{nf}
      
      \caseFact{3} ${\Psi ; \Theta ; \Delta \vdash \tau_1 : \star}$
      
      \caseFact{4} ${\Psi ; \Theta ; \Delta \vdash \tau_2 : \star}$
    }{
      $\Psi ; \Theta ; \Delta \pvdash \tau_1 \oplus \tau_2 \subtynf \tau_1 \oplus \tau_2 : \star \gens \Phi$ and $\Theta ; \Delta \vDash \Phi$
    }
    
    \caseText{By IH on (3) and (4)}
    
    \caseFact{5} $\Psi ; \Theta ; \Delta \pvdash \tau_1 \subtynf \tau_1 : \star \gens \Phi_1$
    
    \caseFact{6} $\Psi ; \Theta ; \Delta \pvdash \tau_2 \subtynf \tau_2 : \star \gens \Phi_2$
    
    \caseFact{7} $\Theta ; \Delta \vDash \Phi_1 \wedge \Phi_2$
    
    \caseText{By AS-Sum on (5) and (6)}
    
    \caseFact{8} $\Psi ; \Theta ; \Delta \pvdash \tau_1 \oplus \tau_2 \subtynf \tau_1 \oplus \tau_2 : \star \gens \Phi_1 \wedge \Phi_2$
    
    \caseText{Goal is immediate from (7), (8)}
  }
  
  \jcase{7}{K-Bang}{
    \jgivengoal{
      \caseFact{1} $\Psi ; \Theta ; \Delta \pvdash !\tau : \star$
      
      \caseFact{2} $\tau \; \texttt{nf}$    
      
      \caseFact{3} $\Psi ; \Theta ; \Delta \vdash \tau : \star$
    }{
      $\Psi ; \Theta ; \Delta \pvdash !\tau \subtynf !\tau : \star \gens \Phi$ and $\Theta ; \Delta \vDash \Phi$
    }
    
    \caseText{By IH on (3)}
    
    \caseFact{4} $\Psi ; \Theta ; \Delta \pvdash \tau \subtynf \tau : \star \gens \Phi$
    
    \caseFact{5} $\Theta ; \Delta \vDash \Phi$
    
    \caseText{By AS-Bang on (4)}
    
    \caseFact{6} $\Psi ; \Theta ; \Delta \pvdash !\tau \subtynf !\tau : \star \gens \Phi$
  }
  
  \jcase{8}{K-IForall}{
    \jgivengoal{
      \caseFact{1} ${\Psi ; \Theta ; \Delta \vdash \forall i : S. \tau : \star}$
      
      \caseFact{2} $\tau \; \texttt{nf}$
      
      \caseFact{3} $\Psi ; \Theta, i : S ; \Delta \vdash \tau : \star$
    }{
      ${\Psi ; \Theta ; \Delta \vdash \forall i :S .\tau \subtynf \forall i : S. \tau: \star \gens \Phi}$ and $\Theta ; \Delta \vDash \Phi$
    }
    
    \caseText{By IH on (3)}
    
    \caseFact{4} $\Psi ; \Theta, i : S ; \Delta \pvdash \tau \subtynf \tau : \star \gens \Phi$
    
    \caseFact{5} $\Theta, i : S;\Delta \vDash \Phi$
    
    \caseText{By AS-IForall on (4)}
    
    \caseFact{6} $\Psi ; \Theta ; \Delta \pvdash \forall i : S.\tau \subty \forall i : S.\tau : \star \gens \forall i : S. \Phi$
    
    \caseText{Equivalently to (5)}
    
    \caseFact{7} $\Theta ; \Delta \vDash \forall i : S. \Phi$
  }
  
  \jcase{9}{K-IExists}{
    \jgivengoal{
      \caseFact{1} ${\Psi ; \Theta ; \Delta \vdash \exists i : S. \tau : \star}$
      
      \caseFact{2} $\tau \; \texttt{nf}$
      
      \caseFact{3} $\Psi ; \Theta, i : S ; \Delta \vdash \tau : \star$
    }{
      ${\Psi ; \Theta ; \Delta \vdash \exists i :S .\tau \subtynf \exists i : S. \tau: \star \gens \Phi}$ and $\Theta ; \Delta \vDash \Phi$
    }
    
    \caseText{By IH on (3)}
    
    \caseFact{4} $\Psi ; \Theta, i : S ; \Delta \pvdash \tau \subtynf \tau : \star \gens \Phi$
    
    \caseFact{5} $\Theta, i : S;\Delta \vDash \Phi$
    
    \caseText{By AS-IExists on (4)}
    
    \caseFact{6} $\Psi ; \Theta ; \Delta \pvdash \exists i : S.\tau \subty \exists i : S.\tau : \star \gens \forall i : S. \Phi$
    
    \caseText{Equivalently to (5)}
    
    \caseFact{7} $\Theta ; \Delta \vDash \forall i : S. \Phi$
  }
  
  \jcase{10}{K-TForall}{
    \jgivengoal{
      \caseFact{1} ${\Psi ; \Theta ; \Delta \vdash \forall \alpha : K. \tau : \star}$
      
      \caseFact{2} $\tau \; \texttt{nf}$
      
      \caseFact{3} $\Psi, \alpha : K ; \Theta ; \Delta \vdash \tau : \star$
    }{
      $\Psi ; \Theta ; \Delta \vdash \forall \alpha : K. \tau \subtynf \forall \alpha : K. \tau : \star \gens \Phi$ and $\Theta ; \Delta \vDash \Phi$
    }
    
    \caseText{By IH on (3)}
    
    \caseFact{4} $\Psi , \alpha : K ; \Theta ; \Delta \pvdash \tau \subtynf \tau : \star \gens \Phi$
    
    \caseFact{5} $\Theta ; \Delta \vDash \Phi$
    
    \caseText{By AS-TForall on (4)}
    
    \caseFact{6} $\Psi ; \Theta ; \Delta \vdash \forall \alpha : K. \tau \subtynf \forall \alpha : K. \tau : \star \gens \Phi$
  }
  
  \jcase{11}{K-List}{
    \jgivengoal{
      \caseFact{1} ${\Psi ; \Theta ; \Delta \pvdash L^I \tau : \star}$
      
      \caseFact{2} $\tau \; \texttt{nf}$
      
      \caseFact{3} $\Psi ; \Theta ; \Delta \vdash \tau : \star$
      
      \caseFact{4} $\Theta ; \Delta \vdash I : \N$
    }{
      $\Psi ; \Theta ; \Delta \vdash L^I\tau \subtynf : L^I\star \gens \Phi$ and $\Theta ;\Delta \vDash \Phi$
    }
    
    \caseText{By IH on (3)}
    
    \caseFact{5} $\Psi ; \Theta ; \Delta \pvdash \tau \subtynf \tau : \star \gens \Phi$
    
    \caseFact{6} $\Theta ; \Delta \vDash \Phi$
    
    \caseText{By AS-List on (5)}
    
    \caseFact{7} $\Psi ; \Theta ; \Delta \vdash L^I \tau \subtynf L^I \tau : \star \gens \Phi \wedge (I = I)$
    
    \caseText{By Theorem~\ref{thm:sound-compl} applied to (4), we may re-establish the presupposition for (7)}
    
    \caseFact{8} $\Psi ; \Theta ; \Delta \pvdash L^I \tau \subtynf L^I \tau : \star \gens \Phi \wedge (I = I)$
    
    \caseText{The goal is immediate from (6), (8), and $\Theta ; \Delta \vDash I = I$}
  }
  
  \jcase{12}{K-Conj}{
    \jgivengoal{
      \caseFact{1} $\Psi ; \Theta ; \Delta \pvdash \Phi' \amp \tau : \star$  
      
      \caseFact{2} $\tau \; \texttt{nf}$
      
      \caseFact{3} $\Psi ; \Theta ; \Delta \vdash \tau : \star$
      
      \caseFact{4} $\Theta ; \Delta \vdash \Phi' \texttt{ wf}$
    }{
      $\Psi ; \Theta ; \Delta \vdash \Phi' \amp \tau \subtynf \Phi' \amp \tau : \star \gens \Phi$ and $\Theta ; \Delta \vDash \Phi$    
    }
    
    \caseText{By IH on (3)}
    
    \caseFact{5} $\Psi ; \Theta ; \Delta \pvdash \tau \subtynf \tau : \star \gens \Phi$
    
    \caseFact{6} $\Theta ; \Delta \vDash \Phi$
    
    \caseText{By AS-Conj on (5)}
    
    \caseFact{7} $\Psi ; \Theta ; \Delta \vdash \Phi' \amp \tau \subtynf \Phi' \amp \tau : \star \gens \Phi \wedge (\Phi' \to \Phi')$
    
    \caseText{By Theorem~\ref{thm:constr-compl} on (4), we may re-establish the presupposition for (7)}
    
    \caseFact{8} $\Psi ; \Theta ; \Delta \vdash \Phi' \amp \tau \subtynf \Phi' \amp \tau : \star \gens \Phi \wedge (\Phi' \to \Phi')$
    
    \caseText{The goal follows from (6), (8), and $\Theta ; \Delta \vDash \Phi' \to \Phi'$}
  }
  
  \jcase{12}{K-Impl}{
    \jgivengoal{
      \caseFact{1} $\Psi ; \Theta ; \Delta \pvdash \Phi' \implies \tau : \star$  
      
      \caseFact{2} $\tau \; \texttt{nf}$
      
      \caseFact{3} $\Psi ; \Theta ; \Delta \vdash \tau : \star$
      
      \caseFact{4} $\Theta ; \Delta \vdash \Phi' \texttt{ wf}$
    }{
      $\Psi ; \Theta ; \Delta \vdash \Phi' \implies \tau \subtynf \Phi' \implies \tau : \star \gens \Phi$ and $\Theta ; \Delta \vDash \Phi$    
    }
    
    \caseText{By IH on (3)}
    
    \caseFact{5} $\Psi ; \Theta ; \Delta \pvdash \tau \subtynf \tau : \star \gens \Phi$
    
    \caseFact{6} $\Theta ; \Delta \vDash \Phi$
    
    \caseText{By AS-Impl on (5)}
    
    \caseFact{7} $\Psi ; \Theta ; \Delta \vdash \Phi' \implies \tau \subtynf \Phi' \implies \tau : \star \gens \Phi \wedge (\Phi' \to \Phi')$
    
    \caseText{By Theorem~\ref{thm:constr-compl} on (4), we may re-establish the presupposition for (7)}
    
    \caseFact{8} $\Psi ; \Theta ; \Delta \vdash \Phi' \implies \tau \subtynf \Phi' \implies \tau : \star \gens \Phi \wedge (\Phi' \to \Phi')$
    
    \caseText{The goal follows from (6), (8), and $\Theta ; \Delta \vDash \Phi' \to \Phi'$}
  }
  
  \jcase{13}{K-Monad}{
    \jgivengoal{
      \caseFact{1} $\Psi ; \Theta ; \Delta \vdash \M(I,\vec{p}) \tau : \star$
      
      \caseFact{2} $\tau \; \texttt{nf}$
      
      \caseFact{3} $\Psi ; \Theta ; \Delta \vdash \tau : \star$
      
      \caseFact{4} $\Theta ; \Delta \vdash I : \mathbb{N}$
      
      \caseFact{5} $\Theta ; \Delta \vdash \vec{p} : \potvec$
    }{
      $\Psi ; \Theta ; \Delta \vdash \M(I,\vec{p}) \tau \subtynf \M(I,\vec{p}) \tau : \star \gens \Phi$ and $\Theta ; \Delta \vDash \Phi$
    }
    
    \caseText{By IH on (3)}
    
    \caseFact{6} $\Psi ; \Theta ; \Delta \pvdash \tau \subtynf \tau : \star \gens \Phi$
    
    \caseFact{7} $\Theta ; \Delta \vDash \Phi$
    
    \caseText{By AS-Monad on (6), followed by Theorem~\ref{thm:sound-compl} on (4) and (5) to establish the presuppositions}
    
    \caseFact{8} $\Psi ; \Theta ; \Delta \pvdash \M(I,\vec{p}) \tau \subtynf \M(I,\vec{p}) \tau : \star \gens \Phi \wedge (I = I) \wedge (\vec{p} \leq \vec{p})$
    
    \caseText{Goal is immediate from (7) and (8)}
  }
  
  \jcase{14}{K-Pot}{
    \jgivengoal{
      \caseFact{1} $\Psi ; \Theta ; \Delta \vdash [I|\vec{p}] \tau : \star$
      
      \caseFact{2} $\tau \; \texttt{nf}$
      
      \caseFact{3} $\Psi ; \Theta ; \Delta \vdash \tau : \star$
      
      \caseFact{4} $\Theta ; \Delta \vdash I : \mathbb{N}$
      
      \caseFact{5} $\Theta ; \Delta \vdash \vec{p} : \potvec$
    }{
      $\Psi ; \Theta ; \Delta \vdash [I|\vec{p}] \tau \subtynf [I|\vec{p}] \tau : \star \gens \Phi$ and $\Theta ; \Delta \vDash \Phi$
    }
    
    \caseText{By IH on (3)}
    
    \caseFact{6} $\Psi ; \Theta ; \Delta \pvdash \tau \subtynf \tau : \star \gens \Phi$
    
    \caseFact{7} $\Theta ; \Delta \vDash \Phi$
    
    \caseText{By AS-Pot on (6), followed by Theorem~\ref{thm:sound-compl} on (4) and (5) to establish the presuppositions}
    
    \caseFact{8} $\Psi ; \Theta ; \Delta \pvdash [I|\vec{p}] \tau \subtynf [I|\vec{p}] \tau : \star \gens \Phi \wedge (I = I) \wedge (\vec{p} \leq \vec{p})$
    
    \caseText{Goal is immediate from (7) and (8)}
  }
  
  \jcase{15}{K-ConstPot}{
    \jgivengoal{
      \caseFact{1} $\Psi ; \Theta ; \Delta \pvdash [I] \; \tau : \star$
      
      \caseFact{2} $\tau \; \texttt{nf}$
      
      \caseFact{3} $\Psi ; \Theta ; \Delta \vdash \tau : \star$
      
      \caseFact{4} $\Theta ; \Delta \vdash I : \mathbb{R}^+$
    }{
      $\Psi ; \Theta ; \Delta \vdash [I]\tau \subtynf [I]\tau: \star \gens \Phi$ and $\Theta ; \Delta \vDash \Phi$
    }
    
    \caseText{By IH on (3)}
    
    \caseFact{5} $\Psi ; \Theta ; \Delta \pvdash \tau \subtynf \tau : \star \gens \Phi$
    
    \caseFact{6} $\Theta ; \Delta \vDash \Phi$
    
    \caseText{By AS-ConstPot on (5) followed by Theorem~\ref{thm:sort-compl} on (4) to re-establish the presupposition}
    
    \caseFact{7} $\Psi ; \Theta ; \Delta \vdash [I] \; \tau \subtynf [I] \; \tau : \star \gens \Phi \wedge (I \leq I)$
    
    \caseText{Goal is immediate by (7), (6), and $\Theta ; \Delta \vDash I \leq I$}
     
  }
  
  \jcase{16}{K-FamLam}{
    \jgivengoal{
      \caseFact{1} $\Psi ; \Theta ; \Delta \pvdash \lambda i : S. \tau : S \to K$
      
      \caseFact{2} $\tau \; \texttt{nf}$
      
      \caseFact{3} $\Psi ; \Theta, i : S ; \Delta \vdash \tau : K$
    }{
      $\Psi ; \Theta ; \Delta \vdash \lambda i : S. \tau \subtynf \tau : S \to K \gens \Phi$ and $\Theta ; \Delta \vDash \Phi$
    }
    
    \caseText{By IH on (3)}
    
    \caseFact{4} $\Psi ; \Theta, i : S; \Delta \pvdash \tau \subtynf \tau : K \gens \Phi$
    
    \caseFact{5} $\Theta , i : S; \Delta \vDash \Phi$
    
    \caseText{By AS-FamLam on (4)}
    
    \caseFact{6} $\Psi ; \Theta ; \Delta \pvdash \lambda i : S. \tau \subtynf \lambda i : S \tau : S \to K \gens \forall i : S. \Phi$
    
    \caseText{Equivalently to (5)}
    
    \caseFact{7} $\Theta ; \Delta \vDash \forall i : S. \Phi$
    
  }
  
  \jcase{17}{K-FamApp}{Immediate by Theorem~\ref{thm:subtyne-refl}}
  
}

\iffalse
 
  
  \item[(K-FamLam)] Suppose  from $\Psi ; \Theta, i : S ; \Delta \vdash \tau : K$. By IH,
  $\Psi ; \Theta, i : S ; \Delta \vdash \tau \subtynf \tau : K \gens \Phi$ with $\Theta, i : S ; \Delta \vDash \Phi$. By AS-FamLam, $\Psi ; \Theta ; \Delta \vdash \lambda i : S. \tau \subtynf \tau : S \to K \gens \forall i : S. \Phi$. But $\Theta, i : S ; \Delta \vDash \Phi$ and so $\Theta ; \Delta \vDash \forall i : S. \Phi$, as required.
  \item[(K-FamApp)] Done by Theorem~\ref{thm:subtyne-refl}
\fi

\subtynftrans*
\jtheorem{Proof of \autoref{thm:subtynf-trans}}{
  \jgivengoal{
    \caseFact{1} $\Psi ; \Theta ; \Delta \pvdash \tau_1 \subtynf \tau_2 : K \gens \Phi_1$
    
    \caseFact{2} $\Psi ; \Theta ; \Delta \pvdash \tau_2 \subtynf \tau_3 : K \gens \Phi_2$
    
    \caseFact{3} $\Theta ; \Delta \vDash \Phi_1 \wedge \Phi_2$
  }{
    $\Psi ; \Theta ; \Delta \pvdash \tau_1 \subtynf \tau_3 : K \gens \Phi$ such that $\Theta ; \Delta \vDash \Phi$  
  }
  
  \caseText{By strong induction on the sum of the sizes of (1) and (2). We note that for a given choice of final rule for (1), the final rule for (2) must be the same, by inspection of the rules generating $\subtynf$. For this reason, we present the proof as a case analysis over the rules for $\subtynf$.
  }
  
  \jcase{1}{AS-Unit}{Immediate.}
  \jcase{2}{AS-Var}{Immediate.}
  
  \jcase{3}{AS-Arr}{
    \jgivengoal{
      \caseFact{1} $\Psi ; \Theta ; \Delta \pvdash \tau_1 \loli \tau_1' \subtynf \tau_2 \loli \tau_2' : \star \gens \Phi_1 \wedge \Phi_2$
      
      \caseFact{2} $\Psi ; \Theta ; \Delta \pvdash \tau_2 \loli \tau_2' \subtynf \tau_3 \loli \tau_3' : \star \gens \Phi_3 \wedge \Phi_4$
      
      \caseFact{3} $\Theta ; \Delta \vdash \bigwedge_{i=1}^4 \Phi_i$
      
      \caseFact{4} $\Psi ; \Theta ; \Delta \vdash \tau_2 \subtynf \tau_1 : \star \gens \Phi_1$
      
      \caseFact{5} $\Psi ; \Theta ; \Delta \vdash \tau_1' \subtynf \tau_2' : \star \gens \Phi_2$
      
      \caseFact{6} $\Psi ; \Theta ; \Delta \vdash \tau_3 \subtynf \tau_2 : \star \gens \Phi_3$
      
      \caseFact{7} $\Psi ; \Theta ; \Delta \vdash \tau_2' \subtynf \tau_3' : \star \gens \Phi_4$
    }{
      $\Psi ; \Theta ; \Delta \pvdash \tau_1 \loli \tau_1' \subtynf \tau_3 \loli \tau_3' : \star \gens \Phi$ and $\Theta ; \Delta \vDash \Phi$
    }
    
    \caseText{By IH on (4) and (6)}
    
    \caseFact{8} $\Psi ; \Theta ; \Delta \pvdash \tau_3 \subtynf \tau_1 : \star \gens \Phi_1'$
    
    \caseFact{9} $\Theta ; \Delta \vDash \Phi_1'$
    
    \caseText{By IH on (5) and (7)}
    
    \caseText{10} $\Psi ; \Theta ; \Delta \pvdash \tau_1' \subtynf \tau_3' : \star \gens \Phi_2'$
    
    \caseFact{11} $\Theta ; \Delta \vDash \Phi_2'$
    
    \caseText{By AS-Arr on (8) and (10)}
    
    \caseFact{12} $\Psi ; \Theta ; \Delta \pvdash \tau_1 \loli \tau_1' \subtynf \tau_3 \loli \tau_3' : \star \gens \Phi_1' \wedge \Phi_2'$
    
    \caseText{The Goal follows by (9), (11), and (12)}
  }
  
  \jcase{4}{AS-Tensor}{
    \jgivengoal{
      \caseFact{1} $\Psi ; \Theta ; \Delta \pvdash \tau_1 \otimes \tau_1' \subtynf \tau_2 \otimes \tau_2' : \star \gens \Phi_1 \wedge \Phi_2$
      
      \caseFact{2} $\Psi ; \Theta ; \Delta \pvdash \tau_2 \otimes \tau_2' \subtynf \tau_3 \otimes \tau_3' : \star \gens \Phi_3 \wedge \Phi_4$
      
      \caseFact{3} $\Theta ; \Delta \vdash \bigwedge_{i=1}^4 \Phi_i$
      
      \caseFact{4} $\Psi ; \Theta ; \Delta \vdash \tau_1 \subtynf \tau_2 : \star \gens \Phi_1$
      
      \caseFact{5} $\Psi ; \Theta ; \Delta \vdash \tau_1' \subtynf \tau_2' : \star \gens \Phi_2$
      
      \caseFact{6} $\Psi ; \Theta ; \Delta \vdash \tau_2 \subtynf \tau_3 : \star \gens \Phi_3$
      
      \caseFact{7} $\Psi ; \Theta ; \Delta \vdash \tau_2' \subtynf \tau_3' : \star \gens \Phi_4$
    }{
      $\Psi ; \Theta ; \Delta \pvdash \tau_1 \otimes \tau_1' \subtynf \tau_3 \otimes \tau_3' : \star \gens \Phi$ and $\Theta ; \Delta \vDash \Phi$
    }
    
    \caseText{By IH on (4) and (6)}
    
    \caseFact{8} $\Psi ; \Theta ; \Delta \pvdash \tau_1 \subtynf \tau_3 : \star \gens \Phi_1'$
    
    \caseFact{9} $\Theta ; \Delta \vDash \Phi_1'$
    
    \caseText{By IH on (5) and (7)}
    
    \caseText{10} $\Psi ; \Theta ; \Delta \pvdash \tau_1' \subtynf \tau_3' : \star \gens \Phi_2'$
    
    \caseFact{11} $\Theta ; \Delta \vDash \Phi_2'$
    
    \caseText{By AS-Tensor on (8) and (10)}
    
    \caseFact{12} $\Psi ; \Theta ; \Delta \pvdash \tau_1 \otimes \tau_1' \subtynf \tau_3 \otimes \tau_3' : \star \gens \Phi_1' \wedge \Phi_2'$
    
    \caseText{The Goal follows by (9), (11), and (12)}
  }
  
  \jcase{5}{AS-With}{
    \jgivengoal{
      \caseFact{1} $\Psi ; \Theta ; \Delta \pvdash \tau_1 \amp \tau_1' \subtynf \tau_2 \amp \tau_2' : \star \gens \Phi_1 \wedge \Phi_2$
      
      \caseFact{2} $\Psi ; \Theta ; \Delta \pvdash \tau_2 \amp \tau_2' \subtynf \tau_3 \amp \tau_3' : \star \gens \Phi_3 \wedge \Phi_4$
      
      \caseFact{3} $\Theta ; \Delta \vdash \bigwedge_{i=1}^4 \Phi_i$
      
      \caseFact{4} $\Psi ; \Theta ; \Delta \vdash \tau_1 \subtynf \tau_2 : \star \gens \Phi_1$
      
      \caseFact{5} $\Psi ; \Theta ; \Delta \vdash \tau_1' \subtynf \tau_2' : \star \gens \Phi_2$
      
      \caseFact{6} $\Psi ; \Theta ; \Delta \vdash \tau_2 \subtynf \tau_3 : \star \gens \Phi_3$
      
      \caseFact{7} $\Psi ; \Theta ; \Delta \vdash \tau_2' \subtynf \tau_3' : \star \gens \Phi_4$
    }{
      $\Psi ; \Theta ; \Delta \pvdash \tau_1 \amp \tau_1' \subtynf \tau_3 \amp \tau_3' : \star \gens \Phi$ and $\Theta ; \Delta \vDash \Phi$
    }
    
    \caseText{By IH on (4) and (6)}
    
    \caseFact{8} $\Psi ; \Theta ; \Delta \pvdash \tau_1 \subtynf \tau_3 : \star \gens \Phi_1'$
    
    \caseFact{9} $\Theta ; \Delta \vDash \Phi_1'$
    
    \caseText{By IH on (5) and (7)}
    
    \caseText{10} $\Psi ; \Theta ; \Delta \pvdash \tau_1' \subtynf \tau_3' : \star \gens \Phi_2'$
    
    \caseFact{11} $\Theta ; \Delta \vDash \Phi_2'$
    
    \caseText{By AS-With on (8) and (10)}
    
    \caseFact{12} $\Psi ; \Theta ; \Delta \pvdash \tau_1 \amp \tau_1' \subtynf \tau_3 \amp \tau_3' : \star \gens \Phi_1' \wedge \Phi_2'$
    
    \caseText{The Goal follows by (9), (11), and (12)}
  }
  
  \jcase{6}{AS-Sum}{
    \jgivengoal{
      \caseFact{1} $\Psi ; \Theta ; \Delta \pvdash \tau_1 \oplus \tau_1' \subtynf \tau_2 \oplus \tau_2' : \star \gens \Phi_1 \wedge \Phi_2$
      
      \caseFact{2} $\Psi ; \Theta ; \Delta \pvdash \tau_2 \oplus \tau_2' \subtynf \tau_3 \oplus \tau_3' : \star \gens \Phi_3 \wedge \Phi_4$
      
      \caseFact{3} $\Theta ; \Delta \vdash \bigwedge_{i=1}^4 \Phi_i$
      
      \caseFact{4} $\Psi ; \Theta ; \Delta \vdash \tau_1 \subtynf \tau_2 : \star \gens \Phi_1$
      
      \caseFact{5} $\Psi ; \Theta ; \Delta \vdash \tau_1' \subtynf \tau_2' : \star \gens \Phi_2$
      
      \caseFact{6} $\Psi ; \Theta ; \Delta \vdash \tau_2 \subtynf \tau_3 : \star \gens \Phi_3$
      
      \caseFact{7} $\Psi ; \Theta ; \Delta \vdash \tau_2' \subtynf \tau_3' : \star \gens \Phi_4$
    }{
      $\Psi ; \Theta ; \Delta \pvdash \tau_1 \oplus \tau_1' \subtynf \tau_3 \oplus \tau_3' : \star \gens \Phi$ and $\Theta ; \Delta \vDash \Phi$
    }
    
    \caseText{By IH on (4) and (6)}
    
    \caseFact{8} $\Psi ; \Theta ; \Delta \pvdash \tau_1 \subtynf \tau_3 : \star \gens \Phi_1'$
    
    \caseFact{9} $\Theta ; \Delta \vDash \Phi_1'$
    
    \caseText{By IH on (5) and (7)}
    
    \caseText{10} $\Psi ; \Theta ; \Delta \pvdash \tau_1' \subtynf \tau_3' : \star \gens \Phi_2'$
    
    \caseFact{11} $\Theta ; \Delta \vDash \Phi_2'$
    
    \caseText{By AS-Sum on (8) and (10)}
    
    \caseFact{12} $\Psi ; \Theta ; \Delta \pvdash \tau_1 \oplus \tau_1' \subtynf \tau_3 \oplus \tau_3' : \star \gens \Phi_1' \wedge \Phi_2'$
    
    \caseText{The Goal follows by (9), (11), and (12)}
  }
  
  \jcase{7}{AS-Bang}{
    \jgivengoal{
      \caseFact{1} $\Psi ; \Theta ; \Delta \pvdash !\tau_1 \subtynf !\tau_2 : \star \gens \Phi_1$
      
      \caseFact{2} $\Psi ; \Theta ; \Delta \pvdash !\tau_2 \subtynf !\tau_3 : \star \gens \Phi_2$
      
      \caseFact{3} $\Theta ; \Delta \vDash \Phi_1 \wedge \Phi_2$
      
      \caseFact{4} $\Psi ; \Theta ; \Delta \vdash \tau_1 \subtynf \tau_2 : \star \gens \Phi_1$
      
      \caseFact{5}  $\Psi ; \Theta ; \Delta \vdash \tau_2 \subtynf \tau_3 : \star \gens \Phi_2$
    }{
      $\Psi ; \Theta ; \Delta \pvdash !\tau_1 \subtynf !\tau_3 : \star \gens \Phi$ and $\Theta ; \Delta \vDash \Phi$
    }
    
    \caseText{By IH on (4) and (5)}
    
    \caseFact{6} $\Psi ; \Theta ; \Delta \pvdash \tau_1 \subtynf \tau_3 : \star \gens Phi$
    
    \caseFact{7} $\Theta ; \Delta \vDash \Phi$
    
    \caseText{By AS-Bang on (6)}
    
    \caseFact{8} $\Psi ; \Theta ; \Delta \pvdash !\tau_1 \subtynf !\tau_3 : \star \gens Phi$
  }
  
  \jcase{8}{AS-IForall}{
    \jgivengoal{
      \caseFact{1} $\Psi ; \Theta ; \Delta \pvdash \forall i : S. \tau_1 \subtynf \forall i : S. \tau_2 : \star \gens \forall i : S. \Phi_1$    
      
      \caseFact{2} $\Psi ; \Theta ; \Delta \pvdash \forall i : S. \tau_2 \subtynf \forall i : S. \tau_3 : \star \gens \forall i : S. \Phi_2$
      
      \caseFact{3} $\Theta ; \Delta \vDash \forall i : S. \Phi_1 \wedge \forall i : S. \Phi_2$
      
      \caseFact{4} $\Psi ; \Theta, i : S ; \Delta \vdash \tau_1 \subtynf \tau_2 : \star \gens \Phi_1$
      
      \caseFact{5} $\Psi ; \Theta, i : S ; \Delta \vdash \tau_2 \subtynf \tau_3 : \star \gens \Phi_3$
    }{
      $\Psi ; \Theta ; \Delta \pvdash \forall i : S. \tau_1 \subtynf \forall i : S. \tau_3 : \star \gens \Phi$ and $\Theta ; \Delta \vDash \Phi$
    }
    
    \caseText{Equivalently to (3), $\Theta, i : S; \Delta \vDash \Phi_1$ and $\Theta, i : S; \Delta \vDash \Phi_2$, and so by IH on (4),(5)}
    
    \caseFact{6} $\Psi ; \Theta, i :S ;\Delta \pvdash \tau_1 \subtynf \tau_3 : \star \gens \Phi$
    
    \caseFact{7} $\Theta, i : S: \Delta \vDash \Phi$
    
    \caseText{By AS-IForall on (6)}
    
    \caseFact{8} $\Psi ; \Theta ; \Delta \pvdash \forall i : S.\tau_1 \subtynf \forall i : S.\tau_3 : \star \gens \forall i : S. \Phi$
    
    \caseText{Equivalently to (7)}
    
    \caseFact{9} $\Theta ; \Delta \vDash \forall i : S. \Phi$
    
    \caseText{The goal follows by (8) and (9)}
  }
  
  \jcase{9}{AS-IExists}{
    \jgivengoal{
      \caseFact{1} $\Psi ; \Theta ; \Delta \pvdash \exists i : S. \tau_1 \subtynf \exists i : S. \tau_2 : \star \gens \forall i : S. \Phi_1$    
      
      \caseFact{2} $\Psi ; \Theta ; \Delta \pvdash \exists i : S. \tau_2 \subtynf \exists i : S. \tau_3 : \star \gens \forall i : S. \Phi_2$
      
      \caseFact{3} $\Theta ; \Delta \vDash \forall i : S. \Phi_1 \wedge \forall i : S. \Phi_2$
      
      \caseFact{4} $\Psi ; \Theta, i : S ; \Delta \vdash \tau_1 \subtynf \tau_2 : \star \gens \Phi_1$
      
      \caseFact{5} $\Psi ; \Theta, i : S ; \Delta \vdash \tau_2 \subtynf \tau_3 : \star \gens \Phi_3$
    }{
      $\Psi ; \Theta ; \Delta \pvdash \exists i : S. \tau_1 \subtynf \exists i : S. \tau_3 : \star \gens \Phi$ and $\Theta ; \Delta \vDash \Phi$
    }
    
    \caseText{Equivalently to (3), $\Theta, i : S; \Delta \vDash \Phi_1$ and $\Theta, i : S; \Delta \vDash \Phi_2$, and so by IH on (4),(5)}
    
    \caseFact{6} $\Psi ; \Theta, i :S ;\Delta \pvdash \tau_1 \subtynf \tau_3 : \star \gens \Phi$
    
    \caseFact{7} $\Theta, i : S: \Delta \vDash \Phi$
    
    \caseText{By AS-IExists on (6)}
    
    \caseFact{8} $\Psi ; \Theta ; \Delta \pvdash \exists i : S.\tau_1 \subtynf \exists i : S.\tau_3 : \star \gens \forall i : S. \Phi$
    
    \caseText{Equivalently to (7)}
    
    \caseFact{9} $\Theta ; \Delta \vDash \forall i : S. \Phi$
    
    \caseText{The goal follows by (8) and (9)}
  }
  
  \jcase{10}{AS-TForall}{
    \jgivengoal{
      \caseFact{1} $\Psi ; \Theta ; \Delta \pvdash \forall \alpha : K. \tau_1 \subtynf \forall \alpha : K. \tau_2 : \star \gens \Phi_1$    
      
      \caseFact{2} $\Psi ; \Theta ; \Delta \pvdash \forall \alpha : K. \tau_2 \subtynf \forall \alpha : K. \tau_3 : \star \gens \Phi_2$
      
      \caseFact{3} $\Theta ; \Delta \vDash \Phi_1 \wedge \Phi_2$
      
      \caseFact{4} $\Psi, \alpha : K ; \Theta ; \Delta \vdash \tau_1 \subtynf \tau_2 : \star \gens \Phi_1$
      
      \caseFact{5} $\Psi, \alpha : K ; \Theta ; \Delta \vdash \tau_2 \subtynf \tau_3 : \star \gens \Phi_2$
    }{
       $\Psi ; \Theta ; \Delta \pvdash \forall \alpha : K. \tau_1 \subtynf \forall \alpha : K. \tau_3 : \star \gens \Phi$     and $\Theta ; \Delta \vDash \Phi$
    }
    
    \caseText{By IH on (4),(5)}    
    
    \caseFact{6} $\Psi , \alpha : K ; \Theta ; \Delta \pvdash \tau_1 \subtynf \tau_3 : \star \gens \Phi$
    
    \caseFact{7} $\Theta ; \Delta \vDash \Phi$
    
    \caseText{By AS-TForall on (6)}
    
    \caseFact{8} $\Psi ; \Theta ; \Delta \pvdash \forall \alpha : K.\tau_1 \subtynf \forall \alpha : K.\tau_3 : \star \gens \Phi$
  }
  
  \jcase{11}{AS-List}{
    \jgivengoal{
      \caseFact{1} $\Psi ; \Theta ; \Delta \pvdash L^I \tau_1 \subtynf L^J \tau_2 : \star \gens \Phi_1 \wedge (I = J)$
      
      \caseFact{2} $\Psi ; \Theta ; \Delta \pvdash L^J \tau_2 \subtynf L^K \tau_3 : \star \gens \Phi_2 \wedge (J = K)$
      
      \caseFact{3} $\Theta ; \Delta \vDash \Phi_1 \wedge \Phi_2 \wedge (I = J) \wedge (J = K)$
      
      \caseFact{4} $\Psi ; \Theta ; \Delta \pvdash \tau_1 \subtynf \tau_2 : \star \gens \Phi_1$
      
      \caseFact{5} $\Psi ; \Theta ; \Delta \pvdash \tau_2 \subtynf \tau_3 : \star \gens \Phi_2$
    }{
      $\Psi ; \Theta ; \Delta \pvdash L^I \tau_1 \subtynf L^K \tau_3 : \star \gens \Phi$ and $\Theta ; \Delta \vDash \Phi$
    }
    
    \caseText{By IH on (4), (5)}
    
    \caseFact{6} $\Psi ; \Theta ; \Delta \pvdash \tau_1 \subtynf \tau_3 : \star \gens \Phi$
    
    \caseFact{7} $\Theta ; \Delta \vDash \Phi$
    
    \caseText{By AS-List on (7)}
    
    \caseFact{8} $\Psi ; \Theta ; \Delta \pvdash \tau_1^I \subtynf \tau_3^K : \star \gens \Phi \wedge (I = K)$
    
    \caseText{By (3)}
    
    \caseFact{9} $\Theta ; \Delta \vDash I = K$
    
    \caseText{The goal follows by (7), (8), (9)}
  }
  
  \jcase{12}{AS-Impl}{
  
    \jgivengoal{
      \caseFact{1} $\Psi ; \Theta ; \Delta \pvdash \Phi_1 \implies \tau_1 \subtynf \Phi_2 \implies \tau_2 : \star \gens \Phi_1' \wedge (\Phi_2 \to \Phi_1)$
      
      \caseFact{2} $\Psi ; \Theta ; \Delta \pvdash \Phi_2 \implies \tau_2 \subtynf \Phi_3 \implies \tau_3 : \star \gens \Phi_2' \wedge (\Phi_3 \to \Phi_2)$
      
      \caseFact{3} $\Theta ; \Delta \vDash \Phi_1' \wedge \Phi_2' \wedge (Phi_3 \to \Phi_2) \wedge (Phi_2 \to \Phi_1)$
      
      \caseFact{4} $\Psi ; \Theta ; \Delta \vdash \tau_1 \subtynf \tau_2 : \star \gens \Phi_1'$
      
      \caseFact{5} $\Psi ; \Theta ; \Delta \vdash \tau_2 \subtynf \tau_3 : \star \gens \Phi_2'$
    }{
      $\Psi ; \Theta ; \Delta \vdash \Phi_1 \implies \tau_1 \subtynf \Phi_3 \implies \tau_3 : \star \gens \Phi$ and $\Theta ; \Delta \vDash \Phi$
    }
    
    \caseText{By IH on (4), (5)}
    
    \caseFact{6} $\Psi ; \Theta ; \Delta \pvdash \tau_1 \subtynf \tau_3 : \star \gens \Phi$
    
    \caseFact{7} $\Theta ; \Delta \vDash \Phi$
    
    \caseText{By AS-Impl on (6)}
    
    \caseFact{8} $\Psi ; \Theta ; \Delta \pvdash \Phi_1 \implies \tau_1 \subtynf \Phi_3 \implies \tau_3 : \star \gens \Phi \wedge (\Phi_3 \to \Phi_1)$
    
    \caseText{By (3)}
    
    \caseFact{9} $\Theta ; \Delta \vDash \Phi_3 \to \Phi_1$
    
    \caseText{The goal follows by (7), (8), (9)}
  }
  
  \jcase{13}{AS-Conj}{
  
    \jgivengoal{
      \caseFact{1} $\Psi ; \Theta ; \Delta \pvdash \Phi_1 \amp \tau_1 \subtynf \Phi_2 \amp \tau_2 : \star \gens \Phi_1' \wedge (\Phi_1 \to \Phi_2)$
      
      \caseFact{2} $\Psi ; \Theta ; \Delta \pvdash \Phi_2 \amp \tau_2 \subtynf \Phi_3 \amp \tau_3 : \star \gens \Phi_2' \wedge (\Phi_2 \to \Phi_3)$
      
      \caseFact{3} $\Theta ; \Delta \vDash \Phi_1' \wedge \Phi_2' \wedge (Phi_1 \to \Phi_3) \wedge (Phi_1 \to \Phi_2)$
      
      \caseFact{4} $\Psi ; \Theta ; \Delta \vdash \tau_1 \subtynf \tau_2 : \star \gens \Phi_1'$
      
      \caseFact{5} $\Psi ; \Theta ; \Delta \vdash \tau_2 \subtynf \tau_3 : \star \gens \Phi_2'$
    }{
      $\Psi ; \Theta ; \Delta \vdash \Phi_1 \amp \tau_1 \subtynf \Phi_3 \amp \tau_3 : \star \gens \Phi$ and $\Theta ; \Delta \vDash \Phi$
    }
    
    \caseText{By IH on (4), (5)}
    
    \caseFact{6} $\Psi ; \Theta ; \Delta \pvdash \tau_1 \subtynf \tau_3 : \star \gens \Phi$
    
    \caseFact{7} $\Theta ; \Delta \vDash \Phi$
    
    \caseText{By AS-Conj on (6)}
    
    \caseFact{8} $\Psi ; \Theta ; \Delta \pvdash \Phi_1 \amp \tau_1 \subtynf \Phi_3 \amp \tau_3 : \star \gens \Phi \wedge (\Phi_1 \to \Phi_3)$
    
    \caseText{By (3)}
    
    \caseFact{9} $\Theta ; \Delta \vDash \Phi_1 \to \Phi_3$
    
    \caseText{The goal follows by (7), (8), (9)}
  }
%{\Psi ; \Theta ; \Delta \vdash \tau_1 \subtynf \tau_2 : \star \gens \Phi}{\Psi ; \Theta ; \Delta \vdash \M(I,\vec{q}) \tau_1 \subtynf \M(J,\vec{p}) \tau_2 : \star \gens (I = J) \wedge (\vec{q} \leq \vec{p}) \wedge \Phi}
  
  \jcase{14}{AS-Monad}{
    \jgivengoal{
      \caseFact{1} $\Psi ; \Theta ; \Delta \pvdash \M(I,\vec{q}) \tau_1 \subtynf \M(J,\vec{p}) \tau_2 : \star \gens (I = J) \wedge (\vec{q} \leq \vec{p}) \wedge \Phi_1$
      
      \caseFact{2} $\Psi ; \Theta ; \Delta \pvdash \M(J,\vec{p}) \tau_2 \subtynf \M(K,\vec{l}) \tau_3 : \star \gens (J = K) \wedge (\vec{p} \leq \vec{l}) \wedge \Phi_2$
      
      \caseFact{3} $\Theta ; \Delta \vDash (I = J) \wedge (J = K) \wedge (\vec{q} \leq \vec{p}) \wedge (\vec{p} \leq \vec{l}) \wedge \Phi_1 \wedge \Phi_2$
      
      \caseFact{4} $\Psi ; \Theta ; \Delta \vdash \tau_1 \subtynf \tau_2 : \star \gens \Phi_1$
      
      \caseFact{5} $\Psi ; \Theta ; \Delta \vdash \tau_2 \subtynf \tau_3 : \star \gens \Phi_2$
    }{
      $\Psi ; \Theta ; \Delta \pvdash \M(I,\vec{q}) \tau_1 \subtynf \M(K,\vec{l}) \tau_3 : \star \gens \Phi$ and $\Theta ; \Delta \vDash \Phi$
    }
    
    \caseText{By IH on (4), (5)}
    
    \caseFact{7} $\Psi ; \Theta ; \Delta \pvdash \tau_1 \subtynf \tau_3 : \star \gens \Phi$
    
    \caseFact{8} $\Theta ; \Delta \vDash \Phi$
    
    \caseText{By (3)}
    
    \caseFact{9} $\Theta ; \Delta \vDash (I = K) \wedge (\vec{q} \leq \vec{l})$
    
    \caseText{By AS-Monad on (7)}
    
    \caseFact{10} $\Psi ; \Theta ; \Delta \pvdash \M(I,\vec{q}) \tau_1 \subtynf \M(K,\vec{p}) \tau_3 : \star \gens (I = K) \wedge (\vec{q} \leq \vec{l}) \wedge \Phi$
    
    \caseText{Goal follows by (8), (9), (10)}
  }
  
  \jcase{15}{AS-Pot}{
    \jgivengoal{
      \caseFact{1} $\Psi ; \Theta ; \Delta \pvdash [I|\vec{q}] \tau_1 \subtynf [J|\vec{p}] \tau_2 : \star \gens (I = J) \wedge (\vec{q} \geq \vec{p}) \wedge \Phi_1$
      
      \caseFact{2} $\Psi ; \Theta ; \Delta \pvdash [J|\vec{p}] \tau_2 \subtynf [K|\vec{l}] \tau_3 : \star \gens (J = K) \wedge (\vec{p} \geq \vec{l}) \wedge \Phi_2$
      
      \caseFact{3} $\Theta ; \Delta \vDash (I = J) \wedge (J = K) \wedge (\vec{q} \geq \vec{p}) \wedge (\vec{p} \geq \vec{l}) \wedge \Phi_1 \wedge \Phi_2$
      
      \caseFact{4} $\Psi ; \Theta ; \Delta \vdash \tau_1 \subtynf \tau_2 : \star \gens \Phi_1$
      
      \caseFact{5} $\Psi ; \Theta ; \Delta \vdash \tau_2 \subtynf \tau_3 : \star \gens \Phi_2$
    }{
      $\Psi ; \Theta ; \Delta \pvdash [I|\vec{q}] \tau_1 \subtynf [K|\vec{l}] \tau_3 : \star \gens \Phi$ and $\Theta ; \Delta \vDash \Phi$
    }
    
    \caseText{By IH on (4), (5)}
    
    \caseFact{7} $\Psi ; \Theta ; \Delta \pvdash \tau_1 \subtynf \tau_3 : \star \gens \Phi$
    
    \caseFact{8} $\Theta ; \Delta \vDash \Phi$
    
    \caseText{By (3)}
    
    \caseFact{9} $\Theta ; \Delta \vDash (I = K) \wedge (\vec{q} \geq \vec{l})$
    
    \caseText{By AS-Pot on (7)}
    
    \caseFact{10} $\Psi ; \Theta ; \Delta \pvdash [I|\vec{q}] \tau_1 \subtynf [K|\vec{p}] \tau_3 : \star \gens (I = K) \wedge (\vec{q} \geq \vec{l}) \wedge \Phi$
    
    \caseText{Goal follows by (8), (9), (10)}
  }
  
  \jcase{16}{AS-ConstPot}{
    \jgivengoal{
      \caseFact{1} $\Psi ; \Theta ; \Delta \pvdash [I] \tau_1 \subtynf [J] \tau_2 : \star \gens (I \leq J) \wedge \Phi_1$
      
      \caseFact{2} $\Psi ; \Theta ; \Delta \pvdash [J] \tau_2 \subtynf [K] \tau_3 : \star \gens (J \leq K) \wedge \Phi_2$
      
      \caseFact{3} $\Theta ; \Delta \vDash (I \leq J) \wedge (J \leq K) \wedge \Phi_1 \wedge \Phi_2$
      
      \caseFact{4} $\Psi ; \Theta ; \Delta \vdash \tau_1 \subtynf \tau_2 : \star \gens \Phi_1$
      
      \caseFact{5} $\Psi ; \Theta ; \Delta \vdash \tau_2 \subtynf \tau_3 : \star \gens \Phi_2$
    }{
      $\Psi ; \Theta ; \Delta \pvdash [I] \tau_1 \subtynf [K] \tau_3 : \star \gens \Phi$ and $\Theta ; \Delta \vDash \Phi$
    }
    
    \caseText{By IH on (4), (5)}
    
    \caseFact{7} $\Psi ; \Theta ; \Delta \pvdash \tau_1 \subtynf \tau_3 : \star \gens \Phi$
    
    \caseFact{8} $\Theta ; \Delta \vDash \Phi$
    
    \caseText{By (3)}
    
    \caseFact{9} $\Theta ; \Delta \vDash (I \leq K)$
    
    \caseText{By AS-ConstPot on (7)}
    
    \caseFact{10} $\Psi ; \Theta ; \Delta \pvdash [I] \tau_1 \subtynf [K] \tau_3 : \star \gens (I \leq K) \wedge \Phi$
    
    \caseText{Goal follows by (8), (9), (10)}
  }
  
  \jcase{17}{AS-FamLam}{
    \jgivengoal{
      \caseFact{1} $\Psi ; \Theta ; \Delta \pvdash \lambda i : S.\tau_1 \subtynf \lambda i : S. \tau_2 : S \to K \gens \forall i : S. \Phi_1$    
      
      \caseFact{2} $\Psi ; \Theta ; \Delta \pvdash \lambda i : S.\tau_2 \subtynf \lambda i : S. \tau_3 : S \to K \gens \forall i : S. \Phi_2$
      
      \caseFact{3} $\Theta ; \Delta \vDash \forall i : S.\Phi_1 \wedge \forall i : S.\Phi_2$
      
      \caseFact{4} $\Psi ; \Theta, i : S; \Delta \vdash \tau_1 \subtynf \tau_2 : K \gens \Phi_1$
      
      \caseFact{5} $\Psi ; \Theta, i : S; \Delta \vdash \tau_2 \subtynf \tau_3 : K \gens \Phi_2$
    }{
      $Psi ; \Theta ; \Delta \pvdash \lambda i : S. \tau_1 \subtynf \lambda i : S. \tau_3 : S \to K \gens \Phi$ and $\Theta ; \Delta \vDash \Phi$    
    }
    
    \caseText{By IH on (4) and (5)}
    
    \caseFact{6}  $\Psi ; \Theta, i : S ; \Delta \pvdash \tau_1 \subtynf \tau_3 : K \gens \Phi$
    
    \caseFact{7} $\Theta, i : S ; \Delta \vDash \Phi$
    
    \caseText{By AS-FamLam on (6)}
    
    \caseFact{8} $\Psi ; \Theta ; \Delta \pvdash \lambda i : S. \tau_1 \subtynf \lambda i : S. \tau_3 : S \to K \gens \forall i : S.\Phi$
    
    \caseText{Equivalently to (7)}
    
    \caseFact{9} $\Theta ; \Delta \vDash \forall i : S. \Phi$
    
    \caseText{The Goal follows from (8) and (9)}
  }
  
  \jcase{18}{AS-FamApp}{
    \jgivengoal{
      \caseFact{1} $\Psi ; \Theta ; \Delta \pvdash \tau_1 \; I \subtynf \tau_2 \; J : K \gens (I = J) \wedge \Phi_1$    
      
      \caseFact{2} $\Psi ; \Theta ; \Delta \pvdash \tau_2 \; J \subtynf \tau_3 \; L : K \gens (J = L) \wedge \Phi_2$
      
      \caseFact{3} $\Theta ; \Delta \vDash (I = J) \wedge (J = L) \wedge \Phi_1 \wedge \Phi_2$
      
      \caseFact{4} $\Psi ; \Theta ; \Delta \vdash \tau_1 \subtynf \tau_2 : S \to K \gens \Phi_1$
      
      \caseFact{5} $\Psi ; \Theta ; \Delta \vdash \tau_2 \subtynf \tau_3 : S \to K \gens \Phi_2$
    }{
      $\Psi ; \Theta ; \Delta \pvdash \tau_1 \; I \subtynf \tau_3 \; L : K \gens \Phi$ and $\Theta ; \Delta \vDash \Phi$
    }
    
    \caseText{By IH on (4) and (5)}
    
    \caseFact{5} $\Psi ; \Theta ; \Delta \pvdash \tau_1 \subtynf \tau_3 : S \to K \gens \Phi$
    
    \caseFact{6} $\Theta ; \Delta \vDash \Phi$
    
    \caseText{By (3)}
    
    \caseFact{7} $\Theta ; \Delta \vDash (I = L)$
    
    \caseText{By AS-FamApp on (5)}
    
    \caseFact{8} $\Psi ; \Theta ; \Delta \pvdash \tau_1 \; I \subtynf \tau_3 \; L : K \gens (I = L) \wedge \Phi$
    
    \caseText{The Goal follows by (6),(7),(8)}
  }
}

\iffalse

  \item[(AS-IForall)] Suppose  and
   from  and . By IH, $\Psi ; \Theta, i : S ; \Delta \vdash \tau_1 \subtynf \tau_3 : \star \gens \Phi_1 \wedge \Phi_2$ with $\Theta, i: S ; \Delta \vDash \Phi_1 \wedge \Phi_2$. By AS-IForall, $\Psi ; \Theta ; \Delta \vdash \forall i : S. \tau_1 \subtynf \forall i : S.\tau_3 : \star \gens \forall i : S.(\Phi_1 \wedge \Phi_2)$. But $\Theta, i : S ; \Delta \vDash \Phi_1 \wedge \Phi_2$ is equivalent to $\Theta ; \Delta \vDash \forall i : S. (\Phi_1 \wedge \Phi_2)$, and so we are done.
  \item[(AS-IExists)]  Suppose $\Psi ; \Theta ; \Delta \vdash \exists i : S. \tau_1 \subtynf \exists i : S. \tau_2 : \star \gens \forall i : S. \Phi_1$ and
  $\Psi ; \Theta ; \Delta \vdash \exists i : S. \tau_2 \subtynf \exists i : S. \tau_3 : \star \gens \forall i : S. \Phi_2$ from $\Psi ; \Theta, i : S ; \Delta \vdash \tau_1 \subtynf \tau_2 : \star \gens \Phi_1$ and $\Psi ; \Theta, i : S ; \Delta \vdash \tau_2 \subtynf \tau_3 : \star \gens \Phi_3$. By IH, $\Psi ; \Theta, i : S ; \Delta \vdash \tau_1 \subtynf \tau_3 : \star \gens \Phi_1 \wedge \Phi_2$ with $\Theta, i: S ; \Delta \vDash \Phi_1 \wedge \Phi_2$. By AS-IExists, $\Psi ; \Theta ; \Delta \vdash \exists i : S. \tau_1 \subtynf \exists i : S.\tau_3 : \star \gens \forall i : S.(\Phi_1 \wedge \Phi_2)$. But $\Theta, i : S ; \Delta \vDash \Phi_1 \wedge \Phi_2$ is equivalent to $\Theta ; \Delta \vDash \forall i : S. (\Phi_1 \wedge \Phi_2)$, and so we are done.
  \item[(AS-TForall)] Suppose $\Psi ; \Theta ; \Delta \vdash \forall \alpha : K. \tau_1 \subtynf \forall \alpha : K. \tau_2 : \star \gens \Phi_1$ and
  $\Psi ; \Theta ; \Delta \vdash \forall \alpha : K. \tau_2 \subtynf \forall \alpha : K. \tau_3 : \star \gens \Phi_2$ from
  $\Psi, \alpha : K ; \Theta ; \Delta \vdash \tau_1 \subtynf \tau_2 : \star \gens \Phi_1$ and
  $\Psi, \alpha : K ; \Theta ; \Delta \vdash \tau_2 \subtynf \tau32 : \star \gens \Phi_2$.
  By IH, $\Psi, \alpha : K ; \Theta ; \Delta \vdash \tau_1 \subtynf \tau_3 : \star \gens \Phi$ with $\Theta ; \Delta \vDash \Phi$. By AS-TForall,
  $\Psi ; \Theta ; \Delta \vdash \forall \alpha : K. \tau_1 \subtynf \forall \alpha : K. \tau_3 : \star \gens \Phi$ as required.
  \item[(AS-List)] Suppose $\Psi ; \Theta ; \Delta \vdash L^I \tau_1 \subtynf L^J \tau_2 : \star \gens I = J \wedge \Phi_1$ and
  $\Psi ; \Theta ; \Delta \vdash L^J \tau_2 \subtynf L^K \tau_3 : \star \gens J = K \wedge \Phi_2$
  from $\Psi ; \Theta ; \Delta \vdash \tau_1 \subtynf \tau_2 : \star \gens \Phi_1$ and
  $\Psi ; \Theta ; \Delta \vdash \tau_2 \subtynf \tau_3 : \star \gens \Phi_2$. Since $\Theta ; \Delta \vdash \Phi_1 \wedge \Phi_2$, we have by IH that
  $\Psi ; \Theta ; \Delta \vdash \tau_1 \subtynf \tau_3 : \star \gens \Phi$ with $\Theta ; \Delta \vDash \Phi$. Since $\Theta ; \Delta \vDash I = J \wedge J = K$ also,
  we have that $\Theta ; \Delta \vDash I = K \wedge \Phi$. Then, by AS-List, we have
  $\Psi ; \Theta ; \Delta \vdash L^I \tau_1 \subtynf L^K \tau_3 : \star \gens I = K \wedge \Phi$, and so we are done.
  \item[(AS-Impl)] Suppose $\Psi ; \Theta ; \Delta \vdash \Phi_1' \implies \tau_1 \subtynf \Phi_2' \implies \tau_2 : \star \gens \Phi_1 \wedge (\Phi_2' \to \Phi_1')$
  and $\Psi ; \Theta ; \Delta \vdash \Phi_2' \implies \tau_2 \subtynf \Phi_3' \implies \tau_3 : \star \gens \Phi_2 \wedge (\Phi_3' \to \Phi_2')$
  from $\Psi ; \Theta ; \Delta \vdash \tau_1 \subtynf \tau_2 : star \gens \Phi_1$ and 
  $\Psi ; \Theta ; \Delta \vdash \tau_2 \subtynf \tau_3 : \star \gens \Phi_3$. By IH,
  $\Psi ; \Theta ; \Delta \vdash \tau_1 \subtynf \tau_3 : \star \gens \Phi$. with $\Theta ; \Delta \vDash \Phi$.
  By AS-Impl, $\Psi ; \Theta ; \Delta \vdash \Phi_1' \implies \tau_1 \subtynf \Phi_3' \implies \tau_3 : \star \gens \Phi \wedge (\Phi_3' \to \Phi_1')$. But since $\Theta ; \Delta \vDash (\Phi_3' \to \Phi_2') \wedge (\Phi_2' \to \Phi_1')$, we have that $\Theta ; \Delta \vDash \Phi \wedge (\Phi_3' \to \Phi_1')$, as required.
  \item[(AS-Conj)] Suppose $\Psi ; \Theta ; \Delta \vdash \Phi_1' \amp \tau_1 \subtynf \Phi_2' \amp \tau_2 : \star \gens \Phi_1 \wedge (\Phi_1' \to \Phi_2')$
  and $\Psi ; \Theta ; \Delta \vdash \Phi_2' \amp \tau_2 \subtynf \Phi_3' \amp \tau_3 : \star \gens \Phi_2 \wedge (\Phi_2' \to \Phi_3')$
  from $\Psi ; \Theta ; \Delta \vdash \tau_1 \subtynf \tau_2 : \star \gens \Phi_1$ and 
  $\Psi ; \Theta ; \Delta \vdash \tau_2 \subtynf \tau_3 : \star \gens \Phi_3$. By IH,
  $\Psi ; \Theta ; \Delta \vdash \tau_1 \subtynf \tau_3 : \star \gens \Phi$. with $\Theta ; \Delta \vDash \Phi$.
  By AS-Conj, $\Psi ; \Theta ; \Delta \vdash \Phi_1' \amp \tau_1 \subtynf \Phi_3' \amp \tau_3 : \star \gens \Phi \wedge (\Phi_1' \to \Phi_3')$. But since $\Theta ; \Delta \vDash (\Phi_1' \to \Phi_2') \wedge (\Phi_2' \to \Phi_3')$, we have that $\Theta ; \Delta \vDash \Phi \wedge (\Phi_1' \to \Phi_3')$, as required.
  \item[(AS-Monad)]
  Suppose $\Psi ; \Theta ; \Delta \vdash \M(I,\vec{q}) \tau_1 \subtynf \M(J,\vec{p}) \tau_2 : \star \gens (I = J) \wedge (\vec{q} \leq \vec{p}) \wedge \Phi_1$
  and $\Psi ; \Theta ; \Delta \vdash \M(J,\vec{p}) \tau_2 \subtynf \M(K,\vec{r}) \tau_3 : \star \gens (J = K) \wedge (\vec{p} \leq \vec{r}) \wedge \Phi_2$
  from $\Psi ; \Theta ; \Delta \vdash \tau_1 \subtynf \tau_2 : \star \gens \Phi_1$ and
  $\Psi ; \Theta ; \Delta \vdash \tau_2 \subtynf \tau_3 : \star \gens \Phi_2$.
  By IH, $\Psi ; \Theta ; \Delta \vdash \tau_1 \subtynf \tau_3 : \star \gens \Phi$, with $\Theta ; \Delta \vDash \Phi$.
  By AS-Monad, $\Psi ; \Theta ; \Delta \vdash \M(I,\vec{q}) \tau_1 \subtynf \M(K,\vec{r})\tau_3 : \star \gens (I = K) \wedge (\vec{q} \leq \vec{r}) \Phi$.
  But since $\Theta ; \Delta \vDash (I = J) \wedge (J = K) \wedge (\vec{q} \leq \vec{p}) \wedge (\vec{p} \leq \vec{r})$, we have $\Theta ; \Delta \vDash (I = K) \wedge (\vec{q} \leq \vec{r}) \Phi$, as required.
  \item[(AS-Pot)] Suppose $\Psi ; \Theta ; \Delta \vdash [I|\vec{q}] \tau_1 \subtynf [J|\vec{p}] \tau_2 : \star \gens (I = J) \wedge (\vec{q} \geq \vec{p}) \wedge \Phi_1$
  and $\Psi ; \Theta ; \Delta \vdash [J|\vec{p}] \tau_2 \subtynf [K|\vec{r}] \tau_3 : \star \gens (J = K) \wedge (\vec{p} \geq \vec{r}) \wedge \Phi_2$
  from $\Psi ; \Theta ; \Delta \vdash \tau_1 \subtynf \tau_2 : \star \gens \Phi_1$ and
  $\Psi ; \Theta ; \Delta \vdash \tau_2 \subtynf \tau_3 : \star \gens \Phi_2$.
  By IH, $\Psi ; \Theta ; \Delta \vdash \tau_1 \subtynf \tau_3 : \star \gens \Phi$, with $\Theta ; \Delta \vDash \Phi$.
  By AS-Pot, $\Psi ; \Theta ; \Delta \vdash [I|\vec{q}] \tau_1 \subtynf [K|\vec{r}] \tau_3 : \star \gens (I = K) \wedge (\vec{q} \geq \vec{r}) \Phi$.
  But since $\Theta ; \Delta \vDash (I = J) \wedge (J = K) \wedge (\vec{q} \geq \vec{p}) \wedge (\vec{p} \geq \vec{r})$, we have $\Theta ; \Delta \vDash (I = K) \wedge (\vec{q} \geq \vec{r}) \Phi$, as required.
  \item[(AS-ConstPot)] Suppose $\Psi ; \Theta ; \Delta \vdash [I] \tau_1 \subtynf [J] \tau_2 : \star \gens \Phi_1 \wedge (J \leq I)$
  and $\Psi ; \Theta ; \Delta \vdash [J] \tau_2 \subtynf [K] \tau_3 : \star \gens \Phi_2 \wedge (K \leq J)$
  from $\Psi ; \Theta ; \Delta \vdash \tau_1 \subtynf \tau_2 : \star \gens \Phi_1$ and $\Psi ; \Theta ; \Delta \vdash \tau_2 \subtynf \tau_3 : \star \gens \Phi_3$.
  By IH, $\Psi ; \Theta ; \Delta \vdash \tau_1 \subtynf \tau_3 : \star \gens \Phi$ with $\Theta ; \Delta \vDash \Phi$.
  By AS-ConstPot, $\Psi ; \Theta ; \Delta \vdash [I] \tau_1 \subtynf [K]\tau_3 : \star \gens \Phi$ with $\Theta ; \Delta \vDash \Phi \wedge (K \leq I)$. But, $\Theta ; \Delta \vDash (K \leq J) \wedge (J \leq I)$, so $\Theta ; \Delta \vDash K \leq I$, and so we are done. 
  \item[(AS-FamLam)] Suppose $\Psi ; \Theta ; \Delta \vdash \lambda i : S. \tau_1 \subtynf \lambda i : S. \tau_2 : S \to K \gens \forall i : S. \Phi_1$
  and $\Psi ; \Theta ; \Delta \vdash \lambda i : S. \tau_2 \subtynf \lambda i : S. \tau_3 : S \to K \gens \forall i : S. \Phi_2$
  from $\Psi ; \Theta, i : S ; \Delta \vdash \tau_1 \subtynf \tau_2 : K \gens \Phi_1$
  and $\Psi ; \Theta, i : S ; \Delta \vdash \tau_2 \subtynf \tau_3 : K \gens \Phi_2$.
  By IH, $\Psi ; \Theta, i :S ; \Delta \vDash \tau_1 \subtynf \tau_3 : K \gens \Phi$ with $\Theta , i : S; \Delta \vDash \Phi$, which is by definition $\Theta ; \Delta \vDash \forall i : S. \Phi$.
  Then by AS-Fam, $\Psi ; \Theta ; \Delta \vdash \lambda i : S. \tau_1 \subtynf \lambda i : S. \tau_3 : S \to K \gens \forall i : S. \Phi$, as required.
  \item[(AS-FamApp)]  Suppose $\Psi ; \Theta ; \Delta \vdash \tau_1 \; I \subtynf \tau_2 \; J : K \gens (I = J) \wedge \Phi_1$
  and $\Psi ; \Theta ; \Delta \vdash \tau_2 \; J \subtynf \tau_3 \; L : K \gens (J = L) \wedge \Phi_2$
  from $\Psi ; \Theta ; \Delta \vdash \tau_1 \subtynf \tau_2 : S \to K \gens \Phi_1$ and
   $\Psi ; \Theta ; \Delta \vdash \tau_2 \subtynf \tau_3 : S \to K \gens \Phi_2$.
   By IH, $\Psi ; \Theta ; \Delta \vdash \tau_1 \subtynf \tau_3 : S \to K \gens \Phi$ with $\Theta ; \Delta \vDash \Phi$.
   By AS-FamApp, $\Psi ; \Theta ; \Delta \vdash \tau_1 \; I \subtynf \tau_3 \; L :  K \gens \Phi \wedge (I = L)$ with $\Theta ; \Delta \vDash \Phi$.
   Since $\Theta ; \Delta \vDash (I = J) \wedge (J = L)$, $\Theta ; \Delta \vDash (I=L)$.
\end{itemize}
\end{proof}

\fi


\begin{theorem}[Index Substitution for Algorithmic Sort Checking]
If $\Theta, i : S ; \Delta \pvdash J : S' \gens \Phi_1$ and $\Theta ; \Delta \pvdash I : S \gens \Phi_2$ with
$\Theta, i : S ; \Delta \vDash \Phi_1$ and $\Theta; \Delta \vDash \Phi_2$, then
$\Theta ; \Delta \pvdash J[I/i] : S' \gens \Phi$ for some $\Theta ; \Delta \vDash \Phi$
\label{idx-idx-algo-subst}
\end{theorem}
\begin{proof}
By Theorem~\ref{thm:sort-sound}, $\Theta, i :S ; \Delta \pvdash J : S'$ and $\Theta ; \Delta \pvdash I : S$.
By Theorem~\ref{thm:idx-idx-subst}, $\Theta ; \Delta \pvdash J[I/i] : S'$.
By Theorem~\ref{thm:sort-compl}, $\Theta ; \Delta \pvdash J[I/i] : S' \gens \Phi'$ for some $\Phi'$ such that $\Theta ; \Delta \vDash \Phi'$.
\end{proof}

\begin{theorem}[Index Substitution for Algorithmic Constraint Well-Formedness]
If $\Theta, i : S; \Delta \pvdash \Phi \; \texttt{wf} \gens \Phi_1$ and $\Theta ; \Delta \pvdash I : S \gens \Phi_2$
with $\Theta, i : S ; \Delta \vDash \Phi_1$ and $\Theta ; \Delta \vDash \Phi_2$, then
$\Theta ; \Delta \pvdash  \Phi[I/i] \; \texttt{wf} \gens \Phi'$ for some $\Theta ; \Delta \vDash \Phi'$
\label{thm:constr-idx-algo-subst}
\end{theorem}
\begin{proof}
By Theorem~\ref{thm:constr-sound}, $\Theta, i : S ; \Delta \pvdash \Phi \; \texttt{wf}$.
By Theorem~\ref{thm:sort-sound}, $\Theta ; \Delta \pvdash I : S$.
By Theorem~\ref{thm:constr-idx-subst}, $\Theta ; \Delta \pvdash \Phi[I/i] \; \texttt{wf}$.
By Theorem~\ref{thm:constr-compl}, $\Theta ; \Delta \pvdash \Phi[I/i] \; \texttt{wf} \gens \Phi'$ for some $\Theta ; \Delta \vDash \Phi'$
\end{proof}

\begin{theorem}[Index Substitution for Algorithmic Type Formation]
If $\Psi ; \Theta, i : S ; \Delta \pvdash \tau : K \gens \Phi_1$ and $\Theta ; \Delta \pvdash I : S \gens \Phi_2$ 
with $\Theta, i : S ; \Delta \vDash \Phi_1$ and $\Theta ; \Delta \vDash \Phi_2$, then
$\Psi ; \Theta ; \Delta \pvdash \tau[I/i] : K \gens \Phi$ for some $\Theta ; \Delta \vDash \Phi$
\label{thm:type-idx-algo-subst}
\end{theorem}
\begin{proof}
By Theorem~\ref{thm:kind-sound}, $\Psi ; \Theta, i : S ; \Delta \pvdash \tau : K$.
By Theorem~\ref{thm:sort-sound}, $\Theta ; \Delta \pvdash I : S$.
By Theorem~\ref{thm:type-idx-subst}, $\Psi ; \Theta ; \Delta \pvdash \tau[I/i] : K$.
Finally, by Theorem~\ref{thm:kind-compl}, $\Psi ; \Theta ; \Delta \pvdash \tau[I/i] : K \gens \Phi$ for some $\Theta ; \Delta \vDash \Phi$
\end{proof}

\subtynfidxsubst*
\jtheorem{Proof of Theorem~\ref{thm:subtynf-idx-subst}}{

  \jgivengoal{
    \caseFact{1} $\Psi ; \Theta, i : S ; \Delta \pvdash \tau_1 \subtynf \tau_2 : K \gens \Phi$
    
    \caseFact{2} $\Theta ; \Delta \vDash \Phi$
    
    \caseFact{3} $\Theta \vdash \Delta \; \texttt{wf}$
    
    \caseFact{4} $\Theta ; \Delta \pvdash I : S \gens \Phi_1$ with $\Theta ; \Delta \vDash \Phi_1$
    
    \caseFact{5} $\Theta ; \Delta \pvdash J : S \gens \Phi_2$ with $\Theta ; \Delta \vDash \Phi_2$ 
    
    \caseFact{6} $\Theta ; \Delta \vDash I = J$  
  }{
    $\Psi ; \Theta ; \Delta \pvdash \tau_1[I/i] \subtynf \tau_2[J/i] : K \gens \Phi'$ for some $\Phi'$ with $\Theta ; \Delta \vDash \Phi'$
  }
  
  \caseText{In most cases, it suffices to show that $\Psi ; \Theta ; \Delta \vdash \tau_1[I/i] \subtynf \tau_2[J/i] : K \gens \Phi'$ for some solvable $\Phi'$
    since Theorem~\ref{thm:idx-ctx-wf-compl} on (3), Theorem~\ref{thm:idx-subst-nf} along with the presuppositions for (1) give the presuppositions for the conclusion,
    using Theorem~\ref{thm:type-idx-algo-subst}. When this is not immediate, we manually reconstruct the presuppositions required.
  }
  
  \jcase{1}{AS-Unit}{Immediate.}
  \jcase{2}{AS-Var}{Immediate.}
  \jcase{3}{AS-Arr}{
   \jgivengoal{
     \caseFact{7} $\Psi ; \Theta, i : S ; \Delta \pvdash \tau_1 \loli \tau_2 \subtynf \tau_1' \loli \tau_2' : \star \gens \Phi_1' \wedge \Phi_2'$
     
     \caseFact{8} $\Psi ; \Theta, i : S ; \Delta \vdash \tau_1' \subtynf \tau_1 : \star \gens \Phi_1'$
     
     \caseFact{9} $\Psi ; \Theta, i : S ; \Delta \vdash \tau_2 \subtynf \tau_2' : \star \gens \Phi_2'$
   }{
     $\Psi ; \Theta ; \Delta \vdash (\tau_1 \loli \tau_2)[I/i] \subtynf (\tau_1' \loli \tau_2')[J/i] : \star \gens \Phi'$ for some $\Theta ; \Delta \vDash \Phi'$
   }
   \caseText{By IH on (8) and (9)}
   
   \caseFact{10} $\Psi ; \Theta ; \Delta \pvdash \tau_1'[J/i] \subtynf \tau_1[I/i] : \star \gens \Phi_1''$
   
   \caseFact{11} $\Theta ; \Delta \vDash \Phi_1''$
   
   \caseFact{12} $\Psi ; \Theta ; \Delta \vdash \tau_2[I/i] \subtynf \tau_2'[J/i] : \star \gens \Phi_2''$
   
   \caseFact{13} $\Theta ; \Delta \vDash \Phi_2''$
   
   \caseText{By AS-Arr on (10) and (12)}
   
   \caseFact{14} $\Psi ; \Theta ; \Delta \pvdash \tau_1[I/i] \loli \tau_2[I/i] \subtynf \tau_1'[J/i] \loli \tau_2'[J/i] : \star \gens \Phi_1'' \wedge \Phi_2''$
   
   \caseText{Goal follows immediately from (14), with $\Phi' = \Phi_1'' \wedge \Phi_2''$}
  }
  \jcase{4}{AS-Tensor}{
   \jgivengoal{
     \caseFact{7} $\Psi ; \Theta, i : S ; \Delta \pvdash \tau_1 \otimes \tau_2 \subtynf \tau_1' \otimes \tau_2' : \star \gens \Phi_1' \wedge \Phi_2'$
     
     \caseFact{8} $\Psi ; \Theta, i : S ; \Delta \vdash \tau_1 \subtynf \tau_1' : \star \gens \Phi_1'$
     
     \caseFact{9} $\Psi ; \Theta, i : S ; \Delta \vdash \tau_2 \subtynf \tau_2' : \star \gens \Phi_2'$
   }{
     $\Psi ; \Theta ; \Delta \vdash (\tau_1 \otimes \tau_2)[I/i] \subtynf (\tau_1' \otimes \tau_2')[J/i] : \star \gens \Phi'$ for some $\Theta ; \Delta \vDash \Phi'$
   }
   \caseText{By IH on (8) and (9)}
   
   \caseFact{10} $\Psi ; \Theta ; \Delta \pvdash \tau_1[J/i] \subtynf \tau_1'[I/i] : \star \gens \Phi_1''$
   
   \caseFact{11} $\Theta ; \Delta \vDash \Phi_1''$
   
   \caseFact{12} $\Psi ; \Theta ; \Delta \vdash \tau_2[I/i] \subtynf \tau_2'[J/i] : \star \gens \Phi_2''$
   
   \caseFact{13} $\Theta ; \Delta \vDash \Phi_2''$
   
   \caseText{By AS-Tensor on (10) and (12)}
   
   \caseFact{14} $\Psi ; \Theta ; \Delta \pvdash \tau_1[I/i] \otimes \tau_2[I/i] \subtynf \tau_1'[J/i] \otimes \tau_2'[J/i] : \star \gens \Phi_1'' \wedge \Phi_2''$
   
   \caseText{Goal follows immediately from (14), with $\Phi' = \Phi_1'' \wedge \Phi_2''$}
  }
  
  \jcase{5}{AS-With}{
   \jgivengoal{
     \caseFact{7} $\Psi ; \Theta, i : S ; \Delta \pvdash \tau_1 \amp \tau_2 \subtynf \tau_1' \amp \tau_2' : \star \gens \Phi_1' \wedge \Phi_2'$
     
     \caseFact{8} $\Psi ; \Theta, i : S ; \Delta \vdash \tau_1 \subtynf \tau_1' : \star \gens \Phi_1'$
     
     \caseFact{9} $\Psi ; \Theta, i : S ; \Delta \vdash \tau_2 \subtynf \tau_2' : \star \gens \Phi_2'$
   }{
     $\Psi ; \Theta ; \Delta \vdash (\tau_1 \amp \tau_2)[I/i] \subtynf (\tau_1' \amp \tau_2')[J/i] : \star \gens \Phi'$ for some $\Theta ; \Delta \vDash \Phi'$
   }
   \caseText{By IH on (8) and (9)}
   
   \caseFact{10} $\Psi ; \Theta ; \Delta \pvdash \tau_1[J/i] \subtynf \tau_1'[I/i] : \star \gens \Phi_1''$
   
   \caseFact{11} $\Theta ; \Delta \vDash \Phi_1''$
   
   \caseFact{12} $\Psi ; \Theta ; \Delta \vdash \tau_2[I/i] \subtynf \tau_2'[J/i] : \star \gens \Phi_2''$
   
   \caseFact{13} $\Theta ; \Delta \vDash \Phi_2''$
   
   \caseText{By AS-With on (10) and (12)}
   
   \caseFact{14} $\Psi ; \Theta ; \Delta \pvdash \tau_1[I/i] \amp \tau_2[I/i] \subtynf \tau_1'[J/i] \amp \tau_2'[J/i] : \star \gens \Phi_1'' \wedge \Phi_2''$
   
   \caseText{Goal follows immediately from (14), with $\Phi' = \Phi_1'' \wedge \Phi_2''$}
  }
  
  \jcase{6}{AS-Sum}{
   \jgivengoal{
     \caseFact{7} $\Psi ; \Theta, i : S ; \Delta \pvdash \tau_1 \oplus \tau_2 \subtynf \tau_1' \oplus \tau_2' : \star \gens \Phi_1' \wedge \Phi_2'$
     
     \caseFact{8} $\Psi ; \Theta, i : S ; \Delta \vdash \tau_1 \subtynf \tau_1' : \star \gens \Phi_1'$
     
     \caseFact{9} $\Psi ; \Theta, i : S ; \Delta \vdash \tau_2 \subtynf \tau_2' : \star \gens \Phi_2'$
   }{
     $\Psi ; \Theta ; \Delta \vdash (\tau_1 \oplus \tau_2)[I/i] \subtynf (\tau_1' \oplus \tau_2')[J/i] : \star \gens \Phi'$ for some $\Theta ; \Delta \vDash \Phi'$
   }
   \caseText{By IH on (8) and (9)}
   
   \caseFact{10} $\Psi ; \Theta ; \Delta \pvdash \tau_1[J/i] \subtynf \tau_1'[I/i] : \star \gens \Phi_1''$
   
   \caseFact{11} $\Theta ; \Delta \vDash \Phi_1''$
   
   \caseFact{12} $\Psi ; \Theta ; \Delta \vdash \tau_2[I/i] \subtynf \tau_2'[J/i] : \star \gens \Phi_2''$
   
   \caseFact{13} $\Theta ; \Delta \vDash \Phi_2''$
   
   \caseText{By AS-Sum on (10) and (12)}
   
   \caseFact{14} $\Psi ; \Theta ; \Delta \pvdash \tau_1[I/i] \oplus \tau_2[I/i] \subtynf \tau_1'[J/i] \oplus \tau_2'[J/i] : \star \gens \Phi_1'' \wedge \Phi_2''$
   
   \caseText{Goal follows immediately from (14), with $\Phi' = \Phi_1'' \wedge \Phi_2''$}
  }
  
  
  \jcase{7}{AS-Bang}{
    \jgivengoal{
       \caseFact{7} $\Psi ; \Theta, i : S ; \Delta \pvdash !\tau_1 \subtynf !\tau_2 : \star \gens \Phi$
       
       \caseFact{8} $\Psi ; \Theta, i : S ; \Delta \vdash \tau_1 \subtynf \tau_2 : \star \gens \Phi$
    }{
      $\Psi ; \Theta ; \Delta \vdash (!\tau_1)[I/i] \subtynf (!\tau_2)[J/i] : \star \gens \Phi'$ for some $\Theta ; \Delta \vDash \Phi'$
    }  
    
    \caseText{By IH on (8)}
    
    \caseFact{9} $\Psi ; \Theta ; \Delta \pvdash \tau_1[I/i] \subtynf \tau_2[J/i] : \star \gens \Phi'$
    
    \caseFact{10} $\Theta ; \Delta \vDash \Phi'$
    
    \caseText{By AS-Bang on (9)}
    
    \caseFact{9} $\Psi ; \Theta ; \Delta \pvdash !\tau_1[I/i] \subtynf !\tau_2[J/i] : \star \gens \Phi'$
    
    \caseText{Goal follows immediately from (9)}
  }
  
  \jcase{8}{AS-IForall}{
    \jgivengoal{
      \caseFact{7} $\Psi ; \Theta, i : S ; \Delta \pvdash \forall j : S'. \tau_1 \subtynf \forall j : S'. \tau_2 : \star \gens \forall j : S'. \Phi$
      
      \caseFact{8} $\Psi ; \Theta, i : S, j : S'; \Delta \vdash \tau_1 \subtynf \tau_2 : \star \gens \Phi$
    }{
      $\Psi ; \Theta ; \Delta \vdash (\forall j : S'. \tau_1)[I/i] \subtynf (\forall j : S'. \tau_2)[J/i] : \star \gens \Phi'$ for some $\Theta ; \Delta \vDash \Phi'$    
    }
   \caseText{By (2)}
   
   \caseFact{9} $\Theta, i : S, j : S ; \Delta \vDash \Phi $
   
   \caseText{IH on (8)}
   
   \caseFact{10} $\Psi ; \Theta j : S'; \Delta \vdash \tau_1[I/i] \subtynf \tau_2[J//i] : \star \gens \Phi'$
   
   \caseFact{11} $\Theta, j : S' ; \Delta \vDash \Phi'$
   
   \caseText{Equivalently to (11)}
   
   \caseFact{12} $\Theta ; \Delta \vDash \forall j : S'. \Phi'$
   
   \caseText{By AS-IForall on (10)}
   
   \caseFact{13} $Psi ; \Theta ; \Delta \vdash \forall j : S'. \tau_1[I/i] \subtynf \forall j : S'.\tau_2[J/i] : \star \gens \forall j : S'.\Phi'$
   
   \caseText{Goal follows by (12) and (13)}
  
  }
  
  \jcase{8}{AS-IExists}{
    \jgivengoal{
      \caseFact{7} $\Psi ; \Theta, i : S ; \Delta \pvdash \exists j : S'. \tau_1 \subtynf \exists j : S'. \tau_2 : \star \gens \forall j : S'. \Phi$
      
      \caseFact{8} $\Psi ; \Theta, i : S, j : S'; \Delta \vdash \tau_1 \subtynf \tau_2 : \star \gens \Phi$
    }{
      $\Psi ; \Theta ; \Delta \vdash (\exists j : S'. \tau_1)[I/i] \subtynf (\exists j : S'. \tau_2)[J/i] : \star \gens \Phi'$ for some $\Theta ; \Delta \vDash \Phi'$    
    }
   \caseText{By (2)}
   
   \caseFact{9} $\Theta, i : S, j : S ; \Delta \vDash \Phi $
   
   \caseText{IH on (8)}
   
   \caseFact{10} $\Psi ; \Theta j : S'; \Delta \vdash \tau_1[I/i] \subtynf \tau_2[J//i] : \star \gens \Phi'$
   
   \caseFact{11} $\Theta, j : S' ; \Delta \vDash \Phi'$
   
   \caseText{Equivalently to (11)}
   
   \caseFact{12} $\Theta ; \Delta \vDash \forall j : S'. \Phi'$
   
   \caseText{By AS-IExists on (10)}
   
   \caseFact{13} $Psi ; \Theta ; \Delta \vdash \exists j : S'. \tau_1[I/i] \subtynf \exists j : S'.\tau_2[J/i] : \star \gens \forall j : S'.\Phi'$
   
   \caseText{Goal follows by (12) and (13)}
  
  }
  
  \jcase{9}{AS-TForall}{
  
    \jgivengoal{
       \caseFact{7} $\Psi ; \Theta, i : S ; \Delta \pvdash \forall \alpha : K. \tau_1 \subtynf \forall \alpha : K. \tau_2 : \star \gens \Phi$
       
       \caseFact{8} $\Psi, \alpha : K ; \Theta, i : S ; \Delta \vdash \tau_1 \subtynf \tau_2 : \star \gens \Phi$
    }{
      $\Psi ; \Theta ; \Delta \vdash (\forall \alpha : K. \tau_1)[I/i] \subtynf (\forall \alpha : K. \tau_2)[J/i] : \star \gens \Phi'$ for some $\Theta ; \Delta \vDash \Phi'$    
    }
    
    \caseText{By IH on (8)}
    
    \caseFact{9} $\Psi, \alpha : K ; \Theta ; \Delta \pvdash \tau_1[I/i] \subtynf \tau_2[I/i] : \star \gens \Phi'$
    
    \caseFact{10} $\Theta ; \Delta \vDash \Phi'$
    
    \caseText{By AS-TForall on (9)}
    
    \caseFact{11} $\Psi ; \Theta ; \Delta \pvdash \forall \alpha : K. \tau_1[I/i] \subtynf \forall \alpha : K. \tau_2[I/i] : \star \gens \Phi'$
    
    \caseText{Goal follows immediately from (11)}
   
  }
  
  \jcase{10}{AS-List}{
    \jgivengoal{
      \caseFact{7} $\Psi ; \Theta, i : S ; \Delta \pvdash L^M \tau_1 \subtynf L^N \tau_2 : \star \gens M = N \wedge \Phi'$ 
      
      \caseFact{8} $\Psi ; \Theta, i : S ; \Delta \vdash \tau_1 \subtynf \tau_2 : \star \gens \Phi'$ 
    }{
      $\Psi ; \Theta ; \Delta \vdash (L^M \tau_1)[I/i] \subtynf (L^N \tau_2)[J/i] : \star \gens \Phi'$ for some $\Theta ; \Delta \vDash \Phi'$
    }  
    \caseText{By (2)}
    
    \caseFact{9} $\Theta, i : S ; \Delta \vDash M = N$
    
    \caseText{From (9) and (4)}
    
    \caseFact{10} $\Theta ; \Delta \vDash M[I/i] = N[J/i]$
    
    \caseText{By IH on (8)}
    
    \caseFact{11} $\Psi ; \Theta ; \Delta \pvdash \tau_1[I/i] \subtynf \tau_2[J/i] : \star \gens \Phi''$
    
    \caseFact{12} $\Theta ; \Delta \vDash \Phi''$
    
    \caseText{By the presupposition for (7) and two inversions}
    
    \caseFact{13} $\Theta, i : S ; \Delta \vdash M : \N \gens \Phi_1$ with $\Theta, i : S ; \Delta \vDash \Phi_1$
    
    \caseFact{14} $\Theta, i : S ; \Delta \vdash N : \N \gens \Phi_2$ with $\Theta, i : S ; \Delta \vDash \Phi_2$

    \caseText{By Theorem~\ref{thm:idx-idx-algo-subst} and Theorem~\ref{thm:idx-compl} on (13) (14) (4) (5)}
    
    \caseFact{15} $\Theta ; \Delta \vdash M[I/i] : \N \gens \Phi_1'$ with $\Theta;\Delta \vDash \Phi_1'$
    
    \caseFact{16} $\Theta ; \Delta \vdash N[J/i] : \N \gens \Phi_2'$ with $\Theta ; \Delta \vDash \Phi_1'$
    
    \caseText{By AS-List on (10) and (11)}
    
    \caseFact{17} $\Psi ; \Theta ; \Delta \vdash L^{M[I/i]} \tau_1[I/i] \subtynf L^{N[J/i]} \tau_2[J/i] : \star \gens (M[I/i] = N[J/i]) \wedge \Phi''$
    
    \caseText{By (15), (16), (17)}
    
    \caseFact{18} $\Psi ; \Theta ; \Delta \pvdash L^{M[I/i]} \tau_1[I/i] \subtynf L^{N[J/i]} \tau_2[J/i] : \star \gens (M[I/i] = N[J/i]) \wedge \Phi''$
    
    \caseText{Goal follows from (18), taking $\Phi' = (M[I/i] = N[J/i]) \wedge \Phi''$}
   
  }
  
  \jcase{11}{AS-Conj}{
    \jgivengoal{
      \caseFact{7} $\Psi ; \Theta, i : S; \Delta \pvdash \Phi_1 \amp \tau_1 \subtynf \Phi_2 \amp \tau_2 : \star \gens \Phi \wedge (Phi_1 \to \Phi_2)$
      
      \caseFact{8} $\Psi ; \Theta, i : S ; \Delta \vdash \tau_1 \subtynf \tau_2 : \star \gens \Phi$
    }{
      $\Psi ; \Theta ; \Delta \vdash (\Phi_1 \amp \tau_1)[I/i] \subtynf (\Phi_2 \amp \tau_2)[J/i] : \star \gens \Phi'$ for some $\Theta ; \Delta \vDash \Phi'$
    }
    \caseText{By IH on (8)}
    
    \caseFact{9} $\Psi ; \Theta ; \Delta \pvdash \tau_1[I/i] \subtynf \tau_2[J/i] : \star \gens \Phi'$
    
    \caseFact{10} $\Theta ; \Delta \vDash \Phi'$
    
    \caseText{Inverting the presupposition that $\Phi_1 \amp \tau_1 $ and $\Phi_2 \amp \tau_2$ are well-formed types from (1), we have}
    
    \caseFact{11} $\Theta, i : S ; \Delta \vdash \Phi_1 \; \texttt{wf} \gens \Phi_1'$ with $\Theta, i : S ; \Delta \vDash \Phi_1'$
    
    \caseFact{12} $\Theta, i : S ; \Delta \vdash \Phi_2 \; \texttt{wf} \gens \Phi_2'$ with $\Theta, i : S ; \Delta \vDash \Phi_2'$
    
    \caseText{By Theorem~\ref{thm:constr-idx-algo-subst}}
    
    \caseFact{13} $\Theta ; \Delta \vDash \Phi_1[I/i] \; \texttt{wf} \gens \Phi_1''$ with $\Theta ; \Delta \vDash \Phi_1''$
    
    \caseFact{14} $\Theta ; \Delta \vDash \Phi_2[J/i] \; \texttt{wf} \gens \Phi_2''$ with $\Theta ; \Delta \vDash \Phi_2''$
    
    \caseText{By AS-Conj on (9)}
    
    \caseFact{15} $\Psi ; \Theta ; \Delta \vdash \Phi_1[I/i] \amp \tau_1[I/i] \subtynf \Phi_2[J/i] \amp \tau_2[I/i] \gens \Phi' \wedge (\Phi_1[I/i] \to \Phi_2[J/i])$
    
    \caseText{By (11) and (12), the presuppositions for (15) hold}
    
    \caseFact{16} $\Psi ; \Theta ; \Delta \pvdash \Phi_1[I/i] \amp \tau_1[I/i] \subtynf \Phi_2[J/i] \amp \tau_2[I/i] \gens \Phi' \wedge (\Phi_1[I/i] \to \Phi_2[J/i])$
    
    \caseText{By (2)}
    
    \caseFact{17} $\Theta, i : S ; \Delta \vDash \Phi_1 \to \Phi_2$
    
    \caseText{By (17) and (6)}
    
    \caseFact{18} $\Theta ; \Delta \vDash \Phi_1[I/i] \to \Phi_2[J/i]$
    
    \caseText{The result follows by (16) and (18), with $\Phi' = \Phi' \wedge (\Phi_1[I/i] \to \Phi_2[J/i])$}    
  
  }
  
  \jcase{12}{AS-Impl}{
    \jgivengoal{
      \caseFact{7} $\Psi ; \Theta, i : S; \Delta \pvdash \Phi_1 \implies \tau_1 \subtynf \Phi_2 \implies \tau_2 : \star \gens \Phi \wedge (Phi_2 \to \Phi_1)$
      
      \caseFact{8} $\Psi ; \Theta, i : S ; \Delta \vdash \tau_1 \subtynf \tau_2 : \star \gens \Phi$
    }{
      $\Psi ; \Theta ; \Delta \vdash (\Phi_1 \implies \tau_1)[I/i] \subtynf (\Phi_2 \amp \tau_2)[J/i] : \star \gens \Phi'$ for some $\Theta ; \Delta \vDash \Phi'$
    }
    \caseText{By IH on (8)}
    
    \caseFact{9} $\Psi ; \Theta ; \Delta \pvdash \tau_1[I/i] \subtynf \tau_2[J/i] : \star \gens \Phi'$
    
    \caseFact{10} $\Theta ; \Delta \vDash \Phi'$
    
    \caseText{Inverting the presupposition that $\Phi_1 \amp \tau_1 $ and $\Phi_2 \amp \tau_2$ are well-formed types from (1), we have}
    
    \caseFact{11} $\Theta, i : S ; \Delta \vdash \Phi_1 \; \texttt{wf} \gens \Phi_1'$ with $\Theta, i : S ; \Delta \vDash \Phi_1'$
    
    \caseFact{12} $\Theta, i : S ; \Delta \vdash \Phi_2 \; \texttt{wf} \gens \Phi_2'$ with $\Theta, i : S ; \Delta \vDash \Phi_2'$
    
    \caseText{By Theorem~\ref{thm:constr-idx-algo-subst}}
    
    \caseFact{13} $\Theta ; \Delta \vDash \Phi_1[I/i] \; \texttt{wf} \gens \Phi_1''$ with $\Theta ; \Delta \vDash \Phi_1''$
    
    \caseFact{14} $\Theta ; \Delta \vDash \Phi_2[J/i] \; \texttt{wf} \gens \Phi_2''$ with $\Theta ; \Delta \vDash \Phi_2''$
    
    \caseText{By AS-Implies on (9)}
    
    \caseFact{15} $\Psi ; \Theta ; \Delta \vdash \Phi_1[I/i] \implies \tau_1[I/i] \subtynf \Phi_2[J/i] \implies \tau_2[I/i] \gens \Phi' \wedge (\Phi_2[J/i] \to \Phi_1[I/i])$
    
    \caseText{By (11) and (12), the presuppositions for (15) hold}
    
    \caseFact{16} $\Psi ; \Theta ; \Delta \pvdash \Phi_1[I/i] \implies \tau_1[I/i] \subtynf \Phi_2[J/i] \implies \tau_2[I/i] \gens \Phi' \wedge (\Phi_2[J/i] \to \Phi_1[I/i])$
    
    \caseText{By (2)}
    
    \caseFact{17} $\Theta, i : S ; \Delta \vDash \Phi_2 \to \Phi_1$
    
    \caseText{By (17) and (6)}
    
    \caseFact{18} $\Theta ; \Delta \vDash \Phi_2[J/i] \to \Phi_1[I/i]$
    
    \caseText{The result follows by (16) and (18), with $\Phi' = \Phi' \wedge (\Phi_2[J/i] \to \Phi_1[I/i])$}    
  
  }
  
  \jcase{13}{AS-Monad}{
    \jgivengoal{
      \caseFact{7} $\Psi ; \Theta, i : S ; \Delta \pvdash \M(M,\vec{q}) \tau_1 \subtynf \M(N,\vec{p}) \tau_2 : \star \gens (M = N) \wedge (\vec{q} \leq \vec{p}) \wedge \Phi$
      
      \caseFact{8} $\Psi ; \Theta, i : S ; \Delta \vdash \tau_1 \subtynf \tau_2 : \star \gens \Phi$
      
    }{
       $\Psi ; \Theta ; \Delta \vdash (\M(M,\vec{q}) \tau_1) [I/i] \subtynf (\M(N,\vec{p}) \tau_2)[J/i] : \star \gens \Phi'$ and $\Theta ; \Delta \vDash \Phi'$
    }
   
   \caseText{From (2)}
   
   \caseFact{9} $\Theta, i : S ; \Delta \vDash M = N$
   
   \caseFact{10} $\Theta, i : S ; \Delta \vDash \vec{q} \leq \vec{p}$
   
   \caseText{By IH on (8)}
   
   \caseFact{11} $\Psi ; \Theta ; \Delta \pvdash \tau_1[I/i] \subtynf \tau_2[J/i] : \star \gens \Phi'$
   
   \caseFact{12} $\Theta ; \Delta \vDash \Phi'$
   
   \caseText{By AS-Monad on (11) and Theorem~\ref{thm:idx-idx-subst} on the presuppositions of (1) for $M,N,\vec{q},\vec{p}$}
   
   \caseFact{13} $\Psi ; \Theta ; \Delta \pvdash \M(M[I/i],\vec{q}[I/i]) \; (\tau_1[I/i]) \subtynf \M(N[J/i],\vec{q}[J/i]) \;(\tau_2[J/i]) : \star \gens (M[I/i] = N[J/i]) \wedge (\vec{q}[I/i] \leq \vec{p}[J/i]) \wedge \Phi'$
   
   \caseText{By (6), (9), and (10)}
   
   \caseFact{14} $\Theta ; \Delta \vDash M[I/i] = N[J/i]$
   
   \caseFact{15} $\Theta ; \Delta \vDash \vec{q}[I/i] \leq \vec{p}[J/i]$
   
   \caseText{The Goal is immediate from (12), (13), (14), (15)}
  }
  
  \jcase{13}{AS-Pot}{
    \jgivengoal{
      \caseFact{7} $\Psi ; \Theta, i : S ; \Delta \pvdash [M|\vec{q}] \tau_1 \subtynf [N|\vec{p}] \tau_2 : \star \gens (M = N) \wedge (\vec{q} \geq \vec{p}) \wedge \Phi$
      
      \caseFact{8} $\Psi ; \Theta, i : S ; \Delta \vdash \tau_1 \subtynf \tau_2 : \star \gens \Phi$
      
    }{
       $\Psi ; \Theta ; \Delta \vdash ([M|\vec{q}] \tau_1) [I/i] \subtynf ([N|\vec{p}] \tau_2)[J/i] : \star \gens \Phi'$ and $\Theta ; \Delta \vDash \Phi'$
    }
   
   \caseText{From (2)}
   
   \caseFact{9} $\Theta, i : S ; \Delta \vDash M = N$
   
   \caseFact{10} $\Theta, i : S ; \Delta \vDash \vec{q} \geq \vec{p}$
   
   \caseText{By IH on (8)}
   
   \caseFact{11} $\Psi ; \Theta ; \Delta \pvdash \tau_1[I/i] \subtynf \tau_2[J/i] : \star \gens \Phi'$
   
   \caseFact{12} $\Theta ; \Delta \vDash \Phi'$
   
   \caseText{By AS-Pot on (11) and Theorem~\ref{thm:idx-idx-subst} on the presuppositions of (1) for $M,N,\vec{q},\vec{p}$}
   
   \caseFact{13} $\Psi ; \Theta ; \Delta \pvdash [M[I/i]|\vec{q}[I/i]] \; (\tau_1[I/i]) \subtynf [N[J/i]|\vec{q}[J/i]] \;(\tau_2[J/i]) : \star \gens (M[I/i] = N[J/i]) \wedge (\vec{q}[I/i] \geq \vec{p}[J/i]) \wedge \Phi'$
   
   \caseText{By (6), (9), and (10)}
   
   \caseFact{14} $\Theta ; \Delta \vDash M[I/i] = N[J/i]$
   
   \caseFact{15} $\Theta ; \Delta \vDash \vec{q}[I/i] \geq \vec{p}[J/i]$
   
   \caseText{The Goal is immediate from (12), (13), (14), (15)}
  }
  
  \jcase{14}{AS-ConstPot}{
     \jgivengoal{
       \caseFact{7} $\Psi ; \Theta, i : S ; \Delta \pvdash [M] \; \tau_1 \subtynf [N] \ ; \tau_2 : \star \gens \Phi \wedge (N \leq M)$
       
       \caseFact{8} $\Psi ; \Theta, i : S; \Delta \vdash \tau_1 \subtynf \tau_2 : \star \gens \Phi$     
     }{
       $\Psi ; \Theta ; \Delta ; \vdash ([M] \; \tau_1)[I/i] \subtynf ([N] \; \tau_2)[J/i] : \star \gens \Phi'$ and $\Theta ; \Delta \vDash \Phi'$     
     }
     
    \caseText{By (2)}
    
    \caseFact{9} $\Theta, i : S ; \Delta \vDash N \leq M$
    
    \caseText{By IH on (8)}
    
    \caseFact{10} $\Psi ; \Theta ; \Delta \pvdash \tau_1[I/i] \subtynf \tau_2[J/i] : \star \gens \Phi'$
    
    \caseFact{11} $\Theta ; \Delta \vDash \Phi'$
    
    \caseText{By AS-ConstPot and Theorem~\ref{thm:idx-idx-subst} for $M,N$}
    
    \caseFact{12} $\Psi ; \Theta ; \Delta \pvdash \left[M[I/i]\right] \; (\tau_1[I/i]) \subtynf \left[N[I/i]\right] \; (\tau_2[J/i]) : \star \gens \Phi' \wedge (N[J/i] \leq M[I/i])$
    
    \caseText{By (9) and (6)}
    
    \caseFact{13} $\Theta ; \Delta \vDash N[J/i] \leq M[I/i]$
    
    \caseText{Goal follows immediately from (11), (12), (13), with $\Phi' = \Phi' \wedge (N[J/i] \leq M[I/i])$}
  }
  
  \jcase{15}{AS-FamLam}{
  
    \jgivengoal{
      \caseFact{7} $\Psi ; \Theta, i : S ; \Delta \pvdash \lambda j : S'. \tau_1 \subtynf \lambda j : S'.\tau_2 : S \to K \gens \forall j : S. \Phi$
      
      \caseFact{8} $\Psi ; \Theta, i : S, j : S' ; \Delta \vdash \tau_1 \subtynf \tau_2 : K \gens \Phi$
    }{
      $\Psi ; \Theta ; \Delta \pvdash (\lambda j : S'.\tau_1)[I/i] \subtynf (\lambda j: S'.\tau_2)[J/i] : S' \to K \gens \Phi'$ and $\Theta ; \Delta \vDash \Phi'$    
    }
    
    \caseText{By IH on (8)}
    
    \caseFact{9} $\Psi ; \Theta, j : S' ; \Delta \pvdash \tau_1[I/i] \subtynf \tau_2[J/i] : K \gens \Phi'$
    
    \caseFact{10} $\Theta, j : S' ; \Delta \vDash \Phi'$
    
    \caseText{By AT-FamLam on (9)}
    
    \caseFact{11} $\Psi ; \Theta ; \Delta \pvdash \forall j : S'.\tau_1[I/i] \subtynf \forall j :S'.\tau_2[J/i] : K \gens \forall j : S'.\Phi'$
  
    \caseText{The goal is immediate by (10) and (11)} 
  }
  
  \jcase{15}{AS-FamApp}{
    \jgivengoal{
       \caseFact{7} $\Psi ; \Theta, i : S; \Delta \pvdash \tau_1 \; M \subtynf \tau_2 \; N : K \gens \Phi \wedge (M = N)$
       
       \caseFact{8} $\Psi ; \Theta, i : S ; \Delta \vdash \tau_1 \subtynf \tau_2 : S' \to K \gens \Phi$
    }{
      $\Psi ; \Theta ; \Delta \pvdash \left(\tau_1 \; M\right)[I/i] \subtynf \left(\tau_2[J/i]\right)[J/i] : K \gens \Phi'$ and $\Theta ; \Delta \vDash \Phi'$
    }
    
    \caseText{By (2)}
    
    \caseFact{9} $\Theta, i : S ; \Delta \vDash M = N$
    
    \caseText{By (9) and (6)}
    
    \caseFact{10}  $\Theta ; \Delta \vDash M[I/i] = N[J/i]$
    
    \caseText{IH on (8)}
    
    \caseFact{11} $\Psi ; \Theta ; \Delta \vdash \tau_1[I/i] \subtynf \tau_2[J/i] : S' \to K \gens \Phi'$
    
    \caseFact{12} $\Theta ; \Delta \vDash \Phi'$
    
    \caseText{By AS-FamApp on (11) and Theorem~\ref{thm:idx-idx-algo-subst} on the proofs that $M,N : S'$ in (7)}
    
    \caseFact{13} $\Psi ; \Theta ; \Delta \pvdash \left(\tau_1[I/i]\right) \; M[I/i] \subtynf \left(\tau_2[J/i]\right) \; N[J/i] : K \gens \Phi' \wedge (M[I/i] = N[J/i])$
    
    \caseText{The Goal follows by (10), (12), (13)}
    
  }

}

\evalapplemma*
\begin{proof}
By inversion on $\Psi ; \Theta ; \Delta \vdash \texttt{eval}(\tau_1) \subtynf \texttt{eval}(\tau_2) : S \to K \gens \Phi$.
\begin{itemize}
  \item For the first case, suppose the derivation was $\Psi ; \Theta ; \Delta \vdash \lambda i : S. \tau_1' \subtynf \tau_2' : S \to K \gens \Phi$
  from $\Psi ; \Theta, i : S ; \Delta \vdash \tau_1' \subtynf \tau_2' : K \gens \Phi'$. By Theorem~\ref{thm:subtynf-idx-subst},
  $\Psi ; \Theta ; \Delta \pvdash \tau_1'[I/i] \subtynf \tau_2'[J/i] : K \gens \Phi'$, for some $\Theta ; \Delta \vDash \Phi'$. But $\texttt{eval}(\tau_1 \; I) = \tau_1'[I/i]$ and $\texttt{eval}(\tau_2 \; J) = \tau_2'[J/i]$.
  \item Now, suppose the derivation was $\Psi ; \Theta ; \Delta \vdash \tau_1' \; L_1 \subtynf \tau_2' \; L_2 : S \to K \gens \Phi \wedge (L_1 = L_2)$, where $\texttt{eval}(\tau_1) = \tau_1' \; L_1$ and $\texttt{eval}(\tau_2) = \tau_2 \; L_2$. These must both be $\texttt{ne}$, since they are both applications, and therefore
  $\texttt{eval}(\tau_1) \; I = \texttt{eval}(\tau_1 \; I)$ and $\texttt{eval}(\tau_2) \; J = \texttt{eval}(\tau_2 \; J)$, as required.
\end{itemize}
\end{proof}

\subtycompl*
\begin{proof}
By induction on $\Psi ; \Theta ; \Delta \vdash \tau_1 \subty \tau_2 : K$.
 \begin{itemize}
   \item[(S-Refl)] Immediate by \autoref{thm:subty-refl}.
   \item[(S-Trans)] Suppose $\Psi ; \Theta ; \Delta \vdash \tau_1 \subty \tau_3 : K$ by way of $\Psi ; \Theta ; \Delta \vdash \tau_1 \subty \tau_2 : K$ and $\Psi ; \Theta ; \Delta \vdash \tau_2 \subty \tau_3 : K$. By IH, $\Psi ; \Theta ; \Delta \vdash \tau_1 \subty \tau_2 : K \gens \Phi_1$ and $\Psi ; \Theta ; \Delta \vdash \tau_2 \subty \tau_3 : K \gens \Phi_2$ with $\Theta ; \Delta \vDash \Phi_1 \wedge \Phi_2$. But by \autoref{thm:subty-trans}, we have
   $\Psi ; \Theta ; \Delta \vdash \tau_1 \subty \tau_3 : K \gens \Phi$ and $\Theta ; \Delta \vDash \Phi$, as required.
   \item[(S-Arr)] Suppose $\Psi ; \Theta ; \Delta \vdash \tau_1 \loli \tau_2 \subty \tau_1' \loli \tau_2' : \star$
   from $\Psi ; \Theta ; \Delta \vdash \tau_1' \subty \tau_1 : \star$ and
   $\Psi ; \Theta ; \Delta \vdash \tau_2 \subty \tau_2' : \star$.
   By IH, $\Psi ; \Theta ; \Delta \vdash \tau_1' \subty \tau_1 : \star \gens \Phi_1$,
   and $\Psi ; \Theta ; \Delta \vdash \tau_2 \subty \tau_2' : \star \gens \Phi_2$.
   By inversion, we have $\Psi ; \Theta ; \Delta \vdash \texttt{eval}(\tau_1') \subtynf \texttt{eval}(\tau_1) : \star \gens \Phi_1$
   $\Psi ; \Theta ; \Delta \vdash \texttt{eval}(\tau_2) \subtynf \texttt{eval}(\tau_2') : \star \gens \Phi_2$.
   By AS-Arr,
   $\Psi ; \Theta ; \Delta \vdash \texttt{eval}(\tau_1) \loli \texttt{eval}(\tau_2) \subtynf \texttt{eval}(\tau_1') \loli \texttt{eval}(\tau_2') : \star \gens \Phi_1 \wedge \Phi_2$. But $\texttt{eval}(\tau_1 \loli \tau_2) = \texttt{eval}(\tau_1) \loli \texttt{eval}(\tau_2)$, and so by AS-Normalize, we have
   $\Psi ; \Theta ; \Delta \vdash \tau_1 \loli \tau_2 \subty \tau_1' \loli \tau_2' : \star \gens \Phi_1 \wedge \Phi_2$, as required.
   
   \item[(S-Tensor)] Suppose $\Psi ; \Theta ; \Delta \vdash \tau_1 \otimes \tau_2 \subty \tau_1' \otimes \tau_2' : \star$ from $\Psi ; \Theta ; \Delta \vdash \tau_1 \subty \tau_1' : \star$ and $\Psi ; \Theta ; \Delta \vdash \tau_2 \subty \tau_2' : \star$. By IH, $\Psi ; \Theta ; \Delta \vdash \tau_1 \subty \tau_1' : \star \gens \Phi_1$ and $\Psi ; \Theta ; \Delta \vdash \tau_2 \subty \tau_2' : \star \gens \Phi_2$ with $\Theta ; \Delta \vDash \Phi_1 \wedge \Phi_2$. Inverting both these, we have that $\Psi ; \Theta ; \Delta \vdash \texttt{eval}(\tau_1) \subtynf \texttt{eval}(\tau_1') : \star \gens \Phi_1$ and $\Psi ; \Theta ; \Delta \vdash \texttt{eval}(\tau_2) \subtynf \texttt{eval}(\tau_1') : \star \gens \Phi_2$. By AS-Tensor, $\Psi ; \Theta ; \Delta \vdash \texttt{eval}(\tau_1) \otimes \texttt{eval}(\tau_1') \subtynf \texttt{eval}(\tau_2) \otimes \texttt{eval}(\tau_2') \star \gens \Phi_1 \wedge \Phi_2$. But $\texttt{eval}(\tau \otimes \sigma) = \texttt{eval}(\tau)\otimes\texttt{eval}(\sigma)$ by definition, and so we have by AS-Normalize $\Psi ; \Theta ; \Delta \vdash \tau_1 \otimes \tau_2 \subty \tau_1' \otimes \tau_2' : \star \gens \Phi_1 \wedge \Phi_2$, as required.
   \item[(S-With)] Suppose $\Psi ; \Theta ; \Delta \vdash \tau_1 \amp \tau_2 \subty \tau_1' \amp \tau_2' : \star$ from $\Psi ; \Theta ; \Delta \vdash \tau_1 \subty \tau_1' : \star$ and $\Psi ; \Theta ; \Delta \vdash \tau_2 \subty \tau_2' : \star$. By IH, $\Psi ; \Theta ; \Delta \vdash \tau_1 \subty \tau_1' : \star \gens \Phi_1$ and $\Psi ; \Theta ; \Delta \vdash \tau_2 \subty \tau_2' : \star \gens \Phi_2$ with $\Theta ; \Delta \vDash \Phi_1 \wedge \Phi_2$. Inverting both these, we have that $\Psi ; \Theta ; \Delta \vdash \texttt{eval}(\tau_1) \subtynf \texttt{eval}(\tau_1') : \star \gens \Phi_1$ and $\Psi ; \Theta ; \Delta \vdash \texttt{eval}(\tau_2) \subtynf \texttt{eval}(\tau_1') : \star \gens \Phi_2$. By AS-With, $\Psi ; \Theta ; \Delta \vdash \texttt{eval}(\tau_1) \amp \texttt{eval}(\tau_1') \subtynf \texttt{eval}(\tau_2) \amp \texttt{eval}(\tau_2') \star \gens \Phi_1 \wedge \Phi_2$. But $\texttt{eval}(\tau \amp \sigma) = \texttt{eval}(\tau)\amp\texttt{eval}(\sigma)$ by definition, and so we have by AS-Normalize $\Psi ; \Theta ; \Delta \vdash \tau_1 \amp \tau_2 \subty \tau_1' \amp \tau_2' : \star \gens \Phi_1 \wedge \Phi_2$, as required.
   \item[(S-Sum)] Suppose $\Psi ; \Theta ; \Delta \vdash \tau_1 \oplus \tau_2 \subty \tau_1' \oplus \tau_2' : \star$ from $\Psi ; \Theta ; \Delta \vdash \tau_1 \subty \tau_1' : \star$ and $\Psi ; \Theta ; \Delta \vdash \tau_2 \subty \tau_2' : \star$. By IH, $\Psi ; \Theta ; \Delta \vdash \tau_1 \subty \tau_1' : \star \gens \Phi_1$ and $\Psi ; \Theta ; \Delta \vdash \tau_2 \subty \tau_2' : \star \gens \Phi_2$ with $\Theta ; \Delta \vDash \Phi_1 \wedge \Phi_2$. Inverting both these, we have that $\Psi ; \Theta ; \Delta \vdash \texttt{eval}(\tau_1) \subtynf \texttt{eval}(\tau_1') : \star \gens \Phi_1$ and $\Psi ; \Theta ; \Delta \vdash \texttt{eval}(\tau_2) \subtynf \texttt{eval}(\tau_1') : \star \gens \Phi_2$. By AS-Sum, $\Psi ; \Theta ; \Delta \vdash \texttt{eval}(\tau_1) \oplus \texttt{eval}(\tau_1') \subtynf \texttt{eval}(\tau_2) \oplus \texttt{eval}(\tau_2') \star \gens \Phi_1 \wedge \Phi_2$. But $\texttt{eval}(\tau \oplus \sigma) = \texttt{eval}(\tau)\oplus\texttt{eval}(\sigma)$ by definition, and so we have by AS-Normalize $\Psi ; \Theta ; \Delta \vdash \tau_1 \oplus \tau_2 \subty \tau_1' \oplus \tau_2' : \star \gens \Phi_1 \wedge \Phi_2$, as required.
   \item[(S-Bang)] Suppose $\Psi ; \Theta ; \Delta \vdash !\tau_1 \subty !\tau_2 : \star$ from $\Psi ; \Theta ; \Delta \vdash \tau_1 \subty \tau_2 : \star$.
   By IH, $\Psi ; \Theta ; \Delta \vdash \tau_1 \subty \tau_2 : \star \gens \Phi$ with $\Theta ; \Delta \vDash \Phi$. By inversion, 
   $\Psi ; \Theta ; \Delta \vdash \texttt{eval}(\tau_1) \subtynf \texttt{eval}(\tau_2) : \star \gens \Phi$. By AS-Bang, 
   $\Psi ; \Theta ; \Delta \vdash !\texttt{eval}(\tau_1) \subtynf !\texttt{eval}(\tau_2) : \star \gens \Phi$. But $\texttt{eval}(!\tau) = !\texttt{eval}(\tau)$, and so
    $\Psi ; \Theta ; \Delta \vdash !\tau_1 \subty !\tau_2 : \star \gens \Phi$, as required.
   \item[(S-IForall)] Suppose $\Psi ; \Theta ; \Delta \vdash \forall i : S. \tau_1 \subty \forall i : S. \tau_2 : \star$ from $\Psi ; \Theta, i : S ; \Delta \vdash \tau_1 \subty \tau_2 : \star$. By IH, $\Psi ; \Theta, i : S ; \Delta \vdash \tau_1 \subty \tau_2 : \star \gens \Phi$ with $\Theta, i : S ; \Delta \vDash \Phi$. Equivalently, $\Theta ; \Delta \vDash \forall i : S. \Phi$. By inversion, $\Psi ; \Theta, i : S ; \Delta \vdash \texttt{eval}(\tau_1) \subtynf \texttt{eval}(\tau_2) : \star \gens \Phi$. by AS-IForall, $\Psi ; \Theta ; \Delta \vdash \forall i : S. \texttt{eval}(\tau_1) \subtynf \forall i : S. \texttt{eval}(\tau_2) : \star \gens \forall i : S. \Phi$. But, $\texttt{eval}(\forall i : S. \tau) = \forall i : S.\texttt{eval}(\tau)$, and so by AS-Normalize, $\Psi ; \Theta ; \Delta \vdash \forall i : S. \tau_1 \subty \forall i : S. \tau_2 : \star \gens \forall i : S. \Phi$, as required.
   \item[(S-IExists)] Suppose $\Psi ; \Theta ; \Delta \vdash \exists i : S. \tau_1 \subty \exists i : S. \tau_2 : \star$ from $\Psi ; \Theta, i : S ; \Delta \vdash \tau_1 \subty \tau_2 : \star$. By IH, $\Psi ; \Theta, i : S ; \Delta \vdash \tau_1 \subty \tau_2 : \star \gens \Phi$ with $\Theta, i : S ; \Delta \vDash \Phi$. Equivalently, $\Theta ; \Delta \vDash \forall i : S. \Phi$. By inversion, $\Psi ; \Theta, i : S ; \Delta \vdash \texttt{eval}(\tau_1) \subtynf \texttt{eval}(\tau_2) : \star \gens \Phi$. by AS-IExists, $\Psi ; \Theta ; \Delta \vdash \exists i : S. \texttt{eval}(\tau_1) \subtynf \exists i : S. \texttt{eval}(\tau_2) : \star \gens \forall i : S. \Phi$. But, $\texttt{eval}(\exists i : S. \tau) = \exists i : S.\texttt{eval}(\tau)$, and so by AS-Normalize, $\Psi ; \Theta ; \Delta \vdash \exists i : S. \tau_1 \subty \exists i : S. \tau_2 : \star \gens \forall i : S. \Phi$, as required.
   \item[(S-TForall)] Suppose $\Psi ; \Theta ; \Delta \vdash \forall \alpha : K. \tau_1 \subty \forall \alpha : K. \tau_2 : \star$ from 
   $\Psi, \alpha : K ; \Theta ; \Delta \vdash \tau_1 \subty \tau_2 : \star$. By IH,
   $\Psi, \alpha : K ; \Theta ; \Delta \vdash \tau_1 \subty \tau_2 : \star \gens \Phi$, with $\Theta ; \Delta \vDash \Phi$. By inversion,
   $\Psi, \alpha : K ; \Theta ; \Delta \vdash \texttt{eval}(\tau_1) \subtynf \texttt{eval}(\tau_2) : \star \gens \Phi$. By AS-TForall,
   $\Psi ; \Theta ; \Delta \vdash \forall \alpha : K. \texttt{eval}(\tau_1) \subtynf \forall \alpha : K. \texttt{eval}(\tau_2) : \star \gens \Phi$.
   But, $\texttt{eval}(\forall \alpha : K.\tau) = \forall \alpha : K. \texttt{eval}(\tau)$, and so
   $\Psi ; \Theta ; \Delta \vdash \forall \alpha : K. \tau_1 \subty \forall \alpha : K. \tau_2 : \star \gens \Phi$ by AS-Normalize, as required.
   \item[(S-List)] Suppose $\Psi ; \Theta ; \Delta \vdash L^I \tau_1 \subty L^J \tau_2 : \star$ from
   $\Psi ; \Theta ; \Delta \vdash \tau_1 \subty \tau_2 : \star$ and $\Theta ; \Delta \vDash I = J$.
   By IH, $\Psi ; \Theta ; \Delta \vdash \tau_1 \subty \tau_2 : \star \gens \Phi$ with $\Theta ; \Delta \vDash \Phi$. Consequently, $\Theta l \Delta \vDash \Phi \wedge (I = J)$. By inversion, $\Psi ; \Theta ; \Delta \vdash \texttt{eval}(\tau_1) \subtynf \texttt{eval}(\tau_2) : \star \gens \Phi$. By AS-List,
   $\Psi ; \Theta ; \Delta \vdash L^I\texttt{eval}(\tau_1) \subtynf L^J\texttt{eval}(\tau_2) : \star \gens \Phi \wedge (I = J)$. But, $\texttt{eval}(L^I \tau) = L^I \texttt{eval}(\tau)$, and so by AS-Normalize $\Psi ; \Theta ; \Delta \vdash L^I \tau_1 \subty L^J \tau_2 : \star \gens \Phi \wedge (I = J)$, as required.
   \item[(S-Impl)] Suppose $\Psi ; \Theta ; \Delta \vdash \Phi_1 \implies \tau_1 \subty \Phi_2 \implies \tau_2 : \star$
   from $\Psi ; \Theta ; \Delta \vdash \tau_1 \subty \tau_2 : \star$ and $\Theta;\Delta \vDash \Phi_2 \to \Phi_1$. By IH,
   $\Psi ; \Theta ; \Delta \vdash \tau_1 \subty \tau_2 : \star \gens \Phi$ with $\Theta ; \Delta \vDash \Phi$. Consequently, $\Theta ; \Delta \vDash \phi \wedge (\Phi_2 \to \Phi_1)$. By inversion, $\Psi ; \Theta ; \Delta \vdash \texttt{eval}(\tau_1) \subtynf \texttt{eval}(\tau_2) : \star \gens \Phi$. By AS-Impl,
   $\Psi ; \Theta ; \Delta \vdash \Phi_1 \implies \texttt{eval}(\tau_1) \subtynf \Phi_2 \implies \texttt{eval}(\tau_2) : \star \gens \Phi \wedge (\Phi_2 \to \Phi_1)$. But, $\texttt{eval}(\Phi \implies \tau) = \Phi \implies \texttt{eval}(\tau)$, and so by AS-Normalize,
   $\Psi ; \Theta ; \Delta \vdash \Phi_1 \implies \tau_1 \subty \Phi_2 \implies \tau_2 : \star \gens \Phi \wedge (\Phi_2 \to \Phi_1)$, as required.
   \item[(S-Conj)]  Suppose $\Psi ; \Theta ; \Delta \vdash \Phi_1 \amp \tau_1 \subty \Phi_2 \amp \tau_2 : \star$
   from $\Psi ; \Theta ; \Delta \vdash \tau_1 \subty \tau_2 : \star$ and $\Theta;\Delta \vDash \Phi_1 \to \Phi_2$. By IH,
   $\Psi ; \Theta ; \Delta \vdash \tau_1 \subty \tau_2 : \star \gens \Phi$ with $\Theta ; \Delta \vDash \Phi$. Consequently, $\Theta ; \Delta \vDash \phi \wedge (\Phi_1 \to \Phi_2)$. By inversion, $\Psi ; \Theta ; \Delta \vdash \texttt{eval}(\tau_1) \subtynf \texttt{eval}(\tau_2) : \star \gens \Phi$. By AS-Conj,
   $\Psi ; \Theta ; \Delta \vdash \Phi_1 \amp \texttt{eval}(\tau_1) \subtynf \Phi_2 \amp \texttt{eval}(\tau_2) : \star \gens \Phi \wedge (\Phi_1 \to \Phi_2)$. But, $\texttt{eval}(\Phi \implies \tau) = \Phi \implies \texttt{eval}(\tau)$, and so by AS-Normalize,
   $\Psi ; \Theta ; \Delta \vdash \Phi_1 \amp \tau_1 \subty \Phi_2 \amp \tau_2 : \star \gens \Phi \wedge (\Phi_1 \to \Phi_2)$, as required.
   \item[(S-Monad)] Suppose $\Psi ; \Theta ; \Delta \vdash \M(I,\vec{q}) \tau_1 \subty \M(J,\vec{p}) \tau_2 : \star$ from $\Psi ; \Theta ; \Delta \vdash \tau_1 \subty \tau_2 : \star$ and $\Theta ; \Delta \vDash (I = J) \wedge (\vec{q} \leq \vec{p})$. By IH, $\Psi ; \Theta ; \Delta \vdash \tau_1 \subty \tau_2 : \star \gens \Phi$ with $\Theta ; \Delta \vDash \Phi$. Consequently, $\Theta ; \Delta \vDash \Phi \wedge (I = J) \wedge (\vec{q} \leq \vec{p})$. By inversion,
   $\Psi ; \Theta ; \Delta \vdash \texttt{eval}(\tau_1) \subtynf \texttt{eval}(\tau_2) : \star \gens \Phi$. By AS-Monad,
   $\Psi ; \Theta ; \Delta \vdash \M(I,\vec{q})\texttt{eval}(\tau_1) \subtynf \M(J,\vec{p})\texttt{eval}(\tau_2) : \star \gens \Phi \wedge (I = J) \wedge (\vec{q} \leq \vec{p})$. But, $\texttt{eval}(M(I,\vec{q})\tau)=\M(I,\vec{q})\texttt{eval}(\tau)$, and so by AS-Normalize.
   $\Psi ; \Theta ; \Delta \vdash \M(I,\vec{q}) \tau_1 \subty \M(J,\vec{p}) \tau_2 : \star \gens \Phi \wedge (I = J) \wedge (\vec{q} \leq \vec{p})$, as required.
   \item[(S-Pot)] Suppose $\Psi ; \Theta ; \Delta \vdash [I|\vec{q}] \tau_1 \subty [J|\vec{p}] \tau_2 : \star$ from $\Psi ; \Theta ; \Delta \vdash \tau_1 \subty \tau_2 : \star$ and $\Theta ; \Delta \vDash (I = J) \wedge (\vec{q} \geq \vec{p})$. By IH, $\Psi ; \Theta ; \Delta \vdash \tau_1 \subty \tau_2 : \star \gens \Phi$ with $\Theta ; \Delta \vDash \Phi$. Consequently, $\Theta ; \Delta \vDash \Phi \wedge (I = J) \wedge (\vec{q} \geq \vec{p})$. By inversion,
   $\Psi ; \Theta ; \Delta \vdash \texttt{eval}(\tau_1) \subtynf \texttt{eval}(\tau_2) : \star \gens \Phi$. By AS-Pot,
   $\Psi ; \Theta ; \Delta \vdash [I|\vec{q}]\texttt{eval}(\tau_1) \subtynf [J|\vec{p}]\texttt{eval}(\tau_2) : \star \gens \Phi \wedge (I = J) \wedge (\vec{q} \geq \vec{p})$. But, $\texttt{eval}([I|\vec{q}]\tau)=[I,\vec{q}]\texttt{eval}(\tau)$, and so by AS-Normalize.
   $\Psi ; \Theta ; \Delta \vdash [I|\vec{q}] \tau_1 \subty [J|\vec{p}] \tau_2 : \star \gens \Phi \wedge (I = J) \wedge (\vec{q} \geq \vec{p})$, as required.
   \item[(S-ConstPot)] Suppose $\Psi ; \Theta ; \Delta \vdash [I] \tau_1 \subty [J] \tau_2 : \star$
   from $\Psi ; \Theta ; \Delta \vdash \tau_1 \subty \tau_2 : \star$ and $\Theta;\Delta \vDash J \leq I$.
   By IH, $\Psi ; \Theta ; \Delta \vdash \tau_1 \subty \tau_2 : \star \gens \Phi$ with $\Theta ; \Delta \vDash I \leq J$. Consequently, $\Theta ; \Delta \vDash \Phi \wedge (I \leq J)$. By inversion, $\Psi ; \Theta ; \Delta \vdash \texttt{eval}(\tau_1) \subtynf \texttt{eval}(\tau_2) : \star \gens \Phi$. By AS-ConstPot,
   $\Psi ; \Theta ; \Delta \vdash [I]\texttt{eval}(\tau_1) \subtynf [J]\texttt{eval}(\tau_2) : \star \gens \Phi \wedge (J \leq I)$. But, $\texttt{eval}([I] \tau) = [I]\texttt{eval}(\tau)$, and so by AS-Normalize, $\Psi ; \Theta ; \Delta \vdash [I] \tau_1 \subty [J] \tau_2 : \star \gens \Phi \wedge (J \leq I)$, as required.
   \item[(S-FamLam)] Suppose $\Psi ; \Theta ; \Delta \vdash \lambda i : S. \tau_1 \subty \lambda i : S. \tau_2 : S \to K$
   from $\Psi ; \Theta, i : S ; \Delta \vdash \tau_1 \subty \tau_2 : K$. By IH, $\Psi ; \Theta, i : S ; \Delta \vdash \tau_1 \subty \tau_2 : K \gens \Phi$ with $\Theta, i : S; \Delta \vDash \Phi$, or $\Theta ; \Delta \vDash \forall i : S. \Phi$. By inversion, $\Psi ; \Theta, i : S ; \Delta \vdash \texttt{eval}(\tau_1) \subtynf \texttt{eval}(\tau_2) : K \gens \Phi$. By AS-FamLam, $\Psi ; \Theta ; \Delta \vdash \lambda i : S . \texttt{eval}(\tau_1) \subtynf \lambda i : S.\texttt{eval}(\tau_2) : S \to K \gens \forall i : S. \Phi$. But, $\texttt{eval}(\lambda i : S. \tau) = \lambda i : S. \texttt{eval}(\tau)$, and so by AS-Normalize,
   $\Psi ; \Theta ; \Delta \vdash \lambda i : S. \tau_1 \subty \lambda i : S. \tau_2 : S \to K \gens \forall i : S. \Phi$, as required.
   \item[(S-FamApp)] Suppose $\Psi ; \Theta ; \Delta \vdash \tau_1 \; I \subty \tau_2 \; J : K$ from $\Psi ; \Theta ; \Delta \vdash \tau_1 \subty \tau_2 : S \to K$ and$\Theta ; \Delta \vDash I = J$. By IH,  $\Psi ; \Theta ; \Delta \vdash \tau_1 \subty \tau_2 : S \to K \gens \Phi$ with $\Theta ; \Delta \vDash \Phi$. By inversion, $\Psi ; \Theta ; \Delta \vdash \texttt{eval}(\tau_1) \subtynf \texttt{eval}(\tau_2) : S \to K \gens \Phi$. By \autoref{thm:eval-app-lemma},
   $\Psi ; \Theta ; \Delta \vdash \texttt{eval}(\tau_1 \; I) \subtynf \texttt{eval}(\tau_2 \; J) : K \gens \Phi'$ with $\Theta ; \Delta \vDash \Phi'$.
   By AS-Normalize, $\Psi ; \Theta ; \Delta \vdash \tau_1 \; I \subty \tau_2 \; J : K \gens \Phi'$, as required.
   
   \item[(S-Fam-Beta1)] Suppose $\Psi ; \Theta ; \Delta \vdash (\lambda i : S. \tau) \; J \subty \tau[J/i] : K$. By AS-Normalize, it suffices to show that
   $\Psi ; \Theta ; \Delta \vdash \texttt{eval}((\lambda i : S. \tau) \; J) \subtynf \texttt{eval}(\tau[J/i]) : K \gens \Phi$ and $\Theta ; \Delta \vDash \Phi$.
   But, $\texttt{eval}((\lambda i : S. \tau) \; J) = \texttt{eval}(\tau)[J/i]$ by definition, and $\texttt{eval}(\tau[J/i]) = \texttt{eval}(\tau)[J/i]$ by \autoref{thm:idx-subst-eval}. By \autoref{thm:norm-thm},  $\Psi ; \Theta ; \Delta \vdash \texttt{eval}(\tau)[J/i] : K$, and so by \autoref{thm:subtynf-refl}, we have that $\Psi ; \Theta ; \Delta \vdash \texttt{eval}(\tau)[J/i] \subtynf \texttt{eval}(\tau)[J/i] : K \gens \Phi$ with $\Theta ; \Delta \vDash \Phi$ as required.
   \item[(S-Fam-Beta2)] Identical to S-FamBeta1.
 \end{itemize}
\end{proof}

\admitsweaken*
\begin{proof}
By a mutual induction on the premises of both cases. We will write this as a case analysis over the AT-* rules.
We will use \autoref{thm:ctx-sub-subset2} liberally, sometimes silently.
\begin{itemize}
  \item[(AT-Var-1)] Suppose $\Psi ; \Theta ; \Delta; \Omega ; \Gamma \vdash x \infers \tau \gens \top, \Gamma \setminus \{x : \tau\}$ from $x : \tau \in \Gamma$.
  By \autoref{thm:ctx-sub-subset2}, there is some $\tau'$ so that $\Psi ; \Theta ; \Delta \vdash \tau' \subty \tau : \star$, and $x : \tau' \in \Gamma'$.
  By AT-Var-1, $\Psi ; \Theta ; \Delta; \Omega' ; \Gamma' \vdash x \infers \tau' \gens \top, \Gamma' \setminus \{x : \tau'\}$.
  By \autoref{thm:subty-compl},  $\Psi ; \Theta ; \Delta \vdash \tau' \subty \tau : \star \gens \Phi$ with $\Theta ; \Delta \vDash \Phi$, and so by AT-Sub,
  $\Psi ; \Theta ; \Delta; \Omega' ; \Gamma' \vdash x \checks \tau \gens \top \wedge \Phi, \Gamma' \setminus \{x : \tau'\}$. Finally, $\Psi ; \Theta ; \Delta \vdash \Gamma' \setminus \{x : \tau'\} \wknto \Gamma' \setminus \Gamma$ and $\Psi ; \Theta ; \Delta \vdash \Gamma' \setminus \{x : \tau'\} \wknto \Gamma \setminus \{x : \tau\}$ since $x : \tau \in \Gamma$, and $\Psi ; \Theta ; \Delta \vdash \Gamma' \wknto \Gamma$, which proves (1). For (2), one use of AT-Anno gives $\Psi ; \Theta ; \Delta; \Omega' ; \Gamma' \vdash (x : \tau) \infers \tau \gens \top \wedge \Phi, \Gamma' \setminus \{x : \tau'\}$. But, $|(x : \tau)| = x$, and so we are done.
  
  \item[(AT-Var-2)] Suppose $\Psi ; \Theta ; \Delta ; \Omega ; \Gamma\vdash x \infers \tau \gens \top, \Gamma$ from $x : \tau \in \Omega$, with $\Psi ; \Theta ; \Delta \vdash \Omega' \wknto \Omega$ with $\Psi ; \Theta ; \Delta \vdash \Gamma' \wknto \Gamma$. By \autoref{thm:ctx-sub-subset2}, there is some $\tau'$ so that $x : \tau' \in \Omega'$ and $\Phi ; \Theta ; \Delta \vdash \tau' \subty \tau : \star$. By \autoref{thm:subty-compl}, $\Phi ; \Theta ; \Delta \vdash \tau' \subty \tau : \star \gens \Phi$ with $\Theta ; \Delta \vDash \Phi$. 
  By AT-Var-2, $\Psi ; \Theta ; \Delta ; \Omega' ; \Gamma' \vdash x \infers \tau' \gens \top, \Gamma'$. By AT-Sub, 
  $\Psi ; \Theta ; \Delta ; \Omega' ; \Gamma' \vdash x \checks \tau \gens \top \wedge \Phi, \Gamma'$, which proves (2). For (1), we apply AT-Anno and note that $|(x : \tau)| = x$.
  
  \item[(AT-Unit)] Immediate.
  \item[(AT-Base)] Immediate.
  \item[(AT-Absurd)] Immediate.
  \item[(AT-Nil)] Immediate.
  
  \item[(AT-Cons)] Suppose $\Psi ; \Theta ; \Delta ; \Omega ; \Gamma\vdash e_1 :: e_2 \checks L^I \tau \gens (I \geq 1) \wedge \Phi_1 \wedge \Phi_2, \Gamma_2$
  from $\Psi ; \Theta ; \Delta ; \Omega ; \Gamma\vdash e_1 \checks \tau \gens \Phi_1, \Gamma_1$ and 
       $\Psi ; \Theta ; \Delta ; \Omega ; \Gamma_1\vdash e_2 \checks L^{I-1} \tau \gens \Phi_2, \Gamma_2$ with 
       $\Theta ; \Delta \vDash (I \geq 1) \wedge \Phi_1 \wedge \Phi_2$, 
       $\Psi ; \Theta ; \Delta \vDash \Gamma' \wknto \Gamma$, and 
       $\Psi ; \Theta ; \Delta \vdash \Omega' \wknto \Omega$.
       By IH, there are $e_1'$, $\Phi_1'$, $\Gamma_1'$ such that $|e_1'| = |e_1|$, 
       $\Theta ; \Delta \vDash \Phi_1'$, 
       $\Psi ; \Theta ; \Delta \vdash \Gamma_1' \wknto \Gamma' \setminus \Gamma$, 
       $\Psi ; \Theta ; \Delta \vdash \Gamma_1' \wknto \Gamma_1$, and
       $\Psi ; \Theta ; \Delta ; \Omega' ; \Gamma' \vdash e_1' \checks \tau \gens \Phi_1',\Gamma_1'$.
       By IH, there are $e_2'$, $\Phi_2'$, $\Gamma_2'$ such that $|e_2'| = |e_2|$,
       $\Theta ; \Delta \vDash \Phi_2'$,
       $\Psi ; \Theta ; \Delta \vdash \Gamma_2' \wknto \Gamma_1' \setminus \Gamma_1$,
       $\Psi ; \Theta ; \Delta \vdash \Gamma_2' \wknto \Gamma_2$, and
       $\Psi ; \Theta ; \Delta ; \Omega' ; \Gamma_1' \vdash e_2' \checks L^{I-1} \tau \gens \Phi_2',\Gamma_2'$.
       By AT-Cons,
       $\Psi ; \Theta ; \Delta ; \Omega' ; \Gamma' \vdash e_1' :: e_2' \checks L^I \tau \gens (I \geq 1) \wedge \Phi_1' \wedge \Phi_2',\Gamma_2'$.
       Since $\Theta ; \Delta \vDash I \geq 1$, we have that $\Theta ; \Delta \vDash (I \geq 1) \wedge \Phi_1' \wedge \Phi_2'$. Further, $|e_1' :: e_2'| = |e_1'| :: |e_2'| = |e_1| :: |e_2| = |e_1 :: e_2|$. Finally, $\Psi ; \Theta ; \Delta \vdash \Gamma_2' \wknto \Gamma_2$ by the second IH, and $\Psi ; \Theta ; \Delta \vdash \Gamma_2' \wknto \Gamma' \setminus \Gamma$ by $\Psi ; \Theta ; \Delta \vdash \Gamma_2' \wknto \Gamma_1' \setminus \Gamma_1$ and $\Psi ; \Theta ; \Delta \vdash \Gamma_1' \wknto \Gamma' \setminus \Gamma$ using \autoref{thm:ctx-sub-subset2}, which completes (1). For (2), AT-Anno gives
       $\Psi ; \Theta ; \Delta ; \Omega' ; \Gamma' \vdash (e_1' :: e_2' : L^I) \infers L^I \tau \gens (I \geq 1) \wedge \Phi_1' \wedge \Phi_2',\Gamma_2'$, as required.
  
  \item[(AT-Match)] Suppose 
   $$\Psi ; \Theta ; \Delta ; \Omega ; \Gamma\vdash \texttt{match}(e,e_1,h.t.e_2) \checks \tau' \gens \Phi_1 \wedge (I = 0 \to \Phi_2) \wedge (I \geq 1 \to \Phi_3), \Gamma_2 \cap (\Gamma_3 \setminus \{h,t\}$$ from 
   $$\Psi ; \Theta ; \Delta ; \Omega ; \Gamma\vdash e \infers L^I \tau \gens \Phi_1, \Gamma_1$$,
   $$\Psi ; \Theta ; \Delta, I = 0 ; \Omega ; \Gamma_1\vdash e_1 \checks \tau' \gens \Phi_2,\Gamma_2$$
   $$\Psi ; \Theta ; \Delta, I \geq 1; \Omega ; \Gamma_1, h : \tau, t : L^{I-1} \tau \vdash e_2 \checks \tau' \gens \Phi_3,\Gamma_3$$ with 
   $$\Theta ; \Delta \vDash \Phi_1 \wedge (I = 0 \to \Phi_2) \wedge (I \geq 1 \to \Phi_3)$$
   $$\Psi ; \Theta ; \Delta \vdash \Omega' \wknto \Omega$$
   $$\Psi ; \Theta ; \Delta \vdash \Gamma' \wknto \Gamma$$
   By IH, there are $e'$, $\Phi_1'$, $\Gamma_1'$ with $|e'| = |e|$, 
   $\Theta ; \Delta \vDash \Phi_1'$,
   $\Psi ; \Theta ; \Delta \vdash \Gamma_1' \wknto \Gamma' \setminus \Gamma$,
   $\Psi ; \Theta ; \Delta \vdash \Gamma_1' \wknto \Gamma_1$, and
   $\Psi ; \Theta ; \Delta ; \Omega' ; \Gamma' \vdash e' \infers L^I \tau \gens \Phi_1',\Gamma_1'$.
   By IH, there are $e_1'$, $\Phi_2'$, $\Gamma_2'$ with $|e_1'| = |e_1|$,
   $\Theta ; \Delta, I = 0 \vDash \Phi_2'$,
   $\Psi ; \Theta ; \Delta, I = 0 \vdash \Gamma_2' \wknto \Gamma_1' \setminus \Gamma_1$,
   $\Psi ; \Theta ; \Delta, I = 0 \vdash \Gamma_2' \wknto \Gamma_2$, and
   $\Psi ; \Theta ; \Delta ; \Omega' ; \Gamma_1' \vdash e_1' \checks \tau' \gens \Phi_2',\Gamma_2'$.
   Since $\Psi ; \Theta ; \Delta \vdash \Gamma_1' \wknto \Gamma_1$, we have that
      $\Psi ; \Theta ; \Delta \vdash \Gamma_1', h : \tau, t : L^{I-1} \tau \wknto \Gamma_1, h : \tau, t : L^{I-1} \tau$,
   and so by IH, there are $e_2'$, $\Phi_3'$, $\Gamma_3'$ such that $|e_2'| = |e_2|$,
   $\Theta ; \Delta, I \geq 1 \vDash \Phi_3'$,
   $\Psi ; \Theta ; \Delta, I \geq 1 \vdash \Gamma_3' \wknto (\Gamma_1',h : \tau, t : L^{I-1} \tau) \setminus (\Gamma_1, h : \tau, t : L^{I-1} \tau)$,
   $\Psi ; \Theta ; \Delta, I \geq 1 \vdash \Gamma_3' \wknto \Gamma_3$, and
   $\Psi ; \Theta ; \Delta ; \Omega' ; \Gamma_1', h : \tau, t : L^{I-1} \tau \vdash e_2' \checks \tau' \gens \Phi_3',\Gamma_3'$.
   By AT-Match, $\Psi ; \Theta ; \Delta ; \Omega' ; \Gamma' \vdash \texttt{match}(e',e1',h.t.e_2') \checks \tau' \gens \Phi_1' \wedge (I = 0 \to \Phi_2') \wedge (I \geq 1 \to \Phi_3'),\Gamma_2' \cap (\Gamma_3' \setminus \{h,t\})$
   Firstly, we note that $|\texttt{match}(e',e1',h.t.e_2')| = \texttt{match}(|e'|,|e_1'|,h.t.|e_2'|) = \texttt{match}(|e|,|e_1|,h.t.|e_2|) = \texttt{match}(e,e_1,h.t.e_2)$. Then, since $\Theta ; \Delta \vDash I = 0 \to \Phi_2'$ and $\Theta ; \Delta \vDash I \geq 1 \to \Phi_3'$, we have $\Theta ; \Delta \vDash \Phi_1' \wedge (I = 0 \to \Phi_2') \wedge (I \geq 1 \to \Phi_3')$.
   Then, 
   $\Phi ; \Theta ; \Delta \vdash \Gamma_2' \cap (\Gamma_3' \setminus \{h,t\}) \wknto \Gamma' \setminus \Gamma$
   because of $\Psi ; \Theta ; \Delta \vdash \Gamma_1' \wknto \Gamma' \setminus \Gamma$,  $\Psi ; \Theta ; \Delta \vdash \Gamma_2' \wknto \Gamma_1' \setminus \Gamma_1$, and $\Psi ; \Theta ; \Delta, I \geq 1 \vdash \Gamma_3' \wknto \Gamma_1' \setminus \Gamma_1$, making heavy use of \autoref{thm:ctx-sub-subset2}.
   Finally, we have   
   $\Phi ; \Theta ; \Delta \vdash \Gamma_2' \cap (\Gamma_3' \setminus \{h,t\}) \wknto \Gamma_2 \cap (\Gamma_3 \setminus \{h,t\}$
   since $\Psi ; \Theta ; \Delta \vdash \Gamma_2' \wknto \Gamma_2$ and $\Psi ; \Theta ; \Delta \vdash \Gamma_3' \wknto \Gamma_3$. We may strengthen away the assumptions in $\Delta$ because of the presuppositions of well-formedness all contexts involved. This completes (1). For (2), we apply AT-Anno to get
   $\Psi ; \Theta ; \Delta ; \Omega' ; \Gamma' \vdash (\texttt{match}(e',e1',h.t.e_2') : \tau') \infers \tau' \gens \Phi_1' \wedge (I = 0 \to \Phi_2') \wedge (I \geq 1 \to \Phi_3'),\Gamma_2' \cap (\Gamma_3' \setminus \{h,t\})$, and we are done.
   
   \item[(AT-ExistI)] Suppose $\Psi ; \Theta ; \Delta ; \Omega ; \Gamma\vdash \texttt{pack}[I](e) \checks \exists i:S.\tau \gens \Phi_1 \wedge \Phi_2, \Gamma''$ from
   $\Theta ; \Delta \vdash I : S \gens \Phi_1$ and
   $\Psi ; \Theta ; \Delta ; \Omega ; \Gamma\vdash e \checks \tau[I/i] \gens \Phi_2,\Gamma''$,
   with $\Theta ; \Delta \vDash \Phi_1 \wedge \Phi_2$,
   $\Psi ; \Theta ; \Delta \vdash \Gamma' \wknto \Gamma$, and
   $\Psi ; \Theta ; \Delta \vdash \Omega' \wknto \Omega$.
   By IH, there are $e'$, $\Phi_2'$, $\Gamma'''$ such that
   $|e'| = |e|$,
   $\Theta ; \Delta \vDash \Phi_2'$,
   $\Psi ; \Theta ; \Delta \vdash \Gamma''' \wknto \Gamma' \setminus \Gamma$, and
   $\Psi ; \Theta ; \Delta \vdash \Gamma''' \wknto \Gamma''$, and
   $\Psi ; \Theta ; \Delta ; \Omega' ; \Gamma'\vdash e' \checks \tau[I/i] \gens \Phi_2',\Gamma'''$.
   By AT-ExistI, $\Psi ; \Theta ; \Delta ; \Omega' ; \Gamma'\vdash \texttt{pack}[I](e') \checks \exists i : S. \tau[I/i] \gens \Phi_1 \wedge \Phi_2',\Gamma'''$.
   Since $|\texttt{pack}[I](e')| = \texttt{pack}[I](|e'|) = \texttt{pack}[I](|e|) = |\texttt{pack}[I](e)|$ and $\Theta ; \Delta \vDash \Phi_1 \wedge \Phi_2'$, this completes (1). For (2), one application of AT-Anno suffices.
   
   \item[(AT-ExistE)]
   Suppose $\Psi ; \Theta ; \Delta ; \Omega ; \Gamma\vdash \texttt{unpack } (i,x) = e_1 \texttt{ in } e_2 \checks \tau' \gens \Phi_1 \wedge (\forall i : S. \Phi_2) , \Gamma_2 \setminus \{x : \tau\}$
   from 
   $\Psi ; \Theta ; \Delta ; \Omega ; \Gamma\vdash e_1 \infers \exists i : S.\tau \gens \Phi_1, \Gamma_1$
   and 
   $\Psi ; \Theta, i : S ; \Delta ; \Omega ; \Gamma_1, x : \tau \vdash e_2 \checks \tau' \gens \Phi_2, \Gamma_2$
   with $\Theta ; \Delta \vDash \Phi_1 \wedge \Phi_2$,
   $\Psi ; \Theta ; \Delta \vdash \Gamma' \wknto \Gamma$, and
   $\Psi ; \Theta ; \Delta \vdash \Omega' \wknto \Omega$.
   By IH, there are $e_1'$, $\Phi_1'$, $\Gamma_1'$ such that 
   $|e_1'| = |e_1|$,
   $\Theta ; \Delta \vDash \Phi_1'$,
   $\Psi ; \Theta ; \Delta \vdash \Gamma_1' \wknto \Gamma' \setminus \Gamma$,
   $\Psi ; \Theta ; \Delta \vdash \Gamma_1' \wknto \Gamma_1$, and
   $\Psi ; \Theta ; \Delta ; \Omega' ; \Gamma'\vdash e_1' \infers \exists i : S.\tau \gens \Phi_1', \Gamma_1'$.
   Since $\Theta ; \Delta \vDash \Phi_1 \wedge (\forall i : S. \Phi_2)$, we have $\Theta, i : S ; \Delta \vDash \Phi_2$. We also have $\Psi ; \Theta ; \Delta \vdash \Gamma_1', x : \tau \wknto \Gamma_1, x : \tau$. From these two facts we have
   by IH that there are $e_2'$, $\Phi_2'$, $\Gamma_2'$ such that
   $|e_2'| = |e_2|$,
   $\Theta, i : S ; \Delta \vDash \Phi_2'$,
   $\Psi ; \Theta, i : S ; \Delta \vdash \Gamma_2' \wknto (\Gamma_1', x : \tau) \setminus (\Gamma_1, x : \tau)$,
   $\Psi ; \Theta, i : S ; \Delta \vdash \Gamma_2' \wknto \Gamma_2$, and that
   $\Psi ; \Theta, i : S ; \Delta ; \Omega' ; \Gamma_1', x : \tau \vdash e_2' \checks \tau' \gens \Phi_2',\Gamma_2'$.
   By AT-ExistE, 
   $\Psi ; \Theta ; \Delta ; \Omega' ; \Gamma' \vdash \texttt{unpack } (i,x) = e_1' \texttt{ in } e_2' \checks \tau' \gens \Phi_1' \wedge (\forall i : S. \Phi_2'), \Gamma_2'$.
   Since $\Theta ; \Delta \vDash \Phi_1'$ and $\Theta, i : S; \Delta \vDash \Phi_2'$, we have $\Theta ; \Delta \vDash \Phi_1' \wedge (\forall i : S. \Phi_2')$.
   Next, we note that $|\texttt{unpack } (i,x) = e_1' \texttt{ in } e_2'| = \texttt{unpack } (i,x) = |e_1'| \texttt{ in } |e_2'| = \texttt{unpack } (i,x) = |e_1| \texttt{ in } |e_2| = |\texttt{unpack } (i,x) = e_1 \texttt{ in } e_2|$. $\Psi ; \Theta, i : S ; \Delta \vdash \Gamma_2' \wknto \Gamma_2$ is immediate from the second IH, and the fact that $\Psi ; \Theta ; \Delta \vdash \Gamma_2' \wknto \Gamma' \setminus \Gamma$ follows from \autoref{thm:ctx-sub-subset2}. This completes (1), and (2) follows immediately from AT-Anno.
   
   \item[(AT-Lam)] Suppose $\Psi ; \Theta ; \Delta ; \Omega ; \Gamma\vdash \lambda x.e \checks \tau_1 \loli \tau_2 \gens \Phi, \Gamma'' \setminus \{x : \tau_1\}$
  from $\Psi ; \Theta ; \Delta ; \Omega ; \Gamma, x : \tau_1 \vdash e \checks \tau_2, \gens \Phi, \Gamma''$, with $\Theta ; \Delta \vDash \Phi$, $\Psi ; \Theta ; \Delta \vdash \Gamma' \wknto \Gamma$, and $\Psi ; \Theta ; \Delta \vdash \Omega' \wknto \Omega$. Since $\Psi ; \Theta ; \Delta \vdash \Gamma',x:\tau_1 \wknto \Gamma, x: \tau_1$, by IH there are $\Phi'$, $e'$, $\Gamma'''$ so that
  $\Theta ; \Delta \vDash \Phi'$, $|e'| = |e|$, $\Psi ; \Theta ; \Delta \vdash \Gamma''' \wknto \Gamma \setminus \Gamma'$, and
  $\Psi ; \Theta ; \Delta \vdash \Gamma''' \wknto \Gamma''$, and $$\Psi ; \Theta ; \Delta ; \Omega' ; \Gamma',x : \tau_1 \vdash e' : \tau_2 \gens \Phi',\Gamma'''$$.
  By AT-Lam, $\Psi ; \Theta ; \Delta ; \Omega' ; \Gamma' \vdash \lambda x.e' : \tau_1 \loli \tau_2 \gens \Phi', \Gamma''' \setminus \{x : \tau_1\}$. Then, $|\lambda x.e'| = \lambda x. |e'| = \lambda x.|e| = |\lambda x.e|$. Finally, $\Psi ; \Theta ; \Delta \vdash \Gamma''' \setminus \{x : \tau_1\} \wknto \Gamma' \setminus \Gamma$
  and $\Psi ; \Theta ; \Delta \vdash \Gamma''' \setminus \{x : \tau_1\} \wknto \Gamma'' \setminus \{x : \tau_1\}$, which proves (1). For (2), one use of AT-Anno suffices.
  
  \item[(AT-App)] Suppose $\Psi ; \Theta ; \Delta ; \Omega ; \Gamma\vdash e_1 \, e_2 \infers  \tau_2 \gens \Phi_1 \wedge \Phi_2, \Gamma_2$ from
  $\Psi ; \Theta ; \Delta ; \Omega ; \Gamma\vdash e_1 \infers \tau_1 \loli \tau_2 \gens \Phi_1, \Gamma_1$ and
  $\Psi ; \Theta ; \Delta ; \Omega ; \Gamma_1\vdash e_2 \checks \tau_1 \gens \Phi_2, \Gamma_2$, with
  $\Theta ; \Delta \vDash \Phi_1 \wedge \Phi_2$,
  $\Psi ; \Theta ; \Delta \vdash \Omega' \wknto \Omega$, and
  $\Psi ; \Theta ; \Delta \vdash \Gamma' \wknto \Gamma$.
  By IH, there are $e_1'$, $\Phi_1'$, and $\Gamma_1'$ such that
  $|e_1'| = |e_1|$,
  $\Theta ; \Delta \vDash \Phi_1'$,
  $\Psi ; \Theta ; \Delta \vdash \Gamma_1' \wknto \Gamma' \setminus \Gamma$,
  $\Psi ; \Theta ; \Delta \vdash \Gamma_1' \wknto \Gamma_1$, and
  $\Psi ; \Theta ; \Delta ; \Omega' ; \Gamma' \vdash e_1' \infers \tau_1 \loli \tau_2 \gens \Phi_1',\Gamma_1'$.
  Since $\Psi ; \Theta ; \Delta \vdash \Gamma_1' \wknto \Gamma_1$,
  we have by IH that there are $e_2'$, $\Phi_2'$, $\Gamma_2'$ such that
  $|e_2'| = |e_2|$,
  $\Theta ; \Delta \vDash \Phi_2'$,
  $\Psi ; \Theta ; \Delta \vdash \Gamma_2' \wknto \Gamma_1' \setminus \Gamma_1$,
  $\Psi ; \Theta ; \Delta \vdash \Gamma_2' \wknto \Gamma_2$, and
  $\Psi ; \Theta ; \Delta ; \Omega' ; \Gamma_1' \vdash e_2' \checks \tau_1 \gens \Phi_2',\Gamma_2'$.
  By AT-App, we have
  $\Psi ; \Theta ; \Delta ; \Omega' ; \Gamma' \vdash e_1' \; e_2' \infers \tau_2 \gens \Phi_1' \wedge \Phi_2', \Gamma_2'$.
  This completes (2). For (1), we invoke \autoref{thm:subty-refl} to get that $\Psi ; \Theta ; \Delta \vdash \tau_2 \subty \tau_2 : \star \gens \Phi'$ with
  $\Theta ; \Delta \vDash \Phi'$. By AT-Sub, we have $\Psi ; \Theta ; \Delta ; \Omega' ; \Gamma' \vdash e_1' \; e_2' \infers \tau_2 \gens \Phi_1' \wedge \Phi_2' \wedge \Phi', \Gamma_2'$, completing (1).
  
  
  \item[(AT-TensorI)] Suppose $\Psi ; \Theta ; \Delta ; \Omega ; \Gamma\vdash \angles{e_1,e_2} \checks \tau_1 \otimes \tau_2 \gens \Phi_1 \wedge \Phi_2,\Gamma_2$
  from $\Psi ; \Theta ; \Delta ; \Omega ; \Gamma\vdash e_1 \checks \tau_1 \gens \Phi_1, \Gamma_1$ and $\Psi ; \Theta ; \Delta ; \Omega ; \Gamma_1\vdash e_2 \checks \tau_2 \gens \Phi_2, \Gamma_2$, with $\Theta ; \Delta \vDash \Phi_1 \wedge \Phi_2$, $\Psi ; \Theta ; \Delta \vdash \Gamma' \wknto \Gamma$, and $\Psi ; \Theta ; \Delta \vdash \Omega' \wknto \Omega$. By IH, there are $\Phi_1'$, $e_1'$, and $\Gamma_1'$ such that
  $\Theta ; \Delta \vDash \Phi_1'$,
  $|e_1'| = |e_1|$,
  $\Psi ; \Theta ; \Delta \vdash \Gamma_1' \wknto \Gamma' \setminus \Gamma$,
  $\Psi ; \Theta ; \Delta \vdash \Gamma_1' \wknto \Gamma_1$,
  $$\Psi ; \Theta ; \Delta ; \Omega' ; \Gamma' \vdash e_1' : \tau_1 \gens \Phi_1', \Gamma_1'$$.
  Since $\Psi ; \Theta ; \Delta \vdash \Gamma_1' \wknto \Gamma_1$, we also have by IH that there are $\Phi_2'$, $e_2'$, $\Gamma_2'$ such that
  $\Theta ; \Delta \vDash \Phi_2'$,
  $|e_2'| = |e_2|$,
  $\Psi ; \Theta ; \Delta \vdash \Gamma_2' \wknto \Gamma_1' \setminus \Gamma_1$, and
  $\Psi ; \Theta ; \Delta \vdash \Gamma_2' \wknto \Gamma_2$
  such that
  $$\Psi ; \Theta ; \Delta ; \Omega' ; \Gamma_1' \vdash e_2' : \tau_2 \gens \Phi_2',\Gamma_2'$$.
  By AT-TensorI, $\Psi ; \Theta ; \Delta ; \Omega' ; \Gamma' \vdash \angles{e_1',e_2'} : \tau_1 \otimes \tau_2 \gens (\Phi_1' \wedge \Phi_2'), \Gamma_2'$.
  We have that $\Theta ; \Delta \vDash \Phi_1' \wedge \Phi_2'$ and $|\angles{e_1',e_2'}| = \angles{|e_1'|,|e_2'|} = \angles{|e_1|,|e_2|} = |\angles{e_1,e_2}|$.
  Finally, $\Psi ; \Theta ; \Delta \vdash \Gamma_2' \wknto \Gamma' \setminus \Gamma$ since $\Psi ; \Theta ; \Delta \vdash \Gamma_2' \wknto \Gamma_1' \setminus \Gamma_1$ and $\Psi ; \Theta ; \Delta \vdash \Gamma_1' \wknto \Gamma' \setminus \Gamma$ by \autoref{thm:ctx-sub-subset2}, completing (1). For (2), a single use of
  AT-Anno gives  $\Psi ; \Theta ; \Delta ; \Omega' ; \Gamma' \vdash (\angles{e_1',e_2'} : \tau_1 \otimes \tau_2) : \tau_1 \otimes \tau_2 \infers (\Phi_1' \wedge \Phi_2'), \Gamma_2'$, as required.
  
  \item[(AT-TensorE)]
  Suppose $\Psi ; \Theta ; \Delta ; \Omega ; \Gamma\vdash \texttt{let } \angles{x,y} = e_1 \texttt{ in } e_2 \checks \tau' \gens \Phi_1 \wedge \Phi_2, \Gamma_2 \setminus \{x,y\}$ from
  $\Psi ; \Theta ; \Delta ; \Omega ; \Gamma\vdash e_1 \infers \tau_1 \otimes \tau_2 \gens \Phi_1, \Gamma_1$ and
  $\Psi ; \Theta ; \Delta ; \Omega ; \Gamma_1,x : \tau_1, y : \tau_2\vdash e_2 \checks \tau' \gens \Phi_2,\Gamma_2$, with
  $\Theta ; \Delta \vDash \Phi_1 \wedge \Phi_2$,
  $\Psi ; \Theta ; \Delta \vdash \Omega' \wknto \Omega$,
  $\Psi ; \Theta ; \Delta \vdash \Gamma' \wknto \Gamma$.
  By IH, there are $e_1'$, $\Phi_1'$, $\Gamma_1'$ such that
  $|e_1'| = |e_1|$,
  $\Theta ; \Delta \vDash \Phi_1'$,
  $\Psi ; \Theta ; \Delta \vdash \Gamma_1' \wknto \Gamma' \setminus \Gamma$,
  $\Psi ; \Theta ; \Delta \vdash \Gamma_1' \wknto \Gamma_1$, and 
  $\Psi ; \Theta ; \Delta ; \Omega' ; \Gamma' \vdash e_1' \infers \tau_1 \otimes \tau_2 \gens \Phi_1', \Gamma_1'$.
  Then, since $\Psi ; \Theta ; \Delta \vdash \Gamma_1', x : \tau_1, y : \tau_2 \wknto \Gamma_1, x : \tau_1, y : \tau_2$,
  we have by IH that there are $e_2'$, $\Phi_2'$, $\Gamma_2'$ such that
  $|e_2'| = |e_2|$,
  $\Theta ; \Delta \vDash \Phi_2'$,
  $\Psi ; \Theta ; \Delta \vdash \Gamma_2' \wknto (\Gamma_1', x : \tau_1, y : \tau_2) \setminus (\Gamma_1, x : \tau_1, y : \tau_2)$,
  $\Psi ; \Theta ; \Delta \vdash \Gamma_2' \wknto \Gamma_2$, and
  $\Psi ; \Theta ; \Delta ; \Omega' ; \Gamma_1', x : \tau_1, y : \tau_2 \vdash e_2' \checks \tau_2 \gens \Phi_2',\Gamma_2'$.
  By AT-TensorE,
  $\Psi ; \Theta ; \Delta ; \Omega' ; \Gamma'\vdash \texttt{let } \angles{x,y} = e_1' \texttt{ in } e_2' \checks \tau' \gens \Phi_1' \wedge \Phi_2', \Gamma_2' \setminus \{x,y\}$.
  Of course, $|\texttt{let } \angles{x,y} = e_1' \texttt{ in } e_2'| = |\texttt{let } \angles{x,y} = e_1 \texttt{ in } e_2|$ and $\Theta ; \Delta \vDash \Phi_1' \wedge \Phi_2'$ as usual, and $\Psi ; \Theta ; \Delta \vdash \Gamma_2' \wknto \Gamma_2$ is immediate by IH. The fact that $\Psi ; \Theta ; \Delta \vdash \Gamma_2' \wknto \Gamma' \setminus \Gamma$ follows from considering the weakening judgments from both IHs, and liberally applying \autoref{thm:ctx-sub-subset2}. This completes (1). For (2), we simply apply AT-Anno, and are done.
  
 
 \item[(AT-WithI)] Suppose $\Psi ; \Theta ; \Delta ; \Omega ; \Gamma \vdash (e_1,e_2) \checks \tau_1 \amp \tau_2 \gens \Phi_1 \wedge \Phi_2, \Gamma_1 \cap \Gamma_2$
 from $\Psi ; \Theta ; \Delta ; \Omega ; \Gamma \vdash e_1 \checks \tau_1 \gens \Phi_1, \Gamma_1$ and 
 $\Psi ; \Theta ; \Delta ; \Omega ; \Gamma \vdash e_2 \checks \tau_2 \gens \Phi_2, \Gamma_2$ with 
 $\Theta ; \Delta \vDash \Phi_1 \wedge \Phi_2$,
 $\Psi ; \Theta ; \Delta \vdash \Omega' \wknto \Omega$, and
 $\Psi ; \Theta ; \Delta \vdash \Gamma' \wknto \Gamma$.
 By IH, there are $e_1'$, $\Phi_1'$, $\Gamma_1'$ such that
 $|e_1'| = |e_1|$,
 $\Theta ; \Delta \vDash \Phi_1'$,
 $\Psi ; \Theta ; \Delta \vdash \Gamma_1' \wknto \Gamma' \setminus \Gamma$,
 $\Psi ; \Theta ; \Delta \vdash \Gamma_1' \wknto \Gamma_1$, and
 $\Psi ; \Theta ; \Delta ; \Omega' ; \Gamma' \vdash e_1' \checks \tau_1 \gens \Phi_1', \Gamma_1'$.
 Again by IH, there are $e_2'$, $\Phi_2'$, $\Gamma_2'$ such that
 $|e_2'| = |e_2|$,
 $\Theta ; \Delta \vDash \Phi_2'$,
 $\Psi ; \Theta ; \Delta \vdash \Gamma_2' \wknto \Gamma' \setminus \Gamma$,
 $\Psi ; \Theta ; \Delta \vdash \Gamma_2' \wknto \Gamma_2$, and
 $\Psi ; \Theta ; \Delta ; \Omega' ; \Gamma' \vdash e_2' \checks \tau_2 \gens \Phi_2', \Gamma_2'$.
 Then, by AT-WithI, 
 $\Psi ; \Theta ; \Delta ; \Omega' ; \Gamma' \vdash (e_1',e_2') \checks \tau_1 \amp \tau_2 \gens \Phi_1' \wedge \Phi_2', \Gamma_1' \cap \Gamma_2'$.
 Then, by \autoref{thm:ctx-sub-subset2}, we have that $\Psi ; \Theta ; \Delta \vdash \Gamma_1' \cap \Gamma_2' \wknto \Gamma_1 \cap \Gamma_2$,
 and that $\Psi ; \Theta ; \Delta \vdash \Gamma_1' \cap \Gamma_2' \wknto \Gamma' \setminus \Gamma$. This completes (1), and (2) follows by AT-Anno.
  
  
  \item[(AT-Fst)] Suppose $\Psi ; \Theta ; \Delta ; \Omega ; \Gamma \vdash \texttt{fst}(e) \infers \tau_1 \gens \Phi,\Gamma''$ by way of
  $\Psi ; \Theta ; \Delta ; \Omega ; \Gamma \vdash e \infers \tau_1 \amp \tau_2 \gens \Phi,\Gamma''$
  with $\Theta ; \Delta \vDash \Phi$,
  $\Psi ; \Theta ; \Delta \vdash \Gamma' \wknto \Gamma$, and
  $\Psi ; \Theta ; \Delta \vdash \Omega' \wknto \Omega$.
  By IH, we have $e'$, $\Phi'$, $\Gamma'''$ such that
  $|e'| = |e|$,
  $\Theta ; \Delta \vDash \Phi'$,
  $\Psi ; \Theta ; \Delta \vdash \Gamma''' \wknto \Gamma' \setminus \Gamma$,
  $\Psi ; \Theta ; \Delta \vdash \Gamma''' \wknto \Gamma''$, and
  $\Psi ; \Theta ; \Delta ; \Omega' ; \Gamma' \vdash e' \infers \tau_1 \amp \tau_2 \gens \Phi',\Gamma'''$.
  By AT-Fst, $\Psi ; \Theta ; \Delta ; \Omega' ; \Gamma' \vdash \texttt{fst}(e') \infers \tau_1 \gens \Phi',\Gamma'''$,
  which completes (2), since $|\texttt{fst}(e')| = \texttt{fst}(|e'|) = \texttt{fst}(|e|) = |\texttt{fst}(e)|$. For (1),
  by \autoref{thm:subty-refl}, there is $\Phi''$ such that $\Psi ; \Theta ; \Delta \vdash \tau_1 \subty \tau_1 : \star \gens \Phi''$ with $\Theta ; \Delta \vDash \Phi''$. By AT-Sub, $\Psi ; \Theta ; \Delta ; \Omega' ; \Gamma' \vdash \texttt{fst}(e') \checks \tau_1 \gens \Phi',\Gamma'''$, completing (1).

  \item[(AT-Snd)] Identical to AT-Fst.
  
  \item[(AT-Inl)] Suppose $\Psi ; \Theta ; \Delta ; \Omega ; \Gamma \vdash \texttt{inl}(e) \checks \tau_1 \oplus \tau_2 \gens \Phi,\Gamma''$
  from $\Psi ; \Theta ; \Delta ; \Omega ; \Gamma \vdash e \checks \tau_1 \gens \Phi,\Gamma''$, and
  $\Theta ; \Delta \vDash \Phi$,
  $\Psi ; \Theta ; \Delta \vdash \Gamma' \wknto \Gamma$, and
  $\Psi ; \Theta ; \Delta \vdash \Omega' \wknto \Omega$.
  By IH, there are $e'$, $\Phi'$, $\Gamma'''$ such that
  $|e'| = |e|$,
  $\Theta ; \Delta \vDash \Phi'$,
  $\Psi ; \Theta ; \Delta \vdash \Gamma''' \wknto \Gamma''$,
  $\Psi ; \Theta ; \Delta \vdash \Gamma''' \wknto \Gamma' \setminus \Gamma$, and
  $\Psi ; \Theta ; \Delta ; \Omega' ; \Gamma' \vdash e' \checks \tau_1 \gens \Phi',\Gamma'''$.
  By AT-Inl, we have
  $\Psi ; \Theta ; \Delta ; \Omega' ; \Gamma' \vdash \texttt{inl}(e') \checks \tau_1 \oplus \tau_2 \gens \Phi',\Phi'''$,
  which completes (1), and (2) is done by AT-Anno.
  
  \item[(AT-Inr)] Identical to AT-Inl.
  
  \item[(AT-Case)] Suppose $\Psi ; \Theta ; \Delta ; \Omega ; \Gamma \vdash \texttt{case}(e,x.e_1,y.e_2) \checks \tau \gens \Phi_1 \wedge \Phi_2 \wedge \Phi_3, (\Gamma_2 \setminus \{x : \tau_1\}) \cap (\Gamma_3 \setminus \{y : \tau_2\})$
  from $\Psi ; \Theta ; \Delta ; \Omega ; \Gamma \vdash e \infers \tau_1 \oplus \tau_2 \gens \Phi_1, \Gamma_1$,
  $\Psi ; \Theta ; \Delta ; \Omega ; \Gamma_1, x: \tau_1 \vdash e_1 \checks \tau \gens \Phi_2,\Gamma_2$, and
  $\Psi ; \Theta ; \Delta ; \Omega ; \Gamma_1, y: \tau_2 \vdash e_2 \checks \tau \gens \Phi_3,\Gamma_3$, and also that
  $\Theta ; \Delta \vDash \Phi_1 \wedge \Phi_2 \wedge \Phi_3$,
  $\Psi ; \Theta ; \Delta \vdash \Gamma' \wknto \Gamma$, and
  $\Psi ; \Theta ; \Delta \vdash \Omega' \wknto \Omega$.
  By IH, we have $e'$, $\Phi_1'$, $\Gamma_1'$ such that
  $|e'| = |e|$,
  $\Theta ; \Delta \vDash \Phi_1'$,
  $\Psi ; \Theta ; \Delta \vdash \Gamma_1' \wknto \Gamma_1$,
  $\Psi ; \Theta ; \Delta \vdash \Gamma_1' \wknto \Gamma' \setminus \Gamma$, and
  $\Psi ; \Theta ; \Delta ; \Omega' ; \Gamma' \vdash e' \infers \tau_1 \oplus \tau_2 \gens \Phi_1',\Gamma_1'$.
  Then, since $\Psi ; \Theta ; \Delta \vdash \Gamma_1', x : \tau_1 \wknto \Gamma_1, x : \tau_1$,
  we have by IH $e_1'$. $\Phi_2'$. $\Gamma_2'$ such that
  $|e_1'| = |e_1|$,
  $\Theta  ; \Delta \vDash \Phi_2'$,
  $\Psi ; \Theta ; \Delta \vdash \Gamma_2' \wknto \Gamma_2$,
  $\Psi ; \Theta ; \Delta \vdash \Gamma_2' \wknto \Gamma_1' \setminus \Gamma_1$, and
  $\Psi ; \Theta ; \Delta ; \Omega' ; \Gamma', x : \tau_1 \vdash e_1' \checks \tau \gens \Phi_2',\Gamma_2'$.
  Similarly, since $\Psi ; \Theta ; \Delta \vdash \Gamma_1', y : \tau_2 \wknto \Gamma_1, y : \tau_2$,
  we have by IH $e_2'$. $\Phi_3'$. $\Gamma_3'$ such that
  $|e_2'| = |e_2|$,
  $\Theta  ; \Delta \vDash \Phi_3'$,
  $\Psi ; \Theta ; \Delta \vdash \Gamma_3' \wknto \Gamma_3$,
  $\Psi ; \Theta ; \Delta \vdash \Gamma_3' \wknto \Gamma_1' \setminus \Gamma_1$, and
  $\Psi ; \Theta ; \Delta ; \Omega' ; \Gamma', x : \tau_2 \vdash e_2' \checks \tau \gens \Phi_3',\Gamma_3'$.
  Then, by AT-Case, we have
  $\Psi ; \Theta ; \Delta ; \Omega' ; \Gamma' \vdash \texttt{case}(e',x.e_1',y.e_2') \checks \tau \gens \Phi_1' \wedge \Phi_2' \wedge \Phi_3', (\Gamma_2' \setminus \{x : \tau_1\}) \cap (\Gamma_3' \setminus \{y : \tau_2\})$.
  Of course, $|\texttt{case}(e',x.e_1',y.e_2')| = |\texttt{case}(e,x.e_1,y.e_2)|$
  since $|e'| = |e|$, $|e_1'| = |e_1|$, and $|e_2'| = |e_2|$.
  Similarly,
  $\Theta ; \Delta \vDash \Phi_1' \wedge \Phi_2' \wedge \Phi_3'$.
  It remains to show that 
  $\Psi ; \Theta ; \Delta \vdash (\Gamma_2' \setminus \{x : \tau_1\}) \cap (\Gamma_3' \setminus \{y : \tau_2\}) \wknto (\Gamma_2 \setminus \{x : \tau_1\}) \cap (\Gamma_3 \setminus \{y : \tau_2\})$
  and
  $\Psi ; \Theta ; \Delta \vdash (\Gamma_2' \setminus \{x : \tau_1\}) \cap (\Gamma_3' \setminus \{y : \tau_2\}) \wknto \Gamma' \setminus \Gamma$,
  but both follow from the six weakening judgments and a few applications of \autoref{thm:ctx-sub-subset2}.
  This completes (1), and (2) follows immediately by AT-Anno.
  
  \item[(AT-Sub)] Suppose 
    $\Psi ; \Theta ; \Delta ; \Omega ; \Gamma \vdash e \checks \tau \gens \Phi_1 \wedge \Phi_2,\Gamma''$ from
   $\Psi ; \Theta ; \Delta ; \Omega ; \Gamma \vdash e \infers \tau' \gens \Phi_1,\Gamma''$ and 
   $\Psi;\Theta;\Delta \vdash \tau' \subty \tau : \star \gens \Phi_2$, and
   $\Theta ; \Delta \vDash \Phi_1 \wedge \Phi_2$,
   $\Psi ; \Theta ; \Delta \vdash \Gamma' \wknto \Gamma$, and
   $\Psi ; \Theta ; \Delta \vdash \Omega' \wknto \Omega$.
   By IH, there are $e'$,$\Phi_1'$, and $\Gamma'''$ such that
   $|e| = |e'|$,
   $\Theta ; \Delta \vDash \Phi_1'$,
   $\Psi ; \Theta ; \Delta \vDash \Gamma''' \wknto \Gamma' \setminus \Gamma$,
   $\Psi ; \Theta ; \Delta \vDash \Gamma''' \wknto \Gamma''$,
   and $\Psi ; \Theta ; \Delta ; \Omega' \Gamma' \vdash e' \infers \tau' \gens \Phi_1',\Gamma'''$.
   by AT-Sub, $\Psi ; \Theta ; \Delta ; \Omega' ; \Gamma' \vdash e' \checks \tau \gens \Phi_1' \wedge \Phi_2,\Gamma'''$, which proves (1).
   For (2), we again use AT-Anno.
   
  \item[(AT-ExpI)] Suppose $\Psi ; \Theta ; \Delta ; \Omega ; \Gamma \vdash !e \checks !\tau \gens \Phi, \Gamma$ from
  $\Psi ; \Theta ; \Delta ; \Omega ; \cdot \vdash e \checks \tau \gens \Phi, \Gamma''$, and
  $\Theta ; \Delta \vDash \Phi$,
  $\Psi ; \Theta ; \Delta \vdash \Gamma' \wknto \Gamma$, and
  $\Psi ; \Theta ; \Delta \vdash \Omega' \wknto \Omega$.
  By IH (not weakening the empty affine environment) there are $e'$, $\Phi'$, $\Gamma'''$ such that 
  $|e'| = |e|$,
  $\Theta ; \Delta \vDash \Phi'$,
  and  $\Psi ; \Theta ; \Delta ; \Omega' ; \cdot \vdash e' \checks \tau \gens \Phi',\Gamma'''$.
  By AT-ExpI,
  $\Psi ; \Theta ; \Delta ; \Omega' ; \Gamma' \vdash !e' \checks \tau \gens \Phi',\Gamma'$.
  This completes (1), since $|!e'| = |!e|$, $\Theta ; \Delta \vDash \Phi'$,
  $\Psi ; \Theta ; \Delta \vdash \Gamma' \wknto \Gamma'$ and $\Psi ; \Theta ; \Delta \vdash \Gamma' \wknto \Gamma' \setminus \Gamma$
  as consequences of \autoref{thm:ctx-sub-subset2}.
  
  \item[(AT-ExpE)] Suppose $\Psi ; \Theta ; \Delta ; \Omega ; \Gamma \vdash \texttt{let } !x = e_1 \texttt{ in } e_2 \checks \tau' \gens \Phi_1 \wedge \Phi_2, \Gamma_2$
  from $\Psi ; \Theta ; \Delta ; \Omega ; \Gamma \vdash e_1 \infers !\tau \gens \Phi_1,\Gamma_1$
  and $\Psi ; \Theta ; \Delta ; \Omega, x : \tau ; \Gamma_1 \vdash e_2 \checks \tau' \gens \Phi_2,\Gamma_2$, with
  $\Theta ; \Delta \vDash \Phi_1 \wedge \Phi_2$,
  $\Psi ; \Theta ; \Delta \vdash \Gamma' \wknto \Gamma$, and
  $\Psi ; \Theta ; \Delta \vdash \Omega' \wknto \Omega$.
  By IH, there are $e_1'$, $\Phi_1'$, $\Gamma_1'$ such that
  $|e_1'| = |e_1|$,
  $\Theta ; \Delta \vDash \Phi_1'$,
  $\Psi ; \Theta ; \Delta \vdash \Gamma_1' \wknto \Gamma' \setminus \Gamma'$,
  $\Psi ; \Theta ; \Delta \vdash \Gamma_1' \wknto \Gamma_1$, and  
  $\Psi ; \Theta ; \Delta ; \Omega' ; \Gamma' \vdash e_1' \infers !\tau \gens \Phi_1',\Gamma_1'$.
  Since $\Psi ; \Theta ; \Delta \vdash \Omega', x : \tau \wknto \Omega, x : \tau$, we have
  by IH we have $e_2'$, $\Phi_2'$, $\Gamma_2'$ such that
  $|e_2'| = |e_2|$,
  $\Theta ; \Delta \vDash \Phi_2'$,
  $\Psi ; \Theta ; \Delta \vdash \Gamma_2' \wknto \Gamma_1' \setminus \Gamma_1$,
  $\Psi ; \Theta ; \Delta \vdash \Gamma_2' \wknto \Gamma_2'$, and
  $\Psi ; \Theta ; \Delta ; \Omega', x : \tau ; \Gamma_1' \vdash e_2' \checks \tau' \gens \Phi_2',\Gamma_2'$.
  By AT-ExpE,
  $\Psi ; \Theta ; \Delta ; \Omega' ; \Gamma' \vdash \texttt{let } !x = e_1' \texttt{ in } e_2' \checks \tau' \gens \Phi_1' \wedge \Phi_2', \Gamma_2'$.
  Applying \autoref{thm:ctx-sub-subset2} to the inductive hypotheses gives us that
  $\Psi ; \Theta ; \Delta \vdash \Gamma_2' \wknto \Gamma' \setminus \Gamma$, completing (1).
  For (2), one application of AT-Anno suffices.
  
  \item[(AT-TAbs)] Suppose $\Psi ; \Theta ; \Delta ; \Omega ; \Gamma \vdash \Lambda \alpha. e \checks \forall \alpha : K.\tau \gens \Phi,\Gamma''$ from
  $\Psi, \alpha : K ; \Theta ; \Delta ; \Omega ; \Gamma \vdash e \checks \tau \gens \Phi, \Gamma''$, and
  $\Theta ; \Delta \vDash \Phi$,
  $\Psi ; \Theta ; \Delta \vdash \Gamma' \wknto \Gamma$, and
  $\Psi ; \Theta ; \Delta \vdash \Omega' \wknto \Omega$.
  By IH, we have $e'$, $\Phi'$, $\Gamma'''$ such that
  $|e'| = |e|$,
  $\Theta ; \Delta \vdash \Phi'$,
  $\Psi, \alpha : K ; \Theta ; \Delta \vdash \Gamma''' \wknto \Gamma' \setminus \Gamma$
  $\Psi, \alpha : K ; \Theta ; \Delta \vdash \Gamma''' \wknto \Gamma''$, and
  $\Psi, \alpha : K ; \Theta ; \Delta ; \Omega' ; \Gamma' \vdash e' \checks \tau \gens \Phi',\Gamma'''$.
  By AT-TAbs, $\Psi ; \Theta ; \Delta ; \Omega' ; \Gamma' \vdash e' \checks \forall \alpha : K. \tau \gens \Phi',\Gamma'''$.
  By \autoref{thm:ctx-sub-streng}, we have that $\Psi ; \Theta ; \Delta \vdash \Gamma''' \wknto \Gamma' \setminus \Gamma$
  and $\Psi  ; \Theta ; \Delta \vdash \Gamma''' \wknto \Gamma''$, which completes (1).
  For (2), one use of AT-Anno suffices.
  
  \item[(AT-TApp)] Suppose $\Psi ; \Theta ; \Delta ; \Omega ; \Gamma \vdash e [\tau'] \infers \tau[\tau'/\alpha] \gens \Phi_1 \wedge \Phi_2, \Gamma''$ from
  $\Psi ; \Theta ; \Delta ; \Omega ; \Gamma \vdash e \infers \forall \alpha : K.\tau \gens \Phi_1, \Gamma''$ and
  $\Psi ; \Theta ; \Delta \vdash \tau' : K \gens \Phi_2$, with
  $\Theta ; \Delta \vDash \Phi_1 \wedge \Phi_2$,
  $\Psi ; \Theta ; \Delta \vdash \Gamma' \wknto \Gamma$, and
  $\Psi ; \Theta ; \Delta \vdash \Omega' \wknto \Omega$.
  By IH, there are $e'$, $\Phi_1'$, $\Gamma'''$ such that 
  $|e'| = |e'|$,
  $\Theta ; \Delta \vDash \Phi_1'$,
  $\Psi ; \Theta ; \Delta \vdash \Gamma''' \wknto \Gamma' \setminus \Gamma$,
  $\Psi ; \Theta ; \Delta \vdash \Gamma''' \wknto \Gamma''$, and
  $\Psi ; \Theta ; \Delta ; \Omega' ; \Gamma' \vdash e' \infers \forall \alpha : K.\tau \gens \Phi_1', \Gamma'''$.
  By AT-TApp,
  $\Psi ; \Theta ; \Delta ; \Omega ; \Gamma \vdash e' [\tau'] \infers \tau[\tau'/\alpha] \gens \Phi_1' \wedge \Phi_2, \Gamma'''$.
  Since $|e' [\tau']| = |e'| [\tau'] = |e [\tau']|$ and $\Theta ; \Delta \vDash \Phi_1' \wedge \Phi_2$, we are done with (1).
  For (2), a single use of AT-Anno completes the proof.
  
  \item[(AT-IAbs)] Suppose $\Psi ; \Theta ; \Delta ; \Omega ; \Gamma \vdash \Lambda i. e \checks \forall i : S. \tau \gens \forall i : S. \Phi, \Gamma''$
  from $\Psi ; \Theta, i : S ; \Delta ; \Omega ; \Gamma \vdash e \checks \tau \gens \Phi, \Gamma''$, with
  $\Theta ; \Delta \vDash \Phi$,
  $\Psi ; \Theta ; \Delta \vdash \Gamma' \wknto \Gamma$, and
  $\Psi ; \Theta ; \Delta \vdash \Omega' \wknto \Omega$.
  By \autoref{thm:ctx-sub-wkn},
  $\Psi ; \Theta, i : S ; \Delta \vdash \Gamma' \wknto \Gamma$ and
  $\Psi ; \Theta, i : S ; \Delta \vdash \Omega' \wknto \Omega$.
  By IH, there are $e'$, $\Phi'$, $\Gamma'''$ such that
  $|e'| = |e|$,
  $\Theta, i : S; \Delta \vDash \Phi'$,
  $\Psi ; \Theta, i : S ; \Delta \vdash \Gamma'' \wknto \Gamma' \setminus \Gamma$,
  $\Psi ; \Theta, i : S ; \Delta \vdash \Gamma'' \wknto \Gamma''$, and
  $\Psi ; \Theta, i : S ; \Delta ; \Omega' ; \Gamma' \vdash e' : \tau \gens \Phi', \Gamma'''$.
  By AT-IAbs,
  $\Psi ; \Theta ; \Delta ; \Omega' ; \Gamma' \vdash \Lambda i. e' : \tau \gens \forall i : S. \Phi', \Gamma'''$.
  By \autoref{thm:ctx-sub-streng},
  $\Psi ; \Theta ; \Delta \vdash \Gamma'' \wknto \Gamma' \setminus \Gamma$ and
  $\Psi ; \Theta ; \Delta \vdash \Gamma'' \wknto \Gamma''$.
  Finally, the fact that $\Theta ; \Delta \vDash \forall i : S. \Phi'$
  completes the proof of (1).
  For (2), AT-Anno suffices.
  
  \item[(AT-IApp)] Suppose $\Psi ; \Theta ; \Delta ; \Omega ; \Gamma \vdash e [I] \infers \tau[I/i] \gens \Phi_1 \wedge \Phi_2,\Gamma''$
  from $\Psi ; \Theta ; \Delta ; \Omega ; \Gamma \vdash e \infers \forall i : S.\tau \gens \Phi_1,\Gamma''$ and
  $\Theta ; \Delta \vdash I : S \gens \Phi_2$, with
  $\Theta ; \Delta \vDash \Phi_1 \wedge \Phi_2$,
  $\Psi ; \Theta ; \Delta \vdash \Gamma' \wknto \Gamma$, and
  $\Psi ; \Theta ; \Delta \vdash \Omega' \wknto \Omega$.
  By IH, there are $e'$, $\Phi_1'$, $\Gamma'''$ such that
  $|e| = |e'|$,
  $\Theta ; \Delta \vDash \Phi_1'$,
  $\Psi ; \Theta ; \Delta \vdash \Gamma''' \wknto \Gamma' \setminus \Gamma$,
  $\Psi ; \Theta ; \Delta \vdash \Gamma''' \wknto \Gamma''$, and
  $\Psi ; \Theta ; \Delta ; \Omega' ; \Gamma' \vdash e \infers \forall i : S. \tau \gens \Phi_1',\Gamma'''$.
  By AT-IApp,
  $\Psi ; \Theta ; \Delta ; \Omega' ; \Gamma' \vdash e [I] \infers \tau[I/i] \gens \Phi_1' \wedge \Phi_2,\Gamma'''$.
  Since $\Theta ; \Delta \vDash \Phi_1' \wedge \Phi_2$ and $|e[I]| = |e|[I] = |e'|[I] = |e'[I]|$, this completes (2).
  For (1), we have by \autoref{thm:subty-refl} some $\Phi_3$ such that $\Psi ; \Theta ; \Delta \vdash \tau[I/i] \subty \tau[I/i] : \star \gens \Phi_3$
  and $\Theta ; \Delta \vDash \Phi_3$.
  By AT-Sub, we  have that $\Psi ; \Theta ; \Delta ; \Omega' ; \Gamma' \vdash e [I] \checks \tau[I/i] \gens \Phi_1' \wedge \Phi_2 \wedge \Phi_3,\Gamma'''$,
  as required for (1).
  
  \item[(AT-Fix)] Suppose $\Psi ; \Theta ; \Delta ; \Omega ; \Gamma \vdash \texttt{fix } x.e \checks \tau \gens \Phi,\Gamma$
  by way of $\Psi ; \Theta ; \Delta ; \Omega, x : \tau ; \cdot \vdash e \checks \tau \gens \Phi,\Gamma''$, with
  $\Theta ; \Delta \vDash \Phi$,
  $\Psi ; \Theta ; \Delta \vdash \Gamma' \wknto \Gamma$, and
  $\Psi ; \Theta ; \Delta \vdash \Omega' \wknto \Omega$. Then,
  $\Psi ; \Theta ; \Delta \vdash \Omega', x : \tau \wknto \Omega, x : \tau$.
  By IH, there are $e'$, $\Phi'$, $\Gamma''$ such that
  $|e| = |e'|$,
  $\Theta ; \Delta \vDash \Phi'$, and
  $\Psi ; \Theta ; \Delta ; \Omega', x : \tau ; \cdot \vdash e' \checks \tau \gens \Phi',\Gamma'''$.
  By AT-Fix,
  $\Psi ; \Theta ; \Delta ; \Omega' ; \Gamma' \vdash \texttt{fix } x.e' \checks \tau \gens \Phi',\Gamma'$.
  Of course, $|\texttt{fix }x.e'| = \texttt{fix }x.|e'| = \texttt{fix }x.|e| = |\texttt{fix }x.e|$.
  Further, $\Psi ; \Theta ; \Delta \vdash \Gamma' \wknto \Gamma' \setminus \Gamma$
  by \autoref{thm:ctx-sub-subset2}, and $\Psi ; \Theta ; \Delta \vdash \Gamma' \wknto \Gamma$ by one of the premises.
  This completes (1), and (2) follows by AT-Anno.
  
  \item[(AT-Ret)] Suppose
  $\Psi ; \Theta ; \Delta ; \Omega ; \Gamma \vdash \texttt{ret } e \checks \M \, \phi(I,\vec{p}) \, \tau \gens \Phi, \Gamma''$ from
  $\Psi ; \Theta ; \Delta ; \Omega ; \Gamma \vdash e \checks \tau \gens \Phi,\Gamma''$, with
  $\Theta ; \Delta \vdash \Phi$,
  $\Psi ; \Theta ; \Delta \vdash \Gamma' \wknto \Gamma$, and
  $\Psi ; \Theta ; \Delta \vdash \Omega' \wknto \Omega$.
  By IH, there are $e'$, $\Phi'$, $\Gamma'''$ such that 
  $|e'| = |e|$,
  $\Theta ; \Delta \vDash \Phi'$,
  $\Psi ;  \Theta ; \Delta \vdash \Gamma''' \wknto \Gamma' \setminus \Gamma$,
  $\Psi ;  \Theta ; \Delta \vdash \Gamma''' \wknto \Gamma''$, and
  $\Psi ; \Theta ; \Delta ; \Omega' ; \Gamma' \vdash e' \checks \tau \gens \Phi',\Gamma'''$.
  By AT-Ret,
  $\Psi ; \Theta ; \Delta ; \Omega' ; \Gamma' \vdash \texttt{ret } e' \checks \M \, \phi(I,\vec{p})\, \tau \gens \Phi',\Gamma'''$,
  completing (1). (2) follows by AT-Anno.
  

  \item[(AT-Bind)] Suppose
  $\Psi ; \Theta ; \Delta ; \Omega ; \Gamma \vdash \texttt{bind } x = e_1 \texttt{ in } e_2 \checks \M \, \phi(I,\vec{q})\, \tau_2 \gens (\vec{q} \geq \vec{p}) \wedge (I =J)  \wedge \Phi_1 \wedge \Phi_2, \Gamma_2 \setminus \{x : \tau_1\}$ from
  $\Psi ; \Theta ; \Delta ; \Omega ; \Gamma \vdash e_1 \infers \M \, \phi(J,\vec{p})\, \tau_1 \gens \Phi_1,\Gamma_1$ and
  $\Psi ; \Theta; \Delta ; \Omega ; \Gamma_1, x:\tau_1 \vdash e_2 \checks \M \, \phi(I,\vec{q} - \vec{p})\, \tau_2 \gens \Phi_2,\Gamma_2$ with
  $\Theta ; \Delta \vDash (\vec{q} \geq \vec{p}) \wedge (I =J)  \wedge \Phi_1 \wedge \Phi_2$,
  $\Psi ; \Theta ; \Delta \vdash \Gamma' \wknto \Gamma$, and
  $\Psi ; \Theta ; \Delta \vdash \Omega' \wknto \Omega$.
  By IH, there are $e_1'$, $\Phi_1'$, $\Gamma_1'$ such that
  $|e_1'| = |e_1|$,
  $\Theta ; \Delta \vDash \Phi_1'$,
  $\Psi ; \Theta ; \Delta \vdash \Gamma_1' \wknto \Gamma' \setminus \Gamma$,
  $\Psi ; \Theta ; \Delta \vdash \Gamma_1' \wknto \Gamma_1$, and
  $\Psi ; \Theta ; \Delta ; \Omega' ; \Gamma' \vdash e_1' \infers \M \, \phi(J,\vec{p}) \, \tau_1 \gens \Phi_1,'\Gamma_1'$.
  We note that $\Psi ; \Theta ; \Delta \vdash \Gamma_1',x:\tau_1 \wknto \Gamma_1, x: \tau_1$, and so
  by IH, there are $e_2'$, $\Phi_2'$, $\Gamma_2'$ such that
  $|e_2'| = |e_2|$,
  $\Theta ; \Delta \vDash \Phi_2'$,
  $\Psi ; \Theta ; \Delta \vdash \Gamma_2' \wknto \Gamma_1' \setminus \Gamma_1$,
  $\Psi ; \Theta ; \Delta \vdash \Gamma_2' \wknto \Gamma_2$, and
  $\Psi ; \Theta; \Delta ; \Omega' ; \Gamma_1', x:\tau_1 \vdash e_2' \checks \M \, \phi(I,\vec{q} - \vec{p})\, \tau_2 \gens \Phi_2',\Gamma_2'$.
  By AT-Bind,
  $\Psi ; \Theta ; \Delta ; \Omega' ; \Gamma' \vdash \texttt{bind } x = e_1' \texttt{ in } e_2' \checks \M \, \phi(I,\vec{q})\, \tau_2 \gens (\vec{q} \geq \vec{p}) \wedge (I =J)  \wedge \Phi_1' \wedge \Phi_2', \Gamma_2' \setminus \{x : \tau_1\}$.
  Of course, $|\texttt{bind } x = e_1 \texttt{ in } e_2| = |\texttt{bind } x = e_1' \texttt{ in } e_2'|$.
  Since $\Theta ; \Delta \vDash (\vec{q} \geq \vec{p}) \wedge (I=J)$, we also have that $\Theta ; \Delta \vDash (\vec{q} \geq \vec{p}) \wedge (I =J)  \wedge \Phi_1' \wedge \Phi_2'$.
  We have $\Psi ; \Theta ; \Delta \vdash \Gamma_2' \wknto \Gamma_2$  by IH, and
  $\Psi ; \Theta ; \Delta \vdash \Gamma_2' \setminus \{x : \tau_1\} \wknto \Gamma' \setminus \Gamma$ follows by applying \autoref{thm:ctx-sub-subset2} to the rest of the weakening premises. This completes (1), and (2) follows by AT-Anno.
  
  \item[(AT-Tick)] Immediate.
  
  \item[(AT-Release)] Suppose
  $\Psi ; \Theta ; \Delta ; \Omega ; \Gamma \vdash \texttt{release } x = e_1 \texttt{ in }e_2 \checks \M \, \phi(I,\vec{p}) \, \tau_2 \gens (I = J) \wedge \Phi_1 \wedge \Phi_2, \Gamma_2 \setminus \{x\}$ from
  $\Psi ; \Theta ; \Delta ; \Omega ; \Gamma \vdash e_1 \infers [J | \vec{q}] \tau_1 \gens \Phi_1,\Gamma_1$ and
  $\Psi ; \Theta ; \Delta ; \Omega ; \Gamma_1, x : \tau \vdash e_2 \checks \M \, \phi(I,\vec{p} + \vec{q}) \, \tau_2 \gens \Phi_2, \Gamma_2$, with
  $\Theta ; \Delta \vDash (I = J) \wedge \Phi_1 \wedge \Phi_2$,
  $\Psi ; \Theta ; \Delta \vdash \Gamma' \wknto \Gamma$, and
  $\Psi ; \Theta ; \Delta \vdash \Omega' \wknto \Omega$.
  By IH, there are $e_1'$, $\Phi_1'$, $\Gamma_1'$ such that
  $|e_1'| = |e_1|$,
  $\Theta ; \Delta \vDash \Phi_1'$,
  $\Psi ; \Theta ; \Delta \vDash \Gamma_1' \wknto \Gamma' \setminus \Gamma$,
  $\Psi ; \Theta ; \Delta \vDash \Gamma_1' \wknto \Gamma_1$, and
  $\Psi ; \Theta ; \Delta ; \Omega' ; \Gamma' \vdash e_1' \infers [J|\vec{q}] \tau_1 \gens \Phi_1',\Gamma_1'$.
  Since $\Psi ; \Theta ; \Delta \vDash \Gamma_1', x : \tau \wknto \Gamma_1, x : \tau$, we have
  by IH that there are $e_2'$, $\Phi_2'$, $\Gamma_2'$ such that
  $|e_2'| = |e_2|$,
  $\Theta ; \Delta \vDash \Phi_2'$,
  $\Psi ; \Theta ; \Delta \vdash \Gamma_2' \wknto \Gamma_1' \setminus \Gamma_1$,
  $\Psi ; \Theta ; \Delta \vdash \Gamma_2' \wknto \Gamma_2$, and
  $\Psi ; \Theta ; \Delta ; \Omega' ; \Gamma_1', x : \tau \vdash e_2' \checks \M \, \phi(I,\vec{p} + \vec{q}) \, \tau_2 \gens \Phi_2',\Gamma_2'$.
  By AT-Release,
  $\Psi ; \Theta ; \Delta ; \Omega' ; \Gamma' \vdash \texttt{release } x = e_1' \texttt{ in }e_2' \checks \M \, \phi(I,\vec{p}) \, \tau_2 \gens (I = J) \wedge \Phi_1' \wedge \Phi_2', \Gamma_2 \setminus \{x\}$.
  Of course, $|\texttt{release } x = e_1' \texttt{ in }e_2'| = |\texttt{release } x = e_1 \texttt{ in }e_2|$.
  Since $\Theta ; \Delta \vDash I = J$, we have $\Theta ; \Delta \vDash (I = J) \wedge \Phi_1' \wedge \Phi_2'$.
  $\Psi ; \Theta ; \Delta \vdash \Gamma_2' \setminus \{x\} \wknto \Gamma_2 \{x\}$ is implied by $\Psi ; \Theta ; \Delta \vdash \Gamma_2' \wknto \Gamma_2$,
  and $\Psi ; \Theta ; \Delta \vdash \Gamma_2' \setminus \{x\} \wknto \Gamma' \setminus \Gamma$ follows from \autoref{thm:ctx-sub-subset2}. This completes (1), and (2) follows from AT-Anno.
  
  \item[(AT-Store)] Suppose
  $\Psi ; \Theta ; \Delta ; \Omega ; \Gamma \vdash \texttt{store}[K|\vec{w}](e) \checks \M \, \phi(I,\vec{q}) \, ([J | \vec{p}] \, \tau) \gens \Phi_1 \wedge \Phi_2 \wedge\Phi_3 \wedge  (\vec{p} \leq \vec{w} \leq \vec{q}) \wedge (I = J = K), \Gamma''$ by way of
  $\Theta ; \Delta \vdash K : \N \gens \Phi_1$,
  $\Theta ; \Delta \vdash \vec{w} : \vec{\mathbb{R}^+} \gens \Phi_2$,
  $\Psi ; \Theta ; \Delta ; \Omega ; \Gamma \vdash e \checks \tau \gens \Phi_3,\Gamma''$, with
  $\Theta ; \Delta \vDash \Phi_1 \wedge \Phi_2 \wedge\Phi_3 \wedge  (\vec{p} \leq \vec{w} \leq \vec{q}) \wedge (I = J = K)$,
  $\Psi ; \Theta ; \Delta \vdash \Gamma' \wknto \Gamma$, and
  $\Psi ; \Theta ; \Delta \vdash \Omega' \wknto \Omega$.
  By IH, there are $e'$, $\Gamma'''$, $\Phi_3'$ such that
  $|e'| = |e|$,
  $\Theta ; \Delta \vDash \Phi_3'$,
  $\Psi ; \Theta ; \Delta \vdash \Gamma''' \wknto \Gamma' \setminus \Gamma$,
  $\Psi ; \Theta ; \Delta \vdash \Gamma''' \wknto \Gamma''$, and
  $\Psi ; \Theta ; \Delta ; \Omega' ; \Gamma' \vdash e' \checks \tau \gens \Phi_3', \Gamma'''$.
  By AT-Store,
  $\Psi ; \Theta ; \Delta ; \Omega' ; \Gamma' \vdash \texttt{store}[K|\vec{w}](e') \checks \M \, \phi(I,\vec{q}) \, ([J | \vec{p}] \, \tau) \gens \Phi_1 \wedge \Phi_2 \wedge\Phi_3' \wedge  (\vec{p} \leq \vec{w} \leq \vec{q}) \wedge (I = J = K), \Gamma'''$,
  which completes (1). For (2), AT-Anno suffices.
  
  \item[(AT-StoreConst)] Suppose $\Psi ; \Theta ; \Delta ; \Omega ; \Gamma \vdash \texttt{store}[J](e) \checks \M \, \phi(K,\vec{p}) \, ([I] \, \tau) \gens (\texttt{const}(I) \leq \texttt{const}(J) \leq \vec{p}) \wedge \Phi_1 \wedge \Phi_2, \Gamma''$ from
  $\Psi ; \Theta ; \Delta ; \Omega ; \Gamma \vdash e \checks \tau \gens \Phi_1,\Gamma''$ and
  $\Theta ; \Delta \vdash J : \mathbb{R} \gens \Phi_2$, with
  $\Theta ; \Delta \vDash (\texttt{const}(I) \leq \texttt{const}(J) \leq \vec{p}) \wedge \Phi_1 \wedge \Phi_2$,
  $\Psi ; \Theta ; \Delta \vdash \Gamma' \wknto \Gamma$, and
  $\Psi ; \Theta ; \Delta \vdash \Omega' \wknto \Omega$.
  By IH, there are $e'$, $\Phi_1'$, $\Gamma'''$ such that
  $|e'| = |e|$,
  $\Theta ; \Delta \vDash \Phi_1'$,
  $\Phi ; \Theta ; \Delta \vdash \Gamma''' \wknto \Gamma' \setminus \Gamma$,
  $\Phi ; \Theta ; \Delta \vdash \Gamma''' \wknto \Gamma''$, and
  $\Phi ; \Theta ; \Delta ; \Omega' ; \Theta' \vdash e' \checks \tau \gens \Phi_1', \Gamma'''$.
  By AT-StoreConst,
  $\Psi ; \Theta ; \Delta ; \Omega' ; \Gamma \vdash \texttt{store}[J](e') \checks \M \, \phi(K,\vec{p}) \, ([I] \, \tau) \gens (\texttt{const}(I) \leq \texttt{const}(J) \leq \vec{p}) \wedge \Phi_1' \wedge \Phi_2, \Gamma''$, which completes (1). For (2), we use AT-Anno.
  
  \item[(AT-ReleaseConst)] Suppose
  $\Psi ; \Theta ; \Delta ; \Omega ; \Gamma \vdash \texttt{release } x = e_1 \texttt{ in }e_2 \checks \M \, \phi(I,\vec{p}) \, \tau_2 \gens \Phi_1 \wedge \Phi_2, \Gamma_2 \setminus \{x\}$ from
  $\Psi ; \Theta ; \Delta ; \Omega ; \Gamma \vdash e_1 \infers [J] \tau_1 \gens \Phi_1,\Gamma_1$, and
  $\Psi ; \Theta ; \Delta ; \Omega ; \Gamma_1, x : \tau_1 \vdash e_2 \checks \M \, \phi(I,\vec{p} + \texttt{const}(J)) \, \tau_2 \gens \Phi_2, \Gamma_2$, with
  $\Theta ; \Delta \vDash \Phi_1 \wedge \Phi_2$,
  $\Psi ; \Theta ; \Delta \vdash \Gamma' \wknto \Gamma$, and
  $\Psi ; \Theta ; \Delta \vdash \Omega' \wknto \Omega$.
  By IH, there are $e_1'$, $\Gamma_1'$, $\Phi_1'$ such that
  $|e_1'| = |e_1|$,
  $\Theta ; \Delta \vDash \Phi_1'$,
  $\Psi ; \Theta ; \Delta \vdash \Gamma_1' \wknto \Gamma' \setminus \Gamma$,
  $\Psi ; \Theta ; \Delta \vdash \Gamma_1' \wknto \Gamma_1$, and
  $\Psi ; \Theta ; \Delta ; \Omega ; \Gamma' \vdash e_1' \infers [J]\tau_1 \gens \Phi_1',\Gamma_1'$.
  Since $\Psi ; \Theta ; \Delta \vdash \Gamma_1',x : \tau_1 \wknto \Gamma_1, x : \tau_1$,
  we have by IH that there are $e_2'$, $\Phi_2'$, $\Gamma_2'$ such that
  $|e_2'| = |e_2|$,
  $\Theta ; \Delta \vDash \Phi_2'$,
  $\Psi ; \Theta ; \Delta \vdash \Gamma_2' \wknto (\Gamma_1',x : \tau_1) \setminus (\Gamma_1, x : \tau_1)$,
  $\Psi ; \Theta ; \Delta \vdash \Gamma_2' \wknto \Gamma_2$, and
  $\Psi ; \Theta ; \Delta ; \Omega' ; \Gamma_1', x : \tau_1 \vdash e_2' \checks \M \, \phi(I,\vec{p} + \texttt{const}(J)) \, \tau_2 \gens \Phi_2', \Gamma_2'$.
  By AT-ReleaseCont,
  $\Psi ; \Theta ; \Delta ; \Omega' ; \Gamma' \vdash \texttt{release } x = e_1' \texttt{ in }e_2' \checks \M \, \phi(I,\vec{p}) \, \tau_2 \gens \Phi_1' \wedge \Phi_2', \Gamma_2' \setminus \{x\}$.
  Of course, $|\texttt{release } x = e_1' \texttt{ in }e_2'| = \texttt{release } x = |e_1'| \texttt{ in }|e_2'| = \texttt{release } x = |e_1| \texttt{ in }|e_2|
  = |\texttt{release } x = e_1 \texttt{ in }e_2|$.
  Also, $\Theta ; \Delta \vDash \Phi_1' \wedge \Phi_2'$.
  $\Psi ; \Theta ; \Delta \vdash \Gamma_2' \wknto \Gamma_2$ holds by IH.
  It remains to show that $\Psi ; \Theta ; \Delta \vdash \Gamma_2' \setminus \{x\} \wknto \Gamma' \setminus \Gamma$.
  This follows by considering the remaining weakening judgments from the IHs with \autoref{thm:ctx-sub-subset2}.
  This completes (1). For (2), as usual, we apply AT-Anno.
  
  \item[(AT-Shift)] Suppose $\Psi ; \Theta ; \Delta ; \Omega ; \Gamma \vdash \texttt{shift}(e) \checks \M \, \phi(I,\vec{q}) \, \tau \gens (I \geq 1) \wedge \Phi, \Gamma''$ from
  $\Psi ; \Theta ; \Delta  ; \Omega ; \Gamma \vdash e \checks \M \, \phi(I - 1,\lhd \vec{q}) \, \tau \gens \Phi, \Gamma''$, with
  $\Theta ; \Delta \vDash (I \geq 1) \wedge \Phi$,
  $\Psi ; \Theta ; \Delta \vdash \Gamma' \wknto \Gamma$,
  $\Psi ; \Theta ; \Delta \vdash \Omega' \wknto \Omega$.
  By IH, there are $e'$, $\Phi'$, $\Gamma'''$ such that
  $|e'| = |e|$,
  $\Theta ; \Delta \vDash \Phi'$,
  $\Psi ; \Theta ; \Delta \vdash \Gamma''' \wknto \Gamma' \setminus \Gamma$,
  $\Psi ; \Theta ; \Delta \vdash \Gamma''' \wknto \Gamma''$,
  $\Psi ; \Theta ; \Delta ; \Omega' ; \Gamma' \vdash e' \checks \M \, \phi(I-1,\lhd \vec{q}) \, \tau \gens \Phi', \Gamma'''$.
  By AT-Shift,
  $\Psi ; \Theta ; \Delta ; \Omega' ; \Gamma' \vdash \texttt{shift}(e') \checks \M \, \phi(I,\vec{q}) \, \tau \gens (I \geq 1) \wedge \Phi', \Gamma'''$,
  which completes (1). For (2), we apply AT-Anno.
  
  \item[(AT-CImpI)] Suppose $\Psi ; \Theta ; \Delta ; \Omega ; \Gamma \vdash \Lambda .e \checks (\Phi' \Rightarrow \tau) \gens (\Phi' \to \Phi),\Gamma''$ from
  $\Psi ; \Theta ; \Delta, \Phi'; \Omega ; \Gamma \vdash e \checks \tau \gens \Phi,\Gamma''$, with
  $\Theta ; \Delta \vDash \Phi$,
  $\Psi ; \Theta ; \Delta \vdash \Gamma' \wknto \Gamma$,
  $\Psi ; \Theta ; \Delta \vdash \Omega' \wknto \Omega$.
  Using \autoref{thm:cxt-sub-wkn}, we have by IH that there are $e'$, $\Phi''$, $\Gamma'''$,
  $|e'| = |e|$,
  $\Theta ; \Delta, \Phi' \vDash \Phi''$,
  $\Psi ; \Theta ; \Delta, \Phi' \vdash \Gamma''' \wknto \Gamma' \setminus \Gamma$,
  $\Psi ; \Theta ; \Delta, \Phi' \vdash \Gamma''' \wknto \Gamma''$, and
  $\Psi ; \Theta ; \Delta, \Phi' ; \Omega' ; \Gamma' \vdash e \checks \tau \gens \Phi'', \Gamma'''$.
  By AT-CImpI,
  $\Psi ; \Theta ; \Delta ; \Omega' ; \Gamma' \vdash \Lambda .e' \checks (\Phi' \Rightarrow \tau) \gens (\Phi' \to \Phi''),\Gamma'''$.
  By definition, $|\Lambda .e'| = \Lambda.|e'| = \Lambda.|e| = |\Lambda.e|$.
  Next, since $\Theta ; \Delta, \Phi' \vDash \Phi''$, we have $\Theta ; \Delta \vDash \Phi' \to \Phi''$.
  The two weakenings again follow by \autoref{thm:ctx-sub-subset2}.
  This completes (1), and (2) follows by AT-Anno.
  
  \item[(AT-CImpE)] Suppose $\Psi ; \Theta ; \Delta ; \Omega ; \Gamma \vdash e \{\} \infers \tau \gens \Phi \wedge \Phi',\Gamma''$ from
  $\Psi ; \Theta ; \Delta ; \Omega ; \Gamma \vdash e \infers (\Phi' \Rightarrow \tau) \gens \Phi,\Gamma''$, with
  $\Theta ; \Delta \vDash \Phi \wedge \Phi'$,
  $\Psi ; \Theta ; \Delta \vdash \Gamma' \wknto \Gamma$, and
  $\Psi ; \Theta ; \Delta \vdash \Omega' \wknto \Omega$.
  By IH, there are $e'$, $\Phi''$. $\Gamma'''$ such that
  $|e'| = |e|$,
  $\Theta ; \Delta \vDash \Phi''$.
  $\Psi ; \Theta ; \Delta \vdash \Gamma''' \wknto \Gamma' \setminus \Gamma$,
  $\Psi ; \Theta ; \Delta \vdash \Gamma''' \wknto \Gamma''$, and
  $\Psi ; \Theta ; \Delta ; \Omega' ; \Gamma' \vdash e' \infers (\Phi' \Rightarrow \tau) \gens \Phi'',\Gamma'''$.
  By AT-CImpE,
  $\Psi ; \Theta ; \Delta ; \Omega' ; \Gamma' \vdash e' \{\} \infers \tau \gens \Phi'' \wedge \Phi',\Gamma'''$,
  which completes the proof of (2). For (1), we use \autoref{thm:subty-refl} to get some $\Phi_1$ with $\Theta ; \Delta \vDash \Phi_1$
  such that $\Psi ; \Theta ; \Delta \vdash \tau \subty \tau : \star \gens \Phi_1$.
  Then, by AT-Sub,
  $\Psi ; \Theta ; \Delta ; \Omega' ; \Gamma' \vdash e' \{\} \checks \tau \gens \Phi'' \wedge \Phi' \wedge \Phi_3,\Gamma'''$, as required.
  
  \item[(AT-CAndI)] Suppose $\Psi ; \Theta ; \Delta ; \Omega ; \Gamma \vdash <e> \checks \Phi' \amp \tau \gens \Phi \wedge \Phi',\Gamma''$ from
  $\Psi ; \Theta ; \Delta ; \Omega ; \Gamma \vdash e \checks \tau \gens \Phi,\Gamma''$, with
  $\Theta ; \Delta \vDash \Phi \wedge \Phi'$,
  $\Psi ; \Theta ; \Delta \vdash \Gamma' \wknto \Gamma$, and
  $\Psi ; \Theta ; \Delta \vdash \Omega' \wknto \Omega$.
  By IH, there are $e'$, $\Phi''$, $\Gamma'''$ such that
  $|e'| = |e|$,
  $\Theta ; \Delta \vDash \Phi''$,
  $\Psi ; \Theta ; \Delta \vdash \Gamma''' \wknto \Gamma' \setminus \Gamma$,
  $\Psi ; \Theta ; \Delta \vdash \Gamma''' \wknto \Gamma''$, and
  $\Psi ; \Theta ; \Delta ; \Omega' ; \Gamma' \vdash e' \checks \tau \gens \Phi'', \Gamma'''$.
  By AT-CAndI,
  $\Psi ; \Theta ; \Delta ; \Omega' ; \Gamma' \vdash <e'> \checks \Phi' \amp \tau \gens \Phi'' \wedge \Phi',\Gamma'''$,
  which completes (1). For (2), one use of AT-Anno suffices.
  
  \item[(AT-CAndE)] Suppose
  $\Psi ; \Theta ; \Delta ; \Omega ; \Gamma \vdash \texttt{clet } x = e_1 \texttt{ in } e_2 \checks \tau' \gens \Phi_1 \wedge (\Phi \to \Phi_2),\Gamma_2 \setminus \{x : \tau\}$ from
  $\Psi ; \Theta ; \Delta ; \Omega ; \Gamma \vdash e_1 \infers \Phi \amp \tau \gens \Phi_1,\Gamma_1$ and
  $\Psi ; \Theta ; \Delta, \Phi ; \Omega ; \Gamma_1, x : \tau \vdash e_2 \checks \tau' \gens \Phi_2, \Gamma_2$, with
  $\Theta ; \Delta \vDash \Phi_1 \wedge (\Phi \to \Phi_2)$,
  $\Psi ; \Theta ; \Delta \vdash \Gamma' \wknto \Gamma$, and
  $\Psi ; \Theta ; \Delta \vdash \Omega' \wknto \Omega$.
  By IH, there are $e_1'$, $\Phi_1'$, $\Gamma_1'$ such that
  $|e_1'| = |e_1|$,
  $\Theta ; \Delta \vDash \Phi_1'$,
  $\Psi ; \Theta ; \Delta \vdash \Gamma_1' \wknto \Gamma' \setminus \Gamma$,
  $\Psi ; \Theta ; \Delta \vdash \Gamma_1' \wknto \Gamma_1$, and
  $\Psi ; \Theta ; \Delta ; \Omega' ; \Gamma' \vdash e_1' \infers \Phi \amp \tau \gens \Phi_1',\Gamma_1'$.
  Since $\Psi ; \Theta ; \Delta \vdash \Gamma_1' \wknto \Gamma_1$, we also have by \autoref{thm:ctx-sub-wkn}
  $\Psi ; \Theta ; \Delta, \Phi \vdash \Gamma_1', x : \tau \wknto \Gamma_1, x : \tau$,
  and so by IH, there are $e_2'$, $\Phi_2'$, $\Gamma_2'$ such that
  $|e_2'| = |e_2|$,
  $\Theta ; \Delta, \Phi \vDash \Phi_2'$,
  $\Psi ; \Theta ; \Delta, \Phi \vdash \Gamma_2' \wknto \Gamma_1' \setminus \Gamma_1$,
  $\Psi ; \Theta ; \Delta, \Phi \vdash \Gamma_2' \wknto \Gamma_2$, and
  $\Psi ; \Theta ; \Delta, \Phi ; \Omega' ; \Gamma_1', x : \tau \vdash e_2' \checks \tau' \gens \Phi_2', \Gamma_2'$.
  By AT-CAndE,
  $\Psi ; \Theta ; \Delta ; \Omega' ; \Gamma' \vdash \texttt{clet } x = e_1' \texttt{ in } e_2' \checks \tau' \gens \Phi_1' \wedge (\Phi \to \Phi_2'),\Gamma_2' \setminus \{x : \tau\}$.
  As usual, we have that $|\texttt{clet } x = e_1' \texttt{ in } e_2'| = |\texttt{clet } x = e_1 \texttt{ in } e_2|$.
  Combining the satisfactions from the premises, we have that $\Theta ; \Delta \vDash \Phi_1' \wedge (\Phi \to \Phi_2')$.
  The fact that $\Psi ; \Theta ; \Delta, \Phi \vdash \Gamma_2' \wknto \Gamma_2$ is immediate from IH, 
  and $\Psi ; \Theta ; \Delta, \Phi \vdash \Gamma_2' \wknto \Gamma' \setminus \Gamma$ follows as usual by considering the rest
  of the weakening judgments with \autoref{thm:ctx-sub-subset2}. This completes (1). For (2), we apply AT-Anno.
   
  \item[(AT-Sub)] Suppose $\Psi ; \Theta ; \Delta ; \Omega ; \Gamma \vdash e \checks \tau \gens \Phi_1 \wedge \Phi_2,\Gamma''$ from
  $\Psi ; \Theta ; \Delta ; \Omega ; \Gamma \vdash e \infers \tau' \gens \Phi_1,\Gamma''$ and
  $\Psi;\Theta;\Delta \vdash \tau' \subty \tau : \star \gens \Phi_2$, with
  $\Theta ; \Delta \vDash \Phi_1 \wedge \Phi_2$,
  $\Psi ; \Theta ; \Delta \vdash \Gamma' \wknto \Gamma$, and
  $\Psi ; \Theta ; \Delta \vdash \Omega' \wknto \Omega$.
  By IH, there are $e'$, $\Phi_1'$, $\Gamma'''$ such that
  $|e'| = |e|$,
  $\Theta ; \Delta \vDash \Phi_1'$,
  $\Psi ; \Theta ; \Delta \vdash \Gamma''' \wknto \Gamma' \setminus \Gamma$,
  $\Psi ; \Theta ; \Delta \vdash \Gamma''' \wknto \Gamma''$, and
  $\Psi ; \Theta ; \Delta ; \Omega' ; \Gamma' \vdash e' \infers \tau' \gens \Phi_1', \Gamma'''$.
  By AT-Sub, $\Psi ; \Theta ; \Delta ; \Omega' ; \Gamma' \vdash e' \checks \tau \gens \Phi_1' \wedge \Phi_2,\Gamma'''$,
  which completes (1). For (2), one use of AT-Anno suffices.
  
  \item[(AT-Anno)] Suppose $\Psi ; \Theta ; \Delta ; \Omega ; \Gamma \vdash (e : \tau) \infers \tau \gens \Phi,\Gamma''$ from 
  $\Psi ; \Theta ; \Delta ; \Omega ; \Gamma \vdash e \checks \tau \gens \Phi,\Gamma''$ with
  $\Theta ; \Delta \vDash \Phi$,
  $\Psi ; \Theta ; \Delta \vdash \Gamma' \wknto \Gamma$, and
  $\Psi ; \Theta ; \Delta \vdash \Omega' \wknto \Omega$.
  By IH, there are $e'$, $\Phi'$, $\Gamma'''$ such that
  $|e| = |e'|$,
  $\Theta ; \Delta \vDash \Phi'$,
  $\Psi ; \Theta ; \Delta \vdash \Gamma''' \wknto \Gamma' \setminus \Gamma$,
  $\Psi ; \Theta ; \Delta \vdash \Gamma''' \wknto \Gamma''$, and
  $\Psi ; \Theta ; \Delta ; \Omega' ; \Gamma' \vdash e' \checks \tau \gens \Phi',\Gamma'''$.
  By AT-Anno, $\Psi ; \Theta ; \Delta ; \Omega' ; \Gamma' \vdash (e' : \tau) \infers \tau \gens \Phi',\Gamma'''$, which completes (2)
  since $|(e' : \tau)| = |e'| = |e| = |(e : \tau)|$. For (1), we use \autoref{thm:subty-refl} to get that $\Psi ; \Theta ; \Delta \vdash \tau \subty  \tau : \star \gens \Phi''$ with $\Theta ; \Delta \vDash \Phi''$, and so by AT-Sub, we have that $\Psi ; \Theta ; \Delta ; \Omega' ; \Gamma' \vdash (e' : \tau) \checks \tau \gens \Phi' \wedge \Phi'',\Gamma'''$, completing (1).
 
  
  
\end{itemize}
\end{proof}

\tycheckcompl*
\begin{proof}
By induction on the derivation of $\Psi;\Theta;\Delta;\Omega;\Gamma \vdash e : \tau$, we prove both claims simultaneously.
In all cases, the erasure property is immediate from the inductive hypotheses-- we will elide this bit of the proof.
\begin{itemize}
  \item[(T-Var-1)] Suppose $\Psi ; \Theta ; \Delta ; \Omega ; \Gamma \vdash x : \tau$ from $x : \tau \in \Gamma$. By AT-Var-1, $\Psi ; \Theta ; \Delta ; \Omega ; \Gamma \vdash x \infers \tau \gens \top,\Gamma \setminus \{x : \tau\}$. Of course, $\Theta ; \Delta \vDash \top$, and $|x| = x$, so (1) is complete. By Theorem~\ref{thm:subty-refl}, $\Psi ; \Theta ; \Delta \vdash \tau \subty \tau : \star \gens \Phi$ with $\Theta ; \Delta \vDash \Phi$. By AT-Sub,  $\Psi ; \Theta ; \Delta ; \Omega ; \Gamma \vdash x \checks \tau \gens \top,\Gamma \setminus \{x : \tau\}$, as required for (2).
  
  \item[(T-Var-2)] Immediate.
  \item[(T-Unit)] Immediate.
  \item[(T-Base)] Immediate.
  \item[(T-Absurd)] Suppose $\Psi ; \Theta ; \Delta ; \Omega ; \Gamma \vdash \texttt{absurd} : \tau$ from $\Theta ; \Delta \vDash \bot$. By AT-Absurd,
  $\Psi ; \Theta ; \Delta ; \Omega ; \Gamma \vdash \texttt{absurd} \checks \tau \gens \bot,\Gamma$. Since $\Theta ; \Delta \vDash \bot$, this completes (1). For (2),
  use AT-Anno to get $\Psi ; \Theta ; \Delta ; \Omega ; \Gamma \vdash (\texttt{absurd} : \tau) \infers \tau \gens \bot,\Gamma$
  
  \item[(T-Nil)] Suppose $\Psi ; \Theta ; \Delta ; \Omega ; \Gamma\vdash \texttt{nil} : L^I \tau$ by way of
  $\Theta ; \Delta \vdash I : \mathbb{N}$ and
  $\Theta;\Delta \vDash I = 0$.
  By Theorem~\ref{thm:sort-compl}, there is a $\Phi$ such that
  $\Theta ; \Delta \vDash \Phi$, and
  $\Theta ; \Delta \vdash I : \mathbb{N} \gens \Phi$.
  By AT-Nil, $\Psi ; \Theta ; \Delta ; \Omega ; \Gamma\vdash \texttt{nil} \checks L^I \tau \gens \Phi \wedge (I = 0),\Gamma$.
  Of course, $\Theta ; \Delta \vDash \Phi \wedge (I = 0)$,
  completing (1). (2) follows by AT-Anno.
  
  \item[(T-Cons)] Suppose
  $\Psi ; \Theta ; \Delta ; \Omega ; \Gamma_1, \Gamma_2\vdash e_1 :: e_2 : L^I \tau$ by way of
  $\Psi ; \Theta ; \Delta ; \Omega ; \Gamma_1\vdash e_1 : \tau$,
  $\Psi ; \Theta ; \Omega ; \Gamma_2\vdash e_2 : L^{I-1} \tau$, and
  $\Theta ; \Delta \vDash I \geq 1$.
  By IH, there are $e_1'$, $\Phi_1$, $\Gamma_1'$ such that
  $|e_1'| = e_1$,
  $\Theta ; \Delta \vDash \Phi_1'$, and
  $\Psi ; \Theta ; \Delta ; \Omega ; \Gamma_1 \vdash e_1' \checks \tau \gens \Phi_1,\Gamma_1'$.
  Since $\Psi ; \Theta ; \Delta \vdash \Gamma_1,\Gamma_2 \wknto \Gamma_1$,
  by Theorem~\ref{thm:admits-weaken} we have $e_1''$, $\Phi_1'$, $\Gamma_1''$ such that
  $|e_1''| = |e_1'|$,
  $\Theta ; \Delta \vDash \Phi_1'$,
  $\Psi ; \Theta ; \Delta \vdash \Gamma_1'' \wknto (\Gamma_1,\Gamma_2) \setminus \Gamma_1$, and
  $\Psi ; \Theta ; \Delta ; \Omega ; \Gamma_1,\Gamma_2 \vdash e_1'' \checks \tau \gens \Phi_1',\Gamma_1''$.
  Since $(\Gamma_1,\Gamma_2) \setminus \Gamma_1 = \Gamma_2$,
  we have that
  $\Psi ; \Theta ; \Delta \vdash \Gamma_1'' \wknto \Gamma_2$.
  By IH, there are $e_2'$, $\Phi_2$, $\Gamma_2'$ such that
  $|e_2'| = e_2$,
  $\Theta ; \Delta \vDash \Phi_2$, and
  $\Psi ; \Theta ; \Delta ; \Omega ; \Gamma_2 \vdash e_2' \checks L^{I-1} \tau \gens \Phi_2,\Gamma_2'$.
  But again by Theorem~\ref{thm:admits-weaken}, we have $e_2''$, $\Phi_2'$, $\Gamma_2''$ such that
  $|e_2''| = |e_2'|$,
  $\Theta ; \Delta \vDash \Phi_2'$, and
  $\Psi ; \Theta ; \Delta ; \Omega ; \Gamma_1'' \vdash e_2' \checks L^{I-1} \tau \gens \Phi_2',\Gamma_2''$.
  By AT-Cons, we have that
  $\Psi ; \Theta ; \Delta ; \Omega ; \Gamma_1,\Gamma_2 \vdash e_1 :: e_2 \checks L^I \tau \gens \Phi_1' \wedge \Phi_2' \wedge (I \geq 1), \Gamma_2''$.
  But, $\Theta ; \Delta \vDash \Phi_1' \wedge \Phi_2' \wedge (I \geq 1)$ and so this completes (1). (2) follows immediately by AT-Anno.
  
  
  \item[(T-Match)] Suppose
  $\Psi ; \Theta ; \Delta ; \Omega ; \Gamma_1,\Gamma_2\vdash \texttt{match}(e,e_1,h.t.e_2) : \tau'$ by way of
  $\Psi ; \Theta ; \Delta ; \Omega ; \Gamma_1\vdash e : L^I \tau$,
  $\Psi ; \Theta ; \Delta, I = 0 ; \Omega ; \Gamma_2\vdash e_1 : \tau'$, and
  $\Psi ; \Theta ; \Delta, I \geq 1; \Omega ; \Gamma_2, h : \tau, t : L^I \tau \vdash e_2 : \tau'$.
  By IH, there are $e'$, $\Phi$, $\Gamma_1'$ such that
  $|e'| = e$,
  $\Theta ; \Delta \vDash \Phi$, and
  $\Psi ; \Theta ; \Delta ; \Omega ; \Gamma_1 \vdash e' \infers L^I \tau \gens \Phi, \Gamma_1'$.
  Since $\Psi ; \Theta ; \Delta \vdash \Gamma_1,\Gamma_2 \wknto \Gamma_1$,
  we have by Theorem~\ref{thm:admits-weaken} that there are $e''$, $\Phi'$, $\Gamma_1''$ such that
  $|e''| = |e'|$,
  $\Theta ; \Delta \vDash \Phi'$,
  $\Psi ; \Theta ; \Delta \vdash \Gamma_1'' \wknto \Gamma_2$, and
  $\Psi ; \Theta ; \Delta ; \Omega ; \Gamma_1,\Gamma_2 \vdash e'' \infers L^I \tau \gens \Phi', \Gamma_1''$.
  Again by IH, there are $e_1'$, $\Phi_1$, $\Gamma_2'$ such that
  $|e_1'| = e_1$,
  $\Theta ; \Delta, I = 0 \vDash \Phi_1$, and
  $\Psi ; \Theta ; \Delta, I = 0 ; \Omega ; \Gamma_2 \vdash e_1' \checks \tau' \gens \Phi_1,\Gamma_2'$.
  By Theorem~\ref{thm:ctx-sub-wkn},
  $\Psi ; \Theta ; \Delta, I = 0 \vdash \Gamma_1'' \wknto \Gamma_2$.
  Then, by Theorem~\ref{thm:admits-weaken}, there are $e_1''$, $\Phi_1'$, $\Gamma_2''$ such that
  $|e_1''| = |e_1'|$,
  $\Theta ; \Delta, I = 0 \vDash \Phi_1'$, and
  $\Psi ; \Theta ; \Delta, I = 0 ; \Omega ; \Gamma_1'' \vdash e_2'' \checks \tau' \gens \Phi_1',\Gamma_2''$.
  Invoking the IH once more, we have $e_2'$, $\Phi_2$, $\Gamma_3'$ such that
  $|e_2'| = e_2$,
  $\Theta ; \Delta, I \geq 1 \vdash \Phi_2$, and
  $\Psi ; \Theta ; \Delta, I \geq 1 ; \Omega ; \Gamma_2, h : \tau, t : L^I \tau \vdash e_2' \checks \tau' \gens \Phi_2,\Gamma_3'$.
  By Theorem~\ref{thm:ctx-sub-wkn} and Theorem~\ref{thm:ctx-sub-subset2},
  $\Psi ; \Theta ; \Delta, I \geq 1 \vdash \Gamma_1'', h : \tau, t : L^I \tau \wknto \Gamma_2, h : \tau, t : L^I \tau$.
  Then, by Theorem~\ref{admits-weaken}, there are $e_2''$, $\Phi_2'$, $\Gamma_3''$ such that
  $|e_2''| = |e_2'|$,
  $\Theta ; \Delta, I \geq 1 \vDash \Phi_2'$, and
  $\Psi ; \Theta ; \Delta, I \geq 1; \Omega ; \Gamma_1'', h : \tau, t : L^I \tau \vdash e_2' \checks \tau' \gens \Phi_2',\Gamma_3''$.
  By AT-Match,
  $\Psi ; \Theta ; \Delta ; \Omega ; \Gamma_1,\Gamma_2\vdash \texttt{match}(e,e_1,h.t.e_2) \checks \tau' \gens \Phi' \wedge (I = 0 \to \Phi_1') \wedge (I \geq 1 \to \Phi_2'), \Gamma_2'' \cap (\Gamma_3'' \setminus \{x,y\})$.
  Since $\Theta ; \Delta \vDash \Phi'$,  $\Theta ; \Delta, I = 0 \vDash \Phi_1'$, and $\Theta ; \Delta, I \geq 1 \vDash \Phi_2'$, we have equivalently that
  $\Theta ; \Delta \vDash \Phi' \wedge (I = 0 \to \Phi_1') \wedge (I \geq 1 \to \Phi_2')$. This completes (1), and (2) follows by AT-Anno.
  
  \item[(T-ExistI)] Suppose $\Psi ; \Theta ; \Delta ; \Omega ; \Gamma\vdash \texttt{pack}[I](e) : \exists i:S.\tau$ from
  $\Theta ; \Delta \vdash I : S$, and
  $\Psi ; \Theta ; \Delta ; \Omega ; \Gamma\vdash e : \tau[I/i]$.
  By Theorem~\ref{thm:idx-compl}, there is some $\Phi_1$ such that
  $\Theta ; \Delta \vDash \Phi_1$, and
  $\Theta ; \Delta \vdash I : S \gens \Phi_1$.
  By IH, there are $e'$, $\Phi_2$, $\Gamma'$ such that
  $\Psi ; \Theta ; \Delta ; \Omega ; \Gamma \vdash e' \checks \tau[I/i] \gens \Phi_2,\Gamma'$.
  By AT-ExistI,
  $\Psi ; \Theta ; \Delta ; \Omega ; \Gamma\vdash \texttt{pack}[I](e') : \exists i:S.\tau \gens \Phi_1 \wedge \Phi_2,\Gamma'$.
  Clearly, $|\texttt{pack}[I](e')| = \texttt{pack}[I](|e'|) = \texttt{pack}[I](e)$, and $\Theta ; \Delta \vDash \Phi_1 \wedge \Phi_2$.
  This completes (1), and (2) follows immediately by AT-Anno.
  
  \item[(T-ExistE)] Suppose
  $\Psi ; \Theta ; \Delta ; \Omega ; \Gamma_1,\Gamma_2\vdash \texttt{unpack } (i,x) = e_1 \texttt{ in } e_2 : \tau'$ from
  $\Psi ; \Theta ; \Delta ; \Omega ; \Gamma_1\vdash e_1 : \exists i : S.\tau$ and
  $\Psi ; \Theta, i : S ; \Delta ; \Omega ; \Gamma_2, x : \tau \vdash e_2 : \tau'$.
  By IH, there are $e_1'$, $\Phi_1$, $\Gamma_1'$ such that
  $|e_1'| = e_1$,
  $\Theta ; \Delta \vDash \Phi_1$, and
  $\Psi ; \Theta ; \Delta ; \Omega ; \Gamma_1 \vdash e_1' \infers \exists i : S. \tau \gens \Phi_1,\Gamma_1'$.
  Since $\Psi ; \Theta ; \Delta \vdash \Gamma_1,\Gamma_2 \wknto \Gamma_1$,
  we have by Theorem~\ref{thm:admits-weaken} that there are $e_1''$, $\Phi_1'$, $\Gamma_1''$ such that
  $|e_1''| = |e_1'|$,
  $\Theta ; \Delta \vDash \Phi_1'$,
  $\Psi ; \Theta ; \Delta \vdash \Gamma_1'' \wknto \Gamma_2$, and
  $\Psi ; \Theta ; \Delta ; \Omega ; \Gamma_1,\Gamma_2 \vdash e_1'' \infers \exists i : S. \tau \gens \Phi_1',\Gamma_1''$.
  By IH, there are $e_2'$, $\Phi_2$, $\Gamma_2'$ such that
  $|e_2'| = e_2$,
  $\Theta, i : S ; \Delta \vDash \Phi_2$, and
  $\Psi ; \Theta, i : S ; \Delta ; \Omega ; \Gamma_2,x : \tau \vdash e_2' \checks \tau' \gens \Phi_2,\Gamma_2'$.
  Since $\Psi ; \Theta ; \Delta \vdash \Gamma_1'',x : \tau \wknto \Gamma_2, x : \tau$,
  we have by Theorem~\ref{thm:admits-weaken} that there are $e_2''$, $\Phi_2'$, $\Gamma_2''$ such that
  $|e_2''| = |e_2'|$,
  $\Theta, i : S ; \Delta \vDash \Phi_2'$, and
  $\Psi ; \Theta, i : S ; \Delta ; \Omega ; \Gamma_1'', x: \tau \vdash e_2'' \checks \tau' \gens \Phi_2',\Gamma_2''$.
  By AT-ExistE,
  $\Psi ; \Theta ; \Delta ; \Omega ; \Gamma_1,\Gamma_2\vdash \texttt{unpack } (i,x) = e_1' \texttt{ in } e_2' : \tau' \gens \Phi_1' \wedge (\forall i : S. \Phi_2'),\Gamma_2'' \setminus \{x\}$,
  which completes (1). For (2), a single use of AT-Anno suffices.
  
  \item[(T-Lam)] Suppose $\Psi ; \Theta ; \Delta ; \Omega ; \Gamma\vdash \lambda x.e : \tau_1 \loli \tau_2$ from $\Psi ; \Theta ; \Delta ; \Omega ; \Gamma, x : \tau_1 \vdash e : \tau_2$. By IH, there are $e'$, $\Phi$, $\Gamma'$ so that $|e'| = e$, $\Theta ; \Delta \vDash \Phi$, and $\Psi ; \Theta ; \Delta ; \Omega ; \Gamma, x : \tau_1 \vdash e' \checks \tau_2 \gens \Phi,\Gamma'$. By AT-Lam, $\Psi ; \Theta ; \Delta ; \Omega ; \Gamma \vdash \lambda x. e' \checks \tau_1 \loli \tau_2 \tau_2 \gens \Phi,\Gamma' \setminus \{x : \tau_1\}$. But, $|\lambda x. e'| = \lambda x. |e'| = \lambda x.e$, which completes the proof of (1). For (2), AT-Anno gives that $\Psi ; \Theta ; \Delta ; \Omega ; \Gamma \vdash (\lambda x. e' : \tau_1 \loli \tau_2) \infers \tau_1 \loli \tau_2 \gens \Phi,\Gamma' \setminus \{x : \tau_1\}$, as required.
  
  \item[(T-App)] Suppose $\Psi ; \Theta ; \Delta ; \Omega ; \Gamma_1,\Gamma_2\vdash e_1 \, e_2 :  \tau_2$ from
  $\Psi ; \Theta ; \Delta ; \Omega ; \Gamma_1\vdash e_1 : \tau_1 \loli \tau_2$ and
  $\Psi ; \Theta ; \Delta ; \Omega ; \Gamma_2\vdash e_2 : \tau_1$.
  By IH, there are $e_1'$, $\Gamma_1'$, $\Phi_1$ such that
  $|e_1'| = e_1$,
  $\Theta ; \Delta \vDash \Phi_1$, and
  $\Psi ; \Theta ; \Delta ; \Omega ; \Gamma_1 \vdash e_1' \infers \tau_1 \loli \tau_2 \gens \Phi_1,\Gamma_1'$.
  Since $\Psi ; \Theta ; \Delta \vdash \Gamma_1,\Gamma_2 \wknto \Gamma_1$,
  we have by Theorem~\ref{thm:admits-weaken} that there are $e_1''$, $\Phi_1'$, $\Gamma_1''$ such that
  $|e_1''| = |e_1'|$,
  $\Theta ; \Delta \vDash \Phi_1'$,
  $\Psi ; \Theta ; \Delta \vdash \Gamma_1'' \wknto \Gamma_2$, and
  $\Psi ; \Theta ; \Delta ; \Omega ; \Gamma_1,\Gamma_1 \vdash e_1'' \infers \tau_1 \loli \tau_2 \gens \Phi_1',\Gamma_1''$.
  Then, by IH, there are $e_2'$, $\Gamma_2'$, and $\Phi_2$ such that
  $|e_2'| = e_2$,
  $\Theta ; \Delta \vDash \Phi_2$,
  $\Psi ; \Theta ; \Delta ; \Omega ; \Gamma_2 \vdash e_2' \checks \tau_1 \gens \Phi_2,\Gamma_2'$.
  But then by Theorem~\ref{thm:admits-weaken}, there are $e_2''$, $\Gamma_2''$, and $\Phi_2'$ such that
  $|e_2''| = |e_2'|$,
  $\Theta ; \Delta \vDash \Phi_2'$, and
  $\Psi ; \Theta ; \Delta ; \Omega; \Gamma_1'' \vdash e_2'' \checks \tau_1 \gens \Phi_2',\Gamma_2''$.
  Then, by AT-App, we have that
  $\Psi ; \Theta ; \Delta ; \Omega ; \Gamma_1,\Gamma_2 \vdash e_1'' \, e_2'' \infers \tau_2 \gens \Phi_1' \wedge \Phi_2',\Gamma_2''$.
  This completes (2). For (1), 
  we apply Theorem~\ref{thm:subty-refl} to find that there is some $\Phi_3$ such that $\Theta ; \Delta \vDash \Phi_3$, and
  $\Psi ; \Theta ; \Delta \vdash \tau_2 \subty \tau_2 : \star \gens \Phi_3$.
  Then, by AT-Sub, we have
  $\Psi ; \Theta ; \Delta ; \Omega ; \Gamma_1,\Gamma_2 \vdash e_1'' \, e_2'' \checks \tau_2 \gens \Phi_1' \wedge \Phi_2' \wedge \Phi_3,\Gamma_2''$
  as required for (2).
  
  \item[(T-TensorI)] Suppose $\Psi ; \Theta ; \Delta ; \Omega ; \Gamma_1,\Gamma_2\vdash \angles{e_1,e_2} ; \tau_1 \otimes \tau_2$ from $\Psi ; \Theta ; \Delta ; \Omega ; \Gamma_1\vdash e_1 : \tau_1$ and $\Psi ; \Theta ; \Delta ; \Omega ; \Gamma_2\vdash e_2 : \tau_2$.
  By IH, there are $e_1'$, $\Phi_1$, $\Gamma_1'$ such that $|e_1'| = e_1$, $\Theta ; \Delta \vDash \Phi_1$, and $\Psi ; \Theta ; \Delta ; \Omega ; \Gamma_1 \vdash e_1' \checks \tau_1 \gens \Phi_1,\Gamma_1'$. By Theorem~\ref{thm:admits-weaken}, there are $e_1''$, $\Phi_1'$, $\Gamma_1''$ such that $|e_1''| = |e_1'|$, $\Theta ; \Delta \vDash \Phi_1'$, $\Psi ; \Theta ; \Delta \vdash \Gamma_1'' \wknto (\Gamma_1,\Gamma_2) \setminus \Gamma_1$ and $\Psi ; \Theta ; \Delta ;\Omega ; \Gamma_1,\Gamma_2 \vdash e_1'' \checks \tau_2 \gens \Phi_1',\Gamma_1''$. But, $(\Gamma_1,\Gamma_2) \setminus \Gamma_1 = \Gamma_2$, since $\Gamma_1$ and $\Gamma_2$ are disjoint, and so we have that $\Psi ; \Theta ; \Delta \vdash \Gamma_1'' \wknto \Gamma_2$.
  By IH, there are $e_2'$, $\Phi_2$, $\Gamma_2'$ such that $|e_2'| = e_2$, $\Theta ; \Delta \vDash \Phi_2$,  and $\Psi ; \Theta ; \Delta ; \Omega ; \Gamma_2 \vdash e_2' \checks \tau_2 \gens \Phi_2,\Gamma_2'$. But by Theorem~\ref{thm:admits-weaken}, there are $e_2''$, $\Phi_2'$, $\Gamma_2''$ such that $|e_2''| = |e_2'|$, $\Theta ; \Delta \vDash \Phi_2'$, and $\Psi ; \Theta ; \Delta ; \Omega ; \Gamma_1'' \vdash e_2'' \checks \tau_2 \gens \Phi_2', \Gamma_2''$. By AT-TensorI,
  $\Psi ; \Theta ; \Delta ; \Omega ; \Gamma_1,\Gamma_2 \vdash \angles{e_1'',e_2''} \checks \tau_1 \otimes \tau_2 \gens \Phi_1' \wedge \Phi_2', \Gamma_2''$.
  We then verify the conditions: $|\angles{e_1'',e_2''}| = \angles{|e_1''|,|e_2''|} = \angles{|e_1'|,|e_2'|} = \angles{e_1,e_2}$, and also $\Theta ; \Delta \vDash \Phi_1' \wedge \Phi_2'$, completing (1). For (2), AT-Anno gives that $\Psi ; \Theta ; \Delta ; \Omega ; \Gamma_1,\Gamma_2 \vdash (\angles{e_1'',e_2''} : \tau_1 \otimes \tau_2) \infers \tau_1 \otimes \tau_2 \gens \Phi_1' \wedge \Phi_2', \Gamma_2''$, as required.
  
  \item[(T-TensorE)] Suppose $\Psi ; \Theta ; \Delta ; \Omega ; \Gamma_1,\Gamma_2\vdash \texttt{let } \angles{x,y} = e_1 \texttt{ in } e_2 : \tau'$ by way of
  $\Psi ; \Theta ; \Delta ; \Omega ; \Gamma_1\vdash e_1 : \tau_1 \otimes \tau_2$ and
  $\Psi ; \Theta ; \Delta ; \Omega ; \Gamma_2,x : \tau_1, y : \tau_2\vdash e_2 : \tau'$.
  By IH, there are $e_1'$, $\Phi_1$, $\Gamma_1'$ such that
  $|e_1'| = e_1$,
  $\Theta ; \Delta \vDash \Phi_1$, and
  $\Psi ; \Theta ; \Delta ; \Omega ; \Gamma_1 \vdash e_1' \infers \tau_1 \otimes \tau_2 \gens \Phi_1,\Gamma_1'$.
  Since $\Psi ; \Theta ; \Delta \vdash \Gamma_1,\Gamma_2 \wknto \Gamma_1$,
  we have by Theorem~\ref{thm:admits-weaken} that there are $e_1''$, $\Phi_1'$, $\Gamma_1''$ such that
  $|e_1''| = |e_1'|$,
  $\Theta ; \Delta \vDash \Phi_1'$,
  $\Psi ; \Theta ; \Delta \vdash \Gamma_1'' \wknto (\Gamma_1,\Gamma_2 \setminus \Gamma_1)$, and
  $\Psi ; \Theta ; \Delta ; \Omega ; \Gamma_1,\Gamma_2 \vdash e_1'' \infers \tau_1 \otimes \tau_2 \gens \Phi_1',\Gamma_1''$.
  Again by IH, there are $e_2'$, $\Phi_2$, $\Gamma_2'$ such that
  $|e_2'| = e_2$,
  $\Theta ; \Delta \vDash \Phi_2$, and
  $\Psi ; \Theta ; \Delta ; \Omega ; \Gamma_2,x : \tau_1, y : \tau_2 \vdash e_2' \checks \tau' \gens \Phi_2,\Gamma_2'$.
  Again by Theorem~\ref{admits-weaken}, since
  $\Psi ; \Theta ; \Delta \vdash \Gamma_1'',x : \tau_1, y : \tau_2 \wknto \Gamma_2,x : \tau_1, y : \tau_2$,
  there are $e_2''$, $\Phi_2'$, $\Gamma_2''$ such that
  $|e_2''| = |e_2'|$,
  $\Theta ; \Delta \vDash \Phi_2'$, and
  $\Psi ; \Theta ; \Delta, \Gamma_1'', x : \tau_1, y : \tau_2 \vdash e_2'' \checks \tau' \gens \Phi_2',\Gamma_2''$.
  By AT-TensorE,
  $\Psi ; \Theta ; \Delta ; \Omega;  \Gamma_1,\Gamma_2 \vdash \texttt{let } \angles{x,y} = e_1'' \texttt{ in } e_2'' \checks \tau' \gens \Phi_1' \wedge \Phi_2',\Gamma_2''$.
  This completes (1), and (2) follows by AT-Anno.
  
  \item[(T-WithI)] Suppose
  $\Psi ; \Theta ; \Delta ; \Omega ; \Gamma \vdash (e_1,e_2) : \tau_1 \amp \tau_2$ by way of
  $\Psi ; \Theta ; \Delta ; \Omega ; \Gamma \vdash e_1 : \tau_1$ and
  $\Psi ; \Theta ; \Delta ; \Omega ; \Gamma \vdash e_2 : \tau_2$.
  By IH, we have for $i = 1,2$ that $e_i'$, $\Phi_i$, $\Gamma_i$ such that
  $|e_i'| = e_i$,
  $\Theta ; \Delta \vDash \Phi_i$, and
  $\Psi ; \Theta ; \Delta ; \Omega ; \Gamma \vdash e_i' : \tau_i \gens \Phi_i,\Gamma_i$.
  By AT-WithI,
  $\Psi ; \Theta ; \Delta ; \Omega ; \Gamma \vdash (e_1',e_2') \checks \tau_1 \amp \tau_2 \gens \Phi_1 \wedge \Phi_2,\Gamma_1 \cap \Gamma_2$.
  This completes (1), and (2) follows by AT-Anno.
  
  \item[(T-Fst)] Suppose
  $\Psi ; \Theta ; \Delta ; \Omega ; \Gamma \vdash \texttt{fst}(e) : \tau_1$ from
  $\Psi ; \Theta ; \Delta ; \Omega ; \Gamma \vdash e : \tau_1 \amp \tau_2$.
  By IH, there are $e'$, $\Phi$, $\Gamma'$ such that
  $|e'| = e$,
  $\Theta ; \Delta \vDash \Phi$, and
  $\Psi ; \Theta ; \Delta ; \Omega ; \Gamma \vdash e' \infers \tau_1 \amp \tau_2 \gens \Phi,\Gamma'$.
  By AT-Fst,
  $\Psi ; \Theta ; \Delta ; \Omega ; \Gamma \vdash \texttt{fst}(e') \infers \tau_1 \gens \Phi,\Gamma'$.
  This completes (2). For (1), we invoke Theorem~\ref{thm:subty-refl} to get some $\Phi'$ such that
  $\Theta ; \Delta \vDash \Phi'$ and $\Psi ; \Theta ; \Delta \vdash \tau_1 \subty \tau_1 : \star \gens \Phi'$.
  Then, by AT-Sub,
  $\Psi ; \Theta ; \Delta ; \Omega ; \Gamma \vdash \texttt{fst}(e') \checks \tau_1 \gens \Phi \wedge \Phi',\Gamma'$,
  which completes (1).
  
  
  \item[(T-Snd)] Identical to T-Fst.
  \item[(T-Inl)] Suppose $\Psi ; \Theta ; \Delta ; \Omega ; \Gamma \vdash \texttt{inl}(e) : \tau_1 \oplus \tau_2$ from
  $\Psi ; \Theta ; \Delta ; \Omega ; \Gamma \vdash e : \tau_1$.
  By IH, there are $e'$, $\Phi$, $\Gamma'$ such that
  $|e'| = e$,
  $\Theta ; \Delta \vDash \Phi$, and
  $\Psi ; \Theta ; \Delta ; \Omega ; \Gamma \vdash e' \checks \tau_1 \gens \Phi,\Gamma'$.
  By AT-Inl,
  $\Psi ; \Theta ; \Delta ; \Omega ; \Gamma \vdash \texttt{inl}(e') \checks \tau_1 \oplus \tau_2 \gens \Phi,\Gamma'$.
  This completes (1), and (2) follows by AT-Anno.
  
  
  \item[(T-Inr)] Identical to T-Inl.
  \item[(T-Case)] Suppose
  $\Psi ; \Theta ; \Delta ; \Omega ; \Gamma_1,\Gamma_2 \vdash \texttt{case}(e,x.e_1,y.e_2) : \tau$ by way of
  $\Psi ; \Theta ; \Delta ; \Omega ; \Gamma_1 \vdash e : \tau_1 \oplus \tau_2$,
  $\Psi ; \Theta ; \Delta ; \Omega ; \Gamma_2, x: \tau_1 \vdash e_1 : \tau$, and
  $\Psi ; \Theta ; \Delta ; \Omega ; \Gamma_2, y: \tau_2 \vdash e_2 : \tau$.
  By IH, there are $e'$, $\Phi$, $\Gamma_1'$ such that
  $|e'| = e$,
  $\Theta ; \Delta \vDash \Phi$, and
  $\Psi ; \Theta ; \Delta ; \Omega ; \Gamma_1 \vdash e' \infers \tau_1 \oplus \tau_2 \gens \Phi,\Gamma_1'$.
  But since $\Psi ; \Theta ; \Delta \vdash \Gamma_1,\Gamma_2 \wknto \Gamma_1$, we have by Theorem~\ref{thm:admits-weaken}
  that there are $e''$, $\Phi'$, $\Gamma_1''$ such that
  $|e''| = |e'|$,
  $\Theta ; \Delta \vDash \Phi'$,
  $\Psi ; \Theta ; \Delta \vdash \Gamma_1'' \wknto (\Gamma_1,\Gamma_2) \setminus \Gamma_1$, and
  $\Psi ; \Theta ; \Delta ; \Omega ; \Gamma_1,\Gamma_2 \vdash e'' \infers \tau_1 \oplus \tau_2 \gens \Phi',\Gamma_1''$.
  Then, again by IH, there are $e_1'$, $\Phi_1$, $\Gamma_2'$ such that
  $|e_1'| = e_1$,
  $\Theta ; \Delta \vDash \Phi_1$, and
  $\Psi ; \Theta ; \Delta ; \Omega ; \Gamma_2, x : \tau_1 \vdash e_1' \checks \tau \gens \Phi_1,\Gamma_2'$.
  But since
  $\Psi ; \Theta ; \Delta \vdash \Gamma_1'', x : \tau_1 \wknto \Gamma_2, x : \tau_1$,
  we have by Theorem~\ref{thm:admits-weaken} that there are $e_1''$, $\Phi_1'$, $\Gamma_2''$ such that
  $|e_1''| = |e_1'|$,
  $\Theta ; \Delta \vDash \Phi_1'$, and
  $\Psi ; \Theta ; \Delta ; \Omega ; \Gamma_1'', x : \tau_1 \vdash e_1'' \checks \tau \gens \Phi_1',\Gamma_2''$.
  Applying IH one last time, we have $e_2'$, $\Phi_2$, $\Gamma_3'$ such that
  $|e_2'| = e_2$,
  $\Theta ; \Delta \vDash \Phi_2$, and
  $\Psi ; \Theta ; \Delta ; \Omega ; \Gamma_2, y : \tau_2 \vdash e_2' \checks \tau \gens \Phi_2,\Gamma_3'$.
  But, once more by Theorem~\ref{thm:admits-weaken}, there are $e_2''$, $\Phi_2'$, $\Gamma_3''$ such that
  $|e_2''| = |e_2'|$,
  $\Theta ; \Delta \vDash \Phi_2'$, and
  $\Psi ; \Theta ; \Delta ; \Omega ; \Gamma_1'', y : \tau_2 \vdash e_2'' \checks \tau \gens \Phi_2',\Gamma_3''$.
  Finally, by AT-Case, we have that
  $\Psi ; \Theta ; \Delta ; \Omega ; \Gamma_1,\Gamma_2 \vdash \texttt{case}(e'',x.e_1'',y.e_2'') \checks \tau \gens \Phi' \wedge \Phi_1' \wedge \Phi_2',(\Gamma_2'' \setminus \{x\}) \cap (\Gamma_3'' \setminus \{y\})$.
  This completes (1), and (2) follows immediately by AT-Anno.
  
  \item[(T-ExpI)] Suppose
  $\Psi ; \Theta ; \Delta ; \Omega ; \Gamma \vdash !e : !\tau$ by way of
  $\Psi ; \Theta ; \Delta ; \Omega ; \cdot \vdash e : \tau$.
  By IH, there are $e'$, $\Phi$, $\Gamma'$ such that
  $|e'| = e$,
  $\Theta ; \Delta \vDash \Phi$,and
  $\Psi ; \Theta ; \Delta ; \Omega; \cdot \vdash e' \checks \tau \gens \Phi,\Gamma'$.
  By AT-ExpI,
  $\Psi ; \Theta ; \Delta ; \Omega; \Gamma \vdash !e' \checks !\tau \gens \Phi,\Gamma$,
  which completes (1), and (2) follows by AT-Anno.        
  
  \item[(T-ExpE)] Suppose
  $\Psi ; \Theta ; \Delta ; \Omega ; \Gamma_1,\Gamma_2 \vdash \texttt{let } !x = e_1 \texttt{ in } e_2 : \tau'$ from
  $\Psi ; \Theta ; \Delta ; \Omega ; \Gamma_1 \vdash e_1  : !\tau$ and
  $\Psi ; \Theta ; \Delta ; \Omega, x : \tau ; \Gamma_2 \vdash e_2 : \tau'$.
  By IH, there are $e_1'$, $\Phi_1$, $\Gamma_1'$ such that
  $|e_1'| = e_1$,
  $\Theta ; \Delta \vDash \Phi_1$, and
  $\Psi ; \Theta ; \Delta ; \Omega ; \Gamma_1 \vdash e_1' \infers !\tau \gens \Phi_1,\Gamma_1'$.
  By Theorem~\ref{thm:admits-weaken}, there are $e_1''$, $\Phi_1'$, $\Gamma_1''$ such that
  $|e_1''| = |e_1'|$,
  $\Theta ; \Delta \vDash \Phi_1'$,
  $\Psi ; \Theta ; \Delta \vdash \Gamma_1'' \wknto (\Gamma_1,\Gamma_2) \setminus \Gamma_1$, and
  $\Psi ; \Theta ; \Delta ; \Omega; \Gamma_1,\Gamma_2 \vdash e_1'' \infers !\tau \gens \Phi_1',\Gamma_1''$.
  Again by IH, there are $e_2'$, $\Phi_2$, $\Gamma_2'$ such that
  $|e_2'| = e_2$,
  $\Theta ; \Delta \vDash \Phi_2$, and
  $\Psi ; \Theta ; \Delta ; \Omega, x: \tau ; \Gamma_2 \vdash e_2' \checks \tau' \gens \Phi_2,\Gamma_2'$.
  Then, since
  $\Psi ; \Theta ; \Delta \vdash \Gamma_1'' \wknto \Gamma_2$,
  we have by Theorem~\ref{admits-weaken} that there are $e_2''$, $\Phi_2'$, $\Gamma_2''$ such that
  $|e_2''| = |e_2|$,
  $\Theta ; \Delta \vDash \Phi_2'$, and
  $\Psi ; \Theta ; \Delta ; \Omega, x : \tau ; \Gamma_1'' \vdash e_2'' \checks \tau' \gens \Phi_2', \Gamma_2''$.
  Finally, by AT-ExpE, we have
  $\Psi ; \Theta ; \Delta ; \Omega ; \Gamma_1,\Gamma_2 \vdash \texttt{let } !x = e_1 \texttt{ in } e_2 \checks \tau' \gens \Phi_1' \wedge \Phi_2', \Gamma_2''$,
  as required for (1). (2) follows by AT-Anno.
  
  \item[(T-TAbs)] Suppose
  $\Psi ; \Theta ; \Delta ; \Omega ; \Gamma \vdash \Lambda \alpha. e : \forall \alpha : K.\tau$ by way of
  $\Psi, \alpha : K ; \Theta ; \Delta ; \Omega ; \Gamma \vdash e : \tau$.
  By IH, there are $e'$,$\Phi$, $\Gamma'$ such that
  $|e'| = e$,
  $\Theta ; \Delta \vDash \Phi$, and
  $\Psi, \alpha : K ; \Theta ; \Delta ; \Omega ; \Gamma \vdash e' \checks \tau \gens \Phi,\Gamma'$.
  By AT-TAbs
  $\Psi ; \Theta ; \Delta ; \Omega ; \Gamma \vdash e' \checks \forall \alpha : K. \tau \gens \Phi,\Gamma'$.
  This completes (1), and for (2), one use of AT-Anno suffices.
    
  \item[(T-TApp)] Suppose
  $\Psi ; \Theta ; \Delta ; \Omega ; \Gamma \vdash e [\tau'] : \tau[\tau'/\alpha]$ by way of
  $\Psi ; \Theta ; \Delta ; \Omega ; \Gamma \vdash e : \forall \alpha : K.\tau$ and
  $\Psi ; \Theta ; \Delta \vdash \tau' : K$.
  By IH, there are $e'$, $\Phi$, $\Gamma'$ such that
  $|e'| = e$,
  $\Theta ; \Delta \vDash \Phi$, and
  $\Psi ; \Theta ; \Delta ; \Omega ; \Gamma \vdash e' \infers \forall \alpha : K.\tau \gens \Phi,\Gamma'$.
  By Theorem~\ref{thm:kind-compl}, there is some $\Phi'$ such that
  $\Theta ; \Delta \vDash \Phi'$, and
  $\Psi ; \Theta ; \Delta \vdash \tau' : K \gens \Phi'$.
  Then, by AT-TApp,
  $\Psi ; \Theta ; \Delta ; \Omega ; \Gamma \vdash e' [\tau'] \infers \tau[\tau'/\alpha] \gens \Phi \wedge \Phi',\Gamma'$.
  This completes (2). For (1), we have
  by Theorem~\ref{thm:subty-refl}, there is some $\Phi''$ with
  $\Theta ; \Delta \vDash \Phi''$ and
  $\Psi ; \Theta ; \Delta \vdash \tau[\tau'/\alpha] \subty \tau[\tau'/\alpha] : \star \gens \Phi''$.
  Then, by AT-Sub,
  $\Psi ; \Theta ; \Delta ; \Omega ; \Gamma \vdash e' [\tau'] \infers \tau[\tau'/\alpha] \gens \Phi \wedge \Phi' \wedge \Phi'',\Gamma'$
  which completes (2).
  \item[(T-IAbs)] Suppose
  $\Psi ; \Theta ; \Delta ; \Omega ; \Gamma \vdash \Lambda i. e : \forall i : S. \tau$ by way of
  $\Psi ; \Theta, i : S ; \Delta ; \Omega ; \Gamma \vdash e : \tau$.
  By IH, there are $e'$, $\Phi$, $\Gamma'$ such that
  $|e'| = e$,
  $\Theta, i : S ; \Delta \vDash \Phi$, and
  $\Psi ; \Theta, i : S ; \Delta ; \Omega; \Gamma \vdash e' \checks \tau \gens \Phi,\Gamma'$.
  By AT-IAbs,
  $\Psi ; \Theta ; \Delta ; \Omega; \Gamma \vdash \Lambda i. e' \checks \forall i : S.\tau \gens \forall i : S.\Phi,\Gamma'$.
  Since $\Theta, i : S ; \Delta \vDash \Phi$, we also have $\Theta ; \Delta \vDash \forall i : S. \Phi$, which proves (1).
  (2) follows immediately by AT-Anno, as always.
  
  
  \item[(T-IApp)] Suppose
  $\Psi ; \Theta ; \Delta ; \Omega ; \Gamma \vdash e [I] : \tau[I/i]$ by way of
  $\Psi ; \Theta ; \Delta ; \Omega ; \Gamma \vdash e : \forall i : S.\tau$ and
  $\Theta ; \Delta \vdash I : S$.
  By IH, there are $e'$, $\Phi$, $\Gamma'$ such that
  $|e'| = e$,
  $\Theta ; \Delta \vDash \Phi$, and
  $\Psi ; \Theta ; \Delta ; \Omega; \Gamma \vdash e \infers \forall i : S. \tau \gens \Phi,\Gamma'$.
  By Theorem~\ref{thm:sort-compl}, there is some $\Phi'$ with
  $\Theta ; \Delta \vDash \Phi'$ and
  $\Theta ; \Delta \vdash I : S \gens \Phi'$.
  By AT-IApp,
  $\Psi ; \Theta ; \Delta ; \Omega; \Gamma \vdash e [I] \infers \tau[I/i] \gens \Phi \wedge \Phi',\Gamma'$.
  This completes (1), and (2) follows by AT-Anno.
  
  
  \item[(T-Fix)] Suppose $\Psi ; \Theta ; \Delta ; \Omega ; \Gamma \vdash \texttt{fix } x.e : \tau$ from
  $\Psi ; \Theta ; \Delta ; \Omega, x : \tau ; \cdot \vdash e : \tau$.
  By IH, there are $e'$, $\Phi$, $\Gamma'$ such that
  $|e'| = e$,
  $\Theta ; \Delta \vDash \Phi$, and
  $\Psi ; \Theta ; \Delta ; \Omega, x : \tau ; \cdot \vdash e' \checks \tau \gens \Phi,\Gamma'$.
  By AT-Fix,
  $\Psi ; \Theta ; \Delta ; \Omega ; \Gamma \vdash \texttt{fix } x.e' : \tau \gens \Phi,\Gamma$,
  as required. Since $|\texttt{fix }x.e'| = \texttt{fix }x.|e'| = \texttt{fix }x.e$, this completes (1). For (2),
  one use of AT-Anno suffices.
  
  \item[(T-Ret)] Suppose
  $\Psi ; \Theta ; \Delta ; \Omega ; \Gamma \vdash \texttt{ret}\; e : \M \, \phi(I,\vec{p}) \, \tau$ by way of
  $\Psi ; \Theta ; \Delta ; \Omega ; \Gamma \vdash e : \tau$.
  By IH, there are $e'$, $\Phi$, $\Gamma'$ such that
  $|e'| = e$,
  $\Theta ; \Delta \vDash \Phi$, and
  $\Psi ; \Theta ; \Delta ; \Omega ; \Gamma \vdash e \checks \tau \gens \Phi,\Gamma'$.
  By AT-Ret,
  $\Psi ; \Theta ; \Delta ; \Omega ; \Gamma \vdash \texttt{ret}\; e \checks \M \, \phi(I,\vec{p}) \, tau \gens \Phi,\Gamma'$.
  This completes (1), and (2) follows by a single use of AT-Anno.
  
  \item[(T-Bind)] 
  Suppose $\Psi ; \Theta ; \Delta ; \Omega ; \Gamma_1,\Gamma_2 \vdash \texttt{bind } x = e_1 \texttt{ in } e_2 : \M \, \phi(I,\vec{p} + \vec{q})\, \tau_2$ by way of
  $\Psi ; \Theta ; \Delta ; \Omega ; \Gamma_1 \vdash e_1 : \M \, \phi(I,\vec{p})\, \tau_1$, and
  $\Psi ; \Theta; \Delta ; \Omega ; \Gamma_2, x:\tau_1 \vdash e_2 : \M \, \phi(I,\vec{q})\, \tau_2$.
  By IH, there are $e_1'$, $\Phi_1$, $\Gamma_1'$ such that
  $|e_1'| = e_1$,
  $\Theta ; \Delta \vDash \Phi_1$, and
  $\Psi  ; \Theta ; \Delta ; \Omega ; \Gamma_1 \vdash e_1' \checks \M \, \phi(I,\vec{p}) \, \tau_1 \gens \Phi_1,\Gamma_1'$.
  Since $\Psi ; \Theta ; \Delta \vdash \Gamma_1,\Gamma_2 \wknto \Gamma_1$,
  we have by Theorem~\ref{thm:admits-weaken} $e_1''$, $\Gamma_1''$, $\Phi_1'$ such that
  $|e_1''| = |e_1'|$,
  $\Theta ; \Delta \vDash \Phi_1'$,
  $\Psi ; \Theta ; \Delta \vdash \Gamma_1'' \wknto (\Gamma_1,\Gamma_2) \setminus \Gamma_1$, and
  $\Psi ; \Theta ; \Delta ; \Omega ; \Gamma_1,\Gamma_2 \vdash e_1'' \checks \M \, \phi(I,\vec{p}) \, \tau_1 \gens \Phi_1',\Gamma_1''$.
  Noting that $(\Gamma_1,\Gamma_2) \setminus \Gamma_1 = \Gamma_2$, we have that
  $\Psi ; \Theta ; \Delta \vdash \Gamma_1'' \wknto \Gamma_2$.
  By IH, there are $e_2'$, $\Phi_2$, $\Gamma_2'$ such that
  $|e_2'| = e_2$,
  $\Theta ; \Delta \vDash \Phi_2$, and
  $\Psi ; \Theta; \Delta ; \Omega ; \Gamma_2, x:\tau_1 \vdash e_2' \checks \M \, \phi(I,\vec{q})\, \tau_2 \gens \Phi_2,\Gamma_2'$.
  But then by Theorem~\ref{thm:admits-weaken}, there are $e_2''$, $\Gamma_2''$, $\Phi_2'$ such that
  $|e_2''| = |e_2'|$,
  $\Theta ; \Delta \vDash \Phi_2'$, and
  $\Psi ; \Theta ; \Delta ; \Omega ; \Gamma_1'', x : \tau_1 \vdash e_2'' \infers \M \, \phi(I,\vec{q}) \, \tau_2 \gens \Phi_2',\Gamma_2''$.
  By Theorem~\ref{thm:subty-refl}, there is some $\Phi_3$ such that $\Theta ; \Delta \vDash \Phi_3$, and
  $\Psi ; \Theta ; \Delta \vdash \tau_2 \subty \tau_2 : \star \gens \Phi_3$.
  Then, by AS-Monad, $\Psi ; \Theta ; \Delta \vdash \M \, \phi(I,\vec{q}) \, \tau_2 \subty \M \, \phi(I,(\vec{p} + \vec{q}) - \vec{p}) \, \tau_2 : \star \gens \Phi_3 \wedge (I = I) \wedge (\vec{q} \leq (\vec{p} + \vec{q}) - \vec{p})$.
  Then, by AT-Sub,
  $\Psi ; \Theta ; \Delta ; \Omega ; \Gamma_1'', x : \tau_1 \vdash e_2'' \checks \M \, \phi(I,(\vec{p} + \vec{q}) - \vec{p}) \, \tau_2 \gens \Phi_2' \wedge  \Phi_3 \wedge (I = I) \wedge (\vec{q} \leq (\vec{p} + \vec{q}) - \vec{p}),\Gamma_2''$.
  Finally, by AT-Bind,
  $\Psi ; \Theta ; \Delta ; \Omega ; \Gamma_1,\Gamma_2 \vdash \texttt{bind } x = e_1'' \texttt{ in } e_2'' \checks \M \, \phi(I,\vec{p} + \vec{q})\, \tau_2 \gens \Phi_1' \wedge \Phi_2' \wedge \Phi_3 \wedge (I = I) \wedge (\vec{q} \leq (\vec{p} + \vec{q}) - \vec{p}), \Gamma_2''$, which completes (1). (2) follows immediately by AT-Anno.
  
  
  \item[(T-Tick)] Suppose $\Psi ; \Theta ; \Delta ; \Omega ; \Gamma \vdash \texttt{tick}[I|\vec{p}] : \M \, \phi(I,\vec{p})\, 1$ by way of
  $\Theta ; \Delta \vdash I : \mathbb{N}$ and
  $\Theta ; \Delta \vdash \vec{p} : \vec{\mathbb{R}^+}$.
  By Theorem~\ref{thm:sort-compl}, there is some $\Phi_1$ such that
  $\Theta ; \Delta \vDash \Phi_1$, and
  $\Theta ; \Delta \vdash I : \mathbb{N} \gens \Phi_1$.
  Again by Theorem~\ref{thm:sort-compl}, there is a $\Phi_2$ such that
  $\Theta ; \Delta \vDash \Phi_2$, and
  $\Theta ; \Delta \vdash \vec{p} : \vec{\mathbb{R}^+}$.
  By AT-Tick,
  $\Psi ; \Theta ; \Delta ; \Omega ; \Gamma \vdash \texttt{tick}[I|\vec{p}] \infers \M \, \phi(I,\vec{p})\, 1 \gens \Phi_1 \wedge \Phi_2$,
  which completes (2).
  For (1), it is easy to see that $\Psi ; \Theta ; \Delta \vdash \M \, \phi(I,\vec{p})\, 1 \subty \M \, \phi(I,\vec{p})\, 1 : \star \gens I = I \wedge \vec{p} \leq \vec{p}$, and of course $\Theta ; \Delta \vDash I = I \wedge \vec{p} \leq \vec{p}$.
  By AT-Sub,
  $\Psi ; \Theta ; \Delta ; \Omega ; \Gamma \vdash \texttt{tick}[I|\vec{p}] \checks \M \, \phi(I,\vec{p})\, 1 \gens \Phi_1 \wedge \Phi_2 \wedge  I = I \wedge \vec{p} \leq \vec{p}$, completing (1).
  
  \item[(T-Release)] Suppose
  $\Psi ; \Theta ; \Delta ; \Omega ; \Gamma_1,\Gamma_2 \vdash \texttt{release } x = e_1 \texttt{ in }e_2 : \M \, \phi(I,\vec{p}) \, \tau_2$ by way of
  $\Psi ; \Theta ; \Delta ; \Omega ; \Gamma_1 \vdash e_1 : [I | \vec{q}] \tau_1$ and
  $\Psi ; \Theta ; \Delta ; \Omega ; \Gamma_2, x : \tau \vdash e_2 : \M \, \phi(I,\vec{p} + \vec{q}) \, \tau_2$.
  By IH, there are $e_1'$, $\Phi_1$, $\Gamma_1'$ such that
  $|e_1'| = e_1$,
  $\Theta ; \Delta \vDash \Phi_1$, and
  $\Psi ; \Theta ; \Delta ; \Omega ; \Gamma_1 \vdash e_1' \infers [I|\vec{q}] \tau_1 \gens \Phi_1,\Gamma_1'$.
  Since $\Psi ; \Theta ;\Delta \vdash \Gamma_1,\Gamma_2 \wknto \Gamma_1$, we have by Theorem~\ref{thm:admits-weaken} that there are $e_1''$, $\Phi_1'$, $\Gamma_1''$ such that
  $|e_1''| = |e_1'|$,
  $\Theta ; \Delta \vDash \Phi_1'$,
  $\Psi ; \Theta ; \Delta \vdash \Gamma_1'' \wknto (\Gamma_1,\Gamma_2) \setminus \Gamma_1$, and
  $\Psi ; \Theta ; \Delta ; \Omega ; \Gamma_1,\Gamma_2 \vdash e_1'' \infers [I|\vec{q}] \tau_1 \gens \Phi_1',\Gamma_1''$.
  Then, by IH, there are $e_2'$, $\Phi_2$, $\Gamma_2'$ such that
  $|e_2'| = e_2$,
  $\Theta ; \Delta \vDash \Phi_2$, and
  $\Psi ; \Theta ; \Delta ; \Omega ; \Gamma_2, x : \tau \vdash e_2' \checks \M \, \phi(I,\vec{p} + \vec{q}) \, \tau_2 \gens \Phi_2,\Gamma_2'$.
  Since $\Psi ; \Theta ; \Delta \vdash \Gamma_1'' \wknto \Gamma_2$,
  we have by Theorem~\ref{thm:admits-weaken} that there are $e_2''$, $\Phi_2'$, $\Gamma_2''$ such that
  $|e_2''| = |e_2'|$,
  $\Theta ; \Delta \vDash \Phi_2'$, and
  $\Psi ; \Theta ; \Delta ; \Omega ; \Gamma_1'', x : \tau \vdash e_2'' \checks \M \, \phi(I,\vec{p} + \vec{q}) \, \tau \gens \Phi_2',\Gamma_2''$.
  Then, by AT-Relase,
  $\Psi ; \Theta ; \Delta ; \Omega ; \Gamma_1,\Gamma_2 \vdash \texttt{release } x = e_1'' \texttt{ in }e_2'' \checks \M \, \phi(I,\vec{p}) \, \tau_2 \gens (I = I) \wedge \Phi_1' \wedge \Phi_2',\Gamma_2'' \setminus \{x\}$.
  This completes (1), and (2) follows by AT-Anno.
  
  
  
  \item[(T-Store)] Suppose
  $\Psi ; \Theta ; \Delta ; \Omega ; \Gamma \vdash \texttt{store}[I|\vec{p}](e) : \M \, \phi(I,\vec{p}) \, ([I| \vec{p}] \, \tau)$ by way of
  $\Theta ; \Delta \vdash I : \mathbb{N}$,
  $\Theta ; \Delta \vdash \vec{p} : \vec{\mathbb{R}^+}$, and
  $\Psi ; \Theta ; \Delta ; \Omega ; \Gamma \vdash e : \tau$.
  By Theorem~\ref{thm:sort-compl}, there are $\Phi_1$, $\Phi_2$ such that
  $\Theta ; \Delta \vdash I : \mathbb{N} \gens \Phi_1$ and
  $\Theta ; \Delta \vdash \vec{p} : \vec{\mathbb{R}^+} \gens \Phi_2$
  with $\Theta ; \Delta \vDash \Phi_1 \wedge \Phi_2$.
  By IH, there are $e'$, $\Phi_3$, $\Gamma'$ such that
  $|e'| = e$,
  $\Theta ; \Delta \vDash \Phi_3$, and
  $\Psi ; \Theta ; \Delta ; \Omega; \Gamma \vdash e' \checks \tau \gens \Phi_3,\Gamma'$.
  By AT-Store,
  $\Psi ; \Theta ; \Delta ; \Omega ; \gamma \vdash \texttt{store}[I|\vec{p}](e') \checks \M \, \phi(I,\vec{p}) \, ([I| \vec{p}] \, \tau) \gens \Phi_1 \wedge \Phi_2 \wedge \Phi_3 \wedge (I \leq I \leq I) \wedge (\vec{p} \leq \vec{p} \leq \vec{p}),\Gamma'$.
  Of course, $\Theta ; \Delta \vDash (I \leq I \leq I) \wedge (\vec{p} \leq \vec{p} \leq \vec{p})$, and so
  we are done with (1). For (2), a single use of AT-Anno suffices.
  
    
  \item[(T-StoreConst)] Suppose
  $\Psi ; \Theta ; \Delta ; \Omega ; \Gamma \vdash \texttt{store}[I](e) : \M \, \phi(K,\texttt{const}(I)) \, ([I] \, \tau)$ by way of
  $\Theta ; \Delta \vdash I : \mathbb{N}$ and
  $\Psi ; \Theta ; \Delta ; \Omega ; \Gamma \vdash e : \tau$.
  By Theorem~\ref{thm:sort-compl}, there is some $\Phi_1$ such that
  $\Theta ; \Delta \vDash \Phi_1$, and
  $\Theta ; \Delta \vdash I : \mathbb{N} \gens \Phi_1$.
  By IH, there are $e'$, $\Phi_2$, and $\Gamma'$ such that
  $|e'| = e$,
  $\Theta ; \Delta \vDash \Phi_2$, and
  $\Psi ; \Theta ; \Delta \vdash e' \checks \tau \gens \Phi_2,\Gamma'$.
  Then, by AT-StoreConst,
  $\Psi ; \Theta ; \Delta ; \Omega ; \Gamma \vdash \texttt{store}[I](e') \checks \M \, \phi(K,\texttt{const}(I)) \, ([I] \, \tau) \gens \Phi_1 \wedge \Phi_2 \wedge (\texttt{const}(I) \leq \texttt{const}(I) \leq \texttt{const}(I)),\Gamma'$.
  Of course, $\Theta ; \Delta \vDash (\texttt{const}(I) \leq \texttt{const}(I) \leq \texttt{const}(I))$,
  which completes (1). For (2), one use of AT-Anno suffices.

  \item[(T-ReleaseConst)] Suppose
  $\Psi ; \Theta ; \Delta ; \Omega ; \Gamma_1,\Gamma_2 \vdash \texttt{release } x = e_1 \texttt{ in }e_2 : \M \, \phi(I,\vec{p}) \, \tau_2$ from
  $\Psi ; \Theta ; \Delta ; \Omega ; \Gamma_1 \vdash e_1 : [J] \tau_1$ and
  $\Psi ; \Theta ; \Delta ; \Omega ; \Gamma_2, x : \tau \vdash e_2 : \M \, \phi(I,\vec{p} + \texttt{const}(J)) \, \tau_2$.
  By IH, there are $e_1'$, $\Phi_1$, $\Gamma_1'$ such that
  $|e_1'| = e_1$,
  $\Theta ; \Delta \vDash \Phi_1$, and
  $\Psi ; \Theta ; \Delta ; \Omega ; \Gamma_1 \vdash e_1' \infers [J] \tau_1 \gens \Phi_1,\Gamma_1'$.
  Since $\Psi ; \Theta ; \Delta \vdash \Gamma_1,\Gamma_2 \wknto \Gamma_1$, we have by Theorem~\ref{thm:admits-weaken}
  that there are $e_1''$, $\Phi_1'$, $\Gamma_1''$ such that
  $|e_1''| = |e_1'|$,
  $\Theta ; \Delta \vDash \Phi_1'$,
  $\Psi ; \Theta ; \Delta \vdash \Gamma_1'' \wknto \Gamma_2$, and
  $\Psi ; \Theta ; \Delta ; \Omega; \Gamma_1,\Gamma_2 \vdash e_1'' \infers [J] \tau_1 \gens \Phi_1',\Gamma_1''$.
  Again by IH, there are $e_2'$, $\Phi_2$, $\Gamma_2'$ such that
  $|e_2'| = e_2$,
  $\Theta ; \Delta \vDash \Phi_2$, and
  $\Psi ; \Theta ; \Delta ; \Omega ; \Gamma_2, x : \tau \vdash e_2' \checks \M \, \phi(I,\vec{p} + \texttt{const}(J)) \, \tau_2 \gens \Phi_2,\Gamma_2'$.
  But, by Theorem~\ref{thm:admits-weaken}, there are $e_2''$, $\Phi_2'$, $\Gamma_2''$ such that
  $|e_2''| = |e_2'|$,
  $\Theta ; \Delta \vDash \Phi_2'$, and
  $\Psi ; \Theta ; \Delta ; \Omega ; \Gamma_1'', x: \tau \vdash e_2'' \checks  \M \, \phi(I,\vec{p} + \texttt{const}(J)) \, \tau_2 \gens \Phi_2',\Gamma_2''$.
  Then, by AT-ReleaseConst, 
  $\Psi ; \Theta ; \Delta ; \Omega ; \Gamma_1,\Gamma_2 \vdash \texttt{release } x = e_1'' \texttt{ in }e_2'' : \M \, \phi(I,\vec{p}) \, \tau_2 \gens \Phi_1' \wedge \Phi_2',\Gamma_2''$.
  This completes (1), and (2) follows by AT-Anno.
  
  \item[(T-Shift)] Suppose $\Psi ; \Theta ; \Delta ; \Omega ; \Gamma \vdash \texttt{shift}(e) : \M \, \phi(I,\vec{p}) \, \tau$ by way of
  $\Psi ; \Theta ; \Delta ; \Omega ; \Gamma \vdash e : \M \, \phi(I - 1,\lhd \vec{p}) \, \tau$ and
  $\Theta ; \Delta \vDash I \geq 1$.
  By IH, there are $e'$, $\Phi$, $\Gamma'$ such that
  $|e'| = e$,
  $\Theta ; \Delta \vDash \Phi$, and
  $\Psi  ; \Theta ; \Delta ; \Omega; \Gamma \vdash e' \checks \M \, \Phi(I-1,\lhd \vec{p}) \, \tau \gens \Phi,\Gamma'$.
  By AT-Shift,
  $\Psi  ; \Theta ; \Delta ; \Omega; \Gamma \vdash \texttt{shift}(e') \checks \M \, \Phi(I,\vec{p}) \, \tau \gens (I \geq 1) \wedge \Phi,\Gamma'$.
  Since $\Theta ; \Delta \vDash (I \geq 1) \wedge \Phi$, this completes (1), and (2) follows by AT-Anno.

  \item[(T-CImpI)] Suppose
  $\Psi ; \Theta ; \Delta ; \Omega ; \Gamma \vdash \Lambda .e : (\Phi' \Rightarrow \tau)$ by way of
  $\Psi ; \Theta ; \Delta,\Phi' ; \Omega ; \Gamma \vdash e : \tau$.
  By IH, there are $e'$, $\Phi$, $\Gamma'$ such that
  $|e'| = e$,
  $\Theta ; \Delta,\Phi' \vDash \Phi$, and
  $\Psi ; \Theta ; \Delta,\Phi'; \Omega ; \Gamma \vdash e' \checks \tau \gens \Phi,\Gamma'$.
  By AT-CImpI,
  $\Psi ; \Theta ; \Delta ; \Omega ; \Gamma \vdash \Lambda. e' \checks \Phi' \implies \tau \gens (\Phi' \to \Phi),\Gamma'$.
  This completes (1), and (2) follows by AT-Anno.
  
  \item[(T-CImpE)] Suppose
  $\Psi ; \Theta ; \Delta ; \Omega ; \Gamma \vdash e \{\} : \tau$ by way of
  $\Psi ; \Theta ; \Delta ; \Omega ; \Gamma \vdash e : \Phi' \Rightarrow \tau$ and
  $\Theta ; \Delta \vDash \Phi'$.
  By IH, there are $e'$, $\Phi$, $\Gamma'$ such that
  $|e'| = e$,
  $\Theta ; \Delta \vDash \Phi$, and
  $\Psi ; \Theta ; \Delta ; \Omega ; \Gamma \vdash e' \infers \Phi' \Rightarrow \tau \gens \Phi,\Gamma'$.
  By AT-CImpE,
  $\Psi ; \Theta ; \Delta ; \Omega ; \Gamma \vdash e \{\} \infers \tau \gens \Phi \wedge \Phi',\Gamma'$.
  Since $\Theta ; \Delta \vDash \Phi'$ and $\Theta ; \Delta \vDash \Phi$, we have $\Theta ; \Delta \vDash \Phi \wedge \Phi'$.
  This completes (2). For (2),
  by Theorem~\ref{thm:subty-refl}, there is some $\Phi''$ with
  $\Theta ; \Delta \vDash \Phi''$, and
  $\Psi ; \Theta ; \Delta \vdash \tau \subty \tau : \star \gens \Phi''$.
  Then, by AT-Sub,
  $\Psi ; \Theta ; \Delta ; \Omega ; \Gamma \vdash e \{\} \checks \tau \gens \Phi \wedge \Phi' \wedge \Phi'',\Gamma'$,
  which completes (1).
  
  \item[(T-CAndI)] Suppose
  $\Psi ; \Theta ; \Delta ; \Omega ; \Gamma \vdash <e> : \Phi' \amp \tau$ by way of
  $\Psi ; \Theta ; \Delta ; \Omega ; \Gamma \vdash e : \tau$ and
  $\Theta ; \Delta \vDash \Phi'$.
  By IH, there are $e'$, $\Phi$, $\Gamma'$ such that
  $|e'| = e$,
  $\Theta ; \Delta \vDash \Phi$, and
  $\Psi  ; \Theta ; \Delta ; \Omega ; \Gamma \vdash e' \checks \tau \gens \Phi,\Gamma'$.
  By AT-CAndI,
  $\Psi; \Theta ; \Delta ; \Omega ; \Gamma \vdash <e'> \checks \Phi' \amp \tau \gens \Phi \wedge \Phi',\Gamma'$.
  Since $\Theta ; \Delta \vDash \Phi$ and $\Theta ; \Delta \vDash \Phi'$, we have $\Theta ; \Delta \vDash \Phi \wedge \Phi'$,
  completing (1). (2) follows by AT-Anno.
  
  \item[(T-CAndE)] Suppose
  $\Psi ; \Theta ; \Delta ; \Omega ; \Gamma_1,\Gamma_2 \vdash \texttt{clet } x = e_1 \texttt{ in } e_2 : \tau'$ by way of
  $\Psi ; \Theta ; \Delta ; \Omega ; \Gamma_1 \vdash e_1 : \Phi' \amp \tau$, and
  $\Psi ; \Theta ; \Delta, \Phi' ; \Omega ; \Gamma_2, x : \tau \vdash e_2 : \tau'$.
  By IH, there are $e_1'$, $\Phi_1$, $\Gamma_1'$ such that
  $|e_1'| = e_1$,
  $\Theta ; \Delta \vDash \Phi_1$, and
  $\Psi ; \Theta ; \Delta ; \Omega ; \Gamma_1 \vdash e_1' \infers \Phi' \amp \tau \gens \Phi_1,\Gamma_1'$.
  By Theorem~\ref{thm:admits-weaken}, there are $e_1''$, $\Phi_1'$, $\Gamma_1''$ such that
  $|e_1''| = |e_1'|$,
  $\Theta ; \Delta \vDash \Phi_1'$,
  $\Psi ; \Theta ; \Delta \vdash \Gamma_1'' \wknto (\Gamma_1,\Gamma_2) \setminus \Gamma_1$, and
  $\Psi ; \Theta ; \Delta ; \Omega ; \Gamma_1,\Gamma_2 \vdash e_1'' \infers \Phi' \amp \tau \gens \Phi_1',\Gamma_1''$.
  Again by IH, there are $e_2'$, $\Phi_2$, $\Gamma_2'$ such that
  $|e_2'| = e_2$,
  $\Theta ; \Delta,\Phi' \vDash \Phi_2$, and
  $\Psi ; \Theta ; \Delta,\Phi' ; \Omega ; \Gamma_2,x : \tau \vdash e_2' \checks \tau' \gens \Phi_2,\Gamma_2'$.
  Since
  $\Psi ; \Theta ; \Delta \vdash \Gamma_1'' \wknto \Gamma_2$, we have by Theorem~\ref{thm:ctx-sub-wkn} and Theorem~\ref{thm:ctx-sub-subset2}
  that
  $\Psi ; \Theta ; \Delta,\Phi' \vdash \Gamma_1'',x : \tau \wknto \Gamma_2,x : \tau$,
  and so by Theorem~\ref{thm:admits-weaken}, there are $e_2''$, $\Phi_2'$, $\Gamma_2''$ such that
  $|e_2''| = |e_2'|$,
  $\Theta ; \Delta, \Phi' \vDash \Phi_2'$,
  $\Psi ; \Theta ; \Delta, \Phi' ; \Omega ;\Gamma_1'',x:\tau \vdash e_2'' \checks \tau' \gens \Phi_2',\Gamma_2''$.
  Then, by AT-CAndE,
  $\Psi ; \Theta ; \Delta ; \Omega ; \Gamma_1,\Gamma_2 \vdash \texttt{clet } x = e_1'' \texttt{ in } e_2'' : \tau' \gens \Phi_1' \wedge (\Phi' \to \Phi_2'),\Gamma_2''$.
  Since $\Theta ; \Delta \vDash \Phi_1'$ and $\Theta ; \Delta, \Phi' \vDash \Phi_2'$,
  we have $\Theta ; \Delta \vDash \Phi_1' \wedge (\Phi' \to \Phi_2')$
  This completes (1), and (2) follows by AT-Anno.
  
  \item[(T-Sub)] Suppose $\Psi ; \Theta ; \Delta ; \Omega ; \Gamma \vdash e : \tau$ from $\Psi ; \Theta ; \Delta ; \Omega ; \Gamma \vdash e : \tau'$ and $\Psi;\Theta;\Delta \vdash \tau' \subty \tau : \star$. By IH, there are $e',\Phi_1,\Gamma'$ so that $|e'| = e$, $\Theta ; \Delta \vDash \Phi_1$, and $\Psi ; \Theta ; \Delta ; \Omega ; \Gamma \vdash e' \infers \tau' \gens \Phi_1,\Gamma'$. By Theorem~\ref{thm:subty-compl}, there is $\Phi_2$ such that $\Theta ; \Delta \vDash \Phi_2$, and $\Psi ; \Theta ; \Delta \vdash \tau' \subty \tau : \star \gens \Phi_2$. By AT-Sub, $\Psi ; \Theta ; \Delta ; \Omega ; \Gamma \vdash e' \checks \tau \gens \Phi_1 \wedge \Phi_2,\Gamma'$, which completes (1). For (2), we apply AT-Anno to get $\Psi ; \Theta ; \Delta ; \Omega ; \Gamma \vdash (e' : \tau) \infers \tau \gens \Phi_1 \wedge \Phi_2,\Gamma'$, and are done.
  
  \item[(T-Weaken)] Suppose $\Psi ; \Theta ; \Delta ; \Omega' ; \Gamma' \vdash e : \tau$ from $\Psi ; \Theta ; \Delta ; \Omega ; \Gamma \vdash e : \tau$, $\Theta ; \Delta \vdash \Omega' \bdby \Omega$, and $\Theta ; \Delta \vdash \Gamma' \bdby \Gamma$. By IH, there are $e'$, $\Phi$, $\Gamma''$ so that $|e'| = e$, $\Theta ; \Delta \vDash \Phi$, and $\Psi ; \Theta ; \Delta ; \Omega ; \Gamma \vdash e : \tau \gens \Phi, \Gamma''$. By Theorem~\ref{thm:admits-weaken}, there are $e_1$, $\Phi_1$, $\Gamma_1$ so that $|e_1| = |e'|$, $\Theta ; \Delta \vDash \Phi_1$, and $\Psi ; \Theta ; \Delta ; \Omega' ; \Gamma' \vdash e_1 \checks \tau \gens \Phi_1,\Gamma_1$, and also that there are $e_2$, $\Phi_2$, $\Gamma_2$ so that $|e_2| = |e'|$, $\Theta ; \Delta \vDash \Phi_2$, and $\Psi ; \Theta ; \Delta ; \Omega' ; \Gamma' \vdash e_2 \infers \tau \gens \Phi_2,\Gamma_2$, which proves (1) and (2).
 
\end{itemize}

\end{proof}

\chapter{}
\label{appendix:b}
\lastructural*
\begin{proof}
All follow by straightforward induction on judgments.
\end{proof}
\fusion*
\begin{proof}
We present terms going in both directions for each caes.
\begin{enumerate}
  \item $x : !^{k_1k_2}_{\ell_1 + k_1\ell_2} \vdash_x \tsfer 1 {k_1k_2} {\ell_1 + k_1\ell_2} y x {\save {k_1} {\ell_1} {(\save {k_2} {\ell_2} {y})}} : !^{k_1}_{\ell_1} !^{k_2}_{\ell_2} A$ and $x : !^{k_1}_{\ell_1} !^{k_2}_{\ell_2} A \vdash_x \tsfer 1 {k_1} {\ell_1} y x {\tsfer {k_1} {k_2} {\ell_2} z y {\save {k_1k_2} {\ell_1 + k_1\ell_2} {z}}} : !^{k_1k_2}_{\ell_1 + k_1\ell_2}$
  
  \item $x : !^k_{\ell_1 + \ell_2}(A \otimes B) \vdash_x \tsfer 1 k {\ell_1 + \ell_2} y x {\asplit {k'} y {z_1} {z_2} {(\save k {\ell_1} {z_1}, \save k {\ell_2} {z_2})}} : !^k_{\ell_1} A \otimes !^k_{\ell_1} B$ and $x : !^k_{\ell_1} A \otimes !^k_{\ell_2} B \vdash_x \asplit 1 x {z_1} {z_2} {\tsfer 1 k {\ell_1} {y_1} {z_1} {\tsfer 1 k {\ell_2} {y_2} {z_2} {\save k {\ell_1 + \ell_2} (y_1,y_2)}}} : !^k_{\ell_1 + \ell_2}(A \otimes B)$.
  
  \item $x : !^k_\ell (A \oplus B) \vdash_x \tsfer 1 k {\ell} y x {\acase k y {z_1} {\inl {(\save k \ell {z_1})}} {z_2} {\inr {(\save k \ell {z_2})}}} : !^k_\ell A \oplus !^k_\ell B$ and $x : !^k_\ell A \oplus !^k_\ell B \vdash_x \acase 1 x {z_1} {\tsfer 1 k {\ell} y {z_1} {\save k \ell {(\inl y)}}} {z_2} {\tsfer 1 k {\ell} y {z_2} {\save k \ell {(\inr y)}}} : !^k_\ell (A \oplus B)$
\end{enumerate}
\end{proof}
\pres*
\begin{proof}
By induction on $M \downarrow v$.
\begin{itemize}
\item (Values): Suppose $\cdot \vdash_a v : A$ and $v \downarrow^{(0,0)} v$. Then, $a + 0 \geq 0$ (because $a \geq 0$), and $\cdot \vdash_{a+0 = a} v : A$.
\item (Tick): Immediate by IH.

\item ($!$-I):  Suppose $\cdot \vdash_b \save k l M : !^k_l A$, and $\save k l M \downarrow^{(\_,kr)} \save k l v$. We must show that $b + kr \geq 0$ and that $\cdot \vdash_{b+kr} \save k l v : !^k_l A$. Inverting the rules, we have that $\cdot \vdash_a M : A$ with $ka+l \leq b$, and that $M \downarrow^{(n,r)} v$. By IH, $\cdot \vdash_{a+r} v : A$ with $a + r \geq 0$. Since $k,l \geq 0$, $0 \leq k(a+r) + l = ka+l+kr$, which, since $ka+l \leq b$, is less than or equal to $b+kr$. So, $b+kr \geq 0$ and $\cdot \vdash_{b+kr} \save k l v : !^k_l A$, as required.

\item ($!$-E): For this case, suppose $\cdot \vdash_{k'a+b} \tsfer {k'} k l x M N : C$, and  $\tsfer{k'} k l x M N \downarrow^{(\_,k'r_1+r_2)} v$. We want to show that: $k'a + b + k'r_1 + r_2 \geq 0$ and $\cdot \vdash_{k'a + b + k'r_1 + r_2} v : C$. By inversion, $\cdot \vdash_a M : !^k_l A$, and $x : A \vdash_{b+k'(kx+l)} N :C$, as well as $M \downarrow^{(\_,r_1)} \save k l {v_1}$ and $N[v_1/x] \downarrow^{(\_,r_2)} v$. By IH, we know that $\cdot \vdash_{a+r_1} \save k l {v_1} : !^k_l A$ and $a+r_1 \geq 0$, so by inversion, there is a $d$ such that $kd+l \leq a+r_1$, and $\cdot \vdash_d v_1 : A$, and so substitution gives that $\cdot \vdash_{b+k'(kd+l)} N[v_1/x] : C$. But $kd+l \leq a+r_1$, so by structural weakening we have $\cdot \vdash_{b+k'a+k'r_1} N[v_1/x] : C$. Again by IH, $\cdot \vdash_{b+r_2 + k'a+k'r_1} v : C$ and $b+r_2 + k'a+k'r_1 \geq 0$, as required.

\item (\waitname): Suppose $\cdot \vdash_a \wait l M : A$ and $\wait l M \downarrow^{(n,r+l)} v$. Inverting, we have $\cdot \vdash_{a+l} M : A$, and $M \downarrow^{(n,r)} v$. By IH, we have that $a + l + r \geq 0$, and $\cdot \vdash_{a+r+l} v$. But, we wanted to show that $a + r + l \geq 0$ and that $\cdot \vdash_{a + r + l} v : A$, and so we are done.
\item (\discname): Suppose $\cdot \vdash_{a+l} \disc l M : A$, and $\disc l M \downarrow^{(\_,r-l)} v$. We want to show that $a + l + r - l = a + r \geq 0$, and that $\cdot \vdash_{a+l+r-l=a+r} v$ Inverting, we have that $\cdot \vdash_a M : A$ and $M \downarrow^{(n,r)} v$. By IH, $a + r \geq 0$ and $\cdot \vdash_{a+r} v : A$, as required.

\item($\otimes$-I): For this case, let $\cdot \vdash_{a+b}(M,N) : A \otimes B$, and $(M,N) \downarrow^{(\_,k_1+k_2)} (v_1,v_2)$. We must show that $a + b + k_1 + k_2 \geq 0$, and that $\cdot \vdash_{a + b + k_1 + k_2 }(v_1,v_2)$
Inverting, we get the four premises $\cdot \vdash_a M : A$, $\cdot \vdash_b N : B$, and $M \downarrow^{(\_,k_1)}v_1$ and $N \downarrow^{(\_,k_2)} v_2$. Using the IH on these two pairs, we get that $a + k_1 \geq 0$, $b + k_2 \geq 0$, $\cdot \vdash_{a+k_1} v_1 : A$, and $\cdot \vdash_{b + k_2} v_2 : B$. Adding the two inequalities and applying $\otimes$-I to the judgments gives the desired result.


\item($\oplus$-E): Suppose $\cdot \vdash_{a+b_1+b_2} \acase {k'} M x {N_1} y {N_2} : C$ and $\acase {k'} M x {N_1} y {N_2} \downarrow^{(\_,k'r_1 + r_2)} v$. 
Inverting the typing judgment, $\cdot \vdash_a M : A \oplus B$, $x : A \vdash_{b_1 + k'x} N_1 : C$ and $y : B \vdash_{b_2 + k'} N_2 : C$. Inverting the evaluation judgment gives two symmetric cases, so suppose that $M \downarrow^{(\_,r_1)} \inl {v_1}$ and $N_1[v_1/x] \downarrow^{(\_,r_2)} v$. By IH, $\cdot \vdash_{a+r_1} \inl {v_1} : A \oplus B$ and $a + r_1 \geq 0$. So, $\cdot \vdash_{a+r_1} v_1 : A$. By substitution, $\cdot \vdash_{b_1 + k'(a+r_1)} N_1[v_1/x] : C$. By IH, $\cdot \vdash_{k'a + b_1 + k'r_1 +  r_2} v : C$ and $k'a + b_1 + k'r_1 +  r_2 \geq 0$. Since $b_2 \geq 0$, by structural weakening, $\cdot \vdash_{k'a + b_1 + b_2 + k'r_1 +  r_2} v : C$ and $k'a + b_1 + b_2 + k'r_1 +  r_2 \geq 0$, as required.

\item ($\loli$-E):  Let $\cdot \vdash_{a+b} M \, N : B$, and $M \, N \downarrow^{(\_,k_1+k_2+k_3)} v$. We want to show that $a + b + k_1 + k_2 + k_3 \geq 0$, and that $\cdot \vdash_{a + b + k_1 + k_2 + k_3} v : B$. We invert both judgments to get $\cdot \vdash_a M : A \loli B$, and $\cdot \vdash_b N : A$, and $M\downarrow^{(\_,k_1)} \lambda x.M'$, and $N \downarrow^{(\_,k_2)} v_1$, and that $M'[v_1/x] \downarrow^{(\_,k_3)} v$. Applying the IH to the first evaluation, we have that $\cdot \vdash_{a+k_1} \lambda x.M' : A \loli B$. Inverting the proof of that judgment, we get that $x : A \vdash_{a+k_1+x} M' : B$. By IH again, $\cdot \vdash_{b+k_2} v_1 : A$, and by substitution, $\cdot \vdash_{a+b+k_1+k_2} M'[v_1/x]$. By IH once more, $a + b + k_1 + k_2 + k_3 \geq 0$, and $\cdot \vdash_{a + b + k_1 + k_2 + k_3} v : B$, as required.

\item ($\N$-E) Suppose $\cdot \vdash_{a+b_1+b_2} \nrec {M} {N_1} {N_2} : C$. By inversion, $\cdot \vdash_a M : \N$, $\cdot \vdash_{b_1} N_1 : 1 \loli C$, and $\cdot \vdash_{b_2} N_2 : !^\infty_0(\N \tensor (1 \loli C) \loli C)$. We have two evaluation cases to consider.
\begin{itemize}
  \item Suppose $\nrec M {N_1} {N_2} \downarrow^{(\_,r_1+r_2+r_3+r_3)} v$ by way of $M \downarrow^{(\_,r_1)} 0 : \N$, $N_1 \downarrow^{(\_,r_2)} \lambda x.N_1'$, $N_2 \downarrow^{(\_,r_3)} \_$, and $N_1'[()/x] \downarrow^{(\_,r_4)} v$. Then, by IH, we have the following:
  \begin{itemize}
    \item $\cdot \vdash_{a+r_1} 0 : \N$, and $a + r_1 \geq 0$
    \item $\cdot \vdash_{b_1+r_2} \lambda x.N_1' : 1 \loli C$, $b_1 + r_2 \geq 0$.
    \item $b_2 + r_3 \geq 0$
  \end{itemize}
  Since $\cdot \vdash_0 () : 1$, $\cdot \vdash_{b_1 + r_1} N_1'[()/x] : C$. By IH, $\cdot \vdash_{b_1 + r_2 + r_4} v : C$. By structural weakening, $\cdot \vdash_{a+b_1+b_2 + r_1 + r_2 + r_3 + r_4} v : C$, as required.
  \item Suppose $\nrec M {N_1} {N_2} \downarrow^{(\_,r_1+r_2+r_3+r_3)} v$ by way of $M \downarrow^{(\_,r_1)} S(v_1) : \N$, $N_1 \downarrow^{(\_,r_2)} \lambda x.N_1'$, $N_2 \downarrow^{(\_,r_3)} \save \infty 0 {(\lambda x.N_2')}$, and $N_2'[(v_1,\lambda z.\nrec {v_1} {\lambda x.N_1'} {\save \infty 0 {(\lambda x.N_2')}})/x] \downarrow^{(\_,r_4)} v$. Then, by IH we have:
  \begin{itemize}
    \item $\cdot \vdash_{a+r_1} S(v_1) : \N$, $a + r_1 \geq 0$
    \item $\cdot \vdash_{b_1 + r_2} \lambda x.N_1' : 1 \loli C$, $b_1 + r_2 \geq 0$
    \item $\cdot \vdash_{b_2 + r_3} \save \infty 0 {(\lambda x.N_2')} : !^\infty_0(\N \otimes (1 \loli C) \loli C)$, and $b_2 + r_3 \geq 0$.
  \end{itemize}
  By $\N$-strengthening, $\cdot \vdash_0 S(v_1) : \N$, and so $\cdot \vdash_0 v_1 : \N$. Since $\cdot \vdash_{b_2 + r_3} \save \infty 0 {(\lambda x.N_2')} : !^\infty_0(\N \otimes (1 \loli C) \loli C)$, there is a $c \geq 0$ so that $\infty \cdot c \leq b_2 + r_3$ with $\cdot \vdash_c \lambda x.N_2' : \N \otimes (1 \loli C) \loli C$. Then, $\cdot \vdash_{b_1 + r_3 + \infty \cdot c} \nrec {v_1} {\lambda x.N_1'} {\save \infty 0 {(\lambda x.N_2')}} : C$, and so $\cdot \vdash_{a+b_1+r_1+r_2 + \infty \cdot c} (v_1,\lambda z.\nrec {v_1} {\lambda x.N_1'} {\save \infty 0 {(\lambda x.N_2')}}) : \N \otimes (1 \loli C)$. Thus, since $x : \N \otimes (1 \loli C) \vdash_{x + c} N_2' : C$,
  $$
  \cdot \vdash_{a+b_1+r_1+r_2 + \infty \cdot c} N_2'[(v_1,\lambda z.\nrec {v_1} {\lambda x.N_1'} {\save \infty 0 {(\lambda x.N_2')}})/x] : C  
  $$
  since $\infty \cdot c + c = \infty \cdot c$. So, by IH, $\cdot \vdash_{a+b_1 + r_1 + r_2 + \infty \cdot c + r_4} v : C$, and because $\infty \cdot c \leq b_2 + r_3$, we have by weakening that $\cdot \vdash_{a+b_1+b_2+r_1+r_2+r_3+r_4} v : C$ as required.
\end{itemize}
  
\item ($[A]$-E) Suppose $\cdot \vdash_{a + b_1 + b_2} \lrec M {N_1} {N_2} : C$. Then, $\cdot \vdash_a M : A$,  $\cdot \vdash_{b_1} N_1 : 1 \to C$, and $\cdot \vdash_{b_2} : N_2 : !^\infty_0(A \otimes (\listty A \amp C) \loli C)$. We have two evaluation cases to consider. 

Firstly, suppose $\lrec M {N_1} {N_2} \downarrow^{(\_,r_1+r_2+r_3+r_4)} v$ by way of $M \downarrow^{(\_,r_1)} v$, $N_1 \downarrow^{(\_,r_2)} \lambda x.N_1'$, $N_2 \downarrow^{(\_,r_3)} v'$, and $N_1'[()/x] \downarrow^{(\_,r_4)} v$. Then, by IH, $a + r_1 \geq 0$, which means that $\cdot \vdash_{a+r_1} () : 1$. By IH, $\cdot \vdash_{b_1 + r_2} \lambda x.N_1'$ and $b_1 + r_2 \geq 0$. So, by inversion and then substitution, $\cdot \vdash_{a+b_1+r_1+r_2} N_1'[()/x] : C$. By IH, $b_2 + r_3 \geq 0$, so by weakening, $\cdot \vdash_{a+b_1+b_2+r_1+r_2+r_3} N_1'[()/x]$. Finally, by IH, $a+b_1+b_2+r_1+r_2+r_3 + r_4 \geq 0$ and $\cdot \vdash_{a+b_1+b_2+r_1+r_2+r_3+r_4} v : C$.

\sloppypar Now, suppose $\lrec M {N_1} {N_2} \downarrow^{(\_,r_1+r_2+r_3+r_4)} v$ by way of $M \downarrow^{(\_,r_1)} \cons{v_1} {v_2}$, $N_1 \downarrow^{(\_,r_2)} \lambda x.N_1'$, $N_2 \downarrow^{(\_,r_3)} \save \infty 0 {(\lambda x.N_2')}$, and $N_2'[(v_1,\amppair {v_2} {\lrec {v_2} {\lambda x.N_1'} {\save \infty 0 {(\lambda x.N_2')}}})/x] \downarrow^{(\_,r_4)} v$. By IH, $\cdot \vdash_{a + r_1} \cons {v_1} {v_2} : \listty A$ and $a + r_1 \geq 0$. By inversion, there are $d_1,d_2 \geq 0$ so that $a+r_1 = d_1 + d_2$ and $\cdot \vdash_{d_1} v_1 : A$ and $\cdot \vdash_{d_2} v_2 : \listty A$. By two more applications of the IH, $\cdot \vdash_{b_1 + r_2} \lambda x.N_1' : 1 \loli C$, $\cdot \vdash_{b_2 + r_3} \save \infty 0 {(\lambda x.N_2')} : !^\infty_0(A\otimes(\listty A \amp C) \loli C)$, with $b_1 + r_2 \geq 0$ and $b_2 + r_3 \geq 0$. By inversion, there is some $c \geq 0$ with $\infty \cdot c \leq b_2 + r_3$ such that $\cdot \vdash_c \lambda x.N_2' : A \otimes (\listty A \amp C) \loli C$. Next,
$$
   \infer{
     \cdot \vdash_{d_2 + b_1 + r_2 + \infty \cdot c} \lrec {v_2} {\lambda x.N_1'} {\save \infty 0 {(\lambda x.N_2')}} : C
   }{
     \cdot \vdash_{d_2} v_2 : \listty A
     &
     \cdot \vdash_{b_1 + r_2} \lambda x.N_1' : 1 \loli C
     &
     \cdot \vdash_{\infty \cdot c} \save \infty 0 {(\lambda x.N_2')} : !^\infty_0 (A \otimes (\listty A \amp C) \loli C)
   }
$$
then, with $\cdot \vdash_{d_2 + b_1 + r_2 + \infty \cdot c} v_2 : \listty A$,
we have that
$$
\cdot \vdash_{d_2 + b_1 + r_2 + \infty \cdot c} \amppair {v_2} {\lrec {v_2} {\lambda x.N_1'} {\save \infty 0 {(\lambda x.N_2')}}} : \listty A \amp C
$$
and since $\cdot \vdash_{d_1} v_1 : A$,
$$
\cdot \vdash_{a + r_1 + b_1 + r_2 + \infty \cdot c} (v_1,\amppair {v_2} {\lrec {v_2} {\lambda x.N_1'} {\save \infty 0 {(\lambda x.N_2')}}}) : A \otimes (\listty A \amp C)
$$
and so by substitution, and using the fact that $c + \infty \cdot c = \infty \cdot c$, $\cdot \vdash_{a + b_1 + r_1 + r_2 + \infty \cdot c} N_2'[(v_1,\amppair {v_2} {\lrec {v_2} {\lambda x.N_1'} {\save \infty 0 {(\lambda x.N_2')}}})/x] : C$. By weakening, since $\infty \cdot c \leq b_2 + r_3$, 
$\cdot \vdash_{a + b_1 + b_2 + r_1 + r_2 + r_3} N_2'[(v_1,\amppair {v_2} {\lrec {v_2} {\lambda x.N_1'} {\save \infty 0 {(\lambda x.N_2')}}})/x] : C$. Finally, by IH, $\cdot \vdash_{a+b_1+b_2+r_1+r_2+r_3+r_4} v : C$, and $a+b_1+b_2+r_1+r_2+r_3+r_4 \geq 0$, as required.

\item ($\amp$-I): Immediate.
\item ($\amp$-E): By symmetry, it suffices to only consider the $\pi_1$ case. Let $\cdot \vdash_a \pi_1 M : A$ and $\pi_1 M \downarrow^{(\_,r_1+r_2)} v$. By inversion, we have that $\cdot \vdash_a M : A \amp B$, and that $M \downarrow^{(\_,r_1)} \amppair {N_1} {N_2}$ and $N_1 \downarrow^{(\_,r_2)}$. We must show that $\cdot \vdash_{a+r_1+r_2} v : A$, and that $a+r_1+r_2 \geq 0$. By IH, $\cdot \vdash_{a+r_1} \amppair {N_1} {N_2} : A \amp B$. Inverting this, we get that $\cdot \vdash_{a+r_1} N_1 : A$, and so again by IH, $\cdot_{a+r_1+r_2} v : A$, and $a+r_1+r_2 \geq 0$, as required.
\end{itemize}
\end{proof}

\valevalzero*
\begin{proof}
By inspection of cases.
\end{proof}
\natstren*
\begin{proof}
By canonical forms, $v = \overline{n}$, proceed by induction on $n$.
\end{proof}
\extrsound*
\begin{proof}
By induction on $\Gamma \vdash_f M : A$
\end{proof}
\weakening*
\begin{proof}
We prove 1 and 2 simultaneously by induction on $A$.
\begin{enumerate}
  \item Suppose $M \bdby^{A,a} E$, and $E \leq_{\mathbb{C} \times \angles{A}} E'$. We need to show that $M \bdby^{A,a} E'$. Suppose $M \downarrow^{(n,r)} v$.
  We need to show:
  \begin{itemize}
    \item $n \leq E'_c - r$
    \item $v \valbd^{A,a+r} E'_p$
  \end{itemize}
  But, since $M \bdby^{A,a} E$
  \begin{itemize}
    \item $n \leq E_c - r$
    \item $v \valbd^{A,a+r} E_p$
  \end{itemize} 
  so, it suffices to show that $E_c \leq_{\mathbb{C}} E'_c$ and $E_p \leq_{\angles A} E'_p$, which is true by the $\pi_1(-)$ and $\pi_2(-)$ congruences, recalling that $(-)_c$ and $(-)_p$ are simply $\pi_1$ and $\pi_2$.
  \item Let $E \leq_{\angles A} E'$. We have a few cases to consider.
  \begin{itemize}
    \item[($!$)] Suppose $\save k l v \valbd^{!^k_l A,a} E$. We must show that $\save k l v \valbd^{!^k_l A,a} E'$. We know that there is a $d \geq 0$ such that $ka+l \leq d$, and $v \valbd^{A,d} E$. So, by IH, $v \valbd^{A,d} E'$, and hence $\save k l v \valbd^{!^k_l A,a} E'$, as required.
    \item[($\loli$)] Suppose $\lambda x.M \valbd^{A\loli B,a} E$. We need to show that $\lambda x.M \valbd^{A\loli B,a} E'$. Let $v \valbd^{A,b} E_v$. Then, $M[v/x] \bdby^{B,a+b} E \; E_v$. Using the application congruence and 1, $M[v/x] \bdby^{B,a+b} E' \; E_v$. Since $v,b,E_v$ were chosen arbitrarily, $\lambda x.M \valbd^{A \loli B,a} E'$ as required.
    \item[($\tensor$)] Suppose $(v_1,v_2) \valbd^{A_1\otimes A_2,a} E$. Then, there are $a_1,a_2$ such that $a_1 + a_2 = a$, and $v_i \valbd^{A_i,a_i} \pi_i E$, and so by $\pi_i$-congruence and the IH, $v_i \valbd^{A_i,a_i} \pi_i E'$, so $(v_1,v_2) \valbd^{A_1\otimes A_2,a} E'$, as required.
    \item[($\listty{A}$)] Both cases are immediate by transitivity.
    \item[($\N$)] Both cases are immediate by transitivity.
    \item[($\oplus$)] Both cases are immediate by transitivity.
    \item[($A \amp B$)] Suppose $\amppair {M_1} {M_2} \valbd^{A_1 \amp A_2,a} E$. Then, for $i \in \{1,2\}$, $M_i \bdby^{A_i,a} \pi_i E$. By $\pi_i$-congruence, $\pi_i E \leq_{\norm{A}} \pi_i E'$, and so by IH from 1, we know that $M_i \bdby^{A_i,a} \pi_i E'$, and are done.
  \end{itemize}
\end{enumerate}
\end{proof}

\credwkn*
\begin{proof}
  We prove the two claims simultaneously.
  \begin{enumerate}
   \item[(1)] Suppose $M \bdby^{A,a_1} E$. To show $M \bdby^{A,a_2} E$, suppose $M \downarrow^{(n,r)} v$. We must show that
   such that
   \begin{itemize}
     \item $n \leq E_c - r$
     \item $v \valbd^{A,a_2+r} E_p$
   \end{itemize}
   But, since $M \bdby^{A,a_1} E$, we have
   \begin{itemize}
     \item $n \leq E_c - r$
     \item $v \valbd^{A,a_1 + r} E_p$
   \end{itemize}
  Since $a_1 \leq a_2$, $a_1 + r \leq a_2+ r$, so we are done by (2).
  \item[(2)] By lexicographic induction on first $A$ and then the size of $v$.
  \begin{itemize}
       \item[($!$)] Let $\save k l v \valbd^{!^k_l A,a_1} E$. Then, there is a $d \geq 0$ such that $kd + l \leq a_1$ and $v \valbd^{A,d} E$. But $kd+l\leq a_1 \leq a_2$, and so $\save k l v \valbd^{!^k_l A,a_2} E$
       \item[($\loli$)] Let $\lambda x.M \valbd^{A \loli B,a_1} E$. Suppose $v \valbd^{A,b} E'$. Then, $M[v/x] \bdby^{B,a_1 + b} E \; E'$, and so by (1), $M[v/x] \bdby^{B,a_2 + b} E \; E'$. Since $v$ was chosen arbitrarily, $\lambda x.M \valbd^{A \loli B,a_2} E$, as required.
    \item[($\tensor$)] Let $(v_1,v_2) \valbd^{A_1 \otimes A_2,a_1} E$. Then, there are $b_1,b_2$ with $b_1 + b_2$ such that $b_1 + b_2 = a_1$, and $v_i \valbd^{A_i,a_1} \pi_i E$ for $i \in \{1,2\}$. By the IH on $v_1$, we have that $v_1 \valbd^{A_1,b_1 + a_2 - a_1} \pi_1 E$, and so $(v_1,v_2) \valbd^{A_1 \otimes A_2,a_2} E$, as required.
    \item[($\listty{A}$)] The empty case is immediate. Suppose $\cons {v_1} {v_2} \valbd^{\listty A, a_1} E$. Then, there are $E_1, E_2,b_1,b_2$ such that $b_1 + b_2 = a_1$, $\cons {E_1} {E_2} \leq E$, $v_1 \valbd^{A,b_1} E_1$, and $v_1 \valbd^{\listty A,b_2} E_2$. By IH, $v_1 \valbd^{A,b_1 + a_2 - a_1} E_1$, and so $\cons {v_1} {v_2} \valbd^{\listty A, a_2} E$.
    \item[($\N$)] The zero case is immediate. Suppose $S(v) \valbd^{\N,a_1} E$. Then, there is $E'$ such that $S(E') \leq E$, and $v \valbd^{\N,a_1} E'$. Since $v$ is a smaller term than $S(v)$, we can apply the IH to see that $v \valbd^{\N,a_2} E'$, and so $S(v) \valbd^{\N,a_2} E$, as desired.
    \item[($\oplus$)] The two cases are symmetric, so we present only one. Suppose $\inl v \valbd^{A \oplus B,a_1} E$. Then we have $E'$ such that $\inl E' \leq E$, and $v \valbd^{A,a_1} E'$, which, by IH, means that $v \valbd^{A,a_2} E'$, and so $\inl v \valbd^{A \oplus B,a_2} E$.
    \item[($A \amp B$)] Immediate by IH.
  \end{itemize}
  \end{enumerate}
\end{proof}

\nreclemma*
\begin{proof}
Proceed by induction on $n$.

For notational simplicity, let $E_2^* = \lambda p. E_2 (\pi_1 p,(\lambda z. \pi_2 p))$

\begin{itemize}
  \item ($n = 0$): To show $\nrec 0 {\lambda x.N_1'} {\save \infty 0 {(\lambda x.N_2')}} \bdby^{C,c_3 + \infty \cdot d} \nrec {E} {E_1} {E_2^*}$, suppose that $\nrec 0 {\lambda x.N_1'} {\save \infty 0 {(\lambda x.N_2')}} \downarrow^{(n,r)} v$ by way of $N_1'[()/x] \downarrow^{(n,r)}$.
  
  We must show that :
  \begin{itemize}
    \item $n \leq \nrec E {E_1} {E_2^*}_c - r$
    \item $v \valbd^{C,c_3 + \infty \cdot d} \nrec E {E_1} {E_2^*}_p$
  \end{itemize}
  We know $N_1[()/x] \bdby^{C,c_3} E_1 \; ()$, since $() \valbd^{1,0} ()$, and so:
  \begin{itemize}
    \item $n \leq (E_1 \; ())_c - r$
    \item $v \valbd^{C,c_3} (E_1 \; ())_p$
  \end{itemize}
  
  Since $0 \valbd^{\N,0} E$, $0 \leq_\N E$, and so $E_1 \; () \leq \nrec 0 {E_1} {E_2^*} \leq \nrec E {E_1} {E_2^*}$.
  
  \item \sloppypar ($n > 0$): Suppose that $\nrec {S(\overline{n})} {\lambda x.N_1'} {\save \infty 0 {(\lambda x.N_2')}} \downarrow^{(n',r)} v$ by way of 
$$
  N_2'[(\overline{n},\lambda z.\nrec {\overline{n}} {\lambda x.N_1'} {\save \infty 0 {(\lambda x.N_2')}})] \downarrow^{(n',r)} v.
$$ 
We must show that:
  \begin{itemize}
    \item $n' \leq \nrec {E} {E_1} {E_2^*}_c - r$;
    \item $v \valbd^{C,c_3+\infty \cdot d + r} \nrec E {E_1} {E_2^*}_p$.
  \end{itemize}
  Since $S(\overline{n}) \valbd^{\N,0} E$, there is an $E'$ so that $S(E') \leq_\N E$, and $\overline{n} \valbd^{\N,0} E'$. For notational convenience, let $E^* = (E',\nrec {E'} {E_1} {E_2^*})$. Note that $E' \leq \pi_1 E^*$, and that $\nrec {E'} {E_1} {E_2^*} \leq \pi_2 E^*$. By IH, $\nrec {\overline{n}} {\lambda x.N_1'} {\save \infty 0 {(\lambda x.N_2')}} \bdby^{C,c_3 + \infty \cdot d} \nrec {E'} {E_1} {E_2^*}$, and thus by weakening $\nrec {\overline{n}} {\lambda x.N_1'} {\save \infty 0 {(\lambda x.N_2')}} \bdby^{C,c_3 + \infty \cdot d} \pi_2 E^*$. For some variable $z$ not free in the term on the left, $\lambda z. \nrec {\overline{n}} {\lambda x.N_1'} {\save \infty 0 {(\lambda x.N_2')}} \valbd^{1 \loli C,c_3 + \infty \cdot d} \lambda z. \pi_2 E^*$, and so $(\overline{n},\lambda z. \nrec {\overline{n}} {\lambda x.N_1'} {\save \infty 0 {(\lambda x.N_2')}}) \valbd^{\N \otimes (1 \loli C),c_3 + \infty \cdot d} (\pi_1 E^*,\lambda z. \pi_2 E^*s)$, and since $\lambda x.N_2' \valbd^{\N \times (1 \loli C) \loli C,d} E_2$, using the fact that $\infty \cdot d + d = \infty \cdot d$
  $$N_2'[(\overline{n},\lambda z. \nrec {\overline{n}} {\lambda x.N_1'} {\save \infty 0 {(\lambda x.N_2')}})/x] \bdby^{C,c_3 + \infty \cdot d} E_2 \; (\pi_1 E^*,\lambda z.\pi_2 E^*)$$
  but,
  \begin{align*}
   E_2 \; (\pi_1 E^*,\lambda z.\pi_2 E^*) &\leq (\lambda p. E_2 \; (\pi_1 p, \lambda z.\pi_2 p)) E^*\\
   &= E_2^* (E',\nrec {E'} {E_1} {E_2^*})\\
   &\leq \nrec {S(E')} {E_1} {E_2^*}\\
   &\leq \nrec {E} {E_1} {E_2^*}
  \end{align*}
  and so we are done by weakening.
\end{itemize}
\end{proof}

\lreclemma*
\begin{proof}
We proceed by induction on the derivation of $\cdot \vdash_d v : \listty A$.
First, suppose $v = []$.  To show that $\lrec {[]} {\lambda x.N_1'} {\save \infty 0 {(\lambda x.N_2')}} \bdby^{C,c_1+d+\infty \cdot c_2} \lrec E {E_1} {\lambda x. E_2 (\pi_1 x,((0,\pi_1 \pi_2 x),\pi_2\pi_2 x))}$, assume that  $\lrec {[]} {\lambda x.N_1'} {\save \infty 0 {(\lambda x.N_2')}} \downarrow^{(n,r)} v$. By inversion, it was by way of $N_1'[()/x] \downarrow^{(n,r)} v$. It suffices to show 
\begin{itemize}
   \item $n \leq \lrec E {E_1} {\lambda x. E_2 (\pi_1 x,((0,\pi_1 \pi_2 x),\pi_2\pi_2 x))}_c - r$
   \item $v\ \valbd^{C,c_1+d+\infty \cdot c_2 + r} \lrec E {E_1} {\lambda x. E_2 (\pi_1 x,((0,\pi_1 \pi_2 x),\pi_2\pi_2 x)) }_p$
\end{itemize}
Since $() \leq_1 ()$, $() \valbd^{1,d} ()$, so $N_1'[()/x] \bdby^{C,c_1+d} E_1 \; ()$, and so
\begin{itemize}
  \item $n \leq (E_1 \; ())_c - r$
  \item $v \valbd^{C,c_1+d+r} (E_1 \; ())_p$
\end{itemize}
But, $\infty \cdot c_2 > 0$ since $c_2 > 0$, and so by credit weakening, $v \valbd^{C,c_1+d+\infty \cdot c_2 +r} (E_1 \; ())_p$. Note that, by assumption, $[] \valbd^{1,d} E$, which means that $[] \leq_{\listty {\angles A}} E$. So,
$$
E_1 \; () \leq \lrec {[]} {E_1} {\lambda x. E_2 (\pi_1 x,((0,\pi_1 \pi_2 x),\pi_2\pi_2 x)) } \leq \lrec E {E_1} {\lambda x. E_2 (\pi_1 x,((0,\pi_1 \pi_2 x),\pi_2\pi_2 x)) }
$$
and so we are done by weakening.

Otherwise, suppose $v = \cons {v_1} {v_2}$. To show that $\lrec {\cons {v_1} {v_2}} {\lambda x.N_1'} {\save \infty 0 {(\lambda x.N_2')}} \bdby^{C,c_1+d+\infty \cdot c_2} \lrec E {E_1} {\lambda x. E_2 (\pi_1 x,((0,\pi_1 \pi_2 x),\pi_2\pi_2 x)) }$, suppose $\lrec {\cons {v_1} {v_2}} {\lambda x.N_1'} {\save \infty 0 {(\lambda x.N_2')}} \downarrow^{(n,r)} v$. By inversion, it was by $N_2'[(v_1,\amppair {v_2} {\lrec {v_2} {\lambda x.N_1'} {\save \infty 0 {(\lambda x.N_2')}}})/x] \downarrow^{(n,r)} v$. It suffices to show:
\begin{itemize}
  \item $n \leq \lrec E {E_1} {\lambda x. E_2 (\pi_1 x,((0,\pi_1 \pi_2 x),\pi_2\pi_2 x)) }_c - r$
  \item $v \valbd^{C,c_1+d+\infty \cdot c_2 + r} \lrec E {E_1} {\lambda x. E_2 (\pi_1 x,((0,\pi_1 \pi_2 x),\pi_2\pi_2 x)) }_p$
\end{itemize}

Since $\cons {v_1} {v_2} \valbd^{\listty A,d} E$, there are $d_1,d_2 \geq 0$ such that $d_1 + d_2 = d$, along with $E'$,$E''$ such that $v_1 \valbd^{A,d_1} E'$ and $v_2 \valbd^{\listty A,d_2} E''$, and $\cons {E'} {E''} \leq E$

By IH, $\lrec {v_2} {\lambda x.N_1'} {\save \infty 0 {(\lambda x.N_2')}} \bdby^{C,c_1 + d_2 + \infty \cdot c_2} \lrec {E''} {E_1} {\ldots}$. 
Since $v_2 \valbd^{\listty A,d_2} E''$, $v_2 \bdby^{\listty A,d_2} (0,E'')$, 
and since $c_1 + \infty \cdot c_2 \geq 0$, we have by credit weakening that $v_2 \bdby^{\listty A,c_1 + d_2 + \infty \cdot c_2} (0,E'')$. 
So, 
$\amppair {v_2} {\lrec {v_2} {\lambda x.N_1'} {\save \infty 0 {(\lambda x.N_2')}}} \valbd^{\listty A \amp C,c_1 + d_2 + \infty \cdot c_2} ((0,E''),\lrec {E''} {E_1} {\ldots})$. 
Further, using the fact that $d_1 + d_2 = d$, 

$$
\begin{array}{l}
(v_1,\amppair {v_2} {\lrec {v_2} {\lambda x.N_1'} {\save \infty 0 {(\lambda x.N_2')}}}) \valbd^{A \otimes (\listty A \amp C), c_1 + d + \infty \cdot c_2}\\ (E',((0,E''),\lrec {E''} {E_1} {\ldots}))
\end{array}
$$
Thus, since $\lambda x.N_2' \valbd^{A \otimes (\listty A \amp C) \loli C,c_2} E_2$, we have (using the fact that $c_2 + \infty \cdot c_2 = \infty \cdot c_2$)
$$
N_2'[(v_1,\amppair {v_2} {\lrec {v_2} {\lambda x.N_1'} {\save \infty 0 {(\lambda x.N_2')}}})/x] \bdby^{C,c_1+d+\infty \cdot c_2} E_2 \; (E',((0,E''),\lrec {E''} {E_1} {\ldots}))
$$
By definition, this means that
\begin{itemize}
  \item $n \leq (E_2 \; (E',((0,E''),\lrec {E''} {E_1} {\ldots})))_c - r$
  \item $v \valbd^{c_1+d+\infty \cdot c_2 + r} (E_2 \; (E',((0,E''),\lrec {E''} {E_1} {\ldots})))_p$
\end{itemize}
We then compute:
\begin{align*}
&E_2 \; (E',((0,E''),\lrec {E''} {E_1} {\ldots}))\\
&\leq (\lambda x. E_2 (\pi_1 x,((0,\pi_1 \pi_2 x),\pi_2\pi_2 x))) \; (E',(E'',\lrec {E''} {E_1} {\lambda x. E_2 (\pi_1 x,((0,\pi_1 \pi_2 x),\pi_2\pi_2 x))}))\\
&\leq \lrec {\cons {E'} {E''}} {E_1} {\lambda x. E_2 (\pi_1 x,((0,\pi_1 \pi_2 x),\pi_2\pi_2 x))s}\\
&\leq \lrec {E} {E_1} {\lambda x. E_2 (\pi_1 x,((0,\pi_1 \pi_2 x),\pi_2\pi_2 x)) }
\end{align*}

and hence we are done by weakening.
\end{proof}

\bounding*
\begin{proof}
By induction on $\Gamma \vdash_f M : A$.
\begin{itemize}

\item[($!$-I)] 
Let $\Gamma \vdash_g \save k l M : !^k_l A$. By inversion, we have $\Gamma \vdash_f M : A$ with $kf + l \leq g$. Let $\theta \subbd^{\Gamma,\sigma} \Theta$. To show $\save k l {M}[\theta] \bdby^{!^k_l A,g[\sigma]} (k\norm{M}[\Theta]_c,\norm{M}[\Theta]_p)$, it suffices to show $\save k l {M}[\theta] \bdby^{!^k_l A,kf[\sigma] + l} (k\norm{M}[\Theta]_c,\norm{M}[\Theta]_p)$ by credit weakening. So, let $\save k l M \downarrow^{(n,kr)} \save k l v$ by way of $M \downarrow^{(n,r)} v$. It suffices to show, using the fact that $kf[\sigma] + l + kr = k(f[\sigma] + r) + l$
\begin{itemize}
  \item $n \leq k\norm{M}[\Theta]_c - kr$
  \item $\save k l v \valbd^{!^k_l A,k(f[\sigma] + r) + l} \norm{M}[\Theta]_p$
\end{itemize}

To show $\save k l v \valbd^{!^k_l A,k(f[\sigma] + r) + l} \norm{M}[\Theta]_p$, it suffices to provide $d \geq 0$ such that $kd+l \leq k(f[\sigma] + r) + l$, and $v \valbd^{A,d} \norm{M}[\Theta]_p$.
By IH, $M[\theta] \bdby^{A,f[\sigma]} \norm{M}[\Theta]$, which means that
\begin{itemize}
 \item $n \leq \norm{M}[\Theta]_c - r$
 \item $v \valbd^{A,f[\sigma] + r} \norm{M}[\Theta]_p$
\end{itemize}
So, $d = f[\sigma] + r$, and the inequality $n \leq kn \leq k\norm{M}[\Theta]_c - kr$ follows by multiplying the above one by $k$ (since $k \geq 1$).

\item[($!$-E)]  Let $\Gamma \vdash_{k'f + g} \tsfer {k'} k l x M N : C$. By inversion, we have that $\Gamma \vdash_f M : !^k_l A$, as well as $\Gamma,x : A\vdash_{g+k'(kx+l)} N : C$. Suppose $\theta \subbd^{\Gamma,\sigma} \Theta$. 
We need to show that $\tsfer {k'} k l x {M[\theta]} {N[\theta]} \bdby^{C,k'f[\sigma] + g[\sigma]} \norm{M}[\Theta]_c +_c \norm{N}[\Theta,\norm{M}[\Theta]_p/x]$. 
Suppose $\tsfer {k'} k l x {M[\theta]} {N[\theta]} \downarrow^{(n_1+n_2,k'r_1+r_2)} v$. By inversion, it was by $M[\theta] \downarrow^{(n_1,r_1)} \save k l {v_1}$ and $N[\theta,v_1/x] \downarrow^{(n_2,r_2)} v$.
It suffices to show that
\begin{itemize}
  \item $n_1 + n_2 \leq k'\norm{M}[\Theta]_c + \norm{N}[\Theta,\norm{M}[\Theta]_p/x]_c - (k'r_1 + r_2)$
  \item $v \valbd^{C,k'f[\sigma] + g[\sigma] + k'r_1 + r_2} \norm{N}[\Theta,\norm{M}[\Theta]_p/x]_p$
\end{itemize}
By IH, we have that $M[\theta] \bdby^{!^k_l A, f[\sigma]} \norm{M}[\Theta]$, which means that there are $b_1,c_1$ with $b_1 + c_1 = f[\sigma]$ and
\begin{itemize}
  \item $n_1 \leq \norm{M}[\Theta]_c - r_1$
  \item $\save k l {v_1} \valbd^{!^k_l A,f[\sigma] + r_1} \norm{M}[\Theta]_p$
\end{itemize}
Since $\save k l {v_1} \valbd^{!^k_l A,f[\sigma] + r_1} \norm{M}[\Theta]_p$, there is a $d \geq 0$ such that $kd+l \leq f[\sigma] + r_1$, and $v_1 \valbd^{A,d} \norm{M}[\Theta]_p$. Thus, $(\theta,v_1/x) \subbd^{(\Gamma,x:A),(\sigma,x\mapsto d)} (\Theta,\norm{M}[\Theta]_p/x)$, and so by IH, $N[\theta,v_1/x] \bdby^{C,g[\sigma] + k'(kd+l)} \norm{N}[\Theta,\norm{M}[\Theta]_p/x]$.
By credit weakening, since $kd+l \leq f[\sigma] + r_1$,  $N[\theta,v_1/x] \bdby^{C,g[\sigma] + k'(f[\sigma] + r_1)} \norm{N}[\Theta,\norm{M}[\Theta]_p/x]$. This gives us that
\begin{itemize}
  \item $n_2 \leq \norm{N}[\Theta,\norm{M}[\Theta]_p/x]_c - r_2$
  \item $v \valbd^{C,g[\sigma] + k'(f[\sigma] + r_1) + r_2} \norm{N}[\Theta,\norm{M}[\Theta]_p/x]_p$
\end{itemize}
To establish the desired inequality, we multiply the first inequality by $k'$, to find that $k'n_1 \leq k'\norm{M}[\Theta]_c - k'r_1$. But $k' \geq 1$,
so $n_1 \leq k'n_1 \leq k'\norm{M}[\Theta]_c - k'r_1$. Therefore, $n_1 + n_2 \leq k'\norm{M}[\Theta]_c + \norm{N}[\Theta,\norm{M}[\Theta]_p/x]_c - (k'r_1 + r_2)$ as required. For value bounding, we note that $g[\sigma] + k'(f[\sigma] + r_1) + r_2 = k'f[\sigma] + g[\sigma] + k'r_1 +r_2$, and are done.

\item[(\discname)] Let $\Gamma \vdash_{f + l} \disc l M : A$.  By inversion, $\Gamma \vdash_f M : A$.
 To show $\disc l M \bdby^{A} (-l) +_c \norm{M}$, suppose $\theta \subbd^{\Gamma,\sigma} \Theta$. 
 To show $\disc l M[\theta] \bdby^{A,f[\sigma] + l} (-l) +_c \norm{M}[\Theta]$, suppose $\disc l M[\theta] \downarrow^{(n,r-l)} v$. 
 By inversion we also have that $M[\theta] \downarrow^{(n,r)} v$. 
 It suffices to show
 \begin{itemize}
   \item $n \leq -l + \norm{M}[\Theta]_c - (r - l)$
   \item $v \valbd^{A,f[\sigma] + l + r - l} \norm{M}[\Theta]_p$
 \end{itemize}
 or, canceling, it suffices to show $n \leq \norm{M}[\Theta]_c  - r$ and $v \valbd^{A,f[\sigma] + r} \norm{M}[\Theta]_p$, which is precisely what we get from the IH.

\item[(\waitname)] Let $\Gamma \vdash_f \wait l M : A$. By inversion, $\Gamma \vdash_{f+l} M : A$. To show $\wait l M \bdby^{A} l +_c \norm{M}$, suppose $\theta \subbd^{\Gamma,\sigma} \Theta$, 
to show $\wait l M[\theta] \bdby^{A,f[\sigma]} l +_c \norm{M}[\Theta]$, 
suppose $\wait l M[\theta] \downarrow^{(n,r+l)} v$. 
By inversion, $M[\theta] \downarrow^{(n,r)} v$.
It suffices to show
\begin{itemize}
  \item $n \leq l + \norm{M}[\theta]_c - (r + l)$
  \item $v \valbd^{f[\sigma] + r + l}$
\end{itemize}
By IH, we have that $M[\theta] \bdby^{A,f[\sigma] + l} \norm{M}[\Theta]$, so
\begin{itemize}
  \item $n \leq \norm{M}[\Theta]_c - r$
  \item $v \valbd^{A,f[\sigma] + l + r}$
\end{itemize}
and so we are done, canceling the $l$s in the first inequality.


\item[(tick)] Immediate from IH, canceling $1$s.

\item[($\loli$-I)] Let $\Gamma \vdash_f \lambda x.M : A \loli B$. By inversion, $\Gamma, x : A \vdash_{f + x} M : B$. 
Let $\theta \subbd^{\Gamma,\sigma} \Theta$. 
To show $\lambda x.M[\theta] \bdby^{A\loli B,f[\sigma]} (0,\lambda x.\norm{M}[\Theta])$, let $\lambda x.M[\theta] \downarrow^{(0,0)} \lambda x.M[\theta]$.
The first condition is trivial ($0 \leq 0$). We need to show that $\lambda x.M \valbd^{f[\sigma]} \lambda x.\norm{M}[\Theta]$. Let $v \valbd^{A,d} E$. We must show that $M[\theta,v/x] \bdby^{B,f[\sigma] + d} (\lambda x.\norm{M}[\Theta]) E$, or by weakening, that $M[\theta,v/x] \bdby^{B,f[\sigma]+d} \norm{M}[\Theta,E/x]$. But, since $v \valbd^{A,d} E$, we have $(\theta,v/x) \subbd^{(\Gamma,x:A),(\sigma, x \mapsto d)} (\Theta,E/x)$, and so by IH, $M[\theta,v/x] \bdby^{B,f[\sigma] + d} \norm{M}[\Theta,E/x]$, as required.

\item[($\loli$-E)] Let $\Gamma \vdash_{f + g} M \, N : B$.  Inversion gives $\Gamma \vdash_f M : A \loli B$ and $\Gamma \vdash_g N : A$. Let $\theta \subbd^{\Gamma,\sigma} \Theta$. We must show $M[\theta] \, N[\theta] \bdby^{B,f[\sigma] + g[\sigma]} (\norm{M}[\Theta]_c + \norm{N}[\Theta]_c) +_c \norm{M}[\Theta]_p\norm{N}[\Theta]_p$. Suppose $M[\theta] \, N[\theta] \downarrow^{(n_1+n_2+n_3,r_1+r_2+r_3)} v$. Inversion gives us that $M[\theta] \downarrow^{(n_1,r_1)} \lambda x.M'$, $N[\theta] \downarrow^{(n_2,r_2)} v_1$, and $M'[v_1/x] \downarrow^{(n_3,r_3)} v$. It remains to show that 
\begin{itemize}
  \item $n_1 + n_2 + n_3 \leq \norm{M}[\Theta]_c + \norm{N}[\Theta]_c + (\norm{M}[\Theta]_p\norm{N}[\Theta]_p)_c - (r_1 + r_2 + r_3)$
  \item $v \valbd^{B,f[\sigma] + g[\sigma] + r_1 + r_2 + r_3} (\norm{M}[\Theta]_p\norm{N}[\Theta]_p)_p$
\end{itemize}
 By the IH applied to $\Gamma \vdash_f M : A \loli B$, we know that $M[\theta] \bdby^{A\loli B,f[\sigma]} \norm{M}[\Theta]$, so
 \begin{itemize}
  \item $n_1 \leq \norm{M}[\Theta]_c - r_1$
  \item $\lambda x.M' \valbd^{A \loli B,f[\sigma] + r_1} \norm{M}[\Theta]_p$.
 \end{itemize}  
 Again applying the IH to $\Gamma \vdash_g N : A$, we know $N[\theta] \bdby^{A,g[\sigma]} \norm{N}[\Theta]$, so
\begin{itemize}
  \item $n_2 \leq \norm{N}[\Theta]_c - r_2$
  \item $v_1 \valbd^{A,g_[\sigma] + r_2} \norm{N}[\Theta]_p$
\end{itemize}
  
But since $\lambda x.M' \valbd^{A \loli B,f[\sigma] + r_1} \norm{M}[\Theta]_p$ and $v_1 \valbd^{A,g[\sigma] + r_2} \norm{N}[\Theta]_p$, we have $M'[v_1/x] \bdby^{B,f[\sigma] + g[\sigma] + r_1 + r_2} \norm{M}[\Theta]_p\norm{N}[\Theta]_p$, which means that:
\begin{itemize}
  \item $n_3 \leq (\norm{M}[\Theta]_p\norm{N}[\Theta]_p)_c - r_3$
  \item $v \valbd^{B,f[\sigma] + g[\sigma] + r_1 + r_2 + r_3} (\norm{M}[\Theta]_p\norm{N}[\Theta]_p)_p$
\end{itemize}
We add the inequalities together, and are done.

\item[($\tensor$-I)] Let $\Gamma \vdash_{f_1 + g_1} (M_1,M_2) : A_1 \tensor A_2$. By inversion, we have that $\Gamma \vdash_{f_i} M_i : A_i$ for $i = 1,2$. Let $\theta \subbd^{\Gamma,\sigma} \Theta$. Towards proving $(M_1[\theta],M_2[\theta]) \bdby^{A_1\tensor A_2,f_1[\sigma] + f_2[\sigma]} (\norm{M_1}[\Theta]_c + \norm{M_2}[\Theta]_c,(\norm{M_1}[\Theta]_p,\norm{M_2}[\Theta]_p))$, assume $(M_1[\theta], M_2[\theta]) \downarrow^{(n_1+n_2,r_1+r_2)} (v_1,v_2)$. By inversion, it must also be that $M_i[\theta] \downarrow^{(n_i,r_i)} v_i$ for $i = 1,2$.
If suffices to show:
\begin{itemize}
  \item $n_1 + n_2 \leq \norm{M_1}[\Theta]_c + \norm{M_2}[\Theta] - (r_1 + r_2)$
  \item $(v_1,v_2) \valbd^{A_1 \tensor A_2,f_1[\sigma] + f_2[\sigma] + r_1 + r_2} (\norm{M_1}[\Theta]_p,\norm{M_2}[\Theta]_p)$
\end{itemize}
By IH, we have that, for $i \in \{1,2\}$
\begin{itemize}
  \item $n_i \leq \norm{M_i}[\Theta]_c - r_i$
  \item $v_i \valbd^{A_i,f[\sigma]_i + r_i} \norm{M_i}[\Theta]_p$
\end{itemize}
Adding the two inequalities and applying the definition of value bounding at $\otimes$, we are done.

\item[($\tensor$-E)]
Let $\Gamma \vdash_{k'f + g} \asplit {k'} M x y N : C$. Inversion gives $\Gamma \vdash_f M : A \tensor B$, and $\Gamma,x : A, y : B \vdash_{g + k'(x + y)} N : C$. Let $\theta \subbd^{\Gamma,\sigma} \Theta$. We must show that
$$
\asplit {k'} {M[\theta]} x y {N[\theta]} \bdby^{C,k'f[\sigma] + g[\sigma]} k'\norm{M}[\Theta]_c +_c \norm{N}[\Theta,\pi_1\norm{M}[\Theta]_p/x,\pi_2\norm{M}[\Theta]_p/y]
$$
Suppose that $\asplit {k'} {M[\theta]} x y {N[\theta]} \downarrow^{(n_1 + n_2,k'r_1 + r_2)} v$ by way of $M[\theta] \downarrow^{(n_1,r_1)} (v_1,v_2)$ and $N[\theta,v_1/x,v_2/y] \downarrow^{(n_2,r_2)} v$. It remains to show that
\begin{itemize}
  \item $n_1 + n_2 \leq k'\norm{M}[\Theta]_c + \norm{N}[\Theta,\pi_1\norm{M}[\Theta]_p/x,\pi_2\norm{M}[\Theta]_p/y]_c - (k'r_1 + r_2)$
  \item $v \valbd^{C,k'f[\sigma] + g[\sigma] + k'r_1 + r_2} \norm{N}[\Theta,\pi_1\norm{M}[\Theta]_p/x,\pi_2\norm{M}[\Theta]_p/y]_p$
\end{itemize}
By IH, $M[\theta] \bdby^{A \otimes B,f[\sigma]} \norm{M}[\Theta]$, so
\begin{itemize}
  \item $n_1 \leq \norm{M}[\Theta]_c - r_1$
  \item $(v_1,v_2) \valbd^{A \otimes B,f[\sigma] + r_1} \norm{M}[\Theta]_p$
\end{itemize}
and so there are $c_1,c_2 \geq 0$ so that $c_1 + c_2 = f[\sigma] + r_1$ and $v_1 \valbd^{A,c_1} \pi_1\norm{M}[\Theta]_p$ and $v_2 \valbd^{B,c_2} \pi_2\norm{M}[\Theta]_p$. So, $(\theta,v_1/x,v_2/y) \subbd^{(\Gamma,x:A,y:B),(\sigma,x\mapsto c_1,y\mapsto c_2)} (\Theta,\pi_1\norm{M}[\Theta]_p/x,\pi_2\norm{M}[\Theta]_p/y)$. So, by IH,
$N[\theta,v_1/x,v_2/y] \bdby^{C,g[\sigma] + k'(f[\sigma] + r_1)} \norm{N}[\Theta,\pi_1\norm{M}[\Theta]_p/x,\pi_2\norm{M}[\Theta]_p/y]$, so
\begin{itemize}
  \item $n_2  \leq \norm{N}[\Theta,\pi_1\norm{M}[\Theta]_p/x,\pi_2\norm{M}[\Theta]_p/y]_c - r_2$
  \item $v \valbd^{C,k'f[\sigma] + g[\sigma] + k'r_1 + r_2} \norm{N}[\Theta,\pi_1\norm{M}[\Theta]_p/x,\pi_2\norm{M}[\Theta]_p/y]_p$
\end{itemize}
Then,
\begin{align*}
 n_1 + n_2 &\leq k'n_1 + n_2\\
 &\leq k'\norm{M}[\Theta]_c + \norm{N}[\Theta,\pi_1\norm{M}[\Theta]_p/x,\pi_2\norm{M}[\Theta]_p/y]_c - (k'r_1 + r_2)
\end{align*}
as required.

\item[($\oplus$-E)] Let $\Gamma \vdash_{k'f + g_1 + g_2} \acase {k'} M x {N_1} y {N_2} : C$. By inversion, $\Gamma \vdash_f M : A \oplus B$, $\Gamma,x:A \vdash_{g_1 + k'x} N_1 : C$, and $\Gamma, y : B \vdash_{g_2 + k'y} N_2 : C$. Let $\theta \subbd^{\Gamma,\sigma} \Theta$. We must show that
$$
\acase {k'} {M[\theta]} x {N_1[\theta]} y {N_2[\theta]} \bdby^{C,k'f[\sigma] + g_1[\sigma] + g_2[\sigma]} k'\norm{M}[\Theta]_c +_c \ccase {\norm{M}[\Theta]_p} x {\norm{N_1}} y {\norm{N_2}}
$$
\sloppypar Because the two cases are symmetric, we consider only the following evaluation: $\acase {k'} {M[\theta]} x {N_1[\theta]} y {N_2[\theta]} \downarrow^{(n_1 + n_2,k'r_1 + r_2)} v$ by way of $M[\theta] \downarrow^{(n_1,r_1)} \inl {v_1}$ and $N_1[\theta,v_1/x] \downarrow^{(n_2,r_2)}v$. We must show that
\begin{itemize}
  \item $n_1 + n_2 \leq k'\norm{M}[\Theta]_c + \ccase {\norm{M}[\Theta]_p} x {\norm{N_1}} y {\norm{N_2}}_c - (k'r_1 + r_2)$
  \item $v \valbd^{C,k'f[\sigma]+g_1[\sigma] + g_2[\sigma] + k'r_1 + r_2} \ccase {\norm{M}[\Theta]_p} x {\norm{N_1}} y {\norm{N_2}}_p$
\end{itemize}
By IH, $M[\theta] \bdby^{A \oplus B,f[\sigma]} \norm{M}[\Theta]$, so
\begin{itemize}
  \item $n_1 \leq \norm{M}[\Theta]_c - r_1$
  \item $\inl {v_1} \valbd^{A \oplus B,f[\sigma] + r_1}\norm{M}[\Theta]_p$
\end{itemize}
so there is an $E$ such that $\inl E \leq_{\angles A + \angles B} \norm{M}[\Theta]_p$ and $v_1 \valbd^{A,f[\sigma] + r_1} E$. So, $(\theta,v_1/x) \subbd^{(\Gamma,x:A),(\sigma,x\mapsto f[\sigma]+r_1)} (\Theta,E/x)$ and hence by IH, $N_1[\theta,v_1/x] \bdby^{g_1[\sigma]  + k'(f[\sigma] + r_!)} \norm{N_1}[\Theta,E/x]$, so
\begin{itemize}
  \item $n_2 \leq \norm{N_1}[\Theta,E/x]_c - r_2$
  \item $v \valbd^{C,k'f[\sigma] + g_1[\sigma] + k'r_1 + r_2} \norm{N_1}[\Theta,E/x]_p$
\end{itemize}
Since $g_2[\sigma] \geq 0$, we have by credit weakening that $v \valbd^{C,k'f[\sigma] + g_1[\sigma] + g_2[\sigma] + k'r_1 + r_2} \norm{N_1}[\Theta,E/x]_p$.
Then, we compute:
\begin{align*}
\norm{N_1}[\Theta,E/x] &\leq_{\norm C} \textsc{\ccase {\inl E} x {\norm{N_1}[\Theta]} y {\norm{N_2}[\Theta]}}\\
&\leq \ccase {\norm{M}[\Theta]_p} x {\norm{N_1}[\Theta]} y {\norm{N_2}[\Theta]}
\end{align*}
which gives us the value bounding condition, and again compute:
\begin{align*}
n_1 + n_2 &\leq k'n_1 + n_2\\
&\leq k'\norm{M}[\Theta]_c + \ccase {\norm{M}[\Theta]_p} x {\norm{N_1}} y {\norm{N_2}}_c - (k'r_1 + r_2)
\end{align*}
which gives us the cost bounding condition.

\item[($\oplus$-I)] The cases for $\inl M$ and $\inr M$ are symmetric, so we let $\Gamma \vdash_f \inl M : A \oplus B$. Inversion gives $\Gamma \vdash_f M : A$. Let $\theta \subbd^{\Gamma,\sigma} \Theta$. 
To show that $\inl {M[\theta]} \bdby^{A \oplus B,f[\sigma]} (\norm{M}[\Theta]_c,\inl{\norm{M}[\Theta]_p})$,
we let $\inl{M[\theta]} \downarrow^{(n,r)} \inl v$. 
Inverting, we have $M[\theta] \downarrow^{(n,r)} v$. 
It suffices to show that $n \leq \norm{M}[\Theta]_c - r$, and that $\inl v \valbd^{A \oplus B,f[\sigma] + r} \inl {\norm{M}[\Theta]_p}$. By IH we have $n \leq \norm{M}[\Theta]_c - r$, and $v \valbd^{A,f[\sigma] + r} \norm{M}[\Theta]_p$. So we are done by the definition of value bounding at $\oplus$ for $\texttt{inl}$.

\item[($\listty A$-I, cons)] Let $\Gamma \vdash_{f+g} \cons M N : \listty A$. 
By inversion, $\Gamma \vdash_f M : A$ and $\Gamma \vdash_g N : \listty A$.
Let $\theta \subbd^{\Gamma,\sigma} \Theta$. 
To show that $\cons M N \bdby^{\listty A, f[\sigma] + g[\sigma]} (\norm{M}[\Theta]_c + \norm{N}[\Theta]_c, \cons {\norm{M}[\Theta]_p} {\norm{N}[\Theta]_p})$, 
let $\cons M N \downarrow^{(n_1 + n_2,r_1 + r_2)} \cons {v_1} {v_2}$. 
By inversion, $M \downarrow^{(n_1,r_1)} v_1$ and $N \downarrow^{(n_2,r_2)} v_2$. It suffices to provide $b,c$ where $c \geq 0$ and $b + c = f[\sigma] + g[\sigma]$ and that
\begin{itemize}
  \item $n_1 + n_2 \leq\norm{M}[\Theta]_c + \norm{N}[\Theta]_c - (r_1 + r_2)$
  \item $\cons {v_1} {v_2} \valbd^{\listty A, f[\sigma] + g[\sigma] + r_1 + r_2} \cons {\norm{M}[\Theta]_p} {\norm{N}[\Theta]_p}$.
\end{itemize}
By IH,
\begin{itemize}
  \item $n_1 \leq \norm{M}[\Theta]_c - r_1$
  \item $v_1 \valbd^{A,f[\sigma] + r_1} \norm{M}[\Theta]_p$
\end{itemize}
and by IH on $N$, there are $b_2,c_2$ with $c_2 \geq 0$ and $b_2 + c_2 = g[\sigma]$ such that
\begin{itemize}
  \item $n_2 \leq \norm{N}[\Theta]_c - r_2$
  \item $v_2 \valbd^{\listty A,g[\sigma] + r_2} \norm{N}[\Theta]_p$
\end{itemize}
Thus, the desired inequality follows from adding the two inductively computed ones, and the value bounding relation for $\cons {v_1} {v_2}$ is immediate by the definition.



\item[($\listty A$-E)] Suppose $\Gamma \vdash_{f+g_1+g_2} \lrec M {N_1} {N_2} : C$. By inversion, $\Gamma \vdash_f M : \listty A$, $\Gamma \vdash_{g_1} N_1 : 1 \loli C$, $\Gamma \vdash_{g_2} N_2 : !^\infty_0 (A \otimes (\listty A \amp C) \loli C)$. Let $\theta \subbd^{\Gamma,\sigma} \Theta$. To show that
\begin{align*}
\lrec {M[\theta]} {N_1[\theta]} {N_2[\theta]} \bdby^{C,f[\sigma] + g_1[\sigma] + g_2[\sigma]} &(\norm{M}[\Theta]_c + \norm{N_1}[\Theta]_c + \norm{N_2}[\Theta]_c) +_c\\ 
&\lrec {\norm{M}[\Theta]_p} {\norm{N_1}[\Theta]_p} {\lambda (a,(as,r)). \norm{N_2}[\Theta]_p\;  (a,((0,as),r))}
\end{align*}
we break into the two evaluation cases. Firstly, suppose that $\lrec {M[\theta]} {N_1[\theta]} {N_2[\theta]} \downarrow^{(n_1+n_2+n_3+n_4,r_1+r_2+r_3+r_4)} v$ by way of $M[\theta] \downarrow^{(n_1,r_1)} \elist$, $N_1[\theta] \downarrow^{(n_2,r_2)} \lambda x.N_1'$, $N_2[\theta] \downarrow^{(n_3,r_3)} \save \infty 0 {(\lambda x.N_2')}$, and $N_1'[()/x] \downarrow^{(n_4,r_4)} v$. From here, denote we denote $\lambda x. \norm{N_2}[\Theta]_p (\pi_1 x,((0,\pi_1 \pi_2 x),\pi_2\pi_2 x))$ as $\norm{N_2}^*$.

 It suffices to show that:
\begin{itemize}
  \item $n_1 + n_2 + n_3 + n_4 \leq \norm{M}[\Theta]_c + \norm{N_1}[\Theta]_c + \norm{N_2}[\Theta]_c + \lrec {\norm{M}[\Theta]_p} {\norm{N_1}[\Theta]_p} {\norm{N_2}^*}_c - (r_1 + r_2 + r_3 + r_4)$
  \item $v \valbd^{C,f[\sigma] + g_1[\sigma] + g_2[\sigma] + r_1 + r_2 + r_3 + r_4} \lrec {\norm{M}[\Theta]_p} {\norm{N_1}[\Theta]_p} {\norm{N_2}^*}_p$
\end{itemize}
By IH, $M[\theta] \bdby^{\listty A,f[\sigma]} \norm{M}[\Theta]$, so
\begin{itemize}
  \item $n_1 \leq \norm{M}[\Theta]_c - r_1$
  \item $\elist \valbd^{f[\sigma] + r_1} \norm{M}[\Theta]_p$
\end{itemize}
The second condition tells us that $\elist \leq \norm{M}[\Theta]_p$.
Again by IH, $N_2[\theta] \bdby^{!^\infty_0(A \otimes (\listty A \amp C) \loli C),g_2[\sigma]} \norm{N_2}[\Theta]$, so
\begin{itemize}
  \item $n_3 \leq \norm{N_2}[\Theta]_p - r_3$
  \item $\save \infty 0 {(\lambda x.N_2')} \valbd^{!^\infty_0(A \otimes (\listty A \amp C) \loli C),g_2[\sigma] + r_3} \norm{N_2}[\Theta]_p$
\end{itemize}
In particular, by preservation, $g_2[\sigma] + r_3 \geq 0$.
Thirdly by IH, $N_1[\theta] \bdby^{1 \loli C,g_1[\sigma]} \norm{N_1}[\Theta]$, which means
\begin{itemize}
  \item $n_2 \leq \norm{N_1}[\Theta]_c - r_2$
  \item $\lambda x.N_1' \valbd^{1 \loli C,g_1[\sigma] + r_2} \norm{N_1}[\Theta]_p$
\end{itemize}
Since $() \leq ()$, $() \valbd^{1,f[\sigma] + r_1} ()$. Hence, $N_2'[()/x] \bdby^{C,g_1[\sigma] + f[\sigma] + r_1 + r_2} \norm{N_1}[\Theta]_p \; ()$. This means that
\begin{itemize}
  \item $n_4 \leq (\norm{N_1}[\Theta]_p \; ())_c - r_4$
  \item $v \valbd^{C,f[\sigma] + g_1[\sigma] + r_1 + r_2 + r_4} (\norm{N_1}[\Theta]_p \; ())_p$
\end{itemize}
But by credit weakening, since $g_2[\sigma] + r_3 \geq 0$, we have 
$v \valbd^{C,f[\sigma] + g_1[\sigma] + g_2[\sigma] + r_1 + r_2 + r_3 + r_4} (\norm{N_1}[\Theta]_p \; ())_p$. But, we can compute:
$$
\norm{N_1}[\Theta]_p \; () \leq \lrec {\elist} {\norm{N_1}[\Theta]_p} {\norm{N_2}[\Theta]_p} \leq \lrec {\norm{M}[\Theta]_p} {\norm{N_1}[\Theta]_p} {\norm{N_2}^*}
$$
and we are done by weakening.\\


Otherwise, assume $M[\theta] \downarrow^{(n_1,r_1)} \cons {v_1} {v_2}$, $N_2[\theta] \downarrow^{(n_2,r_2)} \save \infty 0 {(\lambda x.N_2')}$, $N_1[\theta] \downarrow^{(n_3,r_3)} \lambda x.N_1'$, and $$N_2'[(v_1,\amppair {v_2} {\lrec {v_2} {\lambda x.N_1'} {\save \infty 0 {(\lambda x.N_2')}}})] \downarrow^{(n_4,r_4)} v$$.
Just like the previous case, it suffices to show
\begin{itemize}
  \item $n_1 + n_2 + n_3 + n_4 \leq \norm{M}[\Theta]_c + \norm{N_1}[\Theta]_c + \norm{N_2}[\Theta]_c + \lrec {\norm{M}[\Theta]_p} {\norm{N_1}[\Theta]_p} {\norm{N_2}^*}_c - (r_1 + r_2 + r_3 + r_4)$
  \item $v \valbd^{C,f[\sigma] + g_1[\sigma] + g_2[\sigma] + r_1 + r_2 + r_3 + r_4} \lrec {\norm{M}[\Theta]_p} {\norm{N_1}[\Theta]_p} {\norm{N_2}[\Theta]_p}_p$
\end{itemize}
By IH, $M[\theta] \bdby^{\listty A,f[\sigma]} \norm{M}[\Theta]$, so
\begin{itemize}
  \item $n_1 \leq \norm{M}[\Theta]_c - r_1$
  \item $\cons {v_1} {v_2} \valbd^{\listty A,f[\sigma] + r_1} \norm{M}[\Theta]_p$
\end{itemize}
By the second condition, we know that there are $d_1,d_2 \geq 0$ with $d_1 + d_2 = f[\sigma] + r_1$, and $E_1,E_2$ with $\cons {E_1} {E_2} \leq \norm{M}[\Theta]_p$ such that $v_1 \valbd^{A,d_1} E_1$, and $v_2 \valbd^{\listty A,d_2} E_2$.
By IH, $N_1[\theta] \bdby^{1 \loli C,g_1[\sigma]} \norm{N_1}[\Theta]$, so
\begin{itemize}
  \item $n_3 \leq \norm{N_1}[\Theta]_c - r_3$
  \item $\lambda x.N_1' \valbd^{1 \loli C,g_1[\sigma] + r_3} \norm{N_1}[\Theta]_p$
\end{itemize}
Again by IH, $N_2[\theta] \bdby^{!^\infty_0(A \otimes (\listty A \amp C) \loli C),g_2[\sigma]} \norm{N_2}[\Theta]$, which means that
\begin{itemize}
  \item $n_2 \leq \norm{N_2}[\theta]_c - r_2$
  \item $\save \infty 0 {(\lambda x.N_2')} \valbd^{!^\infty 0 (A \otimes (\listty A \amp C) \loli C),g_2[\sigma] + r_2} \norm{N_2}[\Theta]_p$
\end{itemize}
The second condition means by definition that there is a $c \geq 0$ such that $\infty \cdot c \leq g_2[\sigma] + r_2$, and $\lambda x.N_2' \valbd^{A \otimes (\listty A \amp C) \loli C,c} \norm{N_2}[\Theta]_p$. We claim that 

$$
\begin{array}{l}
N_2'[(v_1,\amppair {v_2} {\lrec {v_2} {\lambda x.N_2'} {\save \infty 0 {(\lambda x.N_2')}}})] \bdby^{f[\sigma] + g_1[\sigma] + g_2[\sigma] + r_1 + r_2 + r_3}\\ \norm{N_2} \; (E_1,((0,E_2),\lrec {E_2} {\norm{N_1}[\Theta]_p} {\norm{N_2}^*}))
\end{array}
$$

To prove this claim, we split into cases on the finitude of $g_2[\sigma] + r_2$.
\begin{itemize}
  \item Suppose $g_2[\sigma] + r_2$ is finite. Then $c = 0$, and $g_2[\sigma] + r_2 = 0$, and so by the list recursor lemma with $E_1 = \norm{N_1}[\Theta]_p$, $E_2 = \norm{N_2}[\Theta]_p$, and $E = E_2$, we have that $\lrec {v_2} {\lambda x.N_1'} {\save \infty 0 {(\lambda x.N_2')}} \bdby^{C,d_2 + g_1[\sigma] + r_3} \lrec {E_2} {\norm{N_1}[\Theta]_p} {\norm{N_2}^*}$. Since $v_2 \valbd^{\listty A,d_2} E_2$, we have that $v_2 \bdby^{\listty A,d_2} (0,E_2)$, and by credit weakening, since $g_1[\sigma] + r_3 \geq 0$, $v_2 \bdby^{\listty A,d_2 + g_1[\sigma] + r_3} (0,E_2)$. Thus:
   $$
   \begin{array}{l}
   \amppair {v_2} {\lrec {v_2} {\lambda x.N_1'} {\save \infty 0 {(\lambda x.N_2')}}} \bdby^{\listty A \amp C,d_2 + g_1[\sigma] + r_3}\\ ((0,E_2),\lrec {E_2} {\norm{N_1}[\Theta]_p} {\norm{N_2}^*})   
   \end{array}
   $$
    Next, since $v_1 \valbd^{A,d_1} E_1$, $d_1 + d_2 = f[\sigma] + r_1$, and $\lambda x.N_2' \valbd^{A \tensor (\listty A \amp C),0} \norm{N_2}[\Theta]_p$,
  $$
   \begin{array}{l}
    N_2'[(v_1,\amppair {v_2} {\lrec {v_2} {\lambda x.N_1'} {\save \infty 0 {(\lambda x.N_2')}}})/x] \bdby^{f[\sigma] + g_1[\sigma] + r_1 + r_3} \\\norm{N_2}[\Theta]_p \; (E_1,((0,E_2),\lrec {E_2} {\norm{N_1}[\Theta]_p} {\norm{N_2}^*}))     
   \end{array}
  $$
  Which is exactly what we wanted to show, since $g_2[\sigma] + r_2 = 0$.
  \item Suppose $g_2[\sigma] + r_2 = \infty$. Then, there is a $c \geq 0$ such that $\infty \cdot c \leq \infty$, and $\lambda x.N_2' \bdby^{A \otimes (\listty A \amp C) \loli C,c} \norm{N_2}[\Theta]_p$ By credit weakening, we may assume $c > 0$. By the list recursor lemma, $\lrec {v_2} {\lambda x.N_1'} {\save \infty 0 {(\lambda x.N_2')}} \bdby^{C,d_2 + g_1[\sigma] + r_3 + \infty \cdot c} \lrec {E_2} {\norm{N_1}[\Theta]_p} {\norm{N_2}^*}$. By the same reasoning as in the previous case, 
  $$
  \begin{array}{l}
  (v_1,\amppair {v_2} {\lrec {v_2} {\lambda x.N_1'} {\save \infty 0 {(\lambda x.N_2')}}}) \bdby^{A \tensor (\listty A \amp C),f[\sigma] + g_1[\sigma] + r_1 + r_3}\\ (E_1,((0,E_2),\lrec {E_2} {\norm{N_1}[\Theta]_p} {\norm{N_2}^*}))  
  \end{array}
  $$ 
  Then, since $\infty \cdot c + c = \infty \cdot c$, 
  $$
  \begin{array}{l}
  N_2'[(v_1,\amppair {v_2} {\lrec {v_2} {\lambda x.N_1'} {\save \infty 0 {(\lambda x.N_2')}}})/x] \bdby^{C,f[\sigma] + g_1[\sigma] + r_1 + r_3 + \infty \cdot c}\\ \norm{N_2}[\Theta]_p \; (E_1,((0,E_2),\lrec {E_2} {\norm{N_1}[\Theta]_p} {\norm{N_2}^*}))  
  \end{array}
  $$
  which, $\infty \cdot c \leq g_2[\sigma] + r_2$, gives us our goal by credit weakening.
\end{itemize}

From this result, we have by definition that
\begin{itemize}
  \item $n_4 \leq (\norm{N_2}[\Theta]_p \; (E_1,((0,E_2),\lrec {E_2} {\norm{N_1}[\Theta]_p} {\norm{N_2}^*})))_c - r_4$
  \item $v \valbd^{C,f[\sigma] + g_1[\sigma] + g_2[\sigma] + r_1 + r_2 + r_3 + r_4} (\norm{N_2}[\Theta]_p \; (E_1,((0,E_2),\lrec {E_2} {\norm{N_1}[\Theta]_p} {\norm{N_2}^*})))_p$
\end{itemize}
Then we can compute:
\begin{align*}
\norm{N_2}[\Theta]_p \; (E_1,((0,E_2),\lrec {E_2} {\norm{N_1}[\Theta]_p} {\norm{N_2}^*})) &\leq \norm{N_2}^* \; (E_1,(E_2,\lrec {E_2} {\norm{N_1}[\Theta]_p} {\norm{N_2}^*}))\\
&\leq \lrec {\cons {E_1} {E_2}} {\norm{N_1}[\Theta]_p} {\norm{N_2}^*}\\
&\leq \lrec {\norm{M}[\Theta]_p} {\norm{N_1}[\Theta]_p} {\norm{N_2}^*}
\end{align*}

and so we are done by weakening.

\item[($\N$-E)] 

Suppose $\Gamma \vdash_{f+g_1+g_2} \nrec M {N_1} {N_2} : C$. 
By inversion, we have that $\Gamma \vdash_f M : \N$, $\Gamma \vdash_{g_1} N_1 : 1 \loli C$, and $\Gamma \vdash_{g_2} N_2 : !^\infty_0 (\N \otimes (1 \loli C) \loli C)$. 
Let $\theta \subbd^{\Gamma,\sigma} \Theta$. For convenience, let $\norm{N_2}[\Theta]_p^* = \lambda p.\norm{N_2}[\Theta]_p(\pi_1 p,\lambda z. \pi_2 p)$.
We must show: 
$$
\begin{array}{l}
\nrec {M[\theta]} {N_1[\theta]} {N_2[\theta]} \bdby^{C,f[\sigma] + g_1[\sigma] + g_2[\sigma]}\\
(\norm{M}[\Theta]_c + \norm{N_1}[\Theta]_c + \norm{N_2}[\Theta]_c) +_c \nrec {\norm{M}[\Theta]_p} {\norm{N_1}[\Theta]_p} {\norm{N_2} [\Theta]_p^*}
\end{array}
$$
In order to show this, we have two evaluation cases to consider.
Suppose 
$$\nrec {M[\theta]} {N_1[\theta]} {N_2[\theta]} \downarrow^{(n_1+n_2+n_3+n_4,r_1+r_2+r_3+r_4)} v$$
by way of $M[\theta] \downarrow^{(n_1,r_1)} 0$, $N_1[\theta] \downarrow^{(n_2,r_2)} \lambda x.N_1'$, $N_2[\theta] \downarrow^{(n_3,r_3)} v'$, and $N_1'[()/x] \downarrow^{(n_4,r_4)} v$. 
It suffices to show that:
\begin{itemize}
  \item $n_1+n_2+n_3+n_4 \leq b + \norm{M}[\Theta]_c + \norm{N_1}[\Theta]_c + \norm{N_2}[\Theta]_c + \nrec {\norm{M}[\Theta]_p} {\norm{N_1}[\Theta]_p} {\norm{N_2} [\Theta]_p^*}_c - (r_1 + r_2 + r_3 + r_4)$
  \item $v \valbd^{C,f[\sigma] + g_1[\sigma] + g_2[\sigma] + r_1 + r_2 + r_3 + r_4} \nrec {\norm{M}[\Theta]_p} {\norm{N_1}[\Theta]_p} {\norm{N_2} [\Theta]_p^*}_p$
\end{itemize}

By IH, $M[\theta] \bdby^{\N,f[\sigma]} \norm{M}[\Theta]$, so
\begin{itemize}
  \item $n_1 \leq \norm{M}[\Theta]_c - r_1$
  \item $0 \valbd^{\N,f[\sigma] + r_1} \norm{M}[\Theta]_p$
\end{itemize}
since $0 \valbd^{\N,f[\sigma] + r_1} \norm{M}[\Theta]_p$, $0 \leq_\N \norm{M}[\Theta]_p$. 
By IH, $N_1 [\theta] \bdby^{1 \loli C,g_1[\sigma]} \norm{N_1}[\Theta]$, so
\begin{itemize}
  \item $n_2 \leq \norm{N_1}[\Theta]_c - r_2$
  \item $\lambda x.N_1' \valbd^{1 \loli C,g_1[\sigma] + r_2} \norm{N_1}[\Theta]_p$
\end{itemize}
By IH, $N_2[\theta] \bdby^{!^\infty_0(\cdots),g_2[\sigma]} \norm{N_2}[\Theta]$, and so
\begin{itemize}
  \item $n_3 \leq \norm{N_2}[\Theta]_c - r_3$
\end{itemize}
We omit the value bounding condition since it does not factor into the rest of the proof. Since $() \leq_1 ()$, $() \valbd^{1,f[\sigma] + r_1} ()$. So: $N_1'[()/x] \bdby^{C,f[\sigma] + g_1[\sigma] + r_1 + r_2} \norm{N_1}[\Theta] \; ()$. Thus,
\begin{itemize}
  \item $n_4 \leq (\norm{N_1}[\Theta] \; ())_c - r_4$
  \item $v \valbd^{C,f[\sigma] + g_1[\sigma] + r_1 + r_2 + r_4} (\norm{N_1}[\Theta] \; ())_p$
\end{itemize}
Since $g_2[\sigma] + r_3 \geq 0$, we know by credit weakening, $v \valbd^{C,f[\sigma] + g_1[\sigma] + g_2[\sigma] + r_1 + r_2 + r_3 + r_4} (\norm{N_1}[\Theta] \; ())_p$.
Since $0 \leq \norm{M}[\Theta]_p$, we compute: 
\begin{align*}
\norm{N_1}[\Theta] \; () &\leq_C \nrec 0 {\norm{N_1}[\Theta]_p} {\norm{N_2}[\Theta]_p^*}\\
&\leq \nrec {\norm{M}[\Theta]_p} {\norm{N_1}[\Theta]_p} {\norm{N_2}[\Theta]_p^*}
\end{align*}

So we are done by weakening.\\

% *** 2nd REC EVAL CASE ***
Suppose $\nrec {M[\theta]} {N_1[\theta]} {N_2[\theta]} \downarrow^{(n_1+n_2+n_3+n_4,r_1+r_2+r_3+r_4)} v$ by way of 
$M[\theta] \downarrow^{(n_1,r_1)} S(v_1)$, 
$N_2[\theta] \downarrow^{(n_2,r_2)} \save \infty 0 {\lambda x.N_2'}$, 
$N_1[\theta] \downarrow^{(n_3,r_3)} \lambda x.N_1'$, and 
$$N_2'[(v_1,\nrec {v_1} {\lambda z. {(\nrec {v_1} {\lambda x.N_1'} {\save \infty 0 {\lambda x.N_2'}})}})/x] \downarrow^{(n_4,r_4)} v$$
\begin{itemize}
  \item $n_1 + n_2 + n_3 + n_4 \leq b + \norm{M}[\Theta]_c + \norm{N_1}[\Theta]_c + \norm{N_2}[\Theta] + \nrec {\norm{M}[\Theta]_p} {\norm{N_1}[\Theta]_p} {\norm{N_2}[\Theta]_p^*}_c - (r_1+r_2+r_3+r_4)$
  \item $v \valbd^{C,f[\sigma] + g_1[\sigma] + g_2[\sigma] + r_1+r_2+r_3+r_4} \nrec {\norm{M}[\Theta]_p} {\norm{N_1}[\Theta]_p} {\norm{N_2}[\Theta]_p^*}_p$
\end{itemize}

%%c_1 = f[\sigma] + r_1
By IH, $M[\theta] \bdby^{\N,f[\sigma]} \norm{M}[\Theta]$, so
\begin{itemize}
  \item $n_1 \leq \norm{M}[\Theta]_c - r_1$
  \item $S(v_1) \valbd^{\N,f[\sigma] + r_1} \norm{M}[\Theta]_p$
\end{itemize}
Since $S(v_1) \valbd^{\N,f[\sigma] + r_1} \norm{M}[\Theta]_p$, there is an $E$ such that $v_1 \valbd^{\N,f[\sigma] + r_1} E$, and $S(E) \leq \norm{M}[\Theta]_p$.

%%c_3 = g_1[\sigma] + r_3
By IH, $N_1[\theta] \bdby^{1 \loli C,g_1[\sigma]} \norm{N_1}[\Theta]$, and so
\begin{itemize}
  \item $n_3 \leq \norm{N_1}[\Theta]_c - r_3$
  \item $\lambda x.N_1' \valbd^{1 \loli C,g_1[\sigma] + r_3} \norm{N_1[\Theta]}_p$
\end{itemize}

%% c_2 = g_2[\sigma] + r_2
By IH, $N_2[\theta] \bdby^{!^\infty_0(\N \otimes (1 \loli C) \loli C),g_2[\sigma]}\norm{N_2}[\Theta]$, so by definition,
\begin{itemize}
  \item $n_2 \leq \norm{N_2}[\Theta]_c - r_2$
  \item $\save \infty 0 {\lambda x.N_2'} \valbd^{!^\infty_0(\N\otimes (1 \loli C) \loli C),g_2[\sigma] + r_2}\norm{N_2}[\Theta]_p$
\end{itemize}
and so there is a $d \geq 0$ so that $\lambda x.N_2' \valbd^{\N \otimes (1 \loli C) \loli C,d} \norm{N_2}[\Theta]_p$, and $\infty \cdot d \leq g_2[\sigma] + r_3$. By the $\N$-recursor lemma, $\nrec {v_1} {\lambda x.N_1'} {\save \infty 0 {(\lambda x.N_2')}} \bdby^{C,g_1[\sigma] + r_2 + \infty \cdot d} \nrec {E} {\norm{N_1}[\Theta]_p} {\norm{N_2}[\Theta]_p^*}$.
Let $E^* = (E,\nrec {E} {\norm{N_1}[\Theta]_p} {\norm{N_2}[\Theta]_p^*})$. Note that $v_1 \valbd^{\N,f[\sigma] + r_1} \pi_1 E^*$ and $$\lambda z. \nrec {v_1} {\lambda x.N_1'} {\save \infty 0 {(\lambda x.N_2')}} \valbd^{1 \loli C,g_1[\sigma] + r_2 + \infty \cdot d} \lambda z. \pi_2 E^*$$, and so:
$$
(v_1,\lambda z. \nrec {v_1} {\lambda x.N_1'} {\save \infty 0 {(\lambda x.N_2')}}) \valbd^{\N \otimes (1 \loli C),f[\sigma] + g_1[\sigma] + \infty \cdot d + r_1 + r_2} (\pi_1  E^*,\lambda z.\pi_2 E^*)
$$.

Thus, because $\lambda z.N_2' \valbd^{\N \otimes (1 \loli C) \loli C,d} \norm{N_2}[\Theta]_p$, and $\infty \cdot d + d = \infty \cdot d$

$$
N_2'[(v_1,\lambda z. \nrec {v_1} {\lambda x.N_1'} {\save \infty 0 {(\lambda x.N_2')}})] \bdby^{C,f[\sigma] + g_1[\sigma] + \infty \cdot d + r_1 + r_2} \norm{N_2}[\Theta]_p(\pi_1  E^*,\lambda z.\pi_2 E^*)
$$
and so:
\begin{itemize}
  \item $n_4 \leq (\norm{N_2}[\Theta]_p \; (\pi_1  E^*,\lambda z.\pi_2 E^*))_c - r_4$
  \item $v \valbd^{C,f[\sigma] + g_1[\sigma] + \infty \cdot d + r_1 + r_2 + r_4}  (\norm{N_2}[\Theta]_p \; (\pi_1  E^*,\lambda z.\pi_2 E^*))_p$
\end{itemize}
but, $\infty \cdot d \leq g_2[\sigma] + r_3$, and
\begin{align*}
  \norm{N_2}[\Theta]_p \; (\pi_1  E^*,\lambda z.\pi_2 E^*) &\leq (\lambda p. \norm{N_2}[\Theta]_p (\pi_1 p,\lambda z.\pi_2 p))E^*\\
  &\leq \norm{N_2}[\Theta]_p^* \; (E,\nrec {E} {\norm{N_1}[\Theta]_p} {\norm{N_2}[\Theta]_p^*})\\
  &\leq \nrec {S(E)} {\norm{N_1}[\Theta]_p} {\norm{N_2}[\Theta]_p}\\
  &\leq \nrec {\norm{N_1}[\Theta]_p} {\norm{N_1}[\Theta]_p} {\norm{N_2}[\Theta]_p}
\end{align*}
and so we are done by weakening and credit weakening.


\item[($\amp$-I)] Suppose $\Gamma \vdash_f \amppair M N : A \amp B$, and let $\theta \subbd^{\Gamma,\sigma} \Theta$. We can invert to find that $\Gamma \vdash_f M : A$ and $\Gamma \vdash_f N : B$. Since $\amppair {M[\theta]} {N[\theta]} \downarrow^{(0,0)} \amppair  {M[\theta]} {N[\theta]} $, to show that $\amppair  {M[\theta]} {N[\theta]}  \bdby^{A \amp B,f[\sigma]} (0,(\norm{M}[\Theta],\norm{N}[\Theta]))$ we must show that $0\leq 0$ (done!) and that $\amppair {M[\theta]} {N[\theta]}  \valbd^{A \amp B,f[\sigma]} (\norm{M}[\Theta],\norm{N}[\Theta])$. For this, it suffices by weakening to show that $M[\theta] \bdby^{A,f[\sigma]} \norm{M}[\Theta]$ and $N[\theta] \bdby^{B,f[\sigma]} \norm{N}[\Theta]$, which are precisely the inductive hypotheses.

\item[($\amp$-E)] By symmetry, it suffices to present the $pi_1$ case. Suppose $\Gamma \vdash_f \pi_1 M : A$. By inversion, $\Gamma \vdash_f M : A \amp B$. Let $\theta \subbd^{\Gamma,\sigma} \Theta$. To show that $\pi_1 M[\theta] \bdby^{A,f[\sigma]} \norm{M}[\Theta]_c +_c \pi_1(\norm{M}[\Theta]_p)$, assume $\pi_1 M[\theta] \downarrow^{(n_1+n_2,r_1+r_2)} v$. By inversion, it was by way of $M[\theta] \downarrow^{(n_1,r_1)} \amppair {N_1} {N_2}$ and $N_1 \downarrow^{(n_2,r_2)} v$. We must show that:
\begin{itemize}
   \item $n_1 + n_2 \leq \norm{M}[\Theta]_c + (\pi_1\norm{M}[\Theta]_p)_c - (r_1 + r_2)$
   \item $v \valbd^{A,f[\sigma] + r_1 + r_2} (\pi_1\norm{M}[\Theta]_p)_p$
\end{itemize}
By IH, $M[\theta] \bdby^{A \amp B,f[\sigma]} \norm{M}[\Theta]$, so
\begin{itemize}
  \item $n_1 \leq \norm{M}[\Theta]_c - r_1$
  \item $\amppair {N_1} {N_2} \valbd^{A \amp B,f[\sigma] + r_1} \norm{M}[\Theta]_p$
\end{itemize}
where the second condition means, in particular, that $N_1 \bdby^{A,f[\sigma] + r_1} \pi_1 \norm{M}[\Theta]_p$. So, since $N_1 \downarrow^{(n_2,r_2)} v$,
\begin{itemize}
  \item $n_2 \leq (\pi_1 \norm{M}[\Theta]_p)_c - r_2$
  \item $v \valbd^{A,f[\sigma] + r_1 + r_2} (\pi_1 \norm{M}[\Theta]_p)_p$
\end{itemize}
as required.

\item[(var)] Suppose $\Gamma, x : A \vdash_{x+f} x : A$. Let $(\theta,v/x) \subbd^{(\Gamma,x:A),(\sigma,x\mapsto a)}(\Theta,E/x)$. We know that $v \valbd^{A,a} E$.
We must show that $v \bdby^{A,a+f[\sigma]} (0,E)$. We know that $v \downarrow^{(0,0)} v$. Of course, $0 \leq 0$. Since $f[\sigma] \geq 0$, we are done by credit weakening.

\end{itemize}
\end{proof}

\auxsemlemma*
\begin{proof}
Let $A,B,C,G$ be posets. Note that these are not required to be in the image of $\scott{\cdot}$. For each case we must show two statements: the function is in fact montonic, and that the functions in its image (an exponential poset) are themselves monotonic.
\begin{enumerate}
  \item Suppose $(f,g) \leq (f',g')$ as elements of ${\left(C^1\right)}^G\times {\left(C^{\N\times C}\right)}^G$. To show that $\texttt{snrec}(f,g) \leq \texttt{snrec}(f',g')$, it suffices to show that for all $\gamma,n$, that $\snrec(f,g)(\gamma,n) \leq \snrec(f',g')(\gamma,n)$. Proceed by induction on $n$.
  \begin{itemize}
    \item $n = 0$. By the definition of $\snrec$, it suffices to show that $f(\gamma)() \leq f'(\gamma)()$, which is true since $f \leq f'$.
    \item $n+1$ By definition of $\snrec$, it suffces to show $g(\gamma)(n,\snrec(f,g)(\gamma,n)) \vee f(\gamma)() \leq g'(\gamma)(n,\snrec(f',g')(\gamma,n)) \vee f(\gamma)()$. We have already shown that $f(\gamma)() \leq f'(\gamma)()$, so it remains to show $g(\gamma)(n,\snrec(f,g)(\gamma,n)) \leq g'(\gamma)(n,\snrec(f',g')(\gamma,n))$. Since $g \leq g'$, $g(\gamma) \leq g'(\gamma)$. By reflexivity, $n \leq n$. By IH, $\snrec(f,g)(\gamma,n) \leq \snrec(f',g')(\gamma)(n)$, and so we are done.
    
    Now, let $(f,g) \in {\left(C^1\right)}^G\times {\left(C^{\N\times C}\right)}^G$. We must show that if $(\gamma,n) \leq (\gamma',n')$ in $G \times \N$, then $\snrec(f,g)(\gamma,n) \leq \snrec(f,g)(\gamma',n')$. Proceed by induction on $n$. We have three cases to consider.
    \begin{itemize}
      \item $n = n' = 0$. By definition of $\snrec$, it suffices to show that $f(\gamma)() \leq f(\gamma')()$, which is true since $\gamma \leq \gamma'$.
      \item $n = 0$, $n' + 1$: By the definition of $\snrec$, we must show that $f(\gamma)() \leq g(\gamma')(n',\snrec(f,g)(\gamma',n')) \vee f(\gamma')()$, for which it suffices to show $f(\gamma)() \leq f(\gamma')()$, which we already argued was true.
      \item $n+1$, $n'+1$. Expanding definitions again and simplifying, it suffices to show that $g(\gamma)(n,\snrec(f,g)(\gamma,n)) \leq g(\gamma')(n',\snrec(f,g)(\gamma',n'))$. Since $g$ is monotonic, $g(\gamma) \leq g(\gamma')$. Since $n + 1 \leq n' + 1$, $n \leq n'$. By IH, $\snrec(f,g)(\gamma,n) \leq \snrec(f,g)(\gamma',n')$, and so $g(\gamma)(n,\snrec(f,g)(\gamma,n)) \leq g(\gamma')(n',\snrec(f,g)(\gamma',n'))$, as required.
    \end{itemize}
  \end{itemize}
  \item Let $(f,g) \leq (f',g) \in {\left(C^1\right)^G} \times {\left(C^{A \times (\N \times C)}\right)^G}$. We want to show that $\slrec(f,g) \leq \slrec(f',g')$. Fix $\gamma \in G$, we prove by induction on $n$ that for all $n \in \N$, $\slrec(f,g)(\gamma,n) \leq \slrec(f',g')(\gamma,n)$.
    \begin{itemize}
      \item $n = 0$: expanding the definition of $\slrec$, we must show that $f(\gamma)() \leq f'(\gamma)()$, which is true because $f \leq f'$.
      \item $n > 0$. It suffices to show that $g(\gamma)(\infty,(n,\slrec(f,g)(\gamma,n))) \leq g'(\gamma)(\infty,(n,\slrec(f',g')(\gamma,n)))$. Since $g \leq g'$, $g(\gamma) \leq g'(\gamma)$. Further, $\infty \leq \infty$, $n \leq n$, and by IH, $\slrec(f,g)(\gamma,n) \leq \slrec(f',g')(\gamma,n)$, as required.
    \end{itemize}
    
    Now, let $(f,g) \in {\left(C^1\right)^G} \times {\left(C^{A \times (\N \times C)}\right)^G}$. We must show that if $(\gamma,n) \leq (\gamma',n')$, $\slrec(f,g)(\gamma,n) \leq \slrec(f,g)(\gamma,n')$. We again prove this by induction on $n$. There are three cases we must consider.
    \begin{itemize}
      \item $n = n' = 0$. Immediate.
      \item $n = 0, n' > 0$. Identical to the similar case for \texttt{snrec}.
      \item $n,n' > 0$. To show that 
      $$g(\gamma)(\infty,(n,\snrec(f,g)(\gamma,n))) \vee f(\gamma)() \leq g(\gamma')(\infty,(n',\snrec(f,g)(\gamma',n))) \vee f(\gamma')()$$
      it suffices to show that $f(\gamma) \leq f(\gamma')$ (which is true because $\gamma \leq \gamma'$ and $f$ is monotonic) and $g(\gamma)(\infty,(n,\snrec(f,g)(\gamma,n))) \leq g(\gamma')(\infty,(n',\snrec(f,g)(\gamma',n')))$. Since $n + 1 \leq n'+1$, $n \leq n'$, and so the desired result follows from IH and the fact that $g(\gamma)$
    \end{itemize}
    
  \item Let $(f,g) \leq (f',g') \in C^{G \times A}\times C^{G \times B}$. We must show that $\scase(f,g) \leq \scase(f',g')$ in $C^{G \times (A + B)}$ Let $(\gamma,x) \in G \times (A+B)$. The two cases for $x$ are symmetrical, so we consider when $x = \inl a$. Then,
    \begin{align*}
      \scase(f,g)(\gamma,\inl a) &= f(\gamma,a) \vee g(\gamma,\infty)\\
                                 &\leq f'(\gamma,a) \vee g(\gamma,\infty)\\
                                 &= \scase(f',g')(\gamma,\inl a)
    \end{align*}
    as required.
    
    Now, fix $(f,g) \in C^{G \times A}\times C^{G \times B}$. We must show that for all $(\gamma,x) \leq (\gamma',y)$, $\scase(f,g)(\gamma,x) \leq \scase(f,g)(\gamma',y)$. We have two symmetric cases to consider, so we present the case where $x = \inl a$ and $y = \inl a'$. Then,
    \begin{align*}
      \scase(f,g)(\gamma,\inl a) &= f(\gamma,a) \vee g(\gamma,\infty)\\
                                 &\leq f(\gamma',a') \vee g(\gamma',\infty)\\
                                 &= \scase(f,g)(\gamma',\inl a')
    \end{align*}
    as required.
\end{enumerate}
\end{proof}
\semsubst*
\begin{proof}
By induction on $\Gamma, x : T_1 \vdash E : T_2$.
\begin{itemize}
  \item (\texttt{nrec}): Suppose $\Gamma, x : T_1 \vdash \nrec E {E_1} {E_2} : T_2$. By inversion, $\Gamma x : T_1 \vdash E : \N $, $\Gamma, x : T_1 \vdash E_1 : 1 \to T_2$, and $\Gamma, x: T_1 \vdash E_2 : \N \times T_2 \to T_2$. By IH,
  \begin{itemize}
   \item  $\scott{\Gamma \vdash E[E'/x] : \N} = (1_{\scott{\Gamma}},\scott{\Gamma \vdash E' : T_1}) ; \scott{\Gamma, x : T_1 \vdash E : \N}$  
   \item $\scott{\Gamma \vdash E_1[E'/x] : 1 \to T_2} = (1_{\scott{\Gamma}},\scott{\Gamma \vdash E' : T_1}) ; \scott{\Gamma, x : T_1 \vdash E_1 : 1 \to T_2}$
   \item $\scott{\Gamma \vdash E_2[E'/x] : \N \times T_2 \to T_2} = (1_{\scott{\Gamma}},\scott{\Gamma \vdash E' : T_1}) ; \scott{\Gamma, x: T_1 \vdash E_2 : \N \times T_2 \to T_2}$
  \end{itemize}. For ease of notation, we let $f = \scott{\Gamma \vdash E' : T_1}$, $g = \scott{\Gamma, x : T_1 \vdash E : \N}$, $h_1 =  \scott{\Gamma, x : T_1 \vdash E_1 : 1 \to T_2}$, and $h_2 = \scott{\Gamma, x: T_1 \vdash E_2 : \N \times T_2 \to T_2}$.
  Then, we compute:
  \begin{align*}
  &\scott{\Gamma \vdash \left(\nrec E {E_1} {E_2}\right)[E'/x] : T_2} \\
  &= \scott{\Gamma \vdash \nrec {E[E'/x]} {E_1[E'/x]} {E_2[E'/x]}}\\
  &= (1_{\scott{\Gamma}}, \scott{\Gamma \vdash E[E'/x] : \N}) ; \snrec(\scott{\Gamma \vdash E_1[E'/x] : 1 \to T_2},\scott{\Gamma \vdash E_2[E'/x] : \N \times T_2 \to T_2})\\
  &= (1_{\scott{\Gamma}},(1_{\scott{\Gamma}},f) ; g) ; \snrec((1_{\scott{\Gamma}},f) ; h_1,(1_{\scott{\Gamma}},f) ; h_2)
  \end{align*}
  It remains to show that
  $$
(1_{\scott{\Gamma}},(1_{\scott{\Gamma}},f) ; g) ; \snrec((1_{\scott{\Gamma}},f) ; h_1,(1_{\scott{\Gamma}},f) ; h_2) = (1_{\scott{\Gamma}},f);(1_{\scott{\Gamma,x:T_1}},g);\snrec(h_1,h_2)  
  $$
  Let $\gamma \in \scott{\Gamma}$.
  Applying the left hand side to $\gamma$, we get
  $$
  \snrec((1_{\scott{\Gamma}},f);h_1,(1_{\scott{\Gamma}},f);h_1)(\gamma,g(\gamma,f(\gamma)))
  $$
  and on the right:
  $$
  \snrec(h_1,h_2)((\gamma,f(\gamma)),g(\gamma,f(\gamma)))  
  $$
  Letting $\gamma' = (\gamma,f(\gamma))$, we must show that $\snrec((1_{\scott{\Gamma}},f);h_1,(1_{\scott{\Gamma}},f);h_1)(\gamma,g(\gamma')) = \snrec(h_1,h_2)(\gamma',g(\gamma'))$. We proceed by induction on $n = g(\gamma')$.
  \begin{itemize}
    \item $n = 0$. 
    \begin{align*}
      \snrec((1_{\scott{\Gamma}},f);h_1,(1_{\scott{\Gamma}},f);h_1)(\gamma,0) &= ((1_{\scott{\Gamma}},f);h_1)(\gamma)()\\
      &= h_1(\gamma,f(\gamma))()\\
      &= h_1(\gamma')()\\
      \snrec(h_1,h_2)(\gamma',0) &= h_1(\gamma')()
    \end{align*}
    as required.
    \item $n+1$:
    \begin{align*}
    &\snrec((1_{\scott{\Gamma}},f);h_1,(1_{\scott{\Gamma}},f);h_1)(\gamma,n+1)\\
    &= ((1_{\scott{\Gamma}},f);h_1)(\gamma)(n,\snrec((1_{\scott{\Gamma}},f);h_1,(1_{\scott{\Gamma}},f);h_1)(\gamma,n)) \vee h_1(\gamma')()\\
    &= h_1(\gamma')(n,\snrec(h_1,h_2)(\gamma',n)) \vee h_1(\gamma')()\\
    &= \snrec(h_1,h_2)(\gamma',n+1)
    \end{align*}
  \end{itemize}
  
  \item (\texttt{lrec}): Suppose $\Gamma, x : T_1 \vdash \lrec {E'} {E_1} {E_2} : T_2$, and $\Gamma \vdash E : T_1$. By inversion, $\Gamma,x:T_1 \vdash E : \listty T$, $\Gamma, x : T_1 \vdash E_1 : 1 \to T_2$, and $\Gamma,x:T_1 \vdash E_2 : T \times (\listty T \times T_2) \to T_2$.
  By IH, we have that:
  \begin{itemize}
    \item $\scott{\Gamma \vdash E'[E/x] : \listty T} = (1_{\scott{\Gamma}},\scott{\Gamma \vdash E : T_1}) ; \scott{\Gamma,x:T_1 \vdash E : \listty T}$
    \item $\scott{\Gamma \vdash E_1[E/x] : 1 \to T_2} = (1_{\scott{\Gamma}},\scott{\Gamma \vdash E : T_1}) ; \scott{\Gamma, x : T_1 \vdash E_1 : 1 \to T_}$
    \item $\scott{\Gamma \vdash E_2[E/x] : T \times (\listty T \times T_2) \to T_2} = (1_{\scott{\Gamma}},\scott{\Gamma \vdash E : T_1}) ; \scott{\Gamma,x:T_1 \vdash E_2 : T \times (\listty T \times T_2) \to T_2}$
  \end{itemize}
  Let $f = \scott{\Gamma \vdash E : T_1}$, $g = \scott{\Gamma,x : T_1 \vdash E' : \listty T}$, $h_1 = \scott{\Gamma,x:T_1 \vdash E_1 : 1 \to T_2}$, and $h_2 = \scott{\Gamma,x:T_1 \vdash E : T \times (\listty T \times T_2) \to T_2}$
  
  
  We must show that 
  $$
   (1_{\scott{\Gamma}},f) ; (1_{\scott{\Gamma} \times \scott{T_1}},g) ; \slrec(h_1,h_2) = (1_{\scott{\Gamma}},(1_{\scott{\Gamma}},f);g) ; \slrec((1_{\scott{\gamma}},f) ; h_1,(1_{\scott{\gamma}},f) ; h_2)
  $$
  
  Let $\gamma \in \scott{\Gamma}$, and let $\gamma' = (\gamma,f(\gamma))$. We must then show that
  
  $$\slrec(h_1,h_2)(\gamma',g(\gamma')) = \slrec((1_{\scott{\gamma}},f) ; h_1,(1_{\scott{\gamma}},f) ; h_2)(\gamma,g(\gamma'))$$
  
  We proceed by induction on $n = g(\gamma')$.
  
  \begin{itemize}
    \item ($n = 0$): The LHS is $\slrec(h_1,h_2)(\gamma',0) = h_1(\gamma')()$, and the RHS is 
    $$\slrec((1_{\scott{\gamma}},f) ; h_1,(1_{\scott{\gamma}},f) ; h_2)(\gamma,0) = h_1(\gamma,f(\gamma))() = h_1(\gamma')()$$.
    \item ($n > 0$): The LHS is:
    \begin{align*}
      &\slrec(h_1,h_2)(\gamma',n+1)\\
      &= h_2(\gamma')(\infty,(n,\slrec(h_1,h_2)(\gamma',n))) \vee h_1(\gamma')()
    \end{align*}
    and the RHS (applying the IH in the 2nd step) is
    \begin{align*}
     &\slrec((1_{\scott{\Gamma}},f) ; h_1,(1_{\scott{\Gamma}},f) ; h_2)(\gamma,n+1)\\
     &= h_2(\gamma')(\infty,(n,\slrec((1_{\scott{\Gamma}},f) ; h_1,(1_{\scott{\Gamma}},f) ; h_2)(\gamma,n))) \vee  h_1(\gamma')()\\
     &= h_2(\gamma')(\gamma,(n,\slrec(h_1,h_2)(\gamma',n))) \vee h_1(\gamma')()
    \end{align*}
    as required.
  \end{itemize}
\end{itemize}
\end{proof}
\begin{proof}
By induction on $\Gamma \vdash E : T$.
\begin{itemize}
  \item ($\texttt{nrec}$): Let $\Gamma \vdash \nrec E {E_1} {E_2} : T$. By inversion, $\Gamma \vdash E : \N$, $\Gamma \vdash E_1 : 1 \to T$, and $\Gamma \vdash E_2 : \N \times C \to C$. By IH, $\scott{\Gamma \vdash E : \N} \in \Hom(\scott{\Gamma},\N)$. Then, $(1_{\Gamma},\scott{\Gamma \vdash E : \N}) \in \Hom(\scott{\Gamma},\scott{\Gamma}\times \N)$. By IH, $\scott{\Gamma \vdash E_1 : 1 \to T} \in \Hom(\scott{\Gamma},\scott{T}^1)$ and $\scott{\Gamma \vdash E_2 : \N \times T \to T} \in \Hom(\scott{\Gamma},\scott{T}^{\N \times \scott{T}})$. Then, by Theorem~\ref{thm:aux-sem-lemma} and composition, $(1_{\Gamma},\scott{\Gamma \vdash E : \N}) ; \snrec(\scott{\Gamma \vdash E_1 : 1 \to T},\scott{\Gamma \vdash E_2 : \N \times T \to T}) \in \Hom(\scott{\Gamma},\scott{T})$, as required.
  
  \item ($\texttt{lrec}$): Let $\Gamma \vdash \lrec E {E_1} {E_2} : T$. By inversion, $\Gamma \vdash E : \listty {T'}$, $\Gamma \vdash E_2 : 1 \to T$, and $\Gamma \vdash E_2 : T' \times (\listty {T'} \times T) \to T$. Applying the IH to all of these premises, we have that $\scott{\Gamma \vdash E : \listty {T'}} \in \Hom(\scott{\Gamma},\N)$, $\scott{\Gamma \vdash E_2 : 1 \to T} \in \Hom(\scott{\Gamma},\scott{T}^1)$, and $\scott{\Gamma \vdash E_2 : T' \times (\listty {T'} \times T) \to T} \in \Hom(\scott{\Gamma},\scott{T}^{\scott{T'}\times(\N\times\scott{T})})$. By Theorem~\ref{thm:aux-sem-lemma} and composition, $(1_{\scott{\Gamma}},\scott{\Gamma \vdash E : \listty {T'}}) ; \slrec(\scott{\Gamma \vdash E_2 : 1 \to T},\scott{\Gamma \vdash E_2 : T' \times (\listty {T'} \times T) \to T}) \in \Hom(\scott{\Gamma},\scott{T})$ as required.
\end{itemize}
\end{proof}

\preordsound*
\begin{proof}
By induction on $\Gamma \vdash E \leq E'$. The new cases ($\snrec$ and $\slrec$) follow easily from the definitions.
\end{proof}
\substext*
\begin{proof}
By induction on $\Delta,\alpha \vdash c' \texttt{ credit}$ and $\Delta,\alpha|\Gamma\vdash_f M : A$, respectively.
\end{proof}
\presext*
\begin{proof}
The cases for all pre-existing rules are identical-- the only new cases are for \texttt{pack}, \texttt{unpack}, and \texttt{trec}. We present only the final case of \texttt{trec}, as it is the most illustritive.
\begin{itemize}
  \item (\texttt{pack}): Suppose that $\cdot | \cdot \vdash_a \pack \alpha \ell M : \exists \alpha$ and $\pack \alpha \ell M \downarrow^{(n,r)} \pack \alpha \ell v$ by way of $\cdot | \cdot \vdash_a M : A[\ell/\alpha]$ and $M \downarrow^{(n,r)} v$. By IH, $\cdot | \cdot \vdash_{a+r} v : A[\ell/\alpha]$ and $a + r \geq 0$. By the rule for $\texttt{pack}$, $\cdot | \cdot \vdash_{a + r} \pack \alpha \ell v : \exists \alpha. A$, as required.
  \item (\texttt{unpack}): Suppose that $\cdot | \cdot \vdash_{a + b} \unpack \alpha x M N : C$ by way of $\cdot | \cdot \vdash_a M : \exists \alpha. A$ and $\alpha | x : A \vdash_{b+x} N : C$ with $\Delta \vdash C$, and that $\unpack \alpha x M N \downarrow^{(n_1+n_2,r_1+r_2)} v$ by way of $M \downarrow^{(n_1,r_1)} \pack \alpha \ell {v_1}$ and $N[\ell/\alpha,v_1/x] \downarrow^{(n_2,r_2)} v$. By IH, $\cdot | \cdot \vdash_{a+r_1} v_1 : A[\ell/\alpha]$. By credit variable substituion, $\cdot | x : A[\ell/\alpha] \vdash_{b+x} N[\ell/\alpha] :C$. By substitution, $\cdot | \cdot \vdash_{b + a + r_1} N[\ell/\alpha,v_1/x] L C$ By IH, $\cdot | \cdot \vdash_{a+b+r_1+r_2} v : C$ and $a + b + r_1 + r_2 \geq 0$ as required.
  \item (\texttt{trec}): Suppose:
  $$
  \infer{\cdot | \cdot \vdash_{a + \sum b_i} \trec M {N_1} {N_2} {N_3} {N_4} {N_4}}
  {
   \begin{array}{l}
  \cdot | \cdot \vdash_f M : \tree A\\
  \cdot | \cdot \vdash_{b_1} N_1 : !^\infty_0(1 \loli C)\\
  \cdot | \cdot \vdash_{b_2} N_2 : !^\infty_0(A \loli C)\\
  \cdot | \cdot \vdash_{b_3} N_3 : !^\infty_0(A \otimes \N \otimes A \otimes N \otimes (\tree A \amp C)^2 \loli C)\\
  \cdot | \cdot \vdash_{b_4} N_4 : !^\infty_0(A \otimes \N \otimes A \otimes N \otimes (\tree A \amp C)^2 \loli C)\\
  \cdot | \cdot \vdash_{b_5} N_5 : !^\infty_0(A \otimes \N \otimes A \otimes N \otimes A \otimes N (\tree A \amp C)^4 \loli C)\\
  \end{array}
  }
  $$
  and
  $$
  \infer{\trec M {N_1} {N_2} {N_3} {N_4} {N_5} \downarrow^{(\sum n_i,\sum r_i)} v}
  {
  \begin{array}{l}
  M \downarrow^{(n_0,r_0)} N(v_1,n_1,N(v_2,n_2,t_{00},t_{01}),N(v_3,n_7,t_{10},t_{11}))\\
  N_i \downarrow^{(n_i,r_i)} \save \infty 0 {v_i'} \qquad (1 \leq i \leq 4)\\
  N_5 \downarrow^{(n_5,r_5)} \save \infty 0 {(\lambda x.N_5')}\\
  N_5'[(v_1,n_1,v_2,n_2,v_3,n_3,\amppair{t_{00},\texttt{trec}({t_{00}},{\save \infty 0 {v_1'}},{\dots},{\save \infty 0 {(\lambda x.N_5')}})},\dots)/x]
  \end{array}
  }  
  $$
  By IH, $\cdot | \cdot \vdash_{a+r_0} N(\dots)$. Hence, there are $d_1,\dots,d_n$, all non-negative, so that $\sum d_i = a + r_0$, and $\cdot | \cdot \vdash_{d_i} w_i: A_i$ where $w_i$ is the $i$th value in the value which $M$ evaluates to (in particular, $\cdot | \cdot \vdash_{d_1} v_1 : A$, and $\cdot | \cdot \vdash_{d_6} t_{00} : \tree A$).  Again by IH, there are $c_1,\dots,c_5$ so that $\infty c_i \leq b_i + r_i$, with $\cdot | \cdot \vdash_{c_i} v_i'$. Thus, $\cdot | \cdot \vdash_{d_6 + \sum c_i} \amppair{t_{00}} {\texttt{trec}(t_00,\save \infty 0 {v_1'},\dots)}$, and similarly for the rest of the subtrees. This immediately implies $$
  \cdot | \cdot \vdash_{\sum {d_i} + 4\sum c_i} (v_1,n_1,v_2,n_2,v_3,n_3,\amppair{t_{00}} {\texttt{trec}(t_00,\save \infty 0 {v_1'},\dots)},\dots) : (A\otimes \N)^3 \otimes (\tree A \amp C)^4  
  $$
  then by substitution
  $$
  \cdot | \cdot \vdash_{\sum d_i + c_5 + 4\sum c_i} N_5'[ (v_1,n_1,v_2,n_2,v_3,n_3,\amppair{t_{00}} {\texttt{trec}(t_00,\save \infty 0 {v_1'},\dots)},\dots)/x] : C
  $$
  The result follows immediately by weakening ($\infty c_i \leq b_i + r_i$) and IH.
\end{itemize}
\end{proof}
\extrsoundex*
\begin{proof}
By induction on $\Delta | \Gamma \vdash_f M : A$.
\end{proof}
\boundingex*
\begin{proof}
$\;$
\begin{itemize}
  \item (\texttt{pack}): Suppose $\Delta | \Gamma \vdash_f \pack \alpha c M : \exists \alpha. A$ by way of $\Delta | \Gamma \vdash_f M : A[c/\alpha]$. Let $\omega \bdby_{\texttt{credit}}^\Delta \Omega$ and $\theta \subbd^{\Gamma[\omega],\sigma} \Theta$. We must show that $\pack \alpha {c[\omega]} {M[\omega,\theta]} \bdby^{\exists \alpha.A[\omega],f[\omega,\sigma]} (\norm{M}_c[\Omega,\Theta],(c[\Omega],\norm{M}_p[\Omega,\Theta]))$. Suppose $\pack \alpha {c[\omega]} {M[\omega,\theta]} \downarrow^{(n,r)} \pack \alpha {c[\omega]} v$ by way of $M[\omega,\theta] \downarrow^{(n,r)} v$. It suffices to show
  \begin{itemize}
    \item $n + r \leq \norm{M}_c[\Omega,\Theta]$
    \item $\pack \alpha {c[\omega]} v \valbd^{\exists \alpha.A[\omega],f[\omega,\sigma] + r} (c[\Omega],\norm{M}_p[\Omega,\Theta])$
  \end{itemize}
  The second item is equivalent to proving that $c[\omega] \leq c[\Omega]$ (which is true because credit terms are monotone), and that $v \valbd^{A[c/\alpha,\omega],f[\omega,\sigma] + r} \norm{M}_p[\Omega,\Theta]$, which follows immediately by IH.
  
  \item (\texttt{unpack}): Suppose that $\Delta | \Gamma \vdash_{f+g} \unpack \alpha x M N : C$ by way of $\Delta | \Gamma \vdash_f M : \exists \alpha.A$ and $\Delta,\alpha | \Gamma,x:A \vdash_{g+x} N : C$ with $\alpha$ not free in $C$. Let $\omega \bdby_\texttt{credit}^\Delta \Omega$ and $\theta \subbd^{\Gamma[\omega],\sigma} \Theta$. Suppose $ \unpack \alpha x {M[\omega,\theta]} {N[\omega,\theta]} \downarrow^{(n_1+n_2,r_1+r_2)} v$ by way of $M[\omega,\theta] \downarrow^{(n_1,r_1)} \pack \alpha \ell {v_1}$ and $N[\omega,\theta,\ell/\alpha,v_1/x]  \downarrow^{(n_2,r_2)} v$. It suffices to show that
  \begin{itemize}
    \item $n_1 + n_2 + r_1 + r_2 \leq \norm{M}_c[\Omega,\Theta] + \norm{N}_c[\Omega,\Theta,\pi_1\norm{M}_p[\Omega,\Theta]\alpha,\pi_2\norm{M}_p[\Omega,\Theta]/x]$
    \item $v \valbd^{C[\omega],f[\omega,\sigma] + g[\omega,\sigma] + r_1 + r_2} \norm{N}_p[\Omega,\Theta,\pi_1\norm{M}_p[\Omega,\Theta]\alpha,\pi_2\norm{M}_p[\Omega,\Theta]/x]$
  \end{itemize}
  By IH, $M[\omega,\theta] \bdby^{\exists \alpha.A[\omega],f[\omega,\sigma]} \norm{M}[\Omega,\Theta]$, and so
  \begin{itemize}
    \item $n_1 + r_1 \leq \norm{M}_c[\Omega,\Theta]$
    \item $\pack \alpha \ell {v_1} \valbd^{\exists \alpha.A[omega],f[\omega,\sigma] + r_1}\norm{M}_p[\Omega,Theta]$
  \end{itemize}
  which means that $\ell \leq \pi_1 \norm{M}_p[\Omega,Theta]$ and that $v_1 \valbd^{A[\omega,\ell/\alpha],f[\omega,\sigma] + r_1} \pi_2 \norm{M}_p[\Omega,Theta]$.
  Hence, $(\omega,\ell/\alpha) \bdby_{\texttt{credit}}^{\Delta,\alpha}(\Omega,\pi_1 \norm{M}_p[\Omega,\Theta]/\alpha)$, and $(\theta,v_1/x) \subbd^{\Gamma[\omega],x:A[\ell/\alpha],\sigma,x\mapsto f[\omega,\sigma] + r_1} (\Theta,\pi_2 \norm{M}_p[\Omega,\Theta]/x)$. Thus, by IH, $$N[\omega,\theta,\ell/\alpha,v_1/x] \bdby^{C[\omega],g[\omega,\sigma] + f[\omega,\sigma] + r_1} \norm{N}[\Omega,\Theta,\pi_1 \norm{M}_p[\Omega,\Theta]/\alpha,\pi_2 \norm{M}_p[\Omega,\Theta]/x]$$ By definition,
  \begin{itemize}
    \item $n_2 + r_2 \leq \norm{N}_c[\Omega,\Theta,\pi_1 \norm{M}_p[\Omega,\Theta]/\alpha,\pi_2 \norm{M}_p[\Omega,\Theta]/x]$
    \item $v \valbd^{C[\omega],f[\omega,\sigma] + g[\omega,\sigma] + r_1 + r_2} \norm{N}_p[\Omega,\Theta,\pi_1 \norm{M}_p[\Omega,\Theta]/\alpha,\pi_2 \norm{M}_p[\Omega,\Theta]/x]$
  \end{itemize}
  as required.
\end{itemize}
\end{proof}

\begin{figure}
  
\begin{mathpar}
\text{$\lambda^A$ rules:} \qquad
\infer{\Delta | \Gamma \vdash_f \texttt{Emp} : \tree A}{}

\infer{\Delta | \Gamma \vdash_{f_1+f_2+g_1+g_2} N(M_1,M_2,N_1,N_2) : \tree A}
{
\Delta | \Gamma \vdash_{f_1} M_1 : A &
\Delta | \Gamma \vdash_{f_2} M_2 : \N &
\Delta | \Gamma \vdash_{g_1} N_1 : \tree A &
\Delta | \Gamma \vdash_{g_2} N_2 : \tree A
}\\

\infer{\Delta | \Gamma \vdash_{f + \sum_{i=1}^5 g_i} \trec M {N_1} {N_2} {N_3} {N_4} {N_5} : C}
{
\begin{array}{l}
\Delta | \Gamma \vdash_f M : \tree A\\
\Delta | \Gamma \vdash_{g_1} N_1 : !^\infty_0(1 \loli C)\\
\Delta | \Gamma \vdash_{g_2} N_2 : !^\infty_0(A \loli C)\\
\Delta | \Gamma \vdash_{g_3} N_3 : !^\infty_0(A \otimes \N \otimes A \otimes N \otimes (\tree A \amp C)^2 \loli C)\\
\Delta | \Gamma \vdash_{g_4} N_4 : !^\infty_0(A \otimes \N \otimes A \otimes N \otimes (\tree A \amp C)^2 \loli C)\\
\Delta | \Gamma \vdash_{g_5} N_5 : !^\infty_0(A \otimes \N \otimes A \otimes N \otimes A \otimes N (\tree A \amp C)^4 \loli C)\\
\end{array}
}

\\

\infer{\trec M {N_1} {N_2} {N_3} {N_4} {N_5} \downarrow^{\left(\sum_{i=1}^7 n_i,\sum_{i=1}^7 r_i\right)} v}{
\begin{array}{l}
M \downarrow^{(n_1,r_1)} N(v_1,n_1,N(v_2,n_2,t_{00},t_{01}),N(v_3,n_3,t_{10},t_{11}))\\
N_1 \downarrow^{(n_2,r_2)} v_1'\\
N_2 \downarrow^{(n_3,r_3)} v_2'\\
N_3 \downarrow^{(n_4,r_4)} v_3'\\
N_4 \downarrow^{(n_5,r_5)} v_4'\\
N_5 \downarrow^{(n_6,r_6)} \save \infty 0 {\lambda x.N_5'}\\
N_5'\left[\left(v_1,n_1,v_2,n_2,v_3,n_3,\amppair{t_{00}} {\trec {t_{00}} {v_1'} {v_2'} {v_3'} {v_4'} {\save \infty 0 {\lambda x.N_5'}}},\dots\right) /x \right] \downarrow^{(n_7,r_7)} v
\end{array}
}\\
\end{mathpar}

\hrule

\begin{mathpar}
\text{$\lambda^\bbbc$ rules:} \qquad
\infer{\Gamma \vdash \texttt{Emp} : \tree T}{}

\infer{\Gamma \vdash N(E_1,E_2,E'_1,E'_2) : \tree T}
{
\Gamma \vdash E_1 : T &
\Gamma \vdash E_2 : \N &
\Gamma \vdash E'_1 : \tree T &
\Gamma \vdash E'_2 : \tree T
}\\

\infer{
\Gamma \vdash \trec {E} {E_1} {E_2} {E_3} {E_4} {E_5} : T'
}
{
\begin{array}{l}
\Gamma \vdash E : \tree T\\
\Gamma \vdash E_1 : 1 \to T'\\
\Gamma \vdash E_2 : T \to T'\\
\Gamma \vdash E_3 : A \times \N \times A \times \N \times (\tree T \times T')^2 \to T'\\
\Gamma \vdash E_4 : A \times \N \times A \times \N \times (\tree T \times T')^2 \to T'\\
\Gamma \vdash E_5 : A \times \N \times A \times \N \times A \times \N \times (\tree T \times T')^4 \to T'\\
\end{array}
}\\

\begin{array}{l}
\norm{\trec M {N_1} {N_2} {N_3} {N_4} {N_5}} = \left(\norm{M}_c + \sum_{i=1}^5 \norm{N_i}_c \right) +_c\\
\hspace{1em} \texttt{trec}(\norm{M}_p,\norm{N_1}_p,\norm{N_2}_p,\lambda (x,n_1,y,n_2,(r_1,t_1),(r_2,t_2)). \norm{N_3}_p (x_1,n_1,y,n_2,((0,r_1),t_1),((0,r_2),t_2)),\dots )
\end{array}
%\norm{\trec M {N_1} {N_2} {N_3} {N_4} {N_5}} = \left(\norm{M}_c + \sum_{i=1}^5 \norm{N_i}_c \right) +_c \texttt{trec}(\norm{M}_p,\norm{N_1}_p,\norm{N_2}_p,\lambda (x,n_1,y,n_2,(r_1,t_1),(r_2,t_2)). \norm{N_3}_p (\_),\dots )

\end{mathpar}



  \caption{$\lambda^A$ and $\lambda^\bbbc$ \texttt{tree} extension, and recurrence extraction}
  \label{fig:trec-rules}
\end{figure}



\end{document}