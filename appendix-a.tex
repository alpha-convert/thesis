%\textbf{SOMETHING IS MESSED UP WITH APPENDIX THEOREM NUMBERING... FIX THIS}
\section{Rules of \dlambdaamor}

\subsection{Wellformedness Judgments}
\begin{mathpar}
\inferrule[WF-CCtxE]{ }{\Theta \vdash \cdot \; \texttt{wf}}

\inferrule[WF-CCtxNe]{\Theta \vdash \Delta \; \texttt{wf}\\ \Theta ; \Delta \vdash \Phi \; \texttt{wf}}{\Theta \vdash \Delta,\Phi \; \texttt{wf}}

\inferrule[WF-TCtxE]{ }{\Psi ; \Theta ; \Delta \vdash \cdot \; \texttt{wf}}

\inferrule[WF-TCtxNE]{\Psi ; \Theta ; \Delta \vdash \Gamma \; \texttt{wf}\\ \Psi ; \Theta ; \Delta \vdash \tau : \star}{\Psi ; \Theta ; \Delta \vdash \Gamma, x : \tau \; \texttt{wf}}
\end{mathpar}

\subsection{Sort Assignment}

\begin{mathpar}
\inferrule[I-Var]{i : S \in \Theta}{\Theta ; \Delta \vdash i : S}

\inferrule[I-Plus]{\Theta ; \Delta \vdash I : bS\\ \Theta ; \Delta \vdash J : bS}{\Theta ; \Delta \vdash I + J : bS}

\inferrule[I-Minus]{\Theta ; \Delta \vdash I : bS\\ \Theta ; \Delta \vdash J : bS\\ \Theta;\Delta \vDash I \geq J}{\Theta ; \Delta \vdash I - J : bS}\\

\inferrule[I-Times-$\mathbb{R}$]{c \in \mathbb{R}^+ \\ \Theta ; \Delta \vdash I : \mathbb{R}^+}{\Theta ; \Delta \vdash c \cdot I : \mathbb{R}^+}

\inferrule[I-Times-$\vec{\mathbb{R}}$]{c \in \mathbb{R}^+ \\ \Theta ; \Delta \vdash I : \vec{\mathbb{R}^+}}{\Theta ; \Delta \vdash c \cdot I : \vec{\mathbb{R}^+}}

\inferrule[I-Times-$\mathbb{N}$]{c \in \mathbb{N} \\ \Theta ; \Delta \vdash I : \mathbb{N}}{\Theta ; \Delta \vdash c \cdot I : \mathbb{N}}\\

\inferrule[I-Shift]{\Theta ; \Delta \vdash I : \vec{\mathbb{R}^+}}{\Theta ; \Delta \vdash \; \lhd I : \vec{\mathbb{R}^+}}

\inferrule[I-Lam]{\Theta, i : bS ; \Delta \vdash I : S}{\Theta ; \Delta \vdash \lambda i : bS. I : bS \to S}

\inferrule[I-App]{\Theta ; \Delta \vdash I : bS \to S\\ \Theta ; \Delta \vdash J : bS}{\Theta ; \Delta \vdash I \; J : S}\\

\inferrule[I-Sum]{\Theta;\Delta \vdash I_0 : \mathbb{N}\\ \Theta;\Delta \vdash I_1 : \mathbb{N}\\ 
                 \Theta,i : \N;\Delta, I_0 \leq i < I_1 \vdash J : bS}
                 {\Theta;\Delta \vdash \sum_{i=I_0}^{I_1} J : bS}
\\
                 
\inferrule[I-ConstVec]{\Theta ; \Delta \vdash I : \mathbb{R}^+}{\Theta ; \Delta \vdash \texttt{const}(I) : \vec{\mathbb{R}^+}}

\inferrule[I-Vec-Lit]{\cdot;\cdot \vDash \bigwedge_i c_i \geq 0}{\Theta ; \Delta \vdash (c_0,\dots,c_k) : \mathbb{R}^+}

\inferrule[I-Nat-Lit]{\cdot;\cdot \vDash n \geq 0}{\Theta ; \Delta \vdash n : \mathbb{N}}

\inferrule[I-PosReal-Lit]{\cdot;\cdot \vDash r \geq 0.0}{\Theta ; \Delta \vdash r : \mathbb{R}^+}
\end{mathpar}

\clearpage

\subsection{Constraint Wellformedness}
\begin{mathpar}
\inferrule[C-Top]{ }{\Theta;\Delta \vdash \top \texttt{ wf}}

\inferrule[C-Bot]{ }{\Theta;\Delta \vdash \bot \texttt{ wf}}

\inferrule[C-Conj]{\Theta ; \Delta \vdash \Phi_1 \texttt{ wf}\\ \Theta ; \Delta \vdash \Phi_2 \texttt{ wf}}{\Theta;\Delta \vdash \Phi_1 \wedge \Phi_2 \texttt{ wf}}

\inferrule[C-Disj]{\Theta ; \Delta \vdash \Phi_1 \texttt{ wf}\\ \Theta ; \Delta \vdash \Phi_2 \texttt{ wf}}{\Theta;\Delta \vdash \Phi_1 \vee \Phi_2 \texttt{ wf}}

\inferrule[C-Impl]{\Theta ; \Delta \vdash \Phi_1 \texttt{ wf}\\ \Theta ; \Delta, \Phi_1 \vdash \Phi_2 \texttt{ wf}}{\Theta;\Delta \vdash \Phi_1 \to \Phi_2 \texttt{ wf}}


\inferrule[C-Forall]{\Theta, i : S ; \Delta \vdash \Phi \texttt{ wf}}{\Theta ; \Delta \vdash \forall i : S. \Phi \texttt{ wf}}

\inferrule[C-Exists]{\Theta, i : S ; \Delta, \Phi \vdash \Phi \texttt{ wf}}{\Theta ; \Delta \vdash \exists i : S. \Phi \texttt{ wf}}

\inferrule[C-Leq]{\Theta ; \Delta \vdash I : bS\\ \Theta ; \Delta \vdash J : bS}{\Theta ; \Delta \vdash I \leq J \texttt{ wf}}

\inferrule[C-Lt]{\Theta ; \Delta \vdash I : bS\\ \Theta ; \Delta \vdash J : bS}{\Theta ; \Delta \vdash I < J \texttt{ wf}}

\inferrule[C-Eq]{\Theta ; \Delta \vdash I : bS\\ \Theta ; \Delta \vdash J : bS}{\Theta ; \Delta \vdash I = J \texttt{ wf}}
\end{mathpar}

\subsection{Kind Assignment}
\begin{mathpar}
\inferrule[K-Var]{\alpha : K \in \Psi}{\Psi ; \Theta ; \Delta \vdash \alpha : K}

\inferrule[K-Unit]{ }{\Psi ; \Theta ; \Delta \vdash 1 : \star}

\inferrule[K-Arr]{\Psi ; \Theta ; \Delta \vdash \tau_1 : \star\\ \Psi ; \Theta ; \Delta \vdash \tau_2 : \star}{\Psi ; \Theta ; \Delta \vdash \tau_1 \loli \tau_2 : \star}

\inferrule[K-Tensor]{\Psi ; \Theta ; \Delta \vdash \tau_1 : \star\\ \Psi ; \Theta ; \Delta \vdash \tau_2 : \star}{\Psi ; \Theta ; \Delta \vdash \tau_1 \otimes \tau_2 : \star}

\inferrule[K-With]{\Psi ; \Theta ; \Delta \vdash \tau_1 : \star\\ \Psi ; \Theta ; \Delta \vdash \tau_2 : \star}{\Psi ; \Theta ; \Delta \vdash \tau_1 \amp \tau_2 : \star}

\inferrule[K-Sum]{\Psi ; \Theta ; \Delta \vdash \tau_1 : \star\\ \Psi ; \Theta ; \Delta \vdash \tau_2 : \star}{\Psi ; \Theta ; \Delta \vdash \tau_1 \oplus \tau_2 : \star}

\inferrule[K-Bang]{\Psi ; \Theta ; \Delta \vdash \tau : \star}{\Psi ; \Theta ; \Delta \vdash !\tau : \star}

\inferrule[K-IForall]{\Psi ; \Theta, i : S ; \Delta \vdash \tau : \star}{\Psi ; \Theta ; \Delta \vdash \forall i : S. \tau : \star}

\inferrule[K-IExists]{\Psi ; \Theta, i : S ; \Delta \vdash \tau : \star}{\Psi ; \Theta ; \Delta \vdash \exists i : S. \tau : \star}

\inferrule[K-TForall]{\Psi, \alpha : K ; \Theta ; \Delta \vdash \tau : \star}{\Psi ; \Theta ; \Delta \vdash \forall \alpha : K. \tau : \star}

\end{mathpar}
\begin{mathpar}

\inferrule[K-List]{\Theta ; \Delta \vdash I : \N \\ \Psi ; \Theta ; \Delta \vdash \tau : \star}{\Psi ; \Theta ; \Delta \vdash L^I \tau : \star}

\inferrule[K-Conj]{\Theta ; \Delta \vdash \Phi \texttt{ wf}\\ \Psi ; \Theta ; \Delta \vdash \tau : \star}{\Psi ; \Theta ; \Delta \vdash \Phi \amp \tau : \star}

\inferrule[K-Impl]{\Theta ; \Delta \vdash \Phi \texttt{ wf}\\ \Psi ; \Theta ; \Delta, \Phi \vdash \tau : \star}{\Psi ; \Theta ; \Delta \vdash \Phi \implies \tau : \star}

\inferrule[K-Monad]{ \Theta ; \Delta \vdash I : \mathbb{N}\\ \Theta ; \Delta \vdash \vec{p} : \vec{\mathbb{R}^+}\\ \Psi ; \Theta ; \Delta \vdash \tau : \star}{\Psi ; \Theta ; \Delta \vdash \M(I,\vec{p}) \tau : \star}

\inferrule[K-Pot]{\Theta ; \Delta \vdash I : \mathbb{N} \\ \Theta ; \Delta \vdash \vec{p} : \vec{\mathbb{R}^+} \\ \Psi ; \Theta ; \Delta \vdash \tau : \star}{\Psi ; \Theta ; \Delta \vdash [I|\vec{p}] \tau : \star}

\inferrule[K-ConstPot]{\Theta ; \Delta \vdash I : \mathbb{R}^+\\ \Psi ; \Theta ; \Delta \vdash \tau : \star}{\Psi ; \Theta ; \Delta \vdash [I] \; \tau : \star}

\inferrule[K-FamLam]{\Psi ; \Theta, i : S ; \Delta \vdash \tau : K}{\Psi ; \Theta ; \Delta \vdash \lambda i : S. \tau : S \to K}

\inferrule[K-FamApp]{\Psi ; \Theta ; \Delta \vdash \tau : S \to K\\ \Theta ; \Delta \vdash I : S}{\Psi ; \Theta ; \Delta \vdash \tau \; I : K}
\end{mathpar}


\subsection{Subtyping}
%Presupposition: When $\Psi ; \Theta ; \Delta \vdash \tau_i : K$, we may judge $\Psi ; \Theta ; \Delta \vdash \tau_1 \subty \tau_2 : K$.

\begin{mathpar}
\inferrule[S-Refl]{ }{\Psi ; \Theta ; \Delta \vdash \tau \subty \tau : K}

\inferrule[S-Trans]{\Psi ; \Theta ; \Delta \vdash \tau_1 \subty \tau_2 : K\\ \Psi ; \Theta ; \Delta \vdash \tau_2 \subty \tau_3 : K}{\Psi ; \Theta ; \Delta \vdash \tau_1 \subty \tau_3 : K}

\inferrule[S-Arr]{\Psi ; \Theta ; \Delta \vdash \tau_1' \subty \tau_1 : \star \\ \Psi ; \Theta ; \Delta \vdash \tau_2 \subty \tau_2' : \star}{\Psi ; \Theta ; \Delta \vdash \tau_1 \loli \tau_2 \subty \tau_1' \loli \tau_2' : \star}

\inferrule[S-Tensor]{\Psi ; \Theta ; \Delta \vdash \tau_1 \subty \tau_1' : \star \\ \Psi ; \Theta ; \Delta \vdash \tau_2 \subty \tau_2' : \star }{\Psi ; \Theta ; \Delta \vdash \tau_1 \otimes \tau_2 \subty \tau_1' \otimes \tau_2' : \star}

\inferrule[S-With]{\Psi ; \Theta ; \Delta \vdash \tau_1 \subty \tau_1' : \star\\ \Psi ; \Theta ; \Delta \vdash \tau_2 \subty \tau_2' : \star}{\Psi ; \Theta ; \Delta \vdash \tau_1 \amp \tau_2 \subty \tau_1' \amp \tau_2' : \star}

\inferrule[S-Sum]{\Psi ; \Theta ; \Delta \vdash \tau_1 \subty \tau_1' : \star\\ \Psi ; \Theta ; \Delta \vdash \tau_2 \subty \tau_2' : \star}{\Psi ; \Theta ; \Delta \vdash \tau_1 \oplus \tau_2 \subty \tau_1' \oplus \tau_2' : \star}

\inferrule[S-Bang]{\Psi ; \Theta ; \Delta \vdash \tau_1 \subty \tau_2 : \star}{\Psi ; \Theta ; \Delta \vdash !\tau_1 \subty !\tau_2 : \star}


\inferrule[S-IForall]{\Psi ; \Theta, i : S ; \Delta \vdash \tau_1 \subty \tau_2 : \star}{\Psi ; \Theta ; \Delta \vdash \forall i : S. \tau_1 \subty \forall i : S. \tau_2 : \star}

\inferrule[S-IExists]{\Psi ; \Theta, i : S ; \Delta \vdash \tau_1 \subty \tau_2 : \star}{\Psi ; \Theta ; \Delta \vdash \exists i : S. \tau_1 \subty \exists i : S. \tau_2 : \star}

\inferrule[S-TForall]{\Psi, \alpha : K ; \Theta ; \Delta \vdash \tau_1 \subty \tau_2 : \star}{\Psi ; \Theta ; \Delta \vdash \forall \alpha : K. \tau_1 \subty \forall \alpha : K. \tau_2 : \star}

\end{mathpar}
\begin{mathpar}

\inferrule[S-List]{\Psi ; \Theta ; \Delta \vdash \tau_1 \subty \tau_2 : \star\\ \Theta ; \Delta \vDash I = J}{\Psi ; \Theta ; \Delta \vdash L^I \tau_1 \subty L^J \tau_2 : \star}

\inferrule[S-Impl]{\Psi ; \Theta ; \Delta \vdash \tau_1 \subty \tau_2 : \star\\ \Theta;\Delta \vDash \Phi_2 \to \Phi_1}{\Psi ; \Theta ; \Delta \vdash \Phi_1 \implies \tau_1 \subty \Phi_2 \implies \tau_2 : \star}

\inferrule[S-Conj]{\Psi ; \Theta ; \Delta \vdash \tau_1 \subty \tau_2 : \star \gens\\ \Theta;\Delta \vDash \Phi_1 \to \Phi_2}{\Psi ; \Theta ; \Delta \vdash \Phi_1 \amp \tau_1 \subty \Phi_2 \amp \tau_2 : \star}

\inferrule[S-Monad]{\Psi ; \Theta ; \Delta \vdash \tau_1 \subty \tau_2 : \star\\ \Theta ; \Delta \vDash I = J\\ \Theta ; \Delta \vDash \vec{q} \leq \vec{p}}{\Psi ; \Theta ; \Delta \vdash \M(I,\vec{q}) \tau_1 \subty \M(J,\vec{p}) \tau_2 : \star}

\inferrule[S-Pot]{\Psi ; \Theta ; \Delta \vdash \tau_1 \subty \tau_2 : \star\\ \Theta ; \Delta \vDash I = J\\ \Theta ; \Delta \vDash \vec{p} \leq \vec{q}}{\Psi ; \Theta ; \Delta \vdash [I|\vec{q}] \tau_1 \subty [J|\vec{p}] \tau_2 : \star}

\inferrule[S-ConstPot]{\Psi ; \Theta ; \Delta \vdash \tau_1 \subty \tau_2 : \star\\ \Theta;\Delta \vDash J \leq I}{\Psi ; \Theta ; \Delta \vdash [I] \tau_1 \subty [J] \tau_2 : \star}

\inferrule[S-FamLam]{\Psi ; \Theta, i : S ; \Delta \vdash \tau_1 \subty \tau_2 : K}{\Psi ; \Theta ; \Delta \vdash \lambda i : S. \tau_1 \subty \lambda i : S. \tau_2 : S \to K}

\inferrule[S-FamApp]{\Psi ; \Theta ; \Delta \vdash \tau_1 \subty \tau_2 : S \to K\\ \Theta ; \Delta \vDash I = J}{\Psi ; \Theta ; \Delta \vdash \tau_1 \; I \subty \tau_2 \; J : K}
\\
\inferrule[S-Fam-Beta1]{ }{\Psi ; \Theta ; \Delta \vdash (\lambda i : S. \tau) \; J \subty \tau[J/i] : K}

\inferrule[S-Fam-Beta2]{ }{\Psi ; \Theta ; \Delta \vdash\tau[J/i]\subty (\lambda i : S. \tau) \; J : K}

\end{mathpar}


\subsection{Context Subsumption}
%Presupposition: $\Theta \vdash \Delta \; \texttt{wf}$, $\Psi ; \Theta ; \Delta \vdash \Gamma \; \texttt{wf}$, and $\Psi ; \Theta ; \Delta \vdash \Omega \; \texttt{wf}$
\begin{mathpar}

\inferrule[CS-Emp]
{ }{\Psi ; \Theta ; \Delta \vdash \Gamma \wknto \cdot}

\inferrule[CS-Var]
{x : \tau' \in \Gamma \\ \Psi ; \Theta ; \Delta \vdash \tau' \subty \tau\\ \Psi ; \Theta ; \Delta \vdash \Gamma \setminus \{x : \tau'\} \wknto \Gamma'}{\Psi ;\Theta ; \Delta \vdash \Gamma \wknto \Gamma', x : \tau}
\end{mathpar}

\subsection{Type Assignment}

%Presupposition: when $\Psi ; \Theta ; \Delta \vdash \tau : \star$, we define $\Psi ; \Theta ; \Delta ; \Omega ; \Gamma\vdash e : \tau$

\begin{mathpar}
\inferrule[T-Var-1]
{x : \tau \in \Gamma}{\Psi ; \Theta ; \Delta ; \Omega ; \Gamma\vdash x : \tau}

\inferrule[T-Var-2]
{x : \tau \in \Omega}{\Psi ; \Theta ; \Delta ; \Omega ; \Gamma\vdash x : \tau}

\inferrule[T-Unit]
{ }{\Psi ; \Theta ; \Delta ; \Omega ; \Gamma\vdash () : 1}

\end{mathpar}
\begin{mathpar}

\inferrule[T-Base]
{ }{\Psi ; \Theta ; \Delta ; \Omega ; \Gamma\vdash c : b}

\inferrule[T-Absurd]
{\Theta ; \Delta \vDash \bot}{
\Psi ; \Theta ; \Delta ; \Omega ; \Gamma \vdash \texttt{absurd} : \tau
}


\inferrule[T-Nil]
{ }
{\Psi ; \Theta ; \Delta ; \Omega ; \Gamma\vdash \texttt{nil} : L^0 \tau}

\inferrule[T-Cons]
{\Psi ; \Theta ; \Delta ; \Omega ; \Gamma_1\vdash e_1 : \tau\\
\Psi ; \Theta ; \Omega ; \Gamma_2\vdash e_2 : L^{I} \tau
}
{\Psi ; \Theta ; \Delta ; \Omega ; \Gamma_1, \Gamma_2\vdash e_1 :: e_2 : L^{I + 1} \tau}

\inferrule[T-Match]
{
\Psi ; \Theta ; \Delta ; \Omega ; \Gamma_1\vdash e : L^I \tau\\
\Psi ; \Theta ; \Delta, I = 0 ; \Omega ; \Gamma_2\vdash e_1 : \tau'\\
\Psi ; \Theta ; \Delta, I \geq 1; \Omega ; \Gamma_2, h : \tau, t : L^I \tau \vdash e_2 : \tau'\\
}
{\Psi ; \Theta ; \Delta ; \Omega ; \Gamma_1,\Gamma_2\vdash \texttt{match}(e,e_1,h.t.e_2) : \tau'}


\inferrule[T-ExistI]
{
\Theta ; \Delta \vdash I : S\\
\Psi ; \Theta ; \Delta ; \Omega ; \Gamma\vdash e : \tau[I/i]\\
}{
\Psi ; \Theta ; \Delta ; \Omega ; \Gamma\vdash \texttt{pack}[I](e) : \exists i:S.\tau
}

\inferrule[T-ExistE]
{
\Psi ; \Theta ; \Delta ; \Omega ; \Gamma_1\vdash e : \exists i : S.\tau\\
\Psi ; \Theta, i : S ; \Delta ; \Omega ; \Gamma_2, x : \tau \vdash e' : \tau'\\
}{
\Psi ; \Theta ; \Delta ; \Omega ; \Gamma_1,\Gamma_2\vdash \texttt{unpack } (i,x) = e \texttt{ in } e' : \tau'
}

\inferrule[T-Lam]
{
\Psi ; \Theta ; \Delta ; \Omega ; \Gamma, x : \tau_1 \vdash e : \tau_2
}{
\Psi ; \Theta ; \Delta ; \Omega ; \Gamma\vdash \lambda x.e : \tau_1 \loli \tau_2
}

\inferrule[T-App]
{
\Psi ; \Theta ; \Delta ; \Omega ; \Gamma_1\vdash e_1 : \tau_1 \loli \tau_2\\
\Psi ; \Theta ; \Delta ; \Omega ; \Gamma_2\vdash e_2 : \tau_1
}{
\Psi ; \Theta ; \Delta ; \Omega ; \Gamma_1,\Gamma_2\vdash e_1 \, e_2 :  \tau_2
}

\inferrule[T-TensorI]
{
\Psi ; \Theta ; \Delta ; \Omega ; \Gamma_1\vdash e_1 : \tau_1\\
\Psi ; \Theta ; \Delta ; \Omega ; \Gamma_2\vdash e_2 : \tau_2\\
}{
\Psi ; \Theta ; \Delta ; \Omega ; \Gamma_1,\Gamma_2\vdash \angles{e_1,e_2} ; \tau_1 \otimes \tau_2
}

\inferrule[T-TensorE]
{
\Psi ; \Theta ; \Delta ; \Omega ; \Gamma_1\vdash e : \tau_1 \otimes \tau_2\\
\Psi ; \Theta ; \Delta ; \Omega ; \Gamma_2,x : \tau_1, y : \tau_2\vdash e' : \tau'
}{
\Psi ; \Theta ; \Delta ; \Omega ; \Gamma_1,\Gamma_2\vdash \texttt{let } \angles{x,y} = e \texttt{ in } e' : \tau'
}

\end{mathpar}

\begin{mathpar}

\inferrule[T-WithI]
{
\Psi ; \Theta ; \Delta ; \Omega ; \Gamma \vdash e_1 : \tau_1\\
\Psi ; \Theta ; \Delta ; \Omega ; \Gamma \vdash e_2 : \tau_2
}{
\Psi ; \Theta ; \Delta ; \Omega ; \Gamma \vdash (e_1,e_2) : \tau_1 \amp \tau_2
}

\inferrule[T-Fst]
{
\Psi ; \Theta ; \Delta ; \Omega ; \Gamma \vdash e : \tau_1 \amp \tau_2
}{
\Psi ; \Theta ; \Delta ; \Omega ; \Gamma \vdash \texttt{fst}(e) : \tau_1
}

\inferrule[T-Snd]
{
\Psi ; \Theta ; \Delta ; \Omega ; \Gamma \vdash e : \tau_1 \amp \tau_2
}{
\Psi ; \Theta ; \Delta ; \Omega ; \Gamma \vdash \texttt{snd}(e) : \tau_2
}

\inferrule[T-Inl]
{
\Psi ; \Theta ; \Delta ; \Omega ; \Gamma \vdash e : \tau_1
}{
\Psi ; \Theta ; \Delta ; \Omega ; \Gamma \vdash \texttt{inl}(e) : \tau_1 \oplus \tau_2
}

\inferrule[T-Inr]
{
\Psi ; \Theta ; \Delta ; \Omega ; \Gamma \vdash e : \tau_2
}{
\Psi ; \Theta ; \Delta ; \Omega ; \Gamma \vdash \texttt{inr}(e) : \tau_1 \oplus \tau_2
}

\inferrule[T-Case]
{
\Psi ; \Theta ; \Delta ; \Omega ; \Gamma_1 \vdash e : \tau_1 \oplus \tau_2\\
\Psi ; \Theta ; \Delta ; \Omega ; \Gamma_2, x: \tau_1 \vdash e_1 : \tau\\
\Psi ; \Theta ; \Delta ; \Omega ; \Gamma_2, y: \tau_2 \vdash e_2 : \tau\\
}{
\Psi ; \Theta ; \Delta ; \Omega ; \Gamma_1,\Gamma_2 \vdash \texttt{case}(e,x.e_1,y.e_2) : \tau
}

\inferrule[T-ExpI]
{
\Psi ; \Theta ; \Delta ; \Omega ; \cdot \vdash e : \tau
}{
\Psi ; \Theta ; \Delta ; \Omega ; \Gamma \vdash !e : !\tau
}

\inferrule[T-ExpE]{
\Psi ; \Theta ; \Delta ; \Omega ; \Gamma_1 \vdash e  : !\tau \\
\Psi ; \Theta ; \Delta ; \Omega, x : \tau ; \Gamma_2 \vdash e' : \tau'
}{
\Psi ; \Theta ; \Delta ; \Omega ; \Gamma_1,\Gamma_2 \vdash \texttt{let } !x = e \texttt{ in } e' : \tau'
}


\inferrule[T-TAbs]
{
\Psi, \alpha : K ; \Theta ; \Delta ; \Omega ; \Gamma \vdash e : \tau
}{
\Psi ; \Theta ; \Delta ; \Omega ; \Gamma \vdash \Lambda \alpha. e : \forall \alpha : K.\tau
}

\inferrule[T-TApp]
{
\Psi ; \Theta ; \Delta ; \Omega ; \Gamma \vdash e : \forall \alpha : K.\tau\\
\Psi ; \Theta ; \Delta \vdash \tau' : K
}{
\Psi ; \Theta ; \Delta ; \Omega ; \Gamma \vdash e [\tau'] : \tau[\tau'/\alpha]
}

\inferrule[T-IAbs]
{
\Psi ; \Theta, i : S ; \Delta ; \Omega ; \Gamma \vdash e : \tau
}{
\Psi ; \Theta ; \Delta ; \Omega ; \Gamma \vdash \Lambda i. e : \forall i : S. \tau
}

\inferrule[T-IApp]
{
\Theta ; \Delta \vdash I : S\\
\Psi ; \Theta ; \Delta ; \Omega ; \Gamma \vdash e : \forall i : S.\tau\\
}{
\Psi ; \Theta ; \Delta ; \Omega ; \Gamma \vdash e [I] : \tau[I/i]
}

\inferrule[T-Fix]
{
\Psi ; \Theta ; \Delta ; \Omega, x : \tau ; \cdot \vdash e : \tau
}{
\Psi ; \Theta ; \Delta ; \Omega ; \Gamma \vdash \texttt{fix } x.e : \tau
}


\inferrule[T-CImpI]
{
\Psi ; \Theta ; \Delta,\Phi' ; \Omega ; \Gamma \vdash e : \tau
}{
\Psi ; \Theta ; \Delta ; \Omega ; \Gamma \vdash \Lambda .e : (\Phi' \Rightarrow \tau)
}

\inferrule[T-CImpE]
{
\Psi ; \Theta ; \Delta ; \Omega ; \Gamma \vdash e : \Phi' \Rightarrow \tau\\
\Theta ; \Delta \vDash \Phi'
}{
\Psi ; \Theta ; \Delta ; \Omega ; \Gamma \vdash e \{\} : \tau
}

\inferrule[T-CAndI]
{
\Psi ; \Theta ; \Delta ; \Omega ; \Gamma \vdash e : \tau\\
\Theta ; \Delta \vDash \Phi'
}{
\Psi ; \Theta ; \Delta ; \Omega ; \Gamma \vdash <e> : \Phi' \amp \tau
}

\inferrule[T-CAndE]
{
\Psi ; \Theta ; \Delta ; \Omega ; \Gamma_1 \vdash e : \Phi' \amp \tau\\
\Psi ; \Theta ; \Delta, \Phi' ; \Omega ; \Gamma_2, x : \tau \vdash e' : \tau'\\
}{
\Psi ; \Theta ; \Delta ; \Omega ; \Gamma_1,\Gamma_2 \vdash \texttt{clet } x = e \texttt{ in } e' : \tau'
}
\end{mathpar}

\begin{mathpar}
\inferrule[T-Tick]
{
\Theta ; \Delta \vdash I : \mathbb{N}\\
\Theta ; \Delta \vdash \vec{p} : \vec{\mathbb{R}^+}
}{
\Psi ; \Theta ; \Delta ; \Omega ; \Gamma \vdash \texttt{tick}[I|\vec{p}] : \M \, (I,\vec{p})\, 1
}

\inferrule[T-Ret]
{
\Psi ; \Theta ; \Delta ; \Omega ; \Gamma \vdash e : \tau
}{
\Psi ; \Theta ; \Delta ; \Omega ; \Gamma \vdash \texttt{ret}\; e : \M \, (I,\vec{0}) \, \tau
}

\inferrule[T-Bind]
{
\Psi ; \Theta ; \Delta ; \Omega ; \Gamma_1 \vdash e_1 : \M \, (I,\vec{p})\, \tau_1\\
\Psi ; \Theta; \Delta ; \Omega ; \Gamma_2, x:\tau_1 \vdash e_2 : \M \, (I,\vec{q})\, \tau_2\\
}{
\Psi ; \Theta ; \Delta ; \Omega ; \Gamma_1,\Gamma_2 \vdash \texttt{bind } x = e_1 \texttt{ in } e_2 : \M \, (I,\vec{p} + \vec{q})\, \tau_2
}

\inferrule[T-Release]
{
\Psi ; \Theta ; \Delta ; \Omega ; \Gamma_1 \vdash e_1 : [I | \vec{q}] \tau_1\\
\Psi ; \Theta ; \Delta ; \Omega ; \Gamma_2, x : \tau \vdash e_2 : \M \, (I,\vec{p} + \vec{q}) \, \tau_2\\
}{
\Psi ; \Theta ; \Delta ; \Omega ; \Gamma_1,\Gamma_2 \vdash \texttt{release } x = e_1 \texttt{ in }e_2 : \M \, (I,\vec{p}) \, \tau_2
}

\inferrule[T-Store]
{
\Theta ; \Delta \vdash I : \mathbb{N}\\
\Theta ; \Delta \vdash \vec{p} : \vec{\mathbb{R}^+}\\
\Psi ; \Theta ; \Delta ; \Omega ; \Gamma \vdash e : \tau\\
}{
\Psi ; \Theta ; \Delta ; \Omega ; \Gamma \vdash \texttt{store}[I|\vec{p}](e) : \M \, (I,\vec{p}) \, ([I| \vec{p}] \, \tau)
}

\inferrule[T-StoreConst]
{
\Psi ; \Theta ; \Delta ; \Omega ; \Gamma \vdash e : \tau\\
\Theta ; \Delta \vdash I : \mathbb{N}\\
}{
\Psi ; \Theta ; \Delta ; \Omega ; \Gamma \vdash \texttt{store}[I](e) : \M \, (K,\texttt{const}(I)) \, ([I] \, \tau)
}

\inferrule[T-ReleaseConst]
{
\Psi ; \Theta ; \Delta ; \Omega ; \Gamma_1 \vdash e_1 : [J] \tau_1\\
\Psi ; \Theta ; \Delta ; \Omega ; \Gamma_2, x : \tau \vdash e_2 : \M \, (I,\vec{p} + \texttt{const}(J)) \, \tau_2
}{
\Psi ; \Theta ; \Delta ; \Omega ; \Gamma_1,\Gamma_2 \vdash \texttt{release } x = e_1 \texttt{ in }e_2 : \M \, (I,\vec{p}) \, \tau_2
}

\inferrule[T-Shift]
{
\Psi ; \Theta ; \Delta ; \Omega ; \Gamma \vdash e : \M \, (I - 1,\lhd \vec{p}) \, \tau\\
\Theta ; \Delta \vDash I \geq 1
}{
\Psi ; \Theta ; \Delta ; \Omega ; \Gamma \vdash \texttt{shift}(e) : \M \, (I,\vec{p}) \, \tau
}

\inferrule[T-Sub]
{
\Psi ; \Theta ; \Delta ; \Omega ; \Gamma \vdash e : \tau'\\
\Psi;\Theta;\Delta \vdash \tau' \subty \tau
}{
\Psi ; \Theta ; \Delta ; \Omega ; \Gamma \vdash e : \tau
}

\inferrule[T-Weaken]
{
\Psi ; \Theta ; \Delta ; \Omega ; \Gamma \vdash e : \tau\\
\Theta ; \Delta \vdash \Omega' \bdby \Omega\\
\Theta ; \Delta \vdash \Gamma' \bdby \Gamma\\
}{
\Psi ; \Theta ; \Delta ; \Omega' ; \Gamma' \vdash e : \tau
}

\end{mathpar}

\subsection{Normal Forms and Normalization}
\begin{mathpar}
\inferrule[N-Var]{ }{\alpha \; \texttt{ne}}

\inferrule[N-Unit]{ }{1 \; \texttt{nf}}

\inferrule[N-Arr]{\tau_1 \; \texttt{nf} \\ \tau_2 \; \texttt{nf}}{\tau_1 \loli \tau_2 \; \texttt{nf}}

\inferrule[N-Tensor]{\tau_1 \; \texttt{nf} \\ \tau_2 \; \texttt{nf}}{\tau_1 \otimes \tau_2 \; \texttt{nf}}

\inferrule[N-With]{\tau_1 \; \texttt{nf} \\ \tau_2 \; \texttt{nf}}{\tau_1 \amp \tau_2 \; \texttt{nf}}

\inferrule[N-Sum]{\tau_1 \; \texttt{nf} \\ \tau_2 \; \texttt{nf}}{\tau_1 \oplus \tau_2 \; \texttt{nf}}

\inferrule[N-Bang]{\tau \; \texttt{nf}}{!\tau \; \texttt{nf}}

\inferrule[N-IForall]{\tau \; \texttt{nf}}{\forall i : S. \tau \; \texttt{nf}}

\inferrule[N-IExists]{\tau \; \texttt{nf}}{\exists i : S. \tau \; \texttt{nf}}

\inferrule[N-TForall]{\tau \; \texttt{nf}}{\forall \alpha : K. \tau \; \texttt{nf}}

\inferrule[N-List]{\tau \; \texttt{nf}}{L^I \tau \; \texttt{nf}}

\inferrule[N-Conj]{\tau \;  \texttt{nf}}{\Phi\amp\tau \; \texttt{nf}}

\inferrule[N-Impl]{\tau \;  \texttt{nf}}{\Phi\implies\tau \; \texttt{nf}}

\inferrule[N-Monad]{\tau \;  \texttt{nf}}{\M(I,\vec{p}) \tau \; \texttt{nf}}

\inferrule[N-Pot]{\tau \;  \texttt{nf}}{[I|\vec{p}] \tau \; \texttt{nf}}

\inferrule[N-ConstPot]{\tau \;  \texttt{nf}}{[I] \tau \; \texttt{nf}}

\inferrule[N-FamLam]{\tau \;  \texttt{nf}}{\lambda i : S. \tau \; \texttt{nf}}

\inferrule[N-FamApp]{\tau \;  \texttt{ne}}{(\tau \; I) \; \texttt{ne}}

\inferrule[N-NeNf]{\tau \; \texttt{ne}}{\tau \; \texttt{nf}}

\end{mathpar}

\begin{align*}
\texttt{eval}(\alpha) &= \alpha\\
\texttt{eval}(1) &= 1\\
\texttt{eval}(\tau_1 \loli \tau_2) &= \texttt{eval}(\tau_1) \loli \texttt{eval}(\tau_2)\\
\texttt{eval}(\tau_1 \otimes \tau_2) &= \texttt{eval}(\tau_1) \otimes \texttt{eval}(\tau_2)\\
\texttt{eval}(\tau_1 \amp \tau_2) &= \texttt{eval}(\tau_1) \amp \texttt{eval}(\tau_2)\\
\texttt{eval}(\tau_1 \oplus \tau_2) &= \texttt{eval}(\tau_1) \oplus \texttt{eval}(\tau_2)\\
\texttt{eval}(!\tau) &= !\texttt{eval}(\tau)\\
\texttt{eval}(\forall i : S. \tau) &= \forall i : S. \texttt{eval}(\tau)\\
\texttt{eval}(\exists i : S. \tau) &= \exists i : S. \texttt{eval}(\tau)\\
\texttt{eval}(\forall \alpha : K. \tau) &= \forall \alpha : K. \texttt{eval}(\tau)\\
\texttt{eval}(L^I\tau) &= L^I(\texttt{eval}(\tau))\\
\texttt{eval}(\Phi \amp \tau) &= \Phi \amp \texttt{eval}(\tau)\\
\texttt{eval}(\Phi \implies \tau) &= \Phi \implies \texttt{eval}(\tau)\\
\texttt{eval}(\M(I,\vec{p})\tau) &= \M(I,\vec{p})(\texttt{eval}(\tau))\\
\texttt{eval}([I|\vec{p}]\tau) &= [I|\vec{p}](\texttt{eval}(\tau))
\end{align*}
\begin{align*}
\texttt{eval}([I]\tau) &= [I](\texttt{eval}(\tau))\\
\texttt{eval}(\lambda i : S. \tau) &= \lambda i : S. \texttt{eval}(\tau)\\
\texttt{eval}(\tau \; I) &= \begin{cases}
   \tau'[I/i], & \texttt{eval}(\tau) = \lambda i : S. \tau' \\
   \texttt{eval}(\tau) \; I & \texttt{eval}(\tau) \; \texttt{ne}
                              \end{cases}
\end{align*}

\section{\dlambdaamor Theorems and Proofs}

\ctxwfstreng*
\begin{proof}
Immediate from the fact that $\Psi ; \Theta ; \Delta \vdash \Gamma \; \texttt{wf}$ exactly when $\Psi ; \Theta ; \Delta \vdash \tau : \star$ for every $x : \tau \in \Gamma$.
\end{proof}

\conwfstreng*
%\begin{proof}
%Immediate by inversion.
%\end{proof}

\begin{theorem}[Raw Admissibility of Weakening for Sort Checking]
If $\Theta ; \Delta \vdash I : S$ and $\Theta' \supseteq \Theta$, then $\Theta' ; \Delta \vdash I : S$
\end{theorem}

\begin{theorem}[Raw Admissibility of Weakening for Constraint Well-Formedness]
If $\Theta ; \Delta \vdash \Phi \; \texttt{ wf}$ and $\Theta' \supseteq \Theta$, then $\Theta' ; \Delta \vdash \Phi \; \texttt{ wf}$
\end{theorem}

\begin{theorem}[Admissibility of Weakening for Constraint Context Well-Formedness]
If $\Theta \vdash \Delta \; \texttt{wf}$ and $\Theta' \supseteq \Theta$, then $\Theta' \vdash \Delta \; \texttt{wf}$
\end{theorem}

\begin{theorem}[Admissibility of Weakening for Sort Checking]
If $\Theta ; \Delta \pvdash I : S$ and $\Theta' \supseteq \Theta$, then $\Theta' ; \Delta \pvdash I : S$
\end{theorem}

\begin{theorem}[Admissibility of Weakening for Constraint Well-Formedness]
If $\Theta ; \Delta \pvdash \Phi \; \texttt{wf}$ and $\Theta' \supseteq \Theta$, then $\Theta' ; \Delta \pvdash \Phi \; \texttt{ wf}$
\end{theorem}

\begin{theorem}[Raw Index Substitution for Constraint-Well-Formedness]
If $\Theta, j : S_1 ; \Delta \vdash I : S_2$ and $\Theta ; \Delta \pvdash J : S_1$, then 
$\Theta ; \Delta[J/j] \vdash I[J/j] : S_2$
\label{thm:raw-idx-idx-subst}
\end{theorem}
\jtheorem{Proof of \autoref{thm:raw-idx-idx-subst}}{
By induction on the derivation of $\Theta, j : S_1 ; \Delta \vdash I : S_2$. The case for I-Var is immediate.

\jcase{1}{I-Plus}{
\jgivengoal{
    \caseFact{1} $\Theta, j : S_1 ; \Delta \vdash I_1 + I_2 : bS$
    
    \caseFact{2} $\Theta, j : S_1 ; \Delta \vdash I_1: bS$
    
    \caseFact{3} $\Theta, j : S_1 ; \Delta \vdash I_2 : bS$
    
    \caseFact{4} $\Theta \vdash \Delta \; \texttt{wf}$
  }{
      $\Theta ; \Delta[J/j] \vdash (I + J) [J/j] : bS$
  }
  \caseText{By IH on (2) and (3)}
  
  \caseFact{5} $\Theta ; \Delta[J/j] \vdash I_1[J/j] : bS$
  
  \caseFact{6} $\Theta ; \Delta[J/j] \vdash I_2[J/j] : bS$
  
  \caseText{By I-Plus}
  
  \caseFact{7} $\Theta ; \Delta[J/j] \vdash I_1[J/j] + I_2[J/j]: bS$
  
  \caseText{Goal follows by (7).}
}

\jcase{2}{I-Minus}{
\jgivengoal{
    \caseFact{1} $\Theta, j : S_1 ; \Delta \vdash I_1 - I_2 : bS$
    
    \caseFact{2} $\Theta, j : S_1 ; \Delta \vdash I_1: bS$
    
    \caseFact{3} $\Theta, j : S_1 ; \Delta \vdash I_2 : bS$
    
    \caseFact{4} $\Theta, j : S_1 ; \Delta \vDash I_1 \geq I_2$
    
    \caseFact{5} $\Theta \vdash \Delta \; \texttt{wf}$
  }{
      $\Theta ; \Delta[J/j] \vdash (I+1 - I_2) [J/j] : bS$
  }
  \caseText{By IH on (2) and (3)}
  
  \caseFact{5} $\Theta ; \Delta[J/j] \vdash I_1[J/j] : bS$
  
  \caseFact{6} $\Theta ; \Delta[J/j] \vdash I_2[J/j] : bS$
  
  \caseText{Instantiating the quantifier in (4) and using (5)}
  
  \caseFact{7} $\Theta ; \Delta[J/j] \vDash I_1[J/j] \geq I_2[J/j]$
  
  \caseText{By I-Minus on (5) (6) (7)}
  
  \caseFact{8} $\Theta ; \Delta[J/j] \vdash I_1[J/j] - I_2[J/j]: bS$
  
  \caseText{Goal follows by (8).}
}

\jcase{3}{I-Times-$\R^+$}{
 \jgivengoal{
   \caseFact{1} $\Theta, j : S_1 ; \Delta \vdash c \cdot I : \R^+$
   
   \caseFact{2} $\Theta, j : S_1 ; \Delta \vdash I : \R^+$
   
   \caseFact{3} $c \in \R^+$
 }{
   $\Theta ; \Delta[I/i] \vdash (c \cdot I)[J/j] : \R^+$
 }
 \caseText{By IH on (2)}
 
 \caseFact{4} $\Theta ; \Delta[J/j] \vdash I[J/j] : \R^+$
 
 \caseText{By I-Times-$\R^+$ on (3) and (4)}
 
 \caseFact{5} $\Theta ; \Delta[J/j] \vdash c \cdot I[J/j] : \R^+$
 
 \caseText{Goal follows by (5)}
}

\jcase{4}{I-Times-$\potvec$}{Identical to I-Times-$\R^+$}

\jcase{5}{I-Times-$\N$}{Identical to I-Times-$\R^+$}

\jcase{6}{I-Shift}{
  \jgivengoal{
    \caseFact{1} $\Theta, j : S ; \Delta \vdash \lhd I : \potvec$
    
    \caseFact{2} $\Theta, j : S ; \Delta \vdash I : \potvec$
  }{
    $\Theta ; \Delta[J/j] \vdash \left(\lhd I\right)[J/j] : \potvec$
  }
  
  \caseText{By IH on (2)}
  
  \caseFact{3} $\Theta ; \Delta[J/j] \vdash I[J/j] : \potvec$
  
  \caseText{By I-Shift on (3)}
  
  \caseFact{4} $\Theta ; \Delta[J/j] \vdash \lhd I[J/j] : \potvec$
  
  \caseText{Goal follows immediately from (4)}
}

\jcase{7}{I-Lam}{
  \jgivengoal{
    \caseFact{1} $\Theta, j : S_1 ; \Delta \vdash \lambda i : S_2. I : S_2 \to S_3$
    
    \caseFact{2} $\Theta, j : S_1, i : S_2 ; \Delta \vdash I : S_3$
  }{
    $\Theta ; \Delta[J/j] \vdash \lambda i : S_2. I[J/j] : S_2 \to S_3$  
  }
  
  \caseText{By IH on (2)}
  
  \caseFact{3} $\Theta, i : S_2 ; \Delta[J/j] \pvdash I[J/j] : S_3$
  
  \caseText{By I-Lam on (3)}
  
  \caseFact{4} $\Theta ; \Delta[J/j] \vdash \lambda i : S_2. I[J/j] : S_2 \to S_3$
  
  \caseText{Goal follows immediately by (4)}
}

\jcase{8}{I-App}{
  \jgivengoal{
    \caseFact{1} $\Theta, j : S ; \Delta \vdash I_1 \; I_2 : S_2$
    
    \caseFact{2} $\Theta, j : S; \Delta \vdash I_1 : S_1 \to S_2$
    
    \caseFact{3} $\Theta, j : S; \Delta \vdash I_2 : S_1$
  }{
    $\Theta ; \Delta[J/j] \vdash (I_1 \; I_2)[J/j] : S_2$,
  }
  
  \caseText{By IH on (2) and (3)}
  
  \caseFact{4} $\Theta ; \Delta[J/j] \vdash I_1[J/j] : S_1 \to S_2$
  
  \caseFact{5} $\Theta ; \Delta[J/j] \vdash I_2[J/j] : S_1$
  
  \caseText{By I-App on (4) and (5)}
   
  \caseFact{6} $\Theta ; \Delta[J/j] \vdash I_1[J/j] \; I_2[J/j] : S_2$
  
  \caseText{Goal follows from (6)}
}

\jcase{9}{I-Const}{
  \jgivengoal{
     \caseFact{1} $\Theta, j : S ; \Delta \vdash \texttt{const}(I) : \potvec$
     
     \caseFact{2} $\Theta, j : S ; \Delta \vdash I : \R^+$
  }{
    $\Theta ; \Delta[J/j] \vdash \texttt{const}(I)[J/j] : \potvec$
  }
  \caseText{By IH on (2)}
  
  \caseFact{3} $\Theta ; \Delta[J/j] \vdash I[J/j] : \R^+$
  
  \caseText{By I-Const on (3)}
  
  \caseFact{4} $\Theta ; \Delta[J/j] \vdash \texttt{const}(I[J/j]) : \potvec$
  
  \caseText{Goal follows from (4)}
}

\jcase{10}{I-$\N$-Lit} {Immediate.}

\jcase{11}{I-$\R^+$-Lit} {Immediate.}

\jcase{12}{I-$\potvec$-Lit} {Immediate.}
}
\iffalse
\begin{proof}
By induction on the derivation of $\Theta, j : S_1 ; \Delta \vdash I : S_2$.
\begin{itemize}
  \item[(I-Var)] Immediate.
  \item[(I-Plus)] Suppose $\Theta, j : S_1 ; \Delta \vdash I_1 + I_2 : bS$ by way of
  $\Theta, j : S_1 ; \Delta \vdash I_1: bS$ and
  $\Theta, j : S_1 ; \Delta \vdash I_2 : bS$.
  Applying IH twice, we have that
   and
  $\Theta ; \Delta \pvdash I_2[J/j] : bS$.
  By I-Plus, $\Theta ; \Delta \pvdash I_1[J/j] + I_2[J/j]: bS$, as required.
  \item[(I-Minus)] Suppose $\Theta, j : S_1 ; \Delta \vdash I_1 - I_2 : bS$ by way of
  $\Theta, j : S_1 ; \Delta \vdash I_1: bS$,
  $\Theta, j : S_1 ; \Delta \vdash I_2 : bS$, and
  $\Theta, j : S_1 ; \Delta \vDash I_1 \geq I_2$
  Applying IH twice, we have that
  $\Theta ; \Delta \pvdash I_1[J/j] : bS$ and
  $\Theta ; \Delta \pvdash I_2[J/j] : bS$.
  Moreover, we have $\Theta ; \Delta \vDash \forall j : S_1. (I_1 \geq I_2)$ since $\Theta \vdash \Delta \; \texttt{wf}$, and so
  $\Theta ; \Delta \vDash I_1[J/j] \geq I_2[J/j]$
  By I-Minus, $\Theta ; \Delta \pvdash I_1[J/j] - I_2[J/j]: bS$, as required.
  \item[(I-Times-$\R^+$)] Suppose $\Theta , j : S \vdash c \cdot I : \R$ from
  $\Theta, i : S \vdash I : \R$ and
  $c \in \R$.
  By IH,
  $\Theta \pvdash I[J/j] : \R$
  and so by I-Times-$\R^+$,
  $\Theta \pvdash c \cdot I[J/j] : \R$
  \item[(I-Times-$\vec{R}^+$)] Identical to I-Times-$\R^+$.
  \item[(I-Times-$\N$)] Identical to I-Times-$\R^+$.
  \item[(I-Shift)] Suppose $\Theta, j : S ; \Delta \vdash \lhd I : \potvec$ by way of $\Theta, j : S ; \Delta \vdash I : \potvec$.
  By IH,
  $\Theta ; \Delta \pvdash I[J/j] : \potvec$,
  and by I-Shift,
  $\Theta ; \Delta \pvdash \lhd I[J/j] : \potvec$
  as required.
  \item[(I-Lam)] Suppose $\Theta, j : S_1 ; \Delta \vdash \lambda i : S_2. I : S_2 \to S_3$ by way of
  $\Theta, j : S_1, i : S_2 ; \Delta \vdash I : S_3$.
  By IH,
  $\Theta, i : S_2 ; \Delta \pvdash I[J/j] : S_3$,
  and by I-Lam,
  $\Theta ; \Delta \pvdash \lambda i : S_2. I[J/j] : S_2 \to S_3$ as required.
  \item[(I-App)] Suppose $\Theta, j : S ; \Delta \vdash I_1 \; I_2 : S_2$ by way of
  $\Theta, j : S; \Delta \vdash I_1 : S_1 \to S_2$ and
  $\Theta, j : S; \Delta \vdash I_2 : S_1$.
  By IH twice, we have that
  $\Theta ; \Delta \pvdash I_1[J/j] : S_1 \to S_2$ and
  $\Theta ; \Delta \pvdash I_2[J/j] : S_1$.
  By I-App,
  $\Theta ; \Delta \pvdash I_1[J/j] \; I_2[J/j] : S_2$, as required.
  \item[(I-Const)] Suppose $\Theta, j : S ; \Delta \vdash \texttt{const}(I) : \potvec$
  by way of
  $\Theta, j : S ; \Delta \vdash I : \R$.
  By IH,
  $\Theta ; \Delta \pvdash I[J/j] : \R$,
  and so by I-Const,
  $\Theta ; \Delta \pvdash \texttt{const}(I)[J/j] : \potvec$
  as required.
  \item[(I-\* -Lit)] Immediate by the presupposition that $\Theta \vdash \Delta \; \texttt{wf}$.
\end{itemize}
\end{proof}
\fi

\begin{theorem}[Index Substitution for Sort Checking]
If $\Theta, j : S_1 ; \Delta \pvdash I : S_2$ and $\Theta ; \Delta \pvdash J : S_1$ then $\Theta ; \Delta \pvdash I[J/j] : S_2$
\label{thm:idx-idx-subst}
\end{theorem}
\begin{proof}
Immediate by \autoref{thm:raw-idx-idx-subst}, using the fact that $\Theta \vdash \Delta \, \texttt{wf}$.
\end{proof}

\begin{theorem}[Raw Index Substitution for Constraint Well-formedness]
If $\Theta, i : S ; \Delta \vdash \Phi \; \texttt{wf}$ and $\Theta ; \Delta \vdash I : S$ then $\Theta ; \Delta[I/i] \vdash \Phi[I/i] \, \texttt{wf}$
\label{thm:raw-constr-idx-subst}
\end{theorem}
\begin{proof}
By induction on the derivation of $\Theta, i ; S ; \Delta \vdash \Phi \; \texttt{wf}$.
\begin{enumerate}
  \item[(C-Top)] Immediate.
  \item[(C-Bot)] Immediate.
  \item[(C-Conj)] Suppose $\Theta, i : S ; \Delta \vdash \Phi_1 \wedge \Phi_2 \; \texttt{wf}$ by way of
  $\Theta, i : S ; \Delta \vdash \Phi_1 \; \texttt{wf}$ and
  $\Theta, i : S ; \Delta \vdash \Phi_2 \; \texttt{wf}$.
  By IH twice,
  $\Theta; \Delta \pvdash \Phi_1[I/i] \; \texttt{wf}$ and
  $\Theta; \Delta \pvdash \Phi_2[I/i] \; \texttt{wf}$.
  Then, by C-Conj,
  $\Theta ; \Delta \pvdash \Phi_1[I/i] \wedge \Phi_2[I/i] \; \texttt{wf}$
  as required.
  \item[(C-Disj)] Suppose $\Theta, i : S ; \Delta \vdash \Phi_1 \vee \Phi_2 \; \texttt{wf}$ by way of
  $\Theta, i : S ; \Delta \vdash \Phi_1 \; \texttt{wf}$ and
  $\Theta, i : S ; \Delta \vdash \Phi_2 \; \texttt{wf}$.
  By IH twice,
  $\Theta; \Delta \pvdash \Phi_1[I/i] \; \texttt{wf}$ and
  $\Theta; \Delta \pvdash \Phi_2[I/i] \; \texttt{wf}$.
  Then, by C-Disj,
  $\Theta ; \Delta \pvdash \Phi_1[I/i] \vee \Phi_2[I/i] \; \texttt{wf}$
  as required.
  
  \item[(C-Impl)]  Suppose $\Theta, i : S ; \Delta \vdash \Phi_1 \to \Phi_2 \; \texttt{wf}$ by way of
  $\Theta, i : S ; \Delta \vdash \Phi_1 \; \texttt{wf}$ and
  $\Theta, i : S ; \Delta, \Phi_1 \vdash \Phi_2 \; \texttt{wf}$.
  By IH twice,
  $\Theta; \Delta \pvdash \Phi_1[I/i] \; \texttt{wf}$ and
  $\Theta; \Delta, \Phi_1 \pvdash \Phi_2[I/i] \; \texttt{wf}$.
  Then, by C-Impl,
  $\Theta ; \Delta \pvdash \Phi_1[I/i] \to \Phi_2[I/i] \; \texttt{wf}$
  as required.
  
  \item[(C-Forall)] Suppose $\Theta, i : S ; \Delta \vdash \forall j : S'. \Phi \; \texttt{wf}$ by way of
  $\Theta, i : S, j : S' ; \Delta \vdash\Phi \; \texttt{wf}$.
  By IH,
  $\Theta, j : S' ; \Delta \pvdash \Phi[I/i] \; \texttt{wf}$.
  By C-Forall,
  $\Theta;\Delta \pvdash \forall j : S'. \Phi[I/i] \; \texttt{wf}$, as required.
  
  \item[(C-Exists)] Suppose $\Theta, i : S ; \Delta \vdash \exists j : S'. \Phi \; \texttt{wf}$ by way of
  $\Theta, i : S, j : S' ; \Delta \vdash\Phi \; \texttt{wf}$.
  By IH,
  $\Theta, j : S' ; \Delta \pvdash \Phi[I/i] \; \texttt{wf}$.
  By C-Exists,
  $\Theta;\Delta \pvdash \exists j : S'. \Phi[I/i] \; \texttt{wf}$, as required.
  
  \item[(C-Eq)] Immediate by \autoref{thm:idx-idx-subst}.
  \item[(C-Leq)] Immediate by \autoref{thm:idx-idx-subst}.
  \item[(C-Lt)] Immediate by \autoref{thm:idx-idx-subst}.
\end{enumerate}
\end{proof}

\begin{theorem}[Index Substitution for Constraint Well-Formedness]
If $\Theta, i : S ; \Delta \pvdash \Phi \; \texttt{wf}$ and $\Theta ; \Delta \pvdash I : S$ then $\Theta ; \Delta \pvdash \Phi[I/i] \; \texttt{wf}.$
\label{thm:constr-idx-subst}
\end{theorem}
\begin{proof}
Immediate by \autoref{thm:raw-constr-idx-subst}, using the fact that $\Theta \vdash \Delta \, \texttt{wf}$.
\end{proof}


\begin{theorem}[Admissibility of Weakening for Type Formation]
If $\Psi ; \Theta ; \Delta \pvdash \tau : K$, $\Psi' \supseteq \Psi$, and $\Theta' \supseteq \Theta$, then
$\Psi' ; \Theta' ; \Delta \pvdash \tau : K$.
\end{theorem}

\begin{theorem}[Admissibility of Weakening for Term Context Wellformedness]
If $\Psi ; \Theta ; \Delta \vdash \Gamma \; \texttt{wf}$, $\Psi' \supseteq \Psi$, and $\Theta' \supseteq \Theta$, then
$\Psi' ; \Theta' ; \Delta \vdash \Gamma \; \texttt{wf}$.
\end{theorem}

\typeidxsubst*
\begin{proof}
By induction on the derivation of $\Psi ; \Theta, i : S ; \Delta \vdash \tau : K$.
\begin{enumerate}
  \item[(K-Var)] Immediate.
  \item[(K-Unit)] Immediate.
  \item[(K-Arr)] Suppose $\Psi ; \Theta, i : S ; \Delta \vdash \tau_1 \loli \tau_2 : \star$ from
  $\Psi ; \Theta, i : S ; \Delta \vdash \tau_1 : \star$ and
  $\Psi ; \Theta, i : S ; \Delta \vdash \tau_2 : \star$.
  By IH twice, we have that
  $\Psi ; \Theta ; \Delta \pvdash \tau_1[I/i] : \star$ and
  $\Psi ; \Theta ; \Delta \pvdash \tau_2[I/i] : \star$.
  But then by K-Arr, we have
  $\Psi ; \Theta ; \Delta \pvdash \tau_1[I/i] \loli \tau_2[I/i]: \star$
  as required.
  \item[(K-Tensor)] Suppose $\Psi ; \Theta, i : S ; \Delta \vdash \tau_1 \otimes \tau_2 : \star$ from
  $\Psi ; \Theta, i : S ; \Delta \vdash \tau_1 : \star$ and
  $\Psi ; \Theta, i : S ; \Delta \vdash \tau_2 : \star$.
  By IH twice, we have that
  $\Psi ; \Theta ; \Delta \pvdash \tau_1[I/i] : \star$ and
  $\Psi ; \Theta ; \Delta \pvdash \tau_2[I/i] : \star$.
  But then by K-Tensor, we have
  $\Psi ; \Theta ; \Delta \pvdash \tau_1[I/i] \otimes \tau_2[I/i]: \star$
  as required.
  \item[(K-With)] Suppose $\Psi ; \Theta, i : S ; \Delta \vdash \tau_1 \amp \tau_2 : \star$ from
  $\Psi ; \Theta, i : S ; \Delta \vdash \tau_1 : \star$ and
  $\Psi ; \Theta, i : S ; \Delta \vdash \tau_2 : \star$.
  By IH twice, we have that
  $\Psi ; \Theta ; \Delta \pvdash \tau_1[I/i] : \star$ and
  $\Psi ; \Theta ; \Delta \pvdash \tau_2[I/i] : \star$.
  But then by K-With, we have
  $\Psi ; \Theta ; \Delta \pvdash \tau_1[I/i] \amp \tau_2[I/i]: \star$
  as required.
  \item[(K-Sum)] Suppose $\Psi ; \Theta, i : S ; \Delta \vdash \tau_1 \oplus \tau_2 : \star$ from
  $\Psi ; \Theta, i : S ; \Delta \vdash \tau_1 : \star$ and
  $\Psi ; \Theta, i : S ; \Delta \vdash \tau_2 : \star$.
  By IH twice, we have that
  $\Psi ; \Theta ; \Delta \pvdash \tau_1[I/i] : \star$ and
  $\Psi ; \Theta ; \Delta \pvdash \tau_2[I/i] : \star$.
  But then by K-Sum, we have
  $\Psi ; \Theta ; \Delta \pvdash \tau_1[I/i] \oplus \tau_2[I/i]: \star$
  as required.
  \item[(K-Bang)] Suppose $\Psi ; \Theta, i : S ; \Delta \vdash !\tau : \star$ by way of
  $\Psi ; \Theta, i : S ; \Delta \vdash \tau : \star$.
  By IH,
  $\Psi ; \Theta ; \Delta \pvdash \tau[I/i] : \star$,
  and so by K-Bang,
  $\Psi ; \Theta ; \Delta \pvdash \tau[I/i] : !\star$.
  \item[(K-IForall)] Suppose $\Psi ; \Theta, i : S;  \Delta \vdash \forall j : S'. \tau : \star$ by way of
  $\Psi ; \Theta, i : S, j : S' ; \Delta \vdash \tau : \star$.
  By IH,
  $\Psi ; \Theta, j : S' ; \Delta \pvdash \tau[I/i] : \star$.
  By K-IForall,
  $\Psi ; \Theta ; \Delta \pvdash \forall j : S', \tau[I/i] : \star$, as required.
  \item[(K-IExists)] Suppose $\Psi ; \Theta, i : S;  \Delta \vdash \exists j : S'. \tau : \star$ by way of
  $\Psi ; \Theta, i : S, j : S' ; \Delta \vdash \tau : \star$.
  By IH,
  $\Psi ; \Theta, j : S' ; \Delta \pvdash \tau[I/i] : \star$.
  By K-IExists,
  $\Psi ; \Theta ; \Delta \pvdash \exists j : S', \tau[I/i] : \star$, as required.
  \item[(K-List)] Suppose $\Psi ; \Theta, i : S ; \Delta \vdash L^J \tau : \star$ by way of
  $\Theta, i : S ; \Delta \vdash J : \N$ and
  $\Psi ; \Theta, i : S ; \Delta \vdash \tau : \star$.
  By \autoref{thm:idx-idx-subst},
  $\Theta ; \Delta \pvdash J[I/i] : \N$.
  By IH,
  $\Psi ; \Theta ; \Delta \pvdash \tau[I/i] : \star$.
  By K-List,
  $\Psi ; \Theta l \Delta \pvdash L^{J[I/i]}\left(\tau[I/i]\right)$,
  as required.
  
  \item[(K-Impl)] Suppose $\Psi ; \Theta, i : S ; \Delta \vdash \Phi \implies \tau : \star$ by way of
  $\Psi ; \Theta, i : S ; \Delta \vdash \tau : \star$ and
  $\Theta, i : S ; \Delta \vdash \Phi \; \texttt{wf}$.
  By IH,
  $\Psi ; \Theta ; \Delta \pvdash \tau[I/i] : \star$.
  By \autoref{thm:constr-idx-subst},
  $\Theta ; \Delta \pvdash \Phi[I/i] \; \texttt{wf}$.
  Then, by K-Conj,
  $\Psi ; \Theta ; \Delta \vdash \Phi[I/i] \amp \tau[I/i] : \star$
  as required.
  Suppose $\Psi ; \Theta, i : S ; \Delta \vdash \Phi \amp \tau : \star$ by way of
  $\Psi ; \Theta, i : S ; \Delta \vdash \tau : \star$ and
  $\Theta, i : S ; \Delta \vdash \Phi \; \texttt{wf}$.
  By IH,
  $\Psi ; \Theta ; \Delta \pvdash \tau[I/i] : \star$.
  By \autoref{thm:constr-idx-subst},
  $\Theta ; \Delta \pvdash \Phi[I/i] \; \texttt{wf}$.
  Then, by K-Implies,
  $\Psi ; \Theta ; \Delta \pvdash \Phi[I/i] \implies \tau[I/i] : \star$
  as required.
  
  \item[(K-Monad)] Suppose
  $\Psi ; \Theta, i : S ; \Delta \vdash \M(J,\vec{p}) \tau : \star$
  by way of
  $\Theta, i : S ; \Delta \vdash J : \mathbb{N}$,
  $\Theta, i : S ; \Delta \vdash \vec{p} : \potvec$, and
  $\Psi ; \Theta, i : S ; \Delta \vdash \tau : \star$.
  By \autoref{thm:idx-idx-subst}.
  $\Theta ; \Delta \pvdash J[I/i] : \mathbb{N}$ and
  $\Theta ; \Delta \pvdash \vec{p}[I/i] : \potvec$.
  By IH,
  $\Psi ; \Theta ; \Delta \pvdash \tau[I/i] : \star$.
  Then, by K-Monad,
  $\Psi ; \Theta  \Delta \pvdash \M(J[I/i],\vec{p}[I/i]) \tau[I/i] : \star$
  
  \item[(K-Pot)] Suppose
  $\Psi ; \Theta, i : S ; \Delta \vdash [J|\vec{p}] \tau : \star$
  by way of
  $\Theta, i : S ; \Delta \vdash J : \mathbb{N}$,
  $\Theta, i : S ; \Delta \vdash \vec{p} : \potvec$, and
  $\Psi ; \Theta, i : S ; \Delta \vdash \tau : \star$.
  By \autoref{thm:idx-idx-subst}.
  $\Theta ; \Delta \pvdash J[I/i] : \mathbb{N}$ and
  $\Theta ; \Delta \pvdash \vec{p}[I/i] : \potvec$.
  By IH,
  $\Psi ; \Theta ; \Delta \pvdash \tau[I/i] : \star$.
  Then, by K-Pot,
  $\Psi ; \Theta  \Delta \pvdash \left[J[I/i],\vec{p}[I/i]\right] \tau[I/i] : \star$
  
  \item[(K-ConstPot)] Suppose
  $\Psi ; \Theta, i : S ; \Delta \vdash [J] \; \tau : \star$
  by way of
  $\Psi ; \Theta, i : S ; \Delta \vdash \tau : \star$ and
  $\Theta, i : S ; \Delta \vdash J : \potvec$.
  By IH,
  $\Psi ; \Theta ; \Delta \pvdash \tau[I/i] : \star$.
  By \autoref{thm:idx-idx-subst},
  $\Theta ; \Delta \pvdash J[I/i] : \potvec$.
  By K-ConstPot,
  $\Psi ; \Theta ; \Delta \pvdash \left[J[I/i]\right] \; \left(\tau[I/i]\right) : \star$
  as required.
  
  \item[(K-FamLam)] Suppose
  $\Psi ; \Theta, i : S ; \Delta \vdash \lambda j : S'. \tau : S' \to K$
  by way of
  $\Psi ; \Theta, i : S, j ; S' ; \Delta \vdash \tau : K$.
  By IH,
  $\Psi ; \Theta, j ; S' ; \Delta \pvdash \tau[I/i] : K$.
  By K-FamLam,
  $\Psi ; \Theta ; \Delta \pvdash \lambda j : S'.\tau[I/i] : S' \to K$
  as required.
  
  \item[(K-FamApp)] Suppose
  $\Psi ; \Theta, i : S ; \Delta \vdash \tau \; J : K$ by way of
  $\Theta, i : S ; \Delta \vdash J : S$ and
  $\Psi ; \Theta, i : S ; \Delta \vdash \tau : S \to K$.
  By \autoref{thm:idx-idx-subst},
  $\Theta ; \Delta \pvdash J[I/i] : S$.
  By IH,
  $\Psi ; \Theta ; \Delta \pvdash \tau[I/i] : S \to K$.
  By K-FamApp,
  $\Psi ; \Theta ; \Delta \pvdash \left(\tau[I/i]\right) \; \left(J[I/i]\right) : K$
  as required.
  
\end{enumerate}
\end{proof}

\begin{theorem}[Admissibility of Weakening for Subtyping]
Suppose $\Psi ; \Theta ; \Delta \pvdash \tau \subty \tau' : K$, $\Psi' \supseteq \Psi$, and $\Theta' \supseteq \Theta$.
Then, $\Psi' ; \Theta' ; \Delta \pvdash \tau \subty \tau'$.
\end{theorem}

\subtystreng*
%\begin{proof}
%Routine induction.
%\end{proof}


\begin{theorem}[Context Subsumption Includes Subset]
If $\Psi ; \Theta ; \Delta \pvdash \Gamma'$ and $\Gamma \subseteq \Gamma'$ as sets, then $\Psi ; \Theta ; \Delta \pvdash \Gamma' \wknto \Gamma$
\label{thm:ctx-sub-subset1}
\end{theorem}
\begin{proof}
By induction on $|\Gamma|$.
If $\Gamma = \emptyset$, this is immediate by CS-Emp.
Now suppose $\Gamma = \Gamma'', x : \tau$. Then, since $\Gamma \subseteq \Gamma'$, we have $x : \tau \in \Gamma'$, and $\Gamma'' \subseteq \Gamma' \setminus \{x : \tau\}$. Moreover, by S-Refl, $\Psi ; \Theta ; \Delta \vdash \tau \subty \tau : \star$, and so we are done by IH.
\end{proof}

\ctxsubsubset*
\begin{proof}
By an easy induction on $|\Gamma|$.
\end{proof}

\begin{theorem}[Admissibility of Weakening for Context Subsumption]
If $\Psi ; \Theta ; \Delta \pvdash \Gamma \wknto \Gamma'$ and $\Psi' \supseteq \Psi$, $\Theta' \supseteq \Theta$, and $\Delta' \supseteq \Delta$, then
$\Psi' ; \Theta' ; \Delta' \pvdash \Gamma \wknto \Gamma'$
\label{thm:ctx-sub-wkn}
\end{theorem}

\begin{theorem}[Strengthening for Context Subsumption]
Suppose $\Psi ; \Theta ; \Delta \vdash \Gamma,\Gamma' \texttt{ wf}$.
If $\Psi' ; \Theta' ; \Delta \pvdash \Gamma \wknto \Gamma'$ with $\Theta' \supseteq \Theta$ and $\Psi' \supseteq \Psi$, then $\Psi ; \Theta ; \Delta \pvdash \Gamma \wknto \Gamma'$.
\end{theorem}
\label{thm:ctx-sub-streng}
\begin{proof}
Immediate by Theorem~\ref{thm:ctx-sub-subset2} and Theorem~\ref{thm:subty-streng}
\end{proof}

\ctxsubswap*
\begin{proof}
By Theorem~\ref{thm:ctx-sub-subset2}, it suffices to show that for every $x : \tau \in \Gamma_2'$, there is some $\tau'$ such that $x : \tau' \in \Gamma_2'$,
and $\Psi ; \Theta ; \Delta \vdash \tau' \subty \tau : \star$. Suppose $x : \tau \in \Gamma_2'$. Then, $x : \tau \in \Gamma_1'$. Further, since $\Psi' ; \Theta' ;  \Delta' \vdash \Gamma_2 \wknto \Gamma_2'$, there is some $\tau'$ such that $x : \tau' \in \Gamma_2$. But then, $x : \tau' \in \Gamma_1$, and so since  $\Psi ; \Theta ; \Delta \vdash \Gamma_1 \wknto \Gamma_1'$, we have that $\Psi ; \Theta ; \Delta \vdash \tau' \subty \tau : \star$, by Theorem~\ref{thm:ctx-sub-subset2}.
\end{proof}

\subsection{Normalization}

\canonforms*
\begin{proof}
Inversion on $\Psi ;\Theta ; \Delta \vdash \tau : S \to K$  and then $\tau \; \texttt{nf}$.
\end{proof}

\idxsubstnf*
\begin{proof}
By an easy simultaneous induction. This is only true because we don't require index terms inside a type to be in normal form for the type to be in normal form.
\end{proof}

\idxsubsteval*
%\begin{proof}
%Induction on $\tau$.
%\end{proof}

%Define $\# \tau$ to be the number of type connectives in $\tau$.

\normthm*
\begin{proof}
%By induction on $\# \tau$. The base cases (base type and type variables) are immediate. For the inductive case, we break into cases on the syntax of $\tau$, inverting $\Psi ;\Theta ; \Delta \vdash \tau : K$-- since this judgment is syntax directed, we may write this as if it were cases over the derivation.

By induction on $\Psi ; \Theta ; \Delta \vdash \tau : K$.

\begin{enumerate}
  \item[(K-Var)] Immediate.
  \item[(K-Unit)] Immediate.  
  \item[(K-Arr)] Suppose ${\Psi ; \Theta ; \Delta \vdash \tau_1 \loli \tau_2 : \star}$ from $\Psi ; \Theta ; \Delta \vdash \tau_1 : \star$ and  $\Psi ; \Theta ; \Delta \vdash \tau_2 : \star$. By IH, we have for $i \in \{1,2\}$
  \begin{enumerate}[1.]
    \item $\Psi ; \Theta ; \Delta \vdash \texttt{eval}(\tau_i) : \star$
    \item $\Psi ; \Theta ; \Delta \vdash \tau_i \equiv \texttt{eval}(\tau_i) : \star$
    \item $\texttt{eval}(\tau_i) \; \texttt{nf}$.
  \end{enumerate}
  Note that $\texttt{eval}(\tau_1 \loli \tau_2) = \texttt{eval}(\tau_1) \loli \texttt{eval}(\tau_2)$. Then,
  \begin{enumerate}[1.]
    \item ${\Psi ; \Theta ; \Delta \vdash \texttt{eval}(\tau_1) \loli \texttt{eval}(\tau_2) : \star}$ by K-Arr
    \item $\Psi ; \Theta ; \Delta \vdash \tau_1 \loli \tau_2 \equiv \texttt{eval}(\tau_1) \loli \tau_2 : \star$ by two uses of S-Arr
    \item Since $\tau_i \; \texttt{nf}$, we have that $\tau_1 \loli \tau_2 \; \texttt{nf}$.
  \end{enumerate}
  as required.
  \item[(K-Tensor)] Suppose ${\Psi ; \Theta ; \Delta \vdash \tau_1 \otimes \tau_2 : \star}$ from $\Psi ; \Theta ; \Delta \vdash \tau_1 : \star$ and  $\Psi ; \Theta ; \Delta \vdash \tau_2 : \star$. By IH, we have for $i \in \{1,2\}$
  \begin{enumerate}[1.]
    \item $\Psi ; \Theta ; \Delta \vdash \texttt{eval}(\tau_i) : \star$
    \item $\Psi ; \Theta ; \Delta \vdash \tau_i \equiv \texttt{eval}(\tau_i) : \star$
    \item $\texttt{eval}(\tau_i) \; \texttt{nf}$.
  \end{enumerate}
  Note that $\texttt{eval}(\tau_1 \otimes \tau_2) = \texttt{eval}(\tau_1) \otimes \texttt{eval}(\tau_2)$. Then,
  \begin{enumerate}[1.]
    \item ${\Psi ; \Theta ; \Delta \vdash \texttt{eval}(\tau_1) \loli \texttt{eval}(\tau_2) : \star}$ by K-Tensor
    \item $\Psi ; \Theta ; \Delta \vdash \tau_1 \otimes \tau_2 \equiv \texttt{eval}(\tau_1) \otimes \tau_2 : \star$ by two uses of S-Tensor
    \item Since $\tau_i \; \texttt{nf}$, we have that $\tau_1 \otimes \tau_2 \; \texttt{nf}$.
  \end{enumerate}
  as required.
  \item[(K-With)] Suppose ${\Psi ; \Theta ; \Delta \vdash \tau_1 \amp \tau_2 : \star}$ from $\Psi ; \Theta ; \Delta \vdash \tau_1 : \star$ and  $\Psi ; \Theta ; \Delta \vdash \tau_2 : \star$. By IH, we have for $i \in \{1,2\}$
  \begin{enumerate}[1.]
    \item $\Psi ; \Theta ; \Delta \vdash \texttt{eval}(\tau_i) : \star$
    \item $\Psi ; \Theta ; \Delta \vdash \tau_i \equiv \texttt{eval}(\tau_i) : \star$
    \item $\texttt{eval}(\tau_i) \; \texttt{nf}$.
  \end{enumerate}
  Note that $\texttt{eval}(\tau_1 \amp \tau_2) = \texttt{eval}(\tau_1) \amp \texttt{eval}(\tau_2)$. Then,
  \begin{enumerate}[1.]
    \item ${\Psi ; \Theta ; \Delta \vdash \texttt{eval}(\tau_1) \amp \texttt{eval}(\tau_2) : \star}$ by K-With
    \item $\Psi ; \Theta ; \Delta \vdash \tau_1 \amp \tau_2 \equiv \texttt{eval}(\tau_1) \amp \tau_2 : \star$ by two uses of S-With
    \item Since $\tau_i \; \texttt{nf}$, we have that $\tau_1 \amp \tau_2 \; \texttt{nf}$.
  \end{enumerate}
  as required.
  \item[(K-Sum)] Suppose ${\Psi ; \Theta ; \Delta \vdash \tau_1 \oplus \tau_2 : \star}$ from $\Psi ; \Theta ; \Delta \vdash \tau_1 : \star$ and  $\Psi ; \Theta ; \Delta \vdash \tau_2 : \star$. By IH, we have for $i \in \{1,2\}$
  \begin{enumerate}[1.]
    \item $\Psi ; \Theta ; \Delta \vdash \texttt{eval}(\tau_i) : \star$
    \item $\Psi ; \Theta ; \Delta \vdash \tau_i \equiv \texttt{eval}(\tau_i) : \star$
    \item $\texttt{eval}(\tau_i) \; \texttt{nf}$.
  \end{enumerate}
  Note that $\texttt{eval}(\tau_1 \amp \tau_2) = \texttt{eval}(\tau_1) \oplus \texttt{eval}(\tau_2)$. Then,
  \begin{enumerate}[1.]
    \item ${\Psi ; \Theta ; \Delta \vdash \texttt{eval}(\tau_1) \oplus \texttt{eval}(\tau_2) : \star}$ by K-Sum
    \item $\Psi ; \Theta ; \Delta \vdash \tau_1 \oplus \tau_2 \equiv \texttt{eval}(\tau_1) \oplus \tau_2 : \star$ by two uses of S-Sum
    \item Since $\tau_i \; \texttt{nf}$, we have that $\tau_1 \oplus \tau_2 \; \texttt{nf}$.
  \end{enumerate}
  as required.
  \item[(K-Bang)] Suppose ${\Psi ; \Theta ; \Delta \vdash !\tau : \star}$ from ${\Psi ; \Theta ; \Delta \vdash \tau : \star}$.
  By IH, we have that
  \begin{enumerate}[1.]
   \item ${\Psi ; \Theta ; \Delta \vdash \texttt{eval}(\tau) : \star}$
   \item ${\Psi ; \Theta ; \Delta \vdash \tau \equiv \texttt{eval}(\tau) : \star}$
   \item $\texttt{eval}(\tau) \; \texttt{nf}$
  \end{enumerate}
  Then, noting that $\texttt{eval}(!\tau) = !\texttt{eval}(\tau)$,
  \begin{enumerate}[1.]
   \item ${\Psi ; \Theta ; \Delta \vdash !\texttt{eval}(\tau) : \star}$ by K-Bang
   \item ${\Psi ; \Theta ; \Delta \vdash !\tau \equiv !\texttt{eval}(\tau) : \star}$ by S-Bang
   \item Since $\texttt{eval}(\tau) \; \texttt{nf}$, $!\texttt{eval}(\tau) \; \texttt{nf}$
  \end{enumerate}
  as required.
  \item[(K-IForall)] Suppose ${\Psi ; \Theta ; \Delta \vdash \forall i : S. \tau : \star}$ from ${\Psi ; \Theta, i : S ; \Delta \vdash \tau : \star}$.
  By IH,
  \begin{enumerate}[1.]
   \item ${\Psi ; \Theta, i : S ; \Delta \vdash \texttt{eval}(\tau) : \star}$.
   \item ${\Psi ; \Theta, i : S ; \Delta \vdash \tau \equiv \texttt{eval}(\tau) : \star}$.
   \item $\texttt{eval}(\tau) \; \texttt{nf}$
  \end{enumerate}
  Then, noting that $\texttt{eval}(\forall i : S. \tau) = \forall i : S.\texttt{eval}(\tau)$, we have that
  \begin{enumerate}[1.]
   \item ${\Psi ; \Theta ; \Delta \vdash \forall i : S. \texttt{eval}(\tau) : \star}$ by K-IForall
   \item ${\Psi ; \Theta ; \Delta \vdash \forall i : S.\tau \equiv \forall i : S. \texttt{eval}(\tau) : \star}$ by S-IForall twice
   \item $\forall i : S. \texttt{eval}(\tau) \; \texttt{nf}$ since $\texttt{eval}(\tau) \; \texttt{nf}$
  \end{enumerate}
  as required.
  \item[(K-IExists)] Suppose ${\Psi ; \Theta ; \Delta \vdash \exists i : S. \tau : \star}$ from ${\Psi ; \Theta, i : S ; \Delta \vdash \tau : \star}$.
  By IH,
  \begin{enumerate}[1.]
   \item ${\Psi ; \Theta, i : S ; \Delta \vdash \texttt{eval}(\tau) : \star}$.
   \item ${\Psi ; \Theta, i : S ; \Delta \vdash \tau \equiv \texttt{eval}(\tau) : \star}$.
   \item $\texttt{eval}(\tau) \; \texttt{nf}$
  \end{enumerate}
  Then, noting that $\texttt{eval}(\exists i : S. \tau) = \exists i : S.\texttt{eval}(\tau)$, we have that
  \begin{enumerate}[1.]
   \item ${\Psi ; \Theta ; \Delta \vdash \exists i : S. \texttt{eval}(\tau) : \star}$ by K-IExists
   \item ${\Psi ; \Theta ; \Delta \vdash \exists i : S.\tau \equiv \forall i : S. \texttt{eval}(\tau) : \star}$ by S-IExists twice
   \item $\exists i : S. \texttt{eval}(\tau) \; \texttt{nf}$ since $\texttt{eval}(\tau) \; \texttt{nf}$
  \end{enumerate}
  as required.
  \item[(K-TForall)] Suppose ${\Psi ; \Theta ; \Delta \vdash \forall \alpha : K. \tau : \star}$ from ${\Psi, \alpha : K; \Theta ; \Delta \vdash \tau : \star}$.
  By IH,
  \begin{enumerate}[1.]
   \item ${\Psi, \alpha : K ; \Theta ; \Delta \vdash \texttt{eval}(\tau) : \star}$.
   \item ${\Psi, \alpha : K ; \Theta ; \Delta \vdash \tau \equiv \texttt{eval}(\tau) : \star}$.
   \item $\texttt{eval}(\tau) \; \texttt{nf}$
  \end{enumerate}
  Then, noting that $\texttt{eval}(\forall \alpha : K. \tau) = \forall \alpha : K.\texttt{eval}(\tau)$, we have that
  \begin{enumerate}[1.]
   \item ${\Psi ; \Theta ; \Delta \vdash \forall \alpha : K. \texttt{eval}(\tau) : \star}$ by K-TForall
   \item ${\Psi ; \Theta ; \Delta \vdash \forall \alpha : K.\tau \equiv \forall \alpha : K. \texttt{eval}(\tau) : \star}$ by S-TForall twice
   \item $\forall \alpha : K. \texttt{eval}(\tau) \; \texttt{nf}$ since $\texttt{eval}(\tau) \; \texttt{nf}$
  \end{enumerate}
  as required.
  \item[(K-List)] Suppose ${\Psi ; \Theta ; \Delta \vdash L^I \tau : \star}$ from $\Psi ; \Theta ; \Delta \vdash \tau : \star$ and $\Theta ; \Delta \vdash I : \N$.
  By IH, we have
  \begin{enumerate}[1.]
    \item $\Psi ; \Theta ; \Delta \vdash \texttt{eval}(\tau) : \star$
    \item $\Psi ; \Theta ; \Delta \vdash \tau \equiv \texttt{eval}(\tau) : \star$
    \item $\texttt{eval}(\tau) \; \texttt{nf}$
  \end{enumerate}
  Then, recalling that $\texttt{eval}\left(L^I \,\tau\right) = L^I\left(\texttt{eval}(\tau)\right)$, we have
  \begin{enumerate}[1.]
    \item $\Psi ; \Theta ; \Delta \vdash L^I\left(\texttt{eval}(\tau)\right) : \star$ by K-List, with $\Theta ; \Delta \vdash I : \N$
    \item $\Psi ; \Theta ; \Delta \vdash L^I \, \tau \equiv L^I\left(\texttt{eval}(\tau)\right) : \star$ by S-List, using the fact that $\Theta ; \Delta \vDash I = I$
    \item $L^I\left(\texttt{eval}(\tau)\right) \; \texttt{nf}$ because $\texttt{eval}(\tau) \; \texttt{nf}$
  \end{enumerate}
  as required.
  \item[(K-Conj)] Suppose ${\Psi ; \Theta ; \Delta \vdash \Phi \amp \tau : \star}$ from $\Psi ; \Theta ; \Delta \vdash \tau : \star$ and $\Theta ; \Delta \vdash \Phi \texttt{ wf}$. By  IH, we have:
  \begin{enumerate}[1.]
    \item $\Psi ; \Theta ; \Delta \vdash \texttt{eval}(\tau) : \star$
    \item $\Psi ; \Theta ; \Delta \vdash \tau \equiv \texttt{eval}(\tau) : \star$
    \item $\texttt{eval}(\tau) \; \texttt{nf}$
  \end{enumerate}
  Then, noting that $\texttt{eval}(\Phi \amp \tau) = \Phi \amp \texttt{eval}(\tau)$, we can conclude:
  \begin{enumerate}[1.]
    \item $\Psi ; \Theta ; \Delta \vdash \Phi \amp\texttt{eval}(\tau) : \star$ by K-Conj with $\Theta ; \Delta \vdash \Phi \texttt{ wf}$.
    \item $\Psi ; \Theta ; \Delta \vdash \Phi \amp \tau \equiv \Phi \amp\texttt{eval}(\tau) : \star$ by two uses of S-Conj, using the fact that $\Theta ; \Delta \vDash \Phi \to \Phi$.
    \item $\Phi \amp\texttt{eval}(\tau) \; \texttt{nf}$ since $\texttt{eval}(\tau) \; \texttt{nf}$
  \end{enumerate}
  as required.
  \item[(K-Impl)] Suppose ${\Psi ; \Theta ; \Delta \vdash \Phi \implies \tau : \star}$ from $\Psi ; \Theta ; \Delta \vdash \tau : \star$ and $\Theta ; \Delta \vdash \Phi \texttt{ wf}$. By  IH, we have:
  \begin{enumerate}[1.]
    \item $\Psi ; \Theta ; \Delta \vdash \texttt{eval}(\tau) : \star$
    \item $\Psi ; \Theta ; \Delta \vdash \tau \equiv \texttt{eval}(\tau) : \star$
    \item $\texttt{eval}(\tau) \; \texttt{nf}$
  \end{enumerate}
  Then, noting that $\texttt{eval}(\Phi \implies \tau) = \Phi \implies \texttt{eval}(\tau)$, we can conclude:
  \begin{enumerate}[1.]
    \item $\Psi ; \Theta ; \Delta \vdash \Phi \implies\texttt{eval}(\tau) : \star$ by K-Impl with $\Theta ; \Delta \vdash \Phi \texttt{ wf}$.
    \item $\Psi ; \Theta ; \Delta \vdash \Phi \implies \tau \equiv \Phi \amp\texttt{eval}(\tau) : \star$ by two uses of S-Impl, using the fact that $\Theta ; \Delta \vDash \Phi \to \Phi$.
    \item $\Phi \implies\texttt{eval}(\tau) \; \texttt{nf}$ since $\texttt{eval}(\tau) \; \texttt{nf}$
  \end{enumerate}
  as required.
  \item[(K-Monad)] Suppose ${\Psi ; \Theta ; \Delta \vdash \M(I,\vec{p}) \tau : \star}$ from $\Psi ; \Theta ; \Delta \vdash \tau : \star$ with $ \Theta ; \Delta \vdash I : \mathbb{N}$ and $\Theta ; \Delta \vdash \vec{p} : \vec{\mathbb{R}^+}$. Then, by IH,
  \begin{enumerate}[1.]
    \item $\Psi ; \Theta ; \Delta \vdash \texttt{eval}(\tau) : \star$
    \item $\Psi ; \Theta ; \Delta \vdash \tau \equiv \texttt{eval}(\tau) : \star$
    \item $\texttt{eval}(\tau) \; \texttt{nf}$
  \end{enumerate}
  Note that $\texttt{eval}(\M(I,\vec{p}) \tau) = \M(I,\vec{p})(\texttt{eval}(\tau))$, and so we may conclude:
  \begin{enumerate}
    \item $\Psi ; \Theta ; \Delta \vdash \M(I,\vec{p})(\texttt{eval}(\tau)) : \star$ by K-Monad with $ \Theta ; \Delta \vdash I : \mathbb{N}$ and $\Theta ; \Delta \vdash \vec{p} : \vec{\mathbb{R}^+}$
    \item $\Psi ; \Theta ; \Delta \vdash \M(I,\vec{p}) \tau \equiv \M(I,\vec{p})(\texttt{eval}(\tau)) : \star$ by two uses of S-Monad, using the fact that $\Theta ; \Delta \vdash (I = I) \wedge (\vec{p} \leq \vec{p})$.
    \item $\M(I,\vec{p})(\texttt{eval}(\tau)) \; \texttt{nf}$ since $\texttt{eval}(\tau) \; \texttt{nf}$
  \end{enumerate}
  \item[(K-Pot)] Suppose ${\Psi ; \Theta ; \Delta \vdash [I|\vec{p}] \tau : \star}$ from $\Psi ; \Theta ; \Delta \vdash \tau : \star$ with $\Theta ; \Delta \vdash I : \mathbb{N}$ and $\Theta ; \Delta \vdash \vec{p} : \vec{\mathbb{R}^+}$. By IH, we have that
    \begin{enumerate}[1.]
    \item $\Psi ; \Theta ; \Delta \vdash \texttt{eval}(\tau) : \star$
    \item $\Psi ; \Theta ; \Delta \vdash \tau \equiv \texttt{eval}(\tau) : \star$
    \item $\texttt{eval}(\tau) \; \texttt{nf}$
  \end{enumerate}
  Then, noting that $\texttt{eval}([I|\vec{p}] \tau) = [I|\vec{p}] (\texttt{eval}(\tau))$, we may conclude that
  \begin{enumerate}[1.]
    \item $\Psi ; \Theta ; \Delta \vdash [I|\vec{p}] (\texttt{eval}(\tau)) : \star$ by K-Pot with $\Theta ; \Delta \vdash I : \mathbb{N}$ and $\Theta ; \Delta \vdash \vec{p} : \vec{\mathbb{R}^+}$
    \item  $\Psi ; \Theta ; \Delta \vdash [I|\vec{p}] \tau \equiv [I|\vec{p}] (\texttt{eval}(\tau)) : \star$ by two uses of S-Pot, using the fact that $\Theta ; \Delta \vdash (I = I) \wedge (\vec{p} \leq \vec{p})$.
  \end{enumerate}
  
  \item[(K-ConstPot)] Suppose ${\Psi ; \Theta ; \Delta \vdash [I] \; \tau : \star}$ from $\Psi ; \Theta ; \Delta \vdash \tau : \star$ and $\Theta ; \Delta \vdash I : \mathbb{R}^+$. By IH,
  \begin{enumerate}[1.]
    \item $\Psi ; \Theta ; \Delta \vdash \texttt{eval}(\tau) : \star$
    \item $\Psi ; \Theta ; \Delta \vdash \tau \equiv \texttt{eval}(\tau) : \star$
    \item $\texttt{eval}(\tau) \; \texttt{nf}$
  \end{enumerate}
  Then, noting that $\texttt{eval}([I] \; \tau) = [I] \; \texttt{eval}(\tau)$, we can conclude:
  \begin{enumerate}
   \item $\Psi ; \Theta ; \Delta \vdash [I] \; \texttt{eval}(\tau) : \star$ by K-ConstPot with $\Theta ; \Delta \vdash I : \mathbb{R}^+$
   \item $\Psi ; \Theta ; \Delta \vdash [I] \; \tau \equiv [I] \; \texttt{eval}(\tau) : \star$ by S-ConstPot, using $\Theta ; \Delta \vDash I = I$.
   \item $[I] \; \texttt{eval}(\tau) \; \texttt{nf}$ because $\texttt{eval}(\tau) \; \texttt{nf}$
  \end{enumerate}
  as required.
  
  \item[(K-FamLam)] Suppose ${\Psi ; \Theta ; \Delta \vdash \lambda i : S. \tau : S \to K}$ from ${\Psi ; \Theta, i : S ; \Delta \vdash \tau : K}$. By IH,
  \begin{enumerate}[1.]
   \item $\Psi ; \Theta, i : S ; \Delta \vdash \texttt{eval}(\tau) : K$
   \item $\Psi ; \Theta, i : S ; \Delta\vdash \tau \equiv \texttt{eval}(\tau) : K$
   \item $\texttt{eval}(\tau) \; \texttt{nf}$
   %\item $\#\texttt{eval}(\tau) \leq \# \tau$
\end{enumerate}
  By definition, $\texttt{eval}(\lambda i : S. \tau) = \lambda i : S. \texttt{eval}(\tau)$. Then, we can proceed to prove the four claims:
  \begin{enumerate}[1.]
    \item By K-FamLam, $\Psi ; \Theta ; \Delta \vdash \lambda i : S. \texttt{eval}(\tau) : S \to K$.
    \item By S-FamLam in both directions, $\Psi ; \Theta ; \Delta\vdash \lambda i : S. \tau \equiv \lambda i : S. \texttt{eval}(\tau) : K$
    \item Since $\texttt{eval}(\tau) \; \texttt{nf}$, $\lambda i : S. \texttt{eval}(\tau) \; \texttt{nf}$ also.
    %\item Finally, $\#\texttt{eval}(\lambda i : S. \tau) = \#(\lambda i : S. \texttt{eval}(\tau)) = 1 + \#\texttt{eval}(\tau) \leq 1 + \#\tau = \#(\lambda i : S. \tau)$
  \end{enumerate}
  
  \item[(K-FamApp)] Suppose ${\Psi ; \Theta ; \Delta \vdash \tau \; I : K}$ from $\Psi ; \Theta ; \Delta \vdash \tau : S \to K$ and  $\Theta ; \Delta \vdash I : S$. By IH,
  we have that
  \begin{enumerate}[1.]
   \item $\Psi ; \Theta ; \Delta \vdash \texttt{eval}(\tau) : S \to K$
   \item $\Psi ; \Theta ; \Delta \vdash \tau \equiv \texttt{eval}(\tau) : S \to K$
   \item $\texttt{eval}(\tau) \; \texttt{nf}$
   %\item $\#\texttt{eval}(\tau) \leq \# \tau$
  \end{enumerate}
  By \autoref{thm:canon-forms}, we have two possibilities for $\texttt{eval}(\tau)$. First, suppose that  $\texttt{eval}(\tau) \; \texttt{ne}$. Then, $\texttt{eval}(\tau \; I) = \texttt{eval}(\tau) \; I$, and so we can easily prove the claims:
  \begin{enumerate}[1.]
   \item By K-FamApp, since $\Theta ; \Delta \vdash I : S$, we have that $\Psi ; \Theta ; \Delta \vdash \texttt{eval}(\tau) \; I : K$.
   \item By S-FamApp in both directions, we have that $\Psi ; \Theta ; \Delta \vdash \tau \; I \equiv \texttt{eval}(\tau) \; I : S \to K$
   \item Since $\texttt{eval}(\tau) \; \texttt{ne}$, we have that $\texttt{eval}(\tau) \; I \; \texttt{ne}$, and so $\texttt{eval}(\tau) \; I \; \texttt{nf}$.
   %\item $\#(\texttt{eval}(\tau) \; I) = 1 + \#\texttt{eval}(\tau) \leq 1 + \#\tau = \#(\tau \; I)$
  \end{enumerate}
  Otherwise, suppose that $\texttt{eval}(\tau) = \lambda i : S. \tau'$ with $\tau' \; \texttt{nf}$ and $\Psi ; \Theta , i : S ; \Delta \vdash \tau' : K$. In this case, $\texttt{eval}(\tau \; I) = \tau'[I/i]$:
  \begin{enumerate}[1.]
   \item By \textbf{Substitution} with $\Theta ; \Delta \vdash I : S$, we have that $\Psi ; \Theta ; \Delta \vdash \tau'[I/i] : K$.
   \item We already know that $\Psi ; \Theta ; \Delta \vdash \tau \equiv \texttt{eval}(\tau) : S \to K$, but $\texttt{eval}(\tau) = \lambda i : S. \tau'$, and so
   $\Psi ; \Theta ; \Delta \vdash \tau \equiv \lambda i : S.\tau' : S \to K$. By S-FamApp, $\Psi ; \Theta ; \Delta \vdash \tau \; I \equiv (\lambda i : S.\tau') \; I : K$. Postcomposing with both directions of S-Fam-Beta-\{1,2\}, we have that $\Psi ; \Theta ; \Delta \vdash \tau \; I \equiv \tau'[I/i] : K$, as required.
   \item Since $\tau'\; \texttt{nf}$, we have by \autoref{thm:idx-subst-nf} that $\tau'[I/i] \; \texttt{nf}$.
  \end{enumerate}
  
\end{enumerate}
\end{proof}

\begin{theorem}
~\begin{itemize}
  \item If $\tau \; \texttt{nf}$, then $\texttt{eval}(\tau) = \tau$
  \item If $\tau \; \texttt{ne}$, then $\texttt{eval}(\tau) = \tau$
\end{itemize}
\label{thm:norm-idemp}
\end{theorem}
%\begin{proof}
%Simultaneous induction.
%\end{proof}

\section{Rules of \bilambdaamor}

\subsection{Algorithmic Wellformedness Judgments}
\begin{mathpar}
\inferrule[AWF-CCtxE]{ }{\Theta \vdash \cdot \; \texttt{wf} \gens \top}

\inferrule[AWF-CCtxNe]{\Theta \vdash \Delta \; \texttt{wf} \gens \Phi_1\\ \Theta ; \Delta \vdash \Phi \; \texttt{wf} \gens \Phi_2}{\Theta \vdash \Delta,\Phi \; \texttt{wf} \gens \Phi_1 \wedge (\bigwedge \Delta \to \Phi_2)}

\inferrule[AWF-TCtxE]{ }{\Psi ; \Theta ; \Delta \vdash \cdot \; \texttt{wf} \gens \top}

\inferrule[AWF-TCtxNE]{\Psi ; \Theta ; \Delta \vdash \Gamma \; \texttt{wf} \gens \Phi_1\\ \Psi ; \Theta ; \Delta \vdash \tau : \star \gens \Phi_2}{\Psi ; \Theta ; \Delta \vdash \Gamma, x : \tau \; \texttt{wf} \gens \Phi_1 \wedge \Phi_2}
\end{mathpar}

\subsection{Algorithmic Sort Checking/Inference}

\begin{mathpar}
\inferrule[AI-Var]{i : S \in \Theta}{\Theta ; \Delta \vdash i : S \gens \top}

\inferrule[AI-Plus]{\Theta ; \Delta \vdash I : bS \gens \Phi_1 \\ \Theta ; \Delta \vdash J : bS \gens \Phi_2}{\Theta ; \Delta \vdash I + J : bS \gens \Phi_1 \wedge \Phi_2}

\inferrule[AI-Minus]{\Theta ; \Delta \vdash I : bS \gens \Phi_1 \\ \Theta ; \Delta \vdash J : bS \gens \Phi_2}{\Theta ; \Delta \vdash I - J : bS \gens \Phi_1 \wedge \Phi_2 \wedge (I \geq J)}\\

\inferrule[AI-Times-$\mathbb{R}$]{c \in \mathbb{R}^+ \\ \Theta ; \Delta \vdash I : \mathbb{R}^+ \gens \Phi}{\Theta ; \Delta \vdash c \cdot I : \mathbb{R}^+ \gens \Phi}

\inferrule[AI-Times-$\vec{\mathbb{R}}$]{c \in \mathbb{R}^+ \\ \Theta ; \Delta \vdash I : \vec{\mathbb{R}^+} \gens \Phi}{\Theta ; \Delta \vdash c \cdot I : \vec{\mathbb{R}^+} \gens \Phi}

\inferrule[AI-Times-$\mathbb{N}$]{c \in \mathbb{N} \\ \Theta ; \Delta \vdash I : \mathbb{N} \gens \Phi}{\Theta ; \Delta \vdash c \cdot I : \mathbb{N} \gens \Phi}\\


\inferrule[AI-Shift]{\Theta ; \Delta \vdash I : \vec{\mathbb{R}^+} \gens \Phi}{\Theta ; \Delta \vdash \; \lhd I : \vec{\mathbb{R}^+} \gens \Phi}

\inferrule[AI-Lam]{\Theta, i : bS ; \Delta \vdash I : S \gens \Phi}{\Theta ; \Delta \vdash \lambda i : bS. I : bS \to S \gens \forall i : S. \Phi}

\inferrule[AI-App]{\Theta ; \Delta \vdash I : bS \to S \gens \Phi_1\\ \Theta ; \Delta \vdash J : bS \gens \Phi_2}{\Theta ; \Delta \vdash I \; J : S \gens \Phi_1 \wedge \Phi_2}\\

\inferrule[AI-Sum]{\Theta;\Delta \vdash I_0 : \mathbb{N} \gens \Phi_1\\ \Theta;\Delta \vdash I_1 : \mathbb{N} \gens \Phi_2\\ 
                 \Theta,i : \N;\Delta, I_0 \leq i < I_1 \vdash J : bS \gens \Phi_3}
                 {\Theta;\Delta \vdash \sum_{i=I_0}^{I_1} J : bS \gens \Phi_1 \wedge \Phi_2 \wedge \forall i : \N.(I_0 \leq i < I_1 \to \Phi_3)}
                 
\end{mathpar}
\begin{mathpar}                 
                 
\inferrule[AI-ConstVec]{\Theta ; \Delta \vdash I : \mathbb{R}^+ \gens \Phi}{\Theta ; \Delta \vdash \texttt{const}(I) : \vec{\mathbb{R}^+} \gens \Phi}

\inferrule[AI-Vec-Lit]{\Phi = \bigwedge_i c_i \geq 0}{\Theta ; \Delta \vdash (c_0,\dots,c_k) : \mathbb{R}^+ \gens \Phi}

\inferrule[AI-Nat-Lit]{ }{\Theta ; \Delta \vdash n : \mathbb{N} \gens n \geq 0}

\inferrule[AI-PosReal-Lit]{ }{\Theta ; \Delta \vdash r : \mathbb{R}^+ \gens r \geq 0}
\end{mathpar}

\subsection{Algorithmic Constraint Wellformedness}

\begin{mathpar}
\inferrule[AC-Top]{ }{\Theta;\Delta \vdash \top \; \texttt{wf}\gens \top}

\inferrule[AC-Bot]{ }{\Theta;\Delta \vdash \bot \; \texttt{wf}\gens \top}

\inferrule[AC-Conj]{\Theta ; \Delta \vdash \Phi_1\; \texttt{wf} \gens \Phi_1' \\ \Theta ; \Delta \vdash \Phi_2\; \texttt{wf} \gens \Phi_2'}{\Theta;\Delta \vdash \Phi_1 \wedge \Phi_2\; \texttt{wf} \gens \Phi_1' \wedge \Phi_2'}

\inferrule[AC-Disj]{\Theta ; \Delta \vdash \Phi_1\; \texttt{wf} \gens \Phi_1' \\ \Theta ; \Delta \vdash \Phi_2\; \texttt{wf} \gens \Phi_2'}{\Theta;\Delta \vdash \Phi_1 \vee \Phi_2\; \texttt{wf} \gens \Phi_1' \wedge \Phi_2'}

\inferrule[AC-Impl]{\Theta ; \Delta \vdash \Phi_1\; \texttt{wf} \gens \Phi_1' \\ \Theta ; \Delta \vdash \Phi_2\; \texttt{wf} \gens \Phi_2'}{\Theta;\Delta,\Phi_1 \vdash \Phi_1 \to \Phi_2\; \texttt{wf} \gens \Phi_1' \wedge (\Phi_1 \to \Phi_2')}

\inferrule[AC-Forall]{\Theta, i : S ; \Delta \vdash \Phi\; \texttt{wf} \gens \Phi'}{\Theta ; \Delta \vdash \forall i : S. \Phi\; \texttt{wf} \gens \forall i : S. \Phi'}

\inferrule[AC-Exists]{\Theta, i : S ; \Delta, \Phi \vdash \Phi\; \texttt{wf} \gens \Phi'}{\Theta ; \Delta \vdash \exists i : S. \Phi\; \texttt{wf} \gens \forall i : S. (\Phi \to \Phi')}

\inferrule[AC-Leq]{\Theta ; \Delta \vdash I : bS \gens \Phi_1\\ \Theta ; \Delta \vdash J : bS \gens \Phi_2}{\Theta ; \Delta \vdash I \leq J \; \texttt{wf} \gens \Phi_1 \wedge \Phi_2}
\inferrule[AC-Lt]{\Theta ; \Delta \vdash I : bS \gens \Phi_1\\ \Theta ; \Delta \vdash J : bS \gens \Phi_2}{\Theta ; \Delta \vdash I < J\; \texttt{wf} \gens \Phi_1 \wedge \Phi_2}

\inferrule[AC-Eq]{\Theta ; \Delta \vdash I : bS \gens \Phi_1\\ \Theta ; \Delta \vdash J : bS \gens \Phi_2}{\Theta ; \Delta \vdash I = J\; \texttt{wf} \gens \Phi_1 \wedge \Phi_2}
\end{mathpar}

\subsection{Algorithmic Kind Checking/Inference}

\begin{mathpar}
\inferrule[AK-Var]{\alpha : K \in \Psi}{\Psi ; \Theta ; \Delta \vdash \alpha : K \gens \top}

\inferrule[AK-Unit]{ }{\Psi ; \Theta ; \Delta \vdash 1 : \star \gens \top}

\end{mathpar}
\begin{mathpar}

\inferrule[AK-Arr]{\Psi ; \Theta ; \Delta \vdash \tau_1 : \star \gens \Phi_1\\ \Psi ; \Theta ; \Delta \vdash \tau_2 : \star \gens \Phi_2}{\Psi ; \Theta ; \Delta \vdash \tau_1 \loli \tau_2 : \star \gens \Phi_1 \wedge \Phi_2}

\inferrule[AK-Tensor]{\Psi ; \Theta ; \Delta \vdash \tau_1 : \star \gens \Phi_1\\ \Psi ; \Theta ; \Delta \vdash \tau_2 : \star \gens \Phi_2}{\Psi ; \Theta ; \Delta \vdash \tau_1 \otimes \tau_2 : \star \gens \Phi_1 \wedge \Phi_2}

\inferrule[AK-With]{\Psi ; \Theta ; \Delta \vdash \tau_1 : \star \gens \Phi_1\\ \Psi ; \Theta ; \Delta \vdash \tau_2 : \star \gens \Phi_2}{\Psi ; \Theta ; \Delta \vdash \tau_1 \amp \tau_2 : \star \gens \Phi_1 \wedge \Phi_2}

\inferrule[AK-Sum]{\Psi ; \Theta ; \Delta \vdash \tau_1 : \star \gens \Phi_1\\ \Psi ; \Theta ; \Delta \vdash \tau_2 : \star \gens \Phi_2}{\Psi ; \Theta ; \Delta \vdash \tau_1 \oplus \tau_2 : \star \gens \Phi_1 \wedge \Phi_2}

\inferrule[AK-Bang]{\Psi ; \Theta ; \Delta \vdash \tau : \star \gens \Phi}{\Psi ; \Theta ; \Delta \vdash !\tau : \star \gens \Phi}

\inferrule[AK-IForall]{\Psi ; \Theta, i : S ; \Delta \vdash \tau : \star \gens \Phi}{\Psi ; \Theta ; \Delta \vdash \forall i : S. \tau : \star \gens \forall i :S.\Phi}

\inferrule[AK-IExists]{\Psi ; \Theta, i : S ; \Delta \vdash \tau : \star \gens \Phi}{\Psi ; \Theta ; \Delta \vdash \exists i : S. \tau : \star \gens \forall i : S. \Phi}

\inferrule[AK-TForall]{\Psi, \alpha : K ; \Theta ; \Delta \vdash \tau : \star \gens \Phi}{\Psi ; \Theta ; \Delta \vdash \forall \alpha : K. \tau : \star \gens \Phi}

\inferrule[AK-List]{\Theta ; \Delta \vdash I : \mathbb{N} \gens \Phi_1 \\ \Psi ; \Theta ; \Delta \vdash \tau : \star \gens \Phi_2}{\Psi ; \Theta ; \Delta \vdash L^I \tau : \star \gens \Phi_1 \wedge \Phi_2}

\inferrule[AK-Conj]{\Theta ; \Delta \vdash \Phi\; \texttt{wf} \gens \Phi_1 \\ \Psi ; \Theta ; \Delta \vdash \tau : \star \gens \Phi_2}{\Psi ; \Theta ; \Delta \vdash \Phi \amp \tau : \star \gens \Phi_1 \wedge \Phi_2}

\inferrule[AK-Impl]{\Theta ; \Delta \vdash \Phi\; \texttt{wf} \gens \Phi_1 \\ \Psi ; \Theta ; \Delta, \Phi \vdash \tau : \star \gens \Phi_2}{\Psi ; \Theta ; \Delta \vdash \Phi \implies \tau : \star \gens \Phi_1 \wedge (\Phi \to \Phi_2)}

\inferrule[AK-Monad]{ \Theta ; \Delta \vdash I : \mathbb{N} \gens \Phi_1\\ \Theta ; \Delta \vdash \vec{p} : \vec{\mathbb{R}^+} \gens \Phi_2\\ \Psi ; \Theta ; \Delta \vdash \tau : \star \gens \Phi_3}{\Psi ; \Theta ; \Delta \vdash \M(I,\vec{p}) \tau : \star \gens \Phi_1 \wedge \Phi_2 \wedge \Phi_3}

\inferrule[AK-Pot]{\Theta ; \Delta \vdash I : \mathbb{N} \gens \Phi_1 \\ \Theta ; \Delta \vdash \vec{p} : \vec{\mathbb{R}^+} \gens \Phi_2\\ \Psi ; \Theta ; \Delta \vdash \tau : \star \gens \Phi_3}{\Psi ; \Theta ; \Delta \vdash [I|\vec{p}] \tau : \star \gens \Phi_1 \wedge \Phi_2 \wedge \Phi_3}

\end{mathpar}

\begin{mathpar}


\inferrule[AK-ConstPot]{\Theta ; \Delta \vdash I : \mathbb{R}^+ \gens \Phi_1 \\ \Psi ; \Theta ; \Delta \vdash \tau : \star \gens \Phi_2}{\Psi ; \Theta ; \Delta \vdash [I] \; \tau : \star \gens \Phi_1 \wedge \Phi_2}

\inferrule[AK-FamLam]{\Psi ; \Theta, i : S ; \Delta \vdash \tau : K \gens \Phi}{\Psi ; \Theta ; \Delta \vdash \lambda i : S. \tau : S \to K \gens \forall i : S. \Phi}

\inferrule[AK-FamApp]{\Theta ; \Delta \vdash I : S \gens \Phi_1 \\ \Psi ; \Theta ; \Delta \vdash \tau : S \to K \gens \Phi_2}{\Psi ; \Theta ; \Delta \vdash \tau \; I : K \gens \Phi_1 \wedge \Phi_2}
\end{mathpar}

\subsection{Algorithmic Subtyping}
%Presupposition: when $\Psi ; \Theta ; \Delta \vdash \tau_i : K \gens \Phi'$ with $\Theta ; \Delta \vDash \Phi'$, we may judge $\Psi ; \Theta ; \Delta \vdash \tau_1 \subty \tau_2 : K \gens \Phi$ and 
%$\Psi ; \Theta ; \Delta \vdash \tau_1 \subtynf \tau_2 : K \gens \Phi$

\begin{mathpar}
\inferrule[AS-Unit]{ }{\Psi ; \Theta ; \Delta \vdash 1 \subtynf 1 : \star \gens \top}

\inferrule[AS-Var]{ }{\Psi ; \Theta ; \Delta \vdash \alpha \subtynf \alpha : \star \gens \top}

\inferrule[AS-Arr]{\Psi ; \Theta ; \Delta \vdash \tau_1' \subtynf \tau_1 : \star \gens \Phi_1 \\ \Psi ; \Theta ; \Delta \vdash \tau_2 \subtynf \tau_2' : \star \gens \Phi_2}{\Psi ; \Theta ; \Delta \vdash \tau_1 \loli \tau_2 \subtynf \tau_1' \loli \tau_2' : \star \gens \Phi_1 \wedge \Phi_2}

\inferrule[AS-Tensor]{\Psi ; \Theta ; \Delta \vdash \tau_1 \subtynf \tau_1' : \star  \gens \Phi_1 \\ \Psi ; \Theta ; \Delta \vdash \tau_2 \subtynf \tau_2' : \star \gens \Phi_2}{\Psi ; \Theta ; \Delta \vdash \tau_1 \otimes \tau_2 \subtynf \tau_1' \otimes \tau_2' : \star \gens \Phi_1 \wedge \Phi_2}

\inferrule[AS-With]{\Psi ; \Theta ; \Delta \vdash \tau_1 \subtynf \tau_1' : \star \gens \Phi_1 \\ \Psi ; \Theta ; \Delta \vdash \tau_2 \subtynf \tau_2' : \star \gens \Phi_2}{\Psi ; \Theta ; \Delta \vdash \tau_1 \amp \tau_2 \subtynf \tau_1' \amp \tau_2' \gens \Phi_1 \wedge \Phi_2}

\inferrule[AS-Sum]{\Psi ; \Theta ; \Delta \vdash \tau_1 \subtynf \tau_1' : \star \gens \Phi_1 \\ \Psi ; \Theta ; \Delta \vdash \tau_2 \subtynf \tau_2' : \star \gens \Phi_2}{\Psi ; \Theta ; \Delta \vdash \tau_1 \oplus \tau_2 \subtynf \tau_1' \oplus \tau_2' : \star \gens \Phi_1 \wedge \Phi_2}

\inferrule[AS-Bang]{\Psi ; \Theta ; \Delta \vdash \tau_1 \subtynf \tau_2 : \star \gens \Phi}{\Psi ; \Theta ; \Delta \vdash !\tau_1 \subtynf !\tau_2 : \star \gens \Phi}

\inferrule[AS-IForall]{\Psi ; \Theta, i : S ; \Delta \vdash \tau_1 \subtynf \tau_2 : \star \gens \Phi}{\Psi ; \Theta ; \Delta \vdash \forall i : S. \tau_1 \subtynf \forall i : S. \tau_2 : \star \gens \forall i : S. \Phi}

\inferrule[AS-IExists]{\Psi ; \Theta, i : S ; \Delta \vdash \tau_1 \subtynf \tau_2 : \star \gens \Phi}{\Psi ; \Theta ; \Delta \vdash \exists i : S. \tau_1 \subtynf \exists i : S. \tau_2 : \star \gens \forall i : S. \Phi}

\inferrule[AS-TForall]{\Psi, \alpha : K ; \Theta ; \Delta \vdash \tau_1 \subtynf \tau_2 : \star \gens \Phi}{\Psi ; \Theta ; \Delta \vdash \forall \alpha : K. \tau_1 \subtynf \forall \alpha : K. \tau_2 : \star \gens \Phi}

\inferrule[AS-List]{\Psi ; \Theta ; \Delta \vdash \tau_1 \subtynf \tau_2 : \star \gens \Phi}{\Psi ; \Theta ; \Delta \vdash L^I \tau_1 \subtynf L^J \tau_2 : \star \gens I = J \wedge \Phi}

\end{mathpar}
\begin{mathpar}

\inferrule[AS-Impl]{\Psi ; \Theta ; \Delta \vdash \tau_1 \subtynf \tau_2 : \star \gens \Phi}{\Psi ; \Theta ; \Delta \vdash \Phi_1 \implies \tau_1 \subtynf \Phi_2 \implies \tau_2 : \star \gens \Phi \wedge (\Phi_2 \to \Phi_1)}

\inferrule[AS-Conj]{\Psi ; \Theta ; \Delta \vdash \tau_1 \subtynf \tau_2 : \star \gens \Phi}{\Psi ; \Theta ; \Delta \vdash \Phi_1 \amp \tau_1 \subtynf \Phi_2 \amp \tau_2 : \star \gens \Phi \wedge (\Phi_1 \to \Phi_2)}

\inferrule[AS-Monad]{\Psi ; \Theta ; \Delta \vdash \tau_1 \subtynf \tau_2 : \star \gens \Phi}{\Psi ; \Theta ; \Delta \vdash \M(I,\vec{q}) \tau_1 \subtynf \M(J,\vec{p}) \tau_2 : \star \gens (I = J) \wedge (\vec{q} \leq \vec{p}) \wedge \Phi}

\inferrule[AS-Pot]{\Psi ; \Theta ; \Delta \vdash \tau_1 \subtynf \tau_2 : \star \gens \Phi}{\Psi ; \Theta ; \Delta \vdash [I|\vec{q}] \tau_1 \subtynf [J|\vec{p}] \tau_2 : \star \gens (I = J) \wedge (\vec{p} \leq \vec{q}) \wedge \Phi}

\inferrule[AS-ConstPot]{\Psi ; \Theta ; \Delta \vdash \tau_1 \subtynf \tau_2 : \star \gens \Phi}{\Psi ; \Theta ; \Delta \vdash [I] \tau_1 \subtynf [J] \tau_2 : \star \gens \Phi \wedge (J \leq I)}

\inferrule[AS-FamLam]{\Psi ; \Theta, i : S ; \Delta \vdash \tau_1 \subtynf \tau_2 : K \gens \Phi}{\Psi ; \Theta ; \Delta \vdash \lambda i : S. \tau_1 \subtynf \lambda i : S. \tau_2 : S \to K \gens \forall i : S. \Phi}

\inferrule[AS-FamApp]{\Psi ; \Theta ; \Delta \vdash \tau_1 \subtynf \tau_2 : S \to K \gens \Phi}{\Psi ; \Theta ; \Delta \vdash \tau_1 \; I \subtynf \tau_2 \; J : K \gens (I = J) \wedge \Phi}

\inferrule[AS-Normalize]{\Psi ; \Theta ; \Delta \vdash \texttt{eval}(\tau_1) \subtynf \texttt{eval}(\tau_2) : K \gens \Phi}{\Psi ; \Theta ; \Delta \vdash \tau_1 \subty \tau_2 : K\gens \Phi}

\end{mathpar}

\subsection{Algorithmic Type Checking/Inference}

%Presupposition: when $\Psi ; \Theta ; \Delta \vdash \tau : \star \gens \Phi'$ with $\Theta ; \Delta \vDash \Phi'$, we define $\Psi ; \Theta ; \Delta ; \Omega ; \Gamma\vdash e \infers\checks \tau \gens \Phi, \Gamma'$

\begin{mathpar}
\inferrule[AT-Var-1]
{x : \tau \in \Gamma}{\Psi ; \Theta ; \Delta ; \Omega ; \Gamma\vdash x \infers \tau \gens \top, \Gamma \setminus \{x : \tau\}}

\inferrule[AT-Var-2]
{x : \tau \in \Omega}{\Psi ; \Theta ; \Delta ; \Omega ; \Gamma\vdash x \infers \tau \gens \top, \Gamma}

\inferrule[AT-Unit]
{ }{\Psi ; \Theta ; \Delta ; \Omega ; \Gamma\vdash () \infers 1 \gens \top, \Gamma}

\end{mathpar}
\begin{mathpar}

\inferrule[AT-Base]
{ }{\Psi ; \Theta ; \Delta ; \Omega ; \Gamma\vdash c \infers b \gens \top, \Gamma}


\inferrule[AT-Absurd]
{ }{
\Psi ; \Theta ; \Delta ; \Omega ; \Gamma \vdash \texttt{absurd} \checks \tau \gens \bot,\Gamma
}

\inferrule[AT-Nil]
{\text{}}
{\Psi ; \Theta ; \Delta ; \Omega ; \Gamma\vdash \texttt{nil} \checks L^I \tau \gens I = 0, \Gamma}

\inferrule[AT-Cons]
{\Psi ; \Theta ; \Delta ; \Omega ; \Gamma\vdash e_1 \checks \tau \gens \Phi_1, \Gamma_1\\
\Psi ; \Theta ; \Delta ; \Omega ; \Gamma_1\vdash e_2 \checks L^{I-1} \tau \gens \Phi_2, \Gamma_2
}
{\Psi ; \Theta ; \Delta ; \Omega ; \Gamma\vdash e_1 :: e_2 \checks L^I \tau \gens (I \geq 1) \wedge \Phi_1 \wedge \Phi_2, \Gamma_2}

\inferrule[AT-Match]
{
\Psi ; \Theta ; \Delta ; \Omega ; \Gamma\vdash e \infers L^I \tau \gens \Phi_1, \Gamma_1\\
\Psi ; \Theta ; \Delta, I = 0 ; \Omega ; \Gamma_1\vdash e_1 \checks \tau' \gens \Phi_2,\Gamma_2\\
\Psi ; \Theta ; \Delta, I \geq 1; \Omega ; \Gamma_1, h : \tau, t : L^{I-1} \tau \vdash e_2 \checks \tau' \gens \Phi_3,\Gamma_3\\
\Phi_\texttt{body} = (I = 0 \to \Phi_2) \wedge (I \geq 1 \to \Phi_3 )\\
\Gamma' = \Gamma_2 \cap (\Gamma_3 \setminus \{h,t\})
}
{\Psi ; \Theta ; \Delta ; \Omega ; \Gamma\vdash \texttt{match}(e,e_1,h.t.e_2) \checks \tau' \gens \Phi_1 \wedge \Phi_\texttt{body}, \Gamma'}


\inferrule[AT-ExistI]
{
\Theta ; \Delta \vdash I : S \gens \Phi_1\\
\Psi ; \Theta ; \Delta ; \Omega ; \Gamma\vdash e \checks \tau[I/i] \gens \Phi_2,\Gamma'
}{
\Psi ; \Theta ; \Delta ; \Omega ; \Gamma\vdash \texttt{pack}[I](e) \checks \exists i:S.\tau \gens \Phi_1 \wedge \Phi_2, \Gamma'
}

\inferrule[AT-ExistE]
{
\Psi ; \Theta ; \Delta ; \Omega ; \Gamma\vdash e \infers \exists i : S.\tau \gens \Phi_1, \Gamma_1\\
\Psi ; \Theta, i : S ; \Delta ; \Omega ; \Gamma_1, x : \tau \vdash e' \checks \tau' \gens \Phi_2, \Gamma_2\\
\Phi = \Phi_1 \wedge (\forall i : S. \Phi_2)
}{
\Psi ; \Theta ; \Delta ; \Omega ; \Gamma\vdash \texttt{unpack } (i,x) = e \texttt{ in } e' \checks \tau' \gens \Phi , \Gamma_2 \setminus \{x : \tau\}
}

\inferrule[AT-Lam]
{
\Psi ; \Theta ; \Delta ; \Omega ; \Gamma, x : \tau_1 \vdash e \checks \tau_2, \gens \Phi, \Gamma'
}{
\Psi ; \Theta ; \Delta ; \Omega ; \Gamma\vdash \lambda x.e \checks \tau_1 \loli \tau_2 \gens \Phi, \Gamma' \setminus \{x : \tau_1\}
}

\inferrule[AT-App]
{
\Psi ; \Theta ; \Delta ; \Omega ; \Gamma\vdash e_1 \infers \tau_1 \loli \tau_2 \gens \Phi_1, \Gamma_1\\
\Psi ; \Theta ; \Delta ; \Omega ; \Gamma_1\vdash e_2 \checks \tau_1 \gens \Phi_2, \Gamma_2
}{
\Psi ; \Theta ; \Delta ; \Omega ; \Gamma\vdash e_1 \, e_2 \infers  \tau_2 \gens \Phi_1 \wedge \Phi_2, \Gamma_2
}

\end{mathpar}
\begin{mathpar}

\inferrule[AT-TensorI]
{
\Psi ; \Theta ; \Delta ; \Omega ; \Gamma\vdash e_1 \checks \tau_1 \gens \Phi_1, \Gamma_1\\
\Psi ; \Theta ; \Delta ; \Omega ; \Gamma_1\vdash e_2 \checks \tau_2 \gens \Phi_2, \Gamma_2\\
}{
\Psi ; \Theta ; \Delta ; \Omega ; \Gamma\vdash \angles{e_1,e_2} \checks \tau_1 \otimes \tau_2 \gens \Phi_1 \wedge \Phi_2,\Gamma_2
}

\inferrule[AT-TensorE]
{
\Psi ; \Theta ; \Delta ; \Omega ; \Gamma\vdash e \infers \tau_1 \otimes \tau_2 \gens \Phi_1, \Gamma_1\\
\Psi ; \Theta ; \Delta ; \Omega ; \Gamma_1,x : \tau_1, y : \tau_2\vdash e' \checks \tau' \gens \Phi_2,\Gamma_2
}{
\Psi ; \Theta ; \Delta ; \Omega ; \Gamma\vdash \texttt{let } \angles{x,y} = e \texttt{ in } e' \checks \tau' \gens \Phi_1 \wedge \Phi_2, \Gamma_2 \setminus \{x,y\}
}

\inferrule[AT-WithI]
{
\Psi ; \Theta ; \Delta ; \Omega ; \Gamma \vdash e_1 \checks \tau_1 \gens \Phi_1, \Gamma_1\\
\Psi ; \Theta ; \Delta ; \Omega ; \Gamma \vdash e_2 \checks \tau_2 \gens \Phi_2, \Gamma_2
}{
\Psi ; \Theta ; \Delta ; \Omega ; \Gamma \vdash (e_1,e_2) \checks \tau_1 \amp \tau_2 \gens \Phi_1 \wedge \Phi_2, \Gamma_1 \cap \Gamma_2
}

\inferrule[AT-Fst]
{
\Psi ; \Theta ; \Delta ; \Omega ; \Gamma \vdash e \infers \tau_1 \amp \tau_2 \gens \Phi,\Gamma'
}{
\Psi ; \Theta ; \Delta ; \Omega ; \Gamma \vdash \texttt{fst}(e) \infers \tau_1 \gens \Phi,\Gamma'
}

\inferrule[AT-Snd]
{
\Psi ; \Theta ; \Delta ; \Omega ; \Gamma \vdash e \infers \tau_1 \amp \tau_2 \gens \Phi,\Gamma'
}{
\Psi ; \Theta ; \Delta ; \Omega ; \Gamma \vdash \texttt{snd}(e) \infers \tau_2 \gens \Phi,\Gamma'
}

\inferrule[AT-Inl]
{
\Psi ; \Theta ; \Delta ; \Omega ; \Gamma \vdash e \checks \tau_1 \gens \Phi,\Gamma'
}{
\Psi ; \Theta ; \Delta ; \Omega ; \Gamma \vdash \texttt{inl}(e) \checks \tau_1 \oplus \tau_2 \gens \Phi,\Gamma'
}

\inferrule[AT-Inr]
{
\Psi ; \Theta ; \Delta ; \Omega ; \Gamma \vdash e \checks \tau_2 \gens \Phi,\Gamma'
}{
\Psi ; \Theta ; \Delta ; \Omega ; \Gamma \vdash \texttt{inr}(e) \checks \tau_1 \oplus \tau_2 \gens \Phi,\Gamma'
}

\inferrule[AT-Case]
{
\Psi ; \Theta ; \Delta ; \Omega ; \Gamma \vdash e \infers \tau_1 \oplus \tau_2 \gens \Phi_1, \Gamma_1\\
\Psi ; \Theta ; \Delta ; \Omega ; \Gamma_1, x: \tau_1 \vdash e_1 \checks \tau \gens \Phi_2,\Gamma_2\\
\Psi ; \Theta ; \Delta ; \Omega ; \Gamma_1, y: \tau_2 \vdash e_2 \checks \tau \gens \Phi_3,\Gamma_3\\
\Gamma' = (\Gamma_2 \setminus \{x : \tau_1\}) \cap (\Gamma_3 \setminus \{y : \tau_2\})
}{
\Psi ; \Theta ; \Delta ; \Omega ; \Gamma \vdash \texttt{case}(e,x.e_1,y.e_2) \checks \tau \gens \Phi_1 \wedge \Phi_2 \wedge \Phi_3, \Gamma'
}

\inferrule[AT-ExpI]
{
\Psi ; \Theta ; \Delta ; \Omega ; \cdot \vdash e \checks \tau \gens \Phi, \Gamma'
}{
\Psi ; \Theta ; \Delta ; \Omega ; \Gamma \vdash !e \checks !\tau \gens \Phi, \Gamma
}

\inferrule[AT-ExpE]{
\Psi ; \Theta ; \Delta ; \Omega ; \Gamma \vdash e \infers !\tau \gens \Phi_1,\Gamma_1\\
\Psi ; \Theta ; \Delta ; \Omega, x : \tau ; \Gamma_1 \vdash e' \checks \tau' \gens \Phi_2,\Gamma_2
}{
\Psi ; \Theta ; \Delta ; \Omega ; \Gamma \vdash \texttt{let } !x = e \texttt{ in } e' \checks \tau' \gens \Phi_1 \wedge \Phi_2, \Gamma_2
}

\end{mathpar}

\begin{mathpar}

\inferrule[AT-TAbs]
{
\Psi, \alpha : K ; \Theta ; \Delta ; \Omega ; \Gamma \vdash e \checks \tau \gens \Phi, \Gamma'
}{
\Psi ; \Theta ; \Delta ; \Omega ; \Gamma \vdash \Lambda \alpha. e \checks \forall \alpha : K.\tau \gens \Phi,\Gamma'
}

\inferrule[AT-TApp]
{
\Psi ; \Theta ; \Delta ; \Omega ; \Gamma \vdash e \infers \forall \alpha : K.\tau \gens \Phi_1, \Gamma'\\
\Psi ; \Theta ; \Delta \vdash \tau' : K \gens \Phi_2
}{
\Psi ; \Theta ; \Delta ; \Omega ; \Gamma \vdash e [\tau'] \infers \tau[\tau'/\alpha] \gens \Phi_1 \wedge \Phi_2, \Gamma'
}

\inferrule[AT-IAbs]
{
\Psi ; \Theta, i : S ; \Delta ; \Omega ; \Gamma \vdash e \checks \tau \gens \Phi, \Gamma'
}{
\Psi ; \Theta ; \Delta ; \Omega ; \Gamma \vdash \Lambda i. e \checks \forall i : S. \tau \gens \forall i : S. \Phi, \Gamma'
}

\inferrule[AT-IApp]
{
\Psi ; \Theta ; \Delta ; \Omega ; \Gamma \vdash e \infers \forall i : S.\tau \gens \Phi_1,\Gamma'\\
\Theta ; \Delta \vdash I : S \gens \Phi_2
}{
\Psi ; \Theta ; \Delta ; \Omega ; \Gamma \vdash e [I] \infers \tau[I/i] \gens \Phi_1 \wedge \Phi_2,\Gamma'
}

\inferrule[AT-Fix]
{
\Psi ; \Theta ; \Delta ; \Omega, x : \tau ; \cdot \vdash e \checks \tau \gens \Phi,\Gamma'
}{
\Psi ; \Theta ; \Delta ; \Omega ; \Gamma \vdash \texttt{fix } x.e \checks \tau \gens \Phi,\Gamma
} 

\inferrule[AT-CImpI]
{
\Psi ; \Theta ; \Delta, \Phi'; \Omega ; \Gamma \vdash e \checks \tau \gens \Phi,\Gamma'
}{
\Psi ; \Theta ; \Delta ; \Omega ; \Gamma \vdash \Lambda .e \checks (\Phi' \Rightarrow \tau) \gens (\Phi' \to \Phi),\Gamma'
}

\inferrule[AT-CImpE]
{
\Psi ; \Theta ; \Delta ; \Omega ; \Gamma \vdash e \infers (\Phi' \Rightarrow \tau) \gens \Phi,\Gamma'
}{
\Psi ; \Theta ; \Delta ; \Omega ; \Gamma \vdash e \{\} \infers \tau \gens \Phi \wedge \Phi',\Gamma'
}

\inferrule[AT-CAndI]
{
\Psi ; \Theta ; \Delta ; \Omega ; \Gamma \vdash e \checks \tau \gens \Phi,\Gamma'
}{
\Psi ; \Theta ; \Delta ; \Omega ; \Gamma \vdash <e> \checks \Phi' \amp \tau \gens \Phi \wedge \Phi',\Gamma'
}

\inferrule[AT-CAndE]
{
\Psi ; \Theta ; \Delta ; \Omega ; \Gamma \vdash e \infers \Phi' \amp \tau \gens \Phi_1,\Gamma_1\\
\Psi ; \Theta ; \Delta, \Phi' ; \Omega ; \Gamma_1, x : \tau \vdash e' \checks \tau' \gens \Phi_2, \Gamma_2\\
}{
\Psi ; \Theta ; \Delta ; \Omega ; \Gamma \vdash \texttt{clet } x = e \texttt{ in } e' \checks \tau' \gens \Phi_1 \wedge (\Phi' \to \Phi_2),\Gamma_2 \setminus \{x : \tau\}
}

\inferrule[AT-Ret]
{
\Psi ; \Theta ; \Delta ; \Omega ; \Gamma \vdash e \checks \tau \gens \Phi,\Gamma'
}{
\Psi ; \Theta ; \Delta ; \Omega ; \Gamma \vdash \texttt{ret } e \checks \M \, \phi(I,\vec{p}) \, \tau \gens \Phi, \Gamma'
}

\inferrule[AT-Bind]
{
\Psi ; \Theta ; \Delta ; \Omega ; \Gamma \vdash e_1 \infers \M \, \phi(J,\vec{p})\, \tau_1 \gens \Phi_1,\Gamma_1\\
\Psi ; \Theta; \Delta ; \Omega ; \Gamma_1, x:\tau_1 \vdash e_2 \checks \M \, \phi(I,\vec{q} - \vec{p})\, \tau_2 \gens \Phi_2,\Gamma_2\\
\Phi = (\vec{q} \geq \vec{p}) \wedge (I =J)  \wedge \Phi_1 \wedge \Phi_2
}{
\Psi ; \Theta ; \Delta ; \Omega ; \Gamma \vdash \texttt{bind } x = e_1 \texttt{ in } e_2 \checks \M \, \phi(I,\vec{q})\, \tau_2 \gens \Phi, \Gamma_2 \setminus \{x : \tau_1\}
}

\end{mathpar}

\begin{mathpar}

\inferrule[AT-Tick]
{
\Theta ; \Delta \vdash I : \N \gens \Phi_1\\
\Theta ; \Delta \vdash \vec{p} : \vec{\mathbb{R}^+} \gens \Phi_1\\
}{
\Psi ; \Theta ; \Delta ; \Omega ; \Gamma \vdash \texttt{tick}[I|\vec{p}] \infers \M \, \phi(I,\vec{p})\, 1 \gens \Phi_1 \wedge \Phi_2,\Gamma
}

\inferrule[AT-Release]
{
\Psi ; \Theta ; \Delta ; \Omega ; \Gamma \vdash e_1 \infers [J | \vec{q}] \tau_1 \gens \Phi_1,\Gamma_1\\
\Psi ; \Theta ; \Delta ; \Omega ; \Gamma_1, x : \tau \vdash e_2 \checks \M \, \phi(I,\vec{p} + \vec{q}) \, \tau_2 \gens \Phi_2, \Gamma_2
}{
\Psi ; \Theta ; \Delta ; \Omega ; \Gamma \vdash \texttt{release } x = e_1 \texttt{ in }e_2 \checks \M \, \phi(I,\vec{p}) \, \tau_2 \gens (I = J \wedge \Phi_1 \wedge \Phi_2), \Gamma_2 \setminus \{x\}
}



\inferrule[AT-Store]
{
\Theta ; \Delta \vdash K : \N \gens \Phi_1\\
\Theta ; \Delta \vdash \vec{w} : \vec{\mathbb{R}^+} \gens \Phi_2\\
\Psi ; \Theta ; \Delta ; \Omega ; \Gamma \vdash e \checks \tau \gens \Phi_3,\Gamma'\\
\Phi =  \Phi_1 \wedge \Phi_2 \wedge\Phi_3 \wedge  (\vec{p} \leq \vec{w} \leq \vec{q}) \wedge (I = J = K)
}{
\Psi ; \Theta ; \Delta ; \Omega ; \Gamma \vdash \texttt{store}[K|\vec{w}](e) \checks \M \, \phi(I,\vec{q}) \, ([J | \vec{p}] \, \tau) \gens \Phi, \Gamma'
}

\inferrule[AT-StoreConst]
{
\Psi ; \Theta ; \Delta ; \Omega ; \Gamma \vdash e \checks \tau \gens \Phi_1,\Gamma'\\
\Theta ; \Delta \vdash J : \mathbb{R} \gens \Phi_2\\
\Phi = (\texttt{const}(I) \leq \texttt{const}(J) \leq \vec{p}) \wedge \Phi_1 \wedge \Phi_2
}{
\Psi ; \Theta ; \Delta ; \Omega ; \Gamma \vdash \texttt{store}[J](e) \checks \M \, \phi(K,\vec{p}) \, ([I] \, \tau) \gens \Phi, \Gamma'
}

\inferrule[AT-ReleaseConst]
{
\Psi ; \Theta ; \Delta ; \Omega ; \Gamma \vdash e_1 \infers [J] \tau_1 \gens \Phi_1,\Gamma_1\\
\Psi ; \Theta ; \Delta ; \Omega ; \Gamma_1, x : \tau_1 \vdash e_2 \checks \M \, \phi(I,\vec{p} + \texttt{const}(J)) \, \tau_2 \gens \Phi_2, \Gamma_2
}{
\Psi ; \Theta ; \Delta ; \Omega ; \Gamma \vdash \texttt{release } x = e_1 \texttt{ in }e_2 \checks \M \, \phi(I,\vec{p}) \, \tau_2 \gens \Phi_1 \wedge \Phi_2, \Gamma_2 \setminus \{x\}
}

\inferrule[AT-Shift]
{
\Psi ; \Theta ; \Delta  ; \Omega ; \Gamma \vdash e \checks \M \, \phi(I - 1,\lhd \vec{q}) \, \tau \gens \Phi, \Gamma'
}{
\Psi ; \Theta ; \Delta ; \Omega ; \Gamma \vdash \texttt{shift}(e) \checks \M \, \phi(I,\vec{q}) \, \tau \gens (I \geq 1) \wedge \Phi, \Gamma'
}



\inferrule[AT-Sub]
{
\Psi ; \Theta ; \Delta ; \Omega ; \Gamma \vdash e \infers \tau' \gens \Phi_1,\Gamma'\\
\Psi;\Theta;\Delta \vdash \tau' \subty \tau : \star \gens \Phi_2
}{
\Psi ; \Theta ; \Delta ; \Omega ; \Gamma \vdash e \checks \tau \gens \Phi_1 \wedge \Phi_2,\Gamma'
}

\inferrule[AT-Anno]
{
\Psi ; \Theta ; \Delta ; \Omega ; \Gamma \vdash e \checks \tau \gens \Phi,\Gamma'
}{
\Psi ; \Theta ; \Delta ; \Omega ; \Gamma \vdash (e : \tau) \infers \tau \gens \Phi,\Gamma'
}
\end{mathpar}


\section{Soundness and Completeness}

\begin{theorem}[Raw Soundness of Sort Checking/Inference]
If $\Theta;\Delta \vdash I : S \gens \Phi$ and $\Theta;\Delta \vDash \Phi$, then $\Theta;\Delta \vdash I : S$ 
\label{thm:raw-sort-sound}
\end{theorem}
\begin{proof}
By induction on $\Theta;\Delta \vdash I : S \gens \Phi$.
\begin{enumerate}
  \item[AI-Var] Immediate.
  \item[AI-Plus] Suppose $\Theta ;\Delta \vdash I + J : bS \gens \Phi_1 \wedge \Phi_2$ and $\Theta ; \Delta \vdash I : bS \gens \Phi_1$, $\Theta ; \Delta \vdash J : bS \gens \Phi_2$. Since $\Theta ; \Delta \vDash \Phi_1 \wedge \Phi_2$, $\Theta ; \Delta \vDash \Phi_i$ for $i=1,2$. Then by IH, $\Theta ; \Delta \vdash I : bS$ and $\Theta ; \Delta \vdash J : bS$, and so by I-Plus, $\Theta ; \Delta \vdash I + J : bS$, as required.
  \item[AI-Minus] Suppose $\Theta ;\Delta \vdash I - J : bS \gens \Phi_1 \wedge \Phi_2 \wedge (I \geq J)$ and $\Theta ; \Delta \vdash I : bS \gens \Phi_1$, $\Theta ; \Delta \vdash J : bS \gens \Phi_2$. Since $\Theta ; \Delta \vDash \Phi_i$ for $i=1,2$, we have by IH that $\Theta ; \Delta \vdash I : bS$ and $\Theta ; \Delta \vdash J : bS$. Further, $\Theta ; \Delta \vDash I \geq J$, and so by I-Minus, $\Theta ;\Delta \vdash I - J : bS$, as required.
  \item[AI-Times-*] Immediate by IH.
  \item[AI-Shift] Immediate by IH.
  \item[AI-Lam] Suppose $\Theta ; \Delta \vdash \lambda i : bS. I : bS \to S \gens \Phi$ by way of $\Theta , i : bS ; \Delta \vdash i : S \gens \Phi$, and $\Theta ; \Delta \vDash \Phi$. By IH, $\Theta, i : bS ; \Delta \vdash I : S$, and so by I-Lam, $\Theta ; \Delta \vdash \lambda i : bS. I : bS \to S$.
  \item[AI-App] Suppose $\Theta ;\Delta \vdash I\; J : S \gens \Phi_1 \wedge \Phi_2$ by way of $\Theta ; \Delta \vdash I : bS \to S \gens \Phi_1$, and $\Theta ; \Delta \vDash J ; bS \gens \Phi_i$ with $i=1,2$. By IH, $\Theta ; \Delta \vdash I : bS \to S$, and $\Theta ; \Delta \vdash J : bS$. Then, by I-App, $\Theta ; \Delta \vdash I \; J : S$, as required.
  \item[AI-Sum] Suppose $\Theta;\Delta \vdash \sum_{i=I_0}^{I_1} J : bS \gens \Phi_1 \wedge \Phi_2 \wedge \forall i : \N.(I_0 \leq i \leq I_1 \to \Phi_3)$ by way of $\Theta;\Delta \vdash I_0 : \mathbb{N} \gens \Phi_1$, $\Theta;\Delta \vdash I_1 : \mathbb{N} \gens \Phi_2$, and $\Theta,i : \N;\Delta, I_0 \leq i \leq I_1 \vdash J : bS \gens \Phi_3$, and 
  $\Theta ; \Delta \vDash \Phi_1$,
  $\Theta ; \Delta \vDash \Phi_2$, and 
  $\Theta ; \Delta \vDash \forall i : \N.(I_0 \leq i \leq I_1 \to \Phi_3)$. By using the IH twice, $\Theta ; \Delta \vdash I_i : \N$ for $i=1,2$. We note that   $\Theta ; \Delta \vDash \forall i : \N.(I_0 \leq i \leq I_1 \to \Phi_3)$ is equivalent to $\Theta, i : \N ; \Delta, I_0 \leq i \leq I_1 \vDash \Phi_3$, and so by IH, $\Theta,i : \N;\Delta, I_0 \leq i \leq I_1 \vdash J : bS$. Applying I-Sum, we have $\Theta;\Delta \vdash \sum_{i=I_0}^{I_1} J : bS$, as required.
  \item[AI-*-Lit] Immediate.
\end{enumerate}
\end{proof}

\begin{theorem}[Raw Soundness of Constraint Well-Formedness]
If $\Theta ; \Delta \vdash \Phi \; \texttt{wf} \gens \Phi'$ and $\Theta ; \Delta \vDash \Phi'$ then $\Theta ; \Delta \vdash \Phi \texttt{ wf}$
\label{thm:raw-constr-sound}
\end{theorem}
\jtheorem{Proof of \autoref{thm:raw-constr-sound}}{
  \jgivengoal{
    \caseFact{1} $\Theta ; \Delta \vdash \Phi \; \texttt{wf} \gens \Phi'$
    
    \caseFact{2} $\Theta ; \Delta \vDash \Phi'$  
  }{
    $\Theta ; \Delta \vdash \Phi \; \texttt{wf}$  
  }
  
  \caseText{By induction on $\Theta ; \Delta \vdash \Phi \; \texttt{wf} \gens \Phi'$}
  
  \jcase{1}{AC-Top}{Immediate.}
  
  \jcase{2}{AC-Bot}{Immediate.}
  
  \jcase{3}{AC-Conj}{
   \jgivengoal{
     \caseFact{1} $\Theta ; \Delta \vdash \Phi_1 \wedge \Phi_2 \; \texttt{wf} \gens \Phi_1' \wedge \Phi_2'$
     
     \caseFact{2} $\Theta ; \Delta \vDash \Phi_1' \wedge \Phi_2'$
     
     \caseFact{3} $\Theta ; \Delta \vdash \Phi_1 \; \texttt{wf} \gens \Phi_1'$
     
     \caseFact{4} $\Theta ; \Delta \vdash \Phi_2 \; \texttt{wf} \gens \Phi_2'$
   }{
     $\Theta ; \Delta \vdash \Phi_1 \wedge \Phi_2 \; \texttt{wf}$   
   }
   
   \caseText{By IH on (3)}
   
   \caseFact{5} $\Theta ; \Delta \vdash \Phi_1 \; \texttt{wf}$
   
   \caseText{By IH on (4)}
   
   \caseFact{6} $\Theta ; \Delta \vdash \Phi_2 \; \texttt{wf}$
   
   \caseText{Goal follows by C-Conj on (5) and (6)}
  }
  
  \jcase{4}{AC-Disj}{
   \jgivengoal{
     \caseFact{1} $\Theta ; \Delta \vdash \Phi_1 \vee \Phi_2 \; \texttt{wf} \gens \Phi_1' \wedge \Phi_2'$
     
     \caseFact{2} $\Theta ; \Delta \vDash \Phi_1' \wedge \Phi_2'$
     
     \caseFact{3} $\Theta ; \Delta \vdash \Phi_1 \; \texttt{wf} \gens \Phi_1'$
     
     \caseFact{4} $\Theta ; \Delta \vdash \Phi_2 \; \texttt{wf} \gens \Phi_2'$
   }{
     $\Theta ; \Delta \vdash \Phi_1 \vee \Phi_2 \; \texttt{wf}$   
   }
   
   \caseText{By IH on (3)}
   
   \caseFact{5} $\Theta ; \Delta \vdash \Phi_1 \; \texttt{wf}$
   
   \caseText{By IH on (4)}
   
   \caseFact{6} $\Theta ; \Delta \vdash \Phi_2 \; \texttt{wf}$
   
   \caseText{Goal follows by C-Disj on (5) and (6)}
  }
  
  \jcase{5}{AC-Impl}{
   \jgivengoal{
     \caseFact{1} $\Theta ; \Delta \vdash \Phi_1 \to \Phi_2 \; \texttt{wf} \gens \Phi_1' \wedge (\Phi_1 \to \Phi_2')$
     
     \caseFact{2} $\Theta ; \Delta \vDash \Phi_1' \wedge (\Phi_1 \to \Phi_2')$
     
     \caseFact{3} $\Theta ; \Delta \vdash \Phi_1 \; \texttt{wf} \gens \Phi_1'$
     
     \caseFact{4} $\Theta ; \Delta, \Phi_1 \vdash \Phi_2 \; \texttt{wf} \gens \Phi_2'$
   }{
     $\Theta ; \Delta \vdash \Phi_1 \to \Phi_2 \; \texttt{wf}$   
   }
   
   \caseText{By IH on (3)}
   
   \caseFact{5} $\Theta ; \Delta \vdash \Phi_1 \; \texttt{wf}$
   
   \caseText{From (2)}
   
   \caseFact{6} $\Theta ; \Delta, \Phi_1 \vDash \Phi_2'$
   
   \caseText{By IH on (4) with (6)}
   
   \caseFact{7} $\Theta ; \Delta, \Phi_1 \vdash \Phi_2 \; \texttt{wf}$
   
   \caseText{Goal follows by C-Impl on (5) and (7)}
  }
  
  \jcase{6}{AC-Forall}{
    \jgivengoal{
      \caseFact{1} $\Theta ; \Delta \vdash \forall i : S. \Phi \; \texttt{wf} \gens \forall i : S. \Phi'$
      
      \caseFact{2} $\Theta ; \Delta \vDash \forall i : S. \Phi'$
      
      \caseFact{3} $\Theta, i : S; \Delta \vdash \Phi \; \texttt{wf} \gens \Phi'$
    }{
      $\Theta ; \Delta \vdash \forall i : S. \Phi \; \texttt{wf}$    
    }
    
    \caseText{Equivalently to (2)}
    
    \caseFact{4} $\Theta, i : S ; \Delta \vDash \Phi'$
    
    \caseText{By IH on (3) with (4)}
    
    \caseFact{5} $\Theta, i : S; \Delta \vdash \Phi \; \texttt{wf}$
    
    \caseText{Goal follows from C-Forall on (5)} 
  }
  
  \jcase{7}{AC-Exists}{
    \jgivengoal{
      \caseFact{1} $\Theta ; \Delta \vdash \exists i : S. \Phi \; \texttt{wf} \gens \forall i : S. \Phi'$
      
      \caseFact{2} $\Theta ; \Delta \vDash \exists i : S. \Phi'$
      
      \caseFact{3} $\Theta, i : S; \Delta \vdash \Phi \; \texttt{wf} \gens \Phi'$
    }{
      $\Theta ; \Delta \vdash \exists i : S. \Phi \; \texttt{wf}$    
    }
    
    \caseText{Equivalently to (2)}
    
    \caseFact{4} $\Theta, i : S ; \Delta \vDash \Phi'$
    
    \caseText{By IH on (3) with (4)}
    
    \caseFact{5} $\Theta, i : S; \Delta \vdash \Phi \; \texttt{wf}$
    
    \caseText{Goal follows from C-Exists on (5)} 
  }
  
  \jcase{8}{AC-Eq}{
    \jgivengoal{
     \caseFact{1} $\Theta ; \Delta \vdash I = J \; \texttt{wf} \gens \Phi_1 \wedge \Phi_2$    
     
     \caseFact{2} $\Theta ; \Delta \vDash \Phi_1 \wedge \Phi_2$
     
     \caseFact{3} $\Theta ; \Delta \vdash I : bS \gens \Phi_1$
     
     \caseFact{4} $\Theta ; \Delta \vdash J : bS \gens \Phi_2$
    }{
      $\Theta ; \Delta \vdash I = J \; \texttt{wf}$
    }
    \caseText{By \autoref{thm:raw-sort-sound} on (3)}
      
    \caseFact{5} $\Theta ; \Delta \vdash I : bS$
      
    \caseText{By \autoref{thm:raw-sort-sound} on (4)}
      
    \caseFact{6} $\Theta ; \Delta \vdash J : bS$
      
    \caseText{Goal follows by C-Eq on (5) and (6)}
  }
  
  \jcase{9}{AC-Leq}{Identical to Case 8}
  
  \jcase{10}{AC-Lt}{Identical to Case 8}

}

\iffalse
\begin{proof}
By induction on $\Theta ; \Delta \vdash \Phi \gens \Phi'$
\begin{itemize}
 \item[AC-Eq] Suppose $\Theta ; \Delta \vdash I = J \gens \Phi_1 \wedge \Phi_2$ from $\Theta ; \Delta \vdash I : bS \gens \Phi_1$ and  $\Theta ; \Delta \vdash J : bS \gens \Phi_2$ with $\Theta ; \Delta \vDash \Phi_1 \wedge \Phi_2$. By \textbf{Soundness of Sort Checking}, $\Theta ; \Delta \vdash I : bS$ and $\Theta ; \Delta \vdash J : bS$, and so by C-Eq, $\Theta ; \Delta \vdash I = J$.
 \item[AC-Leq] Suppose $\Theta ; \Delta \vdash I \leq J \gens \Phi_1 \wedge \Phi_2$ from $\Theta ; \Delta \vdash I : bS \gens \Phi_1$ and  $\Theta ; \Delta \vdash J : bS \gens \Phi_2$ with $\Theta ; \Delta \vDash \Phi_1 \wedge \Phi_2$. By \textbf{Soundness of Sort Checking}, $\Theta ; \Delta \vdash I : bS$ and $\Theta ; \Delta \vdash J : bS$, and so by C-Leq, $\Theta ; \Delta \vdash I \leq J$.
 \item[AC-Lt] Suppose $\Theta ; \Delta \vdash I < J \gens \Phi_1 \wedge \Phi_2$ from $\Theta ; \Delta \vdash I : bS \gens \Phi_1$ and  $\Theta ; \Delta \vdash J : bS \gens \Phi_2$ with $\Theta ; \Delta \vDash \Phi_1 \wedge \Phi_2$. By \textbf{Soundness of Sort Checking}, $\Theta ; \Delta \vdash I : bS$ and $\Theta ; \Delta \vdash J : bS$, and so by C-Lt, $\Theta ; \Delta \vdash I < J$.
\end{itemize}
\end{proof}

\fi

\idxctxwfsound*
\begin{proof}
By induction on $\Theta \vdash \Delta \; \texttt{wf} \gens \Phi$. The base case is immediate. Suppose $\Theta \vdash \Delta, \Phi \; \texttt{wf} \gens \Phi_1 \wedge (\bigwedge \Delta \to \Phi_2)$ with $\Theta ; \cdot \vDash \Phi_1 \wedge (\bigwedge \Delta \to \Phi_2)$, by way of
$\Theta \vdash \Delta \; \texttt{wf} \gens \Phi_1$ and $\Theta ; \Delta \vdash \Phi \; \texttt{wf} \gens \Phi_2$.
Since $\Theta ; \cdot \vDash \Phi_1$, we have by IH that $\Theta \vdash \Delta \; \texttt{wf}$. By Theorem~\ref{thm:raw-sort-sound}, since $\Theta ; \Delta \vDash \Phi_2$, we have that $\Theta ; \Delta \vdash \Phi_2 \; \texttt{wf}$, and so $\Theta \vdash \Delta, \Phi \; \texttt{wf}$, as required.
\end{proof}

\sortsound*
\begin{proof}
Immediate by Theorem~\ref{thm:raw-sort-sound} and Theorem~\ref{thm:idx-ctx-wf-sound}
\end{proof}


\constrsound*
\begin{proof}
Immediate by Theorem~\ref{thm:raw-constr-sound} and Theorem~\ref{thm:idx-ctx-wf-sound}
\end{proof}

\begin{theorem}[Raw Soundness of Kind Checking/Inference]
If $\Psi ; \Theta ; \Delta \vdash \tau : K \gens \Phi$ and $\Theta ; \Delta \vDash \Phi$ then $\Psi ; \Theta ; \Delta \vdash \tau : K$.
\label{thm:raw-kind-sound}
\end{theorem}
\begin{proof}
By induction on $\Psi ; \Theta ; \Delta \vdash \tau : K \gens \Phi$
\begin{itemize}
  \item[AK-Var] Immediate.
  \item[AK-Unit] Immediate.
  \item[AK-Arr] Suppose $\Psi ; \Theta ; \Delta \vdash \tau_1 \loli \tau_2 : \star \gens \Phi_1 \wedge \Phi_2$ from $\Psi ; \Theta ; \Delta \vdash \tau_1 : \star \gens \Phi_1$ and  $\Psi ; \Theta ; \Delta \vdash \tau_2 : \star \gens \Phi_2$ with $\Theta ; \Delta \vDash \Phi_1 \wedge \Phi_2$. By IH, $\Psi ; \Theta ; \Delta \vdash \tau_1 : \star$ and $\Psi ; \Theta ; \Delta \vdash \tau_2 : \star$. By K-Arr, $\Psi ; \Theta ; \Delta \vdash \tau_1 \loli \tau_2 : \star$.
  \item[AK-Tensor]  Suppose $\Psi ; \Theta ; \Delta \vdash \tau_1 \otimes \tau_2 : \star \gens \Phi_1 \wedge \Phi_2$ from $\Psi ; \Theta ; \Delta \vdash \tau_1 : \star \gens \Phi_1$ and  $\Psi ; \Theta ; \Delta \vdash \tau_2 : \star \gens \Phi_2$ with $\Theta ; \Delta \vDash \Phi_1 \wedge \Phi_2$. By IH, $\Psi ; \Theta ; \Delta \vdash \tau_1 : \star$ and $\Psi ; \Theta ; \Delta \vdash \tau_2 : \star$. By K-Tensor, $\Psi ; \Theta ; \Delta \vdash \tau_1 \otimes \tau_2 : \star$.
  \item[AK-With] Suppose $\Psi ; \Theta ; \Delta \vdash \tau_1 \amp \tau_2 : \star \gens \Phi_1 \wedge \Phi_2$ from $\Psi ; \Theta ; \Delta \vdash \tau_1 : \star \gens \Phi_1$ and  $\Psi ; \Theta ; \Delta \vdash \tau_2 : \star \gens \Phi_2$ with $\Theta ; \Delta \vDash \Phi_1 \wedge \Phi_2$. By IH, $\Psi ; \Theta ; \Delta \vdash \tau_1 : \star$ and $\Psi ; \Theta ; \Delta \vdash \tau_2 : \star$. By K-With, $\Psi ; \Theta ; \Delta \vdash \tau_1 \amp \tau_2 : \star$.
  \item[AK-Sum] Suppose $\Psi ; \Theta ; \Delta \vdash \tau_1 \oplus \tau_2 : \star \gens \Phi_1 \wedge \Phi_2$ from $\Psi ; \Theta ; \Delta \vdash \tau_1 : \star \gens \Phi_1$ and  $\Psi ; \Theta ; \Delta \vdash \tau_2 : \star \gens \Phi_2$ with $\Theta ; \Delta \vDash \Phi_1 \wedge \Phi_2$. By IH, $\Psi ; \Theta ; \Delta \vdash \tau_1 : \star$ and $\Psi ; \Theta ; \Delta \vdash \tau_2 : \star$. By K-Sum, $\Psi ; \Theta ; \Delta \vdash \tau_1 \oplus \tau_2 : \star$.
  \item[AK-Bang] Suppose $\Psi ; \Theta ; \Delta \vdash !\tau : \star \gens \Phi$ from $\Psi ; \Theta ; \Delta \vdash \tau : \star \gens \Phi$ with $\Theta ; \Delta \vDash \Phi$. By IH, $\Psi ; \Theta ; \Delta \vdash \tau : \star$ and so by K-Bang,$\Psi ; \Theta ; \Delta \vdash !\tau : \star$ as required.
  \item[AK-IForall] Suppose $\Psi ; \Theta ; \Delta \vdash \forall i : S. \tau : \star \gens \forall i : S. \Phi$ from $\Psi ; \Theta, i : S ; \Delta \vdash \tau : \star \gens \Phi$ with $\Theta ; \Delta \vDash \forall i : S. \Phi$. Equivalently, $\Theta, i : S ; \Delta \vDash \Phi$. By IH, $Psi ; \Theta, i : S ; \Delta \vdash \tau : \star$, and so by K-IForall, $\Psi ; \Theta ; \Delta \vdash \forall i : S. \tau : \star$.
  \item[AK-IExists] Suppose $\Psi ; \Theta ; \Delta \vdash \exists i : S. \tau : \star \gens \forall i : S. \Phi$ from $\Psi ; \Theta, i : S ; \Delta \vdash \tau : \star \gens \Phi$ with $\Theta ; \Delta \vDash \forall i : S. \Phi$. Equivalently, $\Theta, i : S ; \Delta \vDash \Phi$. By IH, $Psi ; \Theta, i : S ; \Delta \vdash \tau : \star$, and so by K-IExists, $\Psi ; \Theta ; \Delta \vdash \exists i : S. \tau : \star$.
  \item[AK-TForall] Suppose $\Psi ; \Theta ; \Delta \vdash \forall \alpha : K. \tau : \star \gens \Phi$ from $\Psi, \alpha : K ; \Theta ; \Delta \vdash \tau : \star \gens \Phi$ with $\Theta ; \Delta \vDash \Phi$. By IH, $\Psi, \alpha : K ; \Theta ; \Delta \vdash \tau : \star$. By K-TForall, $\Psi ; \Theta ; \Delta \vdash \forall \alpha : K. \tau : \star$ as required.
  \item[AK-List] Suppose $\Psi ; \Theta ; \Delta \vdash L^I \tau : \star \gens \Phi_1 \wedge \Phi_2$ from $\Theta ; \Delta \vdash I : \mathbb{N} \gens \Phi_1$ and $\Psi ; \Theta ; \Delta \vdash \tau : \star \gens \Phi_2$ with $\Theta ; \Delta \vDash \Phi_1 \wedge \Phi_2$. By \textbf{soundness of sort checking}, $\Theta ; \Delta \vdash I : \mathbb{N}$. By IH, $\Psi ; \Theta ; \Delta \vdash \tau : \star$. By K-List, $\Psi ; \Theta ; \Delta \vdash L^I \tau : \star$ as required.
  \item[AK-Conj] Suppose $\Psi ; \Theta ; \Delta \vdash \Phi \amp \tau : \star \gens \Phi_1 \wedge \Phi_2$ from $\Theta ; \Delta \vdash \Phi \gens \Phi_1$ and $\Psi ; \Theta ; \Delta \vdash \tau : \star \gens \Phi_2$ with $\Theta ; \Delta \vDash \Phi_1 \wedge \Phi_2$. By \textbf{soundness of constraint-wellformedness}, $\Theta ; \Delta \vdash \Phi \texttt{ wf}$. By IH, $\Psi ; \Theta ; \Delta \vdash \tau : \star$, and so by K-Conj, $\Psi ; \Theta ; \Delta \vdash \Phi \amp \tau : \star$ as required.
  \item[AK-Impl] Suppose $\Psi ; \Theta ; \Delta \vdash \Phi \implies \tau : \star \gens \Phi_1 \wedge (\Phi \to \Phi_2)$ from $\Theta ; \Delta \vdash \Phi \gens \Phi_1$ and $\Psi ; \Theta ; \Delta, \Phi \vdash \tau : \star \gens \Phi_2$ with $\Theta ; \Delta \vDash \Phi_1 \wedge (\Phi \to \Phi_2)$. By \textbf{soundness of constraint wellformedness}, $\Theta ; \Delta \vdash \Phi \texttt{ wf}$. Since $\Theta ; \Delta, \Phi \vDash \Phi_2$, we have by IH that $\Psi ; \Theta ; \Delta, \Phi \vdash \tau : \star$. By K-Impl, $\Psi ; \Theta ; \Delta \vdash \Phi \implies \tau : \star$, as required.
  \item[AK-Monad] Suppose $\Psi ; \Theta ; \Delta \vdash M(I,\vec{p}) \tau : \star \gens \Phi_1 \wedge \Phi_2 \wedge \Phi_3$ from
  \begin{itemize}
    \item $\Theta ; \Delta \vdash I : \mathbb{N} \gens \Phi_1$
    \item $\Theta ; \Delta \vdash \vec{p} : \vec{\mathbb{R}^+} \gens \Phi_2$
    \item $\Psi ; \Theta ; \Delta \vdash \tau : \star \gens \Phi_3$
  \end{itemize}
  with $\Theta ; \Delta \vDash \Phi_1 \wedge \Phi_2 \wedge \Phi_3$.
  By \textbf{soundness of sort checking} $\Theta ; \Delta \vdash I : \mathbb{N}$ and $\Theta ; \Delta \vdash \vec{p} : \vec{\mathbb{R}^+}$. By IH, $\Psi ; \Theta ; \Delta \vdash \tau : \star$. By K-Monad, $\Psi ; \Theta ; \Delta \vdash M(I,\vec{p}) \tau : \star$.
  \item[AK-Pot] Suppose $\Psi ; \Theta ; \Delta \vdash [I|\vec{p}] \tau : \star \gens \Phi_1 \wedge \Phi_2 \wedge \Phi_3$
  from \begin{itemize}
    \item $\Theta ; \Delta \vdash I : \mathbb{N} \gens \Phi_1$ 
    \item $\Theta ; \Delta \vdash \vec{p} : \vec{\mathbb{R}^+} \gens \Phi_2$
    \item $\Psi ; \Theta ; \Delta \vdash \tau : \star \gens \Phi_3$
  \end{itemize}
  with $\Theta ; \Delta \vDash \Phi_1 \wedge \Phi_2 \wedge \Phi_3$. By \textbf{soundness of sort checking}, $\Theta ; \Delta \vdash I : \mathbb{N}$ and $\Theta ; \Delta \vdash \vec{p} : \vec{\mathbb{R}^+}$. By IH, $\Psi ; \Theta ; \Delta \vdash \tau : \star$, and so by K-Pot, $\Psi ; \Theta ; \Delta \vdash [I|\vec{p}] \tau : \star$.
  \item[AK-ConstPot] Suppose $\Psi ; \Theta ; \Delta \vdash [I] \; \tau : \star \gens \Phi_1 \wedge \Phi_2$ from $\Theta ; \Delta \vdash I : \mathbb{R}^+ \gens \Phi_1$ and $\Psi ; \Theta ; \Delta \vdash \tau : \star \gens \Phi_2$ with $\Theta ; \Delta \vDash \Phi_1 \wedge \Phi_2$. By \textbf{soundness of sort checking}, $\Theta ; \Delta \vdash I : \mathbb{R}^+$. By IH, $\Psi ; \Theta ; \Delta \vdash \tau : \star $. By K-ConstPot, $\Psi ; \Theta ; \Delta \vdash [I] \; \tau : \star$.
  \item[AK-FamLam] Suppose $\Psi ; \Theta ; \Delta \vdash \lambda i : S. \tau : S \to K \gens \forall i : S. \Phi$ from $\Psi ; \Theta, i : S ; \Delta \vdash \tau : K \gens \Phi$ with $\Theta ; \Delta \vDash \forall i : S. \Phi$. Equivalently, $\Theta, i :S ; \Delta \vDash \Phi$, and so by IH, $\Psi ; \Theta, i : S ; \Delta \vdash \tau : K$. By K-FamLam, $\Psi ; \Theta ; \Delta \vdash \lambda i : S. \tau : S \to K$ as requried.
  \item[AK-FamApp] Suppose $\Psi ; \Theta ; \Delta \vdash \tau \; I : K \gens \Phi_1 \wedge \Phi_2$ from $\Theta ; \Delta \vdash I : S \gens \Phi_1$ and $ \Psi ; \Theta ; \Delta \vdash \tau : S \to K \gens \Phi_2$ with $\Theta ; \Delta \vDash \Phi_1 \wedge \Phi_2$. By \textbf{soundness of sort checking}, $\Theta ; \Delta \vdash I : S$ and by IH, $\Psi ; \Theta ; \Delta \vdash \tau : S \to K$. By K-FamApp, $\Psi ; \Theta ; \Delta \vdash \tau \; I : K$, as required
  
\end{itemize}
\end{proof}

\kindsound*
\begin{proof}
Immediate by Theorem~\ref{thm:raw-kind-sound} and Theorem~\ref{thm:idx-ctx-wf-sound}
\end{proof}

\begin{theorem}[Raw Soundness of Subtyping for Normal Forms]
If $\Psi ; \Theta ; \Delta \vdash \tau_1 \subtynf \tau_2 : K \gens \Phi$ and $\Theta ; \Delta \vDash \Phi$ then $\Psi ; \Theta ; \Delta \vdash \tau_1 \subty \tau_2 : K$\label{thm:raw-subtynf-sound}
\end{theorem}
\jtheorem{Proof of \autoref{thm:raw-subtynf-sound}}{
  \jgivengoal{
    \caseFact{1} $\Psi ; \Theta ; \Delta \vdash \tau_1 \subtynf \tau_2 : K \gens \Phi$
    
    \caseFact{2} $\Theta ; \Delta \vDash \Phi$
  }{
   $\Psi ; \Theta ; \Delta \vdash \tau_1 \subty \tau_2 : K$  
  }
  
  \jcase{1}{AS-Unit}{Immediate.}
  \jcase{2}{AS-Var}{Immediate.}
  
  \jcase{3}{AS-Arr}{
   \jgivengoal{
     \caseFact{1} $\Psi ; \Theta ; \Delta \vdash \tau_1 \loli \tau_2 \subtynf \tau_1' \loli \tau_2' : \star \gens \Phi_1 \wedge \Phi_2$
     
     \caseFact{2} $\Theta ; \Delta \vDash \Phi_1 \wedge \Phi_2$   
     
     \caseFact{3} $\Psi ; \Theta ; \Delta \vdash \tau_1' \subtynf \tau_1 : \star \gens \Phi_1$
     
     \caseFact{4} $\Psi ; \Theta ; \Delta \vdash \tau_2 \subtynf \tau_2' : \star \gens \Phi_2$
   }{
     $\Psi ; \Theta ; \Delta \vdash \tau_1 \loli \tau_2 \subty \tau_1' \loli \tau_2' : \star$   
   }
   
   \caseText{By IH on (3)}
   
   \caseFact{5} $\Psi ; \Theta ; \Delta \vdash \tau_1' \subty \tau_1 : \star$

   \caseText{By IH on (4)}   
   
   \caseFact{6} $\Psi ; \Theta ; \Delta \vdash \tau_2 \subty \tau_2' : \star$
   
   \caseText{Goal follows by S-Arr on (5) and (6)}
  }
  
  \jcase{4}{AS-Tensor}{
   \jgivengoal{
     \caseFact{1} $\Psi ; \Theta ; \Delta \vdash \tau_1 \otimes \tau_2 \subtynf \tau_1' \otimes \tau_2' : \star \gens \Phi_1 \wedge \Phi_2$
     
     \caseFact{2} $\Theta ; \Delta \vDash \Phi_1 \wedge \Phi_2$   
     
     \caseFact{3} $\Psi ; \Theta ; \Delta \vdash \tau_1 \subtynf \tau_1' : \star \gens \Phi_1$
     
     \caseFact{4} $\Psi ; \Theta ; \Delta \vdash \tau_2 \subtynf \tau_2' : \star \gens \Phi_2$
   }{
     $\Psi ; \Theta ; \Delta \vdash \tau_1 \otimes \tau_2 \subty \tau_1' \otimes \tau_2' : \star$   
   }
   
   \caseText{By IH on (3)}
   
   \caseFact{5} $\Psi ; \Theta ; \Delta \vdash \tau_1' \subty \tau_1 : \star$

   \caseText{By IH on (4)}   
   
   \caseFact{6} $\Psi ; \Theta ; \Delta \vdash \tau_2 \subty \tau_2' : \star$
   
   \caseText{Goal follows by S-Tensor on (5) and (6)}
  }
  
  \jcase{5}{AS-With}{
   \jgivengoal{
     \caseFact{1} $\Psi ; \Theta ; \Delta \vdash \tau_1 \amp \tau_2 \subtynf \tau_1' \amp \tau_2' : \star \gens \Phi_1 \wedge \Phi_2$
     
     \caseFact{2} $\Theta ; \Delta \vDash \Phi_1 \wedge \Phi_2$   
     
     \caseFact{3} $\Psi ; \Theta ; \Delta \vdash \tau_1 \subtynf \tau_1' : \star \gens \Phi_1$
     
     \caseFact{4} $\Psi ; \Theta ; \Delta \vdash \tau_2 \subtynf \tau_2' : \star \gens \Phi_2$
   }{
     $\Psi ; \Theta ; \Delta \vdash \tau_1 \amp \tau_2 \subty \tau_1' \amp \tau_2' : \star$   
   }
   
   \caseText{By IH on (3)}
   
   \caseFact{5} $\Psi ; \Theta ; \Delta \vdash \tau_1' \subty \tau_1 : \star$

   \caseText{By IH on (4)}   
   
   \caseFact{6} $\Psi ; \Theta ; \Delta \vdash \tau_2 \subty \tau_2' : \star$
   
   \caseText{Goal follows by S-With on (5) and (6)}
  }
  
  \jcase{6}{AS-Sum}{
   \jgivengoal{
     \caseFact{1} $\Psi ; \Theta ; \Delta \vdash \tau_1 \oplus \tau_2 \subtynf \tau_1' \oplus \tau_2' : \star \gens \Phi_1 \wedge \Phi_2$
     
     \caseFact{2} $\Theta ; \Delta \vDash \Phi_1 \wedge \Phi_2$   
     
     \caseFact{3} $\Psi ; \Theta ; \Delta \vdash \tau_1 \subtynf \tau_1' : \star \gens \Phi_1$
     
     \caseFact{4} $\Psi ; \Theta ; \Delta \vdash \tau_2 \subtynf \tau_2' : \star \gens \Phi_2$
   }{
     $\Psi ; \Theta ; \Delta \vdash \tau_1 \oplus \tau_2 \subty \tau_1' \oplus \tau_2' : \star$   
   }
   
   \caseText{By IH on (3)}
   
   \caseFact{5} $\Psi ; \Theta ; \Delta \vdash \tau_1' \subty \tau_1 : \star$

   \caseText{By IH on (4)}   
   
   \caseFact{6} $\Psi ; \Theta ; \Delta \vdash \tau_2 \subty \tau_2' : \star$
   
   \caseText{Goal follows by S-Sum on (5) and (6)}
  }
  
  \jcase{7}{AS-Bang}{
   \jgivengoal{
     \caseFact{1} $\Psi ; \Theta ; \Delta \vdash !\tau_1 \subtynf !\tau_2 : \star \gens \Phi$
     
     \caseFact{2} $\Theta ; \Delta \vDash \Phi$
     
     \caseFact{3} $\Psi ; \Theta ; \Delta \vdash \tau_1 \subtynf \tau_2 : \star \gens \Phi$
    
   }{
     $\Psi ; \Theta ; \Delta \vdash !\tau_1 \subty !\tau_2 : \star$
   }
   
   \caseText{By IH on (3)}
   
   \caseFact{4} $\Psi ; \Theta ; \Delta \vdash \tau_1 \subty \tau_2 : \star$
   
   \caseText{Goal follows by S-Bang on (4)}
  }
  
  \jcase{8}{AS-IForall}{
   \jgivengoal{
     \caseFact{1} $\Psi ; \Theta ; \Delta \vdash \forall i : S.\tau_1 \subtynf \forall i : S. \tau_2 : \star \gens \forall i : S. \Phi$
     
     \caseFact{2} $\Theta ; \Delta \vDash \forall i : S. \Phi$
     
     \caseFact{3} $\Psi ; \Theta, i : S ; \Delta \vdash \tau_1 \subtynf \tau_2 : \star \gens \Phi$
    
   }{
     $\Psi ; \Theta ; \Delta \vdash \forall i : S. \tau_1 \subty \forall i : S. \tau_2 : \star$
   }
   
   \caseText{Equivalently to (2)}
   
   \caseFact{4} $\Theta, i : S; \Delta \vDash \Phi$
   
   \caseText{By IH on (3), using (4)}
   
   \caseFact{5} $\Psi ; \Theta ; \Delta \vdash \tau_1 \subty \tau_2 : \star$
   
   \caseText{Goal follows by S-IForall on (5)}
  }
  
  \jcase{9}{AS-IExists}{
   \jgivengoal{
     \caseFact{1} $\Psi ; \Theta ; \Delta \vdash \exists i : S.\tau_1 \subtynf \exists i : S. \tau_2 : \star \gens \forall i : S. \Phi$
     
     \caseFact{2} $\Theta ; \Delta \vDash \forall i : S. \Phi$
     
     \caseFact{3} $\Psi ; \Theta, i : S ; \Delta \vdash \tau_1 \subtynf \tau_2 : \star \gens \Phi$
    
   }{
     $\Psi ; \Theta ; \Delta \vdash \exists i : S. \tau_1 \subty \exists i : S. \tau_2 : \star$
   }
   
   \caseText{Equivalently to (2)}
   
   \caseFact{4} $\Theta, i : S; \Delta \vDash \Phi$
   
   \caseText{By IH on (3), using (4)}
   
   \caseFact{5} $\Psi ; \Theta ; \Delta \vdash \tau_1 \subty \tau_2 : \star$
   
   \caseText{Goal follows by S-IExists on (5)}
  }
  
  \jcase{10}{AS-TForall}{
   \jgivengoal{
     \caseFact{1} $\Psi ; \Theta ; \Delta \vdash \forall \alpha : K. \tau_1 \subtynf \forall \alpha : K. \tau_2 : \star \gens \Phi$
     
     \caseFact{2} $\Theta ; \Delta \vDash \Phi$
     
     \caseFact{3} $\Psi, \alpha : K ; \Theta ; \Delta \vdash \tau_1 \subtynf \tau_2 : \star \gens \Phi$
    
   }{
     $\Psi ; \Theta ; \Delta \vdash \forall \alpha : K. \tau_1 \subty \forall \alpha : K. \tau_2 : \star$
   }
   
   \caseText{By IH on (3)}
   
   \caseFact{4} $\Psi, \alpha : K ; \Theta ; \Delta \vdash \tau_1 \subty \tau_2 : \star$
   
   \caseText{Goal follows by S-TForall on (4)}
  }
  
  \jcase{11}{AS-List}{
   \jgivengoal{
     \caseFact{1} $\Psi ; \Theta ; \Delta \vdash L^I \tau_1 \subtynf L^J \tau_2 : \star \gens \Phi \wedge (I = J)$
     
     \caseFact{2} $\Theta ; \Delta \vDash \Phi \wedge (I = J)$
     
     \caseFact{3} $\Psi ; \Theta ; \Delta \vdash \tau_1 \subtynf \tau_2 : \star \gens \Phi$
   }{
     $\Psi ; \Theta ; \Delta \vdash L^I \tau_1 \subty L^J \tau_2 : \star$
   }
   
   \caseText{By IH on (3)}
   
   \caseFact{4} $\Psi ; \Theta ; \Delta \vdash \tau_1 \subty \tau_2 : \star \gens \Phi$
   
   \caseText{From (2)}
   
   \caseFact{5} $\Theta ; \Delta \vDash I = J$
   
   \caseText{Goal follows by S-List on (4) and (5)}
  }
  
  \jcase{12}{AS-Conj}{
   \jgivengoal{
     \caseFact{1} $\Psi ; \Theta ; \Delta \vdash \Phi_1 \amp \tau_1 \subtynf \Phi_2 \amp \tau_2 : \star \gens \Phi \wedge (\Phi_1 \to \Phi_2)$
     
     \caseFact{2} $\Theta ; \Delta \vDash \Phi \wedge (\Phi_1 \to \Phi_2)$
     
     \caseFact{3} $\Psi ; \Theta ; \Delta \vdash \tau_1 \subtynf \tau_2 : \star \gens \Phi$
   }{
     $\Psi ; \Theta ; \Delta \vdash \Phi_1 \amp \tau_1 \subty \Phi_2 \amp \tau_2 : \star$
   }
   
   \caseText{By IH on (3)}
   
   \caseFact{4} $\Psi ; \Theta ; \Delta \vdash \tau_1 \subty \tau_2 : \star \gens \Phi$
   
   \caseText{From (2)}
   
   \caseFact{5} $\Theta ; \Delta \vDash \Phi_1 \to \Phi_2$
   
   \caseText{Goal follows by S-Conj on (4) and (5)}
  }
  
  \jcase{13}{AS-Impl}{
   \jgivengoal{
     \caseFact{1} $\Psi ; \Theta ; \Delta \vdash \Phi_1 \implies \tau_1 \subtynf \Phi_2 \implies \tau_2 : \star \gens (\Phi_2 \to \Phi) \wedge (\Phi_2 \to \Phi_1)$
     
     \caseFact{2} $\Theta ; \Delta \vDash (\Phi_2 \to \Phi) \wedge (\Phi_2 \to \Phi_1)$
     
     \caseFact{3} $\Psi ; \Theta ; \Delta,\Phi_2 \vdash \tau_1 \subtynf \tau_2 : \star \gens \Phi$
   }{
     $\Psi ; \Theta ; \Delta \vdash \Phi_1 \implies \tau_1 \subty \Phi_2 \implies \tau_2 : \star$
   }
   
   \caseText{By IH on (3)}
   
   \caseFact{4} $\Psi ; \Theta ; \Delta, \Phi_2 \vdash \tau_1 \subty \tau_2 : \star \gens \Phi$
   
   \caseText{From (2)}
   
   \caseFact{5} $\Theta ; \Delta \vDash (\Phi_2 \to \Phi) \wedge (\Phi_2 \to \Phi_1)$
   
   \caseText{Goal follows by S-Impl on (4) and (5)}
  }
  
  \jcase{14}{AS-Monad}{
   \jgivengoal{
    \caseFact{1} $\Psi ; \Theta ; \Delta \vdash \M \, (I,\vec{p}) \, \tau_1 \subtynf \M \, (J,\vec{q}) \, \tau_2 : \star \gens \Phi \wedge (I = J) \wedge (\vec{p} \leq \vec{q})$
    
    \caseFact{2} $\Theta ; \Delta \vDash \Phi \wedge (I = J) \wedge (\vec{p} \leq \vec{q})$
    
    \caseFact{3} $\Psi ; \Theta ; \Delta \vdash \tau_1 \subtynf \tau_2 : \star \gens \Phi$
   }{
     $\Psi ; \Theta ; \Delta \vdash \M \, (I,\vec{p}) \, \tau_1 \subty \M \, (J,\vec{q}) \, \tau_2 : \star$
   }
   
   \caseText{By IH on (3)}
   
   \caseFact{4} $\Psi ; \Theta ; \Delta \vdash \tau_1 \subty \tau_2 : \star$
   
   \caseText{From (2)}
   
   \caseFact{5} $\Theta ; \Delta \vDash I = J$
   
   \caseFact{6} $\Theta ; \Delta \vDash \vec{p} \leq \vec{q}$
   
   \caseText{Goal follows by S-Monad on (4), (5), and (6)}
  }
  
  \jcase{15}{AS-Pot}{
   \jgivengoal{
    \caseFact{1} $\Psi ; \Theta ; \Delta \vdash [I|\vec{p}] \, \tau_1 \subtynf [J|\vec{q}] \, \tau_2 : \star \gens \Phi \wedge (I = J) \wedge (\vec{p} \geq \vec{q})$
    
    \caseFact{2} $\Theta ; \Delta \vDash \Phi \wedge (I = J) \wedge (\vec{p} \geq \vec{q})$
    
    \caseFact{3} $\Psi ; \Theta ; \Delta \vdash \tau_1 \subtynf \tau_2 : \star \gens \Phi$
   }{
     $\Psi ; \Theta ; \Delta \vdash [I|\vec{p}] \, \tau_1 \subty [J|\vec{q}] \, \tau_2 : \star$
   }
   
   \caseText{By IH on (3)}
   
   \caseFact{4} $\Psi ; \Theta ; \Delta \vdash \tau_1 \subty \tau_2 : \star$
   
   \caseText{From (2)}
   
   \caseFact{5} $\Theta ; \Delta \vDash I = J$
   
   \caseFact{6} $\Theta ; \Delta \vDash \vec{p} \geq \vec{q}$
   
   \caseText{Goal follows by S-Pot on (4), (5), and (6)}
  }
  
  \jcase{16}{AS-ConstPot}{
   \jgivengoal{
    \caseFact{1} $\Psi ; \Theta ; \Delta \vdash [I] \, \tau_1 \subtynf [J] \, \tau_2 : \star \gens \Phi \wedge (I \geq J)$
    
    \caseFact{2} $\Theta ; \Delta \vDash \Phi \wedge (I \geq J)$
    
    \caseFact{3} $\Psi ; \Theta ; \Delta \vdash \tau_1 \subtynf \tau_2 : \star \gens \Phi$
   }{
     $\Psi ; \Theta ; \Delta \vdash [I] \, \tau_1 \subty [J] \, \tau_2 : \star$
   }
   
   \caseText{By IH on (3)}
   
   \caseFact{4} $\Psi ; \Theta ; \Delta \vdash \tau_1 \subty \tau_2 : \star$
   
   \caseText{From (2)}
   
   \caseFact{5} $\Theta ; \Delta \vDash I \geq J$
   
   \caseText{Goal follows by S-Const on (4) and (5)}
  }
  
  \jcase{17}{AS-FamLam}{
   \jgivengoal{
    \caseFact{1} $\Psi ; \Theta ; \Delta \vdash \lambda i : S.\tau_1 \subtynf \lambda i :S .\tau_2 :S \to K \gens \forall i : S. \Phi$
    
    \caseFact{2} $\Theta ; \Delta \vDash \forall i : S. \Phi$
    
    \caseFact{3} $\Psi ; \Theta, i : S; \Delta \vdash \tau_1 \subtynf \tau_2 : K \gens \Phi$
   }{
     $\Psi ; \Theta ; \Delta \vdash \lambda i : S.\tau_1 \subty \lambda i :S .\tau_2 :S \to K$
   }
   \caseText{Equivalently to (2)}
   
   \caseFact{4} $\Theta, i : S ; \Delta \vDash \Phi$
   
   \caseText{By IH on (3), using (4)}
   
   \caseFact{5} $\Psi ; \Theta, i : S: \Delta \vdash \tau_1 \subty \tau_2 : K$
   
   \caseText{Goal follows by S-FamLam on (5)}
  }
  
  \jcase{18}{AS-FamApp}{
   \jgivengoal{
     \caseFact{1} $\Psi ; \Theta ; \Delta \vdash \tau_1 \; I \subtynf \tau_2 \; J : K \gens (I = J) \wedge \Phi$ 
     
     \caseFact{2} $\Theta ; \Delta \vDash (I = J) \wedge \Phi$
     
     \caseFact{3} $\Psi ; \Theta ; \Delta \vdash \tau_1 \subtynf \tau_2 : S \to K \gens \Phi$
   }{
     $\Psi ; \Theta ; \Delta \vdash \tau_1 \; I \subty \tau_2 \; J : K$
   }
   
   \caseText{By IH on (3)}
   
   \caseFact{4} $\Psi ; \Theta ; \Delta \vdash \tau_1 \subty \tau_2 ; S \to K$
   
   \caseText{From (2)}
   
   \caseFact{5} $\Theta ; \Delta \vDash I = J$
   
   \caseText{Goal follows by S-FamApp on (4) and (5).}
  }

}

\iffalse
\begin{proof}
 We proceed by induction on $\Psi ; \Theta ; \Delta \vdash \tau_1 \subtynf \tau_2 : K \gens \Phi$.

\begin{itemize}
  \item[AS-Monad] Suppose $\Psi ; \Theta ; \Delta \vdash M(I,\vec{q}) \tau_1 \subtynf M(J,\vec{p}) \tau_2 : \star \gens (I = J) \wedge (\vec{q} \leq \vec{p}) \wedge \Phi$ by way of $\Psi ; \Theta ; \Delta \vdash \tau_1 \subtynf \tau_2 : \star \gens \Phi$ with $\Theta ; \Delta \vDash (I = J)$,  $\Theta ; \Delta \vDash \vec{q} \leq \vec{p}$, and $\Theta ; \Delta \vDash \Phi$. By IH, $\Psi ; \Theta ; \Delta \vdash \tau_1 \subty \tau_2 : \star$, and since $\Theta ; \Delta \vDash (I = J)$ and $\Theta ; \Delta \vDash \vec{q} \leq \vec{p}$, we have by S-Monad that $\Psi ; \Theta ; \Delta \vdash M(I,\vec{q}) \tau_1 \subty M(J,\vec{p}) \tau_2 : \star$.
  \item[AS-Pot] Suppose $\Psi ; \Theta ; \Delta \vdash [I|\vec{q}] \tau_1 \subtynf [J|\vec{p}] \tau_2 : \star \gens (I = J) \wedge (\vec{p} \leq \vec{q}) \wedge \Phi$ by way of $\Psi ; \Theta ; \Delta \vdash \tau_1 \subtynf \tau_2 : \star \gens \Phi$, with  $\Theta ; \Delta \vDash (I = J)$,  $\Theta ; \Delta \vDash \vec{p} \leq \vec{q}$, and $\Theta ; \Delta \vDash \Phi$. By IH, $\Psi ; \Theta ; \Delta \vdash \tau_1 \subty \tau_2 : \star$. Using the fact that $\Theta ; \Delta \vDash (I = J)$ and $\Theta ; \Delta \vDash \vec{p} \leq \vec{q}$, we have by S-Pot that $\Psi ; \Theta ; \Delta \vdash [i|\vec{q}] \tau_1 \subty [j|\vec{p}] \tau_2 : \star$.
  \item[AS-ConstPot] Suppose ${\Psi ; \Theta ; \Delta \vdash [I] \tau_1 \subtynf [J] \tau_2 : \star \gens \Phi \wedge (J \leq I)}$ by way of ${\Psi ; \Theta ; \Delta \vdash \tau_1 \subtynf \tau_2 : \star \gens \Phi}$ with $\Theta ; \Delta \vDash \Phi$ and $\Theta ; \Delta \vDash J \leq I$. By IH, $\Psi ; \Theta ; \Delta \vdash \tau_1 \subty \tau_2 : \star$, and so using $\Theta ; \Delta \vDash J \leq I$, we have by S-ConstPot that $\Psi ; \Theta ; \Delta \vdash [I] \tau_1 \subty [J] \tau_2 : \star$.
  \item[AS-FamLam] Suppose $\Psi ; \Theta ; \Delta \vdash \lambda i : S. \tau_1 \subtynf \lambda i : S. \tau_2 : S \to K \gens \forall i : S. \Phi$ by way of ${\Psi ; \Theta, i : S ; \Delta \vdash \tau_1 \subtynf \tau_2 : K \gens \Phi}$, with $\Theta ; \Delta \vDash \forall i : S. \Phi$. Equivalently, $\Theta, i : S; \Delta \vDash \Phi$, and so by IH, $\Psi ; \Theta, i : S ; \Delta \vdash \tau_1 \subty \tau_2 : K$, which by S-FamLam means $\Psi ; \Theta ; \Delta \vdash \lambda i : S. \tau_1 \subty \lambda i : S. \tau_2 : S \to K $.
  \item[AS-FamApp] Suppose that $\Psi ; \Theta ; \Delta \vdash \tau_1 \; I \subtynf \tau_2 \; J : K \gens (I = J) \wedge \Phi$ by way of $\Psi ; \Theta ; \Delta \vdash \tau_1 \subtynf \tau_2 : S \to K \gens \Phi$ and $i,j : S  \in \Theta$, with $\Theta ; \Delta \vDash I = J$ and $\Theta ; \Delta \vDash \Phi$. By IH, $\Psi ; \Theta ; \Delta \vdash \tau_1 \subty \tau_2 : S \to K \gens \Phi$, and by S-FamApp, $\Psi ; \Theta ; \Delta \vdash \tau_1 \; I \subty \tau_2 \; J : K$.
\end{itemize}

\end{proof}

\fi

\subtynfsound*
\begin{proof}
Immediate by Theorem~\ref{thm:raw-subtynf-sound} and Theorem~\ref{thm:idx-ctx-wf-sound}
\end{proof}

\subtysound*
\begin{proof}
There is only one case: $\Psi ; \Theta ; \Delta \vdash \tau_1 \subty\tau_2 : K \gens \Phi$ by way of $\Psi ; \Theta ; \Delta \vdash \texttt{eval}(\tau_1) \subtynf \texttt{eval}(\tau_2) : K \gens \Phi$ with $\Theta ; \Delta \vDash \Phi$. By Theorem~\ref{thm:subtynf-sound}, $\Psi ; \Theta ; \Delta \vdash \texttt{eval}(\tau_1) \subty \texttt{eval}(\tau_2) : K$. By Theorem~\ref{thm:norm-thm} and two uses of S-Trans, $\Psi ; \Theta ; \Delta \pvdash \tau_1 \subty \tau_2 : K$, as required.
\end{proof}

\begin{theorem}[Output Context is Uniquely Determined]
For any $\Psi,\Theta,\Delta,\Gamma,e,\Phi$, there is at most one $\Gamma'$ so that $\Psi ; \Theta ; \Delta ; \Omega ; \Gamma \vdash e \updownarrow \tau \gens \Phi, \Gamma'$
\label{thm:ctx-uniquely-determined}
\end{theorem}
\begin{proof}
Inspection of rules.
\end{proof}

\lsc*
\begin{proof}
Induction on $\Psi ; \Theta ; \Delta ; \Omega ; \Gamma \vdash e \updownarrow \tau \gens \Phi, \Gamma'$.
\end{proof}

\begin{theorem}[Output Context is Well-Formed]
If $\Psi ; \Theta ; \Delta ; \Omega ; \Gamma \pvdash e \updownarrow \tau \gens \Phi,\Gamma'$ then $\Psi ; \Theta ; \Delta \vdash \Gamma' \; \texttt{wf} \gens \Phi'$ with $\Theta ; \Delta \vDash \Phi'$.
\end{theorem}
\begin{proof}
An easy induction on $\Gamma'$, using the fact that $\Psi ; \Theta ; \Delta \vdash \Gamma \; \texttt{wf} \gens \Phi''$ with $\Theta ; \Delta \vDash \Phi''$, and Theorem~\ref{thm:lsc}
\end{proof}

\begin{theorem}[Algorithmic Well-Formedness of Context Operations]
If $\Gamma_1,\Gamma_2 \subseteq \Gamma$ such that $\Psi ; \Theta ; \Delta \vdash \Gamma \; \texttt{wf}$, then the following are true:
\begin{enumerate}
  \item $\Psi ; \Theta ; \Delta \vdash \Gamma_1,\Gamma_2 \; \texttt{wf}$
  \item $\Psi ; \Theta ; \Delta \vdash \Gamma_1 \cap \Gamma_2 \; \texttt{wf}$
  \item $\Psi ; \Theta ; \Delta \vdash \Gamma_1\setminus\Gamma_2 \; \texttt{wf}$
\end{enumerate}
\end{theorem}

\theorem[Raw Soundness of Type Checking/Inference]{
~\begin{enumerate}
 \item If $\Psi;\Theta;\Delta;\Omega;\Gamma \vdash e \checks \tau \gens \Phi, \Gamma'$ and $\Theta;\Delta \vDash \Phi$ then $\Psi;\Theta;\Delta;\Omega;\Gamma \setminus \Gamma' \vdash |e| : \tau$
 \item If $\Psi;\Theta;\Delta;\Omega;\Gamma \vdash e \infers \tau \gens \Phi, \Gamma'$ and $\Theta;\Delta \vDash \Phi$ then $\Psi;\Theta;\Delta;\Omega;\Gamma \setminus \Gamma' \vdash |e| : \tau$
\end{enumerate}
}
%\textbf{Redo in new style}
\label{thm:raw-tycheck-sound}
\jtheorem{Proof of \autoref{thm:raw-tycheck-sound}}{
\caseText{We prove both claims simultaneously by induction over $\Psi;\Theta;\Delta;\Omega;\Gamma \vdash e \checks \tau \gens \Phi, \Gamma'$ and $\Psi;\Theta;\Delta;\Omega;\Gamma \vdash e \infers \tau \gens \Phi, \Gamma'$.
For brevity, we will often silently invoke \autoref{thm:lsc} and \autoref{thm:ctx-sub-subset-2} silently to build context weakening judgments for T-Weaken.}

 \jcase{1}{AT-Var-1}{
   \jgivengoal{
     \caseFact{1} $\Psi ; \Theta ; \Delta ; \Omega ; \Gamma \vdash x \infers \tau \gens \top,\Gamma \setminus \{x : \tau\}$
     
     \caseFact{2} $\Theta ; \Delta \vDash \top$
     
     \caseFact{3} $x : \tau \in \Gamma$
   }{
     $\Psi ; \Theta ; \Delta ; \Omega ; x : \tau \vdash x : \tau$
   }
   
   \caseText{Immediate by T-Var-1 on (3)}
 }
 
 \jcase{2}{AT-Var-2}{
   \jgivengoal{
     \caseFact{1} $\Psi ; \Theta ; \Delta ; \Omega ; \Gamma \vdash x \infers \tau \gens \top,\Gamma$
     
     \caseFact{2} $\Theta ; \Delta \vDash \top$
     
     \caseFact{3} $x : \tau \in \Omega$
   }{
     $\Psi ; \Theta ; \Delta ; \Omega ; \cdot \vdash x : \tau$
   }
   
   \caseText{Immediate by T-Var-2 on (3)}
 }
 
 \jcase{3}{AT-Unit}{Immediate.}
 
 \jcase{4}{AT-Base}{Immediate.}
 
 \jcase{5}{AT-Absurd}{Immediate.}
 
 \jcase{6}{AT-Nil}{
   \jgivengoal{
     \caseFact{1} $\Psi ; \Theta ; \Delta ; \Omega ; \Gamma \vdash \texttt{nil} \checks L^I \tau \gens I = 0,\Gamma$
     
     \caseFact{2} $\Theta ; \Delta \vDash I = 0$
   }{
     $\Psi ; \Theta ; \Delta ; \cdot \vdash \texttt{nil} : L^I \tau$
   }
   
   \caseText{By T-Nil}
   
   \caseFact{3} $\Psi ; \Theta ; \Delta ; \cdot \vdash \texttt{nil} : L^0 \tau$
   
   \caseText{By S-List, S-Refl and (2)}
   
   \caseFact{4} $\Psi ; \Theta ; \Delta \vdash L^0 \tau \subty L^I \tau : \star$
   
   \caseText{Goal follows by T-Sub on (3) and (4)}
 }
 
 \jcase{7}{AT-Cons}{
   \jgivengoal{
     \caseFact{1} $\Psi ; \Theta ; \Delta ; \Omega ; \Gamma \vdash e_1 :: e_2 \checks L^I \tau \gens (I \geq 1) \wedge \Phi_1 \wedge \Phi_2,\Gamma_2$
     
     \caseFact{2} $\Theta ; \Delta \vDash (I \geq 1) \wedge \Phi_1 \wedge \Phi_2$
     
     \caseFact{3} $\Psi ; \Theta ; \Delta ; \Omega ; \Gamma \vdash e_1 \checks \tau \gens \Phi_1,\Gamma_1$
     
     \caseFact{4} $\Psi ; \Theta ; \Delta ; \Omega ; \Gamma_1 \vdash e_2 \checks L^{I-1} \tau \gens \Phi_2,\Gamma_2$
   }{
     $\Psi ; \Theta ; \Delta ; \Omega ; \Gamma \setminus \Gamma_2 \vdash |e_1| :: |e_2| : L^I \tau$
   }
   
   \caseText{By IH on (3)}
   
   \caseFact{5} $\Psi ; \Theta ; \Delta ; \Omega ; \Gamma \setminus \Gamma_1 \vdash |e_1| : \tau$
   
   \caseText{By IH on (4)}
   
   \caseFact{6} $\Psi ; \Theta ; \Delta ; \Omega ; \Gamma_1 \setminus \Gamma_2 \vdash |e_2| : L^{I-1} \tau$
   
   \caseText{By T-Cons on (5), (6)}
   
   \caseFact{7} $\Psi ; \Theta ; \Delta ; \Omega ; (\Gamma \setminus \Gamma_1), (\Gamma_1 \setminus \Gamma_2) \vdash |e_1| :: |e_2| : L^{(I-1)+1} \, \tau$
   
   \caseText{Since $\Theta ; \Delta \vDash (I-1) + 1 = I$, by S-List}
   
   \caseFact{8} $\Psi ; \Theta ; \Delta \vdash L^{(I-1)+1} \, \tau \subty L^I \, \tau : \star$
   
   \caseText{By T-Sub on (7) and (8)}
   
   \caseFact{9} $\Psi ; \Theta ; \Delta ; \Omega ; (\Gamma \setminus \Gamma_1), (\Gamma_1 \setminus \Gamma_2) \vdash |e_1| :: |e_2| : L^{I} \, \tau$
   
   \caseText{Goal follows by T-Weaken on (9)}
 }
 
 \jcase{8}{AT-Match}{
   \jgivengoal{
    \caseFact{1} $\Psi ; \Theta ; \Delta ; \Omega ; \Gamma\vdash \texttt{match}(e,e_1,h.t.e_2) \checks \tau' \gens \Phi_1 \wedge \Phi_\texttt{body}, \Gamma'$
    
    \caseFact{2} $\Theta ; \Delta \vDash \Phi_1 \wedge \Phi_\texttt{body}$  
    
    \caseFact{3} $\Psi ; \Theta ; \Delta ; \Omega ; \Gamma\vdash e \infers L^I \tau \gens \Phi_1, \Gamma_1$
    
    \caseFact{4} $\Psi ; \Theta ; \Delta, I = 0 ; \Omega ; \Gamma_1\vdash e_1 \checks \tau' \gens \Phi_2,\Gamma_2$
    
    \caseFact{5} $\Psi ; \Theta ; \Delta, I \geq 1; \Omega ; \Gamma_1, h : \tau, t : L^{I-1} \tau \vdash e_2 \checks \tau' \gens \Phi_3,\Gamma_3$
    
    \caseFact{6} $\Phi_\texttt{body} = (I = 0 \to \Phi_2) \wedge (I \geq 1 \to \Phi_3)$
    
    \caseFact{7} $\Gamma' = \Gamma_2 \cap (\Gamma_3 \setminus \{h,t\})$
   }{
     $\Psi ; \Theta ; \Delta ; \Omega ; \Gamma \setminus \Gamma' \vdash \texttt{match}(|e|,|e_1|,h.t.|e_2|) : \tau'$   
   }
   
   \caseText{By IH on (3)}
   
   \caseFact{8} $\Psi ; \Theta ; \Delta ; \Omega ; \Gamma \setminus \Gamma_1 \vdash |e| : L^I \tau$
   
   \caseText{By IH on (4)}
   
   \caseFact{9} $\Psi ; \Theta ; \Delta, I = 0 ; \Omega ; \Gamma_1 \setminus \Gamma_2 \vdash |e_1| : \tau'$
   
   \caseText{By IH on (5)}
   
   \caseFact{10} $\Psi ; \Theta ; \Delta, I \geq 1; \Omega ; (\Gamma_1, h : \tau, t : L^{I-1} \tau) \setminus \Gamma_3 \vdash |e_2| : \tau'$
   
   \caseText{By two applications of T-Weaken on (9) and (10)}
   
   \caseFact{11} $\Psi ; \Theta ; \Delta, I = 0 ; \Omega ; \Gamma_1 \setminus \Gamma' \vdash |e_1| : \tau'$
   
   \caseFact{12} $\Psi ; \Theta ; \Delta, I \geq 1; \Omega ; \Gamma_1 \setminus \Gamma', h : \tau, t : L^{I-1} \tau \vdash |e_2| : \tau'$
   
   \caseText{By T-Match on (8), (11), (12)}
   
   \caseFact{13} $\Psi ; \Theta ; \Delta ; \Omega ; (\Gamma \setminus \Gamma_1),(\Gamma_1 \setminus \Gamma') \vdash \texttt{match}(|e|,e_1,h.t.|e_2|) : \tau'$ 
   
   \caseText{The Goal follows from T-Weakening on (13)}
 }
 
 \jcase{9}{AT-ExistI}{
   \jgivengoal{
     \caseFact{1} $\Psi ; \Theta ; \Delta ; \Omega ; \Gamma\vdash \texttt{pack}[I](e) \checks \exists i:S.\tau \gens \Phi_1 \wedge \Phi_2, \Gamma'$
     
     \caseFact{2} $\Theta ; \Delta \vDash \Phi_1 \wedge \Phi_2$
     
     \caseFact{3} $\Theta ; \Delta \vdash I : S \gens \Phi_1$
     
     \caseFact{4} $\Psi ; \Theta ; \Delta ; \Omega ; \Gamma\vdash e \checks \tau[I/i] \gens \Phi_2,\Gamma'$
   }{
     $\Psi ; \Theta ; \Delta ; \Omega ; \Gamma \setminus \Gamma' \vdash \texttt{pack}[I](|e|) : \exists i:S.\tau$
   }
   
   \caseText{By \autoref{thm:raw-sort-sound} on (3)}
   
   \caseFact{5} $\Theta ; \Delta \vdash I : S$
   
   \caseText{By IH on (4)}
   
   \caseFact{6} $\Psi ; \Theta ; \Delta ; \Omega ; \Gamma \setminus \Gamma' \vdash |e| : \tau[I/i]$
   
   \caseText{Goal follows by T-ExistI on (5) and (6)}
 }
 
 \jcase{10}{AT-ExistE}{
   \jgivengoal{
     \caseFact{1} ${
\Psi ; \Theta ; \Delta ; \Omega ; \Gamma\vdash \texttt{unpack } (i,x) = e \texttt{ in } e' \checks \tau' \gens \Phi, \Gamma_2 \setminus \{x\}
}$ 

    \caseFact{2} $\Theta ; \Delta \vDash \Phi$
    
    \caseFact{3} $\Psi ; \Theta ; \Delta ; \Omega ; \Gamma\vdash e \infers \exists i : S.\tau \gens \Phi_1, \Gamma_1$
    
    \caseFact{4} $\Psi ; \Theta, i : S ; \Delta ; \Omega ; \Gamma_1, x : \tau \vdash e' \checks \tau' \gens \Phi_2, \Gamma_2$
    
    \caseFact{5} $\Phi = \Phi_1 \wedge (\forall i : S. \Phi_2)$ 
   }{
     $\Psi ; \Theta ; \Delta ; \Omega ; \Gamma \setminus (\Gamma_2 \setminus \{x\})\vdash \texttt{unpack } (i,x) = |e| \texttt{ in } |e'| : \tau'$ 
   } 
   
   \caseText{By IH on (3)}
   
   \caseFact{6} $\Psi ; \Theta ; \Delta ; \Omega ; \Gamma \setminus \Gamma_1 \vdash |e| : \exists i : S.\tau$
   
   \caseText{From (2) and (5)}
   
   \caseFact{7} $\Theta ,  i : S ; \Delta \vDash \Phi_2$
   
   \caseText{By IH on (4) using (7)}
   
   \caseFact{8} $\Psi ; \Theta, i : S ; \Delta ; \Omega ; (\Gamma_1, x : \tau) \setminus \Gamma_2 \vdash e' : \tau'$
   
   \caseText{By T-Weaken on  (8)}
   
   \caseFact{9} $\Psi ; \Theta, i : S ; \Delta ; \Omega ; (\Gamma_1 \setminus (\Gamma_2 \setminus \{x : \tau\})), x : \tau \vdash |e'| : \tau'$
   
   \caseText{By T-ExistE on (6) and (9)}
   
   \caseFact{10} $\Psi ; \Theta ; \Delta ; \Omega ; (\Gamma \setminus \Gamma_1), (\Gamma_1 \setminus (\Gamma_2 \setminus \{x : \tau\})) \vdash \texttt{unpack } (i,x) = |e| \texttt{ in } |e'| : \tau'$
   
   \caseText{Goal follows by T-Weaken on (10)}
 }
 
 \jcase{11}{AT-Lam}{
  \jgivengoal{
    \caseFact{1} $\Psi ; \Theta ; \Delta ; \Omega ; \Gamma\vdash \lambda x.e \checks \tau_1 \loli \tau_2 \gens \Phi, \Gamma' \setminus \{x : \tau_1\}$
    
    \caseFact{2} $\Theta ; \Delta \vDash \Phi$
    
    \caseFact{3} $\Psi ; \Theta ; \Delta ; \Omega ; \Gamma, x : \tau_1 \vdash e \checks \tau_2, \gens \Phi, \Gamma'$
  }{
    $\Psi ; \Theta ; \Delta ; \Omega ; \Gamma \setminus (\Gamma' \setminus \{x : \tau_1\}) \vdash \lambda x.|e| : \tau_1 \loli \tau_2$  
  }
  
  \caseText{By IH on (2)}
  
  \caseFact{4} $\Psi ; \Theta ; \Delta ; \Omega ; (\Gamma, x : \tau_1) \setminus \Gamma' \vdash |e| : \tau_2$
  
  \caseText{By T-Weaken on (4)}
  
  \caseFact{5} $\Psi ; \Theta ; \Delta ; \Omega ; (\Gamma \setminus (\Gamma' \setminus \{x : \tau_1\})), x : \tau_1 \vdash |e| : \tau_2$
  
  \caseText{Goal follows by T-Lam on (5)}
 }
 
 \jcase{12}{AT-App}{
  \jgivengoal{
    \caseFact{1} $\Psi ; \Theta ; \Delta ; \Omega ; \Gamma\vdash e_1 \, e_2 \infers  \tau_2 \gens \Phi_1 \wedge \Phi_2, \Gamma_2$
    
    \caseFact{2} $\Theta ; \Delta \vDash \Phi_1 \wedge \Phi_2$
    
    \caseFact{3} $\Psi ; \Theta ; \Delta ; \Omega ; \Gamma\vdash e_1 \infers \tau_1 \loli \tau_2 \gens \Phi_1, \Gamma_1$
    
    \caseFact{4} $\Psi ; \Theta ; \Delta ; \Omega ; \Gamma_1\vdash e_2 \checks \tau_1 \gens \Phi_2, \Gamma_2$
  }{
    $\Psi ; \Theta ; \Delta ; \Omega ; \Gamma \setminus \Gamma_2 \vdash |e_1| \, |e_2| :  \tau_2$
  }
  
  \caseText{By IH on (3)}
  
  \caseFact{5} $\Psi ; \Theta ; \Delta ; \Omega ; \Gamma \setminus \Gamma_1 \vdash |e_1| : \tau_1 \loli \tau_2$
  
  \caseText{By IH on (4)}
  
  \caseFact{6} $\Psi ; \Theta ; \Delta ; \Omega ; \Gamma_1 \setminus \Gamma_2 \vdash |e_2| : \tau_1$
  
  \caseText{By T-App on (5) and (6)}
  
  \caseFact{7} $\Psi ; \Theta ; \Delta ; \Omega ; (\Gamma \setminus \Gamma_1),(\Gamma_1 \setminus \Gamma_2)\vdash |e_1| \, |e_2| :  \tau_2$
  
  \caseText{Goal follows from T-Weaken on (7)}
 }
 
 \jcase{13}{AT-TensorI}{
  \jgivengoal{
    \caseFact{1} $\Psi ; \Theta ; \Delta ; \Omega ; \Gamma\vdash \angles{e_1,e_2} \checks  \tau_1 \otimes \tau_2 \gens \Phi_1 \wedge \Phi_2, \Gamma_2$
    
    \caseFact{2} $\Theta ; \Delta \vDash \Phi_1 \wedge \Phi_2$
    
    \caseFact{3} $\Psi ; \Theta ; \Delta ; \Omega ; \Gamma\vdash e_1 \checks \tau_1 \gens \Phi_1, \Gamma_1$
    
    \caseFact{4} $\Psi ; \Theta ; \Delta ; \Omega ; \Gamma_1\vdash e_2 \checks \tau_2 \gens \Phi_2, \Gamma_2$
  }{
    $\Psi ; \Theta ; \Delta ; \Omega ; \Gamma \setminus \Gamma_2 \vdash \angles{|e_1|,|e_2|} :  \tau_1 \otimes \tau_2$
  }
  
  \caseText{By IH on (3)}
  
  \caseFact{5} $\Psi ; \Theta ; \Delta ; \Omega ; \Gamma \setminus \Gamma_1 \vdash |e_1| : \tau_1$
  
  \caseText{By IH on (4)}
  
  \caseFact{6} $\Psi ; \Theta ; \Delta ; \Omega ; \Gamma_1 \setminus \Gamma_2 \vdash |e_2| : \tau_2$
  
  \caseText{By T-TensorI on (5) and (6)}
  
  \caseFact{7} $\Psi ; \Theta ; \Delta ; \Omega ; (\Gamma \setminus \Gamma_1),(\Gamma_1 \setminus \Gamma_2)\vdash \angles{|e_1|,|e_2|} :  \tau_1 \otimes \tau_2$
  
  \caseText{Goal follows from T-Weaken on (7)}
 }
 
 \jcase{14}{AT-TensorE}{
   \jgivengoal{
    \caseFact{1} $\Psi ; \Theta ; \Delta ; \Omega ; \Gamma\vdash \texttt{let } \angles{x,y} = e \texttt{ in } e' \checks \tau' \gens \Phi_1 \wedge \Phi_2, \Gamma_2 \setminus \{x : \tau_1,y : \tau_2\}$
    
    \caseFact{2} $\Theta ; \Delta \vDash \Phi_1 \wedge \Phi_2$
    
    \caseFact{3} $\Psi ; \Theta ; \Delta ; \Omega ; \Gamma\vdash e \infers \tau_1 \otimes \tau_2 \gens \Phi_1, \Gamma_1$ 
    
    \caseFact{4} $\Psi ; \Theta ; \Delta ; \Omega ; \Gamma_1,x : \tau_1, y : \tau_2\vdash e' \checks \tau' \gens \Phi_2,\Gamma_2$
   }{
     $\Psi ; \Theta ; \Delta ; \Omega ; \Gamma \setminus (\Gamma_2 \setminus \{x : \tau_1,y : \tau_2\})\vdash \texttt{let } \angles{x,y} = |e| \texttt{ in } |e'| : \tau'$
   }
   
   \caseText{By IH on (3)}
   
   \caseFact{5} $\Psi ; \Theta ; \Delta ; \Omega ; \Gamma \setminus \Gamma_1 \vdash |e| : \tau_1 \otimes \tau_2$
  
   \caseText{By IH on (4)}   
   
   \caseFact{6} $\Psi ; \Theta ; \Delta ; \Omega ; (\Gamma_1,x : \tau_1, y : \tau_2) \setminus \Gamma_2 \vdash |e'| : \tau'$
   
   \caseText{By T-Weaken on (6)}
   
   \caseFact{7} $\Psi ; \Theta ; \Delta ; \Omega ; (\Gamma_1 \setminus (\Gamma_2 \setminus \{x : \tau_1, y : \tau_2\})),x : \tau_1, y : \tau_2\ \vdash |e'| : \tau'$
   
   \caseText{Goal follows by T-TensorE and T-Weaken on (5) and (7)}
 }
 
 \jcase{15}{AT-WithI}{
   \jgivengoal{
    \caseFact{1} $\Psi ; \Theta ; \Delta ; \Omega ; \Gamma \vdash (e_1,e_2) \checks \tau_1 \amp \tau_2 \gens \Phi_1 \wedge \Phi_2, \Gamma_1 \cap \Gamma_2$
    
    \caseFact{2} $\Theta ; \Delta \vDash \Phi_1 \wedge \Phi_2$
    
    \caseFact{3} $\Psi ; \Theta ; \Delta ; \Omega ; \Gamma \vdash e_1 \checks \tau_1 \gens \Phi_1, \Gamma_1$
    
    \caseFact{4} $\Psi ; \Theta ; \Delta ; \Omega ; \Gamma \vdash e_2 \checks \tau_2 \gens \Phi_2, \Gamma_2$
   }{
     $\Psi ; \Theta ; \Delta ; \Omega ; \Gamma \setminus (\Gamma_1 \cap \Gamma_2) \vdash (|e_1|,|e_2|) : \tau_1 \amp \tau_2$
   }
   
   \caseText{By IH on (3)}
   
   \caseFact{5} $\Psi ; \Theta ; \Delta ; \Omega ; \Gamma \setminus \Gamma_1 \vdash |e_1| : \tau_1$
   
   \caseText{By IH on (4)}
   
   \caseFact{6} $\Psi ; \Theta ; \Delta ; \Omega ; \Gamma \setminus \Gamma_2 \vdash |e_2| : \tau_2$
   
   \caseText{By T-Weaken on (5) and (6)}
   
   \caseFact{7} $\Psi ; \Theta ; \Delta ; \Omega ; \Gamma \setminus (\Gamma_1 \cap \Gamma_2) \vdash |e_1| : \tau_1$
   
   \caseFact{8} $\Psi ; \Theta ; \Delta ; \Omega ; \Gamma \setminus (\Gamma_1 \cap \Gamma_2) \vdash |e_2| : \tau_2$
   
   \caseText{Goal follows by T-WithI on (7) and (8)}
 }
 
 \jcase{16}{AT-Fst}{
   \jgivengoal{
    \caseFact{1} $\Psi ; \Theta ; \Delta ; \Omega ; \Gamma \vdash \texttt{fst}(e) \infers \tau_1 \gens \Phi,\Gamma'$ 
    
    \caseFact{2} $\Theta ; \Delta \vDash \Phi$
    
    \caseFact{3} $\Psi ; \Theta ; \Delta ; \Omega ; \Gamma \vdash e \infers \tau_1 \amp \tau_2 \gens \Phi,\Gamma'$ 
   }{
     $\Psi ; \Theta ; \Delta ; \Omega ; \Gamma \setminus \Gamma' \vdash \texttt{fst}(|e|) : \tau_1$
   }
   
   \caseText{By IH on (3)}
   
   \caseFact{4} $\Psi ; \Theta ; \Delta ; \Omega ; \Gamma \setminus \Gamma' \vdash |e| : \tau_1 \amp \tau_2$
   
   \caseText{Goal follows by T-Fst on (4)}
 }

 
 \jcase{17}{AT-Snd}{Identical to Case 16.}
 
 \jcase{18}{AT-Inl}{
   \jgivengoal{
    \caseFact{1} $\Psi ; \Theta ; \Delta ; \Omega ; \Gamma \vdash \texttt{inl}(e) \checks \tau_1 \oplus \tau_2 \gens \Phi,\Gamma'$ 
    
    \caseFact{2} $\Theta ; \Delta \vDash \Phi$
    
    \caseFact{3} $\Psi ; \Theta ; \Delta ; \Omega ; \Gamma \vdash e \checks \tau_1 \gens \Phi,\Gamma'$ 
   }{
     $\Psi ; \Theta ; \Delta ; \Omega ; \Gamma \setminus \Gamma' \vdash \texttt{inl}(|e|) : \tau_1 \oplus \tau_2$
   }
   
   \caseText{By IH on (3)}
   
   \caseFact{4} $\Psi ; \Theta ; \Delta ; \Omega ; \Gamma \setminus \Gamma' \vdash |e| : \tau_1$
   
   \caseText{Goal follows by T-Inl on (4)}
 }

 \jcase{19}{AT-Inr}{Identical to Case 18.}
 
 \jcase{20}{AT-Case}{
  \jgivengoal{
   \caseFact{1} $\Psi ; \Theta ; \Delta ; \Omega ; \Gamma \vdash \texttt{case}(e,x.e_1,y.e_2) \checks \tau \gens \Phi_1 \wedge \Phi_2 \wedge \Phi_3, \Gamma'$
   
   \caseFact{2} $\Theta ; \Delta \vDash \Phi_1 \wedge \Phi_2 \wedge \Phi_3$
   
   \caseFact{3} $\Psi ; \Theta ; \Delta ; \Omega ; \Gamma \vdash e \infers \tau_1 \oplus \tau_2 \gens \Phi_1, \Gamma_1$
   
   \caseFact{4} $\Psi ; \Theta ; \Delta ; \Omega ; \Gamma_1, x: \tau_1 \vdash e_1 \checks \tau \gens \Phi_2,\Gamma_2$
   
   \caseFact{5} $\Psi ; \Theta ; \Delta ; \Omega ; \Gamma_1, y: \tau_2 \vdash e_2 \checks \tau \gens \Phi_3,\Gamma_3$
  }{
   $\Psi ; \Theta ; \Delta ; \Omega ; \Gamma \setminus \Gamma' \vdash \texttt{case}(|e|,x.|e_1|,y.|e_2|) : \tau$
  }
  
  \caseText{By IH on (3)}
  
  \caseFact{6} $\Psi ; \Theta ; \Delta ; \Omega ; \Gamma \setminus \Gamma_1 \vdash |e| : \tau_1 \oplus \tau_2$
  
  \caseText{By IH on (4)}
  
  \caseFact{7} $\Psi ; \Theta ; \Delta ; \Omega ; (\Gamma_1, x: \tau_1) \setminus \Gamma_2 \vdash |e_1| : \tau$
  
  \caseText{By IH on (5)}
  
  \caseFact{8} $\Psi ; \Theta ; \Delta ; \Omega ; (\Gamma_1, y: \tau_2) \setminus \Gamma_2 \vdash |e_2| : \tau$
  
  \caseText{By T-Weaken on (7) and then (8)}
  
  \caseFact{9} $\Psi ; \Theta ; \Delta ; \Omega ; \Gamma_1 \setminus \Gamma', x : \tau_1 \vdash |e_1| : \tau$
  
  \caseFact{10} $\Psi ; \Theta ; \Delta ; \Omega ; \Gamma_1 \setminus \Gamma', y : \tau_2 \vdash |e_2| : \tau$
  
  \caseText{By T-Case on (6), (9), and (10)}
  
  \caseFact{11} $\Psi ; \Theta ; \Delta ; \Omega ; (\Gamma \setminus \Gamma_1), (\Gamma_1 \setminus \Gamma') \vdash \texttt{case}(|e|,x.|e_1|,y.|e_2|) : \tau$
  
  \caseText{Goal follows by T-Weaken on (11)}
 }
 
 \jcase{21}{AT-ExpI}{
  \jgivengoal{
    \caseFact{1} $\Psi ; \Theta ; \Delta ; \Omega ; \Gamma \vdash !e \checks !\tau \gens \Phi, \Gamma$
    
    \caseFact{2} $\Theta ; \Delta \vDash \Phi$
    
    \caseFact{3} $\Psi ; \Theta ; \Delta ; \Omega ; \cdot \vdash e \checks \tau \gens \Phi,\Gamma'$
  }{
   $\Psi ; \Theta ; \Delta ; \Omega ; \cdot \vdash !|e| : !\tau$
  }
  
  \caseText{By IH on (3)}
  
  \caseFact {4} $\Psi ; \Theta ; \Delta ; \Omega ; \cdot \vdash |e| : \tau$
  
  \caseText{Goal follows by T-ExpI on (4)}
 }
 
 \jcase{22}{AT-ExpE}{
  \jgivengoal{
    \caseFact{1} $\Psi ; \Theta ; \Delta ; \Omega ; \Gamma \vdash \texttt{let } !x = e \texttt{ in } e' \checks \tau' \gens \Phi_1 \wedge \Phi_2, \Gamma_2$ 
    
    \caseFact{2} $\Theta ; \Delta \vDash \Phi_1 \Phi_2$
    
    \caseFact{3} $\Psi ; \Theta ; \Delta ; \Omega ; \Gamma \vdash e \infers !\tau \gens \Phi_1,\Gamma_1$
    
    \caseFact{4}$\Psi ; \Theta ; \Delta ; \Omega, x : \tau ; \Gamma_1 \vdash e' \checks \tau' \gens \Phi_2,\Gamma_2$
  }{
   $\Psi ; \Theta ; \Delta ; \Omega ; \Gamma  \setminus \Gamma_2 \vdash \texttt{let } !x = |e| \texttt{ in } |e'| : \tau'$
  } 
  
  \caseText{By IH on (3)}
  
  \caseFact{5} $\Psi ; \Theta ; \Delta ; \Omega ; \Gamma \setminus \Gamma_1 \vdash |e| : !\tau$
  
  \caseText{By IH on (4)}
  
  \caseFact{6} $\Psi ; \Theta ; \Delta ; \Omega, x : \tau ; \Gamma_1 \setminus \Gamma_2 \vdash |e'| : \tau'$
  
  \caseText{By T-ExpE on (5) and (6)}
  
  \caseFact{7} $\Psi ; \Theta ; \Delta ; \Omega ; (\Gamma \setminus \Gamma_1),(\Gamma_1  \setminus \Gamma_2) \vdash \texttt{let } !x = |e| \texttt{ in } |e'| : \tau'$
  
  \caseText{Goal follows by T-Weaken on (7)}
 }
 
 \jcase{23}{AT-TAbs}{
  \jgivengoal{
    \caseFact{1} $\Psi ; \Theta ; \Delta ; \Omega ; \Gamma \vdash \Lambda \alpha. e \checks \forall \alpha : K.\tau \gens \Phi,\Gamma'$
    
    \caseFact{2} $\Theta ; \Delta \vDash \Phi$
    
    \caseFact{3} $\Psi, \alpha : K ; \Theta ; \Delta ; \Omega ; \Gamma \vdash e \checks \tau \gens \Phi, \Gamma'$
  }{
    $\Psi ; \Theta ; \Delta ; \Omega ; \Gamma \setminus \Gamma' \vdash \Lambda \alpha. |e| : \forall \alpha : K.\tau$
  }
  
  \caseText{By IH on (3)}
  
  \caseFact{4} $\Psi, \alpha : K ; \Theta ; \Delta ; \Omega ; \Gamma\setminus \Gamma' \vdash |e| : \tau$
  
  \caseText{Goal follows by T-TAbs on (4)}
 }
 
 \jcase{24}{AT-TApp}{
  \jgivengoal{
   \caseFact{1} $\Psi ; \Theta ; \Delta ; \Omega ; \Gamma \vdash e [\tau'] \infers \tau[\tau'/\alpha] \gens \Phi_1 \wedge \Phi_2, \Gamma'$
   
   \caseFact{2} $\Theta ; \Delta \vDash \Phi_1 \wedge \Phi_2$
   
   \caseFact{3} $\Psi ; \Theta ; \Delta ; \Omega ; \Gamma \vdash e \infers \forall \alpha : K.\tau \gens \Phi_1, \Gamma'$ 
   
   \caseFact{4} $\Psi ; \Theta ; \Delta \vdash \tau' : K \gens \Phi_2$
  }{
   $\Psi ; \Theta ; \Delta ; \Omega ; \Gamma \setminus \Gamma' \vdash |e| [\tau'] : \tau[\tau'/\alpha]$
  }
  
  \caseText{By IH on (3)}
  
  \caseFact{5} $\Psi ; \Theta ; \Delta ; \Omega ; \Gamma \setminus \Gamma' \vdash |e| : \forall \alpha : K.\tau$
  
  \caseText{By \autoref{thm:raw-kind-sound} on (5)}
  
  \caseFact{6} $\Psi ; \Theta ; \Delta \vdash \tau' : K$
  
  \caseText{Goal Follows by T-TApp on (5) and (6)}
 }
 
 \jcase{25}{AT-TApp}{
  \jgivengoal{
   \caseFact{1} $\Psi ; \Theta ; \Delta ; \Omega ; \Gamma \vdash \Lambda i. e \checks \forall i : S. \tau \gens \forall i : S. \Phi, \Gamma'$  
   
   \caseFact{2} $\Theta ; \Delta \vDash \forall i : S. \Phi$
   
   \caseFact{3} $\Psi ; \Theta, i : S ; \Delta ; \Omega ; \Gamma \vdash e \checks \tau \gens \Phi, \Gamma'$
  }{
   $\Psi ; \Theta ; \Delta ; \Omega ; \Gamma \setminus \Gamma' \vdash \Lambda i. |e| : \forall i : S. \tau$
  }
  
  \caseText{Equivalently to (2)}
  
  \caseFact{4} $\Theta , i : S; \Delta \vDash \Phi$
  
  \caseText{By IH on (3)}
  
  \caseFact{5} $\Psi ; \Theta, i : S ; \Delta ; \Omega ; \Gamma \setminus \Gamma' \vdash |e| : \tau$
  
  \caseText{Goal follows by T-IAbs on (5)}
 }
 
 \jcase{26}{AT-IApp}{
  \jgivengoal{
   \caseFact{1} $\Psi ; \Theta ; \Delta ; \Omega ; \Gamma \vdash e [I] \infers \tau[I/i] \gens \Phi_1 \wedge \Phi_2,\Gamma'$
   
   \caseFact{2} $\Theta ; \Delta \vDash \Phi_1 \wedge \Phi_2$
   
   \caseFact{3} $\Psi ; \Theta ; \Delta ; \Omega ; \Gamma \vdash e \infers \forall i : S.\tau \gens \Phi_1,\Gamma'$ 
   
   \caseFact{4} $\Theta ; \Delta \vdash I : S \gens \Phi_2$
  }{
   $\Psi ; \Theta ; \Delta ; \Omega ; \Gamma\setminus \Gamma' \vdash |e [I]| : \tau[I/i]$
  }
  
  \caseText{By IH on (3)}
  
  \caseFact{5} $\Psi ; \Theta ; \Delta ; \Omega ; \Gamma \setminus \Gamma' \vdash |e| : \forall i : S. \tau$
  
  \caseText{By \autoref{thm:raw-sort-sound} on (4)}
  
  \caseFact{6} $\Theta ; \Delta \vdash I : S$
  
  \caseText{Goal follows by T-IApp on (5) and (6)}
 }
 
 \jcase{27}{AT-Fix}{
   \jgivengoal{
    \caseFact{1} $\Psi ; \Theta ; \Delta ; \Omega ; \Gamma \vdash \texttt{fix } x.e \checks \tau \gens \Phi,\Gamma$
    
    \caseFact{2} $\Theta ; \Delta \vDash \Phi$
    
    \caseFact{3} $\Psi ; \Theta ; \Delta ; \Omega, x : \tau ; \cdot \vdash e \checks \tau \gens \Phi,\Gamma'$
   }{
    $\Psi ; \Theta ; \Delta ; \Omega ; \cdot \vdash \texttt{fix } x.|e| : \tau$
   }
   
   \caseText{By IH on (3)}
   
   \caseFact{4} $\Psi ; \Theta ; \Delta ; \Omega, x : \tau ; \cdot \vdash |e| : \tau$
   
   \caseText{Goal follows by T-Fix on (4)}
 }
 
 \jcase{28}{AT-Ret}{
  \jgivengoal{
   \caseFact{1} $\Psi ; \Theta ; \Delta ; \Omega ; \Gamma \vdash \texttt{ret } e \checks \M \, (I,\vec{p}) \, \tau \gens \Phi, \Gamma'$ 
   
   \caseFact{2} $\Theta ; \Delta \vDash \Phi$ 
   
   \caseFact{3} $\Psi ; \Theta ; \Delta ; \Omega ; \Gamma \vdash e \checks \tau \gens \Phi,\Gamma'$
  }{
   $\Psi ; \Theta ; \Delta ; \Omega ; \Gamma \setminus \Gamma' \vdash \texttt{ret } |e| : \M \, (I,\vec{p}) \, \tau$
  }
   \caseText{By IH on (3)}
   
   \caseFact{4} $\Psi ; \Theta ; \Delta ; \Omega ; \Gamma \setminus \Gamma' \vdash |e| : \tau$
   
   \caseText{By T-Ret on (4)}
   
   \caseFact{5} $\Psi ; \Theta ; \Delta ; \Omega ; \Gamma \setminus \Gamma' \vdash \texttt{ret}(|e|) :  \M \, (I,\vec{0}) \, \tau$
   
   \caseText{By S-Monad and S-Refl using the fact that $\Theta ; \Delta \vDash \vec{p} \geq \vec{0}$}
   
   \caseFact{6} $\Psi ; \Theta ; \Delta \vdash \M \, (I,\vec{0}) \, \tau \subty \M \, (I,\vec{p}) \, \tau : \star$
   
   \caseText{Goal follows by T-Sub on (5) and (6)}
 }
 
 \jcase{29}{AT-Bind}{
  \jgivengoal{
   \caseFact{1} $\Psi ; \Theta ; \Delta ; \Omega ; \Gamma \vdash \texttt{bind } x = e_1 \texttt{ in } e_2 \checks \M \, (I,\vec{q})\, \tau_2 \gens \Phi, \Gamma_2 \setminus \{x : \tau_1\}$
   
   \caseFact{2} $\Theta ; \Delta \vDash \Phi$
   
   \caseFact{3} $\Psi ; \Theta ; \Delta ; \Omega ; \Gamma \vdash e_1 \infers \M \, (J,\vec{p})\, \tau_1 \gens \Phi_1,\Gamma_1$
   
   \caseFact{4} $\Psi ; \Theta; \Delta ; \Omega ; \Gamma_1, x:\tau_1 \vdash e_2 \checks \M \, (I,\vec{q} - \vec{p})\, \tau_2 \gens \Phi_2,\Gamma_2$
   
   \caseFact{5} $\Phi = (\vec{q} \geq \vec{p}) \wedge (I =J)  \wedge \Phi_1 \wedge \Phi_2$
  }{
   $\Psi ; \Theta ; \Delta ; \Omega ; \Gamma \setminus (\Gamma_2 \setminus \{x : \tau_1\}) \vdash \texttt{bind } x = |e_1| \texttt{ in } |e_2| : \M \, (I,\vec{q})\, \tau_2$  
  }
  
  \caseText{By IH on (3)}
  
  \caseFact{6} $\Psi ; \Theta ; \Delta ; \Omega ; \Gamma \setminus \Gamma_1 \vdash |e_1| : \M \, (J,\vec{p})\, \tau_1$
  
  \caseText{By IH on (4)}
  
  \caseFact{7} $\Psi ; \Theta; \Delta ; \Omega ; (\Gamma_1, x:\tau_1) \setminus \Gamma_2 \vdash |e_2| : \M \, (I,\vec{q} - \vec{p})\, \tau_2$
  
  \caseText{From (2) and (5)}
  
  \caseFact{8} $\Theta ; \Delta \vDash I = J$
  
  \caseText{By S-Monad and S-Refl, using (8)}
  
  \caseFact{9} $\Psi ; \Theta ; \Delta \vdash M(J,\vec{p})\, \tau_1 \subty M(I,\vec{p}) \, \tau_1 \; : \star$
  
  \caseText{By T-Sub on (6) and (9)}
  
  \caseFact{10} $\Psi ; \Theta ; \Delta ; \Omega ; \Gamma \setminus \Gamma_1 \vdash |e_1| : \M \, (I,\vec{p})\, \tau_1$
  
  \caseText{By T-Weaken on (7)}
  
  \caseFact{11} $\Psi ; \Theta; \Delta ; \Omega ; (\Gamma_1 \setminus (\Gamma_2 \setminus \{x\})), x:\tau_1 \vdash |e_2| : \M \, (I,\vec{q} - \vec{p})\, \tau_2$
  
  \caseText{By T-Bind on (10) and (11)}
  
  \caseFact{12} $\Psi ; \Theta ; \Delta ; \Omega (\Gamma \setminus \Gamma_1), (\Gamma_1 \setminus (\Gamma_2 \setminus \{x\})) \vdash \texttt{bind } x = |e_1| \texttt{ in } |e_2| : \M \, (I,\vec{q} - \vec{p} + \vec{p}) \, \tau_2$
  
  \caseText{By S-Monad and S-Refl, using the fact that $\Theta ; \Delta \vDash (\vec{q} - \vec{p}) + \vec{p} = \vec{q}$}
  
  \caseFact{13} $\Psi ; \Theta ; \Delta \vdash \M \, (I,\vec{q} - \vec{p} + \vec{p}) \, \tau_2 \subty \M \, (I,\vec{q}) \, \tau_2 : \star$
  
  \caseText{By T-Sub on (12) and (13)}
  
  \caseFact{13} $\Psi ; \Theta ; \Delta ; \Omega (\Gamma \setminus \Gamma_1), (\Gamma_1 \setminus (\Gamma_2 \setminus \{x\})) \vdash \texttt{bind } x = |e_1| \texttt{ in } |e_2| : \M \, (I,\vec{q}) \, \tau_2$
  
  \caseText{Goal follows by T-Weaken on (13)}
 }
 
 \jcase{30}{AT-Tick}{
  \jgivengoal{
   \caseFact{1} $\Psi ; \Theta ; \Delta ; \Omega ; \Gamma \vdash \texttt{tick}[I|\vec{p}] \infers \M \, (I,\vec{p})\, 1 \gens \Phi_1 \wedge \Phi_2,\Gamma$
   
   \caseFact{2} $\Theta ; \Delta \vDash \Phi_1 \wedge \Phi_2$
   
   \caseFact{3} $\Theta ; \Delta \vdash I : \N \gens \Phi_1$
   
   \caseFact{4} $\Theta ; \Delta \vdash \vec{p} : \potvec \gens \Phi_1$
  }{
    $\Psi ; \Theta ; \Delta ; \Omega ; \cdot \vdash \texttt{tick}[I|\vec{p}] : \M \, (I,\vec{p})\, 1$
  }
  
  \caseText{By \autoref{thm:raw-sort-sound} on (3)}
  
  \caseFact{5} $\Theta ; \Delta \vdash I : \N$
  
  \caseText{By \autoref{thm:raw-sort-sound} on (4)}
  
  \caseFact{6} $\Theta ; \Delta \vdash \vec{p} : \potvec$
  
  \caseText{Goal Follows by T-Tick on (5) and (6)}
 }
 
 \jcase{31}{AT-Release}{
   \jgivengoal{
     \caseFact{1} $\Psi ; \Theta ; \Delta ; \Omega ; \Gamma \vdash \texttt{release } x = e_1 \texttt{ in }e_2 \checks \M \, (I,\vec{p}) \, \tau_2 \gens (I = J \wedge \Phi_1 \wedge \Phi_2), \Gamma_2 \setminus \{x\}$
     
     \caseFact{2} $\Theta ; \Delta \vDash (I = J) \wedge \Phi_1 \wedge \Phi_2$
     
     \caseFact{3} $\Psi ; \Theta ; \Delta ; \Omega ; \Gamma \vdash e_1 \infers [J | \vec{q}] \tau_1 \gens \Phi_1,\Gamma_1$
     
     \caseFact{4} $\Psi ; \Theta ; \Delta ; \Omega ; \Gamma_1, x : \tau_1 \vdash e_2 \checks \M \, (I,\vec{p} + \vec{q}) \, \tau_2 \gens \Phi_2, \Gamma_2$
   }{
    $\Psi ; \Theta ; \Delta ; \Omega ; \Gamma \setminus (\Gamma_2 \setminus \{x : \tau_1\}) \vdash \texttt{release } x = |e_1| \texttt{ in }|e_2| : \M \, (I,\vec{p}) \, \tau_2$
   }
   
   \caseText{By IH on (3)}
   
   \caseFact{5} $\Psi ; \Theta ; \Delta ; \Omega ; \Gamma \setminus \Gamma_1 \vdash |e_1| : [J | \vec{q}] \tau_1$
   
   \caseText{By IH on (4)}
   
   \caseFact{6} $\Psi ; \Theta ; \Delta ; \Omega ; (\Gamma_1, x : \tau_1) \setminus \Gamma_2 \vdash |e_2| : \M \, (I,\vec{p} + \vec{q}) \, \tau_2$
   
   \caseText{From (2)}
   
   \caseFact{7} $\Theta ; \Delta \vDash I = J$
   
   \caseText{By S-Pot, S-Refl, and (7)}
   
   \caseFact{8} $\Psi ; \Theta ; \Delta \vdash [J|\vec{q}] \tau_1 \subty [I|\vec{q}] \tau_1 : \star$
   
   \caseText{By T-Sub on (5) and (8)}
   
   \caseFact{9} $\Psi ; \Theta ; \Delta ; \Omega ; \Gamma \setminus \Gamma_1 \vdash |e_1| : [I | \vec{q}] \tau_1$
   
   \caseText{By T-Weaken on (6)}
   
   \caseFact{10} $\Psi ; \Theta ; \Delta ; \Omega ; \Gamma_1 \setminus (\Gamma_2 \setminus x : \tau_1), x : \tau_1 \vdash |e_2| : \M \, (I,\vec{p} + \vec{q}) \, \tau_2$
   
   \caseText{By T-Release on (9) and (10)}
   
   \caseFact{11} $\Psi ; \Theta ; \Delta ; \Omega ; (\Gamma \setminus \Gamma_1),(\Gamma_1 \setminus (\Gamma_2 \setminus x : \tau_1)) \vdash \texttt{release } x = |e_1| \texttt{ in }|e_2| : \M \, (I,\vec{p}) \, \tau_2$
   
   \caseText{Goal follows by another T-Weaken on (11)}
 }
 
 \jcase{32}{AT-Store}{
  \jgivengoal{
    \caseFact{1} $\Psi ; \Theta ; \Delta ; \Omega ; \Gamma \vdash \texttt{store}[K|\vec{w}](e) \checks \M \, (I,\vec{q}) \, ([J | \vec{p}] \, \tau) \gens \Phi, \Gamma'$
    
    \caseFact{2} $\Theta ; \Delta \vDash \Phi$
    
    \caseFact{3} $\Theta ; \Delta \vdash K : \N \gens \Phi_1$
    
    \caseFact{4} $\Theta ; \Delta \vdash \vec{w} : \potvec \gens \Phi_2$
    
    \caseFact{5} $\Psi ; \Theta ; \Delta ; \Omega ; \Gamma \vdash e \checks \tau \gens \Phi_3,\Gamma'$
    
    \caseFact{6} $\Phi =  \Phi_1 \wedge \Phi_2 \wedge\Phi_3 \wedge  (\vec{p} \leq \vec{w} \leq \vec{q}) \wedge (I = J = K)$
  }{
   $\Psi ; \Theta ; \Delta ; \Omega ; \Gamma \setminus \Gamma' \vdash \texttt{store}[K|\vec{w}](|e|) : \M \, (I,\vec{q}) \, ([J | \vec{p}] \, \tau)$  
  }
  
  \caseText{By \autoref{thm:raw-sort-sound} on (3)}
  
  \caseFact{7} $\Theta ; \Delta \vdash K : \N$
  
  \caseText{By \autoref{thm:raw-sort-sound} on (4)}
  
  \caseFact{8} $\Theta ; \Delta \vdash \vec{w} : \potvec$
  
  \caseText{By IH on (5)}
  
  \caseFact{9} $\Psi ; \Theta ; \Delta ; \Omega ; \Gamma \setminus \Gamma' \vdash |e| : \tau$
  
  \caseText{By T-Store on (7), (8), (9)}
  
  \caseFact{10} $\Psi ; \Theta ; \Delta ; \Omega ; \Gamma \setminus \Gamma' \vdash \texttt{store}[K|\vec{w}](|e|) : \M \, (K,\vec{w}) \, ([K | \vec{w}] \, \tau)$
  
  \caseText{By S-Monad, S-Pot, and S-Refl using (2), (6)}
  
  \caseFact{11} $\Psi ; \Theta ; \Delta \vdash \M \, (K,\vec{w}) \, ([K | \vec{w}] \, \tau) \subty \M \, (I,\vec{q}) \, ([J | \vec{p}] \, \tau) : \star$
  
  \caseText{Goal follows by T-Sub on (10) and (11)}
 }
 
 \jcase{33}{AT-StoreConst}{
  \jgivengoal{
   \caseFact{1}  $\Psi ; \Theta ; \Delta ; \Omega ; \Gamma \vdash \texttt{store}[J](e) \checks \M \, (K,\vec{p}) \, ([I] \, \tau) \gens \Phi, \Gamma'$ 
   
   \caseFact{2} $\Theta ; \Delta \vDash \Phi$
   
   \caseFact{3} $\Psi ; \Theta ; \Delta ; \Omega ; \Gamma \vdash e \checks \tau \gens \Phi_1,\Gamma'$
   
   \caseFact{4} $\Theta ; \Delta \vdash J \checks \mathbb{R} \gens \Phi_2$
   
   \caseFact{5} $\Phi = (\texttt{const}(I) \leq \texttt{const}(J) \leq \vec{p}) \wedge \Phi_1 \wedge \Phi_2$
  }{
   $\Psi ; \Theta ; \Delta ; \Omega ; \Gamma \setminus \Gamma' \vdash \texttt{store}[J](|e|) : \M \, (K,\vec{p}) \, ([I] \, \tau)$
  }
  
  \caseText{By IH on (3)}
  
  \caseFact{6} $\Psi ; \Theta ; \Delta ; \Omega ; \Gamma \setminus \Gamma' \vdash |e| : \tau$
  
  \caseText{By \autoref{thm:raw-sort-sound} on (4)}
  
  \caseFact{7} $\Theta ; \Delta \vdash J : \mathbb{R}$
  
  \caseText{By T-StoreConst on (6) and (7)}
  
  \caseFact{8} $\Psi ; \Theta ; \Delta ; \Omega ; \Gamma \setminus \Gamma' \vdash \texttt{store}[J](|e|) : \M \, (J,\texttt{const}(J)) \, ([J] \, \tau)$
  
  \caseText{By (2) and (5)}
  
  \caseFact{9} $\Theta ; \Delta \vDash \texttt{const}(I) \leq \texttt{const}(J) \leq \vec{p}$
  
  \caseText{By S-Monad, S-Pot, S-Refl, and (9)}
  
  \caseFact{10} $\Psi ; \Theta ; \Delta \vdash \M \, (J,\texttt{const}(J)) \, ([J] \, \tau) \subty \M \, (K,\vec{p}) \, ([I] \, \tau)$
  
  \caseText{Goal follows by T-Sub on (8) and (10)}
 }
 
 \jcase{34}{AT-ReleaseConst}{
  \jgivengoal{
    \caseFact{1} $\Psi ; \Theta ; \Delta ; \Omega ; \Gamma \vdash \texttt{release } x = e_1 \texttt{ in }e_2 \checks \M \, (I,\vec{p}) \, \tau_2 \gens \Phi_1 \wedge \Phi_2, \Gamma_2 \setminus \{x\}$
    
    \caseFact{2} $\Theta ; \Delta \vDash \Phi_1 \wedge \Phi_2$
    
    \caseFact{3} $\Psi ; \Theta ; \Delta ; \Omega ; \Gamma \vdash e_1 \infers [J] \tau_1 \gens \Phi_1,\Gamma_1$
    
    \caseFact{4} $\Psi ; \Theta ; \Delta ; \Omega ; \Gamma_1, x : \tau_1 \vdash e_2 \checks \M \, (I,\vec{p} + \texttt{const}(J)) \, \tau_2 \gens \Phi_2, \Gamma_2$
  }{
   $\Psi ; \Theta ; \Delta ; \Omega ; \Gamma \setminus (\Gamma_2 \setminus \{x\}) \vdash \texttt{release } x = |e_1| \texttt{ in }|e_2| : \M \, (I,\vec{p}) \, \tau_2$
  }
  
  \caseText{By IH on (3)}
  
  \caseFact{5} $\Psi ; \Theta ; \Delta ; \Omega ; \Gamma \setminus \Gamma_1 \vdash |e_1| : [J] \tau_1$
  
  \caseText{By IH on (4)}
  
  \caseFact{6} $\Psi ; \Theta ; \Delta ; \Omega ; (\Gamma_1, x : \tau_1) \setminus \Gamma_2 \vdash |e_2| : \M \, (I,\vec{p} + \texttt{const}(J)) \, \tau_2$
  
  \caseText{By T-Weaken on (6)}
  
  \caseFact{7} $\Psi ; \Theta ; \Delta ; \Omega ; (\Gamma_1 \setminus (\Gamma_2 \setminus \{x : \tau_1\})), x :\tau_1\vdash |e_2| : \M \, (I,\vec{p} + \texttt{const}(J)) \, \tau_2$
  
  \caseText{By T-ReleaseConst on (5) and (7)}
  
  \caseFact{8}  $\Psi ; \Theta ; \Delta ; \Omega ; (\Gamma \setminus \Gamma_1), (\Gamma_1 \setminus (\Gamma_2 \setminus \{x\})) \vdash \texttt{release } x = |e_1| \texttt{ in }|e_2| : \M \, (I,\vec{p}) \, \tau_2$
  
  \caseText{Goal follows by T-Weaken on (8)}
 }
 
 \jcase{35}{AT-Shift}{
  \jgivengoal{
   \caseFact{1} $\Psi ; \Theta ; \Delta ; \Omega ; \Gamma \vdash \texttt{shift}(e) \checks M \, (I,\vec{q}) \, \tau \gens (I \geq 1) \wedge \Phi, \Gamma'$ 
   
   \caseFact{2} $\Theta ; \Delta \vDash (I \geq 1) \wedge \Phi$
   
   \caseFact{3} $\Psi ; \Theta ; \Delta  ; \Omega ; \Gamma \vdash e \checks M \, (I - 1,\lhd \vec{q}) \, \tau \gens \Phi, \Gamma'$ 
  }{
     $\Psi ; \Theta ; \Delta ; \Omega ; \Gamma \setminus \Gamma' \vdash \texttt{shift}(|e|) : M \, (I,\vec{q}) \, \tau$
  }
  
  \caseText{By IH on (3)}
  
  \caseFact{4} $\Psi ; \Theta ; \Delta  ; \Omega ; \Gamma \setminus \Gamma' \vdash |e| : M \, (I - 1,\lhd \vec{q}) \, \tau$
  
  \caseText{From (2)}
  
  \caseFact{5} $\Theta ; \Delta \vDash I \geq 1$
  
  \caseText{Goal follows by T-Shift on (4) and (5)}
 }
 
 \jcase{36}{AT-CImpI}{
  \jgivengoal{
   \caseFact{1} $\Psi ; \Theta ; \Delta ; \Omega ; \Gamma \vdash \Lambda .e \checks (\Phi' \Rightarrow \tau) \gens (\Phi' \to \Phi),\Gamma'$
   
   \caseFact{2} $\Theta ; \Delta \vDash \Phi' \to \Phi$  
   
   \caseFact{3} $\Psi ; \Theta ; \Delta,\Phi' ; \Omega ; \Gamma \vdash e \checks \tau \gens \Phi,\Gamma'$ 
  }{
   $\Psi ; \Theta ; \Delta ; \Omega ; \Gamma \setminus \Gamma' \vdash \Lambda .|e| : \Phi' \Rightarrow \tau$
  } 
  
  \caseText{Equivalently to (2)}
  
  \caseFact{4} $\Theta ; \Delta, \Phi' \vDash \Phi$
  
  \caseText{By IH on (3)}
  
  \caseFact{5} $\Psi ; \Theta ; \Delta,\Phi' ; \Omega ; \Gamma \setminus \Gamma' \vdash |e| : \tau$
  
  \caseText{Goal is immediate by T-CImpI on (5)}
 }
 
 \jcase{37}{AT-CImpE}{
  \jgivengoal{
   \caseFact{1} $\Psi ; \Theta ; \Delta ; \Omega ; \Gamma \vdash e \{\} \infers \tau \gens \Phi \wedge \Phi',\Gamma'$
   
   \caseFact{2} $\Theta ; \Delta \vDash \Phi \wedge \Phi'$  
   
   \caseFact{3} $\Psi ; \Theta ; \Delta ; \Omega ; \Gamma \vdash e \infers (\Phi' \Rightarrow \tau) \gens \Phi,\Gamma'$
  }{
   $\Psi ; \Theta ; \Delta ; \Omega ; \Gamma \setminus \Gamma' \vdash |e| \{\} : \tau$
  } 
  
  \caseText{By IH on (3)}
  
  \caseFact{4} $\Psi ; \Theta ; \Delta ; \Omega ; \Gamma \setminus \Gamma' \vdash |e| : (\Phi' \Rightarrow \tau)$
  
  \caseText{From (2)}
  
  \caseFact{5} $\Theta ; \Delta \vDash \Phi'$
  
  \caseText{Goal follows from T-CImpI on (4) and (5)}
 }
 
 \jcase{38}{AT-CAndI}{
  \jgivengoal{
   \caseFact{1} $\Psi ; \Theta ; \Delta ; \Omega ; \Gamma \vdash <e> \checks \Phi' \amp \tau \gens \Phi \wedge \Phi',\Gamma'$  
   
   \caseFact{2} $\Theta ; \Delta \vDash \Phi \wedge \Phi'$
   
   \caseFact{3} $\Psi ; \Theta ; \Delta ; \Omega ; \Gamma \vdash e \checks \tau \gens \Phi,\Gamma'$
  }{
   $\Psi ; \Theta ; \Delta ; \Omega ; \Gamma \setminus \Gamma' \vdash <|e|> : \Phi' \amp \tau$  
  }
  
  \caseText{By IH on (3)}
  
  \caseFact{4} $\Psi ; \Theta ; \Delta ; \Omega ; \Gamma \setminus \Gamma' \vdash |e| : \tau$
  
  \caseText{From (2)}
  
  \caseFact{5} $\Theta ; \Delta \vDash \Phi'$
  
  \caseText{Goal follows from T-CAndI on (4) and (5)}
 }
 
 \jcase{39}{AT-CAndE}{
  \jgivengoal{
   \caseFact{1} $\Psi ; \Theta ; \Delta ; \Omega ; \Gamma \vdash \texttt{clet } x = e \texttt{ in } e' \checks \tau' \gens \Phi_1 \wedge (\Phi' \to \Phi_2),\Gamma_2 \setminus \{x : \tau\}$
   
   \caseFact{2} $\Theta ; \Delta \vDash \Phi_1 \wedge (\Phi' \to \Phi_2)$  
   
   \caseFact{3} $\Psi ; \Theta ; \Delta ; \Omega ; \Gamma \vdash e \infers \Phi' \amp \tau \gens \Phi_1,\Gamma_1$
   
   \caseFact{4} $\Psi ; \Theta ; \Delta, \Phi' ; \Omega ; \Gamma_1, x : \tau \vdash e' \checks \tau' \gens \Phi_2, \Gamma_2$
  }{
     $\Psi ; \Theta ; \Delta ; \Omega ; \Gamma \setminus (\Gamma_2 \setminus \{x : \tau\}) \vdash \texttt{clet } x = |e| \texttt{ in } |e'| : \tau'$
  }
  
  \caseText{By IH on (3)}
  
  \caseFact{5} $\Psi ; \Theta ; \Delta ; \Omega ; \Gamma \setminus \Gamma_1 \vdash |e| : \Phi' \amp \tau$
  
  \caseText{From (2)}
  
  \caseFact{6} $\Theta ; \Delta, \Phi' \vDash \Phi_2$
  
  \caseText{By IH on (4) using (6)}
  
  \caseFact{7} $\Psi ; \Theta ; \Delta, \Phi' ; \Omega ; (\Gamma_1, x : \tau) \setminus \Gamma_2 \vdash |e'| : \tau'$
  
  \caseText{By T-Weaken on (7)}
  
  \caseFact{8} $\Psi ; \Theta ; \Delta, \Phi' ; \Omega ; (\Gamma_1 \setminus (\Gamma_2 \setminus \{x : \tau\})), x : \tau\vdash |e'| : \tau'$
  
  \caseText{By T-CAndE on (5) and (8)}
  
  \caseFact{9} $\Psi ; \Theta ; \Delta ; \Omega ; (\Gamma \setminus \Gamma_1), (\Gamma_1 \setminus (\Gamma_2 \setminus \{x : \tau\})) \vdash \texttt{clet } x = |e| \texttt{ in } |e'| : \tau'$
  
  \caseText{Goal follows by T-Weaken on (9)}
 }
 
 \jcase{40}{AT-Sub}{
  \jgivengoal{
   \caseFact{1} $\Psi ; \Theta ; \Delta ; \Omega ; \Gamma \vdash e \checks \tau \gens \Phi_1 \wedge \Phi_2,\Gamma'$
   
   \caseFact{2} $\Theta ; \Delta \vDash \Phi_1 \wedge \Phi_2$
   
   \caseFact{3} $\Psi ; \Theta ; \Delta ; \Omega ; \Gamma \vdash e \infers \tau' \gens \Phi_1,\Gamma'$
   
   \caseFact{4} $\Psi;\Theta;\Delta \vdash \tau' \subty \tau : \star \gens \Phi_2$
  }{
   $\Psi ; \Theta ; \Delta ; \Omega ; \Gamma\setminus \Gamma' \vdash |e| : \tau$
  }
  
  \caseText{By IH on (3)}
  
  \caseFact{5} $\Psi ; \Theta ; \Delta ; \Omega ; \Gamma \setminus \Gamma' \vdash |e| : \tau'$
  
  \caseText{By \autoref{thm:subty-sound} on (4)}
  
  \caseFact{6} $\Psi;\Theta;\Delta \vdash \tau' \subty \tau : \star$
  
  \caseText{Goal follows by T-Sub on (5) and (6)}
 }
 
  \jcase{41}{AT-Anno}{
   \jgivengoal{
    \caseFact{1} $\Psi ; \Theta ; \Delta ; \Omega ; \Gamma \vdash (e : \tau) \infers \tau \gens \Phi,\Gamma'$
    
    \caseFact{2} $\Theta ; \Delta \vDash \Phi$   
    
    \caseFact{3} $\Psi ; \Theta ; \Delta ; \Omega ; \Gamma \vdash e \checks \tau \gens \Phi,\Gamma'$
   }{
    $\Psi ; \Theta ; \Delta ; \Omega ; \Gamma \setminus \Gamma' \setminus \vdash |(e : \tau)| : \tau$   
   }
   
   \caseText{By IH on (3)}
   
   \caseFact{4} $\Psi ; \Theta ; \Delta ; \Omega ; \Gamma \setminus \Gamma' \vdash |e| : \tau$
   
   \caseText{Goal follows immediately by (4) since $|(e : \tau)| = |e|$}
  }
}


\tychecksound*
\begin{proof}
Immediate by \ref{thm:raw-tycheck-sound}.
\end{proof}

%%%
% COMPLETENESS
%%%



\begin{theorem}[Raw Completeness of Sort Checking/Inference]
If $\Theta;\Delta \vdash I : S$, then $\Theta;\Delta \vdash I : S \gens \Phi$ and $\Theta;\Delta \vDash \Phi$.
%\textbf{REDO THIS, FIX Proof file Name}
\label{thm:raw-sort-compl}
\end{theorem}
\begin{proof}
By induction on $\Theta;\Delta \vdash I : S$.

\begin{itemize}
  \item[(I-Var)] Immediate by AI-Var with $\Phi = \top$.
  \item[(I-Plus)] Suppose $\Theta ; \Delta \vdash I + J : bS$ from $\Theta ; \Delta \vdash I : bS$ and $\Theta ; \Delta \vdash J : bS$. By IH, $\Theta ; \Delta \vdash I : bS \gens \Phi_1$ and $\Theta ; \Delta \vdash J : bS \gens \Phi_2$ with $\Theta ; \Delta \vDash \Phi_1$ and $\Theta; \Delta \vDash  \Phi_2$. By AI-Plus, $\Theta ; \Delta \vdash I + J : bS \gens \Phi_1 \wedge \Phi_2$
  \item[(I-Minus)] Suppose $\Theta ; \Delta \vdash I - J : bS$ from $\Theta ; \Delta \vdash I : bS$ and $\Theta ; \Delta \vdash J : bS$, and $\Theta ; \Delta \vDash I \geq J$. By IH, $\Theta ; \Delta \vdash I : bS \gens \Phi_1$ and $\Theta ; \Delta \vdash J : bS \gens \Phi_2$ with $\Theta ; \Delta \vDash \Phi_1$ and $\Theta; \Delta \vDash  \Phi_2$. By AI-Minus, $\Theta ; \Delta \vdash I - J : bS \gens \Phi_1 \wedge \Phi_2 \wedge (I \geq J)$
  \item[(I-Times-$\mathbb{R}$)]
  \item[(I-Times-$\vec{\mathbb{R}}$)] 
  \item[(I-Times-$\mathbb{N}$)]
  \item[(I-Shift)] Suppose $\Theta ; \Delta \vdash \; \lhd I : \vec{\mathbb{R}^+}$ from  $\Theta ; \Delta \vdash I : \vec{\mathbb{R}^+}$. By IH, $\Theta ; \Delta \vdash I : \vec{\mathbb{R}^+} \gens \Phi$, and $\Theta ; \Delta \vDash \Phi$.  By AI-Shift,  $\Theta ; \Delta \vdash \; \lhd I : \vec{\mathbb{R}^+} \gens \Phi$, as required.
  \item[(I-Lam)] Suppose $\Theta ; \Delta \vdash \lambda i : bS. I : bS \to S$ from $\Theta, i : bS ; \Delta \vdash I : S$. By IH, $\Theta, i : bS ; \Delta \vdash I : S \gens \Phi$ with $\Theta ; \Delta \vDash  \Phi$. By AI-Lam, $\Theta ; \Delta \vdash \lambda i : bS. I : bS \to S \gens \Phi$.
  \item[(I-App)]
  \item[(I-Sum)]
\end{itemize}
\end{proof}

\begin{theorem}[Raw Completeness of Constraint Well-Formedness]
If $\Theta ; \Delta \vdash \Phi \; \texttt{wf}$, then $\Theta ; \Delta \vdash \Phi \; \texttt{wf} \gens \Phi'$ with $\Theta ; \Delta \vDash \Phi'$
\label{thm:raw-constr-compl}
\end{theorem}

\idxctxwfcompl*
\begin{proof}
By induction on $\Theta \vdash \Delta \; \texttt{wf}$. The base case is immediate.
Suppose $\Theta \vdash \Delta,\Phi \; \texttt{wf}$ by way of $\Theta \vdash \Delta \; \texttt{wf}$ and $\Theta ; \Delta \vdash \Phi \; \texttt{wf}$.
By IH, there is some $\Phi_1$ such that $\Theta \vdash \Delta \; \texttt{wf} \gens \Phi_1$ with $\Theta ; \cdot \vDash \Phi_1$. By Theorem~\ref{thm:raw-constr-compl}, there is $\Phi_2$ such that $\Theta ; \Delta \vdash \Phi \; \texttt{wf} \gens \Phi_2$ with $\Theta ; \Delta \vDash \Phi_2$. Equivalently, $\Theta ; \cdot \vDash \bigwedge \Delta \to \Phi_2$, and so $\Theta ; \cdot \vDash \Phi_1 \wedge (\bigwedge \Delta \to \Phi_2)$. Then, by AWF-CCtx-Ne, $\Theta \vdash \Delta,\Phi \; \texttt{wf} \gens \Phi_1 \wedge (\bigwedge \Delta \to \Phi_2)$, as required.
\end{proof}

\sortcompl*
\begin{proof}
Immediate by Theorem~\ref{thm:raw-sort-compl} and Theorem~\ref{thm:idx-ctx-wf-compl}
\end{proof}

\constrcompl*
\begin{proof}
Immediate by Theorem~\ref{thm:raw-constr-compl} and Theorem~\ref{thm:idx-ctx-wf-compl}
\end{proof}

\begin{theorem}[Raw Completeness of Kind Checking/Inference]
If $\Phi;\Theta;\Delta \vdash \tau : K$, then $\Phi;\Theta;\Delta \vdash \tau : K \gens \Phi$ with $\Theta ; \Delta \vDash \Phi$
\label{thm:raw-kind-compl}
\end{theorem}
\jtheorem{Proof of \autoref{thm:raw-kind-compl}}{
  \jgivengoal{
    \caseFact{1} $\Psi ; \Theta ; \Delta \vdash \tau : K$  
  }{
    $\Phi;\Theta;\Delta \vdash \tau : K \gens \Phi$ and $\Theta ; \Delta \vDash \Phi$  
  }
  
  \jcase{1}{K-Var}{Immediate.}
  \jcase{2}{K-Unit}{Immediate.}
  \jcase{3}{K-Arr}{
    \jgivengoal{
      \caseFact{1} $\Psi ; \Theta ; \Delta \vdash \tau_1 \loli \tau_2 : \star$
      
      \caseFact{2} $\Psi ; \Theta ; \Delta \vdash \tau_1 : \star$
      
      \caseFact{3} $\Psi ; \Theta ; \Delta \vdash \tau_2 : \star$      
    }{
       $\Psi ; \Theta ; \Delta \vdash \tau_1 \loli \tau_2 : \star \gens \Phi'$ and $\Theta ; \Delta \vDash \Phi'$
    }
    \caseText{By IH on (2) and (3)}
    
    \caseFact{4} $\Psi ; \Theta ; \Delta \vdash \tau_1 : \star \gens \Phi_1$
    
    \caseFact{5} $\Psi ; \Theta ; \Delta \vdash \tau_2 : \star \gens \Phi_2$
    
    \caseFact{6} $\Theta ; \Delta \vDash \Phi_1 \wedge \Phi_2$
    
    \caseText{By AK-Arr on (4) and (5)}
    
    \caseFact{7} $\Psi ; \Theta ; \Delta \vdash \tau_1 \loli \tau_2 : \star \gens \Phi_1 \wedge \Phi_2$
    
    \caseText{The Goal follows by (6) and (7), with $\Phi' = \Phi_1 \wedge \Phi_2$}
  }
  
  \jcase{4}{K-Tensor}{
    \jgivengoal{
      \caseFact{1} $\Psi ; \Theta ; \Delta \vdash \tau_1 \otimes \tau_2 : \star$
      
      \caseFact{2} $\Psi ; \Theta ; \Delta \vdash \tau_1 : \star$
      
      \caseFact{3} $\Psi ; \Theta ; \Delta \vdash \tau_2 : \star$
    }{
       $\Psi ; \Theta ; \Delta \vdash \tau_1 \otimes \tau_2 : \star \gens \Phi'$ and $\Theta ; \Delta \vDash \Phi'$
    }
    \caseText{By IH on (2) and (3)}
    
    \caseFact{4} $\Psi ; \Theta ; \Delta \vdash \tau_1 : \star \gens \Phi_1$
    
    \caseFact{5} $\Psi ; \Theta ; \Delta \vdash \tau_2 : \star \gens \Phi_2$
    
    \caseFact{6} $\Theta ; \Delta \vDash \Phi_1 \wedge \Phi_2$
    
    \caseText{By AK-Tensor on (4) and (5)}
    
    \caseFact{7} $\Psi ; \Theta ; \Delta \vdash \tau_1 \otimes \tau_2 : \star \gens \Phi_1 \wedge \Phi_2$
    
    \caseText{The Goal follows by (6) and (7), with $\Phi' = \Phi_1 \wedge \Phi_2$}
  }
  
  \jcase{5}{K-With}{
    \jgivengoal{
      \caseFact{1} $\Psi ; \Theta ; \Delta \vdash \tau_1 \amp \tau_2 : \star$
      
      \caseFact{2} $\Psi ; \Theta ; \Delta \vdash \tau_1 : \star$
      
      \caseFact{3} $\Psi ; \Theta ; \Delta \vdash \tau_2 : \star$
    }{
       $\Psi ; \Theta ; \Delta \vdash \tau_1 \amp \tau_2 : \star \gens \Phi'$ and $\Theta ; \Delta \vDash \Phi'$
    }
    \caseText{By IH on (2) and (3)}
    
    \caseFact{4} $\Psi ; \Theta ; \Delta \vdash \tau_1 : \star \gens \Phi_1$
    
    \caseFact{5} $\Psi ; \Theta ; \Delta \vdash \tau_2 : \star \gens \Phi_2$
    
    \caseFact{6} $\Theta ; \Delta \vDash \Phi_1 \wedge \Phi_2$
    
    \caseText{By K-With on (4) and (5)}
    
    \caseFact{7} $\Psi ; \Theta ; \Delta \vdash \tau_1 \amp \tau_2 : \star \gens \Phi_1 \wedge \Phi_2$
    
    \caseText{The Goal follows by (6) and (7), with $\Phi' = \Phi_1 \wedge \Phi_2$}
  }
  
  \jcase{6}{K-Sum}{
    \jgivengoal{
      \caseFact{1} $\Psi ; \Theta ; \Delta \vdash \tau_1 \oplus \tau_2 : \star$
      
      \caseFact{2} $\Psi ; \Theta ; \Delta \vdash \tau_1 : \star$
      
      \caseFact{3} $\Psi ; \Theta ; \Delta \vdash \tau_2 : \star$
    }{
       $\Psi ; \Theta ; \Delta \vdash \tau_1 \oplus \tau_2 : \star \gens \Phi'$ and $\Theta ; \Delta \vDash \Phi'$
    }
    \caseText{By IH on (2) and (3)}
    
    \caseFact{4} $\Psi ; \Theta ; \Delta \vdash \tau_1 : \star \gens \Phi_1$
    
    \caseFact{5} $\Psi ; \Theta ; \Delta \vdash \tau_2 : \star \gens \Phi_2$
    
    \caseFact{6} $\Theta ; \Delta \vDash \Phi_1 \wedge \Phi_2$
    
    \caseText{By AK-Sum on (4) and (5)}
    
    \caseFact{7} $\Psi ; \Theta ; \Delta \vdash \tau_1 \oplus \tau_2 : \star \gens \Phi_1 \wedge \Phi_2$
    
    \caseText{The Goal follows by (6) and (7), with $\Phi' = \Phi_1 \wedge \Phi_2$}
  }
  
  \jcase{7}{K-Bang}{
    \jgivengoal{
      \caseFact{1} $\Psi ; \Theta ; \Delta \vdash !\tau: \star$
      \caseFact{2} $\Psi ; \Theta ; \Delta \vdash \tau : \star$
    }{
       $\Psi ; \Theta ; \Delta \vdash !\tau : \star \gens \Phi'$ and $\Theta ; \Delta \vDash \Phi'$
    }
    \caseText{By IH on (2)}
    
    \caseFact{3} $\Psi ; \Theta ; \Delta \vdash \tau : \star \gens \Phi'$
    
    \caseFact{4} $\Theta ; \Delta \vDash \Phi'$
    
    \caseText{By AK-Bang on (3)}
    
    \caseFact{4} $\Psi ; \Theta ; \Delta \vdash !\tau : \star \gens \Phi'$
  }
  
  
  \jcase{8}{K-IForall}{
    \jgivengoal{
      \caseFact{1} $\Psi ; \Theta ; \Delta \vdash \forall i : S. \tau : \star$
      
      \caseFact{2} $\Psi ; \Theta, i : S ; \Delta \vdash \tau : \star$
    }{
      $\Psi ; \Theta ; \Delta \vdash \forall i : S. \tau : \star \gens \Phi$ and $\Theta ; \Delta \vDash \Phi$    
    }
    \caseText{By IH on (2)}
    
    \caseFact{3} $\Psi ; \Theta, i : S ; \Delta \vdash \tau : \star \gens \Phi$
    
    \caseFact{4} $\Theta, i : S ; \Delta \vDash \Phi$
    
    \caseText{Equivalently to (4)}
    
    \caseFact{5} $\Theta ; \Delta \vDash \forall i : S. \Phi$
    
    \caseText{By AK-IForall on (3)}
    
    \caseFact{6} $\Psi ; \Theta ; \Delta \vdash \forall i : S. \tau : \star \gens \forall i : S. \Phi$
  }
  
  \jcase{9}{K-IExists}{
    \jgivengoal{
      \caseFact{1} $\Psi ; \Theta ; \Delta \vdash \exists i : S. \tau : \star$
      
      \caseFact{2} $\Psi ; \Theta, i : S ; \Delta \vdash \tau : \star$
    }{
      $\Psi ; \Theta ; \Delta \vdash \exists i : S. \tau : \star \gens \Phi$ and $\Theta ; \Delta \vDash \Phi$    
    }
    \caseText{By IH on (2)}
    
    \caseFact{3} $\Psi ; \Theta, i : S ; \Delta \vdash \tau : \star \gens \Phi$
    
    \caseFact{4} $\Theta, i : S ; \Delta \vDash \Phi$
    
    \caseText{Equivalently to (4)}
    
    \caseFact{5} $\Theta ; \Delta \vDash \forall i : S. \Phi$
    
    \caseText{By AK-IExists on (3)}
    
    \caseFact{6} $\Psi ; \Theta ; \Delta \vdash \exists i : S. \tau : \star \gens \forall i : S. \Phi$
  }
  
  \jcase{10}{K-List}{
    \jgivengoal{
      \caseFact{1} $\Psi ; \Theta ; \Delta \vdash L^I \tau : \star$
      
      \caseFact{2} $\Theta ; \Delta \vdash I : \N$
      
      \caseFact{3} $\Psi ; \Theta ; \Delta \vdash \tau : \star$
    }{
      $\Psi ; \Theta ; \Delta \vdash L^I \tau : \star \gens \Phi$ and $\Theta ; \Delta \vDash \Phi$
    }
    \caseText{By IH on (3)}
    
    \caseFact{4} $\Psi ; \Theta ; \Delta \vdash \tau : \star \gens \Phi_1$
    
    \caseFact{5} $\Theta ; \Delta \vDash \Phi_1$
    
    \caseText{By \autoref{thm:raw-sort-compl} on (2)}
    
    \caseFact{6} $\Theta ; \Delta \vdash I : \N \gens \Phi_2$
    
    \caseFact{7} $\Theta ; \Delta \vDash \Phi_2$
    
    \caseText{By AK-List on (4) and (6)}
    
    \caseFact{8} $\Psi ; \Theta ; \Delta \vdash L^I \tau : \star \gens \Phi_1 \wedge \Phi_2$
    
    \caseText{The goal follows from (5), (7), (8) with $\Phi = \Phi_1 \wedge \Phi_2$}
  }
  
  \jcase{11}{K-Conj}{
    \jgivengoal{
      \caseFact{1} $\Psi ; \Theta ; \Delta \vdash \Phi \amp \tau : \star$
      
      \caseFact{2} $\Theta ; \Delta \vdash \Phi \texttt{ wf}$
      
      \caseFact{3} $\Psi ; \Theta ; \Delta \vdash \tau : \star$
    }{
      $\Psi ; \Theta ; \Delta \vdash \Phi \amp \tau : \star \gens \Phi$ and $\Theta ; \Delta \vDash \Phi$
    }
    \caseText{By \autoref{thm:raw-constr-compl} on (2)}
    
    \caseFact{4} $\Theta ; \Delta \vdash \Phi \; \texttt{wf} \gens \Phi_1$
    
    \caseFact{5} $\Theta ; \Delta \vDash \Phi_!$
    
    \caseText{By IH on (3)}
    
    \caseFact{6} $\Psi ; \Theta ; \Delta \vdash \tau : \star \gens \Phi_2$
    
    \caseFact{7} $\Theta ; \Delta \vDash \Phi_2$
    
    \caseText{By AK-Conj on (4) and (6)}
    
    \caseFact{8} $\Psi ; \Theta ; \Delta \vdash \Phi \amp \tau : \star \gens \Phi_1 \wedge \Phi_2$
    
    \caseText{The Goal follows by (5), (7), (8)}
  }
  
  \jcase{12}{K-Impl}{
    \jgivengoal{
      \caseFact{1} $\Psi ; \Theta ; \Delta \vdash \Phi \implies \tau : \star$
      
      \caseFact{2} $\Theta ; \Delta \vdash \Phi \texttt{ wf}$
      
      \caseFact{3} $\Psi ; \Theta ; \Delta \vdash \tau : \star$
    }{
      $\Psi ; \Theta ; \Delta \vdash \Phi \implies \tau : \star \gens \Phi$ and $\Theta ; \Delta \vDash \Phi$
    }
    \caseText{By \autoref{thm:raw-constr-compl} on (2)}
    
    \caseFact{4} $\Theta ; \Delta \vdash \Phi \; \texttt{wf} \gens \Phi_1$
    
    \caseFact{5} $\Theta ; \Delta \vDash \Phi_!$
    
    \caseText{By IH on (3)}
    
    \caseFact{6} $\Psi ; \Theta ; \Delta \vdash \tau : \star \gens \Phi_2$
    
    \caseFact{7} $\Theta ; \Delta \vDash \Phi_2$
    
    \caseText{By AK-Impl on (4) and (6)}
    
    \caseFact{8} $\Psi ; \Theta ; \Delta \vdash \Phi \implies \tau : \star \gens \Phi_1 \wedge \Phi_2$
    
    \caseText{The Goal follows by (5), (7), (8)}
  }
  

  \jcase{13}{K-Monad}{
    \jgivengoal{
      \caseFact{1} $\Psi ; \Theta ; \Delta \vdash \M(I,\vec{p}) \tau : \star$
      
      \caseFact{2} $\Theta ; \Delta \vdash I : \N$
      
      \caseFact{3} $\Theta ; \Delta \vdash \vec{p} : \potvec$
      
      \caseFact{4} $\Psi ; \Theta ; \Delta \vdash \tau : \star$
    }{
      $\Psi ; \Theta ; \Delta \vdash \M(I,\vec{p}) \tau : \star \gens \Phi$ and $\Theta ; \Delta \vDash \Phi$
    }
    \caseText{By \autoref{thm:raw-sort-compl} on (2) and (3)}
    
    \caseFact{5} $\Theta ; \Delta \vdash I : \N \gens \Phi_1$
    
    \caseFact{6} $\Theta ; \Delta \vdash \vec{p} : \potvec \gens \Phi_2$
    
    \caseFact{7} $\Theta ; \Delta \vDash \Phi_1 \wedge \Phi_2$
    
    \caseText{By IH on (4)}
    
    \caseFact{8} $\Psi ; \Theta ; \Delta \vdash \tau : \star \gens \Phi_3$
    
    \caseFact{9} $\Theta ; \Delta \vDash \Phi_3$
    
    \caseText{By AK-Monad on (5), (6), (8)}
    
    \caseFact{10} $\Psi ; \Theta ; \Delta \vdash \M(I,\vec{p}) \tau : \star \gens \Phi_1 \wedge \Phi_2 \wedge \Phi_3$
    
    \caseText{Goal follows by (7), (9), (10)}
  
  }
  
  \jcase{14}{K-Pot}{
    \jgivengoal{
      \caseFact{1} $\Psi ; \Theta ; \Delta \vdash [I|\vec{p}] \tau : \star$
      
      \caseFact{2} $\Theta ; \Delta \vdash I : \N$
      
      \caseFact{3} $\Theta ; \Delta \vdash \vec{p} : \potvec$
      
      \caseFact{4} $\Psi ; \Theta ; \Delta \vdash \tau : \star$
    }{
      $\Psi ; \Theta ; \Delta \vdash [I|\vec{p}] \tau : \star \gens \Phi$ and $\Theta ; \Delta \vDash \Phi$
    }
    \caseText{By \autoref{thm:raw-sort-compl} on (2) and (3)}
    
    \caseFact{5} $\Theta ; \Delta \vdash I : \N \gens \Phi_1$
    
    \caseFact{6} $\Theta ; \Delta \vdash \vec{p} : \potvec \gens \Phi_2$
    
    \caseFact{7} $\Theta ; \Delta \vDash \Phi_1 \wedge \Phi_2$
    
    \caseText{By IH on (4)}
    
    \caseFact{8} $\Psi ; \Theta ; \Delta \vdash \tau : \star \gens \Phi_3$
    
    \caseFact{9} $\Theta ; \Delta \vDash \Phi_3$
    
    \caseText{By AK-Pot on (5), (6), (8)}
    
    \caseFact{10} $\Psi ; \Theta ; \Delta \vdash [I|\vec{p}] \tau : \star \gens \Phi_1 \wedge \Phi_2 \wedge \Phi_3$
    
    \caseText{Goal follows by (7), (9), (10)}  
  }
  
  \jcase{15}{K-ConstPot}{
  
    \jgivengoal{
      \caseFact{1} $\Psi ; \Theta ; \Delta \vdash [I] \; \tau : \star$
      
      \caseFact{2} $\Theta ; \Delta \vdash I : \R^+$
      
      \caseFact{3} $\Psi ; \Theta ; \Delta \vdash \tau : \star$
    }{
     $\Psi ; \Theta ; \Delta \vdash [I] \; \tau : \star \gens \Phi$ and $\Theta ; \Delta \vDash \Phi$
    }
    
    \caseText{By \autoref{thm:raw-sort-compl} on  (2)}
    
    \caseFact{4} $\Theta ; \Delta \vdash I : \R^+ \gens \Phi_1$
    
    \caseFact{5} $\Theta ; \Delta \vDash \Phi_1$
    
    \caseText{By IH on (3)}
    
    \caseFact{6} $\Psi ; \Theta ; \Delta \vdash \tau : \star \gens \Phi_2$
    
    \caseFact{7} $\Theta ; \Delta \vDash \Phi_2$
    
    \caseText{Applying AK-ConstPot to (4) and (6)}
    
    \caseFact{8} $\Psi ; \Theta ; \Delta \vdash [I] \; \tau : \star \gens \Phi_1 \wedge \Phi_2$
    
    \caseText{Goal follows by (5), (7), (8)}
 
   }
   
   \jcase{16}{K-FamLam}{
     \jgivengoal{
       \caseFact{1} $\Psi ; \Theta ; \Delta \vdash \lambda i : S. \tau : S \to K$
       
       \caseFact{2} $\Psi ; \Theta, i : S ; \Delta \vdash \tau : K$
     }{
       $\Psi ; \Theta ; \Delta \vdash \lambda i : S. \tau : S \to K \gens \Phi$ and $\Theta ; \Delta \vDash \Phi$
     }
     \caseText{By IH on (2)}
     
     \caseFact{3} $\Psi ; \Theta, i : S; \Delta \vdash \tau : K \gens \Phi$
     
     \caseFact{4} $\Theta, i : S; \Delta \vDash \Phi$
     
     \caseText{By AK-FamLam on (3)}
     
     \caseFact{5} $\Psi ; \Theta ; \Delta \vdash \lambda i : S. \tau : S \to K \gens \forall i : S. \Phi$
     
     \caseText{Equivalently to (4)}
     
     \caseFact{6} $\Theta ; \Delta \vDash \forall i : S. \Phi$
   }
   
  
  \jcase{17}{K-FamApp}{
    \jgivengoal{
      \caseFact{1} $\Psi ; \Theta ; \Delta \vdash \tau \; I : K$
      
      \caseFact{2} $\Psi ; \Theta ; \Delta \vdash \tau : S \to K$
      
      \caseFact{3} $\Theta ; \Delta \vdash I : S$
    }{
      $\Psi ; \Theta ; \Delta \vdash \tau \; I : K \gens \Phi$ and $\Theta ; \Delta \vDash \Phi$    
    }
    \caseText{By IH on (2)}
    
    \caseFact{4} $\Psi ; \Theta ; \Delta \vdash \tau : S \to K \gens \Phi_1$
    
    \caseFact{5} $\Theta ; \Delta \vDash \Phi_1$
    
    \caseText{By \autoref{thm:raw-sort-compl} on (3)}
    
    \caseFact{6} $\Theta ; \Delta \vdash I : S \gens \Phi_2$
    
    \caseFact{7} $\Theta ; \Delta \vDash \Phi_2$
    
    \caseText{By AK-FamApp on (4) and (6)}
    
    \caseFact{8} $\Psi ; \Theta ; \Delta \vdash \tau \; I : K \gens \Phi_1 \wedge \Phi_2$
    
    \caseText{Goal is done by (5), (7), and (8)}
  }
}   

\kindcompl*
\begin{proof}
Immediate by Theorem~\ref{thm:raw-kind-compl} and Theorem~\ref{thm:idx-ctx-wf-compl}.
\end{proof}

\subtynerefl*
\jtheorem{Proof of Theorem~\ref{thm:subtyne-refl}}{

\jgivengoal{
  \caseFact{1} $\Psi ; \Theta ;  \Delta \pvdash \tau : K$
  
  \caseFact{2} $\tau \;\texttt{nf}$
}{
  $\Psi ; \Theta ;  \Delta \pvdash \tau\subtynf \tau : K \gens \Phi$ with $\Theta ; \Delta \vDash \Phi$
}

\caseText{By induction on (1) and inversion on (2)}

\jcase{1}{K-Var}{Immediate}

\jcase{2}{K-Unit}{Immediate}

\jcase{3}{K-FamApp}{
  \jgivengoal{
    \caseFact{1} ${\Psi ; \Theta ; \Delta \pvdash \tau \; I : K}$
    
    \caseFact{2} $\tau \; I \; \texttt{ne}$
    
    \caseFact{3} $\Psi ; \Theta ; \Delta \vdash \tau : S \to K$
    
    \caseFact{4} $\Theta ; \Delta \vdash I : S$
  }{
    $\Psi ; \Theta ;  \Delta \pvdash \tau \; I \subtynf \tau \; I: K \gens \Phi$ with $\Theta ; \Delta \vDash \Phi$
  }
  
  \caseText{By inversion on (2)}
  
  \caseFact{5} $\tau \; \texttt{ne}$
  
  \caseText{By IH on (3)}
  
  \caseFact{6} $\Psi ; \Theta ; \Delta \pvdash \tau \subtynf \tau : S \to K \gens \Phi$
  
  \caseFact{7} $\Theta ; \Delta \vDash \Phi$
  
  \caseText{By AK-FamApp on (6)}
  
  \caseFact{8} $\Psi ; \Theta ; \Delta \vdash \tau \; I \subtynf \tau \; I : K \gens \Phi \wedge (I = I)$
  
  \caseText{We re-establish the presupposition for (8) by applying Theorem~\ref{thm:sort-sound} to $\Theta ; \Delta \vdash I : S$ from (1)}
  
  \caseFact{9} $\Psi ; \Theta ; \Delta \pvdash \tau \; I \subtynf \tau \; I : K \gens \Phi \wedge (I = I)$
  
  \caseText{By (7), (9), and the fact that $\Theta ; \Delta \vDash I = I$}
}
  
}

\iffalse
\begin{proof}
By induction on $\Psi ; \Theta ;  \Delta \vdash \tau : K$. Inverting $\tau \;\texttt{ne}$, we see that the only possible cases are K-Var, K-Unit, and K-FamApp. The K-Var and K-Unit cases are immediate by AS-Var and AS-Unit.

Otherwise, suppose ${\Psi ; \Theta ; \Delta \vdash \tau \; I : K}$ from $\Psi ; \Theta ; \Delta \vdash \tau : S \to K$ with $\Theta ; \Delta \vdash I : S$. 

Here, $\tau \; I \; \texttt{ne}$ by way of $\tau \texttt{ne}$, and so by IH, $\Psi ; \Theta ; \Delta \vdash \tau \subty \tau : S \to K \gens \Phi$ with $\Theta ; \Delta \vDash \Phi$. By AS-FamApp, $\Psi ; \Theta ; \Delta \vdash \tau \; I \subty \tau \; I: S \to K \gens \Phi\wedge (I=I)$, and of course $\Theta ; \Delta \vDash I = I$, and so we are done.
\end{proof}
\fi

\subtynfrefl*
\jtheorem{Proof of Theorem~\ref{thm:subtynf-refl}}{

  \jgivengoal{
    \caseFact{1} $\Psi ; \Theta ;  \Delta \pvdash \tau : K$
    
    \caseFact{2} $\tau \;\texttt{nf}$  
  }{
    $\Psi ; \Theta ;  \Delta \pvdash \tau\subtynf \tau : K \gens \Phi$ with $\Theta ; \Delta \vDash \Phi$
  }
  \caseText{By induction on (1), followed by inversion on (2)}
  
  \jcase{1}{K-Var}{Immediate by Theorem~\ref{thm:subtyne-refl}}
  
  \jcase{2}{K-Unit}{Immediate by Theorem~\ref{thm:subtyne-refl}}
  
  \jcase{3}{K-Arr}{
    \jgivengoal{
      \caseFact{1} ${\Psi ; \Theta ; \Delta \pvdash \tau_1 \loli \tau_2 : \star}$
      
      \caseFact{2} $\tau_1$ and $\tau_2$ \texttt{nf}
      
      \caseFact{3} ${\Psi ; \Theta ; \Delta \vdash \tau_1 : \star}$
      
      \caseFact{4} ${\Psi ; \Theta ; \Delta \vdash \tau_2 : \star}$
    }{
      $\Psi ; \Theta ; \Delta \pvdash \tau_1 \loli \tau_2 \subtynf \tau_1 \loli \tau_2 : \star \gens \Phi$ and $\Theta ; \Delta \vDash \Phi$
    }
    
    \caseText{By IH on (3) and (4)}
    
    \caseFact{5} $\Psi ; \Theta ; \Delta \pvdash \tau_1 \subtynf \tau_1 : \star \gens \Phi_1$
    
    \caseFact{6} $\Psi ; \Theta ; \Delta \pvdash \tau_2 \subtynf \tau_2 : \star \gens \Phi_2$
    
    \caseFact{7} $\Theta ; \Delta \vDash \Phi_1 \wedge \Phi_2$
    
    \caseText{By AS-Arr on (5) and (6)}
    
    \caseFact{8} $\Psi ; \Theta ; \Delta \pvdash \tau_1 \loli \tau_2 \subtynf \tau_1 \loli \tau_2 : \star \gens \Phi_1 \wedge \Phi_2$
    
    \caseText{Goal is immediate from (7), (8)}
  }
  
  
  \jcase{4}{K-Tensor}{
    \jgivengoal{
      \caseFact{1} ${\Psi ; \Theta ; \Delta \pvdash \tau_1 \otimes \tau_2 : \star}$
      
      \caseFact{2} $\tau_1$ and $\tau_2$ \texttt{nf}
      
      \caseFact{3} ${\Psi ; \Theta ; \Delta \vdash \tau_1 : \star}$
      
      \caseFact{4} ${\Psi ; \Theta ; \Delta \vdash \tau_2 : \star}$
    }{
      $\Psi ; \Theta ; \Delta \pvdash \tau_1 \otimes \tau_2 \subtynf \tau_1 \otimes \tau_2 : \star \gens \Phi$ and $\Theta ; \Delta \vDash \Phi$
    }
    
    \caseText{By IH on (3) and (4)}
    
    \caseFact{5} $\Psi ; \Theta ; \Delta \pvdash \tau_1 \subtynf \tau_1 : \star \gens \Phi_1$
    
    \caseFact{6} $\Psi ; \Theta ; \Delta \pvdash \tau_2 \subtynf \tau_2 : \star \gens \Phi_2$
    
    \caseFact{7} $\Theta ; \Delta \vDash \Phi_1 \wedge \Phi_2$
    
    \caseText{By AS-Tensor on (5) and (6)}
    
    \caseFact{8} $\Psi ; \Theta ; \Delta \pvdash \tau_1 \otimes \tau_2 \subtynf \tau_1 \otimes \tau_2 : \star \gens \Phi_1 \wedge \Phi_2$
    
    \caseText{Goal is immediate from (7), (8)}
  }
  
  \jcase{5}{K-With}{
    \jgivengoal{
      \caseFact{1} ${\Psi ; \Theta ; \Delta \pvdash \tau_1 \amp \tau_2 : \star}$
      
      \caseFact{2} $\tau_1$ and $\tau_2$ \texttt{nf}
      
      \caseFact{3} ${\Psi ; \Theta ; \Delta \vdash \tau_1 : \star}$
      
      \caseFact{4} ${\Psi ; \Theta ; \Delta \vdash \tau_2 : \star}$
    }{
      $\Psi ; \Theta ; \Delta \pvdash \tau_1 \amp \tau_2 \subtynf \tau_1 \amp \tau_2 : \star \gens \Phi$ and $\Theta ; \Delta \vDash \Phi$
    }
    
    \caseText{By IH on (3) and (4)}
    
    \caseFact{5} $\Psi ; \Theta ; \Delta \pvdash \tau_1 \subtynf \tau_1 : \star \gens \Phi_1$
    
    \caseFact{6} $\Psi ; \Theta ; \Delta \pvdash \tau_2 \subtynf \tau_2 : \star \gens \Phi_2$
    
    \caseFact{7} $\Theta ; \Delta \vDash \Phi_1 \wedge \Phi_2$
    
    \caseText{By AS-With on (5) and (6)}
    
    \caseFact{8} $\Psi ; \Theta ; \Delta \pvdash \tau_1 \amp \tau_2 \subtynf \tau_1 \amp \tau_2 : \star \gens \Phi_1 \wedge \Phi_2$
    
    \caseText{Goal is immediate from (7), (8)}
  }
  
  \jcase{6}{K-Sum}{
    \jgivengoal{
      \caseFact{1} ${\Psi ; \Theta ; \Delta \pvdash \tau_1 \oplus \tau_2 : \star}$
      
      \caseFact{2} $\tau_1$ and $\tau_2$ \texttt{nf}
      
      \caseFact{3} ${\Psi ; \Theta ; \Delta \vdash \tau_1 : \star}$
      
      \caseFact{4} ${\Psi ; \Theta ; \Delta \vdash \tau_2 : \star}$
    }{
      $\Psi ; \Theta ; \Delta \pvdash \tau_1 \oplus \tau_2 \subtynf \tau_1 \oplus \tau_2 : \star \gens \Phi$ and $\Theta ; \Delta \vDash \Phi$
    }
    
    \caseText{By IH on (3) and (4)}
    
    \caseFact{5} $\Psi ; \Theta ; \Delta \pvdash \tau_1 \subtynf \tau_1 : \star \gens \Phi_1$
    
    \caseFact{6} $\Psi ; \Theta ; \Delta \pvdash \tau_2 \subtynf \tau_2 : \star \gens \Phi_2$
    
    \caseFact{7} $\Theta ; \Delta \vDash \Phi_1 \wedge \Phi_2$
    
    \caseText{By AS-Sum on (5) and (6)}
    
    \caseFact{8} $\Psi ; \Theta ; \Delta \pvdash \tau_1 \oplus \tau_2 \subtynf \tau_1 \oplus \tau_2 : \star \gens \Phi_1 \wedge \Phi_2$
    
    \caseText{Goal is immediate from (7), (8)}
  }
  
  \jcase{7}{K-Bang}{
    \jgivengoal{
      \caseFact{1} $\Psi ; \Theta ; \Delta \pvdash !\tau : \star$
      
      \caseFact{2} $\tau \; \texttt{nf}$    
      
      \caseFact{3} $\Psi ; \Theta ; \Delta \vdash \tau : \star$
    }{
      $\Psi ; \Theta ; \Delta \pvdash !\tau \subtynf !\tau : \star \gens \Phi$ and $\Theta ; \Delta \vDash \Phi$
    }
    
    \caseText{By IH on (3)}
    
    \caseFact{4} $\Psi ; \Theta ; \Delta \pvdash \tau \subtynf \tau : \star \gens \Phi$
    
    \caseFact{5} $\Theta ; \Delta \vDash \Phi$
    
    \caseText{By AS-Bang on (4)}
    
    \caseFact{6} $\Psi ; \Theta ; \Delta \pvdash !\tau \subtynf !\tau : \star \gens \Phi$
  }
  
  \jcase{8}{K-IForall}{
    \jgivengoal{
      \caseFact{1} ${\Psi ; \Theta ; \Delta \vdash \forall i : S. \tau : \star}$
      
      \caseFact{2} $\tau \; \texttt{nf}$
      
      \caseFact{3} $\Psi ; \Theta, i : S ; \Delta \vdash \tau : \star$
    }{
      ${\Psi ; \Theta ; \Delta \vdash \forall i :S .\tau \subtynf \forall i : S. \tau: \star \gens \Phi}$ and $\Theta ; \Delta \vDash \Phi$
    }
    
    \caseText{By IH on (3)}
    
    \caseFact{4} $\Psi ; \Theta, i : S ; \Delta \pvdash \tau \subtynf \tau : \star \gens \Phi$
    
    \caseFact{5} $\Theta, i : S;\Delta \vDash \Phi$
    
    \caseText{By AS-IForall on (4)}
    
    \caseFact{6} $\Psi ; \Theta ; \Delta \pvdash \forall i : S.\tau \subty \forall i : S.\tau : \star \gens \forall i : S. \Phi$
    
    \caseText{Equivalently to (5)}
    
    \caseFact{7} $\Theta ; \Delta \vDash \forall i : S. \Phi$
  }
  
  \jcase{9}{K-IExists}{
    \jgivengoal{
      \caseFact{1} ${\Psi ; \Theta ; \Delta \vdash \exists i : S. \tau : \star}$
      
      \caseFact{2} $\tau \; \texttt{nf}$
      
      \caseFact{3} $\Psi ; \Theta, i : S ; \Delta \vdash \tau : \star$
    }{
      ${\Psi ; \Theta ; \Delta \vdash \exists i :S .\tau \subtynf \exists i : S. \tau: \star \gens \Phi}$ and $\Theta ; \Delta \vDash \Phi$
    }
    
    \caseText{By IH on (3)}
    
    \caseFact{4} $\Psi ; \Theta, i : S ; \Delta \pvdash \tau \subtynf \tau : \star \gens \Phi$
    
    \caseFact{5} $\Theta, i : S;\Delta \vDash \Phi$
    
    \caseText{By AS-IExists on (4)}
    
    \caseFact{6} $\Psi ; \Theta ; \Delta \pvdash \exists i : S.\tau \subty \exists i : S.\tau : \star \gens \forall i : S. \Phi$
    
    \caseText{Equivalently to (5)}
    
    \caseFact{7} $\Theta ; \Delta \vDash \forall i : S. \Phi$
  }
  
  \jcase{10}{K-TForall}{
    \jgivengoal{
      \caseFact{1} ${\Psi ; \Theta ; \Delta \vdash \forall \alpha : K. \tau : \star}$
      
      \caseFact{2} $\tau \; \texttt{nf}$
      
      \caseFact{3} $\Psi, \alpha : K ; \Theta ; \Delta \vdash \tau : \star$
    }{
      $\Psi ; \Theta ; \Delta \vdash \forall \alpha : K. \tau \subtynf \forall \alpha : K. \tau : \star \gens \Phi$ and $\Theta ; \Delta \vDash \Phi$
    }
    
    \caseText{By IH on (3)}
    
    \caseFact{4} $\Psi , \alpha : K ; \Theta ; \Delta \pvdash \tau \subtynf \tau : \star \gens \Phi$
    
    \caseFact{5} $\Theta ; \Delta \vDash \Phi$
    
    \caseText{By AS-TForall on (4)}
    
    \caseFact{6} $\Psi ; \Theta ; \Delta \vdash \forall \alpha : K. \tau \subtynf \forall \alpha : K. \tau : \star \gens \Phi$
  }
  
  \jcase{11}{K-List}{
    \jgivengoal{
      \caseFact{1} ${\Psi ; \Theta ; \Delta \pvdash L^I \tau : \star}$
      
      \caseFact{2} $\tau \; \texttt{nf}$
      
      \caseFact{3} $\Psi ; \Theta ; \Delta \vdash \tau : \star$
      
      \caseFact{4} $\Theta ; \Delta \vdash I : \N$
    }{
      $\Psi ; \Theta ; \Delta \vdash L^I\tau \subtynf : L^I\star \gens \Phi$ and $\Theta ;\Delta \vDash \Phi$
    }
    
    \caseText{By IH on (3)}
    
    \caseFact{5} $\Psi ; \Theta ; \Delta \pvdash \tau \subtynf \tau : \star \gens \Phi$
    
    \caseFact{6} $\Theta ; \Delta \vDash \Phi$
    
    \caseText{By AS-List on (5)}
    
    \caseFact{7} $\Psi ; \Theta ; \Delta \vdash L^I \tau \subtynf L^I \tau : \star \gens \Phi \wedge (I = I)$
    
    \caseText{By Theorem~\ref{thm:sound-compl} applied to (4), we may re-establish the presupposition for (7)}
    
    \caseFact{8} $\Psi ; \Theta ; \Delta \pvdash L^I \tau \subtynf L^I \tau : \star \gens \Phi \wedge (I = I)$
    
    \caseText{The goal is immediate from (6), (8), and $\Theta ; \Delta \vDash I = I$}
  }
  
  \jcase{12}{K-Conj}{
    \jgivengoal{
      \caseFact{1} $\Psi ; \Theta ; \Delta \pvdash \Phi' \amp \tau : \star$  
      
      \caseFact{2} $\tau \; \texttt{nf}$
      
      \caseFact{3} $\Psi ; \Theta ; \Delta \vdash \tau : \star$
      
      \caseFact{4} $\Theta ; \Delta \vdash \Phi' \texttt{ wf}$
    }{
      $\Psi ; \Theta ; \Delta \vdash \Phi' \amp \tau \subtynf \Phi' \amp \tau : \star \gens \Phi$ and $\Theta ; \Delta \vDash \Phi$    
    }
    
    \caseText{By IH on (3)}
    
    \caseFact{5} $\Psi ; \Theta ; \Delta \pvdash \tau \subtynf \tau : \star \gens \Phi$
    
    \caseFact{6} $\Theta ; \Delta \vDash \Phi$
    
    \caseText{By AS-Conj on (5)}
    
    \caseFact{7} $\Psi ; \Theta ; \Delta \vdash \Phi' \amp \tau \subtynf \Phi' \amp \tau : \star \gens \Phi \wedge (\Phi' \to \Phi')$
    
    \caseText{By Theorem~\ref{thm:constr-compl} on (4), we may re-establish the presupposition for (7)}
    
    \caseFact{8} $\Psi ; \Theta ; \Delta \vdash \Phi' \amp \tau \subtynf \Phi' \amp \tau : \star \gens \Phi \wedge (\Phi' \to \Phi')$
    
    \caseText{The goal follows from (6), (8), and $\Theta ; \Delta \vDash \Phi' \to \Phi'$}
  }
  
  \jcase{12}{K-Impl}{
    \jgivengoal{
      \caseFact{1} $\Psi ; \Theta ; \Delta \pvdash \Phi' \implies \tau : \star$  
      
      \caseFact{2} $\tau \; \texttt{nf}$
      
      \caseFact{3} $\Psi ; \Theta ; \Delta \vdash \tau : \star$
      
      \caseFact{4} $\Theta ; \Delta \vdash \Phi' \texttt{ wf}$
    }{
      $\Psi ; \Theta ; \Delta \vdash \Phi' \implies \tau \subtynf \Phi' \implies \tau : \star \gens \Phi$ and $\Theta ; \Delta \vDash \Phi$    
    }
    
    \caseText{By IH on (3)}
    
    \caseFact{5} $\Psi ; \Theta ; \Delta \pvdash \tau \subtynf \tau : \star \gens \Phi$
    
    \caseFact{6} $\Theta ; \Delta \vDash \Phi$
    
    \caseText{By AS-Impl on (5)}
    
    \caseFact{7} $\Psi ; \Theta ; \Delta \vdash \Phi' \implies \tau \subtynf \Phi' \implies \tau : \star \gens \Phi \wedge (\Phi' \to \Phi')$
    
    \caseText{By Theorem~\ref{thm:constr-compl} on (4), we may re-establish the presupposition for (7)}
    
    \caseFact{8} $\Psi ; \Theta ; \Delta \vdash \Phi' \implies \tau \subtynf \Phi' \implies \tau : \star \gens \Phi \wedge (\Phi' \to \Phi')$
    
    \caseText{The goal follows from (6), (8), and $\Theta ; \Delta \vDash \Phi' \to \Phi'$}
  }
  
  \jcase{13}{K-Monad}{
    \jgivengoal{
      \caseFact{1} $\Psi ; \Theta ; \Delta \vdash \M(I,\vec{p}) \tau : \star$
      
      \caseFact{2} $\tau \; \texttt{nf}$
      
      \caseFact{3} $\Psi ; \Theta ; \Delta \vdash \tau : \star$
      
      \caseFact{4} $\Theta ; \Delta \vdash I : \mathbb{N}$
      
      \caseFact{5} $\Theta ; \Delta \vdash \vec{p} : \potvec$
    }{
      $\Psi ; \Theta ; \Delta \vdash \M(I,\vec{p}) \tau \subtynf \M(I,\vec{p}) \tau : \star \gens \Phi$ and $\Theta ; \Delta \vDash \Phi$
    }
    
    \caseText{By IH on (3)}
    
    \caseFact{6} $\Psi ; \Theta ; \Delta \pvdash \tau \subtynf \tau : \star \gens \Phi$
    
    \caseFact{7} $\Theta ; \Delta \vDash \Phi$
    
    \caseText{By AS-Monad on (6), followed by Theorem~\ref{thm:sound-compl} on (4) and (5) to establish the presuppositions}
    
    \caseFact{8} $\Psi ; \Theta ; \Delta \pvdash \M(I,\vec{p}) \tau \subtynf \M(I,\vec{p}) \tau : \star \gens \Phi \wedge (I = I) \wedge (\vec{p} \leq \vec{p})$
    
    \caseText{Goal is immediate from (7) and (8)}
  }
  
  \jcase{14}{K-Pot}{
    \jgivengoal{
      \caseFact{1} $\Psi ; \Theta ; \Delta \vdash [I|\vec{p}] \tau : \star$
      
      \caseFact{2} $\tau \; \texttt{nf}$
      
      \caseFact{3} $\Psi ; \Theta ; \Delta \vdash \tau : \star$
      
      \caseFact{4} $\Theta ; \Delta \vdash I : \mathbb{N}$
      
      \caseFact{5} $\Theta ; \Delta \vdash \vec{p} : \potvec$
    }{
      $\Psi ; \Theta ; \Delta \vdash [I|\vec{p}] \tau \subtynf [I|\vec{p}] \tau : \star \gens \Phi$ and $\Theta ; \Delta \vDash \Phi$
    }
    
    \caseText{By IH on (3)}
    
    \caseFact{6} $\Psi ; \Theta ; \Delta \pvdash \tau \subtynf \tau : \star \gens \Phi$
    
    \caseFact{7} $\Theta ; \Delta \vDash \Phi$
    
    \caseText{By AS-Pot on (6), followed by Theorem~\ref{thm:sound-compl} on (4) and (5) to establish the presuppositions}
    
    \caseFact{8} $\Psi ; \Theta ; \Delta \pvdash [I|\vec{p}] \tau \subtynf [I|\vec{p}] \tau : \star \gens \Phi \wedge (I = I) \wedge (\vec{p} \leq \vec{p})$
    
    \caseText{Goal is immediate from (7) and (8)}
  }
  
  \jcase{15}{K-ConstPot}{
    \jgivengoal{
      \caseFact{1} $\Psi ; \Theta ; \Delta \pvdash [I] \; \tau : \star$
      
      \caseFact{2} $\tau \; \texttt{nf}$
      
      \caseFact{3} $\Psi ; \Theta ; \Delta \vdash \tau : \star$
      
      \caseFact{4} $\Theta ; \Delta \vdash I : \mathbb{R}^+$
    }{
      $\Psi ; \Theta ; \Delta \vdash [I]\tau \subtynf [I]\tau: \star \gens \Phi$ and $\Theta ; \Delta \vDash \Phi$
    }
    
    \caseText{By IH on (3)}
    
    \caseFact{5} $\Psi ; \Theta ; \Delta \pvdash \tau \subtynf \tau : \star \gens \Phi$
    
    \caseFact{6} $\Theta ; \Delta \vDash \Phi$
    
    \caseText{By AS-ConstPot on (5) followed by Theorem~\ref{thm:sort-compl} on (4) to re-establish the presupposition}
    
    \caseFact{7} $\Psi ; \Theta ; \Delta \vdash [I] \; \tau \subtynf [I] \; \tau : \star \gens \Phi \wedge (I \leq I)$
    
    \caseText{Goal is immediate by (7), (6), and $\Theta ; \Delta \vDash I \leq I$}
     
  }
  
  \jcase{16}{K-FamLam}{
    \jgivengoal{
      \caseFact{1} $\Psi ; \Theta ; \Delta \pvdash \lambda i : S. \tau : S \to K$
      
      \caseFact{2} $\tau \; \texttt{nf}$
      
      \caseFact{3} $\Psi ; \Theta, i : S ; \Delta \vdash \tau : K$
    }{
      $\Psi ; \Theta ; \Delta \vdash \lambda i : S. \tau \subtynf \tau : S \to K \gens \Phi$ and $\Theta ; \Delta \vDash \Phi$
    }
    
    \caseText{By IH on (3)}
    
    \caseFact{4} $\Psi ; \Theta, i : S; \Delta \pvdash \tau \subtynf \tau : K \gens \Phi$
    
    \caseFact{5} $\Theta , i : S; \Delta \vDash \Phi$
    
    \caseText{By AS-FamLam on (4)}
    
    \caseFact{6} $\Psi ; \Theta ; \Delta \pvdash \lambda i : S. \tau \subtynf \lambda i : S \tau : S \to K \gens \forall i : S. \Phi$
    
    \caseText{Equivalently to (5)}
    
    \caseFact{7} $\Theta ; \Delta \vDash \forall i : S. \Phi$
    
  }
  
  \jcase{17}{K-FamApp}{Immediate by Theorem~\ref{thm:subtyne-refl}}
  
}

\iffalse
 
  
  \item[(K-FamLam)] Suppose  from $\Psi ; \Theta, i : S ; \Delta \vdash \tau : K$. By IH,
  $\Psi ; \Theta, i : S ; \Delta \vdash \tau \subtynf \tau : K \gens \Phi$ with $\Theta, i : S ; \Delta \vDash \Phi$. By AS-FamLam, $\Psi ; \Theta ; \Delta \vdash \lambda i : S. \tau \subtynf \tau : S \to K \gens \forall i : S. \Phi$. But $\Theta, i : S ; \Delta \vDash \Phi$ and so $\Theta ; \Delta \vDash \forall i : S. \Phi$, as required.
  \item[(K-FamApp)] Done by Theorem~\ref{thm:subtyne-refl}
\fi

\subtynftrans*
\jtheorem{Proof of \autoref{thm:subtynf-trans}}{
  \jgivengoal{
    \caseFact{1} $\Psi ; \Theta ; \Delta \pvdash \tau_1 \subtynf \tau_2 : K \gens \Phi_1$
    
    \caseFact{2} $\Psi ; \Theta ; \Delta \pvdash \tau_2 \subtynf \tau_3 : K \gens \Phi_2$
    
    \caseFact{3} $\Theta ; \Delta \vDash \Phi_1 \wedge \Phi_2$
  }{
    $\Psi ; \Theta ; \Delta \pvdash \tau_1 \subtynf \tau_3 : K \gens \Phi$ such that $\Theta ; \Delta \vDash \Phi$  
  }
  
  \caseText{By strong induction on the sum of the sizes of (1) and (2). We note that for a given choice of final rule for (1), the final rule for (2) must be the same, by inspection of the rules generating $\subtynf$. For this reason, we present the proof as a case analysis over the rules for $\subtynf$.
  }
  
  \jcase{1}{AS-Unit}{Immediate.}
  \jcase{2}{AS-Var}{Immediate.}
  
  \jcase{3}{AS-Arr}{
    \jgivengoal{
      \caseFact{1} $\Psi ; \Theta ; \Delta \pvdash \tau_1 \loli \tau_1' \subtynf \tau_2 \loli \tau_2' : \star \gens \Phi_1 \wedge \Phi_2$
      
      \caseFact{2} $\Psi ; \Theta ; \Delta \pvdash \tau_2 \loli \tau_2' \subtynf \tau_3 \loli \tau_3' : \star \gens \Phi_3 \wedge \Phi_4$
      
      \caseFact{3} $\Theta ; \Delta \vdash \bigwedge_{i=1}^4 \Phi_i$
      
      \caseFact{4} $\Psi ; \Theta ; \Delta \vdash \tau_2 \subtynf \tau_1 : \star \gens \Phi_1$
      
      \caseFact{5} $\Psi ; \Theta ; \Delta \vdash \tau_1' \subtynf \tau_2' : \star \gens \Phi_2$
      
      \caseFact{6} $\Psi ; \Theta ; \Delta \vdash \tau_3 \subtynf \tau_2 : \star \gens \Phi_3$
      
      \caseFact{7} $\Psi ; \Theta ; \Delta \vdash \tau_2' \subtynf \tau_3' : \star \gens \Phi_4$
    }{
      $\Psi ; \Theta ; \Delta \pvdash \tau_1 \loli \tau_1' \subtynf \tau_3 \loli \tau_3' : \star \gens \Phi$ and $\Theta ; \Delta \vDash \Phi$
    }
    
    \caseText{By IH on (4) and (6)}
    
    \caseFact{8} $\Psi ; \Theta ; \Delta \pvdash \tau_3 \subtynf \tau_1 : \star \gens \Phi_1'$
    
    \caseFact{9} $\Theta ; \Delta \vDash \Phi_1'$
    
    \caseText{By IH on (5) and (7)}
    
    \caseText{10} $\Psi ; \Theta ; \Delta \pvdash \tau_1' \subtynf \tau_3' : \star \gens \Phi_2'$
    
    \caseFact{11} $\Theta ; \Delta \vDash \Phi_2'$
    
    \caseText{By AS-Arr on (8) and (10)}
    
    \caseFact{12} $\Psi ; \Theta ; \Delta \pvdash \tau_1 \loli \tau_1' \subtynf \tau_3 \loli \tau_3' : \star \gens \Phi_1' \wedge \Phi_2'$
    
    \caseText{The Goal follows by (9), (11), and (12)}
  }
  
  \jcase{4}{AS-Tensor}{
    \jgivengoal{
      \caseFact{1} $\Psi ; \Theta ; \Delta \pvdash \tau_1 \otimes \tau_1' \subtynf \tau_2 \otimes \tau_2' : \star \gens \Phi_1 \wedge \Phi_2$
      
      \caseFact{2} $\Psi ; \Theta ; \Delta \pvdash \tau_2 \otimes \tau_2' \subtynf \tau_3 \otimes \tau_3' : \star \gens \Phi_3 \wedge \Phi_4$
      
      \caseFact{3} $\Theta ; \Delta \vdash \bigwedge_{i=1}^4 \Phi_i$
      
      \caseFact{4} $\Psi ; \Theta ; \Delta \vdash \tau_1 \subtynf \tau_2 : \star \gens \Phi_1$
      
      \caseFact{5} $\Psi ; \Theta ; \Delta \vdash \tau_1' \subtynf \tau_2' : \star \gens \Phi_2$
      
      \caseFact{6} $\Psi ; \Theta ; \Delta \vdash \tau_2 \subtynf \tau_3 : \star \gens \Phi_3$
      
      \caseFact{7} $\Psi ; \Theta ; \Delta \vdash \tau_2' \subtynf \tau_3' : \star \gens \Phi_4$
    }{
      $\Psi ; \Theta ; \Delta \pvdash \tau_1 \otimes \tau_1' \subtynf \tau_3 \otimes \tau_3' : \star \gens \Phi$ and $\Theta ; \Delta \vDash \Phi$
    }
    
    \caseText{By IH on (4) and (6)}
    
    \caseFact{8} $\Psi ; \Theta ; \Delta \pvdash \tau_1 \subtynf \tau_3 : \star \gens \Phi_1'$
    
    \caseFact{9} $\Theta ; \Delta \vDash \Phi_1'$
    
    \caseText{By IH on (5) and (7)}
    
    \caseText{10} $\Psi ; \Theta ; \Delta \pvdash \tau_1' \subtynf \tau_3' : \star \gens \Phi_2'$
    
    \caseFact{11} $\Theta ; \Delta \vDash \Phi_2'$
    
    \caseText{By AS-Tensor on (8) and (10)}
    
    \caseFact{12} $\Psi ; \Theta ; \Delta \pvdash \tau_1 \otimes \tau_1' \subtynf \tau_3 \otimes \tau_3' : \star \gens \Phi_1' \wedge \Phi_2'$
    
    \caseText{The Goal follows by (9), (11), and (12)}
  }
  
  \jcase{5}{AS-With}{
    \jgivengoal{
      \caseFact{1} $\Psi ; \Theta ; \Delta \pvdash \tau_1 \amp \tau_1' \subtynf \tau_2 \amp \tau_2' : \star \gens \Phi_1 \wedge \Phi_2$
      
      \caseFact{2} $\Psi ; \Theta ; \Delta \pvdash \tau_2 \amp \tau_2' \subtynf \tau_3 \amp \tau_3' : \star \gens \Phi_3 \wedge \Phi_4$
      
      \caseFact{3} $\Theta ; \Delta \vdash \bigwedge_{i=1}^4 \Phi_i$
      
      \caseFact{4} $\Psi ; \Theta ; \Delta \vdash \tau_1 \subtynf \tau_2 : \star \gens \Phi_1$
      
      \caseFact{5} $\Psi ; \Theta ; \Delta \vdash \tau_1' \subtynf \tau_2' : \star \gens \Phi_2$
      
      \caseFact{6} $\Psi ; \Theta ; \Delta \vdash \tau_2 \subtynf \tau_3 : \star \gens \Phi_3$
      
      \caseFact{7} $\Psi ; \Theta ; \Delta \vdash \tau_2' \subtynf \tau_3' : \star \gens \Phi_4$
    }{
      $\Psi ; \Theta ; \Delta \pvdash \tau_1 \amp \tau_1' \subtynf \tau_3 \amp \tau_3' : \star \gens \Phi$ and $\Theta ; \Delta \vDash \Phi$
    }
    
    \caseText{By IH on (4) and (6)}
    
    \caseFact{8} $\Psi ; \Theta ; \Delta \pvdash \tau_1 \subtynf \tau_3 : \star \gens \Phi_1'$
    
    \caseFact{9} $\Theta ; \Delta \vDash \Phi_1'$
    
    \caseText{By IH on (5) and (7)}
    
    \caseText{10} $\Psi ; \Theta ; \Delta \pvdash \tau_1' \subtynf \tau_3' : \star \gens \Phi_2'$
    
    \caseFact{11} $\Theta ; \Delta \vDash \Phi_2'$
    
    \caseText{By AS-With on (8) and (10)}
    
    \caseFact{12} $\Psi ; \Theta ; \Delta \pvdash \tau_1 \amp \tau_1' \subtynf \tau_3 \amp \tau_3' : \star \gens \Phi_1' \wedge \Phi_2'$
    
    \caseText{The Goal follows by (9), (11), and (12)}
  }
  
  \jcase{6}{AS-Sum}{
    \jgivengoal{
      \caseFact{1} $\Psi ; \Theta ; \Delta \pvdash \tau_1 \oplus \tau_1' \subtynf \tau_2 \oplus \tau_2' : \star \gens \Phi_1 \wedge \Phi_2$
      
      \caseFact{2} $\Psi ; \Theta ; \Delta \pvdash \tau_2 \oplus \tau_2' \subtynf \tau_3 \oplus \tau_3' : \star \gens \Phi_3 \wedge \Phi_4$
      
      \caseFact{3} $\Theta ; \Delta \vdash \bigwedge_{i=1}^4 \Phi_i$
      
      \caseFact{4} $\Psi ; \Theta ; \Delta \vdash \tau_1 \subtynf \tau_2 : \star \gens \Phi_1$
      
      \caseFact{5} $\Psi ; \Theta ; \Delta \vdash \tau_1' \subtynf \tau_2' : \star \gens \Phi_2$
      
      \caseFact{6} $\Psi ; \Theta ; \Delta \vdash \tau_2 \subtynf \tau_3 : \star \gens \Phi_3$
      
      \caseFact{7} $\Psi ; \Theta ; \Delta \vdash \tau_2' \subtynf \tau_3' : \star \gens \Phi_4$
    }{
      $\Psi ; \Theta ; \Delta \pvdash \tau_1 \oplus \tau_1' \subtynf \tau_3 \oplus \tau_3' : \star \gens \Phi$ and $\Theta ; \Delta \vDash \Phi$
    }
    
    \caseText{By IH on (4) and (6)}
    
    \caseFact{8} $\Psi ; \Theta ; \Delta \pvdash \tau_1 \subtynf \tau_3 : \star \gens \Phi_1'$
    
    \caseFact{9} $\Theta ; \Delta \vDash \Phi_1'$
    
    \caseText{By IH on (5) and (7)}
    
    \caseText{10} $\Psi ; \Theta ; \Delta \pvdash \tau_1' \subtynf \tau_3' : \star \gens \Phi_2'$
    
    \caseFact{11} $\Theta ; \Delta \vDash \Phi_2'$
    
    \caseText{By AS-Sum on (8) and (10)}
    
    \caseFact{12} $\Psi ; \Theta ; \Delta \pvdash \tau_1 \oplus \tau_1' \subtynf \tau_3 \oplus \tau_3' : \star \gens \Phi_1' \wedge \Phi_2'$
    
    \caseText{The Goal follows by (9), (11), and (12)}
  }
  
  \jcase{7}{AS-Bang}{
    \jgivengoal{
      \caseFact{1} $\Psi ; \Theta ; \Delta \pvdash !\tau_1 \subtynf !\tau_2 : \star \gens \Phi_1$
      
      \caseFact{2} $\Psi ; \Theta ; \Delta \pvdash !\tau_2 \subtynf !\tau_3 : \star \gens \Phi_2$
      
      \caseFact{3} $\Theta ; \Delta \vDash \Phi_1 \wedge \Phi_2$
      
      \caseFact{4} $\Psi ; \Theta ; \Delta \vdash \tau_1 \subtynf \tau_2 : \star \gens \Phi_1$
      
      \caseFact{5}  $\Psi ; \Theta ; \Delta \vdash \tau_2 \subtynf \tau_3 : \star \gens \Phi_2$
    }{
      $\Psi ; \Theta ; \Delta \pvdash !\tau_1 \subtynf !\tau_3 : \star \gens \Phi$ and $\Theta ; \Delta \vDash \Phi$
    }
    
    \caseText{By IH on (4) and (5)}
    
    \caseFact{6} $\Psi ; \Theta ; \Delta \pvdash \tau_1 \subtynf \tau_3 : \star \gens Phi$
    
    \caseFact{7} $\Theta ; \Delta \vDash \Phi$
    
    \caseText{By AS-Bang on (6)}
    
    \caseFact{8} $\Psi ; \Theta ; \Delta \pvdash !\tau_1 \subtynf !\tau_3 : \star \gens Phi$
  }
  
  \jcase{8}{AS-IForall}{
    \jgivengoal{
      \caseFact{1} $\Psi ; \Theta ; \Delta \pvdash \forall i : S. \tau_1 \subtynf \forall i : S. \tau_2 : \star \gens \forall i : S. \Phi_1$    
      
      \caseFact{2} $\Psi ; \Theta ; \Delta \pvdash \forall i : S. \tau_2 \subtynf \forall i : S. \tau_3 : \star \gens \forall i : S. \Phi_2$
      
      \caseFact{3} $\Theta ; \Delta \vDash \forall i : S. \Phi_1 \wedge \forall i : S. \Phi_2$
      
      \caseFact{4} $\Psi ; \Theta, i : S ; \Delta \vdash \tau_1 \subtynf \tau_2 : \star \gens \Phi_1$
      
      \caseFact{5} $\Psi ; \Theta, i : S ; \Delta \vdash \tau_2 \subtynf \tau_3 : \star \gens \Phi_3$
    }{
      $\Psi ; \Theta ; \Delta \pvdash \forall i : S. \tau_1 \subtynf \forall i : S. \tau_3 : \star \gens \Phi$ and $\Theta ; \Delta \vDash \Phi$
    }
    
    \caseText{Equivalently to (3), $\Theta, i : S; \Delta \vDash \Phi_1$ and $\Theta, i : S; \Delta \vDash \Phi_2$, and so by IH on (4),(5)}
    
    \caseFact{6} $\Psi ; \Theta, i :S ;\Delta \pvdash \tau_1 \subtynf \tau_3 : \star \gens \Phi$
    
    \caseFact{7} $\Theta, i : S: \Delta \vDash \Phi$
    
    \caseText{By AS-IForall on (6)}
    
    \caseFact{8} $\Psi ; \Theta ; \Delta \pvdash \forall i : S.\tau_1 \subtynf \forall i : S.\tau_3 : \star \gens \forall i : S. \Phi$
    
    \caseText{Equivalently to (7)}
    
    \caseFact{9} $\Theta ; \Delta \vDash \forall i : S. \Phi$
    
    \caseText{The goal follows by (8) and (9)}
  }
  
  \jcase{9}{AS-IExists}{
    \jgivengoal{
      \caseFact{1} $\Psi ; \Theta ; \Delta \pvdash \exists i : S. \tau_1 \subtynf \exists i : S. \tau_2 : \star \gens \forall i : S. \Phi_1$    
      
      \caseFact{2} $\Psi ; \Theta ; \Delta \pvdash \exists i : S. \tau_2 \subtynf \exists i : S. \tau_3 : \star \gens \forall i : S. \Phi_2$
      
      \caseFact{3} $\Theta ; \Delta \vDash \forall i : S. \Phi_1 \wedge \forall i : S. \Phi_2$
      
      \caseFact{4} $\Psi ; \Theta, i : S ; \Delta \vdash \tau_1 \subtynf \tau_2 : \star \gens \Phi_1$
      
      \caseFact{5} $\Psi ; \Theta, i : S ; \Delta \vdash \tau_2 \subtynf \tau_3 : \star \gens \Phi_3$
    }{
      $\Psi ; \Theta ; \Delta \pvdash \exists i : S. \tau_1 \subtynf \exists i : S. \tau_3 : \star \gens \Phi$ and $\Theta ; \Delta \vDash \Phi$
    }
    
    \caseText{Equivalently to (3), $\Theta, i : S; \Delta \vDash \Phi_1$ and $\Theta, i : S; \Delta \vDash \Phi_2$, and so by IH on (4),(5)}
    
    \caseFact{6} $\Psi ; \Theta, i :S ;\Delta \pvdash \tau_1 \subtynf \tau_3 : \star \gens \Phi$
    
    \caseFact{7} $\Theta, i : S: \Delta \vDash \Phi$
    
    \caseText{By AS-IExists on (6)}
    
    \caseFact{8} $\Psi ; \Theta ; \Delta \pvdash \exists i : S.\tau_1 \subtynf \exists i : S.\tau_3 : \star \gens \forall i : S. \Phi$
    
    \caseText{Equivalently to (7)}
    
    \caseFact{9} $\Theta ; \Delta \vDash \forall i : S. \Phi$
    
    \caseText{The goal follows by (8) and (9)}
  }
  
  \jcase{10}{AS-TForall}{
    \jgivengoal{
      \caseFact{1} $\Psi ; \Theta ; \Delta \pvdash \forall \alpha : K. \tau_1 \subtynf \forall \alpha : K. \tau_2 : \star \gens \Phi_1$    
      
      \caseFact{2} $\Psi ; \Theta ; \Delta \pvdash \forall \alpha : K. \tau_2 \subtynf \forall \alpha : K. \tau_3 : \star \gens \Phi_2$
      
      \caseFact{3} $\Theta ; \Delta \vDash \Phi_1 \wedge \Phi_2$
      
      \caseFact{4} $\Psi, \alpha : K ; \Theta ; \Delta \vdash \tau_1 \subtynf \tau_2 : \star \gens \Phi_1$
      
      \caseFact{5} $\Psi, \alpha : K ; \Theta ; \Delta \vdash \tau_2 \subtynf \tau_3 : \star \gens \Phi_2$
    }{
       $\Psi ; \Theta ; \Delta \pvdash \forall \alpha : K. \tau_1 \subtynf \forall \alpha : K. \tau_3 : \star \gens \Phi$     and $\Theta ; \Delta \vDash \Phi$
    }
    
    \caseText{By IH on (4),(5)}    
    
    \caseFact{6} $\Psi , \alpha : K ; \Theta ; \Delta \pvdash \tau_1 \subtynf \tau_3 : \star \gens \Phi$
    
    \caseFact{7} $\Theta ; \Delta \vDash \Phi$
    
    \caseText{By AS-TForall on (6)}
    
    \caseFact{8} $\Psi ; \Theta ; \Delta \pvdash \forall \alpha : K.\tau_1 \subtynf \forall \alpha : K.\tau_3 : \star \gens \Phi$
  }
  
  \jcase{11}{AS-List}{
    \jgivengoal{
      \caseFact{1} $\Psi ; \Theta ; \Delta \pvdash L^I \tau_1 \subtynf L^J \tau_2 : \star \gens \Phi_1 \wedge (I = J)$
      
      \caseFact{2} $\Psi ; \Theta ; \Delta \pvdash L^J \tau_2 \subtynf L^K \tau_3 : \star \gens \Phi_2 \wedge (J = K)$
      
      \caseFact{3} $\Theta ; \Delta \vDash \Phi_1 \wedge \Phi_2 \wedge (I = J) \wedge (J = K)$
      
      \caseFact{4} $\Psi ; \Theta ; \Delta \pvdash \tau_1 \subtynf \tau_2 : \star \gens \Phi_1$
      
      \caseFact{5} $\Psi ; \Theta ; \Delta \pvdash \tau_2 \subtynf \tau_3 : \star \gens \Phi_2$
    }{
      $\Psi ; \Theta ; \Delta \pvdash L^I \tau_1 \subtynf L^K \tau_3 : \star \gens \Phi$ and $\Theta ; \Delta \vDash \Phi$
    }
    
    \caseText{By IH on (4), (5)}
    
    \caseFact{6} $\Psi ; \Theta ; \Delta \pvdash \tau_1 \subtynf \tau_3 : \star \gens \Phi$
    
    \caseFact{7} $\Theta ; \Delta \vDash \Phi$
    
    \caseText{By AS-List on (7)}
    
    \caseFact{8} $\Psi ; \Theta ; \Delta \pvdash \tau_1^I \subtynf \tau_3^K : \star \gens \Phi \wedge (I = K)$
    
    \caseText{By (3)}
    
    \caseFact{9} $\Theta ; \Delta \vDash I = K$
    
    \caseText{The goal follows by (7), (8), (9)}
  }
  
  \jcase{12}{AS-Impl}{
  
    \jgivengoal{
      \caseFact{1} $\Psi ; \Theta ; \Delta \pvdash \Phi_1 \implies \tau_1 \subtynf \Phi_2 \implies \tau_2 : \star \gens (\Phi_2 \to \Phi_1') \wedge (\Phi_2 \to \Phi_1)$
      
      \caseFact{2} $\Psi ; \Theta ; \Delta \pvdash \Phi_2 \implies \tau_2 \subtynf \Phi_3 \implies \tau_3 : \star \gens (\Phi_3 \to \Phi_2') \wedge (\Phi_3 \to \Phi_2)$
      
      \caseFact{3} $\Theta ; \Delta \vDash \Phi_1' \wedge \Phi_2' \wedge (Phi_3 \to \Phi_2) \wedge (Phi_2 \to \Phi_1)$
      
      \caseFact{4} $\Psi ; \Theta ; \Delta, \Phi_2 \pvdash \tau_1 \subtynf \tau_2 : \star \gens \Phi_1'$
      
      \caseFact{5} $\Psi ; \Theta ; \Delta, \Phi_3 \pvdash \tau_2 \subtynf \tau_3 : \star \gens \Phi_2'$
    }{
      $\Psi ; \Theta ; \Delta \vdash \Phi_1 \implies \tau_1 \subtynf \Phi_3 \implies \tau_3 : \star \gens \Phi$ and $\Theta ; \Delta \vDash \Phi$
    }
    
    \textbf{FIXME FIXME}
    
    \caseText{By \autoref{thm:subty-assump-precomp} on (4)}
    
    \caseFact{6} $\Psi ; \Theta ; \Delta, \Phi_3 \pvdash \tau_1 \subtynf \tau_2 : \star \gens \Phi_1''$
    
    \caseFact{7} $\Theta ; \Delta \vDash \Phi_1''$
    
    \caseText{By IH on (5), (6)}
    
    \caseFact{8} $\Psi ; \Theta ; \Delta, \Phi_3 \pvdash \tau_1 \subtynf \tau_3 : \star \gens \Phi$
    
    \caseFact{9} $\Theta ; \Delta \vDash \Phi$
    
    \caseText{By AS-Impl on (8)}
    
    \caseFact{10} $\Psi ; \Theta ; \Delta \pvdash \Phi_1 \implies \tau_1 \subtynf \Phi_3 \implies \tau_3 : \star \gens \Phi \wedge (\Phi_3 \to \Phi_1)$
    
    \caseText{By (3)}
    
    \caseFact{11} $\Theta ; \Delta \vDash \Phi_3 \to \Phi_1$
    
    \caseText{The goal follows by (7), (8), (9)}
  }
  
  \jcase{13}{AS-Conj}{
  
    \jgivengoal{
      \caseFact{1} $\Psi ; \Theta ; \Delta \pvdash \Phi_1 \amp \tau_1 \subtynf \Phi_2 \amp \tau_2 : \star \gens \Phi_1' \wedge (\Phi_1 \to \Phi_2)$
      
      \caseFact{2} $\Psi ; \Theta ; \Delta \pvdash \Phi_2 \amp \tau_2 \subtynf \Phi_3 \amp \tau_3 : \star \gens \Phi_2' \wedge (\Phi_2 \to \Phi_3)$
      
      \caseFact{3} $\Theta ; \Delta \vDash \Phi_1' \wedge \Phi_2' \wedge (Phi_1 \to \Phi_3) \wedge (Phi_1 \to \Phi_2)$
      
      \caseFact{4} $\Psi ; \Theta ; \Delta \vdash \tau_1 \subtynf \tau_2 : \star \gens \Phi_1'$
      
      \caseFact{5} $\Psi ; \Theta ; \Delta \vdash \tau_2 \subtynf \tau_3 : \star \gens \Phi_2'$
    }{
      $\Psi ; \Theta ; \Delta \vdash \Phi_1 \amp \tau_1 \subtynf \Phi_3 \amp \tau_3 : \star \gens \Phi$ and $\Theta ; \Delta \vDash \Phi$
    }
    
    \caseText{By IH on (4), (5)}
    
    \caseFact{6} $\Psi ; \Theta ; \Delta \pvdash \tau_1 \subtynf \tau_3 : \star \gens \Phi$
    
    \caseFact{7} $\Theta ; \Delta \vDash \Phi$
    
    \caseText{By AS-Conj on (6)}
    
    \caseFact{8} $\Psi ; \Theta ; \Delta \pvdash \Phi_1 \amp \tau_1 \subtynf \Phi_3 \amp \tau_3 : \star \gens \Phi \wedge (\Phi_1 \to \Phi_3)$
    
    \caseText{By (3)}
    
    \caseFact{9} $\Theta ; \Delta \vDash \Phi_1 \to \Phi_3$
    
    \caseText{The goal follows by (7), (8), (9)}
  }
%{\Psi ; \Theta ; \Delta \vdash \tau_1 \subtynf \tau_2 : \star \gens \Phi}{\Psi ; \Theta ; \Delta \vdash \M(I,\vec{q}) \tau_1 \subtynf \M(J,\vec{p}) \tau_2 : \star \gens (I = J) \wedge (\vec{q} \leq \vec{p}) \wedge \Phi}
  
  \jcase{14}{AS-Monad}{
    \jgivengoal{
      \caseFact{1} $\Psi ; \Theta ; \Delta \pvdash \M(I,\vec{q}) \tau_1 \subtynf \M(J,\vec{p}) \tau_2 : \star \gens (I = J) \wedge (\vec{q} \leq \vec{p}) \wedge \Phi_1$
      
      \caseFact{2} $\Psi ; \Theta ; \Delta \pvdash \M(J,\vec{p}) \tau_2 \subtynf \M(K,\vec{l}) \tau_3 : \star \gens (J = K) \wedge (\vec{p} \leq \vec{l}) \wedge \Phi_2$
      
      \caseFact{3} $\Theta ; \Delta \vDash (I = J) \wedge (J = K) \wedge (\vec{q} \leq \vec{p}) \wedge (\vec{p} \leq \vec{l}) \wedge \Phi_1 \wedge \Phi_2$
      
      \caseFact{4} $\Psi ; \Theta ; \Delta \vdash \tau_1 \subtynf \tau_2 : \star \gens \Phi_1$
      
      \caseFact{5} $\Psi ; \Theta ; \Delta \vdash \tau_2 \subtynf \tau_3 : \star \gens \Phi_2$
    }{
      $\Psi ; \Theta ; \Delta \pvdash \M(I,\vec{q}) \tau_1 \subtynf \M(K,\vec{l}) \tau_3 : \star \gens \Phi$ and $\Theta ; \Delta \vDash \Phi$
    }
    
    \caseText{By IH on (4), (5)}
    
    \caseFact{7} $\Psi ; \Theta ; \Delta \pvdash \tau_1 \subtynf \tau_3 : \star \gens \Phi$
    
    \caseFact{8} $\Theta ; \Delta \vDash \Phi$
    
    \caseText{By (3)}
    
    \caseFact{9} $\Theta ; \Delta \vDash (I = K) \wedge (\vec{q} \leq \vec{l})$
    
    \caseText{By AS-Monad on (7)}
    
    \caseFact{10} $\Psi ; \Theta ; \Delta \pvdash \M(I,\vec{q}) \tau_1 \subtynf \M(K,\vec{p}) \tau_3 : \star \gens (I = K) \wedge (\vec{q} \leq \vec{l}) \wedge \Phi$
    
    \caseText{Goal follows by (8), (9), (10)}
  }
  
  \jcase{15}{AS-Pot}{
    \jgivengoal{
      \caseFact{1} $\Psi ; \Theta ; \Delta \pvdash [I|\vec{q}] \tau_1 \subtynf [J|\vec{p}] \tau_2 : \star \gens (I = J) \wedge (\vec{q} \geq \vec{p}) \wedge \Phi_1$
      
      \caseFact{2} $\Psi ; \Theta ; \Delta \pvdash [J|\vec{p}] \tau_2 \subtynf [K|\vec{l}] \tau_3 : \star \gens (J = K) \wedge (\vec{p} \geq \vec{l}) \wedge \Phi_2$
      
      \caseFact{3} $\Theta ; \Delta \vDash (I = J) \wedge (J = K) \wedge (\vec{q} \geq \vec{p}) \wedge (\vec{p} \geq \vec{l}) \wedge \Phi_1 \wedge \Phi_2$
      
      \caseFact{4} $\Psi ; \Theta ; \Delta \vdash \tau_1 \subtynf \tau_2 : \star \gens \Phi_1$
      
      \caseFact{5} $\Psi ; \Theta ; \Delta \vdash \tau_2 \subtynf \tau_3 : \star \gens \Phi_2$
    }{
      $\Psi ; \Theta ; \Delta \pvdash [I|\vec{q}] \tau_1 \subtynf [K|\vec{l}] \tau_3 : \star \gens \Phi$ and $\Theta ; \Delta \vDash \Phi$
    }
    
    \caseText{By IH on (4), (5)}
    
    \caseFact{7} $\Psi ; \Theta ; \Delta \pvdash \tau_1 \subtynf \tau_3 : \star \gens \Phi$
    
    \caseFact{8} $\Theta ; \Delta \vDash \Phi$
    
    \caseText{By (3)}
    
    \caseFact{9} $\Theta ; \Delta \vDash (I = K) \wedge (\vec{q} \geq \vec{l})$
    
    \caseText{By AS-Pot on (7)}
    
    \caseFact{10} $\Psi ; \Theta ; \Delta \pvdash [I|\vec{q}] \tau_1 \subtynf [K|\vec{p}] \tau_3 : \star \gens (I = K) \wedge (\vec{q} \geq \vec{l}) \wedge \Phi$
    
    \caseText{Goal follows by (8), (9), (10)}
  }
  
  \jcase{16}{AS-ConstPot}{
    \jgivengoal{
      \caseFact{1} $\Psi ; \Theta ; \Delta \pvdash [I] \tau_1 \subtynf [J] \tau_2 : \star \gens (I \leq J) \wedge \Phi_1$
      
      \caseFact{2} $\Psi ; \Theta ; \Delta \pvdash [J] \tau_2 \subtynf [K] \tau_3 : \star \gens (J \leq K) \wedge \Phi_2$
      
      \caseFact{3} $\Theta ; \Delta \vDash (I \leq J) \wedge (J \leq K) \wedge \Phi_1 \wedge \Phi_2$
      
      \caseFact{4} $\Psi ; \Theta ; \Delta \vdash \tau_1 \subtynf \tau_2 : \star \gens \Phi_1$
      
      \caseFact{5} $\Psi ; \Theta ; \Delta \vdash \tau_2 \subtynf \tau_3 : \star \gens \Phi_2$
    }{
      $\Psi ; \Theta ; \Delta \pvdash [I] \tau_1 \subtynf [K] \tau_3 : \star \gens \Phi$ and $\Theta ; \Delta \vDash \Phi$
    }
    
    \caseText{By IH on (4), (5)}
    
    \caseFact{7} $\Psi ; \Theta ; \Delta \pvdash \tau_1 \subtynf \tau_3 : \star \gens \Phi$
    
    \caseFact{8} $\Theta ; \Delta \vDash \Phi$
    
    \caseText{By (3)}
    
    \caseFact{9} $\Theta ; \Delta \vDash (I \leq K)$
    
    \caseText{By AS-ConstPot on (7)}
    
    \caseFact{10} $\Psi ; \Theta ; \Delta \pvdash [I] \tau_1 \subtynf [K] \tau_3 : \star \gens (I \leq K) \wedge \Phi$
    
    \caseText{Goal follows by (8), (9), (10)}
  }
  
  \jcase{17}{AS-FamLam}{
    \jgivengoal{
      \caseFact{1} $\Psi ; \Theta ; \Delta \pvdash \lambda i : S.\tau_1 \subtynf \lambda i : S. \tau_2 : S \to K \gens \forall i : S. \Phi_1$    
      
      \caseFact{2} $\Psi ; \Theta ; \Delta \pvdash \lambda i : S.\tau_2 \subtynf \lambda i : S. \tau_3 : S \to K \gens \forall i : S. \Phi_2$
      
      \caseFact{3} $\Theta ; \Delta \vDash \forall i : S.\Phi_1 \wedge \forall i : S.\Phi_2$
      
      \caseFact{4} $\Psi ; \Theta, i : S; \Delta \vdash \tau_1 \subtynf \tau_2 : K \gens \Phi_1$
      
      \caseFact{5} $\Psi ; \Theta, i : S; \Delta \vdash \tau_2 \subtynf \tau_3 : K \gens \Phi_2$
    }{
      $Psi ; \Theta ; \Delta \pvdash \lambda i : S. \tau_1 \subtynf \lambda i : S. \tau_3 : S \to K \gens \Phi$ and $\Theta ; \Delta \vDash \Phi$    
    }
    
    \caseText{By IH on (4) and (5)}
    
    \caseFact{6}  $\Psi ; \Theta, i : S ; \Delta \pvdash \tau_1 \subtynf \tau_3 : K \gens \Phi$
    
    \caseFact{7} $\Theta, i : S ; \Delta \vDash \Phi$
    
    \caseText{By AS-FamLam on (6)}
    
    \caseFact{8} $\Psi ; \Theta ; \Delta \pvdash \lambda i : S. \tau_1 \subtynf \lambda i : S. \tau_3 : S \to K \gens \forall i : S.\Phi$
    
    \caseText{Equivalently to (7)}
    
    \caseFact{9} $\Theta ; \Delta \vDash \forall i : S. \Phi$
    
    \caseText{The Goal follows from (8) and (9)}
  }
  
  \jcase{18}{AS-FamApp}{
    \jgivengoal{
      \caseFact{1} $\Psi ; \Theta ; \Delta \pvdash \tau_1 \; I \subtynf \tau_2 \; J : K \gens (I = J) \wedge \Phi_1$    
      
      \caseFact{2} $\Psi ; \Theta ; \Delta \pvdash \tau_2 \; J \subtynf \tau_3 \; L : K \gens (J = L) \wedge \Phi_2$
      
      \caseFact{3} $\Theta ; \Delta \vDash (I = J) \wedge (J = L) \wedge \Phi_1 \wedge \Phi_2$
      
      \caseFact{4} $\Psi ; \Theta ; \Delta \vdash \tau_1 \subtynf \tau_2 : S \to K \gens \Phi_1$
      
      \caseFact{5} $\Psi ; \Theta ; \Delta \vdash \tau_2 \subtynf \tau_3 : S \to K \gens \Phi_2$
    }{
      $\Psi ; \Theta ; \Delta \pvdash \tau_1 \; I \subtynf \tau_3 \; L : K \gens \Phi$ and $\Theta ; \Delta \vDash \Phi$
    }
    
    \caseText{By IH on (4) and (5)}
    
    \caseFact{5} $\Psi ; \Theta ; \Delta \pvdash \tau_1 \subtynf \tau_3 : S \to K \gens \Phi$
    
    \caseFact{6} $\Theta ; \Delta \vDash \Phi$
    
    \caseText{By (3)}
    
    \caseFact{7} $\Theta ; \Delta \vDash (I = L)$
    
    \caseText{By AS-FamApp on (5)}
    
    \caseFact{8} $\Psi ; \Theta ; \Delta \pvdash \tau_1 \; I \subtynf \tau_3 \; L : K \gens (I = L) \wedge \Phi$
    
    \caseText{The Goal follows by (6),(7),(8)}
  }
}

\iffalse

  \item[(AS-IForall)] Suppose  and
   from  and . By IH, $\Psi ; \Theta, i : S ; \Delta \vdash \tau_1 \subtynf \tau_3 : \star \gens \Phi_1 \wedge \Phi_2$ with $\Theta, i: S ; \Delta \vDash \Phi_1 \wedge \Phi_2$. By AS-IForall, $\Psi ; \Theta ; \Delta \vdash \forall i : S. \tau_1 \subtynf \forall i : S.\tau_3 : \star \gens \forall i : S.(\Phi_1 \wedge \Phi_2)$. But $\Theta, i : S ; \Delta \vDash \Phi_1 \wedge \Phi_2$ is equivalent to $\Theta ; \Delta \vDash \forall i : S. (\Phi_1 \wedge \Phi_2)$, and so we are done.
  \item[(AS-IExists)]  Suppose $\Psi ; \Theta ; \Delta \vdash \exists i : S. \tau_1 \subtynf \exists i : S. \tau_2 : \star \gens \forall i : S. \Phi_1$ and
  $\Psi ; \Theta ; \Delta \vdash \exists i : S. \tau_2 \subtynf \exists i : S. \tau_3 : \star \gens \forall i : S. \Phi_2$ from $\Psi ; \Theta, i : S ; \Delta \vdash \tau_1 \subtynf \tau_2 : \star \gens \Phi_1$ and $\Psi ; \Theta, i : S ; \Delta \vdash \tau_2 \subtynf \tau_3 : \star \gens \Phi_3$. By IH, $\Psi ; \Theta, i : S ; \Delta \vdash \tau_1 \subtynf \tau_3 : \star \gens \Phi_1 \wedge \Phi_2$ with $\Theta, i: S ; \Delta \vDash \Phi_1 \wedge \Phi_2$. By AS-IExists, $\Psi ; \Theta ; \Delta \vdash \exists i : S. \tau_1 \subtynf \exists i : S.\tau_3 : \star \gens \forall i : S.(\Phi_1 \wedge \Phi_2)$. But $\Theta, i : S ; \Delta \vDash \Phi_1 \wedge \Phi_2$ is equivalent to $\Theta ; \Delta \vDash \forall i : S. (\Phi_1 \wedge \Phi_2)$, and so we are done.
  \item[(AS-TForall)] Suppose $\Psi ; \Theta ; \Delta \vdash \forall \alpha : K. \tau_1 \subtynf \forall \alpha : K. \tau_2 : \star \gens \Phi_1$ and
  $\Psi ; \Theta ; \Delta \vdash \forall \alpha : K. \tau_2 \subtynf \forall \alpha : K. \tau_3 : \star \gens \Phi_2$ from
  $\Psi, \alpha : K ; \Theta ; \Delta \vdash \tau_1 \subtynf \tau_2 : \star \gens \Phi_1$ and
  $\Psi, \alpha : K ; \Theta ; \Delta \vdash \tau_2 \subtynf \tau32 : \star \gens \Phi_2$.
  By IH, $\Psi, \alpha : K ; \Theta ; \Delta \vdash \tau_1 \subtynf \tau_3 : \star \gens \Phi$ with $\Theta ; \Delta \vDash \Phi$. By AS-TForall,
  $\Psi ; \Theta ; \Delta \vdash \forall \alpha : K. \tau_1 \subtynf \forall \alpha : K. \tau_3 : \star \gens \Phi$ as required.
  \item[(AS-List)] Suppose $\Psi ; \Theta ; \Delta \vdash L^I \tau_1 \subtynf L^J \tau_2 : \star \gens I = J \wedge \Phi_1$ and
  $\Psi ; \Theta ; \Delta \vdash L^J \tau_2 \subtynf L^K \tau_3 : \star \gens J = K \wedge \Phi_2$
  from $\Psi ; \Theta ; \Delta \vdash \tau_1 \subtynf \tau_2 : \star \gens \Phi_1$ and
  $\Psi ; \Theta ; \Delta \vdash \tau_2 \subtynf \tau_3 : \star \gens \Phi_2$. Since $\Theta ; \Delta \vdash \Phi_1 \wedge \Phi_2$, we have by IH that
  $\Psi ; \Theta ; \Delta \vdash \tau_1 \subtynf \tau_3 : \star \gens \Phi$ with $\Theta ; \Delta \vDash \Phi$. Since $\Theta ; \Delta \vDash I = J \wedge J = K$ also,
  we have that $\Theta ; \Delta \vDash I = K \wedge \Phi$. Then, by AS-List, we have
  $\Psi ; \Theta ; \Delta \vdash L^I \tau_1 \subtynf L^K \tau_3 : \star \gens I = K \wedge \Phi$, and so we are done.
  \item[(AS-Impl)] Suppose $\Psi ; \Theta ; \Delta \vdash \Phi_1' \implies \tau_1 \subtynf \Phi_2' \implies \tau_2 : \star \gens \Phi_1 \wedge (\Phi_2' \to \Phi_1')$
  and $\Psi ; \Theta ; \Delta \vdash \Phi_2' \implies \tau_2 \subtynf \Phi_3' \implies \tau_3 : \star \gens \Phi_2 \wedge (\Phi_3' \to \Phi_2')$
  from $\Psi ; \Theta ; \Delta \vdash \tau_1 \subtynf \tau_2 : star \gens \Phi_1$ and 
  $\Psi ; \Theta ; \Delta \vdash \tau_2 \subtynf \tau_3 : \star \gens \Phi_3$. By IH,
  $\Psi ; \Theta ; \Delta \vdash \tau_1 \subtynf \tau_3 : \star \gens \Phi$. with $\Theta ; \Delta \vDash \Phi$.
  By AS-Impl, $\Psi ; \Theta ; \Delta \vdash \Phi_1' \implies \tau_1 \subtynf \Phi_3' \implies \tau_3 : \star \gens \Phi \wedge (\Phi_3' \to \Phi_1')$. But since $\Theta ; \Delta \vDash (\Phi_3' \to \Phi_2') \wedge (\Phi_2' \to \Phi_1')$, we have that $\Theta ; \Delta \vDash \Phi \wedge (\Phi_3' \to \Phi_1')$, as required.
  \item[(AS-Conj)] Suppose $\Psi ; \Theta ; \Delta \vdash \Phi_1' \amp \tau_1 \subtynf \Phi_2' \amp \tau_2 : \star \gens \Phi_1 \wedge (\Phi_1' \to \Phi_2')$
  and $\Psi ; \Theta ; \Delta \vdash \Phi_2' \amp \tau_2 \subtynf \Phi_3' \amp \tau_3 : \star \gens \Phi_2 \wedge (\Phi_2' \to \Phi_3')$
  from $\Psi ; \Theta ; \Delta \vdash \tau_1 \subtynf \tau_2 : \star \gens \Phi_1$ and 
  $\Psi ; \Theta ; \Delta \vdash \tau_2 \subtynf \tau_3 : \star \gens \Phi_3$. By IH,
  $\Psi ; \Theta ; \Delta \vdash \tau_1 \subtynf \tau_3 : \star \gens \Phi$. with $\Theta ; \Delta \vDash \Phi$.
  By AS-Conj, $\Psi ; \Theta ; \Delta \vdash \Phi_1' \amp \tau_1 \subtynf \Phi_3' \amp \tau_3 : \star \gens \Phi \wedge (\Phi_1' \to \Phi_3')$. But since $\Theta ; \Delta \vDash (\Phi_1' \to \Phi_2') \wedge (\Phi_2' \to \Phi_3')$, we have that $\Theta ; \Delta \vDash \Phi \wedge (\Phi_1' \to \Phi_3')$, as required.
  \item[(AS-Monad)]
  Suppose $\Psi ; \Theta ; \Delta \vdash \M(I,\vec{q}) \tau_1 \subtynf \M(J,\vec{p}) \tau_2 : \star \gens (I = J) \wedge (\vec{q} \leq \vec{p}) \wedge \Phi_1$
  and $\Psi ; \Theta ; \Delta \vdash \M(J,\vec{p}) \tau_2 \subtynf \M(K,\vec{r}) \tau_3 : \star \gens (J = K) \wedge (\vec{p} \leq \vec{r}) \wedge \Phi_2$
  from $\Psi ; \Theta ; \Delta \vdash \tau_1 \subtynf \tau_2 : \star \gens \Phi_1$ and
  $\Psi ; \Theta ; \Delta \vdash \tau_2 \subtynf \tau_3 : \star \gens \Phi_2$.
  By IH, $\Psi ; \Theta ; \Delta \vdash \tau_1 \subtynf \tau_3 : \star \gens \Phi$, with $\Theta ; \Delta \vDash \Phi$.
  By AS-Monad, $\Psi ; \Theta ; \Delta \vdash \M(I,\vec{q}) \tau_1 \subtynf \M(K,\vec{r})\tau_3 : \star \gens (I = K) \wedge (\vec{q} \leq \vec{r}) \Phi$.
  But since $\Theta ; \Delta \vDash (I = J) \wedge (J = K) \wedge (\vec{q} \leq \vec{p}) \wedge (\vec{p} \leq \vec{r})$, we have $\Theta ; \Delta \vDash (I = K) \wedge (\vec{q} \leq \vec{r}) \Phi$, as required.
  \item[(AS-Pot)] Suppose $\Psi ; \Theta ; \Delta \vdash [I|\vec{q}] \tau_1 \subtynf [J|\vec{p}] \tau_2 : \star \gens (I = J) \wedge (\vec{q} \geq \vec{p}) \wedge \Phi_1$
  and $\Psi ; \Theta ; \Delta \vdash [J|\vec{p}] \tau_2 \subtynf [K|\vec{r}] \tau_3 : \star \gens (J = K) \wedge (\vec{p} \geq \vec{r}) \wedge \Phi_2$
  from $\Psi ; \Theta ; \Delta \vdash \tau_1 \subtynf \tau_2 : \star \gens \Phi_1$ and
  $\Psi ; \Theta ; \Delta \vdash \tau_2 \subtynf \tau_3 : \star \gens \Phi_2$.
  By IH, $\Psi ; \Theta ; \Delta \vdash \tau_1 \subtynf \tau_3 : \star \gens \Phi$, with $\Theta ; \Delta \vDash \Phi$.
  By AS-Pot, $\Psi ; \Theta ; \Delta \vdash [I|\vec{q}] \tau_1 \subtynf [K|\vec{r}] \tau_3 : \star \gens (I = K) \wedge (\vec{q} \geq \vec{r}) \Phi$.
  But since $\Theta ; \Delta \vDash (I = J) \wedge (J = K) \wedge (\vec{q} \geq \vec{p}) \wedge (\vec{p} \geq \vec{r})$, we have $\Theta ; \Delta \vDash (I = K) \wedge (\vec{q} \geq \vec{r}) \Phi$, as required.
  \item[(AS-ConstPot)] Suppose $\Psi ; \Theta ; \Delta \vdash [I] \tau_1 \subtynf [J] \tau_2 : \star \gens \Phi_1 \wedge (J \leq I)$
  and $\Psi ; \Theta ; \Delta \vdash [J] \tau_2 \subtynf [K] \tau_3 : \star \gens \Phi_2 \wedge (K \leq J)$
  from $\Psi ; \Theta ; \Delta \vdash \tau_1 \subtynf \tau_2 : \star \gens \Phi_1$ and $\Psi ; \Theta ; \Delta \vdash \tau_2 \subtynf \tau_3 : \star \gens \Phi_3$.
  By IH, $\Psi ; \Theta ; \Delta \vdash \tau_1 \subtynf \tau_3 : \star \gens \Phi$ with $\Theta ; \Delta \vDash \Phi$.
  By AS-ConstPot, $\Psi ; \Theta ; \Delta \vdash [I] \tau_1 \subtynf [K]\tau_3 : \star \gens \Phi$ with $\Theta ; \Delta \vDash \Phi \wedge (K \leq I)$. But, $\Theta ; \Delta \vDash (K \leq J) \wedge (J \leq I)$, so $\Theta ; \Delta \vDash K \leq I$, and so we are done. 
  \item[(AS-FamLam)] Suppose $\Psi ; \Theta ; \Delta \vdash \lambda i : S. \tau_1 \subtynf \lambda i : S. \tau_2 : S \to K \gens \forall i : S. \Phi_1$
  and $\Psi ; \Theta ; \Delta \vdash \lambda i : S. \tau_2 \subtynf \lambda i : S. \tau_3 : S \to K \gens \forall i : S. \Phi_2$
  from $\Psi ; \Theta, i : S ; \Delta \vdash \tau_1 \subtynf \tau_2 : K \gens \Phi_1$
  and $\Psi ; \Theta, i : S ; \Delta \vdash \tau_2 \subtynf \tau_3 : K \gens \Phi_2$.
  By IH, $\Psi ; \Theta, i :S ; \Delta \vDash \tau_1 \subtynf \tau_3 : K \gens \Phi$ with $\Theta , i : S; \Delta \vDash \Phi$, which is by definition $\Theta ; \Delta \vDash \forall i : S. \Phi$.
  Then by AS-Fam, $\Psi ; \Theta ; \Delta \vdash \lambda i : S. \tau_1 \subtynf \lambda i : S. \tau_3 : S \to K \gens \forall i : S. \Phi$, as required.
  \item[(AS-FamApp)]  Suppose $\Psi ; \Theta ; \Delta \vdash \tau_1 \; I \subtynf \tau_2 \; J : K \gens (I = J) \wedge \Phi_1$
  and $\Psi ; \Theta ; \Delta \vdash \tau_2 \; J \subtynf \tau_3 \; L : K \gens (J = L) \wedge \Phi_2$
  from $\Psi ; \Theta ; \Delta \vdash \tau_1 \subtynf \tau_2 : S \to K \gens \Phi_1$ and
   $\Psi ; \Theta ; \Delta \vdash \tau_2 \subtynf \tau_3 : S \to K \gens \Phi_2$.
   By IH, $\Psi ; \Theta ; \Delta \vdash \tau_1 \subtynf \tau_3 : S \to K \gens \Phi$ with $\Theta ; \Delta \vDash \Phi$.
   By AS-FamApp, $\Psi ; \Theta ; \Delta \vdash \tau_1 \; I \subtynf \tau_3 \; L :  K \gens \Phi \wedge (I = L)$ with $\Theta ; \Delta \vDash \Phi$.
   Since $\Theta ; \Delta \vDash (I = J) \wedge (J = L)$, $\Theta ; \Delta \vDash (I=L)$.
\end{itemize}
\end{proof}

\fi


\begin{theorem}[Index Substitution for Algorithmic Sort Checking]
If $\Theta, i : S ; \Delta \pvdash J : S' \gens \Phi_1$ and $\Theta ; \Delta \pvdash I : S \gens \Phi_2$ with
$\Theta, i : S ; \Delta \vDash \Phi_1$ and $\Theta; \Delta \vDash \Phi_2$, then
$\Theta ; \Delta \pvdash J[I/i] : S' \gens \Phi$ for some $\Theta ; \Delta \vDash \Phi$
\label{idx-idx-algo-subst}
\end{theorem}
\begin{proof}
By Theorem~\ref{thm:sort-sound}, $\Theta, i :S ; \Delta \pvdash J : S'$ and $\Theta ; \Delta \pvdash I : S$.
By Theorem~\ref{thm:idx-idx-subst}, $\Theta ; \Delta \pvdash J[I/i] : S'$.
By Theorem~\ref{thm:sort-compl}, $\Theta ; \Delta \pvdash J[I/i] : S' \gens \Phi'$ for some $\Phi'$ such that $\Theta ; \Delta \vDash \Phi'$.
\end{proof}

\begin{theorem}[Index Substitution for Algorithmic Constraint Well-Formedness]
If $\Theta, i : S; \Delta \pvdash \Phi \; \texttt{wf} \gens \Phi_1$ and $\Theta ; \Delta \pvdash I : S \gens \Phi_2$
with $\Theta, i : S ; \Delta \vDash \Phi_1$ and $\Theta ; \Delta \vDash \Phi_2$, then
$\Theta ; \Delta \pvdash  \Phi[I/i] \; \texttt{wf} \gens \Phi'$ for some $\Theta ; \Delta \vDash \Phi'$
\label{thm:constr-idx-algo-subst}
\end{theorem}
\begin{proof}
By Theorem~\ref{thm:constr-sound}, $\Theta, i : S ; \Delta \pvdash \Phi \; \texttt{wf}$.
By Theorem~\ref{thm:sort-sound}, $\Theta ; \Delta \pvdash I : S$.
By Theorem~\ref{thm:constr-idx-subst}, $\Theta ; \Delta \pvdash \Phi[I/i] \; \texttt{wf}$.
By Theorem~\ref{thm:constr-compl}, $\Theta ; \Delta \pvdash \Phi[I/i] \; \texttt{wf} \gens \Phi'$ for some $\Theta ; \Delta \vDash \Phi'$
\end{proof}

\begin{theorem}[Index Substitution for Algorithmic Type Formation]
If $\Psi ; \Theta, i : S ; \Delta \pvdash \tau : K \gens \Phi_1$ and $\Theta ; \Delta \pvdash I : S \gens \Phi_2$ 
with $\Theta, i : S ; \Delta \vDash \Phi_1$ and $\Theta ; \Delta \vDash \Phi_2$, then
$\Psi ; \Theta ; \Delta \pvdash \tau[I/i] : K \gens \Phi$ for some $\Theta ; \Delta \vDash \Phi$
\label{thm:type-idx-algo-subst}
\end{theorem}
\begin{proof}
By Theorem~\ref{thm:kind-sound}, $\Psi ; \Theta, i : S ; \Delta \pvdash \tau : K$.
By Theorem~\ref{thm:sort-sound}, $\Theta ; \Delta \pvdash I : S$.
By Theorem~\ref{thm:type-idx-subst}, $\Psi ; \Theta ; \Delta \pvdash \tau[I/i] : K$.
Finally, by Theorem~\ref{thm:kind-compl}, $\Psi ; \Theta ; \Delta \pvdash \tau[I/i] : K \gens \Phi$ for some $\Theta ; \Delta \vDash \Phi$
\end{proof}

\subtynfidxsubst*
\jtheorem{Proof of \autoref{thm:subtynf-idx-subst}}{

  \jgivengoal{
    \caseFact{1} $\Psi ; \Theta, i : S ; \Delta \pvdash \tau_1 \subtynf \tau_2 : K \gens \Phi$
    
    \caseFact{2} $\Theta ; \Delta \vDash \Phi$
    
    \caseFact{3} $\Theta \vdash \Delta \; \texttt{wf}$
    
    \caseFact{4} $\Theta ; \Delta \pvdash I : S \gens \Phi_1$ with $\Theta ; \Delta \vDash \Phi_1$
    
    \caseFact{5} $\Theta ; \Delta \pvdash J : S \gens \Phi_2$ with $\Theta ; \Delta \vDash \Phi_2$ 
    
    \caseFact{6} $\Theta ; \Delta \vDash I = J$  
  }{
    $\Psi ; \Theta ; \Delta \pvdash \tau_1[I/i] \subtynf \tau_2[J/i] : K \gens \Phi'$ for some $\Phi'$ with $\Theta ; \Delta \vDash \Phi'$
  }
  
  \caseText{In most cases, it suffices to show that $\Psi ; \Theta ; \Delta \vdash \tau_1[I/i] \subtynf \tau_2[J/i] : K \gens \Phi'$ for some solvable $\Phi'$
    since \autoref{thm:idx-ctx-wf-compl} on (3), \autoref{thm:idx-subst-nf} along with the presuppositions for (1) give the presuppositions for the conclusion,
    using \autoref{thm:type-idx-algo-subst}. When this is not immediate, we manually reconstruct the presuppositions required.
  }
  
  \jcase{1}{AS-Unit}{Immediate.}
  \jcase{2}{AS-Var}{Immediate.}
  \jcase{3}{AS-Arr}{
   \jgivengoal{
     \caseFact{7} $\Psi ; \Theta, i : S ; \Delta \pvdash \tau_1 \loli \tau_2 \subtynf \tau_1' \loli \tau_2' : \star \gens \Phi_1' \wedge \Phi_2'$
     
     \caseFact{8} $\Psi ; \Theta, i : S ; \Delta \vdash \tau_1' \subtynf \tau_1 : \star \gens \Phi_1'$
     
     \caseFact{9} $\Psi ; \Theta, i : S ; \Delta \vdash \tau_2 \subtynf \tau_2' : \star \gens \Phi_2'$
   }{
     $\Psi ; \Theta ; \Delta \vdash (\tau_1 \loli \tau_2)[I/i] \subtynf (\tau_1' \loli \tau_2')[J/i] : \star \gens \Phi'$ for some $\Theta ; \Delta \vDash \Phi'$
   }
   \caseText{By IH on (8) and (9)}
   
   \caseFact{10} $\Psi ; \Theta ; \Delta \pvdash \tau_1'[J/i] \subtynf \tau_1[I/i] : \star \gens \Phi_1''$
   
   \caseFact{11} $\Theta ; \Delta \vDash \Phi_1''$
   
   \caseFact{12} $\Psi ; \Theta ; \Delta \vdash \tau_2[I/i] \subtynf \tau_2'[J/i] : \star \gens \Phi_2''$
   
   \caseFact{13} $\Theta ; \Delta \vDash \Phi_2''$
   
   \caseText{By AS-Arr on (10) and (12)}
   
   \caseFact{14} $\Psi ; \Theta ; \Delta \pvdash \tau_1[I/i] \loli \tau_2[I/i] \subtynf \tau_1'[J/i] \loli \tau_2'[J/i] : \star \gens \Phi_1'' \wedge \Phi_2''$
   
   \caseText{Goal follows immediately from (14), with $\Phi' = \Phi_1'' \wedge \Phi_2''$}
  }
  \jcase{4}{AS-Tensor}{
   \jgivengoal{
     \caseFact{7} $\Psi ; \Theta, i : S ; \Delta \pvdash \tau_1 \otimes \tau_2 \subtynf \tau_1' \otimes \tau_2' : \star \gens \Phi_1' \wedge \Phi_2'$
     
     \caseFact{8} $\Psi ; \Theta, i : S ; \Delta \vdash \tau_1 \subtynf \tau_1' : \star \gens \Phi_1'$
     
     \caseFact{9} $\Psi ; \Theta, i : S ; \Delta \vdash \tau_2 \subtynf \tau_2' : \star \gens \Phi_2'$
   }{
     $\Psi ; \Theta ; \Delta \vdash (\tau_1 \otimes \tau_2)[I/i] \subtynf (\tau_1' \otimes \tau_2')[J/i] : \star \gens \Phi'$ for some $\Theta ; \Delta \vDash \Phi'$
   }
   \caseText{By IH on (8) and (9)}
   
   \caseFact{10} $\Psi ; \Theta ; \Delta \pvdash \tau_1[J/i] \subtynf \tau_1'[I/i] : \star \gens \Phi_1''$
   
   \caseFact{11} $\Theta ; \Delta \vDash \Phi_1''$
   
   \caseFact{12} $\Psi ; \Theta ; \Delta \vdash \tau_2[I/i] \subtynf \tau_2'[J/i] : \star \gens \Phi_2''$
   
   \caseFact{13} $\Theta ; \Delta \vDash \Phi_2''$
   
   \caseText{By AS-Tensor on (10) and (12)}
   
   \caseFact{14} $\Psi ; \Theta ; \Delta \pvdash \tau_1[I/i] \otimes \tau_2[I/i] \subtynf \tau_1'[J/i] \otimes \tau_2'[J/i] : \star \gens \Phi_1'' \wedge \Phi_2''$
   
   \caseText{Goal follows immediately from (14), with $\Phi' = \Phi_1'' \wedge \Phi_2''$}
  }
  
  \jcase{5}{AS-With}{
   \jgivengoal{
     \caseFact{7} $\Psi ; \Theta, i : S ; \Delta \pvdash \tau_1 \amp \tau_2 \subtynf \tau_1' \amp \tau_2' : \star \gens \Phi_1' \wedge \Phi_2'$
     
     \caseFact{8} $\Psi ; \Theta, i : S ; \Delta \vdash \tau_1 \subtynf \tau_1' : \star \gens \Phi_1'$
     
     \caseFact{9} $\Psi ; \Theta, i : S ; \Delta \vdash \tau_2 \subtynf \tau_2' : \star \gens \Phi_2'$
   }{
     $\Psi ; \Theta ; \Delta \vdash (\tau_1 \amp \tau_2)[I/i] \subtynf (\tau_1' \amp \tau_2')[J/i] : \star \gens \Phi'$ for some $\Theta ; \Delta \vDash \Phi'$
   }
   \caseText{By IH on (8) and (9)}
   
   \caseFact{10} $\Psi ; \Theta ; \Delta \pvdash \tau_1[J/i] \subtynf \tau_1'[I/i] : \star \gens \Phi_1''$
   
   \caseFact{11} $\Theta ; \Delta \vDash \Phi_1''$
   
   \caseFact{12} $\Psi ; \Theta ; \Delta \vdash \tau_2[I/i] \subtynf \tau_2'[J/i] : \star \gens \Phi_2''$
   
   \caseFact{13} $\Theta ; \Delta \vDash \Phi_2''$
   
   \caseText{By AS-With on (10) and (12)}
   
   \caseFact{14} $\Psi ; \Theta ; \Delta \pvdash \tau_1[I/i] \amp \tau_2[I/i] \subtynf \tau_1'[J/i] \amp \tau_2'[J/i] : \star \gens \Phi_1'' \wedge \Phi_2''$
   
   \caseText{Goal follows immediately from (14), with $\Phi' = \Phi_1'' \wedge \Phi_2''$}
  }
  
  \jcase{6}{AS-Sum}{
   \jgivengoal{
     \caseFact{7} $\Psi ; \Theta, i : S ; \Delta \pvdash \tau_1 \oplus \tau_2 \subtynf \tau_1' \oplus \tau_2' : \star \gens \Phi_1' \wedge \Phi_2'$
     
     \caseFact{8} $\Psi ; \Theta, i : S ; \Delta \vdash \tau_1 \subtynf \tau_1' : \star \gens \Phi_1'$
     
     \caseFact{9} $\Psi ; \Theta, i : S ; \Delta \vdash \tau_2 \subtynf \tau_2' : \star \gens \Phi_2'$
   }{
     $\Psi ; \Theta ; \Delta \vdash (\tau_1 \oplus \tau_2)[I/i] \subtynf (\tau_1' \oplus \tau_2')[J/i] : \star \gens \Phi'$ for some $\Theta ; \Delta \vDash \Phi'$
   }
   \caseText{By IH on (8) and (9)}
   
   \caseFact{10} $\Psi ; \Theta ; \Delta \pvdash \tau_1[J/i] \subtynf \tau_1'[I/i] : \star \gens \Phi_1''$
   
   \caseFact{11} $\Theta ; \Delta \vDash \Phi_1''$
   
   \caseFact{12} $\Psi ; \Theta ; \Delta \vdash \tau_2[I/i] \subtynf \tau_2'[J/i] : \star \gens \Phi_2''$
   
   \caseFact{13} $\Theta ; \Delta \vDash \Phi_2''$
   
   \caseText{By AS-Sum on (10) and (12)}
   
   \caseFact{14} $\Psi ; \Theta ; \Delta \pvdash \tau_1[I/i] \oplus \tau_2[I/i] \subtynf \tau_1'[J/i] \oplus \tau_2'[J/i] : \star \gens \Phi_1'' \wedge \Phi_2''$
   
   \caseText{Goal follows immediately from (14), with $\Phi' = \Phi_1'' \wedge \Phi_2''$}
  }
  
  
  \jcase{7}{AS-Bang}{
    \jgivengoal{
       \caseFact{7} $\Psi ; \Theta, i : S ; \Delta \pvdash !\tau_1 \subtynf !\tau_2 : \star \gens \Phi$
       
       \caseFact{8} $\Psi ; \Theta, i : S ; \Delta \vdash \tau_1 \subtynf \tau_2 : \star \gens \Phi$
    }{
      $\Psi ; \Theta ; \Delta \vdash (!\tau_1)[I/i] \subtynf (!\tau_2)[J/i] : \star \gens \Phi'$ for some $\Theta ; \Delta \vDash \Phi'$
    }  
    
    \caseText{By IH on (8)}
    
    \caseFact{9} $\Psi ; \Theta ; \Delta \pvdash \tau_1[I/i] \subtynf \tau_2[J/i] : \star \gens \Phi'$
    
    \caseFact{10} $\Theta ; \Delta \vDash \Phi'$
    
    \caseText{By AS-Bang on (9)}
    
    \caseFact{9} $\Psi ; \Theta ; \Delta \pvdash !\tau_1[I/i] \subtynf !\tau_2[J/i] : \star \gens \Phi'$
    
    \caseText{Goal follows immediately from (9)}
  }
  
  \jcase{8}{AS-IForall}{
    \jgivengoal{
      \caseFact{7} $\Psi ; \Theta, i : S ; \Delta \pvdash \forall j : S'. \tau_1 \subtynf \forall j : S'. \tau_2 : \star \gens \forall j : S'. \Phi$
      
      \caseFact{8} $\Psi ; \Theta, i : S, j : S'; \Delta \vdash \tau_1 \subtynf \tau_2 : \star \gens \Phi$
    }{
      $\Psi ; \Theta ; \Delta \vdash (\forall j : S'. \tau_1)[I/i] \subtynf (\forall j : S'. \tau_2)[J/i] : \star \gens \Phi'$ for some $\Theta ; \Delta \vDash \Phi'$    
    }
   \caseText{By (2)}
   
   \caseFact{9} $\Theta, i : S, j : S ; \Delta \vDash \Phi $
   
   \caseText{IH on (8)}
   
   \caseFact{10} $\Psi ; \Theta j : S'; \Delta \vdash \tau_1[I/i] \subtynf \tau_2[J//i] : \star \gens \Phi'$
   
   \caseFact{11} $\Theta, j : S' ; \Delta \vDash \Phi'$
   
   \caseText{Equivalently to (11)}
   
   \caseFact{12} $\Theta ; \Delta \vDash \forall j : S'. \Phi'$
   
   \caseText{By AS-IForall on (10)}
   
   \caseFact{13} $Psi ; \Theta ; \Delta \vdash \forall j : S'. \tau_1[I/i] \subtynf \forall j : S'.\tau_2[J/i] : \star \gens \forall j : S'.\Phi'$
   
   \caseText{Goal follows by (12) and (13)}
  
  }
  
  \jcase{8}{AS-IExists}{
    \jgivengoal{
      \caseFact{7} $\Psi ; \Theta, i : S ; \Delta \pvdash \exists j : S'. \tau_1 \subtynf \exists j : S'. \tau_2 : \star \gens \forall j : S'. \Phi$
      
      \caseFact{8} $\Psi ; \Theta, i : S, j : S'; \Delta \vdash \tau_1 \subtynf \tau_2 : \star \gens \Phi$
    }{
      $\Psi ; \Theta ; \Delta \vdash (\exists j : S'. \tau_1)[I/i] \subtynf (\exists j : S'. \tau_2)[J/i] : \star \gens \Phi'$ for some $\Theta ; \Delta \vDash \Phi'$    
    }
   \caseText{By (2)}
   
   \caseFact{9} $\Theta, i : S, j : S ; \Delta \vDash \Phi $
   
   \caseText{IH on (8)}
   
   \caseFact{10} $\Psi ; \Theta j : S'; \Delta \vdash \tau_1[I/i] \subtynf \tau_2[J//i] : \star \gens \Phi'$
   
   \caseFact{11} $\Theta, j : S' ; \Delta \vDash \Phi'$
   
   \caseText{Equivalently to (11)}
   
   \caseFact{12} $\Theta ; \Delta \vDash \forall j : S'. \Phi'$
   
   \caseText{By AS-IExists on (10)}
   
   \caseFact{13} $Psi ; \Theta ; \Delta \vdash \exists j : S'. \tau_1[I/i] \subtynf \exists j : S'.\tau_2[J/i] : \star \gens \forall j : S'.\Phi'$
   
   \caseText{Goal follows by (12) and (13)}
  
  }
  
  \jcase{9}{AS-TForall}{
  
    \jgivengoal{
       \caseFact{7} $\Psi ; \Theta, i : S ; \Delta \pvdash \forall \alpha : K. \tau_1 \subtynf \forall \alpha : K. \tau_2 : \star \gens \Phi$
       
       \caseFact{8} $\Psi, \alpha : K ; \Theta, i : S ; \Delta \vdash \tau_1 \subtynf \tau_2 : \star \gens \Phi$
    }{
      $\Psi ; \Theta ; \Delta \vdash (\forall \alpha : K. \tau_1)[I/i] \subtynf (\forall \alpha : K. \tau_2)[J/i] : \star \gens \Phi'$ for some $\Theta ; \Delta \vDash \Phi'$    
    }
    
    \caseText{By IH on (8)}
    
    \caseFact{9} $\Psi, \alpha : K ; \Theta ; \Delta \pvdash \tau_1[I/i] \subtynf \tau_2[I/i] : \star \gens \Phi'$
    
    \caseFact{10} $\Theta ; \Delta \vDash \Phi'$
    
    \caseText{By AS-TForall on (9)}
    
    \caseFact{11} $\Psi ; \Theta ; \Delta \pvdash \forall \alpha : K. \tau_1[I/i] \subtynf \forall \alpha : K. \tau_2[I/i] : \star \gens \Phi'$
    
    \caseText{Goal follows immediately from (11)}
   
  }
  
  \jcase{10}{AS-List}{
    \jgivengoal{
      \caseFact{7} $\Psi ; \Theta, i : S ; \Delta \pvdash L^M \tau_1 \subtynf L^N \tau_2 : \star \gens M = N \wedge \Phi'$ 
      
      \caseFact{8} $\Psi ; \Theta, i : S ; \Delta \vdash \tau_1 \subtynf \tau_2 : \star \gens \Phi'$ 
    }{
      $\Psi ; \Theta ; \Delta \vdash (L^M \tau_1)[I/i] \subtynf (L^N \tau_2)[J/i] : \star \gens \Phi'$ for some $\Theta ; \Delta \vDash \Phi'$
    }  
    \caseText{By (2)}
    
    \caseFact{9} $\Theta, i : S ; \Delta \vDash M = N$
    
    \caseText{From (9) and (4)}
    
    \caseFact{10} $\Theta ; \Delta \vDash M[I/i] = N[J/i]$
    
    \caseText{By IH on (8)}
    
    \caseFact{11} $\Psi ; \Theta ; \Delta \pvdash \tau_1[I/i] \subtynf \tau_2[J/i] : \star \gens \Phi''$
    
    \caseFact{12} $\Theta ; \Delta \vDash \Phi''$
    
    \caseText{By the presupposition for (7) and two inversions}
    
    \caseFact{13} $\Theta, i : S ; \Delta \vdash M : \N \gens \Phi_1$ with $\Theta, i : S ; \Delta \vDash \Phi_1$
    
    \caseFact{14} $\Theta, i : S ; \Delta \vdash N : \N \gens \Phi_2$ with $\Theta, i : S ; \Delta \vDash \Phi_2$

    \caseText{By \autoref{thm:idx-idx-algo-subst} and \autoref{thm:idx-compl} on (13) (14) (4) (5)}
    
    \caseFact{15} $\Theta ; \Delta \vdash M[I/i] : \N \gens \Phi_1'$ with $\Theta;\Delta \vDash \Phi_1'$
    
    \caseFact{16} $\Theta ; \Delta \vdash N[J/i] : \N \gens \Phi_2'$ with $\Theta ; \Delta \vDash \Phi_1'$
    
    \caseText{By AS-List on (10) and (11)}
    
    \caseFact{17} $\Psi ; \Theta ; \Delta \vdash L^{M[I/i]} \tau_1[I/i] \subtynf L^{N[J/i]} \tau_2[J/i] : \star \gens (M[I/i] = N[J/i]) \wedge \Phi''$
    
    \caseText{By (15), (16), (17)}
    
    \caseFact{18} $\Psi ; \Theta ; \Delta \pvdash L^{M[I/i]} \tau_1[I/i] \subtynf L^{N[J/i]} \tau_2[J/i] : \star \gens (M[I/i] = N[J/i]) \wedge \Phi''$
    
    \caseText{Goal follows from (18), taking $\Phi' = (M[I/i] = N[J/i]) \wedge \Phi''$}
   
  }
  
  \jcase{11}{AS-Conj}{
    \jgivengoal{
      \caseFact{7} $\Psi ; \Theta, i : S; \Delta \pvdash \Phi_1 \amp \tau_1 \subtynf \Phi_2 \amp \tau_2 : \star \gens \Phi \wedge (Phi_1 \to \Phi_2)$
      
      \caseFact{8} $\Psi ; \Theta, i : S ; \Delta \vdash \tau_1 \subtynf \tau_2 : \star \gens \Phi$
    }{
      $\Psi ; \Theta ; \Delta \vdash (\Phi_1 \amp \tau_1)[I/i] \subtynf (\Phi_2 \amp \tau_2)[J/i] : \star \gens \Phi'$ for some $\Theta ; \Delta \vDash \Phi'$
    }
    \caseText{By IH on (8)}
    
    \caseFact{9} $\Psi ; \Theta ; \Delta \pvdash \tau_1[I/i] \subtynf \tau_2[J/i] : \star \gens \Phi'$
    
    \caseFact{10} $\Theta ; \Delta \vDash \Phi'$
    
    \caseText{Inverting the presupposition that $\Phi_1 \amp \tau_1 $ and $\Phi_2 \amp \tau_2$ are well-formed types from (1), we have}
    
    \caseFact{11} $\Theta, i : S ; \Delta \vdash \Phi_1 \; \texttt{wf} \gens \Phi_1'$ with $\Theta, i : S ; \Delta \vDash \Phi_1'$
    
    \caseFact{12} $\Theta, i : S ; \Delta \vdash \Phi_2 \; \texttt{wf} \gens \Phi_2'$ with $\Theta, i : S ; \Delta \vDash \Phi_2'$
    
    \caseText{By \autoref{thm:constr-idx-algo-subst}}
    
    \caseFact{13} $\Theta ; \Delta \vDash \Phi_1[I/i] \; \texttt{wf} \gens \Phi_1''$ with $\Theta ; \Delta \vDash \Phi_1''$
    
    \caseFact{14} $\Theta ; \Delta \vDash \Phi_2[J/i] \; \texttt{wf} \gens \Phi_2''$ with $\Theta ; \Delta \vDash \Phi_2''$
    
    \caseText{By AS-Conj on (9)}
    
    \caseFact{15} $\Psi ; \Theta ; \Delta \vdash \Phi_1[I/i] \amp \tau_1[I/i] \subtynf \Phi_2[J/i] \amp \tau_2[I/i] \gens \Phi' \wedge (\Phi_1[I/i] \to \Phi_2[J/i])$
    
    \caseText{By (11) and (12), the presuppositions for (15) hold}
    
    \caseFact{16} $\Psi ; \Theta ; \Delta \pvdash \Phi_1[I/i] \amp \tau_1[I/i] \subtynf \Phi_2[J/i] \amp \tau_2[I/i] \gens \Phi' \wedge (\Phi_1[I/i] \to \Phi_2[J/i])$
    
    \caseText{By (2)}
    
    \caseFact{17} $\Theta, i : S ; \Delta \vDash \Phi_1 \to \Phi_2$
    
    \caseText{By (17) and (6)}
    
    \caseFact{18} $\Theta ; \Delta \vDash \Phi_1[I/i] \to \Phi_2[J/i]$
    
    \caseText{The result follows by (16) and (18), with $\Phi' = \Phi' \wedge (\Phi_1[I/i] \to \Phi_2[J/i])$}    
  
  }
  
  \jcase{12}{AS-Impl}{
    \jgivengoal{
      \caseFact{7} $\Psi ; \Theta, i : S; \Delta \pvdash \Phi_1 \implies \tau_1 \subtynf \Phi_2 \implies \tau_2 : \star \gens \Phi \wedge (Phi_2 \to \Phi_1)$
      
      \caseFact{8} $\Psi ; \Theta, i : S ; \Delta \vdash \tau_1 \subtynf \tau_2 : \star \gens \Phi$
    }{
      $\Psi ; \Theta ; \Delta \vdash (\Phi_1 \implies \tau_1)[I/i] \subtynf (\Phi_2 \amp \tau_2)[J/i] : \star \gens \Phi'$ for some $\Theta ; \Delta \vDash \Phi'$
    }
    \caseText{By IH on (8)}
    
    \caseFact{9} $\Psi ; \Theta ; \Delta \pvdash \tau_1[I/i] \subtynf \tau_2[J/i] : \star \gens \Phi'$
    
    \caseFact{10} $\Theta ; \Delta \vDash \Phi'$
    
    \caseText{Inverting the presupposition that $\Phi_1 \amp \tau_1 $ and $\Phi_2 \amp \tau_2$ are well-formed types from (1), we have}
    
    \caseFact{11} $\Theta, i : S ; \Delta \vdash \Phi_1 \; \texttt{wf} \gens \Phi_1'$ with $\Theta, i : S ; \Delta \vDash \Phi_1'$
    
    \caseFact{12} $\Theta, i : S ; \Delta \vdash \Phi_2 \; \texttt{wf} \gens \Phi_2'$ with $\Theta, i : S ; \Delta \vDash \Phi_2'$
    
    \caseText{By \autoref{thm:constr-idx-algo-subst}}
    
    \caseFact{13} $\Theta ; \Delta \vDash \Phi_1[I/i] \; \texttt{wf} \gens \Phi_1''$ with $\Theta ; \Delta \vDash \Phi_1''$
    
    \caseFact{14} $\Theta ; \Delta \vDash \Phi_2[J/i] \; \texttt{wf} \gens \Phi_2''$ with $\Theta ; \Delta \vDash \Phi_2''$
    
    \caseText{By AS-Implies on (9)}
    
    \caseFact{15} $\Psi ; \Theta ; \Delta \vdash \Phi_1[I/i] \implies \tau_1[I/i] \subtynf \Phi_2[J/i] \implies \tau_2[I/i] \gens \Phi' \wedge (\Phi_2[J/i] \to \Phi_1[I/i])$
    
    \caseText{By (11) and (12), the presuppositions for (15) hold}
    
    \caseFact{16} $\Psi ; \Theta ; \Delta \pvdash \Phi_1[I/i] \implies \tau_1[I/i] \subtynf \Phi_2[J/i] \implies \tau_2[I/i] \gens \Phi' \wedge (\Phi_2[J/i] \to \Phi_1[I/i])$
    
    \caseText{By (2)}
    
    \caseFact{17} $\Theta, i : S ; \Delta \vDash \Phi_2 \to \Phi_1$
    
    \caseText{By (17) and (6)}
    
    \caseFact{18} $\Theta ; \Delta \vDash \Phi_2[J/i] \to \Phi_1[I/i]$
    
    \caseText{The result follows by (16) and (18), with $\Phi' = \Phi' \wedge (\Phi_2[J/i] \to \Phi_1[I/i])$}    
  
  }
  
  \jcase{13}{AS-Monad}{
    \jgivengoal{
      \caseFact{7} $\Psi ; \Theta, i : S ; \Delta \pvdash \M(M,\vec{q}) \tau_1 \subtynf \M(N,\vec{p}) \tau_2 : \star \gens (M = N) \wedge (\vec{q} \leq \vec{p}) \wedge \Phi$
      
      \caseFact{8} $\Psi ; \Theta, i : S ; \Delta \vdash \tau_1 \subtynf \tau_2 : \star \gens \Phi$
      
    }{
       $\Psi ; \Theta ; \Delta \vdash (\M(M,\vec{q}) \tau_1) [I/i] \subtynf (\M(N,\vec{p}) \tau_2)[J/i] : \star \gens \Phi'$ and $\Theta ; \Delta \vDash \Phi'$
    }
   
   \caseText{From (2)}
   
   \caseFact{9} $\Theta, i : S ; \Delta \vDash M = N$
   
   \caseFact{10} $\Theta, i : S ; \Delta \vDash \vec{q} \leq \vec{p}$
   
   \caseText{By IH on (8)}
   
   \caseFact{11} $\Psi ; \Theta ; \Delta \pvdash \tau_1[I/i] \subtynf \tau_2[J/i] : \star \gens \Phi'$
   
   \caseFact{12} $\Theta ; \Delta \vDash \Phi'$
   
   \caseText{By AS-Monad on (11) and \autoref{thm:idx-idx-subst} on the presuppositions of (1) for $M,N,\vec{q},\vec{p}$}
   
   \caseFact{13} $\Psi ; \Theta ; \Delta \pvdash \M(M[I/i],\vec{q}[I/i]) \; (\tau_1[I/i]) \subtynf \M(N[J/i],\vec{q}[J/i]) \;(\tau_2[J/i]) : \star \gens (M[I/i] = N[J/i]) \wedge (\vec{q}[I/i] \leq \vec{p}[J/i]) \wedge \Phi'$
   
   \caseText{By (6), (9), and (10)}
   
   \caseFact{14} $\Theta ; \Delta \vDash M[I/i] = N[J/i]$
   
   \caseFact{15} $\Theta ; \Delta \vDash \vec{q}[I/i] \leq \vec{p}[J/i]$
   
   \caseText{The Goal is immediate from (12), (13), (14), (15)}
  }
  
  \jcase{13}{AS-Pot}{
    \jgivengoal{
      \caseFact{7} $\Psi ; \Theta, i : S ; \Delta \pvdash [M|\vec{q}] \tau_1 \subtynf [N|\vec{p}] \tau_2 : \star \gens (M = N) \wedge (\vec{q} \geq \vec{p}) \wedge \Phi$
      
      \caseFact{8} $\Psi ; \Theta, i : S ; \Delta \vdash \tau_1 \subtynf \tau_2 : \star \gens \Phi$
      
    }{
       $\Psi ; \Theta ; \Delta \vdash ([M|\vec{q}] \tau_1) [I/i] \subtynf ([N|\vec{p}] \tau_2)[J/i] : \star \gens \Phi'$ and $\Theta ; \Delta \vDash \Phi'$
    }
   
   \caseText{From (2)}
   
   \caseFact{9} $\Theta, i : S ; \Delta \vDash M = N$
   
   \caseFact{10} $\Theta, i : S ; \Delta \vDash \vec{q} \geq \vec{p}$
   
   \caseText{By IH on (8)}
   
   \caseFact{11} $\Psi ; \Theta ; \Delta \pvdash \tau_1[I/i] \subtynf \tau_2[J/i] : \star \gens \Phi'$
   
   \caseFact{12} $\Theta ; \Delta \vDash \Phi'$
   
   \caseText{By AS-Pot on (11) and \autoref{thm:idx-idx-subst} on the presuppositions of (1) for $M,N,\vec{q},\vec{p}$}
   
   \caseFact{13} $\Psi ; \Theta ; \Delta \pvdash [M[I/i]|\vec{q}[I/i]] \; (\tau_1[I/i]) \subtynf [N[J/i]|\vec{q}[J/i]] \;(\tau_2[J/i]) : \star \gens (M[I/i] = N[J/i]) \wedge (\vec{q}[I/i] \geq \vec{p}[J/i]) \wedge \Phi'$
   
   \caseText{By (6), (9), and (10)}
   
   \caseFact{14} $\Theta ; \Delta \vDash M[I/i] = N[J/i]$
   
   \caseFact{15} $\Theta ; \Delta \vDash \vec{q}[I/i] \geq \vec{p}[J/i]$
   
   \caseText{The Goal is immediate from (12), (13), (14), (15)}
  }
  
  \jcase{14}{AS-ConstPot}{
     \jgivengoal{
       \caseFact{7} $\Psi ; \Theta, i : S ; \Delta \pvdash [M] \; \tau_1 \subtynf [N] \ ; \tau_2 : \star \gens \Phi \wedge (N \leq M)$
       
       \caseFact{8} $\Psi ; \Theta, i : S; \Delta \vdash \tau_1 \subtynf \tau_2 : \star \gens \Phi$     
     }{
       $\Psi ; \Theta ; \Delta ; \vdash ([M] \; \tau_1)[I/i] \subtynf ([N] \; \tau_2)[J/i] : \star \gens \Phi'$ and $\Theta ; \Delta \vDash \Phi'$     
     }
     
    \caseText{By (2)}
    
    \caseFact{9} $\Theta, i : S ; \Delta \vDash N \leq M$
    
    \caseText{By IH on (8)}
    
    \caseFact{10} $\Psi ; \Theta ; \Delta \pvdash \tau_1[I/i] \subtynf \tau_2[J/i] : \star \gens \Phi'$
    
    \caseFact{11} $\Theta ; \Delta \vDash \Phi'$
    
    \caseText{By AS-ConstPot and \autoref{thm:idx-idx-subst} for $M,N$}
    
    \caseFact{12} $\Psi ; \Theta ; \Delta \pvdash \left[M[I/i]\right] \; (\tau_1[I/i]) \subtynf \left[N[I/i]\right] \; (\tau_2[J/i]) : \star \gens \Phi' \wedge (N[J/i] \leq M[I/i])$
    
    \caseText{By (9) and (6)}
    
    \caseFact{13} $\Theta ; \Delta \vDash N[J/i] \leq M[I/i]$
    
    \caseText{Goal follows immediately from (11), (12), (13), with $\Phi' = \Phi' \wedge (N[J/i] \leq M[I/i])$}
  }
  
  \jcase{15}{AS-FamLam}{
  
    \jgivengoal{
      \caseFact{7} $\Psi ; \Theta, i : S ; \Delta \pvdash \lambda j : S'. \tau_1 \subtynf \lambda j : S'.\tau_2 : S \to K \gens \forall j : S. \Phi$
      
      \caseFact{8} $\Psi ; \Theta, i : S, j : S' ; \Delta \vdash \tau_1 \subtynf \tau_2 : K \gens \Phi$
    }{
      $\Psi ; \Theta ; \Delta \pvdash (\lambda j : S'.\tau_1)[I/i] \subtynf (\lambda j: S'.\tau_2)[J/i] : S' \to K \gens \Phi'$ and $\Theta ; \Delta \vDash \Phi'$    
    }
    
    \caseText{By IH on (8)}
    
    \caseFact{9} $\Psi ; \Theta, j : S' ; \Delta \pvdash \tau_1[I/i] \subtynf \tau_2[J/i] : K \gens \Phi'$
    
    \caseFact{10} $\Theta, j : S' ; \Delta \vDash \Phi'$
    
    \caseText{By AT-FamLam on (9)}
    
    \caseFact{11} $\Psi ; \Theta ; \Delta \pvdash \forall j : S'.\tau_1[I/i] \subtynf \forall j :S'.\tau_2[J/i] : K \gens \forall j : S'.\Phi'$
  
    \caseText{The goal is immediate by (10) and (11)} 
  }
  
  \jcase{15}{AS-FamApp}{
    \jgivengoal{
       \caseFact{7} $\Psi ; \Theta, i : S; \Delta \pvdash \tau_1 \; M \subtynf \tau_2 \; N : K \gens \Phi \wedge (M = N)$
       
       \caseFact{8} $\Psi ; \Theta, i : S ; \Delta \vdash \tau_1 \subtynf \tau_2 : S' \to K \gens \Phi$
    }{
      $\Psi ; \Theta ; \Delta \pvdash \left(\tau_1 \; M\right)[I/i] \subtynf \left(\tau_2[J/i]\right)[J/i] : K \gens \Phi'$ and $\Theta ; \Delta \vDash \Phi'$
    }
    
    \caseText{By (2)}
    
    \caseFact{9} $\Theta, i : S ; \Delta \vDash M = N$
    
    \caseText{By (9) and (6)}
    
    \caseFact{10}  $\Theta ; \Delta \vDash M[I/i] = N[J/i]$
    
    \caseText{IH on (8)}
    
    \caseFact{11} $\Psi ; \Theta ; \Delta \vdash \tau_1[I/i] \subtynf \tau_2[J/i] : S' \to K \gens \Phi'$
    
    \caseFact{12} $\Theta ; \Delta \vDash \Phi'$
    
    \caseText{By AS-FamApp on (11) and \autoref{thm:idx-idx-algo-subst} on the proofs that $M,N : S'$ in (7)}
    
    \caseFact{13} $\Psi ; \Theta ; \Delta \pvdash \left(\tau_1[I/i]\right) \; M[I/i] \subtynf \left(\tau_2[J/i]\right) \; N[J/i] : K \gens \Phi' \wedge (M[I/i] = N[J/i])$
    
    \caseText{The Goal follows by (10), (12), (13)}
    
  }

}

\evalapplemma*
\begin{proof}
By inversion on $\Psi ; \Theta ; \Delta \vdash \texttt{eval}(\tau_1) \subtynf \texttt{eval}(\tau_2) : S \to K \gens \Phi$.
\begin{itemize}
  \item For the first case, suppose the derivation was $\Psi ; \Theta ; \Delta \vdash \lambda i : S. \tau_1' \subtynf \tau_2' : S \to K \gens \Phi$
  from $\Psi ; \Theta, i : S ; \Delta \vdash \tau_1' \subtynf \tau_2' : K \gens \Phi'$. By Theorem~\ref{thm:subtynf-idx-subst},
  $\Psi ; \Theta ; \Delta \pvdash \tau_1'[I/i] \subtynf \tau_2'[J/i] : K \gens \Phi'$, for some $\Theta ; \Delta \vDash \Phi'$. But $\texttt{eval}(\tau_1 \; I) = \tau_1'[I/i]$ and $\texttt{eval}(\tau_2 \; J) = \tau_2'[J/i]$.
  \item Now, suppose the derivation was $\Psi ; \Theta ; \Delta \vdash \tau_1' \; L_1 \subtynf \tau_2' \; L_2 : S \to K \gens \Phi \wedge (L_1 = L_2)$, where $\texttt{eval}(\tau_1) = \tau_1' \; L_1$ and $\texttt{eval}(\tau_2) = \tau_2 \; L_2$. These must both be $\texttt{ne}$, since they are both applications, and therefore
  $\texttt{eval}(\tau_1) \; I = \texttt{eval}(\tau_1 \; I)$ and $\texttt{eval}(\tau_2) \; J = \texttt{eval}(\tau_2 \; J)$, as required.
\end{itemize}
\end{proof}

\subtycompl*
\jtheorem{Proof of \autoref{thm:subty-compl}}{
  \caseFact{1} By induction on $\Psi ; \Theta ; \Delta \vdash \tau_1 \subty \tau_2 : K$.
  
  \jcase{1}{S-Refl}{
    Immediate by \autoref{thm:subty-refl}.
  }
  
  \jcase{2}{S-Trans}{
    \jgivengoal{
      \caseFact{1} $\Psi ; \Theta ; \Delta \pvdash \tau_1 \subty \tau_3 : K$
      
      \caseFact{2} $\Psi ; \Theta ; \Delta \pvdash \tau_1 \subty \tau_2 : K$
      
      \caseFact{3} $\Psi ; \Theta ; \Delta \pvdash \tau_2 \subty \tau_3 : K$
    }{
      $\Psi ; \Theta ; \Delta \pvdash \tau_1 \subty \tau_3 : K \gens \Phi$ and $\Theta ; \Delta \vDash \Phi$
    }
    
    \caseText{By IH on (2)}
    
    \caseFact{4} $\Psi ; \Theta ; \Delta \pvdash \tau_1 \subty \tau_2 : K \gens \Phi_1$
    
    \caseFact{5} $\Theta ; \Delta \vDash \Phi_1$
    
    \caseText{By IH on (3)}
    
    \caseFact{6} $\Psi ; \Theta ; \Delta \pvdash \tau_2 \subty \tau_3 : K \gens \Phi_2$
    
    \caseFact{7} $\Theta ; \Delta \vDash \Phi_2$
    
    \caseText{Goal follows by \autoref{subty-trans} on (4), (5), (6), (7)}
  }
  
  \jcase{3}{S-Arr}{
    \jgivengoal{
     \caseFact{1} $\Psi ; \Theta ; \Delta \pvdash \tau_1 \loli \tau_2 \subty \tau_1' \loli \tau_2' : \star$    
     
     \caseFact{2} $\Psi ; \Theta ; \Delta \pvdash \tau_1' \subty \tau_1 : \star$
     
     \caseFact{3} $\Psi ; \Theta ; \Delta \pvdash \tau_2 \subty \tau_2' : \star$
    }{
     $\Psi ; \Theta ; \Delta \pvdash \tau_1 \loli \tau_2 \subty \tau_1' \loli \tau_2' : \star \gens \Phi$ and $\Theta ; \Delta \vDash \Phi$
    }
    
    \caseText{By IH on (2)} 
    
    \caseFact{4} $\Psi ; \Theta ; \Delta \pvdash \tau_1' \subty \tau_1 : \star \gens \Phi_1$
    
    \caseFact{5} $\Theta ; \Delta \vDash \Phi_1$
    
    \caseText{By IH on (3)}
    
    \caseFact{6} $\Psi ; \Theta ; \Delta \vdash \tau_2 \subty \tau_2' : \star \gens \Phi_2$
    
    \caseFact{7} $\Theta ; \Delta \vDash \Phi_2$
    
    \caseText{Inverting (4) and (6)}
    
    \caseFact{8} $\Psi ; \Theta ; \Delta \pvdash \texttt{eval}(\tau_1') \subtynf \texttt{eval}(\tau_1) : \star \gens \Phi_1$
    
    \caseFact{9} $\Psi ; \Theta ; \Delta \pvdash \texttt{eval}(\tau_2) \subtynf \texttt{eval}(\tau_2') : \star \gens \Phi_2$
    
    \caseText{By AS-Arr on (8) and (9)}
    
    \caseFact{10} $\Psi ; \Theta ; \Delta \pvdash \texttt{eval}(\tau_1) \loli \texttt{eval}(\tau_2) \subtynf \texttt{eval}(\tau_1') \loli \texttt{eval}(\tau_2') : \star \gens \Phi_1 \wedge \Phi_2$
    
    \caseText{Goal follows by AS-Normalize on (10), taking $\Phi = \Phi_1 \wedge \Phi_2$}
  }
  
  \jcase{4}{S-Tensor}{
    \jgivengoal{
     \caseFact{1} $\Psi ; \Theta ; \Delta \pvdash \tau_1 \otimes \tau_2 \subty \tau_1' \otimes \tau_2' : \star$    
     
     \caseFact{2} $\Psi ; \Theta ; \Delta \pvdash \tau_1 \subty \tau_1' : \star$
     
     \caseFact{3} $\Psi ; \Theta ; \Delta \pvdash \tau_2 \subty \tau_2' : \star$
    }{
     $\Psi ; \Theta ; \Delta \pvdash \tau_1 \otimes \tau_2 \subty \tau_1' \otimes \tau_2' : \star \gens \Phi$ and $\Theta ; \Delta \vDash \Phi$
    }
    
    \caseText{By IH on (2)} 
    
    \caseFact{4} $\Psi ; \Theta ; \Delta \pvdash \tau_1 \subty \tau_1' : \star \gens \Phi_1$
    
    \caseFact{5} $\Theta ; \Delta \vDash \Phi_1$
    
    \caseText{By IH on (3)}
    
    \caseFact{6} $\Psi ; \Theta ; \Delta \vdash \tau_2 \subty \tau_2' : \star \gens \Phi_2$
    
    \caseFact{7} $\Theta ; \Delta \vDash \Phi_2$
    
    \caseText{Inverting (4) and (6)}
    
    \caseFact{8} $\Psi ; \Theta ; \Delta \pvdash \texttt{eval}(\tau_1) \subtynf \texttt{eval}(\tau_1') : \star \gens \Phi_1$
    
    \caseFact{9} $\Psi ; \Theta ; \Delta \pvdash \texttt{eval}(\tau_2) \subtynf \texttt{eval}(\tau_2') : \star \gens \Phi_2$
    
    \caseText{By AS-Tensor on (8) and (9)}
    
    \caseFact{10} $\Psi ; \Theta ; \Delta \pvdash \texttt{eval}(\tau_1) \otimes \texttt{eval}(\tau_2) \subtynf \texttt{eval}(\tau_1') \otimes \texttt{eval}(\tau_2') : \star \gens \Phi_1 \wedge \Phi_2$
    
    \caseText{Goal follows by AS-Normalize on (10), taking $\Phi = \Phi_1 \wedge \Phi_2$}
  }
  
  \jcase{5}{S-With}{
    \jgivengoal{
     \caseFact{1} $\Psi ; \Theta ; \Delta \pvdash \tau_1 \amp \tau_2 \subty \tau_1' \amp \tau_2' : \star$    
     
     \caseFact{2} $\Psi ; \Theta ; \Delta \pvdash \tau_1 \subty \tau_1' : \star$
     
     \caseFact{3} $\Psi ; \Theta ; \Delta \pvdash \tau_2 \subty \tau_2' : \star$
    }{
     $\Psi ; \Theta ; \Delta \pvdash \tau_1 \amp \tau_2 \subty \tau_1' \amp \tau_2' : \star \gens \Phi$ and $\Theta ; \Delta \vDash \Phi$
    }
    
    \caseText{By IH on (2)} 
    
    \caseFact{4} $\Psi ; \Theta ; \Delta \pvdash \tau_1 \subty \tau_1' : \star \gens \Phi_1$
    
    \caseFact{5} $\Theta ; \Delta \vDash \Phi_1$
    
    \caseText{By IH on (3)}
    
    \caseFact{6} $\Psi ; \Theta ; \Delta \vdash \tau_2 \subty \tau_2' : \star \gens \Phi_2$
    
    \caseFact{7} $\Theta ; \Delta \vDash \Phi_2$
    
    \caseText{Inverting (4) and (6)}
    
    \caseFact{8} $\Psi ; \Theta ; \Delta \pvdash \texttt{eval}(\tau_1) \subtynf \texttt{eval}(\tau_1') : \star \gens \Phi_1$
    
    \caseFact{9} $\Psi ; \Theta ; \Delta \pvdash \texttt{eval}(\tau_2) \subtynf \texttt{eval}(\tau_2') : \star \gens \Phi_2$
    
    \caseText{By AS-With on (8) and (9)}
    
    \caseFact{10} $\Psi ; \Theta ; \Delta \pvdash \texttt{eval}(\tau_1) \amp \texttt{eval}(\tau_2) \subtynf \texttt{eval}(\tau_1') \amp \texttt{eval}(\tau_2') : \star \gens \Phi_1 \wedge \Phi_2$
    
    \caseText{Goal follows by AS-Normalize on (10), taking $\Phi = \Phi_1 \wedge \Phi_2$}
  }
  
  \jcase{6}{S-Sum}{
    \jgivengoal{
     \caseFact{1} $\Psi ; \Theta ; \Delta \pvdash \tau_1 \oplus \tau_2 \subty \tau_1' \oplus \tau_2' : \star$    
     
     \caseFact{2} $\Psi ; \Theta ; \Delta \pvdash \tau_1 \subty \tau_1' : \star$
     
     \caseFact{3} $\Psi ; \Theta ; \Delta \pvdash \tau_2 \subty \tau_2' : \star$
    }{
     $\Psi ; \Theta ; \Delta \pvdash \tau_1 \oplus \tau_2 \subty \tau_1' \oplus \tau_2' : \star \gens \Phi$ and $\Theta ; \Delta \vDash \Phi$
    }
    
    \caseText{By IH on (2)} 
    
    \caseFact{4} $\Psi ; \Theta ; \Delta \pvdash \tau_1 \subty \tau_1' : \star \gens \Phi_1$
    
    \caseFact{5} $\Theta ; \Delta \vDash \Phi_1$
    
    \caseText{By IH on (3)}
    
    \caseFact{6} $\Psi ; \Theta ; \Delta \vdash \tau_2 \subty \tau_2' : \star \gens \Phi_2$
    
    \caseFact{7} $\Theta ; \Delta \vDash \Phi_2$
    
    \caseText{Inverting (4) and (6)}
    
    \caseFact{8} $\Psi ; \Theta ; \Delta \pvdash \texttt{eval}(\tau_1) \subtynf \texttt{eval}(\tau_1') : \star \gens \Phi_1$
    
    \caseFact{9} $\Psi ; \Theta ; \Delta \pvdash \texttt{eval}(\tau_2) \subtynf \texttt{eval}(\tau_2') : \star \gens \Phi_2$
    
    \caseText{By AS-Sum on (8) and (9)}
    
    \caseFact{10} $\Psi ; \Theta ; \Delta \pvdash \texttt{eval}(\tau_1) \oplus \texttt{eval}(\tau_2) \subtynf \texttt{eval}(\tau_1') \oplus \texttt{eval}(\tau_2') : \star \gens \Phi_1 \wedge \Phi_2$
    
    \caseText{Goal follows by AS-Normalize on (10), taking $\Phi = \Phi_1 \wedge \Phi_2$}
  }
  
  \jcase{7}{S-Bang}{
    \jgivengoal{
     \caseFact{1} $\Psi ; \Theta ; \Delta \pvdash !\tau_1\subty!\tau_2 : \star$    
     
     \caseFact{2} $\Psi ; \Theta ; \Delta \pvdash \tau_1 \subty \tau_2 : \star$
    
    }{
     $\Psi ; \Theta ; \Delta \pvdash !\tau_1\subty!\tau_2 : \star \gens \Phi$ and $\Theta ; \Delta \vDash \Phi$
    }
    
    \caseText{By IH on (2)} 
    
    \caseFact{4} $\Psi ; \Theta ; \Delta \pvdash \tau_1 \subty \tau_2 : \star \gens \Phi$
    
    \caseFact{5} $\Theta ; \Delta \vDash \Phi$
    
    \caseText{Inverting (5)}
    
    \caseFact{6} $\Psi ; \Theta ; \Delta \pvdash \texttt{eval}(\tau_1) \subtynf \texttt{eval}(\tau_2) : \star \gens \Phi$
 
    \caseText{By AS-Bang on (6)}
    
    \caseFact{7} $\Psi ; \Theta ; \Delta \pvdash !\texttt{eval}(\tau_1) \subtynf !\texttt{eval}(\tau_2) : \star \gens \Phi$
    
    \caseText{Goal follows by AS-Normalize on (7)}
  }
  
  \jcase{8}{S-IForall}{
    \jgivengoal{
     \caseFact{1} $\Psi ; \Theta ; \Delta \pvdash \forall i : S.\tau_1\subty\forall i : S.\tau_2 : \star$    
     
     \caseFact{2} $\Psi ; \Theta, i : S ; \Delta \pvdash \tau_1 \subty \tau_2 : \star$
    
    }{
     $\Psi ; \Theta ; \Delta \pvdash \forall i : S.\tau_1\subty\forall i : S.\tau_2 : \star \gens \Phi$ and $\Theta ; \Delta \vDash \Phi$
    }
    
    \caseText{By IH on (2)} 
    
    \caseFact{4} $\Psi ; \Theta, i : S ; \Delta \pvdash \tau_1 \subty \tau_2 : \star \gens \Phi$
    
    \caseFact{5} $\Theta, i : S ; \Delta \vDash \Phi$
    
    \caseText{Inverting (5)}
    
    \caseFact{6} $\Psi ; \Theta, i : S ; \Delta \pvdash \texttt{eval}(\tau_1) \subtynf \texttt{eval}(\tau_2) : \star \gens \Phi$
 
    \caseText{By AS-IForall on (6)}
    
    \caseFact{7} $\Psi ; \Theta ; \Delta \pvdash \forall i : S. \texttt{eval}(\tau_1) \subtynf \forall i : S. \texttt{eval}(\tau_2) : \star \gens \forall i : S.\Phi$
    
    \caseText{Goal follows by AS-Normalize on (7)}
  }
  
  \jcase{9}{S-IExists}{
    \jgivengoal{
     \caseFact{1} $\Psi ; \Theta ; \Delta \pvdash \exists i : S.\tau_1\subty\exists i : S.\tau_2 : \star$    
     
     \caseFact{2} $\Psi ; \Theta, i : S ; \Delta \pvdash \tau_1 \subty \tau_2 : \star$
    
    }{
     $\Psi ; \Theta ; \Delta \pvdash \exists i : S.\tau_1\subty\exists i : S.\tau_2 : \star \gens \Phi$ and $\Theta ; \Delta \vDash \Phi$
    }
    
    \caseText{By IH on (2)} 
    
    \caseFact{4} $\Psi ; \Theta, i : S ; \Delta \pvdash \tau_1 \subty \tau_2 : \star \gens \Phi$
    
    \caseFact{5} $\Theta, i : S ; \Delta \vDash \Phi$
    
    \caseText{Inverting (5)}
    
    \caseFact{6} $\Psi ; \Theta, i : S ; \Delta \pvdash \texttt{eval}(\tau_1) \subtynf \texttt{eval}(\tau_2) : \star \gens \Phi$
 
    \caseText{By AS-IExists on (6)}
    
    \caseFact{7} $\Psi ; \Theta ; \Delta \pvdash \exists i : S. \texttt{eval}(\tau_1) \subtynf \exists i : S. \texttt{eval}(\tau_2) : \star \gens \forall i : S.\Phi$
    
    \caseText{Goal follows by AS-Normalize on (7)}
  }
  
  \jcase{10}{S-TForall}{
    \jgivengoal{
     \caseFact{1} $\Psi ; \Theta ; \Delta \pvdash \forall \alpha : K.\tau_1\subty\forall \alpha : K.\tau_2 : \star$    
     
     \caseFact{2} $\Psi,\alpha : K ; \Theta ; \Delta \pvdash \tau_1 \subty \tau_2 : \star$
    
    }{
     $\Psi ; \Theta ; \Delta \pvdash \forall \alpha : K.\tau_1\subty\forall \alpha : K.\tau_2 : \star \gens \Phi$ and $\Theta ; \Delta \vDash \Phi$
    }
    
    \caseText{By IH on (2)} 
    
    \caseFact{4} $\Psi, \alpha : K ; \Theta ; \Delta \pvdash \tau_1 \subty \tau_2 : \star \gens \Phi$
    
    \caseFact{5} $\Theta ; \Delta \vDash \Phi$
    
    \caseText{Inverting (5)}
    
    \caseFact{6} $\Psi , \alpha : K ; \Theta ; \Delta \pvdash \texttt{eval}(\tau_1) \subtynf \texttt{eval}(\tau_2) : \star \gens \Phi$
 
    \caseText{By AS-IForall on (6)}
    
    \caseFact{7} $\Psi ; \Theta ; \Delta \pvdash \forall \alpha : K. \texttt{eval}(\tau_1) \subtynf \forall \alpha : K. \texttt{eval}(\tau_2) : \star \gens \Phi$
    
    \caseText{Goal follows by AS-Normalize on (7)}
  }
  
  \jcase{11}{S-List}{
   \jgivengoal{
    \caseFact{1} $\Psi ; \Theta ; \Delta \pvdash L^I \tau_1 \subty L^J \tau_2 : \star$
    
    \caseFact{2} $\Psi ; \Theta ; \Delta \pvdash \tau_1 \subty \tau_2 : \star$
    
    \caseFact{3} $\Theta ; \Delta \vDash I = J$
   }{
   $\Psi ; \Theta ; \Delta \pvdash L^I \tau_1 \subty L^J \tau_2 : \star \gens \Phi$ and $\Theta ; \Delta \vDash \Phi$  
   }
     \caseText{By IH on (2)}
     
     \caseFact{5} $\Psi ; \Theta ; \Delta \vdash \tau_1 \subty \tau_2 : \star \gens \Phi$
     
     \caseFact{6} $\Theta ; \Delta \vDash \Phi$
     
     \caseText{By (3) and (6)}
     
     \caseFact{7} $\Theta ; \Delta \vDash \Phi \wedge (I = J)$
     
     \caseText{By inversion on (5)}
     
     \caseFact{8} $\Psi ; \Theta ; \Delta \vdash \texttt{eval}(\tau_1) \subtynf \texttt{eval}(\tau_2) : \star \gens \Phi$
     
     \caseText{By AS-List on (8)}
     
     \caseFact{9} $\Psi ; \Theta ; \Delta \vdash L^I\texttt{eval}(\tau_1) \subtynf L^J\texttt{eval}(\tau_2) : \star \gens \Phi \wedge (I = J)$
     
     \caseText{Goal follows by AS-Normalize on (9)}
  }
  
  \jcase{12}{S-Impl}{
    \jgivengoal{
      \caseFact{1} $\Psi ; \Theta ; \Delta \pvdash \Phi_1 \implies \tau_1 \subty \Phi_2 \implies \tau_2 : \star$
      
      \caseFact{2} $\Psi ; \Theta ; \Delta \pvdash \tau_1 \subty \tau_2 : \star$
      
      \caseFact{3} $\Theta;\Delta \vDash \Phi_2 \to \Phi_1$
    }{
     $\Psi ; \Theta ; \Delta \pvdash \Phi_1 \implies \tau_1 \subty \Phi_2 \implies \tau_2 : \star \gens \Phi$ and $\Theta  ; \Delta \vDash \Phi$    
    }
     \caseText{By IH on (2)}    
     
     \caseFact{4} $\Psi ; \Theta ; \Delta \vdash \tau_1 \subty \tau_2 : \star \gens \Phi$
     
     \caseFact{5} $\Theta ; \Delta \vDash \Phi$
     
     \caseText{By (3) and (5)}
     
     \caseFact{6} $\Theta ; \Delta \vDash \Phi \wedge (\Phi_2 \to \Phi_1)$
     
     \caseText{Inverting (4)}
     
     \caseFact{7} $\Psi ; \Theta ; \Delta \vdash \texttt{eval}(\tau_1) \subtynf \texttt{eval}(\tau_2) : \star \gens \Phi$
     
     \caseText{By AS-Impl on (7)}
     
     \caseFact{8} $\Psi ; \Theta ; \Delta \vdash \Phi_1 \implies \texttt{eval}(\tau_1) \subtynf \Phi_2 \implies \texttt{eval}(\tau_2) : \star \gens \Phi \wedge (\Phi_2 \to \Phi_1)$
     
     \caseFact{Goal follows by AS-Normalize on (8)}
  }
  
  \jcase{13}{S-Conj}{
    \jgivengoal{
      \caseFact{1} $\Psi ; \Theta ; \Delta \pvdash \Phi_1 \amp \tau_1 \subty \Phi_2 \amp \tau_2 : \star$
      
      \caseFact{2} $\Psi ; \Theta ; \Delta \pvdash \tau_1 \subty \tau_2 : \star$
      
      \caseFact{3} $\Theta;\Delta \vDash \Phi_1 \to \Phi_2$
    }{
    $\Psi ; \Theta ; \Delta \pvdash \Phi_1 \amp \tau_1 \subty \Phi_2 \amp \tau_2 : \star \gens \Phi$ and $\Theta ; \Delta \vDash \Phi$    
    }
     \caseText{By IH on (2)}    
     
     \caseFact{4} $\Psi ; \Theta ; \Delta \vdash \tau_1 \subty \tau_2 : \star \gens \Phi$
     
     \caseFact{5} $\Theta ; \Delta \vDash \Phi$
     
     \caseText{By (3) and (5)}
     
     \caseFact{6} $\Theta ; \Delta \vDash \Phi \wedge (\Phi_1 \to \Phi_2)$
     
     \caseText{Inverting (4)}
     
     \caseFact{7} $\Psi ; \Theta ; \Delta \vdash \texttt{eval}(\tau_1) \subtynf \texttt{eval}(\tau_2) : \star \gens \Phi$
     
     \caseText{By AS-Conj on (7)}
     
     \caseFact{8} $\Psi ; \Theta ; \Delta \vdash \Phi_1 \amp \texttt{eval}(\tau_1) \subtynf \Phi_2 \amp \texttt{eval}(\tau_2) : \star \gens \Phi \wedge (\Phi_1 \to \Phi_2)$
     
     \caseFact{Goal follows by AS-Normalize on (8)}
  }
  
  \jcase{14}{S-Monad}{
   \jgivengoal{
    \caseFact{1} $\Psi ; \Theta ; \Delta \pvdash \M(I,\vec{q}) \tau_1 \subty \M(J,\vec{p}) \tau_2 : \star$
    
    \caseFact{2} $\Psi ; \Theta ; \Delta \pvdash \tau_1 \subty \tau_2 : \star$
    
    \caseFact{3} $\Theta ; \Delta \vDash (I = J) \wedge (\vec{q} \leq \vec{p})$
   }{
    $\Psi ; \Theta ; \Delta \pvdash \M(I,\vec{q}) \tau_1 \subty \M(J,\vec{p}) \tau_2 : \star \gens \Phi$ and $\Theta ; \Delta \vDash \Phi$   
   }
    \caseText{By IH on (2)}  
    
    \caseFact{4} $\Psi ; \Theta ; \Delta \pvdash \tau_1 \subty \tau_2 : \star \gens \Phi$
    
    \caseFact{5} $\Theta ; \Delta \vDash \Phi$ 
    
    \caseText{From (3) and (5)}   
    
    \caseFact{6} $\Theta ; \Delta \vDash \Phi \wedge (I = J) \wedge (\vec{q} \leq \vec{p})$
    
    \caseText{By inversion on (4)}
    
    \caseFact{7} $\Psi ; \Theta ; \Delta \pvdash \texttt{eval}(\tau_1) \subtynf \texttt{eval}(\tau_2) : \star \gens \Phi$
    
    \caseText{By AS-Monad on (7)}
    
    \caseFact{8} $\Psi ; \Theta ; \Delta \pvdash \M(I,\vec{q})\texttt{eval}(\tau_1) \subtynf \M(J,\vec{p})\texttt{eval}(\tau_2) : \star \gens \Phi \wedge (I = J) \wedge (\vec{q} \leq \vec{p})$
    
    \caseText{Goal follows by AS-Normalize on (8)}
  }
  
  \jcase{15}{S-Pot}{
   \jgivengoal{
    \caseFact{1} $\Psi ; \Theta ; \Delta \pvdash [I|\vec{q}] \, \tau_1 \subty [J|\vec{p}] \, \tau_2 : \star$
    
    \caseFact{2} $\Psi ; \Theta ; \Delta \pvdash \tau_1 \subty \tau_2 : \star$
    
    \caseFact{3} $\Theta ; \Delta \vDash (I = J) \wedge (\vec{q} \geq \vec{p})$
   }{
    $\Psi ; \Theta ; \Delta \pvdash [I|\vec{q}]\, \tau_1 \subty [J|\vec{p}] \, \tau_2 : \star \gens \Phi$ and $\Theta ; \Delta \vDash \Phi$   
   }
    \caseText{By IH on (2)}  
    
    \caseFact{4} $\Psi ; \Theta ; \Delta \pvdash \tau_1 \subty \tau_2 : \star \gens \Phi$
    
    \caseFact{5} $\Theta ; \Delta \vDash \Phi$ 
    
    \caseText{From (3) and (5)}   
    
    \caseFact{6} $\Theta ; \Delta \vDash \Phi \wedge (I = J) \wedge (\vec{q} \geq \vec{p})$
    
    \caseText{By inversion on (4)}
    
    \caseFact{7} $\Psi ; \Theta ; \Delta \pvdash \texttt{eval}(\tau_1) \subtynf \texttt{eval}(\tau_2) : \star \gens \Phi$
    
    \caseText{By AS-Pot on (7)}
    
    \caseFact{8} $\Psi ; \Theta ; \Delta \pvdash [I|\vec{q}]\texttt{eval}(\tau_1) \subtynf [J|\vec{p}]\texttt{eval}(\tau_2) : \star \gens \Phi \wedge (I = J) \wedge (\vec{q} \geq \vec{p})$
    
    \caseText{Goal follows by AS-Normalize on (8)}
  }
  
  \jcase{16}{S-ConstPot}{
   \jgivengoal{
    \caseFact{1} $\Psi ; \Theta ; \Delta \pvdash [I] \, \tau_1 \subty [J] \, \tau_2 : \star$
    
    \caseFact{2} $\Psi ; \Theta ; \Delta \pvdash \tau_1 \subty \tau_2 : \star$
    
    \caseFact{3} $\Theta ; \Delta \vDash (I \geq J)$
   }{
    $\Psi ; \Theta ; \Delta \pvdash [I]\, \tau_1 \subty [J] \, \tau_2 : \star \gens \Phi$ and $\Theta ; \Delta \vDash \Phi$   
   }
    \caseText{By IH on (2)}  
    
    \caseFact{4} $\Psi ; \Theta ; \Delta \pvdash \tau_1 \subty \tau_2 : \star \gens \Phi$
    
    \caseFact{5} $\Theta ; \Delta \vDash \Phi$ 
    
    \caseText{From (3) and (5)}   
    
    \caseFact{6} $\Theta ; \Delta \vDash \Phi \wedge (I \geq J)$
    
    \caseText{By inversion on (4)}
    
    \caseFact{7} $\Psi ; \Theta ; \Delta \pvdash \texttt{eval}(\tau_1) \subtynf \texttt{eval}(\tau_2) : \star \gens \Phi$
    
    \caseText{By AS-ConstPot on (7)}
    
    \caseFact{8} $\Psi ; \Theta ; \Delta \pvdash [I]\texttt{eval}(\tau_1) \subtynf [J]\texttt{eval}(\tau_2) : \star \gens \Phi \wedge (I \geq J)$
    
    \caseText{Goal follows by AS-Normalize on (8)}
  }
  
  \jcase{17}{S-FamLam}{
   \jgivengoal{
    \caseFact{1} $\Psi ; \Theta ; \Delta \pvdash \lambda i : S. \tau_1 \subty \lambda i : S. \tau_2 : S \to K$ 
    
    \caseFact{2} $\Psi ; \Theta, i : S ; \Delta \pvdash \tau_1 \subty \tau_2 : \star$
   }{
     $\Psi ; \Theta ; \Delta \pvdash \lambda i : S. \tau_1 \subty \lambda i : S. \tau_2 : S \to K \gens \Phi$ and $\Theta ; \Delta \vDash \Phi$ 
   }
    \caseText{By IH on (2)}
    
    \caseFact{3} $\Psi ; \Theta, i : S ; \Delta \pvdash \tau_1 \subty \tau_2 : K \gens \Phi$
    
    \caseFact{4} $\Theta, i : S ; \Delta \vDash \Phi$
    
    \caseText{Equivalently to (4)}
    
    \caseFact{5} $Theta ; Delta \vDash \forall i : S. \Phi$
    
    \caseText{By inversion on (3)}
    
    \caseFact{6} $\Psi ; \Theta, i : S ; \Delta \pvdash \texttt{eval}(\tau_1) \subtynf \texttt{eval}(\tau_2) : K \gens \Phi$
    
    \caseText{By AS-FamLam on (6)}
    
    \caseFact{7} $\Psi ; \Theta ; \Delta \pvdash \lambda i : S . \texttt{eval}(\tau_1) \subtynf \lambda i : S.\texttt{eval}(\tau_2) : S \to K \gens \forall i : S. \Phi$
    
    \caseText{Goal follows by AS-Noramlize on (7)}
  }
  
  \jcase{18}{S-FamApp}{
   \jgivengoal{
    \caseFact{1} $\Psi ; \Theta ; \Delta \pvdash \tau_1 \; I \subty \tau_2 \; J : K$
    
    \caseFact{2} $\Psi ; \Theta ; \Delta \pvdash \tau_1 \subty \tau_2 : S \to K$
    
    \caseFact{3} $\Theta ; \Delta \vDash I = J$
   }{
    $\Psi  ;\Theta ; \Delta \pvdash \tau_1 \; I \subty \tau_2 \; J : K \gens \Phi$ and $\Theta ; \Delta \vDash \Phi$
   }
   
   \caseText{By IH on (2)}
   
   \caseFact{4} $\Psi ; \Theta ; \Delta \pvdash \tau_1 \subty \tau_2 : S \to K \gens \Phi$
   
   \caseFact{5} $\Theta ; \Delta \vDash \Phi$
   
   \caseText{Inverting (4)}
   
   \caseFact{6} $\Psi ; \Theta ; \Delta \pvdash \texttt{eval}(\tau_1) \subtynf \texttt{eval}(\tau_2) : S \to K \gens \Phi$
   
   \caseText{By \autoref{thm:eval-app-lemma} on (3) and (6)}
   
   \caseFact{7} $\Psi ; \Theta ; \Delta \pvdash \texttt{eval}(\tau_1 \; I) \subtynf \texttt{eval}(\tau_2 \; J) : K \gens \Phi'$
   
   \caseFact{8} $\Theta ; \Delta \vDash \Phi'$
   
   \caseText{Goal follows from AS-Normalize on (7)}
  }
  
  \jcase{19}{S-Fam-Beta1}{
   \jgivengoal{
     \caseFact{1} $\Psi ; \Theta ; \Delta \pvdash (\lambda i : S. \tau) \; J \subty \tau[J/i] : K$
   }{
    $\Psi ; \Theta ; \Delta \vdash (\lambda i : S. \tau) \; J \subty \tau[J/i] : K \gens \Phi$ and $\Theta ; \Delta \vDash \Phi$
   }
   
   \caseText{By the presupposition for (1)}
   
   \caseFact{2} $\Psi ; \Theta ; \Delta \pvdash \tau[J/i] : K$
   
   \caseText{By \autoref{thm:norm-thm} on (2)}
   
   \caseFact{3} $\Psi ; \Theta ; \Delta \pvdash \texttt{eval}(\tau[J/i]) : K$
   
   \caseFact{4} $\texttt{eval}(\tau[J/i]) \; \texttt{nf}$
   
   \caseText{By \autoref{thm:subtynf-refl} on (3) and (4)}
   
   \caseFact{5} $\Psi ; \Theta ; \Delta \pvdash \texttt{eval}(\tau[J/i]) \subtynf \texttt{eval}(\tau[J/i]) : K \gens \Phi$
   
   \caseFact{6} $\Theta ; \Delta \vDash \Phi$
   
   \caseText{By \autoref{thm:idx-subst-eval}}
   
   \caseFact{7} $\texttt{eval}(\tau[J/i]) = \texttt{eval}(\tau)[J/i]$
   
   \caseText{By definition of \texttt{eval}}
   
   \caseFact{8} $\texttt{eval}(\tau)[J/i] = \texttt{eval}((\lambda i : S. \tau) \; J)$
   
   \caseText{Combining (7) and (8) in (5)}
   
   \caseFact{9} $\Psi ; \Theta ; \Delta \pvdash \texttt{eval}((\lambda i : S. \tau) \; J) \subtynf \texttt{eval}(\tau[J/i]) : K \gens \Phi$
   
   \caseText{Goal follows by AS-Normalize on  (9)}
  }
  
  \jcase{20}{S-Fam-Beta2}{Identical to case 19}
  
}


\admitsweaken*
\begin{proof}
By a mutual induction on the premises of both cases. We will write this as a case analysis over the AT-* rules.
We will use \autoref{thm:ctx-sub-subset-2} liberally, sometimes silently.
\begin{itemize}
  \item[(AT-Var-1)] Suppose $\Psi ; \Theta ; \Delta; \Omega ; \Gamma \vdash x \infers \tau \gens \top, \Gamma \setminus \{x : \tau\}$ from $x : \tau \in \Gamma$.
  By \autoref{thm:ctx-sub-subset-2}, there is some $\tau'$ so that $\Psi ; \Theta ; \Delta \vdash \tau' \subty \tau : \star$, and $x : \tau' \in \Gamma'$.
  By AT-Var-1, $\Psi ; \Theta ; \Delta; \Omega' ; \Gamma' \vdash x \infers \tau' \gens \top, \Gamma' \setminus \{x : \tau'\}$.
  By \autoref{thm:subty-compl},  $\Psi ; \Theta ; \Delta \vdash \tau' \subty \tau : \star \gens \Phi$ with $\Theta ; \Delta \vDash \Phi$, and so by AT-Sub,
  $\Psi ; \Theta ; \Delta; \Omega' ; \Gamma' \vdash x \checks \tau \gens \top \wedge \Phi, \Gamma' \setminus \{x : \tau'\}$. Finally, $\Psi ; \Theta ; \Delta \vdash \Gamma' \setminus \{x : \tau'\} \wknto \Gamma' \setminus \Gamma$ and $\Psi ; \Theta ; \Delta \vdash \Gamma' \setminus \{x : \tau'\} \wknto \Gamma \setminus \{x : \tau\}$ since $x : \tau \in \Gamma$, and $\Psi ; \Theta ; \Delta \vdash \Gamma' \wknto \Gamma$, which proves (1). For (2), one use of AT-Anno gives $\Psi ; \Theta ; \Delta; \Omega' ; \Gamma' \vdash (x : \tau) \infers \tau \gens \top \wedge \Phi, \Gamma' \setminus \{x : \tau'\}$. But, $|(x : \tau)| = x$, and so we are done.
  
  \item[(AT-Var-2)] Suppose $\Psi ; \Theta ; \Delta ; \Omega ; \Gamma\vdash x \infers \tau \gens \top, \Gamma$ from $x : \tau \in \Omega$, with $\Psi ; \Theta ; \Delta \vdash \Omega' \wknto \Omega$ with $\Psi ; \Theta ; \Delta \vdash \Gamma' \wknto \Gamma$. By \autoref{thm:ctx-sub-subset-2}, there is some $\tau'$ so that $x : \tau' \in \Omega'$ and $\Phi ; \Theta ; \Delta \vdash \tau' \subty \tau : \star$. By \autoref{thm:subty-compl}, $\Phi ; \Theta ; \Delta \vdash \tau' \subty \tau : \star \gens \Phi$ with $\Theta ; \Delta \vDash \Phi$. 
  By AT-Var-2, $\Psi ; \Theta ; \Delta ; \Omega' ; \Gamma' \vdash x \infers \tau' \gens \top, \Gamma'$. By AT-Sub, 
  $\Psi ; \Theta ; \Delta ; \Omega' ; \Gamma' \vdash x \checks \tau \gens \top \wedge \Phi, \Gamma'$, which proves (2). For (1), we apply AT-Anno and note that $|(x : \tau)| = x$.
  
  \item[(AT-Unit)] Immediate.
  \item[(AT-Base)] Immediate.
  \item[(AT-Absurd)] Immediate.
  \item[(AT-Nil)] Immediate.
  
  \item[(AT-Cons)] Suppose $\Psi ; \Theta ; \Delta ; \Omega ; \Gamma\vdash e_1 :: e_2 \checks L^I \tau \gens (I \geq 1) \wedge \Phi_1 \wedge \Phi_2, \Gamma_2$
  from $\Psi ; \Theta ; \Delta ; \Omega ; \Gamma\vdash e_1 \checks \tau \gens \Phi_1, \Gamma_1$ and 
       $\Psi ; \Theta ; \Delta ; \Omega ; \Gamma_1\vdash e_2 \checks L^{I-1} \tau \gens \Phi_2, \Gamma_2$ with 
       $\Theta ; \Delta \vDash (I \geq 1) \wedge \Phi_1 \wedge \Phi_2$, 
       $\Psi ; \Theta ; \Delta \vDash \Gamma' \wknto \Gamma$, and 
       $\Psi ; \Theta ; \Delta \vdash \Omega' \wknto \Omega$.
       By IH, there are $e_1'$, $\Phi_1'$, $\Gamma_1'$ such that $|e_1'| = |e_1|$, 
       $\Theta ; \Delta \vDash \Phi_1'$, 
       $\Psi ; \Theta ; \Delta \vdash \Gamma_1' \wknto \Gamma' \setminus \Gamma$, 
       $\Psi ; \Theta ; \Delta \vdash \Gamma_1' \wknto \Gamma_1$, and
       $\Psi ; \Theta ; \Delta ; \Omega' ; \Gamma' \vdash e_1' \checks \tau \gens \Phi_1',\Gamma_1'$.
       By IH, there are $e_2'$, $\Phi_2'$, $\Gamma_2'$ such that $|e_2'| = |e_2|$,
       $\Theta ; \Delta \vDash \Phi_2'$,
       $\Psi ; \Theta ; \Delta \vdash \Gamma_2' \wknto \Gamma_1' \setminus \Gamma_1$,
       $\Psi ; \Theta ; \Delta \vdash \Gamma_2' \wknto \Gamma_2$, and
       $\Psi ; \Theta ; \Delta ; \Omega' ; \Gamma_1' \vdash e_2' \checks L^{I-1} \tau \gens \Phi_2',\Gamma_2'$.
       By AT-Cons,
       $\Psi ; \Theta ; \Delta ; \Omega' ; \Gamma' \vdash e_1' :: e_2' \checks L^I \tau \gens (I \geq 1) \wedge \Phi_1' \wedge \Phi_2',\Gamma_2'$.
       Since $\Theta ; \Delta \vDash I \geq 1$, we have that $\Theta ; \Delta \vDash (I \geq 1) \wedge \Phi_1' \wedge \Phi_2'$. Further, $|e_1' :: e_2'| = |e_1'| :: |e_2'| = |e_1| :: |e_2| = |e_1 :: e_2|$. Finally, $\Psi ; \Theta ; \Delta \vdash \Gamma_2' \wknto \Gamma_2$ by the second IH, and $\Psi ; \Theta ; \Delta \vdash \Gamma_2' \wknto \Gamma' \setminus \Gamma$ by $\Psi ; \Theta ; \Delta \vdash \Gamma_2' \wknto \Gamma_1' \setminus \Gamma_1$ and $\Psi ; \Theta ; \Delta \vdash \Gamma_1' \wknto \Gamma' \setminus \Gamma$ using \autoref{thm:ctx-sub-subset-2}, which completes (1). For (2), AT-Anno gives
       $\Psi ; \Theta ; \Delta ; \Omega' ; \Gamma' \vdash (e_1' :: e_2' : L^I) \infers L^I \tau \gens (I \geq 1) \wedge \Phi_1' \wedge \Phi_2',\Gamma_2'$, as required.
  
  \item[(AT-Match)] Suppose 
   $$\Psi ; \Theta ; \Delta ; \Omega ; \Gamma\vdash \texttt{match}(e,e_1,h.t.e_2) \checks \tau' \gens \Phi_1 \wedge (I = 0 \to \Phi_2) \wedge (I \geq 1 \to \Phi_3), \Gamma_2 \cap (\Gamma_3 \setminus \{h,t\}$$ from 
   $$\Psi ; \Theta ; \Delta ; \Omega ; \Gamma\vdash e \infers L^I \tau \gens \Phi_1, \Gamma_1$$,
   $$\Psi ; \Theta ; \Delta, I = 0 ; \Omega ; \Gamma_1\vdash e_1 \checks \tau' \gens \Phi_2,\Gamma_2$$
   $$\Psi ; \Theta ; \Delta, I \geq 1; \Omega ; \Gamma_1, h : \tau, t : L^{I-1} \tau \vdash e_2 \checks \tau' \gens \Phi_3,\Gamma_3$$ with 
   $$\Theta ; \Delta \vDash \Phi_1 \wedge (I = 0 \to \Phi_2) \wedge (I \geq 1 \to \Phi_3)$$
   $$\Psi ; \Theta ; \Delta \vdash \Omega' \wknto \Omega$$
   $$\Psi ; \Theta ; \Delta \vdash \Gamma' \wknto \Gamma$$
   By IH, there are $e'$, $\Phi_1'$, $\Gamma_1'$ with $|e'| = |e|$, 
   $\Theta ; \Delta \vDash \Phi_1'$,
   $\Psi ; \Theta ; \Delta \vdash \Gamma_1' \wknto \Gamma' \setminus \Gamma$,
   $\Psi ; \Theta ; \Delta \vdash \Gamma_1' \wknto \Gamma_1$, and
   $\Psi ; \Theta ; \Delta ; \Omega' ; \Gamma' \vdash e' \infers L^I \tau \gens \Phi_1',\Gamma_1'$.
   By IH, there are $e_1'$, $\Phi_2'$, $\Gamma_2'$ with $|e_1'| = |e_1|$,
   $\Theta ; \Delta, I = 0 \vDash \Phi_2'$,
   $\Psi ; \Theta ; \Delta, I = 0 \vdash \Gamma_2' \wknto \Gamma_1' \setminus \Gamma_1$,
   $\Psi ; \Theta ; \Delta, I = 0 \vdash \Gamma_2' \wknto \Gamma_2$, and
   $\Psi ; \Theta ; \Delta ; \Omega' ; \Gamma_1' \vdash e_1' \checks \tau' \gens \Phi_2',\Gamma_2'$.
   Since $\Psi ; \Theta ; \Delta \vdash \Gamma_1' \wknto \Gamma_1$, we have that
      $\Psi ; \Theta ; \Delta \vdash \Gamma_1', h : \tau, t : L^{I-1} \tau \wknto \Gamma_1, h : \tau, t : L^{I-1} \tau$,
   and so by IH, there are $e_2'$, $\Phi_3'$, $\Gamma_3'$ such that $|e_2'| = |e_2|$,
   $\Theta ; \Delta, I \geq 1 \vDash \Phi_3'$,
   $\Psi ; \Theta ; \Delta, I \geq 1 \vdash \Gamma_3' \wknto (\Gamma_1',h : \tau, t : L^{I-1} \tau) \setminus (\Gamma_1, h : \tau, t : L^{I-1} \tau)$,
   $\Psi ; \Theta ; \Delta, I \geq 1 \vdash \Gamma_3' \wknto \Gamma_3$, and
   $\Psi ; \Theta ; \Delta ; \Omega' ; \Gamma_1', h : \tau, t : L^{I-1} \tau \vdash e_2' \checks \tau' \gens \Phi_3',\Gamma_3'$.
   By AT-Match, $\Psi ; \Theta ; \Delta ; \Omega' ; \Gamma' \vdash \texttt{match}(e',e1',h.t.e_2') \checks \tau' \gens \Phi_1' \wedge (I = 0 \to \Phi_2') \wedge (I \geq 1 \to \Phi_3'),\Gamma_2' \cap (\Gamma_3' \setminus \{h,t\})$
   Firstly, we note that $|\texttt{match}(e',e1',h.t.e_2')| = \texttt{match}(|e'|,|e_1'|,h.t.|e_2'|) = \texttt{match}(|e|,|e_1|,h.t.|e_2|) = \texttt{match}(e,e_1,h.t.e_2)$. Then, since $\Theta ; \Delta \vDash I = 0 \to \Phi_2'$ and $\Theta ; \Delta \vDash I \geq 1 \to \Phi_3'$, we have $\Theta ; \Delta \vDash \Phi_1' \wedge (I = 0 \to \Phi_2') \wedge (I \geq 1 \to \Phi_3')$.
   Then, 
   $\Phi ; \Theta ; \Delta \vdash \Gamma_2' \cap (\Gamma_3' \setminus \{h,t\}) \wknto \Gamma' \setminus \Gamma$
   because of $\Psi ; \Theta ; \Delta \vdash \Gamma_1' \wknto \Gamma' \setminus \Gamma$,  $\Psi ; \Theta ; \Delta \vdash \Gamma_2' \wknto \Gamma_1' \setminus \Gamma_1$, and $\Psi ; \Theta ; \Delta, I \geq 1 \vdash \Gamma_3' \wknto \Gamma_1' \setminus \Gamma_1$, making heavy use of \autoref{thm:ctx-sub-subset-2}.
   Finally, we have   
   $\Phi ; \Theta ; \Delta \vdash \Gamma_2' \cap (\Gamma_3' \setminus \{h,t\}) \wknto \Gamma_2 \cap (\Gamma_3 \setminus \{h,t\}$
   since $\Psi ; \Theta ; \Delta \vdash \Gamma_2' \wknto \Gamma_2$ and $\Psi ; \Theta ; \Delta \vdash \Gamma_3' \wknto \Gamma_3$. We may strengthen away the assumptions in $\Delta$ because of the presuppositions of well-formedness all contexts involved. This completes (1). For (2), we apply AT-Anno to get
   $\Psi ; \Theta ; \Delta ; \Omega' ; \Gamma' \vdash (\texttt{match}(e',e1',h.t.e_2') : \tau') \infers \tau' \gens \Phi_1' \wedge (I = 0 \to \Phi_2') \wedge (I \geq 1 \to \Phi_3'),\Gamma_2' \cap (\Gamma_3' \setminus \{h,t\})$, and we are done.
   
   \item[(AT-ExistI)] Suppose $\Psi ; \Theta ; \Delta ; \Omega ; \Gamma\vdash \texttt{pack}[I](e) \checks \exists i:S.\tau \gens \Phi_1 \wedge \Phi_2, \Gamma''$ from
   $\Theta ; \Delta \vdash I : S \gens \Phi_1$ and
   $\Psi ; \Theta ; \Delta ; \Omega ; \Gamma\vdash e \checks \tau[I/i] \gens \Phi_2,\Gamma''$,
   with $\Theta ; \Delta \vDash \Phi_1 \wedge \Phi_2$,
   $\Psi ; \Theta ; \Delta \vdash \Gamma' \wknto \Gamma$, and
   $\Psi ; \Theta ; \Delta \vdash \Omega' \wknto \Omega$.
   By IH, there are $e'$, $\Phi_2'$, $\Gamma'''$ such that
   $|e'| = |e|$,
   $\Theta ; \Delta \vDash \Phi_2'$,
   $\Psi ; \Theta ; \Delta \vdash \Gamma''' \wknto \Gamma' \setminus \Gamma$, and
   $\Psi ; \Theta ; \Delta \vdash \Gamma''' \wknto \Gamma''$, and
   $\Psi ; \Theta ; \Delta ; \Omega' ; \Gamma'\vdash e' \checks \tau[I/i] \gens \Phi_2',\Gamma'''$.
   By AT-ExistI, $\Psi ; \Theta ; \Delta ; \Omega' ; \Gamma'\vdash \texttt{pack}[I](e') \checks \exists i : S. \tau[I/i] \gens \Phi_1 \wedge \Phi_2',\Gamma'''$.
   Since $|\texttt{pack}[I](e')| = \texttt{pack}[I](|e'|) = \texttt{pack}[I](|e|) = |\texttt{pack}[I](e)|$ and $\Theta ; \Delta \vDash \Phi_1 \wedge \Phi_2'$, this completes (1). For (2), one application of AT-Anno suffices.
   
   \item[(AT-ExistE)]
   Suppose $\Psi ; \Theta ; \Delta ; \Omega ; \Gamma\vdash \texttt{unpack } (i,x) = e_1 \texttt{ in } e_2 \checks \tau' \gens \Phi_1 \wedge (\forall i : S. \Phi_2) , \Gamma_2 \setminus \{x : \tau\}$
   from 
   $\Psi ; \Theta ; \Delta ; \Omega ; \Gamma\vdash e_1 \infers \exists i : S.\tau \gens \Phi_1, \Gamma_1$
   and 
   $\Psi ; \Theta, i : S ; \Delta ; \Omega ; \Gamma_1, x : \tau \vdash e_2 \checks \tau' \gens \Phi_2, \Gamma_2$
   with $\Theta ; \Delta \vDash \Phi_1 \wedge \Phi_2$,
   $\Psi ; \Theta ; \Delta \vdash \Gamma' \wknto \Gamma$, and
   $\Psi ; \Theta ; \Delta \vdash \Omega' \wknto \Omega$.
   By IH, there are $e_1'$, $\Phi_1'$, $\Gamma_1'$ such that 
   $|e_1'| = |e_1|$,
   $\Theta ; \Delta \vDash \Phi_1'$,
   $\Psi ; \Theta ; \Delta \vdash \Gamma_1' \wknto \Gamma' \setminus \Gamma$,
   $\Psi ; \Theta ; \Delta \vdash \Gamma_1' \wknto \Gamma_1$, and
   $\Psi ; \Theta ; \Delta ; \Omega' ; \Gamma'\vdash e_1' \infers \exists i : S.\tau \gens \Phi_1', \Gamma_1'$.
   Since $\Theta ; \Delta \vDash \Phi_1 \wedge (\forall i : S. \Phi_2)$, we have $\Theta, i : S ; \Delta \vDash \Phi_2$. We also have $\Psi ; \Theta ; \Delta \vdash \Gamma_1', x : \tau \wknto \Gamma_1, x : \tau$. From these two facts we have
   by IH that there are $e_2'$, $\Phi_2'$, $\Gamma_2'$ such that
   $|e_2'| = |e_2|$,
   $\Theta, i : S ; \Delta \vDash \Phi_2'$,
   $\Psi ; \Theta, i : S ; \Delta \vdash \Gamma_2' \wknto (\Gamma_1', x : \tau) \setminus (\Gamma_1, x : \tau)$,
   $\Psi ; \Theta, i : S ; \Delta \vdash \Gamma_2' \wknto \Gamma_2$, and that
   $\Psi ; \Theta, i : S ; \Delta ; \Omega' ; \Gamma_1', x : \tau \vdash e_2' \checks \tau' \gens \Phi_2',\Gamma_2'$.
   By AT-ExistE, 
   $\Psi ; \Theta ; \Delta ; \Omega' ; \Gamma' \vdash \texttt{unpack } (i,x) = e_1' \texttt{ in } e_2' \checks \tau' \gens \Phi_1' \wedge (\forall i : S. \Phi_2'), \Gamma_2'$.
   Since $\Theta ; \Delta \vDash \Phi_1'$ and $\Theta, i : S; \Delta \vDash \Phi_2'$, we have $\Theta ; \Delta \vDash \Phi_1' \wedge (\forall i : S. \Phi_2')$.
   Next, we note that $|\texttt{unpack } (i,x) = e_1' \texttt{ in } e_2'| = \texttt{unpack } (i,x) = |e_1'| \texttt{ in } |e_2'| = \texttt{unpack } (i,x) = |e_1| \texttt{ in } |e_2| = |\texttt{unpack } (i,x) = e_1 \texttt{ in } e_2|$. $\Psi ; \Theta, i : S ; \Delta \vdash \Gamma_2' \wknto \Gamma_2$ is immediate from the second IH, and the fact that $\Psi ; \Theta ; \Delta \vdash \Gamma_2' \wknto \Gamma' \setminus \Gamma$ follows from \autoref{thm:ctx-sub-subset-2}. This completes (1), and (2) follows immediately from AT-Anno.
   
   \item[(AT-Lam)] Suppose $\Psi ; \Theta ; \Delta ; \Omega ; \Gamma\vdash \lambda x.e \checks \tau_1 \loli \tau_2 \gens \Phi, \Gamma'' \setminus \{x : \tau_1\}$
  from $\Psi ; \Theta ; \Delta ; \Omega ; \Gamma, x : \tau_1 \vdash e \checks \tau_2, \gens \Phi, \Gamma''$, with $\Theta ; \Delta \vDash \Phi$, $\Psi ; \Theta ; \Delta \vdash \Gamma' \wknto \Gamma$, and $\Psi ; \Theta ; \Delta \vdash \Omega' \wknto \Omega$. Since $\Psi ; \Theta ; \Delta \vdash \Gamma',x:\tau_1 \wknto \Gamma, x: \tau_1$, by IH there are $\Phi'$, $e'$, $\Gamma'''$ so that
  $\Theta ; \Delta \vDash \Phi'$, $|e'| = |e|$, $\Psi ; \Theta ; \Delta \vdash \Gamma''' \wknto \Gamma \setminus \Gamma'$, and
  $\Psi ; \Theta ; \Delta \vdash \Gamma''' \wknto \Gamma''$, and $$\Psi ; \Theta ; \Delta ; \Omega' ; \Gamma',x : \tau_1 \vdash e' : \tau_2 \gens \Phi',\Gamma'''$$.
  By AT-Lam, $\Psi ; \Theta ; \Delta ; \Omega' ; \Gamma' \vdash \lambda x.e' : \tau_1 \loli \tau_2 \gens \Phi', \Gamma''' \setminus \{x : \tau_1\}$. Then, $|\lambda x.e'| = \lambda x. |e'| = \lambda x.|e| = |\lambda x.e|$. Finally, $\Psi ; \Theta ; \Delta \vdash \Gamma''' \setminus \{x : \tau_1\} \wknto \Gamma' \setminus \Gamma$
  and $\Psi ; \Theta ; \Delta \vdash \Gamma''' \setminus \{x : \tau_1\} \wknto \Gamma'' \setminus \{x : \tau_1\}$, which proves (1). For (2), one use of AT-Anno suffices.
  
  \item[(AT-App)] Suppose $\Psi ; \Theta ; \Delta ; \Omega ; \Gamma\vdash e_1 \, e_2 \infers  \tau_2 \gens \Phi_1 \wedge \Phi_2, \Gamma_2$ from
  $\Psi ; \Theta ; \Delta ; \Omega ; \Gamma\vdash e_1 \infers \tau_1 \loli \tau_2 \gens \Phi_1, \Gamma_1$ and
  $\Psi ; \Theta ; \Delta ; \Omega ; \Gamma_1\vdash e_2 \checks \tau_1 \gens \Phi_2, \Gamma_2$, with
  $\Theta ; \Delta \vDash \Phi_1 \wedge \Phi_2$,
  $\Psi ; \Theta ; \Delta \vdash \Omega' \wknto \Omega$, and
  $\Psi ; \Theta ; \Delta \vdash \Gamma' \wknto \Gamma$.
  By IH, there are $e_1'$, $\Phi_1'$, and $\Gamma_1'$ such that
  $|e_1'| = |e_1|$,
  $\Theta ; \Delta \vDash \Phi_1'$,
  $\Psi ; \Theta ; \Delta \vdash \Gamma_1' \wknto \Gamma' \setminus \Gamma$,
  $\Psi ; \Theta ; \Delta \vdash \Gamma_1' \wknto \Gamma_1$, and
  $\Psi ; \Theta ; \Delta ; \Omega' ; \Gamma' \vdash e_1' \infers \tau_1 \loli \tau_2 \gens \Phi_1',\Gamma_1'$.
  Since $\Psi ; \Theta ; \Delta \vdash \Gamma_1' \wknto \Gamma_1$,
  we have by IH that there are $e_2'$, $\Phi_2'$, $\Gamma_2'$ such that
  $|e_2'| = |e_2|$,
  $\Theta ; \Delta \vDash \Phi_2'$,
  $\Psi ; \Theta ; \Delta \vdash \Gamma_2' \wknto \Gamma_1' \setminus \Gamma_1$,
  $\Psi ; \Theta ; \Delta \vdash \Gamma_2' \wknto \Gamma_2$, and
  $\Psi ; \Theta ; \Delta ; \Omega' ; \Gamma_1' \vdash e_2' \checks \tau_1 \gens \Phi_2',\Gamma_2'$.
  By AT-App, we have
  $\Psi ; \Theta ; \Delta ; \Omega' ; \Gamma' \vdash e_1' \; e_2' \infers \tau_2 \gens \Phi_1' \wedge \Phi_2', \Gamma_2'$.
  This completes (2). For (1), we invoke \autoref{thm:subty-refl} to get that $\Psi ; \Theta ; \Delta \vdash \tau_2 \subty \tau_2 : \star \gens \Phi'$ with
  $\Theta ; \Delta \vDash \Phi'$. By AT-Sub, we have $\Psi ; \Theta ; \Delta ; \Omega' ; \Gamma' \vdash e_1' \; e_2' \infers \tau_2 \gens \Phi_1' \wedge \Phi_2' \wedge \Phi', \Gamma_2'$, completing (1).
  
  
  \item[(AT-TensorI)] Suppose $\Psi ; \Theta ; \Delta ; \Omega ; \Gamma\vdash \angles{e_1,e_2} \checks \tau_1 \otimes \tau_2 \gens \Phi_1 \wedge \Phi_2,\Gamma_2$
  from $\Psi ; \Theta ; \Delta ; \Omega ; \Gamma\vdash e_1 \checks \tau_1 \gens \Phi_1, \Gamma_1$ and $\Psi ; \Theta ; \Delta ; \Omega ; \Gamma_1\vdash e_2 \checks \tau_2 \gens \Phi_2, \Gamma_2$, with $\Theta ; \Delta \vDash \Phi_1 \wedge \Phi_2$, $\Psi ; \Theta ; \Delta \vdash \Gamma' \wknto \Gamma$, and $\Psi ; \Theta ; \Delta \vdash \Omega' \wknto \Omega$. By IH, there are $\Phi_1'$, $e_1'$, and $\Gamma_1'$ such that
  $\Theta ; \Delta \vDash \Phi_1'$,
  $|e_1'| = |e_1|$,
  $\Psi ; \Theta ; \Delta \vdash \Gamma_1' \wknto \Gamma' \setminus \Gamma$,
  $\Psi ; \Theta ; \Delta \vdash \Gamma_1' \wknto \Gamma_1$,
  $$\Psi ; \Theta ; \Delta ; \Omega' ; \Gamma' \vdash e_1' : \tau_1 \gens \Phi_1', \Gamma_1'$$.
  Since $\Psi ; \Theta ; \Delta \vdash \Gamma_1' \wknto \Gamma_1$, we also have by IH that there are $\Phi_2'$, $e_2'$, $\Gamma_2'$ such that
  $\Theta ; \Delta \vDash \Phi_2'$,
  $|e_2'| = |e_2|$,
  $\Psi ; \Theta ; \Delta \vdash \Gamma_2' \wknto \Gamma_1' \setminus \Gamma_1$, and
  $\Psi ; \Theta ; \Delta \vdash \Gamma_2' \wknto \Gamma_2$
  such that
  $$\Psi ; \Theta ; \Delta ; \Omega' ; \Gamma_1' \vdash e_2' : \tau_2 \gens \Phi_2',\Gamma_2'$$.
  By AT-TensorI, $\Psi ; \Theta ; \Delta ; \Omega' ; \Gamma' \vdash \angles{e_1',e_2'} : \tau_1 \otimes \tau_2 \gens (\Phi_1' \wedge \Phi_2'), \Gamma_2'$.
  We have that $\Theta ; \Delta \vDash \Phi_1' \wedge \Phi_2'$ and $|\angles{e_1',e_2'}| = \angles{|e_1'|,|e_2'|} = \angles{|e_1|,|e_2|} = |\angles{e_1,e_2}|$.
  Finally, $\Psi ; \Theta ; \Delta \vdash \Gamma_2' \wknto \Gamma' \setminus \Gamma$ since $\Psi ; \Theta ; \Delta \vdash \Gamma_2' \wknto \Gamma_1' \setminus \Gamma_1$ and $\Psi ; \Theta ; \Delta \vdash \Gamma_1' \wknto \Gamma' \setminus \Gamma$ by \autoref{thm:ctx-sub-subset-2}, completing (1). For (2), a single use of
  AT-Anno gives  $\Psi ; \Theta ; \Delta ; \Omega' ; \Gamma' \vdash (\angles{e_1',e_2'} : \tau_1 \otimes \tau_2) : \tau_1 \otimes \tau_2 \infers (\Phi_1' \wedge \Phi_2'), \Gamma_2'$, as required.
  
  \item[(AT-TensorE)]
  Suppose $\Psi ; \Theta ; \Delta ; \Omega ; \Gamma\vdash \texttt{let } \angles{x,y} = e_1 \texttt{ in } e_2 \checks \tau' \gens \Phi_1 \wedge \Phi_2, \Gamma_2 \setminus \{x,y\}$ from
  $\Psi ; \Theta ; \Delta ; \Omega ; \Gamma\vdash e_1 \infers \tau_1 \otimes \tau_2 \gens \Phi_1, \Gamma_1$ and
  $\Psi ; \Theta ; \Delta ; \Omega ; \Gamma_1,x : \tau_1, y : \tau_2\vdash e_2 \checks \tau' \gens \Phi_2,\Gamma_2$, with
  $\Theta ; \Delta \vDash \Phi_1 \wedge \Phi_2$,
  $\Psi ; \Theta ; \Delta \vdash \Omega' \wknto \Omega$,
  $\Psi ; \Theta ; \Delta \vdash \Gamma' \wknto \Gamma$.
  By IH, there are $e_1'$, $\Phi_1'$, $\Gamma_1'$ such that
  $|e_1'| = |e_1|$,
  $\Theta ; \Delta \vDash \Phi_1'$,
  $\Psi ; \Theta ; \Delta \vdash \Gamma_1' \wknto \Gamma' \setminus \Gamma$,
  $\Psi ; \Theta ; \Delta \vdash \Gamma_1' \wknto \Gamma_1$, and 
  $\Psi ; \Theta ; \Delta ; \Omega' ; \Gamma' \vdash e_1' \infers \tau_1 \otimes \tau_2 \gens \Phi_1', \Gamma_1'$.
  Then, since $\Psi ; \Theta ; \Delta \vdash \Gamma_1', x : \tau_1, y : \tau_2 \wknto \Gamma_1, x : \tau_1, y : \tau_2$,
  we have by IH that there are $e_2'$, $\Phi_2'$, $\Gamma_2'$ such that
  $|e_2'| = |e_2|$,
  $\Theta ; \Delta \vDash \Phi_2'$,
  $\Psi ; \Theta ; \Delta \vdash \Gamma_2' \wknto (\Gamma_1', x : \tau_1, y : \tau_2) \setminus (\Gamma_1, x : \tau_1, y : \tau_2)$,
  $\Psi ; \Theta ; \Delta \vdash \Gamma_2' \wknto \Gamma_2$, and
  $\Psi ; \Theta ; \Delta ; \Omega' ; \Gamma_1', x : \tau_1, y : \tau_2 \vdash e_2' \checks \tau_2 \gens \Phi_2',\Gamma_2'$.
  By AT-TensorE,
  $\Psi ; \Theta ; \Delta ; \Omega' ; \Gamma'\vdash \texttt{let } \angles{x,y} = e_1' \texttt{ in } e_2' \checks \tau' \gens \Phi_1' \wedge \Phi_2', \Gamma_2' \setminus \{x,y\}$.
  Of course, $|\texttt{let } \angles{x,y} = e_1' \texttt{ in } e_2'| = |\texttt{let } \angles{x,y} = e_1 \texttt{ in } e_2|$ and $\Theta ; \Delta \vDash \Phi_1' \wedge \Phi_2'$ as usual, and $\Psi ; \Theta ; \Delta \vdash \Gamma_2' \wknto \Gamma_2$ is immediate by IH. The fact that $\Psi ; \Theta ; \Delta \vdash \Gamma_2' \wknto \Gamma' \setminus \Gamma$ follows from considering the weakening judgments from both IHs, and liberally applying \autoref{thm:ctx-sub-subset-2}. This completes (1). For (2), we simply apply AT-Anno, and are done.
  
 
 \item[(AT-WithI)] Suppose $\Psi ; \Theta ; \Delta ; \Omega ; \Gamma \vdash (e_1,e_2) \checks \tau_1 \amp \tau_2 \gens \Phi_1 \wedge \Phi_2, \Gamma_1 \cap \Gamma_2$
 from $\Psi ; \Theta ; \Delta ; \Omega ; \Gamma \vdash e_1 \checks \tau_1 \gens \Phi_1, \Gamma_1$ and 
 $\Psi ; \Theta ; \Delta ; \Omega ; \Gamma \vdash e_2 \checks \tau_2 \gens \Phi_2, \Gamma_2$ with 
 $\Theta ; \Delta \vDash \Phi_1 \wedge \Phi_2$,
 $\Psi ; \Theta ; \Delta \vdash \Omega' \wknto \Omega$, and
 $\Psi ; \Theta ; \Delta \vdash \Gamma' \wknto \Gamma$.
 By IH, there are $e_1'$, $\Phi_1'$, $\Gamma_1'$ such that
 $|e_1'| = |e_1|$,
 $\Theta ; \Delta \vDash \Phi_1'$,
 $\Psi ; \Theta ; \Delta \vdash \Gamma_1' \wknto \Gamma' \setminus \Gamma$,
 $\Psi ; \Theta ; \Delta \vdash \Gamma_1' \wknto \Gamma_1$, and
 $\Psi ; \Theta ; \Delta ; \Omega' ; \Gamma' \vdash e_1' \checks \tau_1 \gens \Phi_1', \Gamma_1'$.
 Again by IH, there are $e_2'$, $\Phi_2'$, $\Gamma_2'$ such that
 $|e_2'| = |e_2|$,
 $\Theta ; \Delta \vDash \Phi_2'$,
 $\Psi ; \Theta ; \Delta \vdash \Gamma_2' \wknto \Gamma' \setminus \Gamma$,
 $\Psi ; \Theta ; \Delta \vdash \Gamma_2' \wknto \Gamma_2$, and
 $\Psi ; \Theta ; \Delta ; \Omega' ; \Gamma' \vdash e_2' \checks \tau_2 \gens \Phi_2', \Gamma_2'$.
 Then, by AT-WithI, 
 $\Psi ; \Theta ; \Delta ; \Omega' ; \Gamma' \vdash (e_1',e_2') \checks \tau_1 \amp \tau_2 \gens \Phi_1' \wedge \Phi_2', \Gamma_1' \cap \Gamma_2'$.
 Then, by \autoref{thm:ctx-sub-subset-2}, we have that $\Psi ; \Theta ; \Delta \vdash \Gamma_1' \cap \Gamma_2' \wknto \Gamma_1 \cap \Gamma_2$,
 and that $\Psi ; \Theta ; \Delta \vdash \Gamma_1' \cap \Gamma_2' \wknto \Gamma' \setminus \Gamma$. This completes (1), and (2) follows by AT-Anno.
  
  
  \item[(AT-Fst)] Suppose $\Psi ; \Theta ; \Delta ; \Omega ; \Gamma \vdash \texttt{fst}(e) \infers \tau_1 \gens \Phi,\Gamma''$ by way of
  $\Psi ; \Theta ; \Delta ; \Omega ; \Gamma \vdash e \infers \tau_1 \amp \tau_2 \gens \Phi,\Gamma''$
  with $\Theta ; \Delta \vDash \Phi$,
  $\Psi ; \Theta ; \Delta \vdash \Gamma' \wknto \Gamma$, and
  $\Psi ; \Theta ; \Delta \vdash \Omega' \wknto \Omega$.
  By IH, we have $e'$, $\Phi'$, $\Gamma'''$ such that
  $|e'| = |e|$,
  $\Theta ; \Delta \vDash \Phi'$,
  $\Psi ; \Theta ; \Delta \vdash \Gamma''' \wknto \Gamma' \setminus \Gamma$,
  $\Psi ; \Theta ; \Delta \vdash \Gamma''' \wknto \Gamma''$, and
  $\Psi ; \Theta ; \Delta ; \Omega' ; \Gamma' \vdash e' \infers \tau_1 \amp \tau_2 \gens \Phi',\Gamma'''$.
  By AT-Fst, $\Psi ; \Theta ; \Delta ; \Omega' ; \Gamma' \vdash \texttt{fst}(e') \infers \tau_1 \gens \Phi',\Gamma'''$,
  which completes (2), since $|\texttt{fst}(e')| = \texttt{fst}(|e'|) = \texttt{fst}(|e|) = |\texttt{fst}(e)|$. For (1),
  by \autoref{thm:subty-refl}, there is $\Phi''$ such that $\Psi ; \Theta ; \Delta \vdash \tau_1 \subty \tau_1 : \star \gens \Phi''$ with $\Theta ; \Delta \vDash \Phi''$. By AT-Sub, $\Psi ; \Theta ; \Delta ; \Omega' ; \Gamma' \vdash \texttt{fst}(e') \checks \tau_1 \gens \Phi',\Gamma'''$, completing (1).

  \item[(AT-Snd)] Identical to AT-Fst.
  
  \item[(AT-Inl)] Suppose $\Psi ; \Theta ; \Delta ; \Omega ; \Gamma \vdash \texttt{inl}(e) \checks \tau_1 \oplus \tau_2 \gens \Phi,\Gamma''$
  from $\Psi ; \Theta ; \Delta ; \Omega ; \Gamma \vdash e \checks \tau_1 \gens \Phi,\Gamma''$, and
  $\Theta ; \Delta \vDash \Phi$,
  $\Psi ; \Theta ; \Delta \vdash \Gamma' \wknto \Gamma$, and
  $\Psi ; \Theta ; \Delta \vdash \Omega' \wknto \Omega$.
  By IH, there are $e'$, $\Phi'$, $\Gamma'''$ such that
  $|e'| = |e|$,
  $\Theta ; \Delta \vDash \Phi'$,
  $\Psi ; \Theta ; \Delta \vdash \Gamma''' \wknto \Gamma''$,
  $\Psi ; \Theta ; \Delta \vdash \Gamma''' \wknto \Gamma' \setminus \Gamma$, and
  $\Psi ; \Theta ; \Delta ; \Omega' ; \Gamma' \vdash e' \checks \tau_1 \gens \Phi',\Gamma'''$.
  By AT-Inl, we have
  $\Psi ; \Theta ; \Delta ; \Omega' ; \Gamma' \vdash \texttt{inl}(e') \checks \tau_1 \oplus \tau_2 \gens \Phi',\Phi'''$,
  which completes (1), and (2) is done by AT-Anno.
  
  \item[(AT-Inr)] Identical to AT-Inl.
  
  \item[(AT-Case)] Suppose $\Psi ; \Theta ; \Delta ; \Omega ; \Gamma \vdash \texttt{case}(e,x.e_1,y.e_2) \checks \tau \gens \Phi_1 \wedge \Phi_2 \wedge \Phi_3, (\Gamma_2 \setminus \{x : \tau_1\}) \cap (\Gamma_3 \setminus \{y : \tau_2\})$
  from $\Psi ; \Theta ; \Delta ; \Omega ; \Gamma \vdash e \infers \tau_1 \oplus \tau_2 \gens \Phi_1, \Gamma_1$,
  $\Psi ; \Theta ; \Delta ; \Omega ; \Gamma_1, x: \tau_1 \vdash e_1 \checks \tau \gens \Phi_2,\Gamma_2$, and
  $\Psi ; \Theta ; \Delta ; \Omega ; \Gamma_1, y: \tau_2 \vdash e_2 \checks \tau \gens \Phi_3,\Gamma_3$, and also that
  $\Theta ; \Delta \vDash \Phi_1 \wedge \Phi_2 \wedge \Phi_3$,
  $\Psi ; \Theta ; \Delta \vdash \Gamma' \wknto \Gamma$, and
  $\Psi ; \Theta ; \Delta \vdash \Omega' \wknto \Omega$.
  By IH, we have $e'$, $\Phi_1'$, $\Gamma_1'$ such that
  $|e'| = |e|$,
  $\Theta ; \Delta \vDash \Phi_1'$,
  $\Psi ; \Theta ; \Delta \vdash \Gamma_1' \wknto \Gamma_1$,
  $\Psi ; \Theta ; \Delta \vdash \Gamma_1' \wknto \Gamma' \setminus \Gamma$, and
  $\Psi ; \Theta ; \Delta ; \Omega' ; \Gamma' \vdash e' \infers \tau_1 \oplus \tau_2 \gens \Phi_1',\Gamma_1'$.
  Then, since $\Psi ; \Theta ; \Delta \vdash \Gamma_1', x : \tau_1 \wknto \Gamma_1, x : \tau_1$,
  we have by IH $e_1'$. $\Phi_2'$. $\Gamma_2'$ such that
  $|e_1'| = |e_1|$,
  $\Theta  ; \Delta \vDash \Phi_2'$,
  $\Psi ; \Theta ; \Delta \vdash \Gamma_2' \wknto \Gamma_2$,
  $\Psi ; \Theta ; \Delta \vdash \Gamma_2' \wknto \Gamma_1' \setminus \Gamma_1$, and
  $\Psi ; \Theta ; \Delta ; \Omega' ; \Gamma', x : \tau_1 \vdash e_1' \checks \tau \gens \Phi_2',\Gamma_2'$.
  Similarly, since $\Psi ; \Theta ; \Delta \vdash \Gamma_1', y : \tau_2 \wknto \Gamma_1, y : \tau_2$,
  we have by IH $e_2'$. $\Phi_3'$. $\Gamma_3'$ such that
  $|e_2'| = |e_2|$,
  $\Theta  ; \Delta \vDash \Phi_3'$,
  $\Psi ; \Theta ; \Delta \vdash \Gamma_3' \wknto \Gamma_3$,
  $\Psi ; \Theta ; \Delta \vdash \Gamma_3' \wknto \Gamma_1' \setminus \Gamma_1$, and
  $\Psi ; \Theta ; \Delta ; \Omega' ; \Gamma', x : \tau_2 \vdash e_2' \checks \tau \gens \Phi_3',\Gamma_3'$.
  Then, by AT-Case, we have
  $\Psi ; \Theta ; \Delta ; \Omega' ; \Gamma' \vdash \texttt{case}(e',x.e_1',y.e_2') \checks \tau \gens \Phi_1' \wedge \Phi_2' \wedge \Phi_3', (\Gamma_2' \setminus \{x : \tau_1\}) \cap (\Gamma_3' \setminus \{y : \tau_2\})$.
  Of course, $|\texttt{case}(e',x.e_1',y.e_2')| = |\texttt{case}(e,x.e_1,y.e_2)|$
  since $|e'| = |e|$, $|e_1'| = |e_1|$, and $|e_2'| = |e_2|$.
  Similarly,
  $\Theta ; \Delta \vDash \Phi_1' \wedge \Phi_2' \wedge \Phi_3'$.
  It remains to show that 
  $\Psi ; \Theta ; \Delta \vdash (\Gamma_2' \setminus \{x : \tau_1\}) \cap (\Gamma_3' \setminus \{y : \tau_2\}) \wknto (\Gamma_2 \setminus \{x : \tau_1\}) \cap (\Gamma_3 \setminus \{y : \tau_2\})$
  and
  $\Psi ; \Theta ; \Delta \vdash (\Gamma_2' \setminus \{x : \tau_1\}) \cap (\Gamma_3' \setminus \{y : \tau_2\}) \wknto \Gamma' \setminus \Gamma$,
  but both follow from the six weakening judgments and a few applications of \autoref{thm:ctx-sub-subset-2}.
  This completes (1), and (2) follows immediately by AT-Anno.
  
  \item[(AT-Sub)] Suppose 
    $\Psi ; \Theta ; \Delta ; \Omega ; \Gamma \vdash e \checks \tau \gens \Phi_1 \wedge \Phi_2,\Gamma''$ from
   $\Psi ; \Theta ; \Delta ; \Omega ; \Gamma \vdash e \infers \tau' \gens \Phi_1,\Gamma''$ and 
   $\Psi;\Theta;\Delta \vdash \tau' \subty \tau : \star \gens \Phi_2$, and
   $\Theta ; \Delta \vDash \Phi_1 \wedge \Phi_2$,
   $\Psi ; \Theta ; \Delta \vdash \Gamma' \wknto \Gamma$, and
   $\Psi ; \Theta ; \Delta \vdash \Omega' \wknto \Omega$.
   By IH, there are $e'$,$\Phi_1'$, and $\Gamma'''$ such that
   $|e| = |e'|$,
   $\Theta ; \Delta \vDash \Phi_1'$,
   $\Psi ; \Theta ; \Delta \vDash \Gamma''' \wknto \Gamma' \setminus \Gamma$,
   $\Psi ; \Theta ; \Delta \vDash \Gamma''' \wknto \Gamma''$,
   and $\Psi ; \Theta ; \Delta ; \Omega' \Gamma' \vdash e' \infers \tau' \gens \Phi_1',\Gamma'''$.
   by AT-Sub, $\Psi ; \Theta ; \Delta ; \Omega' ; \Gamma' \vdash e' \checks \tau \gens \Phi_1' \wedge \Phi_2,\Gamma'''$, which proves (1).
   For (2), we again use AT-Anno.
   
  \item[(AT-ExpI)] Suppose $\Psi ; \Theta ; \Delta ; \Omega ; \Gamma \vdash !e \checks !\tau \gens \Phi, \Gamma$ from
  $\Psi ; \Theta ; \Delta ; \Omega ; \cdot \vdash e \checks \tau \gens \Phi, \Gamma''$, and
  $\Theta ; \Delta \vDash \Phi$,
  $\Psi ; \Theta ; \Delta \vdash \Gamma' \wknto \Gamma$, and
  $\Psi ; \Theta ; \Delta \vdash \Omega' \wknto \Omega$.
  By IH (not weakening the empty affine environment) there are $e'$, $\Phi'$, $\Gamma'''$ such that 
  $|e'| = |e|$,
  $\Theta ; \Delta \vDash \Phi'$,
  and  $\Psi ; \Theta ; \Delta ; \Omega' ; \cdot \vdash e' \checks \tau \gens \Phi',\Gamma'''$.
  By AT-ExpI,
  $\Psi ; \Theta ; \Delta ; \Omega' ; \Gamma' \vdash !e' \checks \tau \gens \Phi',\Gamma'$.
  This completes (1), since $|!e'| = |!e|$, $\Theta ; \Delta \vDash \Phi'$,
  $\Psi ; \Theta ; \Delta \vdash \Gamma' \wknto \Gamma'$ and $\Psi ; \Theta ; \Delta \vdash \Gamma' \wknto \Gamma' \setminus \Gamma$
  as consequences of \autoref{thm:ctx-sub-subset-2}.
  
  \item[(AT-ExpE)] Suppose $\Psi ; \Theta ; \Delta ; \Omega ; \Gamma \vdash \texttt{let } !x = e_1 \texttt{ in } e_2 \checks \tau' \gens \Phi_1 \wedge \Phi_2, \Gamma_2$
  from $\Psi ; \Theta ; \Delta ; \Omega ; \Gamma \vdash e_1 \infers !\tau \gens \Phi_1,\Gamma_1$
  and $\Psi ; \Theta ; \Delta ; \Omega, x : \tau ; \Gamma_1 \vdash e_2 \checks \tau' \gens \Phi_2,\Gamma_2$, with
  $\Theta ; \Delta \vDash \Phi_1 \wedge \Phi_2$,
  $\Psi ; \Theta ; \Delta \vdash \Gamma' \wknto \Gamma$, and
  $\Psi ; \Theta ; \Delta \vdash \Omega' \wknto \Omega$.
  By IH, there are $e_1'$, $\Phi_1'$, $\Gamma_1'$ such that
  $|e_1'| = |e_1|$,
  $\Theta ; \Delta \vDash \Phi_1'$,
  $\Psi ; \Theta ; \Delta \vdash \Gamma_1' \wknto \Gamma' \setminus \Gamma'$,
  $\Psi ; \Theta ; \Delta \vdash \Gamma_1' \wknto \Gamma_1$, and  
  $\Psi ; \Theta ; \Delta ; \Omega' ; \Gamma' \vdash e_1' \infers !\tau \gens \Phi_1',\Gamma_1'$.
  Since $\Psi ; \Theta ; \Delta \vdash \Omega', x : \tau \wknto \Omega, x : \tau$, we have
  by IH we have $e_2'$, $\Phi_2'$, $\Gamma_2'$ such that
  $|e_2'| = |e_2|$,
  $\Theta ; \Delta \vDash \Phi_2'$,
  $\Psi ; \Theta ; \Delta \vdash \Gamma_2' \wknto \Gamma_1' \setminus \Gamma_1$,
  $\Psi ; \Theta ; \Delta \vdash \Gamma_2' \wknto \Gamma_2'$, and
  $\Psi ; \Theta ; \Delta ; \Omega', x : \tau ; \Gamma_1' \vdash e_2' \checks \tau' \gens \Phi_2',\Gamma_2'$.
  By AT-ExpE,
  $\Psi ; \Theta ; \Delta ; \Omega' ; \Gamma' \vdash \texttt{let } !x = e_1' \texttt{ in } e_2' \checks \tau' \gens \Phi_1' \wedge \Phi_2', \Gamma_2'$.
  Applying \autoref{thm:ctx-sub-subset-2} to the inductive hypotheses gives us that
  $\Psi ; \Theta ; \Delta \vdash \Gamma_2' \wknto \Gamma' \setminus \Gamma$, completing (1).
  For (2), one application of AT-Anno suffices.
  
  \item[(AT-TAbs)] Suppose $\Psi ; \Theta ; \Delta ; \Omega ; \Gamma \vdash \Lambda \alpha. e \checks \forall \alpha : K.\tau \gens \Phi,\Gamma''$ from
  $\Psi, \alpha : K ; \Theta ; \Delta ; \Omega ; \Gamma \vdash e \checks \tau \gens \Phi, \Gamma''$, and
  $\Theta ; \Delta \vDash \Phi$,
  $\Psi ; \Theta ; \Delta \vdash \Gamma' \wknto \Gamma$, and
  $\Psi ; \Theta ; \Delta \vdash \Omega' \wknto \Omega$.
  By IH, we have $e'$, $\Phi'$, $\Gamma'''$ such that
  $|e'| = |e|$,
  $\Theta ; \Delta \vdash \Phi'$,
  $\Psi, \alpha : K ; \Theta ; \Delta \vdash \Gamma''' \wknto \Gamma' \setminus \Gamma$
  $\Psi, \alpha : K ; \Theta ; \Delta \vdash \Gamma''' \wknto \Gamma''$, and
  $\Psi, \alpha : K ; \Theta ; \Delta ; \Omega' ; \Gamma' \vdash e' \checks \tau \gens \Phi',\Gamma'''$.
  By AT-TAbs, $\Psi ; \Theta ; \Delta ; \Omega' ; \Gamma' \vdash e' \checks \forall \alpha : K. \tau \gens \Phi',\Gamma'''$.
  By \autoref{thm:ctx-sub-streng}, we have that $\Psi ; \Theta ; \Delta \vdash \Gamma''' \wknto \Gamma' \setminus \Gamma$
  and $\Psi  ; \Theta ; \Delta \vdash \Gamma''' \wknto \Gamma''$, which completes (1).
  For (2), one use of AT-Anno suffices.
  
  \item[(AT-TApp)] Suppose $\Psi ; \Theta ; \Delta ; \Omega ; \Gamma \vdash e [\tau'] \infers \tau[\tau'/\alpha] \gens \Phi_1 \wedge \Phi_2, \Gamma''$ from
  $\Psi ; \Theta ; \Delta ; \Omega ; \Gamma \vdash e \infers \forall \alpha : K.\tau \gens \Phi_1, \Gamma''$ and
  $\Psi ; \Theta ; \Delta \vdash \tau' : K \gens \Phi_2$, with
  $\Theta ; \Delta \vDash \Phi_1 \wedge \Phi_2$,
  $\Psi ; \Theta ; \Delta \vdash \Gamma' \wknto \Gamma$, and
  $\Psi ; \Theta ; \Delta \vdash \Omega' \wknto \Omega$.
  By IH, there are $e'$, $\Phi_1'$, $\Gamma'''$ such that 
  $|e'| = |e'|$,
  $\Theta ; \Delta \vDash \Phi_1'$,
  $\Psi ; \Theta ; \Delta \vdash \Gamma''' \wknto \Gamma' \setminus \Gamma$,
  $\Psi ; \Theta ; \Delta \vdash \Gamma''' \wknto \Gamma''$, and
  $\Psi ; \Theta ; \Delta ; \Omega' ; \Gamma' \vdash e' \infers \forall \alpha : K.\tau \gens \Phi_1', \Gamma'''$.
  By AT-TApp,
  $\Psi ; \Theta ; \Delta ; \Omega ; \Gamma \vdash e' [\tau'] \infers \tau[\tau'/\alpha] \gens \Phi_1' \wedge \Phi_2, \Gamma'''$.
  Since $|e' [\tau']| = |e'| [\tau'] = |e [\tau']|$ and $\Theta ; \Delta \vDash \Phi_1' \wedge \Phi_2$, we are done with (1).
  For (2), a single use of AT-Anno completes the proof.
  
  \item[(AT-IAbs)] Suppose $\Psi ; \Theta ; \Delta ; \Omega ; \Gamma \vdash \Lambda i. e \checks \forall i : S. \tau \gens \forall i : S. \Phi, \Gamma''$
  from $\Psi ; \Theta, i : S ; \Delta ; \Omega ; \Gamma \vdash e \checks \tau \gens \Phi, \Gamma''$, with
  $\Theta ; \Delta \vDash \Phi$,
  $\Psi ; \Theta ; \Delta \vdash \Gamma' \wknto \Gamma$, and
  $\Psi ; \Theta ; \Delta \vdash \Omega' \wknto \Omega$.
  By \autoref{thm:ctx-sub-wkn},
  $\Psi ; \Theta, i : S ; \Delta \vdash \Gamma' \wknto \Gamma$ and
  $\Psi ; \Theta, i : S ; \Delta \vdash \Omega' \wknto \Omega$.
  By IH, there are $e'$, $\Phi'$, $\Gamma'''$ such that
  $|e'| = |e|$,
  $\Theta, i : S; \Delta \vDash \Phi'$,
  $\Psi ; \Theta, i : S ; \Delta \vdash \Gamma'' \wknto \Gamma' \setminus \Gamma$,
  $\Psi ; \Theta, i : S ; \Delta \vdash \Gamma'' \wknto \Gamma''$, and
  $\Psi ; \Theta, i : S ; \Delta ; \Omega' ; \Gamma' \vdash e' : \tau \gens \Phi', \Gamma'''$.
  By AT-IAbs,
  $\Psi ; \Theta ; \Delta ; \Omega' ; \Gamma' \vdash \Lambda i. e' : \tau \gens \forall i : S. \Phi', \Gamma'''$.
  By \autoref{thm:ctx-sub-streng},
  $\Psi ; \Theta ; \Delta \vdash \Gamma'' \wknto \Gamma' \setminus \Gamma$ and
  $\Psi ; \Theta ; \Delta \vdash \Gamma'' \wknto \Gamma''$.
  Finally, the fact that $\Theta ; \Delta \vDash \forall i : S. \Phi'$
  completes the proof of (1).
  For (2), AT-Anno suffices.
  
  \item[(AT-IApp)] Suppose $\Psi ; \Theta ; \Delta ; \Omega ; \Gamma \vdash e [I] \infers \tau[I/i] \gens \Phi_1 \wedge \Phi_2,\Gamma''$
  from $\Psi ; \Theta ; \Delta ; \Omega ; \Gamma \vdash e \infers \forall i : S.\tau \gens \Phi_1,\Gamma''$ and
  $\Theta ; \Delta \vdash I : S \gens \Phi_2$, with
  $\Theta ; \Delta \vDash \Phi_1 \wedge \Phi_2$,
  $\Psi ; \Theta ; \Delta \vdash \Gamma' \wknto \Gamma$, and
  $\Psi ; \Theta ; \Delta \vdash \Omega' \wknto \Omega$.
  By IH, there are $e'$, $\Phi_1'$, $\Gamma'''$ such that
  $|e| = |e'|$,
  $\Theta ; \Delta \vDash \Phi_1'$,
  $\Psi ; \Theta ; \Delta \vdash \Gamma''' \wknto \Gamma' \setminus \Gamma$,
  $\Psi ; \Theta ; \Delta \vdash \Gamma''' \wknto \Gamma''$, and
  $\Psi ; \Theta ; \Delta ; \Omega' ; \Gamma' \vdash e \infers \forall i : S. \tau \gens \Phi_1',\Gamma'''$.
  By AT-IApp,
  $\Psi ; \Theta ; \Delta ; \Omega' ; \Gamma' \vdash e [I] \infers \tau[I/i] \gens \Phi_1' \wedge \Phi_2,\Gamma'''$.
  Since $\Theta ; \Delta \vDash \Phi_1' \wedge \Phi_2$ and $|e[I]| = |e|[I] = |e'|[I] = |e'[I]|$, this completes (2).
  For (1), we have by \autoref{thm:subty-refl} some $\Phi_3$ such that $\Psi ; \Theta ; \Delta \vdash \tau[I/i] \subty \tau[I/i] : \star \gens \Phi_3$
  and $\Theta ; \Delta \vDash \Phi_3$.
  By AT-Sub, we  have that $\Psi ; \Theta ; \Delta ; \Omega' ; \Gamma' \vdash e [I] \checks \tau[I/i] \gens \Phi_1' \wedge \Phi_2 \wedge \Phi_3,\Gamma'''$,
  as required for (1).
  
  \item[(AT-Fix)] Suppose $\Psi ; \Theta ; \Delta ; \Omega ; \Gamma \vdash \texttt{fix } x.e \checks \tau \gens \Phi,\Gamma$
  by way of $\Psi ; \Theta ; \Delta ; \Omega, x : \tau ; \cdot \vdash e \checks \tau \gens \Phi,\Gamma''$, with
  $\Theta ; \Delta \vDash \Phi$,
  $\Psi ; \Theta ; \Delta \vdash \Gamma' \wknto \Gamma$, and
  $\Psi ; \Theta ; \Delta \vdash \Omega' \wknto \Omega$. Then,
  $\Psi ; \Theta ; \Delta \vdash \Omega', x : \tau \wknto \Omega, x : \tau$.
  By IH, there are $e'$, $\Phi'$, $\Gamma''$ such that
  $|e| = |e'|$,
  $\Theta ; \Delta \vDash \Phi'$, and
  $\Psi ; \Theta ; \Delta ; \Omega', x : \tau ; \cdot \vdash e' \checks \tau \gens \Phi',\Gamma'''$.
  By AT-Fix,
  $\Psi ; \Theta ; \Delta ; \Omega' ; \Gamma' \vdash \texttt{fix } x.e' \checks \tau \gens \Phi',\Gamma'$.
  Of course, $|\texttt{fix }x.e'| = \texttt{fix }x.|e'| = \texttt{fix }x.|e| = |\texttt{fix }x.e|$.
  Further, $\Psi ; \Theta ; \Delta \vdash \Gamma' \wknto \Gamma' \setminus \Gamma$
  by \autoref{thm:ctx-sub-subset-2}, and $\Psi ; \Theta ; \Delta \vdash \Gamma' \wknto \Gamma$ by one of the premises.
  This completes (1), and (2) follows by AT-Anno.
  
  \item[(AT-Ret)] Suppose
  $\Psi ; \Theta ; \Delta ; \Omega ; \Gamma \vdash \texttt{ret } e \checks \M \, \phi(I,\vec{p}) \, \tau \gens \Phi, \Gamma''$ from
  $\Psi ; \Theta ; \Delta ; \Omega ; \Gamma \vdash e \checks \tau \gens \Phi,\Gamma''$, with
  $\Theta ; \Delta \vdash \Phi$,
  $\Psi ; \Theta ; \Delta \vdash \Gamma' \wknto \Gamma$, and
  $\Psi ; \Theta ; \Delta \vdash \Omega' \wknto \Omega$.
  By IH, there are $e'$, $\Phi'$, $\Gamma'''$ such that 
  $|e'| = |e|$,
  $\Theta ; \Delta \vDash \Phi'$,
  $\Psi ;  \Theta ; \Delta \vdash \Gamma''' \wknto \Gamma' \setminus \Gamma$,
  $\Psi ;  \Theta ; \Delta \vdash \Gamma''' \wknto \Gamma''$, and
  $\Psi ; \Theta ; \Delta ; \Omega' ; \Gamma' \vdash e' \checks \tau \gens \Phi',\Gamma'''$.
  By AT-Ret,
  $\Psi ; \Theta ; \Delta ; \Omega' ; \Gamma' \vdash \texttt{ret } e' \checks \M \, \phi(I,\vec{p})\, \tau \gens \Phi',\Gamma'''$,
  completing (1). (2) follows by AT-Anno.
  

  \item[(AT-Bind)] Suppose
  $\Psi ; \Theta ; \Delta ; \Omega ; \Gamma \vdash \texttt{bind } x = e_1 \texttt{ in } e_2 \checks \M \, \phi(I,\vec{q})\, \tau_2 \gens (\vec{q} \geq \vec{p}) \wedge (I =J)  \wedge \Phi_1 \wedge \Phi_2, \Gamma_2 \setminus \{x : \tau_1\}$ from
  $\Psi ; \Theta ; \Delta ; \Omega ; \Gamma \vdash e_1 \infers \M \, \phi(J,\vec{p})\, \tau_1 \gens \Phi_1,\Gamma_1$ and
  $\Psi ; \Theta; \Delta ; \Omega ; \Gamma_1, x:\tau_1 \vdash e_2 \checks \M \, \phi(I,\vec{q} - \vec{p})\, \tau_2 \gens \Phi_2,\Gamma_2$ with
  $\Theta ; \Delta \vDash (\vec{q} \geq \vec{p}) \wedge (I =J)  \wedge \Phi_1 \wedge \Phi_2$,
  $\Psi ; \Theta ; \Delta \vdash \Gamma' \wknto \Gamma$, and
  $\Psi ; \Theta ; \Delta \vdash \Omega' \wknto \Omega$.
  By IH, there are $e_1'$, $\Phi_1'$, $\Gamma_1'$ such that
  $|e_1'| = |e_1|$,
  $\Theta ; \Delta \vDash \Phi_1'$,
  $\Psi ; \Theta ; \Delta \vdash \Gamma_1' \wknto \Gamma' \setminus \Gamma$,
  $\Psi ; \Theta ; \Delta \vdash \Gamma_1' \wknto \Gamma_1$, and
  $\Psi ; \Theta ; \Delta ; \Omega' ; \Gamma' \vdash e_1' \infers \M \, \phi(J,\vec{p}) \, \tau_1 \gens \Phi_1,'\Gamma_1'$.
  We note that $\Psi ; \Theta ; \Delta \vdash \Gamma_1',x:\tau_1 \wknto \Gamma_1, x: \tau_1$, and so
  by IH, there are $e_2'$, $\Phi_2'$, $\Gamma_2'$ such that
  $|e_2'| = |e_2|$,
  $\Theta ; \Delta \vDash \Phi_2'$,
  $\Psi ; \Theta ; \Delta \vdash \Gamma_2' \wknto \Gamma_1' \setminus \Gamma_1$,
  $\Psi ; \Theta ; \Delta \vdash \Gamma_2' \wknto \Gamma_2$, and
  $\Psi ; \Theta; \Delta ; \Omega' ; \Gamma_1', x:\tau_1 \vdash e_2' \checks \M \, \phi(I,\vec{q} - \vec{p})\, \tau_2 \gens \Phi_2',\Gamma_2'$.
  By AT-Bind,
  $\Psi ; \Theta ; \Delta ; \Omega' ; \Gamma' \vdash \texttt{bind } x = e_1' \texttt{ in } e_2' \checks \M \, \phi(I,\vec{q})\, \tau_2 \gens (\vec{q} \geq \vec{p}) \wedge (I =J)  \wedge \Phi_1' \wedge \Phi_2', \Gamma_2' \setminus \{x : \tau_1\}$.
  Of course, $|\texttt{bind } x = e_1 \texttt{ in } e_2| = |\texttt{bind } x = e_1' \texttt{ in } e_2'|$.
  Since $\Theta ; \Delta \vDash (\vec{q} \geq \vec{p}) \wedge (I=J)$, we also have that $\Theta ; \Delta \vDash (\vec{q} \geq \vec{p}) \wedge (I =J)  \wedge \Phi_1' \wedge \Phi_2'$.
  We have $\Psi ; \Theta ; \Delta \vdash \Gamma_2' \wknto \Gamma_2$  by IH, and
  $\Psi ; \Theta ; \Delta \vdash \Gamma_2' \setminus \{x : \tau_1\} \wknto \Gamma' \setminus \Gamma$ follows by applying \autoref{thm:ctx-sub-subset-2} to the rest of the weakening premises. This completes (1), and (2) follows by AT-Anno.
  
  \item[(AT-Tick)] Immediate.
  
  \item[(AT-Release)] Suppose
  $\Psi ; \Theta ; \Delta ; \Omega ; \Gamma \vdash \texttt{release } x = e_1 \texttt{ in }e_2 \checks \M \, \phi(I,\vec{p}) \, \tau_2 \gens (I = J) \wedge \Phi_1 \wedge \Phi_2, \Gamma_2 \setminus \{x\}$ from
  $\Psi ; \Theta ; \Delta ; \Omega ; \Gamma \vdash e_1 \infers [J | \vec{q}] \tau_1 \gens \Phi_1,\Gamma_1$ and
  $\Psi ; \Theta ; \Delta ; \Omega ; \Gamma_1, x : \tau \vdash e_2 \checks \M \, \phi(I,\vec{p} + \vec{q}) \, \tau_2 \gens \Phi_2, \Gamma_2$, with
  $\Theta ; \Delta \vDash (I = J) \wedge \Phi_1 \wedge \Phi_2$,
  $\Psi ; \Theta ; \Delta \vdash \Gamma' \wknto \Gamma$, and
  $\Psi ; \Theta ; \Delta \vdash \Omega' \wknto \Omega$.
  By IH, there are $e_1'$, $\Phi_1'$, $\Gamma_1'$ such that
  $|e_1'| = |e_1|$,
  $\Theta ; \Delta \vDash \Phi_1'$,
  $\Psi ; \Theta ; \Delta \vDash \Gamma_1' \wknto \Gamma' \setminus \Gamma$,
  $\Psi ; \Theta ; \Delta \vDash \Gamma_1' \wknto \Gamma_1$, and
  $\Psi ; \Theta ; \Delta ; \Omega' ; \Gamma' \vdash e_1' \infers [J|\vec{q}] \tau_1 \gens \Phi_1',\Gamma_1'$.
  Since $\Psi ; \Theta ; \Delta \vDash \Gamma_1', x : \tau \wknto \Gamma_1, x : \tau$, we have
  by IH that there are $e_2'$, $\Phi_2'$, $\Gamma_2'$ such that
  $|e_2'| = |e_2|$,
  $\Theta ; \Delta \vDash \Phi_2'$,
  $\Psi ; \Theta ; \Delta \vdash \Gamma_2' \wknto \Gamma_1' \setminus \Gamma_1$,
  $\Psi ; \Theta ; \Delta \vdash \Gamma_2' \wknto \Gamma_2$, and
  $\Psi ; \Theta ; \Delta ; \Omega' ; \Gamma_1', x : \tau \vdash e_2' \checks \M \, \phi(I,\vec{p} + \vec{q}) \, \tau_2 \gens \Phi_2',\Gamma_2'$.
  By AT-Release,
  $\Psi ; \Theta ; \Delta ; \Omega' ; \Gamma' \vdash \texttt{release } x = e_1' \texttt{ in }e_2' \checks \M \, \phi(I,\vec{p}) \, \tau_2 \gens (I = J) \wedge \Phi_1' \wedge \Phi_2', \Gamma_2 \setminus \{x\}$.
  Of course, $|\texttt{release } x = e_1' \texttt{ in }e_2'| = |\texttt{release } x = e_1 \texttt{ in }e_2|$.
  Since $\Theta ; \Delta \vDash I = J$, we have $\Theta ; \Delta \vDash (I = J) \wedge \Phi_1' \wedge \Phi_2'$.
  $\Psi ; \Theta ; \Delta \vdash \Gamma_2' \setminus \{x\} \wknto \Gamma_2 \{x\}$ is implied by $\Psi ; \Theta ; \Delta \vdash \Gamma_2' \wknto \Gamma_2$,
  and $\Psi ; \Theta ; \Delta \vdash \Gamma_2' \setminus \{x\} \wknto \Gamma' \setminus \Gamma$ follows from \autoref{thm:ctx-sub-subset-2}. This completes (1), and (2) follows from AT-Anno.
  
  \item[(AT-Store)] Suppose
  $\Psi ; \Theta ; \Delta ; \Omega ; \Gamma \vdash \texttt{store}[K|\vec{w}](e) \checks \M \, \phi(I,\vec{q}) \, ([J | \vec{p}] \, \tau) \gens \Phi_1 \wedge \Phi_2 \wedge\Phi_3 \wedge  (\vec{p} \leq \vec{w} \leq \vec{q}) \wedge (I = J = K), \Gamma''$ by way of
  $\Theta ; \Delta \vdash K : \N \gens \Phi_1$,
  $\Theta ; \Delta \vdash \vec{w} : \vec{\mathbb{R}^+} \gens \Phi_2$,
  $\Psi ; \Theta ; \Delta ; \Omega ; \Gamma \vdash e \checks \tau \gens \Phi_3,\Gamma''$, with
  $\Theta ; \Delta \vDash \Phi_1 \wedge \Phi_2 \wedge\Phi_3 \wedge  (\vec{p} \leq \vec{w} \leq \vec{q}) \wedge (I = J = K)$,
  $\Psi ; \Theta ; \Delta \vdash \Gamma' \wknto \Gamma$, and
  $\Psi ; \Theta ; \Delta \vdash \Omega' \wknto \Omega$.
  By IH, there are $e'$, $\Gamma'''$, $\Phi_3'$ such that
  $|e'| = |e|$,
  $\Theta ; \Delta \vDash \Phi_3'$,
  $\Psi ; \Theta ; \Delta \vdash \Gamma''' \wknto \Gamma' \setminus \Gamma$,
  $\Psi ; \Theta ; \Delta \vdash \Gamma''' \wknto \Gamma''$, and
  $\Psi ; \Theta ; \Delta ; \Omega' ; \Gamma' \vdash e' \checks \tau \gens \Phi_3', \Gamma'''$.
  By AT-Store,
  $\Psi ; \Theta ; \Delta ; \Omega' ; \Gamma' \vdash \texttt{store}[K|\vec{w}](e') \checks \M \, \phi(I,\vec{q}) \, ([J | \vec{p}] \, \tau) \gens \Phi_1 \wedge \Phi_2 \wedge\Phi_3' \wedge  (\vec{p} \leq \vec{w} \leq \vec{q}) \wedge (I = J = K), \Gamma'''$,
  which completes (1). For (2), AT-Anno suffices.
  
  \item[(AT-StoreConst)] Suppose $\Psi ; \Theta ; \Delta ; \Omega ; \Gamma \vdash \texttt{store}[J](e) \checks \M \, \phi(K,\vec{p}) \, ([I] \, \tau) \gens (\texttt{const}(I) \leq \texttt{const}(J) \leq \vec{p}) \wedge \Phi_1 \wedge \Phi_2, \Gamma''$ from
  $\Psi ; \Theta ; \Delta ; \Omega ; \Gamma \vdash e \checks \tau \gens \Phi_1,\Gamma''$ and
  $\Theta ; \Delta \vdash J : \mathbb{R} \gens \Phi_2$, with
  $\Theta ; \Delta \vDash (\texttt{const}(I) \leq \texttt{const}(J) \leq \vec{p}) \wedge \Phi_1 \wedge \Phi_2$,
  $\Psi ; \Theta ; \Delta \vdash \Gamma' \wknto \Gamma$, and
  $\Psi ; \Theta ; \Delta \vdash \Omega' \wknto \Omega$.
  By IH, there are $e'$, $\Phi_1'$, $\Gamma'''$ such that
  $|e'| = |e|$,
  $\Theta ; \Delta \vDash \Phi_1'$,
  $\Phi ; \Theta ; \Delta \vdash \Gamma''' \wknto \Gamma' \setminus \Gamma$,
  $\Phi ; \Theta ; \Delta \vdash \Gamma''' \wknto \Gamma''$, and
  $\Phi ; \Theta ; \Delta ; \Omega' ; \Theta' \vdash e' \checks \tau \gens \Phi_1', \Gamma'''$.
  By AT-StoreConst,
  $\Psi ; \Theta ; \Delta ; \Omega' ; \Gamma \vdash \texttt{store}[J](e') \checks \M \, \phi(K,\vec{p}) \, ([I] \, \tau) \gens (\texttt{const}(I) \leq \texttt{const}(J) \leq \vec{p}) \wedge \Phi_1' \wedge \Phi_2, \Gamma''$, which completes (1). For (2), we use AT-Anno.
  
  \item[(AT-ReleaseConst)] Suppose
  $\Psi ; \Theta ; \Delta ; \Omega ; \Gamma \vdash \texttt{release } x = e_1 \texttt{ in }e_2 \checks \M \, \phi(I,\vec{p}) \, \tau_2 \gens \Phi_1 \wedge \Phi_2, \Gamma_2 \setminus \{x\}$ from
  $\Psi ; \Theta ; \Delta ; \Omega ; \Gamma \vdash e_1 \infers [J] \tau_1 \gens \Phi_1,\Gamma_1$, and
  $\Psi ; \Theta ; \Delta ; \Omega ; \Gamma_1, x : \tau_1 \vdash e_2 \checks \M \, \phi(I,\vec{p} + \texttt{const}(J)) \, \tau_2 \gens \Phi_2, \Gamma_2$, with
  $\Theta ; \Delta \vDash \Phi_1 \wedge \Phi_2$,
  $\Psi ; \Theta ; \Delta \vdash \Gamma' \wknto \Gamma$, and
  $\Psi ; \Theta ; \Delta \vdash \Omega' \wknto \Omega$.
  By IH, there are $e_1'$, $\Gamma_1'$, $\Phi_1'$ such that
  $|e_1'| = |e_1|$,
  $\Theta ; \Delta \vDash \Phi_1'$,
  $\Psi ; \Theta ; \Delta \vdash \Gamma_1' \wknto \Gamma' \setminus \Gamma$,
  $\Psi ; \Theta ; \Delta \vdash \Gamma_1' \wknto \Gamma_1$, and
  $\Psi ; \Theta ; \Delta ; \Omega ; \Gamma' \vdash e_1' \infers [J]\tau_1 \gens \Phi_1',\Gamma_1'$.
  Since $\Psi ; \Theta ; \Delta \vdash \Gamma_1',x : \tau_1 \wknto \Gamma_1, x : \tau_1$,
  we have by IH that there are $e_2'$, $\Phi_2'$, $\Gamma_2'$ such that
  $|e_2'| = |e_2|$,
  $\Theta ; \Delta \vDash \Phi_2'$,
  $\Psi ; \Theta ; \Delta \vdash \Gamma_2' \wknto (\Gamma_1',x : \tau_1) \setminus (\Gamma_1, x : \tau_1)$,
  $\Psi ; \Theta ; \Delta \vdash \Gamma_2' \wknto \Gamma_2$, and
  $\Psi ; \Theta ; \Delta ; \Omega' ; \Gamma_1', x : \tau_1 \vdash e_2' \checks \M \, \phi(I,\vec{p} + \texttt{const}(J)) \, \tau_2 \gens \Phi_2', \Gamma_2'$.
  By AT-ReleaseCont,
  $\Psi ; \Theta ; \Delta ; \Omega' ; \Gamma' \vdash \texttt{release } x = e_1' \texttt{ in }e_2' \checks \M \, \phi(I,\vec{p}) \, \tau_2 \gens \Phi_1' \wedge \Phi_2', \Gamma_2' \setminus \{x\}$.
  Of course, $|\texttt{release } x = e_1' \texttt{ in }e_2'| = \texttt{release } x = |e_1'| \texttt{ in }|e_2'| = \texttt{release } x = |e_1| \texttt{ in }|e_2|
  = |\texttt{release } x = e_1 \texttt{ in }e_2|$.
  Also, $\Theta ; \Delta \vDash \Phi_1' \wedge \Phi_2'$.
  $\Psi ; \Theta ; \Delta \vdash \Gamma_2' \wknto \Gamma_2$ holds by IH.
  It remains to show that $\Psi ; \Theta ; \Delta \vdash \Gamma_2' \setminus \{x\} \wknto \Gamma' \setminus \Gamma$.
  This follows by considering the remaining weakening judgments from the IHs with \autoref{thm:ctx-sub-subset-2}.
  This completes (1). For (2), as usual, we apply AT-Anno.
  
  \item[(AT-Shift)] Suppose $\Psi ; \Theta ; \Delta ; \Omega ; \Gamma \vdash \texttt{shift}(e) \checks \M \, \phi(I,\vec{q}) \, \tau \gens (I \geq 1) \wedge \Phi, \Gamma''$ from
  $\Psi ; \Theta ; \Delta  ; \Omega ; \Gamma \vdash e \checks \M \, \phi(I - 1,\lhd \vec{q}) \, \tau \gens \Phi, \Gamma''$, with
  $\Theta ; \Delta \vDash (I \geq 1) \wedge \Phi$,
  $\Psi ; \Theta ; \Delta \vdash \Gamma' \wknto \Gamma$,
  $\Psi ; \Theta ; \Delta \vdash \Omega' \wknto \Omega$.
  By IH, there are $e'$, $\Phi'$, $\Gamma'''$ such that
  $|e'| = |e|$,
  $\Theta ; \Delta \vDash \Phi'$,
  $\Psi ; \Theta ; \Delta \vdash \Gamma''' \wknto \Gamma' \setminus \Gamma$,
  $\Psi ; \Theta ; \Delta \vdash \Gamma''' \wknto \Gamma''$,
  $\Psi ; \Theta ; \Delta ; \Omega' ; \Gamma' \vdash e' \checks \M \, \phi(I-1,\lhd \vec{q}) \, \tau \gens \Phi', \Gamma'''$.
  By AT-Shift,
  $\Psi ; \Theta ; \Delta ; \Omega' ; \Gamma' \vdash \texttt{shift}(e') \checks \M \, \phi(I,\vec{q}) \, \tau \gens (I \geq 1) \wedge \Phi', \Gamma'''$,
  which completes (1). For (2), we apply AT-Anno.
  
  \item[(AT-CImpI)] Suppose $\Psi ; \Theta ; \Delta ; \Omega ; \Gamma \vdash \Lambda .e \checks (\Phi' \Rightarrow \tau) \gens (\Phi' \to \Phi),\Gamma''$ from
  $\Psi ; \Theta ; \Delta, \Phi'; \Omega ; \Gamma \vdash e \checks \tau \gens \Phi,\Gamma''$, with
  $\Theta ; \Delta \vDash \Phi$,
  $\Psi ; \Theta ; \Delta \vdash \Gamma' \wknto \Gamma$,
  $\Psi ; \Theta ; \Delta \vdash \Omega' \wknto \Omega$.
  Using \autoref{thm:cxt-sub-wkn}, we have by IH that there are $e'$, $\Phi''$, $\Gamma'''$,
  $|e'| = |e|$,
  $\Theta ; \Delta, \Phi' \vDash \Phi''$,
  $\Psi ; \Theta ; \Delta, \Phi' \vdash \Gamma''' \wknto \Gamma' \setminus \Gamma$,
  $\Psi ; \Theta ; \Delta, \Phi' \vdash \Gamma''' \wknto \Gamma''$, and
  $\Psi ; \Theta ; \Delta, \Phi' ; \Omega' ; \Gamma' \vdash e \checks \tau \gens \Phi'', \Gamma'''$.
  By AT-CImpI,
  $\Psi ; \Theta ; \Delta ; \Omega' ; \Gamma' \vdash \Lambda .e' \checks (\Phi' \Rightarrow \tau) \gens (\Phi' \to \Phi''),\Gamma'''$.
  By definition, $|\Lambda .e'| = \Lambda.|e'| = \Lambda.|e| = |\Lambda.e|$.
  Next, since $\Theta ; \Delta, \Phi' \vDash \Phi''$, we have $\Theta ; \Delta \vDash \Phi' \to \Phi''$.
  The two weakenings again follow by \autoref{thm:ctx-sub-subset-2}.
  This completes (1), and (2) follows by AT-Anno.
  
  \item[(AT-CImpE)] Suppose $\Psi ; \Theta ; \Delta ; \Omega ; \Gamma \vdash e \{\} \infers \tau \gens \Phi \wedge \Phi',\Gamma''$ from
  $\Psi ; \Theta ; \Delta ; \Omega ; \Gamma \vdash e \infers (\Phi' \Rightarrow \tau) \gens \Phi,\Gamma''$, with
  $\Theta ; \Delta \vDash \Phi \wedge \Phi'$,
  $\Psi ; \Theta ; \Delta \vdash \Gamma' \wknto \Gamma$, and
  $\Psi ; \Theta ; \Delta \vdash \Omega' \wknto \Omega$.
  By IH, there are $e'$, $\Phi''$. $\Gamma'''$ such that
  $|e'| = |e|$,
  $\Theta ; \Delta \vDash \Phi''$.
  $\Psi ; \Theta ; \Delta \vdash \Gamma''' \wknto \Gamma' \setminus \Gamma$,
  $\Psi ; \Theta ; \Delta \vdash \Gamma''' \wknto \Gamma''$, and
  $\Psi ; \Theta ; \Delta ; \Omega' ; \Gamma' \vdash e' \infers (\Phi' \Rightarrow \tau) \gens \Phi'',\Gamma'''$.
  By AT-CImpE,
  $\Psi ; \Theta ; \Delta ; \Omega' ; \Gamma' \vdash e' \{\} \infers \tau \gens \Phi'' \wedge \Phi',\Gamma'''$,
  which completes the proof of (2). For (1), we use \autoref{thm:subty-refl} to get some $\Phi_1$ with $\Theta ; \Delta \vDash \Phi_1$
  such that $\Psi ; \Theta ; \Delta \vdash \tau \subty \tau : \star \gens \Phi_1$.
  Then, by AT-Sub,
  $\Psi ; \Theta ; \Delta ; \Omega' ; \Gamma' \vdash e' \{\} \checks \tau \gens \Phi'' \wedge \Phi' \wedge \Phi_3,\Gamma'''$, as required.
  
  \item[(AT-CAndI)] Suppose $\Psi ; \Theta ; \Delta ; \Omega ; \Gamma \vdash <e> \checks \Phi' \amp \tau \gens \Phi \wedge \Phi',\Gamma''$ from
  $\Psi ; \Theta ; \Delta ; \Omega ; \Gamma \vdash e \checks \tau \gens \Phi,\Gamma''$, with
  $\Theta ; \Delta \vDash \Phi \wedge \Phi'$,
  $\Psi ; \Theta ; \Delta \vdash \Gamma' \wknto \Gamma$, and
  $\Psi ; \Theta ; \Delta \vdash \Omega' \wknto \Omega$.
  By IH, there are $e'$, $\Phi''$, $\Gamma'''$ such that
  $|e'| = |e|$,
  $\Theta ; \Delta \vDash \Phi''$,
  $\Psi ; \Theta ; \Delta \vdash \Gamma''' \wknto \Gamma' \setminus \Gamma$,
  $\Psi ; \Theta ; \Delta \vdash \Gamma''' \wknto \Gamma''$, and
  $\Psi ; \Theta ; \Delta ; \Omega' ; \Gamma' \vdash e' \checks \tau \gens \Phi'', \Gamma'''$.
  By AT-CAndI,
  $\Psi ; \Theta ; \Delta ; \Omega' ; \Gamma' \vdash <e'> \checks \Phi' \amp \tau \gens \Phi'' \wedge \Phi',\Gamma'''$,
  which completes (1). For (2), one use of AT-Anno suffices.
  
  \item[(AT-CAndE)] Suppose
  $\Psi ; \Theta ; \Delta ; \Omega ; \Gamma \vdash \texttt{clet } x = e_1 \texttt{ in } e_2 \checks \tau' \gens \Phi_1 \wedge (\Phi \to \Phi_2),\Gamma_2 \setminus \{x : \tau\}$ from
  $\Psi ; \Theta ; \Delta ; \Omega ; \Gamma \vdash e_1 \infers \Phi \amp \tau \gens \Phi_1,\Gamma_1$ and
  $\Psi ; \Theta ; \Delta, \Phi ; \Omega ; \Gamma_1, x : \tau \vdash e_2 \checks \tau' \gens \Phi_2, \Gamma_2$, with
  $\Theta ; \Delta \vDash \Phi_1 \wedge (\Phi \to \Phi_2)$,
  $\Psi ; \Theta ; \Delta \vdash \Gamma' \wknto \Gamma$, and
  $\Psi ; \Theta ; \Delta \vdash \Omega' \wknto \Omega$.
  By IH, there are $e_1'$, $\Phi_1'$, $\Gamma_1'$ such that
  $|e_1'| = |e_1|$,
  $\Theta ; \Delta \vDash \Phi_1'$,
  $\Psi ; \Theta ; \Delta \vdash \Gamma_1' \wknto \Gamma' \setminus \Gamma$,
  $\Psi ; \Theta ; \Delta \vdash \Gamma_1' \wknto \Gamma_1$, and
  $\Psi ; \Theta ; \Delta ; \Omega' ; \Gamma' \vdash e_1' \infers \Phi \amp \tau \gens \Phi_1',\Gamma_1'$.
  Since $\Psi ; \Theta ; \Delta \vdash \Gamma_1' \wknto \Gamma_1$, we also have by \autoref{thm:ctx-sub-wkn}
  $\Psi ; \Theta ; \Delta, \Phi \vdash \Gamma_1', x : \tau \wknto \Gamma_1, x : \tau$,
  and so by IH, there are $e_2'$, $\Phi_2'$, $\Gamma_2'$ such that
  $|e_2'| = |e_2|$,
  $\Theta ; \Delta, \Phi \vDash \Phi_2'$,
  $\Psi ; \Theta ; \Delta, \Phi \vdash \Gamma_2' \wknto \Gamma_1' \setminus \Gamma_1$,
  $\Psi ; \Theta ; \Delta, \Phi \vdash \Gamma_2' \wknto \Gamma_2$, and
  $\Psi ; \Theta ; \Delta, \Phi ; \Omega' ; \Gamma_1', x : \tau \vdash e_2' \checks \tau' \gens \Phi_2', \Gamma_2'$.
  By AT-CAndE,
  $\Psi ; \Theta ; \Delta ; \Omega' ; \Gamma' \vdash \texttt{clet } x = e_1' \texttt{ in } e_2' \checks \tau' \gens \Phi_1' \wedge (\Phi \to \Phi_2'),\Gamma_2' \setminus \{x : \tau\}$.
  As usual, we have that $|\texttt{clet } x = e_1' \texttt{ in } e_2'| = |\texttt{clet } x = e_1 \texttt{ in } e_2|$.
  Combining the satisfactions from the premises, we have that $\Theta ; \Delta \vDash \Phi_1' \wedge (\Phi \to \Phi_2')$.
  The fact that $\Psi ; \Theta ; \Delta, \Phi \vdash \Gamma_2' \wknto \Gamma_2$ is immediate from IH, 
  and $\Psi ; \Theta ; \Delta, \Phi \vdash \Gamma_2' \wknto \Gamma' \setminus \Gamma$ follows as usual by considering the rest
  of the weakening judgments with \autoref{thm:ctx-sub-subset-2}. This completes (1). For (2), we apply AT-Anno.
   
  \item[(AT-Sub)] Suppose $\Psi ; \Theta ; \Delta ; \Omega ; \Gamma \vdash e \checks \tau \gens \Phi_1 \wedge \Phi_2,\Gamma''$ from
  $\Psi ; \Theta ; \Delta ; \Omega ; \Gamma \vdash e \infers \tau' \gens \Phi_1,\Gamma''$ and
  $\Psi;\Theta;\Delta \vdash \tau' \subty \tau : \star \gens \Phi_2$, with
  $\Theta ; \Delta \vDash \Phi_1 \wedge \Phi_2$,
  $\Psi ; \Theta ; \Delta \vdash \Gamma' \wknto \Gamma$, and
  $\Psi ; \Theta ; \Delta \vdash \Omega' \wknto \Omega$.
  By IH, there are $e'$, $\Phi_1'$, $\Gamma'''$ such that
  $|e'| = |e|$,
  $\Theta ; \Delta \vDash \Phi_1'$,
  $\Psi ; \Theta ; \Delta \vdash \Gamma''' \wknto \Gamma' \setminus \Gamma$,
  $\Psi ; \Theta ; \Delta \vdash \Gamma''' \wknto \Gamma''$, and
  $\Psi ; \Theta ; \Delta ; \Omega' ; \Gamma' \vdash e' \infers \tau' \gens \Phi_1', \Gamma'''$.
  By AT-Sub, $\Psi ; \Theta ; \Delta ; \Omega' ; \Gamma' \vdash e' \checks \tau \gens \Phi_1' \wedge \Phi_2,\Gamma'''$,
  which completes (1). For (2), one use of AT-Anno suffices.
  
  \item[(AT-Anno)] Suppose $\Psi ; \Theta ; \Delta ; \Omega ; \Gamma \vdash (e : \tau) \infers \tau \gens \Phi,\Gamma''$ from 
  $\Psi ; \Theta ; \Delta ; \Omega ; \Gamma \vdash e \checks \tau \gens \Phi,\Gamma''$ with
  $\Theta ; \Delta \vDash \Phi$,
  $\Psi ; \Theta ; \Delta \vdash \Gamma' \wknto \Gamma$, and
  $\Psi ; \Theta ; \Delta \vdash \Omega' \wknto \Omega$.
  By IH, there are $e'$, $\Phi'$, $\Gamma'''$ such that
  $|e| = |e'|$,
  $\Theta ; \Delta \vDash \Phi'$,
  $\Psi ; \Theta ; \Delta \vdash \Gamma''' \wknto \Gamma' \setminus \Gamma$,
  $\Psi ; \Theta ; \Delta \vdash \Gamma''' \wknto \Gamma''$, and
  $\Psi ; \Theta ; \Delta ; \Omega' ; \Gamma' \vdash e' \checks \tau \gens \Phi',\Gamma'''$.
  By AT-Anno, $\Psi ; \Theta ; \Delta ; \Omega' ; \Gamma' \vdash (e' : \tau) \infers \tau \gens \Phi',\Gamma'''$, which completes (2)
  since $|(e' : \tau)| = |e'| = |e| = |(e : \tau)|$. For (1), we use \autoref{thm:subty-refl} to get that $\Psi ; \Theta ; \Delta \vdash \tau \subty  \tau : \star \gens \Phi''$ with $\Theta ; \Delta \vDash \Phi''$, and so by AT-Sub, we have that $\Psi ; \Theta ; \Delta ; \Omega' ; \Gamma' \vdash (e' : \tau) \checks \tau \gens \Phi' \wedge \Phi'',\Gamma'''$, completing (1).
 
  
  
\end{itemize}
\end{proof}

\tycheckcompl*
\jtheorem{Proof of \autoref{thm:tycheck-compl}}{
By induction on the derivation of $\Psi;\Theta;\Delta;\Omega;\Gamma \vdash e : \tau$, we prove both claims simultaneously. In each case, we will only prove one of the two conclusions, based on the direction of the corresponding algorithmic rule. In cases where the rule is checking claim (2) follows immediately by AT-Anno. In cases where the rule is inferring, (1) follows by \autoref{thm:subty-refl} and then AT-Sub. In all cases, the erasure property is immediate from the inductive hypotheses-- we will elide this bit of the proof.

 \jcase{1}{T-Var-1}{
  \jgivengoalThreeExists{
   \caseFact{1} $\Psi ; \Theta ; \Delta ; \Omega ; \Gamma \vdash x : \tau$
   
   \caseFact{2} $x : \tau \in \Gamma$
  }{
   $e'$, $\Phi'$, $\Gamma'$
  }{
   $|e'| = x$
  }{
   $\Psi ; \Theta ; \Delta \vDash \Phi'$   
  }{
   $\Psi ; \Theta ; \Delta ; \Omega ; \Gamma \pvdash e' \infers \tau \gens \Phi,\Gamma'$  
  }
   Immediate
 }
 
 \jcase{2}{T-Var-2}{Immediate}
 
 \jcase{3}{T-Unit}{Immediate}
 
 \jcase{4}{T-Base}{Immediate}
 
 \jcase{5}{T-Absurd}{Immediate}
 
 \jcase{6}{T-Nil}{
  \jgivengoalThreeExists{
   \caseFact{1} $\Psi ; \Theta ; \Delta ; \Omega ; \Gamma\pvdash \texttt{nil} : L^I \tau$
   
   \caseFact{2} $\Theta ; \Delta \vdash I : \mathbb{N}$
   
   \caseFact{3} $\Theta;\Delta \vDash I = 0$
  }{
   $e'$, $\Phi'$,$\Gamma'$  
  }{
   $|e'| = \texttt{nil}$  
  }{
   $\Theta ; \Delta \vDash \Phi'$  
  }{
   $\Psi ; \Theta ; \Delta ; \Omega ; \Gamma \pvdash e' \checks L^I \tau \gens \Phi',\Gamma'$  
  }
  
  \caseText{By \autoref{thm:sort-compl}, there is a $\Phi$ such that}
  
  \caseFact{4} $\Theta ; \Delta \vdash I : \mathbb{N} \gens \Phi$
  
  \caseFact{5} $\Theta ; \Delta \vDash \Phi$
  
  \caseText{Goals follow by AT-Nil, letting $e' = \texttt{nil}$, $\Phi' = \Phi \wedge (I = 0)$, and $\Gamma' = \Gamma$}
 }
 
 \jcase{7}{T-Cons}{
  \jgivengoalThreeExists{
   \caseFact{1} $\Psi ; \Theta ; \Delta ; \Omega ; \Gamma_1,\Gamma_2\pvdash e_1 :: e_2 : L^I \tau$
   
   \caseFact{2} $\Psi ; \Theta ; \Delta ; \Omega ; \Gamma_1\pvdash e_1 : \tau$
   
   \caseFact{3} $\Psi ; \Theta ; \Delta ; \Omega ; \Gamma_2\pvdash e_2 : L^{I-1} \tau$
   
   \caseFact{4} $\Theta ; \Delta \vDash I \geq 1$
  }{
   $e'$, $\Phi'$,$\Gamma'$  
  }{
   $|e'| = e_1 :: e_2$
  }{
   $\Theta ; \Delta \vDash \Phi'$  
  }{
   $\Psi ; \Theta ; \Delta ; \Omega ; \Gamma_1,\Gamma_2 \pvdash e' \checks L^I \tau \gens \Phi',\Gamma'$  
  }
  
  \caseText{By IH, on (2) there are $e_1'$, $\Phi_1$, $\Gamma_1'$ such that}
  
  \caseFact{5} $|e_1'| = e_1$
  
  \caseFact{6} $\Theta ; \Delta \vDash \Phi_1'$
  
  \caseFact{7} $\Psi ; \Theta ; \Delta ; \Omega ; \Gamma_1 \vdash e_1' \checks \tau \gens \Phi_1,\Gamma_1'$
  
  \caseText{By \autoref{thm:admits-weaken} on (7), there are $e_1''$, $\Phi_1'$, $\Gamma_1''$ such that}
  
  \caseFact{8} $|e_1''| = |e_1'|$
  
  \caseFact{9} $\Theta ; \Delta \vDash \Phi_1'$
  
  \caseFact{10} $\Psi ; \Theta ; \Delta \vdash \Gamma_1'' \wknto (\Gamma_1,\Gamma_2) \setminus \Gamma_1$
  
  \caseFact{11} $\Psi ; \Theta ; \Delta ; \Omega ; \Gamma_1,\Gamma_2 \vdash e_1'' \checks \tau \gens \Phi_1',\Gamma_1''$
  
  \caseText{By IH, there are $e_2'$, $\Phi_2$, $\Gamma_2'$ such that}
  
  \caseFact{12} $|e_2'| = e_2$
  
  \caseFact{13} $\Theta ; \Delta \vDash \Phi_2$
  
  \caseFact{14} $\Psi ; \Theta ; \Delta ; \Omega ; \Gamma_2 \vdash e_2' \checks L^{I-1} \tau \gens \Phi_2,\Gamma_2'$
  
  \caseText{Again by \autoref{thm:admits-weaken} on (14), we have $e_2''$, $\Phi_2'$, $\Gamma_2''$ such that}
  
  \caseFact{15} $|e_2''| = |e_2'|$
  
  \caseFact{16} $\Theta ; \Delta \vDash \Phi_2'$
  
  \caseFact{17} $\Psi ; \Theta ; \Delta ; \Omega ; \Gamma_1'' \vdash e_2'' \checks L^{I-1} \tau \gens \Phi_2',\Gamma_2''$
  
  \caseText{By AT-Cons on (11) and (17)}
  
  \caseFact{18} $\Psi ; \Theta ; \Delta ; \Omega ; \Gamma_1,\Gamma_2 \vdash e_1'' :: e_2'' \checks L^I \tau \gens \Phi_1' \wedge \Phi_2' \wedge (I \geq 1), \Gamma_2''$
  
  \caseText{Goals follow by $e' =  e_1'' :: e_2''$, $\Phi' = \Phi_1' \wedge \Phi_2' \wedge (I \geq 1)$, and $\Gamma' = \Gamma_2''$}
 }
 
 \jcase{8}{T-Match}{
  \jgivengoalThreeExists{
   \caseFact{1} $\Psi ; \Theta ; \Delta ; \Omega ; \Gamma_1,\Gamma_2\vdash \texttt{match}(e,e_1,h.t.e_2) : \tau'$
   
   \caseFact{2} $\Psi ; \Theta ; \Delta ; \Omega ; \Gamma_1\vdash e : L^I \tau$
   
   \caseFact{3} $\Psi ; \Theta ; \Delta, I = 0 ; \Omega ; \Gamma_2\vdash e_1 : \tau'$
   
   \caseFact{4} $\Psi ; \Theta ; \Delta, I \geq 1; \Omega ; \Gamma_2, h : \tau, t : L^I \tau \vdash e_2 : \tau'$
  }{
   $e'$, $\Phi'$,$\Gamma'$  
  }{
   $|e'| = \texttt{match}(e,e_1,h.t.e_2)$
  }{
   $\Theta ; \Delta \vDash \Phi'$  
  }{
   $\Psi ; \Theta ; \Delta ; \Omega ; \Gamma_1,\Gamma_2 \pvdash e' \checks \tau' \gens \Phi',\Gamma'$  
  }
  
  \caseText{By IH on (2)}
  
  \caseFact{5} $|e'| = e$
  
  \caseFact{6} $\Theta ; \Delta \vDash \Phi$
  
  \caseFact{7} $\Psi ; \Theta ; \Delta ; \Omega ; \Gamma_1 \vdash e' \infers L^I \tau \gens \Phi, \Gamma_1'$
  
  \caseText{By \autoref{thm:admits-weaken} on (7), there are $e''$, $\Phi'$, $\Gamma_1''$ such that}
  
  \caseFact{8} $|e''| = |e'|$
  
  \caseFact{9} $\Theta ; \Delta \vDash \Phi'$
  
  \caseFact{10} $\Psi ; \Theta ; \Delta \vdash \Gamma_1'' \wknto \Gamma_2$
  
  \caseFact{11} $\Psi ; \Theta ; \Delta ; \Omega ; \Gamma_1,\Gamma_2 \vdash e'' \infers L^I \tau \gens \Phi', \Gamma_1''$
  
  \caseText{By IH on (3) there are $e_1'$, $\Phi_1$, $\Gamma_2'$ such that}
  
  \caseFact{12} $|e_1'| = e_1$

  \caseFact{13} $\Theta ; \Delta, I = 0 \vDash \Phi_1$

  \caseFact{14} $\Psi ; \Theta ; \Delta, I = 0 ; \Omega ; \Gamma_2 \vdash e_1' \checks \tau' \gens \Phi_1,\Gamma_2'$

  \caseText{By \autoref{thm:ctx-sub-wkn} on (10)}
  
  \caseFact{15} $\Psi ; \Theta ; \Delta, I = 0 \vdash \Gamma_1'' \wknto \Gamma_2$.

  \caseText{Then, by \autoref{thm:admits-weaken}, there are $e_1''$, $\Phi_1'$, $\Gamma_2''$ such that}

  \caseFact{16} $|e_1''| = |e_1'|$,
  
  \caseFact{17} $\Theta ; \Delta, I = 0 \vDash \Phi_1'$

  \caseFact{18} $\Psi ; \Theta ; \Delta, I = 0 ; \Omega ; \Gamma_1'' \vdash e_1'' \checks \tau' \gens \Phi_1',\Gamma_2''$

  \caseText{By IH again on (3) we have $e_2'$, $\Phi_2$, $\Gamma_3'$ such that}

  \caseFact{19}  $|e_2'| = e_2$

  \caseFact{20} $\Theta ; \Delta, I \geq 1 \vdash \Phi_2$

  \caseFact{21} $\Psi ; \Theta ; \Delta, I \geq 1 ; \Omega ; \Gamma_2, h : \tau, t : L^I \tau \vdash e_2' \checks \tau' \gens \Phi_2,\Gamma_3'$

  \caseText{By \autoref{thm:ctx-sub-wn} and \autoref{thm:ctx-sub-subset-2}}

  \caseFact{22} $\Psi ; \Theta ; \Delta, I \geq 1 \vdash \Gamma_1'', h : \tau, t : L^I \tau \wknto \Gamma_2, h : \tau, t : L^I \tau$

  \caseText{By \autoref{thm:admits-weaken} on (21) and (22), there are are $e_2''$, $\Phi_2'$, $\Gamma_3''$ such that}

  \caseFact{23} $|e_2''| = |e_2'|$

  \caseFact{24} $\Theta ; \Delta, I \geq 1 \vDash \Phi_2'$

  \caseFact{25} $\Psi ; \Theta ; \Delta, I \geq 1; \Omega ; \Gamma_1'', h : \tau, t : L^I \tau \vdash e_2'' \checks \tau' \gens \Phi_2',\Gamma_3''$

  \caseText{Goals follow by AT-Match on (11), (18), (25)}
  
 }

\jcase{9}{T-ExistI}{
  \jgivengoalThreeExists{
   \caseFact{1} $\Psi ; \Theta ; \Delta ; \Omega ; \Gamma\pvdash \texttt{pack}[I](e) : \exists i : S. \tau$
   
   \caseFact{2} $\Theta ; \Delta \vdash I : S$
   
   \caseFact{3} $\Psi ; \Theta ; \Delta ; \Omega ; \Gamma \pvdash e : \tau[I/i]$
  }{
   $e'$, $\Phi'$,$\Gamma'$  
  }{
   $|e'| = \texttt{pack}[I](e) $  
  }{
   $\Theta ; \Delta \vDash \Phi'$  
  }{
   $\Psi ; \Theta ; \Delta ; \Omega ; \Gamma \pvdash e' \checks \exists i : S. \tau \gens \Phi',\Gamma'$  
  }
  
  \caseText{By \autoref{thm:sort-compl} on (2) , there is a $\Phi_1$ such that}
  
  \caseFact{4} $\Theta ; \Delta \vdash I : S \gens \Phi_1$
  
  \caseFact{5} $\Theta ; \Delta \vDash \Phi_1$
  
  \caseText{By IH on (3), there are $e'$, $\Phi_2$, $\Gamma'$ such that}

  \caseFact{6} $|e'| = e$

  \caseFact{7} $\Theta ; \Delta \vDash \Phi_2$

  \caseFact{8} $\Psi ; \Theta ; \Delta ; \Omega ; \Gamma \vdash e' \checks \tau[I/i] \gens \Phi_2,\Gamma'$

  \caseText{Goals follow by AT-ExistI on (4) and (8)}
 }

 \jcase{10}{T-ExistE}{
  \jgivengoalThreeExists{
   \caseFact{1} $\Psi ; \Theta ; \Delta ; \Omega ; \Gamma_1,\Gamma_2\vdash \texttt{unpack } (i,x) = e_1 \texttt{ in } e_2 : \tau'$ 
   
   \caseFact{2} $\Psi ; \Theta ; \Delta ; \Omega ; \Gamma_1\vdash e_1 : \exists i : S.\tau$
   
   \caseFact{3} $\Psi ; \Theta, i : S ; \Delta ; \Omega ; \Gamma_2, x : \tau \vdash e_2 : \tau'$
  }{
   $e'$, $\Phi'$,$\Gamma'$  
  }{
   $|e'| = \texttt{unpack } (i,x) = e_1 \texttt{ in } e_2$  
  }{
   $\Theta ; \Delta \vDash \Phi'$  
  }{
   $\Psi ; \Theta ; \Delta ; \Omega ; \Gamma_1,\Gamma_2 \pvdash e' \checks \tau' \gens \Phi',\Gamma'$  
  }

  \caseText{By IH on (2), there are $e_1'$, $\Phi_1$, $\Gamma_1'$ such that}

  \caseFact{4} $|e_1'| = e_1$

  \caseFact{5} $\Theta ; \Delta \vDash \Phi_1$

  \caseFact{6} $\Psi ; \Theta ; \Delta ; \Omega ; \Gamma_1 \vdash e_1' \infers \exists i : S. \tau \gens \Phi_1,\Gamma_1'$

  \caseText{By \autoref{thm:admits-weaken} there are $e_1''$, $\Phi_1'$, $\Gamma_1''$ such that}

  \caseFact{7} $|e_1''| = |e_1'|$
  
  \caseFact{8} $\Theta ; \Delta \vDash \Phi_1'$

  \caseFact{9} $\Psi ; \Theta ; \Delta \vdash \Gamma_1'' \wknto \Gamma_2$

  \caseFact{10} $\Psi ; \Theta ; \Delta ; \Omega ; \Gamma_1,\Gamma_2 \vdash e_1'' \infers \exists i : S. \tau \gens \Phi_1',\Gamma_1''$

  \caseText{By IH on (3), there are $e_2'$, $\Phi_2$, $\Gamma_2'$ such that}

  \caseFact{11} $|e_2'| = e_2$

  \caseFact{12} $\Theta, i : S ; \Delta \vDash \Phi_2$

  \caseFact{13} $\Psi ; \Theta, i : S ; \Delta ; \Omega ; \Gamma_2,x : \tau \vdash e_2' \checks \tau' \gens \Phi_2,\Gamma_2'$.

  \caseText{By \autoref{thm:ctx-sub-subset-2} and \autoref{thm:admits-weaken} on (9) and (13) there are $e_2''$, $\Phi_2'$, $\Gamma_2''$ such that}

  \caseFact{14} $|e_2''| = |e_2'|$

  \caseFact{15} $\Theta, i : S ; \Delta \vDash \Phi_2'$

  \caseFact{16} $\Psi ; \Theta, i : S ; \Delta ; \Omega ; \Gamma_1'', x: \tau \vdash e_2'' \checks \tau' \gens \Phi_2',\Gamma_2''$

  \caseText{Goals follow by AT-ExistE on (10) and (16)}
 }

 \jcase{11}{T-Lam}{
  \jgivengoalThreeExists{
    \caseFact{1} $\Psi ; \Theta ; \Delta ; \Omega ; \Gamma\vdash \lambda x.e : \tau_1 \loli \tau_2$ 

    \caseFact{2} $\Psi ; \Theta ; \Delta ; \Omega ; \Gamma, x : \tau_1 \vdash e : \tau_2$
  }{
    $e'$, $\Phi'$,$\Gamma'$  
  }{
    $|e'| = \lambda x.e$
  }{
    $\Theta ; \Delta \vDash \Phi'$  
  }{
    $\Psi ; \Theta ; \Delta ; \Omega ; \Gamma \pvdash e' \checks \tau' \gens \Phi',\Gamma'$  
  }

  \caseText{By IH on (2) there are $e'$, $\Phi$, $\Gamma'$ so that}

  \caseFact{3} $|e'| = e$

  \caseFact{4} $\Theta ; \Delta \vDash \Phi$

  \caseFact{5} $\Psi ; \Theta ; \Delta ; \Omega ; \Gamma, x : \tau_1 \vdash e' \checks \tau_2 \gens \Phi,\Gamma'$

  \caseText{Goals are immediate by AT-Lam}
 }

 \jcase{12}{T-App}{
  \jgivengoalThreeExists{
    \caseFact{1} $\Psi ; \Theta ; \Delta ; \Omega ; \Gamma_1,\Gamma_2\vdash e_1 \; e_2 : \tau_2$ 

    \caseFact{2} $\Psi ; \Theta ; \Delta ; \Omega ; \Gamma_1 \vdash e_1 : \tau_1 \loli \tau_2$

    \caseFact{3} $\Psi ; \Theta ; \Delta ; \Omega ; \Gamma_2 \vdash e_2 : \tau_1$
  }{
    $e'$, $\Phi'$,$\Gamma'$  
  }{
    $|e'| = e_1 \; e_2$
  }{
    $\Theta ; \Delta \vDash \Phi'$  
  }{
    $\Psi ; \Theta ; \Delta ; \Omega ; \Gamma_1,\Gamma_2 \pvdash e' \infers \tau' \gens \Phi',\Gamma'$  
  }

  \caseText{By IH on (2) there are $e_1'$, $\Gamma_1'$, $\Phi_1$ such that} 

  \caseFact{4} $|e_1'| = e_1$

  \caseFact{5} $\Theta ; \Delta \vDash \Phi_1$

  \caseFact{6} $\Psi ; \Theta ; \Delta ; \Omega ; \Gamma_1 \vdash e_1' \infers \tau_1 \loli \tau_2 \gens \Phi_1,\Gamma_1'$

  \caseText{By \autoref{thm:admits-weaken} on (6), there are $e_1''$, $\Phi_1'$, $\Gamma_1''$ such that}

  \caseFact{7} $|e_1''| = |e_1'|$

  \caseFact{8} $\Theta ; \Delta \vDash \Phi_1'$

  \caseFact{9} $\Psi ; \Theta ; \Delta \vdash \Gamma_1'' \wknto \Gamma_2$

  \caseFact{10} $\Psi ; \Theta ; \Delta ; \Omega ; \Gamma_1,\Gamma_1 \vdash e_1'' \infers \tau_1 \loli \tau_2 \gens \Phi_1',\Gamma_1''$

  \caseText{By IH on (3), there are $e_2'$, $\Gamma_2'$, and $\Phi_2$ such that} 

  \caseFact{11} $|e_2'| = e_2$

  \caseFact{12} $\Theta ; \Delta \vDash \Phi_2$

  \caseFact{13} $\Psi ; \Theta ; \Delta ; \Omega ; \Gamma_2 \vdash e_2' \checks \tau_1 \gens \Phi_2,\Gamma_2'$

  \caseText{By \autoref{thm:admits-weaken} on (9) and (13), there are $e_2''$, $\Gamma_2''$, and $\Phi_2'$ such that} 

  \caseFact{14} $|e_2''| = |e_2'|$

  \caseFact{15} $\Theta ; \Delta \vDash \Phi_2'$

  \caseFact{16} $\Psi ; \Theta ; \Delta ; \Omega; \Gamma_1'' \vdash e_2'' \checks \tau_1 \gens \Phi_2',\Gamma_2''$

  \caseText{Goals follow by AT-App on (10) and (16)}
 }

 \jcase{13}{T-TensorI}{
  \jgivengoalThreeExists{
    \caseFact{1}  $\Psi ; \Theta ; \Delta ; \Omega ; \Gamma_1,\Gamma_2\vdash \angles{e_1,e_2} ; \tau_1 \otimes \tau_2$

    \caseFact{2} $\Psi ; \Theta ; \Delta ; \Omega ; \Gamma_1\vdash e_1 : \tau_1$ 

    \caseFact{3} $\Psi ; \Theta ; \Delta ; \Omega ; \Gamma_2\vdash e_2 : \tau_2$

  }{
    $e'$, $\Phi'$,$\Gamma'$  
  }{
    $|e'| = \angles{e_1,e_2}$
  }{
    $\Theta ; \Delta \vDash \Phi'$  
  }{
    $\Psi ; \Theta ; \Delta ; \Omega ; \Gamma_1,\Gamma_2 \pvdash e' \checks \tau' \gens \Phi',\Gamma'$  
  }

  \caseText{By IH on (2), there are $e_1'$, $\Phi_1$, $\Gamma_1'$ such that}

  \caseFact{4} $|e_1'| = e_1$

  \caseFact{5} $\Theta ; \Delta \vDash \Phi_1$

  \caseFact{6} $\Psi ; \Theta ; \Delta ; \Omega ; \Gamma_1 \vdash e_1' \checks \tau_1 \gens \Phi_1,\Gamma_1'$

  \caseText{By \autoref{thm:admits-weaken} on (6), there are $e_1''$, $\Phi_1'$, $\Gamma_1''$ such that}

  \caseFact{7} $|e_1''| = |e_1'|$

  \caseFact{8} $\Theta ; \Delta \vDash \Phi_1'$

  \caseFact{9} $\Psi ; \Theta ; \Delta \vdash \Gamma_1'' \wknto (\Gamma_1,\Gamma_2) \setminus \Gamma_1$

  \caseFact{10} $\Psi ; \Theta ; \Delta ;\Omega ; \Gamma_1,\Gamma_2 \vdash e_1'' \checks \tau_2 \gens \Phi_1',\Gamma_1''$. 

  \caseText{By IH on (3), there are $e_2'$, $\Phi_2$, $\Gamma_2'$ such that}

  \caseFact{11} $|e_2'| = e_2$

  \caseFact{12} $\Theta ; \Delta \vDash \Phi_2$

  \caseFact{13} $\Psi ; \Theta ; \Delta ; \Omega ; \Gamma_2 \vdash e_2' \checks \tau_2 \gens \Phi_2,\Gamma_2'$

  \caseText{But by \autoref{thm:admits-weaken} on (13) and (9), there are $e_2''$, $\Phi_2'$, $\Gamma_2''$ such that }

  \caseFact{14} $|e_2''| = |e_2'|$

  \caseFact{15} $\Theta ; \Delta \vDash \Phi_2'$

  \caseFact{16} $\Psi ; \Theta ; \Delta ; \Omega ; \Gamma_1'' \vdash e_2'' \checks \tau_2 \gens \Phi_2', \Gamma_2''$

  \caseText{Goals follow by AT-TensorI on (16)}
 }

 \jcase{14}{T-TensorE}{
  \jgivengoalThreeExists{
    \caseFact{1}  $\Psi ; \Theta ; \Delta ; \Omega ; \Gamma_1,\Gamma_2\vdash \texttt{let } \angles{x,y} = e_1 \texttt{ in } e_2 : \tau'$

    \caseFact{2} $\Psi ; \Theta ; \Delta ; \Omega ; \Gamma_1\vdash e_1 : \tau_1 \otimes \tau_2$

    \caseFact{3} $\Psi ; \Theta ; \Delta ; \Omega ; \Gamma_2,x : \tau_1, y : \tau_2\vdash e_2 : \tau'$

  }{
    $e'$, $\Phi'$,$\Gamma'$  
  }{
    $|e'| = \texttt{let } \angles{x,y} = e_1 \texttt{ in } e_2$
  }{
    $\Theta ; \Delta \vDash \Phi'$  
  }{
    $\Psi ; \Theta ; \Delta ; \Omega ; \Gamma_1,\Gamma_2 \pvdash e' \checks \tau' \gens \Phi',\Gamma'$  
  }

 \caseText{By IH on (2), there are $e_1'$, $\Phi_1$, $\Gamma_1'$ such that}

 \caseFact{4} $|e_1'| = e_1$

 \caseFact{5} $\Theta ; \Delta \vDash \Phi_1$

 \caseFact{6} $\Psi ; \Theta ; \Delta ; \Omega ; \Gamma_1 \vdash e_1' \infers \tau_1 \otimes \tau_2 \gens \Phi_1,\Gamma_1'$

 \caseText{By \autoref{thm:admits-weaken} that there are $e_1''$, $\Phi_1'$, $\Gamma_1''$ such that}

 \caseFact{7} $|e_1''| = |e_1'|$

 \caseFact{8} $\Theta ; \Delta \vDash \Phi_1'$

 \caseFact{9} $\Psi ; \Theta ; \Delta \vdash \Gamma_1'' \wknto (\Gamma_1,\Gamma_2 \setminus \Gamma_1)$

 \caseFact{10} $\Psi ; \Theta ; \Delta ; \Omega ; \Gamma_1,\Gamma_2 \vdash e_1'' \infers \tau_1 \otimes \tau_2 \gens \Phi_1',\Gamma_1''$.

 \caseText{By IH on (3), there are $e_2'$, $\Phi_2$, $\Gamma_2'$ such that}

 \caseFact{11} $|e_2'| = e_2$

 \caseFact{12} $\Theta ; \Delta \vDash \Phi_2$

 \caseFact{13} $\Psi ; \Theta ; \Delta ; \Omega ; \Gamma_2,x : \tau_1, y : \tau_2 \vdash e_2' \checks \tau' \gens \Phi_2,\Gamma_2'$

 \caseText{By \autoref{thm:ctx-sub-subset-2} on (9) \autoref{thm:admits-weaken} on (13), there are $e_2''$, $\Phi_2'$, $\Gamma_2''$ such that}

 \caseFact{14} $|e_2''| = |e_2'|$

 \caseFact{15} $\Theta ; \Delta \vDash \Phi_2'$

 \caseFact{16} $\Psi ; \Theta ; \Delta, \Gamma_1'', x : \tau_1, y : \tau_2 \vdash e_2'' \checks \tau' \gens \Phi_2',\Gamma_2''$

 \caseText{Goals follow by AT-TensorE on (10) and (16)}
 }

 \jcase{15}{T-WithI}{
  \jgivengoalThreeExists{
    \caseFact{1} $\Psi ; \Theta ; \Delta ; \Omega ; \Gamma \vdash (e_1,e_2) : \tau_1 \amp \tau_2$

    \caseFact{2} $\Psi ; \Theta ; \Delta ; \Omega ; \Gamma \vdash e_1 : \tau_1$

    \caseFact{3} $\Psi ; \Theta ; \Delta ; \Omega ; \Gamma \vdash e_2 : \tau_2$.

  }{
    $e'$, $\Phi'$,$\Gamma'$  
  }{
    $|e'| = (e_1,e_2)$
  }{
    $\Theta ; \Delta \vDash \Phi'$  
  }{
    $\Psi ; \Theta ; \Delta ; \Omega ; \Gamma \pvdash e' \checks \tau_1 \amp \tau_2 \gens \Phi',\Gamma'$  
  }

  \caseText{By IH on (2), we have $e_1'$, $\Phi_1$, $\Gamma_1$ such that}

  \caseFact{4} $|e_1'| = e_1$

  \caseFact{5} $\Theta ; \Delta \vDash \Phi_1$

  \caseFact{6} $\Psi ; \Theta ; \Delta ; \Omega ; \Gamma \vdash e_1' : \tau_1 \gens \Phi_1,\Gamma_1$.

  \caseText{By IH on (3), we have $e_2'$, $\Phi_2$, $\Gamma_2$ such that}

  \caseFact{4} $|e_2'| = e_2$

  \caseFact{5} $\Theta ; \Delta \vDash \Phi_2$

  \caseFact{6} $\Psi ; \Theta ; \Delta ; \Omega ; \Gamma \vdash e_2' : \tau_2 \gens \Phi_2,\Gamma_2$.

  \caseText{Goals follow by AT-WithI}
 }

 \jcase{16}{T-Fst}{
  \jgivengoalThreeExists{
    \caseFact{1}  $\Psi ; \Theta ; \Delta ; \Omega ; \Gamma\vdash \texttt{fst}(e) : \tau_1$

    \caseFact{2} $\Psi ; \Theta ; \Delta ; \Omega ; \Gamma \vdash e : \tau_1 \amp \tau_2$ 

  }{
    $e'$, $\Phi'$,$\Gamma'$  
  }{
    $|e'| = \texttt{fst}(e)$
  }{
    $\Theta ; \Delta \vDash \Phi'$  
  }{
    $\Psi ; \Theta ; \Delta ; \Omega ; \Gamma \pvdash e' \infers \tau_1 \gens \Phi',\Gamma'$  
  }

  \caseText{By IH on (2), there are $e'$, $\Phi$, $\Gamma'$ such that}

  \caseFact{3} $|e'| = e$

  \caseFact{4} $\Theta ; \Delta \vDash \Phi$

  \caseFact{5} $\Psi ; \Theta ; \Delta ; \Omega ; \Gamma \vdash e' \infers \tau_1 \amp \tau_2 \gens \Phi,\Gamma'$

  \caseText{Goals follow by AT-Fst}
 }

 \jcase{17}{T-Snd}{Identical to case (16)}

 \jcase{18}{T-Inl}{
  \jgivengoalThreeExists{
    \caseFact{1}  $\Psi ; \Theta ; \Delta ; \Omega ; \Gamma vdash \texttt{inl}(e) : \tau_1 \oplus \tau_2$

    \caseFact{2} $\Psi ; \Theta ; \Delta ; \Omega ; \Gamma\vdash e : \tau_1$ 

  }{
    $e'$, $\Phi'$,$\Gamma'$  
  }{
    $|e'| = \texttt{inl}(e)$
  }{
    $\Theta ; \Delta \vDash \Phi'$  
  }{
    $\Psi ; \Theta ; \Delta ; \Omega ; \Gamma \pvdash e' \checks \tau_1 \oplus \tau_2 \gens \Phi',\Gamma'$  
  }

  \caseText{By IH on (2), there are $e'$, $\Phi$, $\Gamma'$ such that}

  \caseFact{3} $|e'| = e$

  \caseFact{4} $\Theta ; \Delta \vDash \Phi$

  \caseFact{5} $\Psi ; \Theta ; \Delta ; \Omega ; \Gamma \vdash e' \checks \tau_1 \gens \Phi,\Gamma'$

  \caseText{Goals follow by AT-Inl}
 }

 \jcase{19}{T-Snd}{Identical to case (18)}

 \jcase{20}{T-Case}{
  \jgivengoalThreeExists{
    \caseFact{1} $\Psi ; \Theta ; \Delta ; \Omega ; \Gamma_1,\Gamma_2 \vdash \texttt{case}(e,x.e_1,y.e_2) : \tau$

    \caseFact{2} $\Psi ; \Theta ; \Delta ; \Omega ; \Gamma_1 \vdash e : \tau_1 \oplus \tau_2$

    \caseFact{3} $\Psi ; \Theta ; \Delta ; \Omega ; \Gamma_2, x: \tau_1 \vdash e_1 : \tau$

    \caseFact{4} $\Psi ; \Theta ; \Delta ; \Omega ; \Gamma_2, y: \tau_2 \vdash e_2 : \tau$

  }{
    $e'$, $\Phi'$,$\Gamma'$  
  }{
    $|e'| = \texttt{case}(e,x.e_1,y.e_2)$
  }{
    $\Theta ; \Delta \vDash \Phi'$  
  }{
    $\Psi ; \Theta ; \Delta ; \Omega ; \Gamma_1,\Gamma_2 \pvdash e' \checks \tau \gens \Phi',\Gamma'$  
  }

  \caseText{By IH on (2), there are $e'$, $\Phi$, $\Gamma_1'$}

  \caseFact{5} $|e'| = e$

  \caseFact{6} $\Theta ; \Delta \vDash \Phi$

  \caseFact{7} $\Psi ; \Theta ; \Delta ; \Omega ; \Gamma_1 \vdash e' \infers \tau_1 \oplus \tau_2 \gens \Phi,\Gamma_1'$

  \caseText{By \autoref{thm:admits-weaken} there are $e''$, $\Phi'$, $\Gamma_1''$ such that}

  \caseFact{8} $|e''| = |e'|$

  \caseFact{9} $\Theta ; \Delta \vDash \Phi'$

  \caseFact{10} $\Psi ; \Theta ; \Delta \vdash \Gamma_1'' \wknto (\Gamma_1,\Gamma_2) \setminus \Gamma_1$

  \caseFact{11} $\Psi ; \Theta ; \Delta ; \Omega ; \Gamma_1,\Gamma_2 \vdash e'' \infers \tau_1 \oplus \tau_2 \gens \Phi',\Gamma_1''$.

  \caseText{By IH on (3), there are $e_1'$, $\Phi_1$, $\Gamma_2'$ such that}

  \caseFact{12} $|e_1'| = e_1$

  \caseFact{13} $\Theta ; \Delta \vDash \Phi_1$

  \caseFact{14} $\Psi ; \Theta ; \Delta ; \Omega ; \Gamma_2, x : \tau_1 \vdash e_1' \checks \tau \gens \Phi_1,\Gamma_2'$

  \caseText{By \autoref{thm:ctx-sub-subset-2} and \autoref{thm:admits-weaken} on (10) and (14), there are $e_1''$, $\Phi_1'$, $\Gamma_2''$ such that}

  \caseFact{15} $|e_1''| = |e_1'|$

  \caseFact{16} $\Theta ; \Delta \vDash \Phi_1'$

  \caseFact{17} $\Psi ; \Theta ; \Delta ; \Omega ; \Gamma_1'', x : \tau_1 \vdash e_1'' \checks \tau \gens \Phi_1',\Gamma_2''$

  \caseText{By IH on (4), we have $e_2'$, $\Phi_2$, $\Gamma_3'$ such that}
  
  \caseFact{18} $|e_2'| = e_2$
  
  \caseFact{19} $\Theta ; \Delta \vDash \Phi_2$

  \caseFact{20} $\Psi ; \Theta ; \Delta ; \Omega ; \Gamma_2, y : \tau_2 \vdash e_2' \checks \tau \gens \Phi_2,\Gamma_3'$

  \caseText{By \autoref{thm:ctx-sub-subset-2} and \autoref{thm:admits-weaken} on (10) and (20), there are $e_2''$, $\Phi_2'$, $\Gamma_3''$ such that}

  \caseFact{21} $|e_2''| = |e_2'|$

  \caseFact{22} $\Theta ; \Delta \vDash \Phi_2'$

  \caseFact{23} $\Psi ; \Theta ; \Delta ; \Omega ; \Gamma_1'', y : \tau_2 \vdash e_2'' \checks \tau \gens \Phi_2',\Gamma_3''$

  \caseText{Goals follow by AT-Case on (11), (17), and (23)}
 }

 \jcase{21}{T-ExpI}{
  \jgivengoalThreeExists{
    \caseFact{1}  $\Psi ; \Theta ; \Delta ; \Omega ; \Gamma\vdash !e : !\tau$

    \caseFact{2} $\Psi ; \Theta ; \Delta ; \Omega ; \cdot \vdash e : \tau$ 

  }{
    $e'$, $\Phi'$,$\Gamma'$  
  }{
    $|e'| = !e$
  }{
    $\Theta ; \Delta \vDash \Phi'$  
  }{
    $\Psi ; \Theta ; \Delta ; \Omega ; \Gamma \pvdash e' \checks \tau_1 \gens \Phi',\Gamma'$  
  }

  \caseText{By IH on (2), there are $e'$, $\Phi$, $\Gamma'$ such that}

  \caseFact{3} $|e'| = e$

  \caseFact{4} $\Theta ; \Delta \vDash \Phi$

  \caseFact{5} $\Psi ; \Theta ; \Delta ; \Omega; \cdot \vdash e' \checks \tau \gens \Phi,\Gamma'$

  \caseText{Goals follow by AT-ExpI on (5)}
 }

 \jcase{22}{T-ExpE}{
  \jgivengoalThreeExists{
    \caseFact{1} $\Psi ; \Theta ; \Delta ; \Omega ; \Gamma_1,\Gamma_2 \vdash \texttt{let } !x = e_1 \texttt{ in } e_2 : \tau'$

    \caseFact{2} $\Psi ; \Theta ; \Delta ; \Omega ; \Gamma_1 \vdash e_1  : !\tau$

    \caseFact{3} $\Psi ; \Theta ; \Delta ; \Omega, x : \tau ; \Gamma_2 \vdash e_2 : \tau'$.

  }{
    $e'$, $\Phi'$,$\Gamma'$  
  }{
    $|e'| = \texttt{let } !x = e_1 \texttt{ in } e_2$
  }{
    $\Theta ; \Delta \vDash \Phi'$  
  }{
    $\Psi ; \Theta ; \Delta ; \Omega ; \Gamma_1,\Gamma_2 \pvdash e' \checks \tau' \gens \Phi',\Gamma'$  
  }

  \caseText{By IH on (2), there are $e_1'$, $\Phi_1$, $\Gamma_1'$ such that}
  
  \caseFact{4} $|e_1'| = e_1$

  \caseFact{5} $\Theta ; \Delta \vDash \Phi_1$

  \caseFact{6} $\Psi ; \Theta ; \Delta ; \Omega ; \Gamma_1 \vdash e_1' \infers !\tau \gens \Phi_1,\Gamma_1'$.

  \caseText{By \autoref{thm:admits-weaken}, there are $e_1''$, $\Phi_1'$, $\Gamma_1''$ such that}o
   
  \caseFact{7} $|e_1''| = |e_1'|$

  \caseFact{8} $\Theta ; \Delta \vDash \Phi_1'$

  \caseFact{9} $\Psi ; \Theta ; \Delta \vdash \Gamma_1'' \wknto (\Gamma_1,\Gamma_2) \setminus \Gamma_1$

  \caseFact{10} $\Psi ; \Theta ; \Delta ; \Omega; \Gamma_1,\Gamma_2 \vdash e_1'' \infers !\tau \gens \Phi_1',\Gamma_1''$

  \caseText{By IH on (3), there are $e_2'$, $\Phi_2$, $\Gamma_2'$ such that}

  \caseFact{11} $|e_2'| = e_2$

  \caseFact{12} $\Theta ; \Delta \vDash \Phi_2$

  \caseFact{13} $\Psi ; \Theta ; \Delta ; \Omega, x: \tau ; \Gamma_2 \vdash e_2' \checks \tau' \gens \Phi_2,\Gamma_2'$

  \caseText{By \autoref{admits-weaken} on (9) and (13), there are $e_2''$, $\Phi_2'$, $\Gamma_2''$ such that}

  \caseFact{14} $|e_2''| = |e_2|$

  \caseFact{15} $\Theta ; \Delta \vDash \Phi_2'$

  \caseFact{16} $\Psi ; \Theta ; \Delta ; \Omega, x : \tau ; \Gamma_1'' \vdash e_2'' \checks \tau' \gens \Phi_2', \Gamma_2''$

  \caseText{Goals follow by AT-ExpE on (10) and (16)}
 }

 \jcase{23}{T-TAbs}{
  \jgivengoalThreeExists{
    \caseFact{1} $\Psi ; \Theta ; \Delta ; \Omega ; \Gamma \vdash \Lambda \alpha. e : \forall \alpha : K.\tau$

    \caseFact{2} $\Psi, \alpha : K ; \Theta ; \Delta ; \Omega ; \Gamma \vdash e : \tau$
  }{
    $e'$, $\Phi'$,$\Gamma'$  
  }{
    $|e'| = \Lambda \alpha . e$
  }{
    $\Theta ; \Delta \vDash \Phi'$  
  }{
    $\Psi ; \Theta ; \Delta ; \Omega ; \Gamma \pvdash e' \checks \forall \alpha : K. \tau \gens \Phi',\Gamma'$  
  }

  \caseText{By IH on (2), there are $e'$,$\Phi$, $\Gamma'$ such that}

  \caseFact{3} $|e'| = e$

  \caseFact{4} $\Theta ; \Delta \vDash \Phi$

  \caseFact{5} $\Psi, \alpha : K ; \Theta ; \Delta ; \Omega ; \Gamma \vdash e' \checks \tau \gens \Phi,\Gamma'$

  \caseText{Goals follow by AT-TAbs}
 }

 \jcase{24}{T-TApp}{
  \jgivengoalThreeExists{
    \caseFact{1} $\Psi ; \Theta ; \Delta ; \Omega ; \Gamma \vdash e [\tau'] : \tau[\tau'/\alpha]$

    \caseFact{2} $\Psi ; \Theta ; \Delta ; \Omega ; \Gamma \vdash e : \forall \alpha : K.\tau$

    \caseFact{3} $\Psi ; \Theta ; \Delta \vdash \tau' : K$.
  }{
    $e'$, $\Phi'$,$\Gamma'$  
  }{
    $|e'| = e \, [\tau]$
  }{
    $\Theta ; \Delta \vDash \Phi'$  
  }{
    $\Psi ; \Theta ; \Delta ; \Omega ; \Gamma \pvdash e' \infers \tau[\tau'/\alpha] \gens \Phi',\Gamma'$  
  }

  \caseText{By IH on (2), there are $e'$, $\Phi$, $\Gamma'$ such that}

  \caseFact{4} $|e'| = e$

  \caseFact{5} $\Theta ; \Delta \vDash \Phi$

  \caseFact{6} $\Psi ; \Theta ; \Delta ; \Omega ; \Gamma \vdash e' \infers \forall \alpha : K.\tau \gens \Phi,\Gamma'$

  \caseText{By \autoref{thm:kind-compl} on (3), there is some $\Phi'$ such that}

  \caseFact{7} $\Theta ; \Delta \vDash \Phi'$

  \caseFact{8} $\Psi ; \Theta ; \Delta \vdash \tau' : K \gens \Phi'$

  \caseText{Goals follow by AT-TApp on (6) and (8)}
 }

\jcase{25}{T-IAbs}{
  \jgivengoalThreeExists{
    \caseFact{1}  $\Psi ; \Theta ; \Delta ; \Omega ; \Gamma \vdash \Lambda i. e : \forall i : S. \tau$

    \caseFact{2} $\Psi ; \Theta, i : S ; \Delta ; \Omega ; \Gamma \vdash e : \tau$
  }{
    $e'$, $\Phi'$,$\Gamma'$  
  }{
    $|e'| = \Lambda \i . e$
  }{
    $\Theta ; \Delta \vDash \Phi'$  
  }{
    $\Psi ; \Theta ; \Delta ; \Omega ; \Gamma \pvdash e' \checks \forall \alpha : K. \tau \gens \Phi',\Gamma'$  
  }

  \caseText{By IH on (2), there are $e'$, $\Phi$, $\Gamma'$ such that}

  \caseFact{3} $|e'| = e$

  \caseFact{4} $\Theta, i : S ; \Delta \vDash \Phi$

  \caseFact{5} $\Psi ; \Theta, i : S ; \Delta ; \Omega; \Gamma \vdash e' \checks \tau \gens \Phi,\Gamma'$

  \caseText{Goals follow by AT-IAbs on (5)}
 }

\jcase{26}{T-IApp}{
  \jgivengoalThreeExists{
    \caseFact{1} $\Psi ; \Theta ; \Delta ; \Omega ; \Gamma \vdash e [I] : \tau[I/i]$

    \caseFact{2} $\Psi ; \Theta ; \Delta ; \Omega ; \Gamma \vdash e : \forall i : S.\tau$

    \caseFact{3} $\Theta ; \Delta \vdash I : S$
  }{
    $e'$, $\Phi'$,$\Gamma'$  
  }{
    $|e'| = e \, [I]$
  }{
    $\Theta ; \Delta \vDash \Phi'$  
  }{
    $\Psi ; \Theta ; \Delta ; \Omega ; \Gamma \pvdash e' \infers \tau[I/i] \gens \Phi',\Gamma'$  
  }

  \caseText{By IH on (2), there are $e'$, $\Phi$, $\Gamma'$ such that}

  \caseFact{4} $|e'| = e$
  
  \caseFact{5} $\Theta ; \Delta \vDash \Phi$

  \caseFact{6} $\Psi ; \Theta ; \Delta ; \Omega; \Gamma \vdash e \infers \forall i : S. \tau \gens \Phi,\Gamma'$

  \caseText{By \autoref{thm:sort-compl} on (3), there is some $\Phi'$ such that}

  \caseFact{7} $\Theta ; \Delta \vDash \Phi'$

  \caseFact{8} $\Theta ; \Delta \vdash I : S \gens \Phi'$

  \caseText{Goals follow by AT-IApp on (6) and (8)}
 }

 \jcase{27}{T-Fix}{
  \jgivengoalThreeExists{
    \caseFact{1} $\Psi ; \Theta ; \Delta ; \Omega ; \Gamma \vdash \texttt{fix}\; x.e : \tau$

    \caseFact{2} $\Psi ; \Theta ; \Delta ; \Omega, x : \tau ; \cdot \vdash e : \tau$
  }{
    $e'$, $\Phi'$,$\Gamma'$  
  }{
    $|e'| = \texttt{fix}\; x.e$
  }{
    $\Theta ; \Delta \vDash \Phi'$  
  }{
    $\Psi ; \Theta ; \Delta ; \Omega ; \Gamma \pvdash e' \checks \tau \gens \Phi',\Gamma'$  
  }

  \caseText{By IH on (2), there are $e'$, $\Phi$, $\Gamma'$ such that}

  \caseFact{3} $|e'| = e$

  \caseFact{4} $\Theta ; \Delta \vDash \Phi$

  \caseFact{5} $\Psi ; \Theta ; \Delta ; \Omega, x : \tau ; \cdot \vdash e' \checks \tau \gens \Phi,\Gamma'$

  \caseText{Goals follow by AT-Fix on (5)}

 }

\jcase{28}{T-Ret}{
  \jgivengoalThreeExists{
    \caseFact{1} $\Psi ; \Theta ; \Delta ; \Omega ; \Gamma \vdash \texttt{ret}\; e : \M \, (I,\vec{p}) \, \tau$ by way of

    \caseFact{2} $\Psi ; \Theta ; \Delta ; \Omega ; \Gamma \vdash e : \tau$.
  }{
    $e'$, $\Phi'$,$\Gamma'$  
  }{
    $|e'| = \texttt{ret}\; e $
  }{
    $\Theta ; \Delta \vDash \Phi'$  
  }{
    $\Psi ; \Theta ; \Delta ; \Omega ; \Gamma \pvdash e' \checks \tau \gens \Phi',\Gamma'$  
  }

  \caseText{By IH on (2), there are $e'$, $\Phi$, $\Gamma'$ such that}

  \caseFact{3} $|e'| = e$

  \caseFact{4} $\Theta ; \Delta \vDash \Phi$

  \caseFact{5} $\Psi ; \Theta ; \Delta ; \Omega ; \Gamma \vdash e' \checks M \, (I,\vec{p}) \, \tau \gens \Phi,\Gamma'$

  \caseText{Goals follow by AT-Ret on (5)}
 }

 \jcase{29}{T-Bind}{
  \jgivengoalThreeExists{
    \caseFact{1} $\Psi ; \Theta ; \Delta ; \Omega ; \Gamma_1,\Gamma_2 \vdash \texttt{bind } x = e_1 \texttt{ in } e_2 : \M \, \phi(I,\vec{p} + \vec{q})\, \tau_2$

    \caseFact{2} $\Psi ; \Theta ; \Delta ; \Omega ; \Gamma_1 \vdash e_1 : \M \, \phi(I,\vec{p})\, \tau_1$

    \caseFact{3} $\Psi ; \Theta; \Delta ; \Omega ; \Gamma_2, x:\tau_1 \vdash e_2 : \M \, \phi(I,\vec{q})\, \tau_2$

  }{
    $e'$, $\Phi'$,$\Gamma'$  
  }{
    $|e'| = \texttt{bind } x = e_1 \texttt{ in } e_2$
  }{
    $\Theta ; \Delta \vDash \Phi'$  
  }{
    $\Psi ; \Theta ; \Delta ; \Omega ; \Gamma_1,\Gamma_2 \pvdash e' \checks \M \, \phi(I,\vec{p} + \vec{q})\, \tau_2 \gens \Phi',\Gamma'$  
  }

  \caseText{By IH on (2), there are $e_1'$, $\Phi_1$, $\Gamma_1'$ such that}

  \caseFact{4} $|e_1'| = e_1$

  \caseFact{5} $\Theta ; \Delta \vDash \Phi_1$

  \caseFact{6} $\Psi  ; \Theta ; \Delta ; \Omega ; \Gamma_1 \vdash e_1' \checks \M \, \phi(I,\vec{p}) \, \tau_1 \gens \Phi_1,\Gamma_1'$

  \caseText{By \autoref{thm:admits-weaken} on (6), there are $e_1''$, $\Gamma_1''$, $\Phi_1'$ such that}

  \caseFact{7} $|e_1''| = |e_1'|$

  \caseFact{8} $\Theta ; \Delta \vDash \Phi_1'$

  \caseFact{9} $\Psi ; \Theta ; \Delta \vdash \Gamma_1'' \wknto (\Gamma_1,\Gamma_2) \setminus \Gamma_1$

  \caseFact{10} $\Psi ; \Theta ; \Delta ; \Omega ; \Gamma_1,\Gamma_2 \vdash e_1'' \checks \M \, \phi(I,\vec{p}) \, \tau_1 \gens \Phi_1',\Gamma_1''$

  \caseText{By IH on (3), there are $e_2'$, $\Phi_2$, $\Gamma_2'$ such that}
  
  \caseFact{11} $|e_2'| = e_2$

  \caseFact{12} $\Theta ; \Delta \vDash \Phi_2$

  \caseFact{13} $\Psi ; \Theta; \Delta ; \Omega ; \Gamma_2, x:\tau_1 \vdash e_2' \checks \M \, \phi(I,\vec{q})\, \tau_2 \gens \Phi_2,\Gamma_2'$
  
  \caseText{By \autoref{thm:admits-weaken} on (9) and (11), there are $e_2''$, $\Gamma_2''$, $\Phi_2'$ such that}

  \caseFact{14} $|e_2''| = |e_2'|$

  \caseFact{15} $\Theta ; \Delta \vDash \Phi_2'$

  \caseFact{16} $\Psi ; \Theta ; \Delta ; \Omega ; \Gamma_1'', x : \tau_1 \vdash e_2'' \infers \M \, \phi(I,\vec{q}) \, \tau_2 \gens \Phi_2',\Gamma_2''$

  \caseText{By \autoref{thm:subty-refl}, there is some $\Phi_3$ such that $\Theta ; \Delta \vDash \Phi_3$ and}

  \caseFact{17} $\Psi ; \Theta ; \Delta \vdash \tau_2 \subty \tau_2 : \star \gens \Phi_3$.

  \caseText{By AS-Monad on (17)}

  \caseFact{18} $\Psi ; \Theta ; \Delta \vdash \M \, \phi(I,\vec{q}) \, \tau_2 \subty \M \, \phi(I,(\vec{p} + \vec{q}) - \vec{p}) \, \tau_2 : \star \gens \Phi_3 \wedge (I = I) \wedge (\vec{q} \leq (\vec{p} + \vec{q}) - \vec{p})$

  \caseText{By AT-Sub on (16) and (18)}

  \caseFact{19} $\Psi ; \Theta ; \Delta ; \Omega ; \Gamma_1'', x : \tau_1 \vdash e_2'' \checks \M \, \phi(I,(\vec{p} + \vec{q}) - \vec{p}) \, \tau_2 \gens \Phi_2' \wedge  \Phi_3 \wedge (I = I) \wedge (\vec{q} \leq (\vec{p} + \vec{q}) - \vec{p}),\Gamma_2''$.

  \caseText{Goals follow by AT-Bind on (10) and (19)}
 }

 \jcase{30}{T-Tick}{
  \jgivengoalThreeExists{
    \caseFact{1} $\Psi ; \Theta ; \Delta ; \Omega ; \Gamma \vdash \texttt{tick}[I|\vec{p}] : \M \, \phi(I,\vec{p})\, 1$

    \caseFact{2} $\Theta ; \Delta \vdash I : \N$

    \caseFact{3} $\Theta ; \Delta \vdash \vec{p} : \potvec$
  }{
    $e'$, $\Phi'$,$\Gamma'$  
  }{
    $|e'| = \texttt{tick}[I|\vec{p}]$
  }{
    $\Theta ; \Delta \vDash \Phi'$
  }{
    $\Psi ; \Theta ; \Delta ; \Omega ; \Gamma\pvdash e' \infers \M \, \phi(I,\vec{p})\, 1 \gens \Phi',\Gamma'$  
  }

  \caseText{By \autoref{thm:sort-compl} on (2), there is some $\Phi_1$ such that}

  \caseFact{4} $\Theta ; \Delta \vDash \Phi_1$

  \caseFact{5} $\Theta ; \Delta \vdash I : \mathbb{N} \gens \Phi_1$

  \caseText{By \autoref{thm:sort-compl} on (3), there is some $\Phi_2$ such that}

  \caseFact{6} $\Theta ; \Delta \vDash \Phi_2$, and

  \caseFact{7} $\Theta ; \Delta \vdash \vec{p} : \potvec$.

  \caseText{Goals follow by AT-Tick on (5) and (7)}
 }

 \jcase{31}{T-Release}{
  \jgivengoalThreeExists{
    \caseFact{1} $\Psi ; \Theta ; \Delta ; \Omega ; \Gamma_1,\Gamma_2 \vdash \texttt{release } x = e_1 \texttt{ in }e_2 : \M \, (I,\vec{p}) \, \tau_2$

    \caseFact{2} $\Psi ; \Theta ; \Delta ; \Omega ; \Gamma_1 \vdash e_1 : [I | \vec{q}] \tau_1$

    \caseFact{3} $\Psi ; \Theta ; \Delta ; \Omega ; \Gamma_2, x : \tau \vdash e_2 : \M \, \phi(I,\vec{p} + \vec{q}) \, \tau_2$

  }{
    $e'$, $\Phi'$,$\Gamma'$  
  }{
    $|e'| = \texttt{release } x = e_1 \texttt{ in }e_2$
  }{
    $\Theta ; \Delta \vDash \Phi'$  
  }{
    $\Psi ; \Theta ; \Delta ; \Omega ; \Gamma_1,\Gamma_2 \pvdash e' \checks \M \, (I,\vec{p}) \, \tau_2 \gens \Phi',\Gamma'$  
  }

  \caseText{By IH on (2), there are $e_1'$, $\Phi_1$, $\Gamma_1'$ such that}

  \caseFact{4} $|e_1'| = e_1$

  \caseFact{5} $\Theta ; \Delta \vDash \Phi_1$

  \caseFact{6} $\Psi ; \Theta ; \Delta ; \Omega ; \Gamma_1 \vdash e_1' \infers [I|\vec{q}] \tau_1 \gens \Phi_1,\Gamma_1'$

  \caseText{By \autoref{thm:admits-weaken} on (6) there are $e_1''$, $\Phi_1'$, $\Gamma_1''$ such that}

  \caseFact{7} $|e_1''| = |e_1'|$

  \caseFact{8} $\Theta ; \Delta \vDash \Phi_1'$

  \caseFact{9} $\Psi ; \Theta ; \Delta \vdash \Gamma_1'' \wknto (\Gamma_1,\Gamma_2) \setminus \Gamma_1$

  \caseFact{10} $\Psi ; \Theta ; \Delta ; \Omega ; \Gamma_1,\Gamma_2 \vdash e_1'' \infers [I|\vec{q}] \tau_1 \gens \Phi_1',\Gamma_1''$

  \caseText{By IH on (3), there are $e_2'$, $\Phi_2$, $\Gamma_2'$ such that}

  \caseFact{11} $|e_2'| = e_2$

  \caseFact{12} $\Theta ; \Delta \vDash \Phi_2$

  \caseFact{13} $\Psi ; \Theta ; \Delta ; \Omega ; \Gamma_2, x : \tau \vdash e_2' \checks \M \, \phi(I,\vec{p} + \vec{q}) \, \tau_2 \gens \Phi_2,\Gamma_2'$

  \caseText{By \autoref{thm:admits-weaken} on (9) and (13), there are $e_2''$, $\Phi_2'$, $\Gamma_2''$ such that}

  \caseFact{14} $|e_2''| = |e_2'|$

  \caseFact{15} $\Theta ; \Delta \vDash \Phi_2'$

  \caseFact{16} $\Psi ; \Theta ; \Delta ; \Omega ; \Gamma_1'', x : \tau \vdash e_2'' \checks \M \, \phi(I,\vec{p} + \vec{q}) \, \tau \gens \Phi_2',\Gamma_2''$

  \caseText{Goals follow by AT-Release on (10) and (16)}

 }

 \jcase{32}{T-Store}{
  \jgivengoalThreeExists{
    \caseFact{1} $\Psi ; \Theta ; \Delta ; \Omega ; \Gamma \vdash \texttt{store}[I|\vec{p}](e) : \M \, \phi(I,\vec{p}) \, ([I| \vec{p}] \, \tau)$

    \caseFact{2} $\Theta ; \Delta \vdash I : \mathbb{N}$

    \caseFact{3} $\Theta ; \Delta \vdash \vec{p} : \potvec$

    \caseFact{4} $\Psi ; \Theta ; \Delta ; \Omega ; \Gamma \vdash e : \tau$

  }{
    $e'$, $\Phi'$,$\Gamma'$  
  }{
    $|e'| = \texttt{store}[I|\vec{p}](e) $
  }{
    $\Theta ; \Delta \vDash \Phi'$  
  }{
    $\Psi ; \Theta ; \Delta ; \Omega ; \Gamma \pvdash e' \checks  \M \, \phi(I,\vec{p}) \, ([I| \vec{p}] \, \tau) \gens \Phi',\Gamma'$  
  }

  \caseText{By \autoref{thm:sort-compl} on (2), there is $\Phi_1$ such that}

  \caseFact{5} $\Theta ; \Delta \vdash I : \N \gens \Phi_1$

  \caseFact{6} $\Theta ; \Delta \vDash \Phi_1$

  \caseText{By \autoref{thm:sort-compl} on (3), there is $\Phi_2$ such that}

  \caseFact{7} $\Theta ; \Delta \vdash \vec{p} : \potvec \gens \Phi_2$

  \caseFact{8} $\Theta ; \Delta \vDash \Phi_2$

  \caseText{By IH on (4), there are $e'$, $\Phi_3$, $\Gamma'$ such that}

  \caseFact{9} $|e'| = e$

  \caseFact{10} $\Theta ; \Delta \vDash \Phi_3$

  \caseFact{11} $\Psi ; \Theta ; \Delta ; \Omega; \Gamma \vdash e' \checks \tau \gens \Phi_3,\Gamma'$

  \caseText{Goals follow by AT-Store on (6), (8), and (11)}
 }

 \jcase{33}{T-StoreConst}{
  \jgivengoalThreeExists{
    \caseFact{1} $\Psi ; \Theta ; \Delta ; \Omega ; \Gamma \vdash \texttt{store}[I](e) : \M \, (K,\texttt{const}(I)) \, ([I] \, \tau)$

    \caseFact{2} $\Theta ; \Delta \vdash I : \N$

    \caseFact{3} $\Psi ; \Theta ; \Delta ; \Omega ; \Gamma \vdash e : \tau$.

  }{
    $e'$, $\Phi'$,$\Gamma'$  
  }{
    $|e'| = \texttt{store}[I](e) $
  }{
    $\Theta ; \Delta \vDash \Phi'$  
  }{
    $\Psi ; \Theta ; \Delta ; \Omega ; \Gamma \pvdash e' \checks  \M \, (K,\texttt{const}(I)) \, ([I] \, \tau)\Phi',\Gamma'$  
  }

  \caseText{By \autoref{thm:sort-compl}, there is some $\Phi_1$ such that}

  \caseFact{4} $\Theta ; \Delta \vDash \Phi_1$
  
  \caseFact{5} $\Theta ; \Delta \vdash I : \mathbb{N} \gens \Phi_1$

  \caseText{By IH on (3), there are $e'$, $\Phi_2$, and $\Gamma'$ such that}

  \caseFact{6} $|e'| = e$

  \caseFact{7} $\Theta ; \Delta \vDash \Phi_2$

  \caseFact{8} $\Psi ; \Theta ; \Delta \vdash e' \checks \tau \gens \Phi_2,\Gamma'$

  \caseText{Goals follow immediately from AT-StoreConst on (5) and (8)}

 }


}

\iffalse
  \item[(T-ReleaseConst)] Suppose
  $\Psi ; \Theta ; \Delta ; \Omega ; \Gamma_1,\Gamma_2 \vdash \texttt{release } x = e_1 \texttt{ in }e_2 : \M \, \phi(I,\vec{p}) \, \tau_2$ from
  $\Psi ; \Theta ; \Delta ; \Omega ; \Gamma_1 \vdash e_1 : [J] \tau_1$ and
  $\Psi ; \Theta ; \Delta ; \Omega ; \Gamma_2, x : \tau \vdash e_2 : \M \, \phi(I,\vec{p} + \texttt{const}(J)) \, \tau_2$.
  By IH, there are $e_1'$, $\Phi_1$, $\Gamma_1'$ such that
  $|e_1'| = e_1$,
  $\Theta ; \Delta \vDash \Phi_1$, and
  $\Psi ; \Theta ; \Delta ; \Omega ; \Gamma_1 \vdash e_1' \infers [J] \tau_1 \gens \Phi_1,\Gamma_1'$.
  Since $\Psi ; \Theta ; \Delta \vdash \Gamma_1,\Gamma_2 \wknto \Gamma_1$, we have by \autoref{thm:admits-weaken}
  that there are $e_1''$, $\Phi_1'$, $\Gamma_1''$ such that
  $|e_1''| = |e_1'|$,
  $\Theta ; \Delta \vDash \Phi_1'$,
  $\Psi ; \Theta ; \Delta \vdash \Gamma_1'' \wknto \Gamma_2$, and
  $\Psi ; \Theta ; \Delta ; \Omega; \Gamma_1,\Gamma_2 \vdash e_1'' \infers [J] \tau_1 \gens \Phi_1',\Gamma_1''$.
  Again by IH, there are $e_2'$, $\Phi_2$, $\Gamma_2'$ such that
  $|e_2'| = e_2$,
  $\Theta ; \Delta \vDash \Phi_2$, and
  $\Psi ; \Theta ; \Delta ; \Omega ; \Gamma_2, x : \tau \vdash e_2' \checks \M \, \phi(I,\vec{p} + \texttt{const}(J)) \, \tau_2 \gens \Phi_2,\Gamma_2'$.
  But, by \autoref{thm:admits-weaken}, there are $e_2''$, $\Phi_2'$, $\Gamma_2''$ such that
  $|e_2''| = |e_2'|$,
  $\Theta ; \Delta \vDash \Phi_2'$, and
  $\Psi ; \Theta ; \Delta ; \Omega ; \Gamma_1'', x: \tau \vdash e_2'' \checks  \M \, \phi(I,\vec{p} + \texttt{const}(J)) \, \tau_2 \gens \Phi_2',\Gamma_2''$.
  Then, by AT-ReleaseConst, 
  $\Psi ; \Theta ; \Delta ; \Omega ; \Gamma_1,\Gamma_2 \vdash \texttt{release } x = e_1'' \texttt{ in }e_2'' : \M \, \phi(I,\vec{p}) \, \tau_2 \gens \Phi_1' \wedge \Phi_2',\Gamma_2''$.
  This completes (1), and (2) follows by AT-Anno.
  
  \item[(T-Shift)] Suppose $\Psi ; \Theta ; \Delta ; \Omega ; \Gamma \vdash \texttt{shift}(e) : \M \, \phi(I,\vec{p}) \, \tau$ by way of
  $\Psi ; \Theta ; \Delta ; \Omega ; \Gamma \vdash e : \M \, \phi(I - 1,\lhd \vec{p}) \, \tau$ and
  $\Theta ; \Delta \vDash I \geq 1$.
  By IH, there are $e'$, $\Phi$, $\Gamma'$ such that
  $|e'| = e$,
  $\Theta ; \Delta \vDash \Phi$, and
  $\Psi  ; \Theta ; \Delta ; \Omega; \Gamma \vdash e' \checks \M \, \Phi(I-1,\lhd \vec{p}) \, \tau \gens \Phi,\Gamma'$.
  By AT-Shift,
  $\Psi  ; \Theta ; \Delta ; \Omega; \Gamma \vdash \texttt{shift}(e') \checks \M \, \Phi(I,\vec{p}) \, \tau \gens (I \geq 1) \wedge \Phi,\Gamma'$.
  Since $\Theta ; \Delta \vDash (I \geq 1) \wedge \Phi$, this completes (1), and (2) follows by AT-Anno.

  \item[(T-CImpI)] Suppose
  $\Psi ; \Theta ; \Delta ; \Omega ; \Gamma \vdash \Lambda .e : (\Phi' \Rightarrow \tau)$ by way of
  $\Psi ; \Theta ; \Delta,\Phi' ; \Omega ; \Gamma \vdash e : \tau$.
  By IH, there are $e'$, $\Phi$, $\Gamma'$ such that
  $|e'| = e$,
  $\Theta ; \Delta,\Phi' \vDash \Phi$, and
  $\Psi ; \Theta ; \Delta,\Phi'; \Omega ; \Gamma \vdash e' \checks \tau \gens \Phi,\Gamma'$.
  By AT-CImpI,
  $\Psi ; \Theta ; \Delta ; \Omega ; \Gamma \vdash \Lambda. e' \checks \Phi' \implies \tau \gens (\Phi' \to \Phi),\Gamma'$.
  This completes (1), and (2) follows by AT-Anno.
  
  \item[(T-CImpE)] Suppose
  $\Psi ; \Theta ; \Delta ; \Omega ; \Gamma \vdash e \{\} : \tau$ by way of
  $\Psi ; \Theta ; \Delta ; \Omega ; \Gamma \vdash e : \Phi' \Rightarrow \tau$ and
  $\Theta ; \Delta \vDash \Phi'$.
  By IH, there are $e'$, $\Phi$, $\Gamma'$ such that
  $|e'| = e$,
  $\Theta ; \Delta \vDash \Phi$, and
  $\Psi ; \Theta ; \Delta ; \Omega ; \Gamma \vdash e' \infers \Phi' \Rightarrow \tau \gens \Phi,\Gamma'$.
  By AT-CImpE,
  $\Psi ; \Theta ; \Delta ; \Omega ; \Gamma \vdash e \{\} \infers \tau \gens \Phi \wedge \Phi',\Gamma'$.
  Since $\Theta ; \Delta \vDash \Phi'$ and $\Theta ; \Delta \vDash \Phi$, we have $\Theta ; \Delta \vDash \Phi \wedge \Phi'$.
  This completes (2). For (2),
  by \autoref{thm:subty-refl}, there is some $\Phi''$ with
  $\Theta ; \Delta \vDash \Phi''$, and
  $\Psi ; \Theta ; \Delta \vdash \tau \subty \tau : \star \gens \Phi''$.
  Then, by AT-Sub,
  $\Psi ; \Theta ; \Delta ; \Omega ; \Gamma \vdash e \{\} \checks \tau \gens \Phi \wedge \Phi' \wedge \Phi'',\Gamma'$,
  which completes (1).
  
  \item[(T-CAndI)] Suppose
  $\Psi ; \Theta ; \Delta ; \Omega ; \Gamma \vdash <e> : \Phi' \amp \tau$ by way of
  $\Psi ; \Theta ; \Delta ; \Omega ; \Gamma \vdash e : \tau$ and
  $\Theta ; \Delta \vDash \Phi'$.
  By IH, there are $e'$, $\Phi$, $\Gamma'$ such that
  $|e'| = e$,
  $\Theta ; \Delta \vDash \Phi$, and
  $\Psi  ; \Theta ; \Delta ; \Omega ; \Gamma \vdash e' \checks \tau \gens \Phi,\Gamma'$.
  By AT-CAndI,
  $\Psi; \Theta ; \Delta ; \Omega ; \Gamma \vdash <e'> \checks \Phi' \amp \tau \gens \Phi \wedge \Phi',\Gamma'$.
  Since $\Theta ; \Delta \vDash \Phi$ and $\Theta ; \Delta \vDash \Phi'$, we have $\Theta ; \Delta \vDash \Phi \wedge \Phi'$,
  completing (1). (2) follows by AT-Anno.
  
  \item[(T-CAndE)] Suppose
  $\Psi ; \Theta ; \Delta ; \Omega ; \Gamma_1,\Gamma_2 \vdash \texttt{clet } x = e_1 \texttt{ in } e_2 : \tau'$ by way of
  $\Psi ; \Theta ; \Delta ; \Omega ; \Gamma_1 \vdash e_1 : \Phi' \amp \tau$, and
  $\Psi ; \Theta ; \Delta, \Phi' ; \Omega ; \Gamma_2, x : \tau \vdash e_2 : \tau'$.
  By IH, there are $e_1'$, $\Phi_1$, $\Gamma_1'$ such that
  $|e_1'| = e_1$,
  $\Theta ; \Delta \vDash \Phi_1$, and
  $\Psi ; \Theta ; \Delta ; \Omega ; \Gamma_1 \vdash e_1' \infers \Phi' \amp \tau \gens \Phi_1,\Gamma_1'$.
  By \autoref{thm:admits-weaken}, there are $e_1''$, $\Phi_1'$, $\Gamma_1''$ such that
  $|e_1''| = |e_1'|$,
  $\Theta ; \Delta \vDash \Phi_1'$,
  $\Psi ; \Theta ; \Delta \vdash \Gamma_1'' \wknto (\Gamma_1,\Gamma_2) \setminus \Gamma_1$, and
  $\Psi ; \Theta ; \Delta ; \Omega ; \Gamma_1,\Gamma_2 \vdash e_1'' \infers \Phi' \amp \tau \gens \Phi_1',\Gamma_1''$.
  Again by IH, there are $e_2'$, $\Phi_2$, $\Gamma_2'$ such that
  $|e_2'| = e_2$,
  $\Theta ; \Delta,\Phi' \vDash \Phi_2$, and
  $\Psi ; \Theta ; \Delta,\Phi' ; \Omega ; \Gamma_2,x : \tau \vdash e_2' \checks \tau' \gens \Phi_2,\Gamma_2'$.
  Since
  $\Psi ; \Theta ; \Delta \vdash \Gamma_1'' \wknto \Gamma_2$, we have by \autoref{thm:ctx-sub-wkn} and \autoref{thm:ctx-sub-subset2}
  that
  $\Psi ; \Theta ; \Delta,\Phi' \vdash \Gamma_1'',x : \tau \wknto \Gamma_2,x : \tau$,
  and so by \autoref{thm:admits-weaken}, there are $e_2''$, $\Phi_2'$, $\Gamma_2''$ such that
  $|e_2''| = |e_2'|$,
  $\Theta ; \Delta, \Phi' \vDash \Phi_2'$,
  $\Psi ; \Theta ; \Delta, \Phi' ; \Omega ;\Gamma_1'',x:\tau \vdash e_2'' \checks \tau' \gens \Phi_2',\Gamma_2''$.
  Then, by AT-CAndE,
  $\Psi ; \Theta ; \Delta ; \Omega ; \Gamma_1,\Gamma_2 \vdash \texttt{clet } x = e_1'' \texttt{ in } e_2'' : \tau' \gens \Phi_1' \wedge (\Phi' \to \Phi_2'),\Gamma_2''$.
  Since $\Theta ; \Delta \vDash \Phi_1'$ and $\Theta ; \Delta, \Phi' \vDash \Phi_2'$,
  we have $\Theta ; \Delta \vDash \Phi_1' \wedge (\Phi' \to \Phi_2')$
  This completes (1), and (2) follows by AT-Anno.
  
  \item[(T-Sub)] Suppose $\Psi ; \Theta ; \Delta ; \Omega ; \Gamma \vdash e : \tau$ from $\Psi ; \Theta ; \Delta ; \Omega ; \Gamma \vdash e : \tau'$ and $\Psi;\Theta;\Delta \vdash \tau' \subty \tau : \star$. By IH, there are $e',\Phi_1,\Gamma'$ so that $|e'| = e$, $\Theta ; \Delta \vDash \Phi_1$, and $\Psi ; \Theta ; \Delta ; \Omega ; \Gamma \vdash e' \infers \tau' \gens \Phi_1,\Gamma'$. By \autoref{thm:subty-compl}, there is $\Phi_2$ such that $\Theta ; \Delta \vDash \Phi_2$, and $\Psi ; \Theta ; \Delta \vdash \tau' \subty \tau : \star \gens \Phi_2$. By AT-Sub, $\Psi ; \Theta ; \Delta ; \Omega ; \Gamma \vdash e' \checks \tau \gens \Phi_1 \wedge \Phi_2,\Gamma'$, which completes (1). For (2), we apply AT-Anno to get $\Psi ; \Theta ; \Delta ; \Omega ; \Gamma \vdash (e' : \tau) \infers \tau \gens \Phi_1 \wedge \Phi_2,\Gamma'$, and are done.
  
  \item[(T-Weaken)] Suppose $\Psi ; \Theta ; \Delta ; \Omega' ; \Gamma' \vdash e : \tau$ from $\Psi ; \Theta ; \Delta ; \Omega ; \Gamma \vdash e : \tau$, $\Theta ; \Delta \vdash \Omega' \bdby \Omega$, and $\Theta ; \Delta \vdash \Gamma' \bdby \Gamma$. By IH, there are $e'$, $\Phi$, $\Gamma''$ so that $|e'| = e$, $\Theta ; \Delta \vDash \Phi$, and $\Psi ; \Theta ; \Delta ; \Omega ; \Gamma \vdash e : \tau \gens \Phi, \Gamma''$. By \autoref{thm:admits-weaken}, there are $e_1$, $\Phi_1$, $\Gamma_1$ so that $|e_1| = |e'|$, $\Theta ; \Delta \vDash \Phi_1$, and $\Psi ; \Theta ; \Delta ; \Omega' ; \Gamma' \vdash e_1 \checks \tau \gens \Phi_1,\Gamma_1$, and also that there are $e_2$, $\Phi_2$, $\Gamma_2$ so that $|e_2| = |e'|$, $\Theta ; \Delta \vDash \Phi_2$, and $\Psi ; \Theta ; \Delta ; \Omega' ; \Gamma' \vdash e_2 \infers \tau \gens \Phi_2,\Gamma_2$, which proves (1) and (2).
 
\end{itemize}

\end{proof}
\fi