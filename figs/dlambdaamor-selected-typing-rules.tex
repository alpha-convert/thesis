\begin{mathpar}
\inferrule[T-Var-1]
{x : \tau \in \Gamma}{\Psi ; \Theta ; \Delta ; \Omega ; \Gamma\vdash x : \tau}

\inferrule[T-TensorI]
{
\Psi ; \Theta ; \Delta ; \Omega ; \Gamma_1\vdash e_1 : \tau_1\\
\Psi ; \Theta ; \Delta ; \Omega ; \Gamma_2\vdash e_2 : \tau_2\\
}{
\Psi ; \Theta ; \Delta ; \Omega ; \Gamma_1,\Gamma_2\vdash \angles{e_1,e_2} ; \tau_1 \otimes \tau_2
}

\inferrule[T-TensorE]
{
\Psi ; \Theta ; \Delta ; \Omega ; \Gamma_1\vdash e : \tau_1 \otimes \tau_2\\
\Psi ; \Theta ; \Delta ; \Omega ; \Gamma_2,x : \tau_1, y : \tau_2\vdash e' : \tau'
}{
\Psi ; \Theta ; \Delta ; \Omega ; \Gamma_1,\Gamma_2\vdash \texttt{let } \angles{x,y} = e \texttt{ in } e' : \tau'
}

\\

\inferrule[T-Nil]
{ }
{\Psi ; \Theta ; \Delta ; \Omega ; \Gamma\vdash \texttt{nil} : L^0 \tau}

\inferrule[T-Cons]
{\Psi ; \Theta ; \Delta ; \Omega ; \Gamma_1\vdash e_1 : \tau\\
\Psi ; \Theta ; \Omega ; \Gamma_2\vdash e_2 : L^{I} \tau
}
{\Psi ; \Theta ; \Delta ; \Omega ; \Gamma_1, \Gamma_2\vdash e_1 :: e_2 : L^{I + 1} \tau}
\\

\inferrule[T-Match]
{
\Psi ; \Theta ; \Delta ; \Omega ; \Gamma_1\vdash e : L^I \tau\\
\Psi ; \Theta ; \Delta, I = 0 ; \Omega ; \Gamma_2\vdash e_1 : \tau'\\
\Psi ; \Theta ; \Delta, I \geq 1; \Omega ; \Gamma_2, h : \tau, t : L^I \tau \vdash e_2 : \tau'\\
}
{\Psi ; \Theta ; \Delta ; \Omega ; \Gamma_1,\Gamma_2\vdash \texttt{match}(e,e_1,h.t.e_2) : \tau'}


\inferrule[T-Ret]
{
\Psi ; \Theta ; \Delta ; \Omega ; \Gamma \vdash e : \tau
}{
\Psi ; \Theta ; \Delta ; \Omega ; \Gamma \vdash \texttt{ret}\; e : \M \, (I,\vec{0}) \, \tau
}

\inferrule[T-Tick]
{
\Theta ; \Delta \vdash I : \mathbb{N}\\
\Theta ; \Delta \vdash \vec{p} : \vec{\mathbb{R}^+}
}{
\Psi ; \Theta ; \Delta ; \Omega ; \Gamma \vdash \texttt{tick}[I|\vec{p}] : \M \, (I,\vec{p})\, 1
}


\inferrule[T-Bind]
{
\Psi ; \Theta ; \Delta ; \Omega ; \Gamma_1 \vdash e_1 : \M \, (I,\vec{p})\, \tau_1\\
\Psi ; \Theta; \Delta ; \Omega ; \Gamma_2, x:\tau_1 \vdash e_2 : \M \, (I,\vec{q})\, \tau_2\\
}{
\Psi ; \Theta ; \Delta ; \Omega ; \Gamma_1,\Gamma_2 \vdash \texttt{bind } x = e_1 \texttt{ in } e_2 : \M \, (I,\vec{p} + \vec{q})\, \tau_2
}

\inferrule[T-Release]
{
\Psi ; \Theta ; \Delta ; \Omega ; \Gamma_1 \vdash e_1 : [I | \vec{q}] \tau_1\\
\Psi ; \Theta ; \Delta ; \Omega ; \Gamma_2, x : \tau \vdash e_2 : \M \, (I,\vec{p} + \vec{q}) \, \tau_2\\
}{
\Psi ; \Theta ; \Delta ; \Omega ; \Gamma_1,\Gamma_2 \vdash \texttt{release } x = e_1 \texttt{ in }e_2 : \M \, (I,\vec{p}) \, \tau_2
}
\\

\inferrule[T-Store]
{
\Theta ; \Delta \vdash I : \mathbb{N}\\
\Theta ; \Delta \vdash \vec{p} : \vec{\mathbb{R}^+}\\
\Psi ; \Theta ; \Delta ; \Omega ; \Gamma \vdash e : \tau\\
}{
\Psi ; \Theta ; \Delta ; \Omega ; \Gamma \vdash \texttt{store}[I|\vec{p}](e) : \M \, (I,\vec{p}) \, ([I| \vec{p}] \, \tau)
}


\inferrule[T-Sub]
{
\Psi ; \Theta ; \Delta ; \Omega ; \Gamma \vdash e : \tau'\\
\Psi;\Theta;\Delta \vdash \tau' \subty \tau
}{
\Psi ; \Theta ; \Delta ; \Omega ; \Gamma \vdash e : \tau
}

\inferrule[T-Weaken]
{
\Psi ; \Theta ; \Delta ; \Omega ; \Gamma \vdash e : \tau\\
\Theta ; \Delta \vdash \Omega' \bdby \Omega\\
\Theta ; \Delta \vdash \Gamma' \bdby \Gamma\\
}{
\Psi ; \Theta ; \Delta ; \Omega' ; \Gamma' \vdash e : \tau
}
\end{mathpar}